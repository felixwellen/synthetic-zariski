This is an incomplete draft on work in progress by Ingo Blechschmidt, Felix Cherubini and David Wärn (\cite{chech-draft}).

In pure mathematics,
it is a common practice to simplify questions about complicated objects
by assigning them more simply objects in a systematic way that faithfully represents some features of interest.
One particular, but still surprisingly broad applicable instantiation of this appraoch,
is the assignment of a sequence of abelian groups, the cohomology groups, to spaces, sheaves and other things.
Over the last century, cohomology was first discovered in concrete examples, then generalized and streamlined,
a process that culminated in the presentation of cohomology groups as mapping spaces in higher toposes.

This is a representation, we can easily and elementary use through the interpretation of homotopy type theory in higher toposes.
In \cite{evan-master-thesis} results about cohomology theories like the Mayer-Vietoris-Sequence were proven and computations were carried out,
in \cite{floris-thesis}, the Serre-Spectral-Sequence was constructed and used.
The latter also introduced cohomology with non-constant coefficients,
which are the right level of generality for the applications we have in mind.
We are particularly interested in computing cohomology groups of sheaves in algebraic geometry,
which can be done synthetically using the foundation laid out by \cite{draft}
building on work and ideas of Ingo Blechschmidt (\cite{ingo-thesis}), Anders Kock (\cite{kock-sdg})
and David Jaz Myers (\cite{myers-talk1}, \cite{myers-talk2}).

In this setup, the basic spaces in algebraic geometry, schemes, are just sets with a particular property \cite{draft}[def of scheme],
and instead of sheaves on a type $X$, we consider, more generally maps $A:X\to\AbGroup$ to the type of abelian groups.
The cohomology groups are then defined as dependent function types with values in Eilenberg-MacLane-Spaces
\[ H^n(X,A)\colonequiv \|(x:X)\to K(A(x),n)\|_0\]
-- a definition first suggested by \cite{mike-blogpost}.
Due to its simplicity, this is very convenient to work with.
One common way to calculate cohomology groups $H^n(X,\mathcal F)$ is to
use results about the cohomology of simple subspaces $U_i\subseteq X$.
A computational result on the case with two subspaces $U,V\subseteq X$ is known as the \notion{Mayer-Vietoris-Sequence}.
In general this sequence helps to calculate the cohomology groups of a pushout
and was constructed for cohomology with constant coefficients in a group in \cite{evan-master-thesis}.
We generalize this result to non-constant coefficients (\cref{mayer-vietoris-sequence})
with a slick proof the second author learned in parts from Urs Schreiber in the course of his phd-thesis.

\v{C}ech Cohomology, in the sense of this work,
is a generalization of the Mayer-Vietoris Sequence in the case, where $U,V$ are actually subtypes of a set,
to a space $X$ which is the union of fintely many subtypes $U_i\subseteq X$, i.e. $\bigcup_{i}U_i=X$.
From a \emph{synthetic homotopy theory}, this is not very interesting,
but it is very interesting for our intended applications in \emph{synthetic algebraic geometry}.
In the latter subhject, it was unclear for a long time how one could set up a theory of cohomology,
since the classical treatment relies on (non-)constructions, which need the axiom of choice.

In \cite{draft} this problem is circumvented,
by using a justified axiom which allows a bit of choice which is related to the topology of the relevant topos
and, secondly, as mentioned above, by using higher types to define work with cohomology.

We present two approaches to a proof of a sufficiently general isomorphism between \v{C}ech Cohomology groups
and cohomology groups defined using higher types.
The first appraoch is more conceptual, more general and makes use of the higher types with have available in HoTT.
It is also related to how one would produce a \v{C}ech Cohomology theorem in higher category theory:
the space is represented as a colimit, so mapping into the coefficients should yield a limit description of
the (untruncated) cohomology of the whole space.

The second approach very roughly follows old classical treatments of Grothendieck (\cite{tohoku1957}) and Buchsbaum (\cite{buchsbaum}).
Inspired by this, we aim to show that both cohomology defined using higher types and \v{C}ech cohomology satisfies the universal property
of a \notion{universal $\partial$-functor} in some furtunate but still quite relevant situations.
While this approach is far less general, it also seems to need far less involved calculations.
