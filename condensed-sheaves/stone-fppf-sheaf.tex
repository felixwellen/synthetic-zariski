In this section we work in the Zariski topos.

\subsection{Fppf sheaves in the Zariski topos}

\begin{definition}
The fppf topology consists of $\Spec(B)$ for all fppf boolean algebra $B$.
\end{definition}

The checking that this is a topology is the same as in algebraic geometry.

\begin{lemma}\label{2-modules-flat}
Any $2$-module is flat.
\end{lemma}

\begin{proof}
This is true for any discrete field, the idea is that any module is a filtered colimit of finitely presented modules, finitely presented modules over a discrete field are free therefore flat, and a filtered colimit of flat modules is flat.
\end{proof}

\begin{remark}
We think that any module over a boolean algebra is flat. We were not able to prove this yet.
\end{remark}

\begin{lemma}
A boolean algebra $B$ is fppf if and only if $0\not=_B1$.
\end{lemma}

\begin{proof}
By \cref{2-modules-flat} we have that $B$ is always flat, so we have to prove that:
\[\prod_{M:2-\mathrm{Mod}} B\otimes M = 0 \to M=0\]
if and only if:
\[0\not=_B1\]
The direct is clear, conversly if $0\not=_B1$ then:
\[2\to B\]
is injective, but by \cref{2-modules-flat} we have that any $M$ is flat so that:
\[M\to M\otimes B\]
is injective which is enough to conclude.
\end{proof}

\begin{lemma}
The fppf topology is subcanonical.
\end{lemma}

\begin{proof}
We could just reuse the proof from SAG, but there probably are simpler ones. TODO
\end{proof}

Once again next theorem relies on the unproven \cref{sheaves-internal}.

\begin{theorem}
The Fppf topos satisfy the following:
\begin{itemize}
\item For all c.p. $B$ such that $0\not=_B1$, we have that $\propTrunc{\Spec(B)}$
\item For all c.p. boolean ring $C$, the map:
\[C\to 2^{\Spec(C)}\]
is an equivalence.
\item For all c.p. boolean ring $C$, the type $\Spec(C)$ has representable choice.
\end{itemize}
\end{theorem}

\begin{lemma}
For all c.p. boolean algebras $A,B$, a map:
\[\Spec(B)\to \Spec(A)\]
is surjective if and only if the corresponding map:
\[A\to B\]
is injective.
\end{lemma}

\begin{proof}
Assume a surjective map:
\[\Spec(B)\to \Spec(A)\]
corresponding to:
\[f:A\to B\]
Then given $a,b:A$ such that $f(a)=f(b)$ we have that:
\[\prod_{x:\Spec(A)} x(a) = x(b)\]
indeed for any $x:\Spec(A)$ we want to prove a proposition, so we can assume $y:\Spec(B)$ such that $y\circ f = x$ and then $f(a)=f(b)$ allows us to conclude. Therefore $a=b$.

Conversely if the map is injective TODO (would be a consequence of all modules over boolean algebras being flat)
\end{proof}


\subsection{Dependent choice from Zariski sheaves to Fppf sheaves}

\begin{lemma}\label{fppf-stable-sigma}
Assume:
\[\Spec(B) \to \Spec(A)\]
an fppf cover of affine scheme. If $A$ is fppf, so is $B$.
\end{lemma}

\begin{proof}
TODO easy
\end{proof}

\begin{lemma}\label{fppf-cover-transfinite-composition}
Assume given:
\[\cdots \to A_2\to A_1 \to A_0 \]
a tower of fppf covers, then:
\[\lim_i A_i \to A_0\]
is an fppf cover.
\end{lemma}

\begin{proof}
It is enough to check that given a tower of maps of c.p. algebra:
\[B_0\to B_1 \to B_2\to\cdots \]
such that $B_0$ is fppf and the induced $\Spec(B_{i+1}) \to \Spec(B_i)$ is an fppf cover, we have that $\mathrm{colim}_iB_i$ is a c.p. fppf algebra. The c.p. part is clear, and we have $0\not=_{B_i}1$ for all $i$ by inductively using \cref{fppf-stable-sigma}, so $0=1$ in $\mathrm{colim}_iB_i$ is a contradiction.
\end{proof}

Next lemma more or less that if dependent choice holds in the Zariski topos, it holds in the fppf topos:

\begin{lemma}
Assume given a tower of fppf sheaves:
\[\cdots \to X_2\to X_1 \to X_0 \]
where the maps are fppf surjective. Assuming dependent choice in the Zariski topos, we are able to show that the map:
\[\lim_i X_i\to X_0\]
is fppf surjective.
\end{lemma}

\begin{proof}
Consider the tower:
\[\cdots \to M_2\to M_1 \to M_0\]
where:
\[M_0 = X_0\]
and $M_{n+1}$ is defined as the type of commuting dotted arrows such that:
 \begin{center}
    \begin{tikzcd}
      X_{n+1} \ar[r] & X_0\\
      & A_{n+1}\ar[ul,dotted] \ar[u,swap,"fppf\ cover",dotted]
          \end{tikzcd}
  \end{center}
  Using fppf local choice, we can show that the maps $M_{n+1}\to M_n$ are surjective, so that by dependent choice $\lim_i M_i \to X_0$ is surjective. But given a point in its fiber means giving a commutative triangle:
 \begin{center}
    \begin{tikzcd}
      \lim_iX_i \ar[r] & X_0\\
      & \lim_iA_i\ar[ul,dotted] \ar[u,swap,dotted]
          \end{tikzcd}
  \end{center}
  and we conclude by \cref{fppf-cover-transfinite-composition}.
\end{proof}



