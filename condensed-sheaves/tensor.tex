

\subsection{Bounded distributive lattices}

\begin{definition}
A bounded distributive lattice is a set $X$ with $0,1:X$ and $\lor,\land$ binary operations such that:
\begin{itemize}
\item $\lor$ and $\land$ are idempotent commutative and associative.
\item $1$ is an identity for $\land$ and $0$ is an identity for $\lor$.
\item  $a\lor(a\land b) = a$ (or equivalently $a\land(a\lor b) = a$).
\item $a\land(b\lor c) = (a\land b)\lor(a\land c)$
\end{itemize}
\end{definition}

\begin{definition}
Given $A,B$ bounded distributive lattices, we define $A\boxtimes B$ as the type of formal sups $\bigvee_i a_i\boxtimes b_i$ for $a_i:A$ and $b_i:B$.

Then we define $\lor$ a binary operation on $A\boxtimes B$ as expected, and $\land$ by:
\[(\bigvee_ia_i\boxtimes b_i)\land (\bigvee_jc_j\boxtimes d_j) = \bigvee_{i,j}(a_i\land c_j)\boxtimes (b_i\land d_j)\]
as well as $0=0\boxtimes 0$ and $1=1\boxtimes 1$. 
\end{definition}

Note that $A\boxtimes B$ is not a distributive lattice. We want to correct that.

\begin{definition}
Inductively, we say that:
\[a\boxtimes b \leq \bigvee_ic_i\boxtimes d_i\]
if one of the following hold:
\begin{itemize}
\item $a\leq c_i$ and $b\leq d_i$ for some $i$.
\item $a=\bigvee_k a_k$ and for all $k$ we have $a_k\boxtimes b\leq \bigvee_ic_i\boxtimes d_i$.
\item $b=\bigvee_k b_k$ and for all $k$ we have $a\boxtimes b_k\leq \bigvee_ic_i\boxtimes d_i$.
\end{itemize}
Then we define:
\[\bigvee_ia_i\boxtimes b_i \leq \bigvee_ic_i\boxtimes d_i\]
as:
\[\forall i.\ a_i\boxtimes b_i \leq \bigvee_ic_i\boxtimes d_i\]
\end{definition}

\begin{lemma}\label{and-or-respect-leq}
Assume given $x,y,z:A\otimes B$. If $x\leq y$ then:
\[x\lor z \leq y\lor z\]
\[z\lor z \leq z\lor y\]
\[x\land z \leq y\land z\]
\[z\land z \leq z\land y\]
\end{lemma}

\begin{proof}
For the results about $\lor$, it is clear from the definition of $\leq$ that it is enough to prove that $a\otimes b\leq y\lor z$ when $a\otimes b \leq y$. We prove this by induction on $a\otimes b\leq y$.

For the results about $\land$, we see by the first results and direct computations that it is enough to prove it when $x=a\otimes b$ and $y=e\otimes f$. We prove this by induction on $a\otimes b \leq y$.
\end{proof}

\begin{definition}
We define $A\otimes B$ as the set quotient of $A\boxtimes B$ by the relation $x\leq y$ and $y\leq x$.

By \Cref{and-or-respect-leq} we have induced operations $0,1,\land,\lor$.
\end{definition}

\begin{lemma}
We have that $A\otimes B$ is a bounded distributive lattice, and that:
\[i_A:A\to A\otimes B\]
\[a\mapsto a\otimes 1\]
and:
\[i_B : B\to A\otimes B\]
\[b\mapsto 1\otimes b\]
are morphisms of bounded distributive lattice.
\end{lemma}

\begin{proof}
By direct computation.
\end{proof}

\begin{lemma}
We have that $A\otimes B$ is the coproduct of $A$ and $B$ in the category of bounded distributive lattice.
\end{lemma}

\begin{proof}
We need to check that given $f:A\to C$ and $g:B\to C$, there is a unique map $\psi:A\otimes B\to C$ such that $\psi\circ i_A = f$ and $\psi\circ i_B$.

It is clear that if such a $\psi$ exists, it is unique, as if $\psi(a\otimes 1) = f(a)$ and $\psi(1\otimes b) = g(b)$ then:
\[\psi(\bigvee_ia_i\otimes b_i) = \bigvee_i f(a_i)\land g(b_i)\]
Now we need to check that this formula indeed define a morphism in $A\otimes B\to C$.

First we check it respects the equivalence relation. It is enough to prove that if $a\boxtimes b \leq \bigvee_i c_i\otimes d_i$ then:
\[f(a)\land g(b) \leq \bigvee_i f(c_i)\land f(d_i)\] 
We proceed by induction on  $a\boxtimes b \leq \bigvee_i c_i\otimes d_i$.

Checking that $\psi$ is a morphism of lattice is straightforward.
\end{proof}

\begin{lemma}\label{leq-transitive}
The relation $\leq$ is reflexive and transitive on $A\boxtimes B$. Therefore the relation ($x\leq y$ and $y\leq x$) is an equivalence relation.
\end{lemma}

\begin{proof}
Reflexivity is straightforward, transitivity is by induction on $\leq$.
\end{proof}

As is customary in a lattice, we write $x\leq y$ in $A\otimes B$ as a shorthand for $x\land y = x$ in $A\otimes B$.

\begin{lemma}\label{leq-otimes}
Given $x,y:A\boxtimes B$ we have $[x]\leq [y]$ in $A\otimes B$ if and only if $x\leq y$ in $A\boxtimes B$.
\end{lemma}

\begin{proof}
By \Cref{leq-transitive} we have that $[x]\land [y]=[x]$ is and only if $x\land y\leq x$ and $x\leq x\land y$, which is equivalent to $x\leq y$.
\end{proof}

This is the result I am interested in:

\begin{proposition}
Given $a,c:A$ and $b,d:B$, we have that:
\[(a\otimes b) \leq (c\otimes d)\]
in $A\otimes B$ if and only if $a=0$ or $b=0$ or ($a\leq c$ and $b\leq d$).
\end{proposition}

\begin{proof}
By \Cref{leq-otimes} we have that $(a\otimes b) \leq (c\otimes d)$ in $A\otimes B$ if and only if $(a\boxtimes b) \leq (c\boxtimes d)$ in $A\boxtimes B$, and then we proceed by induction $\leq$.
\end{proof}

\begin{remark}
This mean that given a bounded distributive lattice $A$, tensoring with $A$ is an exact functor. Can we relativise to bounded distributive $A$-lattice? Can we extend to modules to get flatness of every module?
\end{remark}

\subsection{Boolean algebras}

\begin{definition}
Assume given $a:A$. A complement for $a$ is a $b:A$ such that $a \lor b = 1$ and $a\land b = 0$.
\end{definition}

\begin{lemma}\label{complement-unique}
Any $a:A$ has at most one complement.
\end{lemma}

\begin{proof}
We just need to check that there is at most one $b:A$ such that $a\lor b=1$ and $a\land b = 0$. But given two such $b,b'$ we have that:
\[b = b\land (a\lor b') = (b\land a) \lor (b\land b') = b\land b'\]
so $b\leq b'$ and for the same reason we have that $b'\leq b$ so that $b=b'$.
\end{proof}

\begin{definition}
A boolean algebra is a bounded distributive lattice where every element has a complement.
\end{definition}

\begin{lemma}\label{being-boolean-proposition}
A bounded distributive lattice being a boolean algebra is a proposition.
\end{lemma}

\begin{proof}
By \Cref{complement-unique}.
\end{proof}

\begin{lemma}
If $A$ and $B$ are boolean algebras, then so is $A\otimes B$. Then $A\otimes B$ is the coproduct of $A$ and $B$ in the category of boolean algebras.
\end{lemma}

\begin{proof}
If $x$ and $y$ have complements $x'$ and $y'$, then $x'\land y'$ is a complement for $x\lor y$. So it is enough to check that any $a\otimes b$ has a complement. But $(\neg a\otimes b) \lor (a\otimes \neg b) \lor (\neg a\otimes \neg b)$ is a complement for $a\otimes b$ by straightforward computations.
\end{proof}
