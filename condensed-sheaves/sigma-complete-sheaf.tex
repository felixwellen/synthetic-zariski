We work in a presheaf topos on a $\sigma$-complete boolean algebra.

\begin{definition}
A type $X$ is a sheaf if for all $e_1,\cdots,e_n:B$ a fundamental system of idempotent in $B$, we have that $X$ is $(e_1=1\lor\cdots\lor e_n=1)$-local.
\end{definition}

We will study the topos of such sheaves.

\subsection{Limited principle of omniscience}

\begin{lemma}
The boolean algebra $B$ is a sheaf.
\end{lemma}

\begin{proof}
Assume given $e_1,\cdots,e_n$ a fundamental system of idempotents. Giving a map:
\[\phi:e_1=1\lor\cdots\lor e_n=1 \to B\]
is equivalent to giving a map:
\[\psi : e_1=1+\cdots + e_n=1 \to B\]
as if $e_j = 1$ and $e_k=1$ for $k\not= l$, since $e_je_k=0$ we have that $0=_B1$ and $B=1$ so that $\psi$ agrees on both proofs. But then this is equivalent to:
\[B^{e_1=1}\times\cdots B^{e_n=1}\]
which by assumption in the presheaf model is equivalent to:
\[B_{e_1}\times\cdots B_{e_n}\]
which in turn is equivalent to $B$ using algebra (TODO check).
\end{proof}

This means we can interpret $B$ as itself in the sheaf topos.

\begin{lemma}\label{B-is-2-sigma-complete}
In the sheaf topos, we have that $B=2$.
\end{lemma}

\begin{proof}
First we prove that the interpretation of $0\not=_B1$ holds in the sheaf topos. The interpretation of $0=_B1$ is itself, and assuming $0=_B1$, the empty idempotent system implies that any sheaf is contractible and we can conclude.

Now we prove the interpretation of:
\[\forall(e:B).\ e=1 \lor e=0\]
in the sheaf topos. But we conclude by considering the fundamental system of idempotent $e,1-e$.
\end{proof}

MISTAKE Next lemma is probably wrong.

\begin{lemma}\label{N-sheaf-sigma-complete}
The type $\N$ is a sheaf.
\end{lemma}

\begin{proof}
TODO it would be enough to have that the interpretation of $\N$ is the sheafification of $\N$.
\end{proof}

\begin{theorem}[LPO]
In the sheaf topos, we have that $2$ is $\sigma$-complete.
\end{theorem}

\begin{proof}
By \cref{B-is-2} we know it is enough to show that $B$ is $\sigma$-complete in the sheaf topos. We write $\hat{\N}$ for the interpretation of $\N$ in the presheaf topos, and $\iota:\N\to\hat{\N}$ the canonical map.

We define $\hat{\lor} : (\hat{\N} \to B) \to B$ by $\hat{\lor}(\phi) = \lor(\phi\circ \iota)$.

Given $\phi : \hat{N}\to B$, it is clear that $\hat{\lor}(\phi)$ is smaller than any bound of $\phi$, we need to check that for all $n:\hat{\N}$ we have that:
\[\phi(n) \leq \lor(\phi\circ \iota)\]
TODO
\end{proof}

\subsection{Dependent choice}

\begin{lemma}
Let $X$ be a sheaf, then for any system of idempotent $e_1,\cdots, e_n$ we have that $X$ is $(e_1=1+\cdots+e_n=1)$-local.
\end{lemma}

\begin{proof}
Giving a map:
\[\phi:e_1=1+\cdots+e_n=1 \to X\] 
is equivalent to giving a map:
\[\psi:e_1=1\lor\cdots\lor e_n=1 \to X\]
as if we have $j\not=l$ such that $e_j=1$ and $e_k=1$, the fact that $e_je_k=0$ implies $0=1$ so that $X$ is contractible and $\phi$ agrees on both. 
\end{proof}

This is a shorter but less general proof of \cref{zariski-sum-local}.

\begin{lemma}\label{prop-trunc-sheaf-sigma-complete}
Let $X$ be a sheaf, then $\propTrunc{X}$ is a sheaf. 
\end{lemma}

\begin{proof}
Assuming that for any $e:B$ we have that $e=1$ has choice, we just reason as in \cref{zariski-stable-truncation}.
\end{proof}

This means that when interpreting something in the sheaf topos, we just interpret propositional truncation as itself.

\begin{theorem}
Dependent choice holds in the sheaf topos.
\end{theorem}

\begin{proof}
It just depends on $\N$ and $\propTrunc{\_}$, so by \cref{N-sheaf-sigma-complete} and \cref{prop-trunc-sheaf-sigma-complete} we can conclude.
\end{proof}