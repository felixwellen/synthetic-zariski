We work in a presheaf topos on a $\sigma$-complete boolean algebra.


\subsection{Sheaves}

Since $B$ is $\sigma$-complete, we can define fundamental systems of idempotents indexed by $\N$. We say that a type $I$ is countable if it is finite or merely equal to $\N$.

\begin{definition}
A type $X$ is a sheaf if for all $I$ countable and $(e_i)_{i:I}$ fundamental system of idempotent in $B$, we have that $X$ is $(\exists (i:I). e_i=1)$-local.
\end{definition}

\begin{lemma}\label{countable-cover-cech}
Assume $X$ a type and $I$ countable with $(P_i)_{i:I}$ a sequence of proposition such that for all $i\not=j$ in $I$ we have that $P_i\land P_j$ implies $X$ contractible. Then the canonical map:
\[X^{\exists(i:I). P_i} \to X^{\Sigma(i:I).P_i}  \]
is an equivalence.
\end{lemma}

\begin{proof}
We have that:
\[ \Sigma(i:\N). P_i = \mathrm{colim}_n \Sigma(i\leq n). P_i\]
\[ \exists(i:\N). P_i = \mathrm{colim}_n \exists(i\leq n). P_i\]
so it is enough to prove the proposition for finite sets. If $I$ is empty it is clear, let us prove it for $I=\mathrm{Fin}(n+1)$. We have to prove that the canonical map: 
\[X^{\exists(i:\mathrm{Fin}(n+1)). P_i}\to X^{\Sigma(i : \mathrm{Fin}(n+1)). P_i}\]
is an equivalence. We have that:
\[\exists(i: \mathrm{Fin}(n+1)). P_i = \exists(i:\mathrm{Fin}(n)). P_i \lor P_{n}\]
By induction it is enough to prove that the map:
\[X^{\exists(i:\mathrm{Fin}(n+1)). P_i} \to X^{\exists(i:\mathrm{Fin}(n)).P_i + P_{n}}\]
is an equivalence. But we have the pushout square:
\begin{center}
\begin{tikzcd}
\exists(i:\mathrm{Fin}(n)).P_i  \land P_{n}\ar[d]\ar[r]& P_{n}\ar[d]\\
\exists(i:\mathrm{Fin}(n)).P_i \ar[r] & \exists(i:\mathrm{Fin}(n+1)).P_i  \\
\end{tikzcd}
\end{center}
so we have the pullback square:
\begin{center}
\begin{tikzcd}
X^{\exists(i:\mathrm{Fin}(n+1)).P_i} \ar[r]\ar[d]& X^{P_{n}}\ar[d]\\
X^{\exists(i\mathrm{Fin}(n)).P_i} \ar[r] & X^{\exists(i\mathrm{Fin}(n)).P_i  \land P_{n}}\\
\end{tikzcd}
\end{center}
and $\exists(i:\mathrm{Fin}(n)).P_i  \land P_{n}$ implies $X$ contractible by hypothesis, so that $X^{\exists(i\mathrm{Fin}(n)).P_i  \land P_{n}}$ is contractible and the map:
\[X^{\exists(i:\mathrm{Fin}(n+1)). P_i} \to X^{\exists(i:\mathrm{Fin}(n)).P_i + P_{n}}\]
is an equivalence.
\end{proof}

\begin{lemma}\label{sigma-complete-sheaf-simpler-aux}
Given a type $X$ such that $0=1$ implies $X$ contractible, for all $I$ countable and $(e_i)_{i:I}$ fundamental system of idempotents in $B$ we have that the canonical map:
\[X^{\exists(i:I). e_i=1} \to X^{\Sigma(i:I). e_i=1}  \]
is an equivalence.
\end{lemma}

\begin{proof}
We just apply \cref{countable-cover-cech}.
\end{proof}

\begin{lemma}\label{sigma-complete-sheaf-simpler-aux-2}
Given a set $X$ such that $0=1$ implies $X$ contractible, for all proposition $P$ the canonical map:
\[X^{P\lor (0=1)} \to X^P\]
is an equivalence.
\end{lemma}

\begin{proof}
As in \cref{sigma-complete-sheaf-simpler-aux} we have that the map:
\[X^{P\lor(0=1)}\to X^{P+(0=1)}\]
is an equivalence so it is enough to prove that the map:
\[X^{P+ (0=1)} \to X^P\]
is an equivalence, but this is the case as $X^{0=1}$ is contractible.
\end{proof}

\begin{lemma}\label{sigma-complete-sheaf-simpler}
Given a type $X$, the following are equivalent:
\begin{enumerate}[(i)]
\item $X$ is a sheaf.
\item We have that $0=1$ implies $X$ contractible and for all $(e_i)_{i:\N}$ fundamental system of idempotent in $B$ we have that $X$ is $(\Sigma(i:\N). e_i=1)$-local.
\end{enumerate}
\end{lemma}

\begin{proof}
We prove (i) implies (ii). If $0=1$ then any sheaf is contractible by considering the empty system of idempotent, and then we conclude by \cref{sigma-complete-sheaf-simpler-aux}.

We prove (ii) implies (i). Assume given $I$ countable and $(e_i)_{i:I}$ a fundamental system of idempotent in $B$. If $I$ is empty then $0=1$ and $X$ is contractible so it is a sheaf. If $I$ is infinite, by \cref{sigma-complete-sheaf-simpler-aux} we know that $X$ is $(\exists(i:I). e_i=1)$-local. If $I$ is finite non-empty we extend $(e_i)_{i:I}$ to $\N$ by adding $0$. Then $X$ is $(\exists(i:I).e_i=1) \lor (0=1)$-local and we can conclude by \cref{sigma-complete-sheaf-simpler-aux-2}.
\end{proof}

We will study the topos of such sheaves.


\subsection{Interpretation in the sheaf topos}

\begin{lemma}\label{prop-trunc-sheaf-sigma-complete}
Let $X$ be a sheaf, then $\propTrunc{X}$ is a sheaf. 
\end{lemma}

\begin{proof}
We use \cref{sigma-complete-sheaf-simpler}. If $0=1$ implies $X$ contractible, then it implies $\propTrunc{X}$ is contractible.

Given $(e_i)_{i:\N}$ fundamental system of idempotent in $B$, we check that $\propTrunc{X}$ is $(\sum_{i:\N} e_i=1)$-local. Since $\propTrunc{X}$ is a proposition we just need to prove $\propTrunc{X}$ assuming that:
\[\left(\sum_{i:\N} e_i=1\right) \to \propTrunc{X}\]
But we can merely fill the diagram:
\begin{center}
\begin{tikzcd}
\sum_{i:\N} e_i=1\ar[d]\ar[r]\ar[rd,dotted] & \propTrunc{X}\\
1\ar[dotted,r] & X\ar[u]\\
\end{tikzcd}
\end{center}
where the dotted diagonal arrow comes from $\sum_{i:\N} e_i=1$ having choice (by DC plus representable having choice) and the dotted down arrow comes from $X$ being a sheaf.
\end{proof}

This means that the propositional truncation can be interpreted as itself in the sheaf topos.

We write $\hat{X}$ for the intepretation of $X$ in the sheaf topos. 

\begin{lemma}
The boolean algebra $B$ is a sheaf.
\end{lemma}

\begin{proof}
We use \cref{sigma-complete-sheaf-simpler}. It is clear that if $0=1$ then $B$ is contractible.

Given $(e_i)_{i:\N}$ fundamental system of idempotent in $B$, we check that the canonical map:
\[B\to B^{\sum_{i:\N} e_i=1}\]
is an equivalence. This map is equivalent to the map:
\[B\to \prod_{i:\N} B^{e_i=1}\]
which by the assumed duality is equivalent to the map:
\[B\to \prod_{i:\N} B_{e_i}\]
\[b \mapsto (b\land e_i)_{i:\N}\]
We can check that:
\[\prod_{i:\N} B_{e_i}\to B\]
\[(x_i)_{i:\N}\mapsto \lor_{i:\N} x_i\]
is inverse to this map using $a\land(\lor_{i:\N} b_i) = \lor_{i:\N} (a\land b_i)$ which is true as:
\[a\land(\lor_{i:\N} b_i)\leq\lor_{i:\N} (a\land b_i) \]
if and only if:
\[\lor_{i:\N} b_i\leq\neg a\lor (\lor_{i:\N} (a\land b_i)) \]
which holds.
\end{proof}

This means that we can define $\hat{B}$ in the sheaf topos simply as $B$ with this proof that it is a sheaf.

\begin{lemma}\label{bot-sheaf-sigma-complete}
We have that $\hat{\bot}$ is equivalent to $0=1$.
\end{lemma}

\begin{proof}
It is clear that $\hat{\bot}$ is a proposition and that:
\[0=1 \to \hat{\bot}\]
as $0=1$ implies all sheaves contractible by considering the empty system of idempotent. On the other hand since $B$ is a sheaf we have that $0=_B 1$ is a sheaf, so to prove that:
\[\hat{\bot}\to 0=1\]
it is enough to prove that:
\[\bot\to 0=1\]
\end{proof}

So we can interpret $\bot$ as $0=1$. We have a similar result for $\hat\N$, indeed:

\begin{lemma}\label{N-sheafification-sigma-complete}
We have that $\hat\N$ is equivalent to the sheafification of $\N$,
\end{lemma}

\begin{proof}
We just check by direct computation. Omitted, holds for any lex modality.
\end{proof}

It should be noted that $\hat\N$ and the sheafification of $\N$ do not compute the same. Nevertheless when proving the interpretation of a formula we can assume $\hat\N$ is the sheafification of $\N$, as long as the formula does not depend on the computation rules for $\N$.


\subsection{Limited principle of omniscience}

\begin{lemma}\label{B-is-2-sigma-complete}
In the sheaf topos, we have that $B=2$.
\end{lemma}

\begin{proof}
First we prove the interpretation of:
\[0\not=_B1\] 
in the sheaf topos. This is true as $0=1$ is interpreted as itself, and $\bot$ as $0=1$ by \cref{bot-sheaf-sigma-complete}.

The interpretation of:
\[\forall(e:B).\ e=1 \lor e=0\]
holds in the sheaf topos by considering the fundamental system of idempotents $e,1-e$.

These two properties imply $B=2$.
\end{proof}

The proof of the next theorem is more verbose than needed. The key point is that by \cref{N-sheafification-sigma-complete} we have that $\hat\N$ is equivalent to the sheafification of $\N$. 

\begin{lemma}\label{2-sigma-complete-sheaf}
In the sheaf topos, we have that $2$ is $\sigma$-complete.
\end{lemma}

\begin{proof}
By \cref{B-is-2-sigma-complete} we know it is enough to show that $B$ is $\sigma$-complete in the sheaf topos. 

By \cref{N-sheafification-sigma-complete}, we can assume $\hat\N$ the interpretation of $\N$ in the sheaf topos is the sheafification of $\N$.
\begin{itemize}
\item We need to define an inhabitant of:
\[(\hat\N \to B) \to B\]
but this type is actually equivalent to:
\[(\N\to B) \to B\]
so we can just use $B$ being $\sigma$-complete.
\item We need to prove that:
\[\forall (\alpha:\hat\N\to B)(n:\hat\N).\ \phi(n)\leq \hat{\lor}\alpha\]
but this type is actually equivalent to:
\[\forall (\alpha:\N\to B)(n:\N).\ \phi(n)\leq \lor\alpha\]
as $B$ and inequalites in $B$ are sheaves.
\item Similarly we need to prove:
\[\forall (e:B)(\alpha:\hat\N\to B).\ \big((\forall(n:\hat\N). \alpha(n)\leq e) \to \hat{\lor} \alpha \leq e\big)\]
but this type is equivalent to:
\[\forall (e:B)(\alpha:\N\to B).\ \big((\forall(n:\N). \alpha(n)\leq e) \to \lor \alpha \leq e\big)\]
as inequalities in $B$ are sheaves.
\end{itemize}
\end{proof}

We write $\iota: \N\to \hat\N$ the sheafification of $\N$.

\begin{lemma}\label{sup-gives-exists}
In the sheaf topos, given $(e_i)_{i:\N}$ a sequence of booleans we have that:
\[(\lor_{i:\N}e_i) = 1 \to \exists(i:\N).\ e_i=1\]
\end{lemma}

\begin{proof}
It is enough to prove this for $(e_i)$ in $\N_\infty$, i.e. assuming that if $k\not=l$ then $e_ke_l = 0$. By 
\cref{B-is-2-sigma-complete} we can replace $2$ by $B$, after interpretation we have to prove that given:
\[e:\hat\N\to B\] 
such that:
\begin{itemize}
\item For all $k,l:\hat\N$ if $k=l\to 0=1$ then $e_ke_l=0$.
\item $(\hat{\lor}_{i:\hat\N} e_i) = 1$.
\end{itemize}
then we have that:
\[\exists(i:\hat\N). e_i=1\]
Now we notice that $\hat\N$ is the sheafification of $\N$ and that:
\begin{itemize} 
\item The type:
\[\forall(k,l:\hat\N).\ (k=l\to 0=1)\to e_ke_l=0\]
is equivalent to:
\[\forall(k,l:\N).\ (k=l\to 0=1)\to e_{\iota(k)}e_{\iota(l)}=0\]
which implies:
\[\forall(k,l:\N).\ k\not=l\to e_{\iota(k)}e_{\iota(l)}=0\]
\item $\hat{\lor}_{i:\hat\N} e_i$ is defined as $\lor_{i:\N} e_{\iota(i)}$.
\item We have that:
\[\exists (i:\hat\N). e_i=1\]
is the sheafification of $\exists(i:\N).e_{\iota(i)} = 1$, which is a straightforward consequence of both types being sheaves that are propositions.
\end{itemize}
So it is enough to prove that given $(e_i)_{i:\N}$ a fundamental system of idempotents in $B$ we have the sheafification of $\exists(i:\N).e_i = 1$. This is immediate by the definition of a sheaf..
\end{proof}

\begin{theorem}[LPO]
In the sheaf topos, given $(e_i)_{i:\N}$ a sequence of booleans we have that:
\[\forall(i:\N).\ e_i=0 \lor \exists(i:\N).\ e_i=1\]
\end{theorem}

\begin{proof}
By \cref{2-sigma-complete-sheaf} we have a boolean $\lor_{i:\N}e_i$. If $(\lor_{i:\N}e_i) = 0$ then $\forall(i:\N).\ e_i=0$, if $(\lor_{i:\N}e_i) = 1$ we conclude by \cref{sup-gives-exists}.
\end{proof}


\subsection{Dependent choice}

\begin{theorem}[DC]
Dependent choice holds in the sheaf topos.
\end{theorem}

\begin{proof}
Recall that by \cref{prop-trunc-sheaf-sigma-complete} the propositional truncation can be interpreted as itself, and by \cref{N-sheafification-sigma-complete} we have that $\N$ can be interpreted as the sheafification of $\N$.

Then we see that the interpretation of dependent choice in the sheaf topos is equivalent to dependent choice for sheaves in the presheaf topos. We omit the details.
\end{proof}