% latexmk -pdflatex='xelatex %O %S' -pvc -pdf slides.tex
\documentclass{beamer}

\beamertemplatenavigationsymbolsempty

% für xelatex:
\usepackage{xltxtra}
\usepackage{unicode-math}

% literatur
\usepackage[backend=biber,style=alphabetic]{biblatex}

\addbibresource{../util/literature.bib}

\usepackage{../util/zariski}

\usepackage{csquotes}
\usepackage{hyperref}
\usepackage{tikz}
\usetikzlibrary{cd,arrows,shapes,calc,through,backgrounds,matrix,trees,decorations.pathmorphing,positioning,automata}
\usepackage{graphicx}
\usepackage{color}

\usepackage{mathpartir}
\newcommand{\yields}{\vdash}
\newcommand{\cbar}{\, | \,}


% für tabellen
\usepackage{booktabs}

\title[ConCoh]
{\v{C}ech Cohomology in Homotopy Type Theory}
\author[Author, Anders] 
{Ingo Blechschmidt, Felix Cherubini, David Wärn}

\begin{document}

\date{}
\begin{frame}
  \titlepage
\end{frame}

\begin{frame}
  \vspace{1cm}
  \begin{center}
    \begin{tikzpicture} 
      [ 
      node distance=.8cm, 
      circled/.style={draw, ellipse, ultra thick, fill=blue!12},
      edge from parent/.style={very thick,draw=black,-latex},
      plaintext/.style={}
      ]

      \tikzstyle{level 1}=[sibling angle=81,level distance=4cm]

      \node[circled] (hott) {HoTT+Axioms} [counterclockwise from=207]
      child { 
        node[circled] (grp) {$\infty$-Groupoids} 
        edge from parent [-latex] node (grp) 
        {
          \begin{tikzpicture}[scale=0.8, rotate=25]
            \draw[-, red, ultra thick] (0,0) to (1,1);
            \draw[-, red, ultra thick] (1,0) to (0,1);
          \end{tikzpicture}
        } 
      }
      child { node[matrix, circled, inner sep=0pt] (smgrp) 
        {
          \node[circled, minimum height=0.7cm, minimum width=3cm] (mfd) {Schemes};
          \node[below of=mfd, text width=3cm,
          node distance=0.9cm] 
          {``Cubical Zariski-sheaves''}; \\
        }
      }
      ;
    \end{tikzpicture}
  \end{center}
  \vspace{1cm}
  {\footnotesize * Schemes = quasi-compact, quasi-separated schemes of finite type}
\end{frame}

\begin{frame}
  \frametitle{Reminder: The 3 Axioms}
  \textbf{Axiom:} We have a local, commutative ring $R$. \\

  \pause
  \vspace{0.25cm}
  For a finitely presented $R$-algebra $A$, define:
  \[ \Spec(A):\equiv \Hom_{R\text{-algebra}}(A,R)\]
  \pause
  \textbf{Axiom (synthetic quasi-coherence (SQC)):} \\
  For any finitely presented $R$-algebra $A$, the map
  \[ a\mapsto (\varphi\mapsto \varphi(a)):A \xrightarrow{\sim} R^{\Spec(A)} \]
  is an equivalence.

  \pause
  \textbf{Axiom (Zariski-local choice):}\\
  For every surjective $\pi$, there merely exist local sections $s_i$
  \[ \begin{tikzcd}[ampersand replacement=\&, column sep=small]
    \& E \ar[d, two heads, "\pi"] \\
    D(f_i) \ar[r, hook] \ar[ur, bend left, dashed, "s_i"] \& \Spec(A)
  \end{tikzcd} \]
  with $f_1, \dots, f_n : A$ coprime.
\end{frame}

\begin{frame}
  \vspace{0.25cm}
  For $A : X \to \mathrm{Ab}$, define \emph{cohomology} as:
  \[ H^n(X, A) :\equiv \Big\| \prod_{x:X}K(A_x,n) \Big\|_{\mathrm{set}} \]
  \begin{itemize}
  \item LES and MVS.
  \item $H^n(X,M)=0$ if $n>0$, $X$ is affine and $M:X\to \Mod{R}_{\mathrm{wqc}}$.
  \item $H^n(X,M)$ coincides with \v{C}ech-Cohomology (for \emph{separated} schemes).
  \item By work of Blechschmidt and Wärn: A scheme $X$ is affine if and only if
      \[ H^n(X, M) = 0 \]
      for all $M : X \to R\text{-}\mathrm{Mod}_{\mathrm{wqc}}$ and $n > 0$.
  \end{itemize}
\end{frame}

\begin{frame}
  \frametitle{Example}
  Ideally an example showing how \v{C}ech-Cohomology works.
  Or: Explanation of the proof ideas...
\end{frame}

\begin{frame}
  \centering
  Thank you!
\end{frame}

\end{document}
