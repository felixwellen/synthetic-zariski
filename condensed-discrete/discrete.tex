\section{Weakly discrete types}

There is a tower:
\[N_0 \leftarrow N_1 \leftarrow N_2 \leftarrow \cdots\]
where $N_k = \mathrm{Fin(k+1)}$ the map $N_{k+1}\to N_k$ sends $k$ to $k-1$ and is the identity on other element. Then:
\[\mathrm{lim}_kN_k = \Noo\]

\begin{definition}
A type $X$ is called weakly discrete if the canonical map:
\[\mathrm{colim}_kX^{N_k} \to X^{\Noo}\]
is an equivalence.
\end{definition}

Discrete clearly implies weakly discrete. We look at truncated types:

\begin{lemma}
Any proposition is weakly discrete.
\end{lemma}

\begin{proof}
As $\Noo$ and $N_k$ are all inhabited.
\end{proof}

\begin{lemma}\label{weakly-discrete-sets}
If $X$ is a set, $X$ is discrete if and only if every convergent sequence in $X$ is eventually constant, or equivalently if any map $\Noo\to X$ merely factor through a finite set.
\end{lemma}

\begin{proof}
If $X$ is a set, the map:
\[\mathrm{colim}_kX^{N_k} \to X^{\Noo}\]
is always an embedding, and the conditions precisely means it is surjective.
\end{proof}

\begin{remark}
Presumably, we have that $X$ truncated is weakly discrete if and only if a map from $\Noo$ to any iterated identity type of $X$ is eventually constant.
\end{remark}

\begin{lemma}
Weakly discrete types are stable under $\Sigma$-types and identity types.
\end{lemma}

\begin{proof}
By commutation of sequential colimits with $\Sigma$-types and identity types.
\end{proof}

\begin{lemma}
Any type with open identity types is weakly discrete.
\end{lemma}

\begin{proof}
TODO
\end{proof}


\section{Discrete types}

\begin{definition}
A type $X$ is called discrete given any tower $(S_k)_{k:\N}$ of finite types, the canonical map:
\[\mathrm{colim}_kX^{S_k} \to X^{\mathrm{lim}_kS_k}\]
is an equivalence.
\end{definition}

\begin{lemma}
If $X$ is a proposition the following are equivalent:
\begin{enumerate}[(i)]
\item $X$ is discrete.
\item For all $(D_n)_{n:\N}$ decidable propositions:
\[((\forall_nD_n)\to P) \leftrightarrow \exists_n(D_n\to P)\]
\item For all $C$ closed proposition:
\[(C\to P) \leftrightarrow \not C\lor P\]
\end{enumerate}
\end{lemma}

\begin{proof}
TODO
\end{proof}

\begin{corollary}
Any open proposition is discrete.
\end{corollary}

\begin{lemma}
If $X$ is a set, $X$ is discrete if and only if for all $S:\Stone$, any map $S\to X$ merely factor through a finite type.
\end{lemma}

\begin{proof}
As in \Cref{weakly-discrete-sets}.
\end{proof}

\begin{lemma}
Weakly discrete types are stable under $\Sigma$-types and identity types.
\end{lemma}

\begin{proof}
By commutation of sequential colimits with $\Sigma$-types and identity types.
\end{proof}

\begin{lemma}
Any overtly discrete type is discrete.
\end{lemma}

\begin{proof}
TODO
\end{proof}

\begin{lemma}
Given any overtly discrete abelian group $A$, the type $BA$ is discrete.
\end{lemma}

\begin{proof}
TODO
\end{proof}

Questions:
\begin{itemize}
\item If $S$ is Stone and $A$ is a discrete abelian group, do we have $H^1(S,A) = 0$? 
\item What about higher cohomology groups? 
\item What about $H^1(\I,A)$? 
\item Is any discrete type $\I$-local? 
\item Do we have that discrete and overt sets are the same as overtly discrete sets?
\end{itemize}