Based on Barton-Commelin axioms. Going toward:

\begin{definition}
A type is overtly discrete if it merely is a sequential colimit of finite type.
\end{definition}

We write $\ODisc$ for the type of overtly discrete sets.

\begin{proposition}
  If $E:\ODisc$ and $e_0,e_1:E$ then $e_0=_E e_1$ is open. If $P$ is a proposition, then $P:\ODisc$
  iff $P:\Open$.
\end{proposition}

\begin{proposition}
  If $B$ is a Boolean algebra, we have $B:\Boole$ iff $B:\ODisc$.
\end{proposition}

\begin{theorem}
Given $X,Y:\mathrm{ODisc}$, the fiber of the map:
\[ (\mathrm{Open}\to \mathrm{ODisc}) \to \mathrm{ODisc}\times \mathrm{ODisc}\]
\[ P \mapsto (P(\bot),P(\top))\]
over $(X,Y)$ is:
\[X\to Y\]
\end{theorem}

Intuitively this means that the type of open propositions is a directed interval for overtly discrete types.

In particular, any map $\Open\rightarrow\Open$ is monotone.

\begin{theorem}[Tychonoff]
  If $E:\ODisc$ and $X:E\rightarrow\CHaus$ then $\Pi_EX:\CHaus$. If $X:\CHaus$ and $E:X\rightarrow\ODisc$ then
  $\Pi_XE:\ODisc.$ In particular if $E:X\rightarrow\Open$ then $\Pi_XE:\Open$.
\end{theorem}

