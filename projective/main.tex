% latexmk -pdf -pvc main.tex
\documentclass{../util/zariski-small}



\title{Projective Space in Synthetic Algebraic Geometry}

\begin{document}


\author{Felix Cherubini, Thierry Coquand, Matthias Ritter and David Wärn}

\maketitle

\begin{abstract}
  Synthetic algebraic geometry is a new approach to algebraic geometry. It consists in using homotopy type theory extended with three axioms, together with the interpretation of these in a higher version of the Zariski topos, in order to do algebraic geometry internally to this topos.
  
  In this article, we will show basic properties of projective n-space $\bP^n$ in synthetic algebraic geometry.
  In particular, we show that the automorphism group of $\bP^n$ is $\PGL_{n+1}(R)$ and that the picard group is $\Z$.
  We will provide different proofs of the latter statement, where the most synthetic approach naturally leads to the refined statement that the type of line bundles on $\bP^n$ is the higher type $\Z\times K(R^\times,1)$, where $K(R^\times,1)$ is a delooping of the group of units of the internal base ring $R$.
\end{abstract}

\section*{A brief introduction to synthetic algebraic geometry}

In mathematics, it is common practice to assume a fixed set theory, usually with the axiom of choice, as a common basis. While it is a great advantage to work in one common language and share a lot of the basic constructions, the dual approach of adapting the  ``base language'' to particular mathematical domains is sometimes more concise, provides a new perspective and encourages new proof techniques which would be hard to find otherwise.
We use the word ``synthetic'' to indicate that the latter approach is used,
as it was used by Kock and Lawvere to describe developement of mathematics internal to certain categories \cite{lawvere-categorical-dynamics}, in particular toposes -- a program which dates back as far as 1967.

Already in the 70s in the same program, Anders Kock suggested to use the language of higher-order logic \cite{Church40} to describe the Zariski topos, the collection of sheaves for the Zariski topology \cite{Kock74,kockreyes}, which is the first occurence of synthetic algebraic geometry.
Kock's approach allowed for a more suggestive and geometrical description of schemes.
There is in particular a ``generic local ring'' $R$, which, as a sheaf, associates to any algebra $A$ its underlying set and, as described in \cite{kockreyes}, the projective space $\bP^n$ is then the set of lines in $R^{n+1}$.

Just using category theory is not the same as reasoning synthetically -- for the latter the goal is usually to derive results exclusively in one system,
as Kock and Lawvere did with differential geometry in his work.
The distinction with just using an abstraction like categories is important, since the translation from the synthetic language and back can become cumbersome -- although it is still the goal to derive statements about ordinary mathematical objects in the end.

Starting with Kock and Lawvere's work, more differential geometry was developed synthetically \cite{kock-sdg} along with a study of the models of the theory \cite{moerdijk-reyes}.
One basic axiom of the theory, called the Kock-Lawvere axiom, allows for reasoning with nilpotent infinitesimals. Our version of synthetic algebraic geometry uses a generalisation of this axiom called the duality axiom. Let us now describe the Kock-Lawvere axiom.

The Kock-Lawvere axiom is added to a basic language which can be interpreted in good enough categories, for example toposes. More precisely, we need basic objects like $\emptyset$, $\{\ast\}$ and $\N$ as well as natural constructions like $A\times B$ or $A^B$ for objects $A$, $B$. These constructions come with data, like the projections in the case of $A\times B$, satisfying natural laws. We also need predicates $P(x)$ for elements $x:A$ so we can form subobjects like $\{x:A\mid P(x)\}$.
In this language, we assume there is a fixed ring $R$, which can be thought of as the real numbers. We define $\D(1)=\{x\in R\mid x^2=0\}$ to be the set of all square-zero elements of $R$, then the Kock-Lawvere axiom gives us a bijection
\[ e : R \times R  \to R^{\D(1)} \]
which commutes with evaluation at $0$ and projection to the first factor.
The intuition is that $\D(1)$ is so small that any function on it is linear and therefore determined by its value and its derivative at $0\in\D(1)$.
With this axiom, the derivative at $0:R$ of a function $f : R \to R$ may then be defined as $\pi_2(e^{-1}(f_{\vert \D(1)}))$. This is the start of a convenient development of differential calculus, which doesn't require any further structures on $R$ or other objects. This is the core of the synthetic method: we can work with these differential spaces as if they were sets.

To give an example, the tangent bundle of a manifold $M$ can be defined as $M^{\D(1)}$ and vector fields as sections of the map $M^{\D(1)}\to M$ evaluating at $0$. Then it is easy to see that a vector field is the same as a map $\zeta:\D(1)\to M^M$ with $\zeta(0)=\id_M$, which can be interpreted as an infinitesimal transformation of the identity map. This style of reasoning with spaces as if they were sets is also central in current synthetic algebraic geometry. 

The Kock-Lawvere axiom above are incompatible with the law of excluded middle (LEM) and therefore also with the axiom of choice (AC). Indeed they imply all maps from say $R$ to $R$ are differentiable, which contradicts LEM. 
However restricted versions of LEM and AC are compatible with this axiom. A very basic example is that equality of natural numbers is decidable, meaning that two natural numbers are either equal or not equal. We will latter go back to why full LEM and AC tend to be incompatible with synthetic approach to various geometry.

%TODO: Hugo is here in the rereading

The use of nilpotent elements to capture infinitesimal quantities as mentioned above was inspired by the Grothendieck school of algebraic geometry and Anders Kock also worked with an extended axiom \cite{Kock74,kockreyes} suitable for synthetic algebraic geometry, where the role of $\D(1)$ above can be taken by any finitely presented affine scheme. In his 2017 doctoral thesis, Ingo Blechschmidt noticed a property holding internally in the Zariski-topos, which he called synthetic quasi-coherence. It is generalised and internalised version of what Kock used. In 2018, David Jaz Myers\footnote{Myers' never published on the subject, but communicated his ideas to Felix Cherubini and in talks to a larger audience \cite{myers-talk1,myers-talk2}.} started working with a specialization of Blechschmidt's synthetic quasi-coherence, which is what we now call \emph{duality axiom}.

To state the duality axiom we need to go from the space $\D(1)$ to spaces that are the common zeros of some finite system of polynomial equations over $R$. Such a space can be encoded independently of the choice of polynomials as a finitely presented $R$-algebra, i.e.\ an $R$-algebra $A$ which is of the form $R[X_1,\dots,X_n]/(P_1,\dots,P_l)$ for some numbers $n,l$ and polynomials $P_i\in R[X_1,\dots,X_n]$.
Then the set of roots of the system is given by the type $\Hom_{\Alg{R}}(A,R)$ of $R$-algebra homomorphisms from $A$ to the base ring. We denote this type by $\Spec A$.
Now the duality axiom states that $\Spec$ is the inverse to exponentiating with $R$, i.e.\ for all 
finitely presented $R$-algebras $A$ the following is an isomorphism:
\[ (a\mapsto (\varphi\mapsto \varphi(a))) : A\to R^{\Spec A}\rlap{.}\]

Myers used homotopy type theory as a base language, which is now the standard in synthetic algebraic geometry. Now we introduce homotopy type theory, in the next paragraphs we will explain how it fits with synthetic algebraic geometry. Homotopy type theory is a language for synthetic homotopy theory.
This means that when using it, we can think of the basic objects of the theory, that is types, directly as homotopy types. This should be contrasted with the usual practice in algebraic topology, which is to implement these homotopy types as topological spaces or Kan complexes.
So the rules of homotopy type theory allow to work with types in very much the same way as one would work with homotopy types in traditional mathematics. % -- with the clear exception of the law of excluded middle and the axiom of choice - although the former and restricted versions of the latter can be assumed.

On the other hand we also use homotopy type theory because it allows to reason synthetically about spaces, as plain type theory does. A key point is that we do not use the law of excluded middle (LEM) or the axiom of choice (AC), which are incompatible with types being interpreted as spaces. Indeed on one hand LEM allows us to find a complement of each subset of a given type $A$, which exposes $A$ as a coproduct.
This is not true for spaces, for example, $\R$ is not the coproduct of the topological subspaces $\{0\}$ and $\R/\{0\}$.
On the other hand AC states that any surjection has a section. This is also not true for any sensible notion of space, in particular it would trivialise all cohomology.
Thus, constructive reasoning in the sense of not using LEM and AC is a necessity if we want to types to be understood as having a spatial structure. It turns out that this is the only obstruction, so the rules of type theory allow to work with type as one would work with spaces in algebraic geometry.

In synthetic algebraic geometry, we work inside homotopy type theory so that types behave both as homotopy types and as spaces from algebraic geometry. This means that we are mixing two synthetic approaches, combining their advantages,
which rests on the possibility of interpreting homotopy type theory in any higher topos \cite{shulman2019all} and not just the higher topos of $\infty$-groupoids. More precisely we think of the higher topos of Zariski sheaves with value in homotopy type. %We remind readers of this by using the notation $x:X$ which can often be thought of as $x\in X$.
The general idea of using homotopy type theory to combine some kind of synthetic, spatial reasoning with synthetic homotopy theory, goes back at least to 2014, to Mike Shulman and Urs Schreiber \cite{Schreiber_2014}.
Schreiber suggested to the HoTT community at various occasions to make use of HoTT as the internal language of higher toposes, where specifities of the topos are accessed in the language via modalities.
This approach was shown to be quite effective and intuitive in Shulman's \cite{shulman-Brouwer-fixed-point} work on mixing synthetic homotopy theory in the form of HoTT and a synthetic approach to topology using a triple of modalities -- a structure called cohesion by Lawvere \cite{Lawvere2007}. A more detailed introduction to homotopy type theory for a general mathematical audience, with an emphasis on this mix of homotopical and spatial structure can be found in \cite{shulman-logic-of-spaces}.

One of the main advantages of using specifically homotopy type theory and not plain type theory, is using synthetic homotopical reasoning to make cohomological computations. Indeed one of Schreiber's motivation was to make use of the modern perspective on cohomology as the connected components of a space of maps in a higher topos. This can be mimicked in HoTT as follows: Given $X$ a type, $A$ an abelian group and $n:\N$, we define the $n$-th cohomology group of $X$ with coefficients in $A$ as
\[ H^n(X,A):=\| X\to K(A,n) \|_0\]
 where $\|\_\|_0$ is the $0$-truncation, an operation which turns any type into a $0$-type, that is a type with trivial higher structure. The type $K(A,n)$ is the $n$-th Eilenberg MacLane space, which can always be constructed for any abelian group $A$ and comes with an isomorphism $\Omega^n(K(A,n))\simeq A$.
With this definition of cohomology groups we can use synthetic homotopy theory to reason about cohomology, which had already been done successfully for the cohomology of homotopy types like spheres and finite cell complexes. It also works for the cohomology of $0$-types such as spaces in synthetic algebraic geometry.
This internal version of cohomology does not agree with the external version mentioned above, indeed it is a sheaf of groups instead of a single group, and it is indexed by an internal natural number instead of an external one. Nevertheless, internal cohomology turned out to be quite useful in practice.

In 2022, trying to use this approach to calculate cohomology groups in synthetic algebraic geometry led to the discovery of what is now called Zariski-local choice \cite{draft},
which is an additional axiom that holds in the higher Zariski-topos.
It is a weakening of the axiom of choice. In homotopy type theory, the axiom of choice can be formulated as follows: For any surjective map $f:X\to Y$, there exists a section, i.e.\ a map $s:Y\to X$ such that $f\circ s=\id_Y$.
Zariski-local choice also states the existence of a section, but only Zariski-locally and only for surjections into an affine scheme: For any surjection $f:E\to \Spec A$,
there exists a Zariski-cover $U_1,\dots,U_n$ of $\Spec A$ and maps $s_i:U_i\to E$ such that $f(s_i(x))=x$ for all $x\in U_i$.

In homotopy type theory, we use the propositional truncation $\|\_\|$ to define surjections and more generally what we mean by ``exists''.
Propositional truncation turns an arbitrary type $A$ into a type $\|A\|$ with the property that $x=y$ for all $x,y:\|A\|$.
Types with this property are called propositions or (-1)-types in homotopy type theory.
Using a univalent universe of types $\mathcal U$ we have that surjection into a type $A$ are the same as type families $F:A\to \mathcal U$, such that we have $\|F(x)\|$ for all $x: A$.
Using type families instead of maps allows us to drop the condition that the maps we get are sections, since we can express it using dependent function types and we arrive at the formulation of Zariski-local choice given below in the list of axioms.
In this instance and many others, homotopy type theory is much more convenient for formal reasoning, which is an advantage when formalizing synthetic algebraic geometry.

In total, the system we use for synthetic algebraic geometry consists of the extension of homotopy type theory postulating a fixed commutative ring $R$ satisfying these three axioms (see below for an explanation of the first one):

\begin{center}
\begin{axiom}[Locality]%
  \label{loc-axiom}
  $R$ is a local ring, i.e.\ $1\neq 0$ and whenever $x+y$ is invertible then $x$ is invertible or $y$ is invertible.
\end{axiom}

\begin{axiom}[Duality]%
  \label{duality-axiom}
  For any finitely presented $R$-algebra $A$, the homomorphism
  \[ a \mapsto (\varphi\mapsto \varphi(a)) : A \to (\Spec A \to R)\]
  is an isomorphism of $R$-algebras.
\end{axiom}

\begin{axiom}[Zariski-local choice]%
  \label{Z-choice-axiom}
  Let $A$ be a finitely presented $R$-algebra
  and let $B : \Spec A \to \mU$ be a family of inhabited types.
  Then there exists a Zariski-cover $U_1,\dots,U_n\subseteq \Spec A$
  together with dependent functions $s_i : (x : U_i)\to B(x)$.
\end{axiom}
\end{center}

As we explained above the duality axiom is a generalisation of the Kock-Lawvere axiom, which was used for convenient infinitesimal computations. It has a lot of consequences. In line with classical algebraic geometry, it shows that we have an anti-equivalence between finitely presented $R$-algebras and affine schemes of finite presentation over $R$.
More surprisingly, it implies that all functions in $R\to R$ are polynomials, as well as various the properties of the base ring $R$.
For example, for all $x:R$, we have that $x$ is invertible if and only if we have $x\neq 0$.
It also implies that any map from an affine schemes to the natural numbers is bounded.

Surprisingly, the Zariski-local choice axiom was also usable to solve problems which have no obvious connection to cohomology.
For example, it implies that pointwise open subsets of an affine scheme are the same as subsets which are given by unions of non-vanishing sets of functions on the scheme.
In more detail, we an open proposition is a proposition of the form $r_1\neq 0 \vee\dots\vee r_n\neq 0$ where $r_i:R$.
Then a subset $U$ of a type $X$  is called open if the proposition $x\in U$ is open for all $x:X$.
Given an open subset $U$ of $\Spec A$, using Zariski-local choice we turn these elements $r_1,\dots,r_n$ of the base ring into functions defined Zariski-locally on $\Spec A$.
We can then even prove that $U$ is a union of non-vanishing sets $D(f_i)$ of global functions $f_i:\Spec A \to R$.
An analogous result holds for closed propositionsand vanishing sets of functions on affine schemes, where closed propositions are propositions of the form $r_1=0\wedge\dots\wedge r_n=0$ where $r_i:R$.

This connection between pointwise and Zariski-local openness is crucial to make the synthetic definition of a scheme work well:
A scheme is a type $X$, that merely has a finite open cover by affine schemes.
To produce interesting examples, it is necessary to use the locality axiom.
This is related to the Zariski topology and ensures that classical examples of Zariski covers can be reproduced.
A central example is the projective spaces $\bP^n$, which can be defined as the quotients of $R^{n+1}/\{0\}$ by the action of $R^\times$ by scaling.
A cover of $\bP^n$ is given by sets of equivalence classes of the form $\{[x_0:\cdots:x_n] \vert x_i\neq 0 \}$, which is clearly open using the pointwise definition.
To see that it is a cover, one has to note that for $x:R^{n+1}$, we have that $x\neq 0$ is equivalent to one of the entries $x_i$ being different from $0$. In synthetic algebraic geometry, this is the case for the base ring $R$ and the proof uses that $R$ is a local ring.


\section*{Content of this article}
Grothendieck advocated for a functor of points approach to schemes early in his
project of foundation of algebraic geometry (see the introduction of \cite{EGAIV1}).
In this approach, a scheme is defined as a special kind of (covariant) set valued functor
on the category of finitely presented commutative ring. This functor should in particular
be a sheaf w.r.t.\ the Zariski topology. As a typical example, the projective space $\bP^n$
is the functor, which to a ring $A$ associates the set of finitely presented sub-modules of $A^{n+1}$, which are direct factors \cite{Demazure,Eisenbud,Jantzen}.

In the 70s, Anders Kock suggested to use the language of higher-order logic \cite{Church40}
to describe the Zariski topos, the collection of sheaves for the Zariski topology \cite{Kock74,kockreyes}.
This allows for
a more suggestive and geometrical description of schemes, that has now seen as special kind
of types satisfying some properties and a map of schemes in this setting is just any map.
There is in particular a ``generic
local ring'' $R$, which associates to $A$ its underlying set. As described in \cite{kockreyes}
the projective space $\bP^n$ is then the set of lines in $R^{n+1}$.

A natural question is if we can show in this setting that the automorphism group of $\bP^n$
is  $PGL_{n+1}(R)$.
More generally, can we show that any map $\bP^n\rightarrow \bP^m$ is given by $m+1$ homogeneous
polynomials of same degree in $n+1$ variables?
It is possible from this to deduce the corresponding result about $\bP^n$ as defined
as functor of points (but the maps are now {\em natural transformations}) or about $\bP^n$ as
defined as a scheme (but the maps are now {\em maps of schemes}).
(This result, though fundamental, is surprisingly not in \cite{Hartshorne}.)
One goal of this paper is to present such a proof.

In \cite{draft}, we presented an axiomatisation of the Zariski {\em higher topos} \cite{lurie-htt},
using instead of the language of higher-order logic the language of dependent type theory
with univalence \cite{hott}. The first axiom is that we have a local ring $R$. We define
then an affine scheme to be a type of the form $Sp(A) = Hom_{R-alg}(A,R)$ for some finitely presented
$R$-algebra $A$. The second axiom, inspired from the work of Ingo Blechschmidt \cite{ingo-thesis},
states that the evaluation map $A\rightarrow R^{Sp(A)}$ is a bijection. The last axiom states
that each $Sp(A)$ satisfies some form of local choice \cite{draft}. We can then define a notion
of {\em open} proposition, with the corresponding notion of open subset, and define a scheme as a type
covered by a finite number of open subsets that are affine schemes. We can then in particular define
$\bP^n$ as in \cite{kockreyes} and show that it is a scheme.
We show in this setting, dependent type theory with univalence extended with these 3 axioms,
the above result about maps between $\bP^n$ and $\bP^m$ and the result about automorphisms of $\bP^n$.

Interestingly, though these results are
about the Zariski $1$-topos, the proof makes use of types that
are not (homotopy) sets (in the sense of \cite{hott}),
since it proceeds in characterizing $\bP^n\rightarrow\KR$, where $\KR$ is the delooping
(thus a type which is not a set) of the multiplicative group of units of $R$.
More technically, we also use such higher types as an alternative of the technique
of Quillen patching \cite{Quillen,lombardi-quitte,Lam}.



%% Schemes as special kind of sheaf for Zariski topos.

%% Even nicer in a type theoretic framework

%% Anders Kock property of Zariski topos.

%% Zariski topos higher logic

%% Definition of $\bP^n$ as a set of lines in $R^{n+1}$ coincides with the definition
%% of projective as functor of points (Demazure? Eisenbud?)

%% ``Geometric'' definition

%% Meyers, Blechschmidt use of type theory with univalence

%% Axiomatisation of the Zariski (higher) topos

%% A scheme is defined as a type satisfying some property and a map of schemes is {\em any} function
%% between the corresponding types



\paragraph{Acknowledgements.}
The idea to use the topological characterization of stone spaces as totally disconnected, compact Hausdorff spaces to prove \Cref{stone-sigma-closed} was explained to us by Martín Escardó.
We profited a lot from a discussion with Reid Barton and Johann Commelin. 
David Wärn noticed that Markov's principle (\Cref{MarkovPrinciple}) holds. 
At TYPES 2024, we had an interesting discussion with Bas Spitters on the topic of the article.
Work on this article was supported by the ForCUTT project, ERC advanced grant 101053291.


\section[Definition of projective space and some linear algebra]{Definition of $\bP^n$ and some linear algebra}

We give two definitions of projective space, which differ only in size.

\begin{definition}
  \begin{enumerate}
  \item An $n$-dimensional $R$-\notion{vector space} is an $R$-module $V$,
    such that $\| V = R^n \|$. 
  \item We write $\Vect{R}{n}$ for the type of these vector spaces and $V\setminus\{0\}$ for the type
    \[ \sum_{x:V}x\neq 0\]
  \item A \notion{vector bundle} on a type $X$ is a map $V:X\to \Vect{R}{n}$. 
  \end{enumerate}
\end{definition}

The following defines projective space as the space of lines in a vector space.
This is a large type.
We will see below, that there is also a small definition of the same type.

\begin{definition}
  \begin{enumerate}
  \item   A \notion{line} in a $R$-vector space $V$ is a subtype $\mathcal L:V\to \Prop$,
    such that there exists an $x:V\setminus\{0\}$ with
    \[ \prod_{y:V}\left(\mathcal L (y) \Leftrightarrow \exists c:R.y=c\cdot x\right)\]
  \item The space of all lines in a fixed $n$-dimensional vector space $V$ is the projectivization of $V$:
    \[ \bP(V)\colonequiv \sum_{\mathcal L:V\to \Prop} \mathcal L \text{ is a line}  \]
  \item \notion{Projective $n$-space} is the projectivization of $\bA^{n+1}$,
    $\bP^n \colonequiv \bP(\bA^{n+1})$.
  \end{enumerate}
\end{definition}

Lines are closed subschemes (not defined yet):

\begin{proposition}
  For any line $\mathcal L : \bA^{2}\to \Prop$, there is a degree one polynomial $P\in R[X_0,X_1]$ such that
  for all $x:\bA^{2}$, $\mathcal L(x)$ is equivalent to $P(x)=0$.
\end{proposition}
\begin{proof}
  Take $P$ to be the polynomial, given by inner product 
\end{proof}

\ignore{
\begin{definition}
  Let $n:\NN$. Projective $n$-space is the type given 
\end{definition}
}

\begin{theorem}[\axiomref{sqc},\axiomref{loc}]
  $\bP^n$ is a scheme.
\end{theorem}

\begin{proof}
  \dots
\end{proof}

\begin{lemma}
  All functions $\bP^1 \to R$ are constant.
\end{lemma}

\begin{proof}
  \dots
\end{proof}

\begin{lemma}[using (\axiomref{sqc}), (\axiomref{loc})]
  Let $p \neq q \in \bP^n$ be given.
  Then there exists a map $f : \bP^1 \to \bP^n$
  such that $f([0 : 1]) = p$, $f([1 : 0]) = q$.
\end{lemma}

\begin{proof}
  What we want to prove is a proposition,
  so we can assume chosen $a, b \in \bA^{n+1} \setminus \{0\}$
  with $p = [a]$, $q = [b]$.
  Then we set
  \[ f([x, y]) \colonequiv [xa + yb] \rlap{.}\]
  Let us check that $xa + yb \neq 0$.
  By \dots,
  we have that $x$ or $y$ is invertible
  and both $a$ and $b$ have at least one invertible entry.
  If $xa = - yb$
  then it follows that $x$ and $y$ are both invertible
  and therefore $a$ and $b$ would be linearly equivalent,
  contradicting the assumption $p \neq q$.
  Of course $f$ is also well-defined
  with respect to linear equivalence in the pair $(x, y)$.
\end{proof}

\begin{lemma}
  Let $n \geq 1$.
  For every point $p \in \bP^n$,
  we have $p \neq [1 : 0 : 0 : \dots]$
  or $p \neq [0 : 1 : 0 : \dots]$.
\end{lemma}

\begin{proof}
  Let $p = [a]$ with $a \in \bA^{n+1} \setminus \{0\}$.
  By \dots,
  there is an $i \in \{0, \dots, n\}$ with $a_i \neq 0$.
  If $i = 0$ then $p \neq [0 : 1 : 0 : \dots]$,
  if $i \geq 1$ then $p \neq [1 : 0 : 0 : \dots]$.
\end{proof}

\begin{theorem}
  All functions $\bP^n \to R$ are constant,
  that is,
  \[ H^0(\bP^n, R) \colonequiv (\bP^n \to R) = R \rlap{.} \]
\end{theorem}

\begin{proof}
  \dots
\end{proof}


%\section{Horrocks Theorem}
%We will need the following special case of Horrocks Theorem, for a commutative ring $A$.

\begin{lemma}\label{Horrocks}
  If an ideal of $A[X]$ divides a principal ideal $(f)$ with $f$ monic then it is itself a principal ideal.
\end{lemma}

Let $I$ and $J$ be such that $I\cdot J = (f)$. We can then write $f = \Sigma u_iv_i$ with $u_i$ in $I$ and
$v_i$ in $J$. We then have $I = (u_1,\dots,u_n)$ and $J=(v_1,\dots,v_n)$.
The strategy of the proof is to build comaximal monoids $S_1,\dots,S_l$ in $A$ \cite{LQ} such
that $I$ is generated by a monic polynomial in each $A_{S_j}[X]$.

\subsection{Formal computation of gcd}

 We start by describing a general technique introduced in \cite{LQ}.

If we have a list $u_1,\dots,u_n$ of polynomials over a field we can compute the gcd
so that $(g) = (u_1,\dots,u_n)$ and $g$ is $0$ or a monic polynomial.

In general if we are now over a ring $R$, we can interpret this computation formally as
follows. We build a binary tree of root $R$. At each node of the tree we have a f.p. extension $A$
of $R$. If we want to decide whether an element $a$ in $R$ is invertible or $0$
\footnote{A priori it is an element
of $A$, but we can always assume that this element comes from an element of $R$.}
we open two branches: one with $A\rightarrow A/(a)$ (intuitively we force $a$ to be $0$)
and the other with $A\rightarrow A_a = A[1/a]$ (intuitively we force $a$ to be invertible).

In this way we have at each leaf a f.p. extension $R\rightarrow A$ and in $A$ we have
$g$, a monic polynomial in $A[X]$ or $0$, such that $(g) = (u_1,\dots,u_n)$ in $A[X]$.
Over each branch we have a list of elements $a_1,\dots,a_n$ of $R$ that we force to be
invertible, and a list of elements $b_1,\dots,b_m$ of $R$ that we force to be $0$.
We associate to this branch the multiplicative monoid generated by $a_1\dots a_n$
and $1 + (b_1,\dots,b_m)$. In this way, we build a list of monoids $S_1,\dots,S_l$
that are {\em comaximal} \cite{LQ}: if $s_i$ in $S_i$ then $1 = (s_1,\dots,s_l)$.

\subsection{Application to Horrocks' Theorem}

We assume $f = \Sigma u_iv_i$ and $fp_{ij} = u_iv_j$ with $\Sigma p_{ii} = 1$
in $A[X]$. The goal is to build comaximal monoids $S_1,\dots,S_l$ with $I$ generated by
a monic polynomial in $A_{S_j}[X]$.

We first build a binary tree which corresponds to the formal computation of the gcd of
$u_1,\dots,u_n$ as described above. To each branch we associate an element that
we force to be invertible and a list of elements $b_1,\dots,b_m$ that we force to be $0$.
We write $S$ the multiplicative monoid generated by $a$ and $1 + (b_1,\dots,b_m)$.
We also have a monic polynomial $\gamma$ in $A_S[X]$ 
such that $I = (\gamma)$ in $A_S[X]/(b_1,\dots,b_m)$.

 Note that $I = (u_1,\dots,u_n)$ contains $f$.

\begin{lemma}
  If $p$ is a polynomial in $I$ which is monic in $A_S[X]/(b_1,\dots,b_m)$ of degree $<deg(f)$
  then there exists $h$ monic in
  $A_S[X]$ and in $(u_1,\dots,u_n)$ and such that $p=h$ mod. $(b_1,\dots,b_m)$.
\end{lemma}

\begin{proof}
  (Same proof as in Lam \cite{Lam}.) Let $N$ be the degree of $f$.
  If $I$ also contains a polynomial $q$ which is monic
  mod. $L$ of degree $N-1$, we can kill all coefficients (in $L$) of degree $\geqslant N$
  using $f$, and we get that $I$ also contains a monic polynomial of degree $N-1$
  and equal to $q$ mod. $L$.
  Similarly $I$ will also contains a monic polynomial of degree $N-2$, and so on, until
  we get $h$ monic in $(u_1,\dots,u_n)$ and equal to $p$ mod. $L$.
\end{proof}

By this Lemma, we get a monic polynomial $h$ in $(u_1,\dots,u_n)$ in $A_S[X]$
and such that $I=(h)$ in $A_S[X]/(b_1,\dots,b_m)$.

\begin{lemma}
 $I = (h)$ in $A_{S}[X]$.
\end{lemma}

\begin{proof}
  Let $L$ be $(b_1,\dots,b_m)$ in $A_S[X]$.
  Since $I$ contains $I\cap L$ and $I\cdot J = (f)$ with $f$ regular, we can find $K$
  such that $I\cdot K = I\cap L$.
  We then have $I\cdot K = 0$ mod. $L$ and hence $K = 0$ mod. $L$ since $I$ contains $f$
  which is monic.
  This means $I\cap L = I\cdot L$. Then we have $I = hA_S[X] + I\cdot L$.
  The result then follows from the fact that $h$ is monic and from Nakayama, as in Lam \cite{Lam}:
  the module $M = I/(h)$ is a finitely generated module over $A_S$ and satisfies
  $M\subseteq ML$.
\end{proof}

\begin{corollary}
  We can find comaximal monoids $S_1,\dots,S_l$ such that $I$ is principal and generated by a
  monic polynomial in each $A_{S_j}[X]$.
  In particular, if $A$ is a local ring, we have that $I$ is principal in $A[X]$ and hence $P$
  presents a free module in $A[X]$.
\end{corollary}


\section{Line bundles on affine schemes}

\subsection{Regular sections and regular closed subschemes}

In classical algebraic geometry,
there is the concept of a \notion{generic section} of a line bundle.
Informally, the generic sections have the smallest possible vanishing set.
The following definition corresponds to this notion:

\begin{definition}%
  \label{regular-section}
  Let $X$ be a type and $\mathcal L:X\to \Mod{R}$ a line bundle.
  A section
  \[ s:\prod_{x:X}\mathcal L_x \]
  is \notion{regular}, there merely is a trivializing affine cover $U_1=\Spec A_1,\dots,U_n=\Spec A_n$
  of $\mathcal L$, such that each trivialized restriction
  \[ s_i:\Spec A_i\to R \]
  is a regular element (\cref{regular-element}) of $(\Spec A_i\to R) = A_i$.
\end{definition}

\begin{lemma}%
  \label{regular-zariski-local}
  Let $s:\Spec A\to R$.
  $s$ being regular is Zariski-local, i.e.
  for all Zariski-covers $U_1,\dots,U_n$ of $\Spec A$,
  $s$ is regular, if and only if it is regular on all $U_i$.
\end{lemma}

\begin{proof}
  It is enough to check this for a localization at $f:A$.
  Let
  \[ \frac{s}{1}\cdot\frac{g}{f^k}=0\rlap{.} \]
  then $f^lsg=0$, which implies $f^lg=0$ by regularity of $s$ and therefore $\frac{g}{f^l}=0$.
\end{proof}

\begin{proposition}%
  The choice of trivializing cover in \cref{regular-section}
  is irrelevant.
\end{proposition}

\begin{proof}
  By \cref{regular-zariski-local}.
\end{proof}

From a line bundle together with a regular section,
we can produce a closed subtype of a special kind:

\begin{definition}%
  Let $X$ be a scheme.
  A \notion{regular closed subtype} of $X$ is a closed subtype
  $C:X\to \Prop$, such that there merely is an affine open cover $U_1=\Spec A_1,\dots,U_n=\Spec A_n$,
  and $C\cap U_i$ is $V(f_i)$ for a regular $f_i:A_i$.
\end{definition}

\begin{lemma}%
  Let $f,g:A$, $f$ be regular and $V(f)=V(g)$,
  then $g$ is regular and there is a unique unit $\alpha:A^\times$, such that $\alpha f=g$.
\end{lemma}

\begin{proof}
  $V(f)=V(g)$ implies there are $\alpha,\beta:A$ such that
  $\alpha f = g$ and $\beta g = f$.
  But then: $f=\beta g=\beta\alpha f$.
  So by regularity of $f$, $\beta\alpha=1$.
  By \cref{units-products-regular}, units are regular and products of regular elements are regular,
  so $g$ is regular.
  Uniqueness of $\alpha$ follows from regularity.
\end{proof}

\begin{theorem}[using \axiomref{Z-choice}]%
  Let $X$ be a scheme.
  For any regular closed subscheme $C$,
  there is a line bundle with regular section $(\mathcal L,s)$ on $X$,
  such that $C=V(s)$.
\end{theorem}

\begin{proof}
  Let $U_1=\Spec A_1,\dots,U_n=\Spec A_n$ be a cover by standard  affine opens such that we have
  regular $f_i$ with $C\cap U_i=V(f_i)$. 
  We define $\mathcal L$ to be the trivial line bundle $\_\mapsto R$ on each $U_i$
  and by giving automorphisms on the intersections $U_i\cap U_j\colonequiv U_{ij}=\Spec A_{ij}$.
  On $U_{ij}$, $C$ is given by $V(\frac{f_i}{1})$ and $V(\frac{f_j}{1})$ which are both regular.
  Therefore, there is a unit $\alpha:A_{ij}^\times$ such that $\alpha\frac{f_i}{1}=\frac{f_j}{1}$,
  which we can also view as a map $U_{ij}\to R^\times$ and since $R^\times$
  is equivalent to the automorphism group of $R$ as an $R$-module,
  this provides the identetification we need to construct $\mathcal L$.
  Under the identification, the local regular sections are identified, so we get a global section $s$ of $\mathcal L$,
  which is locally regular.
\end{proof}


\section{Application of the Veronese embedding}
For given $d>0$, we introduce the Veronese map $V: \bP^n\rightarrow \bP^N$ with $N = \binom{n+d}{n}-1$.
We write an element of $\bP^N$ as a sequence of elements $z_{i_0,\dots,i_n}$ in $R$, not all $0$,
indexed by $i_0,\dots,i_n$ such that $d = i_0+\dots+i_n$
It is defined by $V(x0:\dots:x_n)$ to be the elements $z_{i_0,\dots,i_n} = x_0^{i_0}\dots x_n^{i_n}$. It is well-defined
since we have $x_i^d\neq 0$ for some $i$.

\begin{proposition}\label{veronese}
  $V$ is a bijection between $\bP^n$ and the closed subset $V(n,d)$ of $\bP^N$ determined by the quadratic
  equations $z_Iz_J = z_Kz_L$ for $I+J=K+L$.
\end{proposition}

\begin{proof}
  We have $\bP^N$ covered by the open $U_0,\dots,U_n$ with $U_l$ set of $z_I$ with $z_I\neq 0$
  for some $I = i_0,\dots,i_n$ with $i_l>0$. On $U_l$ we define the map $g_l(z_I) = (x_0:\dots:x_n)$
  with $x_l = z_I$ and $x_k = z_{J_k}$ with $J_k = j_0,\dots,j_n$ and $j_l = i_l-1$ and $j_k = i_k+1$
  and $j_p = i_p$ for $p\neq l,k$. Using the quadratic relations, we see that the map $g_l$
  are compatible and define an inverse of the Veronese map.
\end{proof}

\begin{corollary}\label{affine}
  Let $q(x_0,\dots,x_n)$ be a homogeneous polynomial of degree $d>0$ (the polynomial $q$ might be $0$). Then $S_q$ the subset
  of $\bP^n$ of $(x_0:\dots:x_n)$ such that $q(x_0,\dots,x_n)\neq 0$
  defines an {\em affine} subset of $\bP^n$.
\end{corollary}

\begin{proof}
  To simplify the notation, we present the argument in the case $n=1$ and $d=2$, but the same can be done
  in general. We have $q = ax_0^2 + bx_0x_1+cx_1^2$. The Veronese embedding is $(x_0:x_1)\mapsto (x_0^2:x_0x_1:x_2^2)$
  and the subset $q\neq 0$ of $\bP^1$ is in bijection with the subset $ay_0+by_1+cy_2\neq 0, y_0y_2 = y_1^2$ of $\bP^2$.
  This subset is itself in bijection with the {\em affine} subset $ay_0+by_1+cy_2 = 1, ~y_0y_2=y_1^2$ of $R^3$.
\end{proof}

Note that we have the following description of $R^{S_q}$.

 \begin{lemma}
   $R^{S_q}$ is  $R[X_0,\dots,X_n][1/q]_0$.
\end{lemma}

 \begin{proof}
   Let $T_q$ be the set of $x$ in $R^{n+1}$ such that $q(x) \neq 0$. We can see $S_q$ as the quotient of $T_q$
   by the relation $x = sy$ for some unit $s$. So $R^{S_q}$ is the subset of elements $u$ in $R^{T_q} = R[X_0,\dots,X_n][1/q] = A$
   such that $u(sx) = u(x)$ for $s$ unit. Let $d$ be the formal degree of $q$, and let us write $u$ as $\Sigma a_i/q^l$ with
   $a_i$ homogeneous component of degree $i$. We have $s\neq 0\rightarrow \Sigma_{i\neq dl} a_is^i/s^{dl}q^l = 0$ and hence
   $a_i = 0$ in $A$ if $i\neq dl$.
 \end{proof}
 
 \begin{corollary}
   $R[X_0,\dots,X_n][1/q]_0$ is a finitely presented $R$-algebra.
 \end{corollary}
 


\section[Picard group of projective space]{Picard group of $\bP^1$}
%% The goal of this note is to show that $\Pic(\bP^n) = \Z$ in the setting of Synthetic Algebraic
%% Geometry \cite{draft}. We actually present a strengthening of this result,
%% which in particular states the equivalence
%% $$(\bP^1\to{\KR})\simeq (\Z\times \KR)$$
%% %We will also present Matthias' strengthened version of this statement, which states that the
%% %map $\Z\times \KR\rightarrow {\KR}^{\bP^1},~p,l\mapsto (x\mapsto l\times \OO(p)(x))$
%% %is an equivalence.

%% One application is that $\Aut(\bP^n)$ is $\PGL_{n+1}$.

%% For the case $n=1$,
%% we follow the proof of Horrock's Lemma as presented in Lam's book on Serre's problem \cite{Lam}
%% on projective modules\footnote{This argument is different from the one presented in Lombardi-Quitt\'e \cite{lombardi-quitte}; we instead give a constructive version of the proof of Nashier-Nichols \cite{Nashier}.}.
%% For the general case, we don't follow the Quillen patching technique
%% presented in the 1976 paper \cite{Quillen}, but instead present an argument which uses
%% our description of ${\bP^1}\to{\KR}$.
%% We then explain how we can deduce that $\Aut(\bP^n)$ is $\PGL_{n+1}$.

%%  One point of this work is to show that all these results can be proven axiomatically in the
%%  setting of univalent type theory with the 3 axioms described in \cite{draft}.
%%  ALREADY IN THE INTRODUCTION

 The following result also holds for a general \emph{connected}\footnote{If $e(1-e) = 0$ then $e=0$ or $e=1$.} ring,
 without assuming a finite presentation. 

 \begin{lemma}\label{stand}
   Let $A$ be a connected, finitely presented $R$-algebra, then
   an invertible element of $A[X,1/X]$ can be written $X^N\Sigma a_nX^n$ with $N$ in $\Z$
   and $a_0$ unit and $a_n$ nilpotent if $n\neq 0$.
 \end{lemma}

 \begin{proof}
   Let $P:A[X,1/X]$ be invertible, with inverse $Q:A[X,1/X]$, and $P=\sum_ia_iX^i$ and $Q=\sum_ib_iX^i$.
   By duality we can view the coefficients as functions $a_i,b_i:\Spec(A)\to R$.
   For all $x:\Spec(A)$, we get an invertible $P_x:R[X,1/X]$ by evaluating the coefficients of $P$ at $x$.
   Then $P_x\cdot Q_x=1$ and in particular $1=\sum_{i+j=0}a_ib_j$, so by locality of $R$, we have $i$ such that $a_i$ is invertible.
   Without loss of generality, we assume $i=0$ and want to show $\neg\neg (a_j=0)$ for $j\neq 0$.
   Since we prove a negated proposition, we can assume that we have $l,k$ minimal with $a_l$ and $b_k$ invertible.
   Then we must have $k+l=0$ because we would have $0=a_lb_k$ otherwise.
   $k$ was minimal, so it is $0$ and $l$ is $0$ as well.
   The same reasoning applies for a maximal choice of $k$.
 \end{proof}
 
 Using this Lemma we deduce the following.

\begin{lemma}\label{nilpotent}
  Any invertible element of $A[X,1/X]$ can be written uniquely as a product
  $uX^l(1+a)(1+b)$ with $l$ in $\Z$, $u$ in $A^{\times}$ and $a$ (resp. $b$)
  polynomial in $A[X]$ (resp. $1/XA[1/X]$) with only nilpotent coefficients.
\end{lemma}

\begin{proof}
  Write $\Sigma v_nX^n$ the invertible element of $A[X,1/X]$.
  W.l.o.g. we can assume that the polynomial is of the form $1 + \Sigma v_nX^n$ with
  all $v_n,~n\in \Z$ nilpotent.
  We let $J$ be the ideal generated by these nilpotent elements.
  We have some $N$ such that $J^N = 0$.
  
  We first multiply by the inverse of $1 + \Sigma_{n<0}v_nX^n$, making all coefficients of
  $X^n,~n<0$ in $J^2$.
  We keep doing this until all these elements are $0$.
  %We then do the same, killing all coefficients of $X^n$ for $n>0$.
  We have then written the invertible polynomials on the form $(1+a)(1+b)$.

  Such a decomposition is unique: if we have $(1+a)(1+b)$ in $A^{\times}$ with $a = \Sigma_{n\geqslant 0}a_nX^n$
  and $b = \Sigma_{n<0}b_nX^n$ then we have $a_n = 0$ for $n>0$ and $b_n = 0$ for $n<0$.
\end{proof}

\begin{corollary}\label{Pic1}
  We have $\prod_{L:\bP^1\rightarrow \KR}\Sigma_{p:\Z}\|L = \OO(p)\|$
\end{corollary}

\begin{proof}
A line bundle $L([x_0,x_1])$ on $\bP^1$ is trivial on each of the affine charts $x_0\neq 0$ and $x_1\neq 0$ by Corollary \ref{c1}, so
it is characterised by an invertible Laurent polynomial on $R$, and the result follows from Lemma \ref{nilpotent}.
\end{proof}

We can then state the following strengthening.

\begin{proposition}\label{Matthias}
  The map $\KR\times\Z\rightarrow (\bP^1\rightarrow \KR)$
  which associates to $(l_0,d)$ the map $x\mapsto l_0\otimes \OO(d)(x)$ is an equivalence.
\end{proposition}

\begin{proof}
  Corollary \ref{Pic1} shows that this map is essentially surjective.
  So we can conclude by showing that the map is also an embedding.
  For $(l,d),(l',d'):\KR\times\Z$ let us first consider the case $d=d'$. 
  Then we merely have $(l,d)=(\ast,d)$ and $(l',d')=(\ast,d)$,
  so it is enough to note that the induced map on loop spaces based at $(\ast,d)$ is an equivalence by \Cref{const}.
  Now let $d\neq d'$. To conclude we have to show $\OO(k)$ is different from $\OO(0)$ for $k\neq 0$.
  It is enough to show that for $k>0$ the bundle $\OO(k)$ has at least two linear independent sections,
  since we know $\OO(0)$ only has constant sections by \Cref{const}.
  This follows from the fact that $\OO(k)(x)$ is $\Hom_{\Mod{R}}(Rx^{\otimes k},R)$ and has all projections as sections.
  %% which we can naturally identify with $\Hom_{\Mod{R}}(R(x_0^k,x_1^k),R)$ for $x=(x_0,x_1)$.
  %% Then $s_0(r\cdot (x_0^k,x_1^k))\colonequiv r\cdot x_o^k$ and $s_1(r\cdot (x_0^k,x_1^k))\colonequiv r\cdot x_1^k$ define two sections
  %% which are independent since $s_0([0:1])=0\neq s_1([0:1])$ and $s_1([1:0])=0\neq s_1([1:0])$.
  %TC: I think it is clearer not so say too much here. 
\end{proof}

 It is a curious remark that $\KR\rightarrow \KR$ is also equivalent
 to $\KR\times \Hom_{\mathrm{Group}}(R^\times,R^\times) = \KR\times\Z$.

\begin{corollary}\label{Matthias1}
  We have $\prod_{L:\bP^1\rightarrow \KR}\prod_{x:R}L([1:x]) = L([0:1])$.
\end{corollary}

\begin{proof}
  By the equivalence in \Cref{Matthias}, we have
  \[ \prod_{L:\bP^1\to \KR} \,\prod_{x : \bP^1}  L(x)=l_0\otimes \OO(d)(x) \]
  for some $(l_0,d)$ corresponding to $L$.
  $\OO(d)([0:1])$ can be identified with $R^1$ and $\OO(d)$ is trivial on $R$,
  so we have $L([1:x])=l_0=L([0:1])$ for all $x:R$.
\end{proof}

\section{Line bundles on $\bP^n$}
We will prove $\Pic(\bP^n)=\Z$ and a strengthening thereof in this section by mostly algebraic means.
In \Cref{geometric-proof} we will give a shorter geometric proof.

We can now reformulate Quillen's argument for Theorem 2' \cite{Quillen} in our setting.

\begin{proposition}\label{trivial}
  For all $V:\bP^n\rightarrow \KR$ we have ${\prod_{s:R^n}V([1:s]) = V([0:1:0:\cdots :0])}$.
\end{proposition}

\begin{proof}
  We define $L:R^{n-1}\rightarrow (\bP^1 \to \KR)$ by $L~t~[x_0:x_1] = V([x_0:x_1:x_0t])$.
  Let $s=(s_1,\dots,s_{n}):R^{n}$. We apply Corollary \ref{Matthias1} and we get
  \[
   V([1:s]) = L~(s_2,\dots,s_n)~[1:s_1] = L~(s_2,\dots,s_n)~[0:1] = V([0:1:0:\cdots :0])
   \rlap{.}
  \]
\end{proof}

 Note that the use of Corollary \ref{Matthias1} replaces the use of the ``Quillen patching''
 \cite{lombardi-quitte} introduced in \cite{Quillen}.

\medskip

Let $T$ be the ring of polynomials $u = \Sigma_p u(p)X^p$ with
$X^p = X_0^{p_0}\dots X_n^{p_n}$ with $\Sigma p_i = 0$. We write $T_l$ for the subring
of $T$ which contains only monomials $X^p$ with $p_i\geqslant 0$ if $i\neq l$
and $T_{lm}$ the subring of $T$ 
which contains only monomials $X^p$ with $p_i\geqslant 0$ if $i\neq l$ and $i\neq m$.

Note that $T_l$ is the polynomial ring $T_l = R[X_0/X_l,\dots,X_n/X_l]$.

A line bundle on $\bP^n$ is given by compatible line bundles on each $\Spec(T_l)$.

By \Cref{trivial}, a line bundle on $\bP^n$ is trivial on each $\Spec(T_l)$.
So it is determined by $t_{ij}$ invertible in $T_i[X_i/X_j] = T_j[X_j/X_i] = T_{ij}$
such that $t_{ik} = t_{ij}t_{jk}$ and $t_{ii} = 1$. 
Using \Cref{stand} we can assume without loss of generality, that
$t_{ij} = (X_i/X_j)^{N_{ij}} u_{ij}$, for some $N_{ij}$ in $\Z$, where $u_{ij}(p)$ is invertible for $p = 0$
and all other coefficients $u_{ij}(p)$ for $p\neq 0$
are nilpotent. By looking at the relation  $t_{ik} = t_{ij}t_{jk}$ when we quotient by nilpotent elements, we see that
$N_{ij} = N$ does not depend on $i,j$.
The result $\Pic(\bP^n) = \Z$ will then follow from the following result.

\begin{proposition}
  There exists $s_i$ invertible in $T_i$ such that $u_{ij} = s_i/s_j$ 
\end{proposition}

\begin{proof}
%  (We follow essentially David's argument.)
  Each $u_{ij}$ is such that $u_{ij}(p)$ unit for $p=0$ and
  all $u_{ij}(p)$ nilpotent for $p\neq 0$.

  Like in the proof of \Cref{nilpotent}, we can change $u_{01}$ so that
  we have $u_{01}(p) = 0$ if $p\neq 0$ and $p_0\geqslant 0$ or $p_1\geqslant 0$ by multiplying $u_{01}$ by a unit in $T_0$ and
  a unit in $T_1$. Let us show for instance how to force $u_{01}(p) = 0$ if $p\neq 0$ and $p_1\geqslant 0$ by multiplying $u_{01}$
  by a unit in $T_0$. Let $M$ be the ideal generated by $u_{01}(p)$ for $p\neq 0$, which is a nilpotent ideal. If we
  multiply $u_{01}$ by $u_{01}(0) - \Sigma_{p_1\geqslant 0} u_{01}(p)$
  we change $u_{01}$ to $u'_{01}$ where all $u_{01}'(p)$, for $p_1\geqslant 0$ and $p\neq 0$, are in $M^2$. We iterate this process
  and since $M$ is nilpotent, we force $u_{01}(p) = 0$ or $p\neq 0$ and $p_1\geqslant 0$.

  We can thus assume that $u_{01}(p) = 0$ if $p\neq 0$ and $p_0\geqslant 0$ or $p_1\geqslant 0$.
  
  We claim then that, in this case, $u_{01}$ has to be a unit. For this we show that $u_{01}(p) = 0$
  if $p_l>0$ for each $l\neq 0,1$. 
  This is obtained by looking at the relation $u_{01}= u_{0l}u_{l1}$. Let $L$ be the ideal generated by
  coefficients $u_{0l}(p)$ and $u_{1l}(p)$ with $p_l>0$ and $I$
  the ideal generated by all nilpotent coefficients of $u_{0l}$ and $u_{l1}$.
  Thanks to the form of $u_{01}$ we must have $L\subseteq LI$ and so $L=0$ by Nakayama. Indeed we have
  $$u_{01}(p) = u_{0l}(p)u_{l1}(0) + u_{0l}(0)u_{l1}(p) + \Sigma_{q+r = p, q\neq 0, r\neq 0}u_{0l}(q)u_{l1}(r)$$
  and we use this to show that $u_{0l}(p)$ is in $LI$.
  Since $p_l>0$, we have $u_{0l}(p) = 0$ if $p_0\geqslant 0$, hence we can assume $p_0<0$.
  We also have $u_{0l}(p)$ if $p_1<0$ and we can assume $p_1\geqslant 0$.
  This implies $u_{l1}(p) = 0$ (since $p_0<0$) and $u_{01}(p) = 0$ (since $p_0<0$ and $0\leqslant p_1$).
  We get thus
  $$u_{0l}(p)u_{l1}(0) = - \Sigma_{q+r = p, q\neq 0, r\neq 0}u_{0l}(q)u_{l1}(r)$$
  and each member in the sum $u_{0l}(q)u_{l1}(r)$ is in $IL$ since $q_l+r_l = p_l>0$ and hence $q_l>0$ or $r_l>0$.

  We thus deduce $L=0$ by Nakayama. We get, for $p_l>0$
  $$u_{01}(p) = u_{0l}(p)u_{l1}(0) + u_{0l}(0)u_{l1}(p)$$
  and if $p_0<0$ and $p_1<0$ we have $u_{0l}(p) = u_{l1}(p) = 0$.

  This implies that all coefficients $u_{01}(p)$ such that $p_l>0$ are $0$.

  Since this holds for each $l>1$ we have that $u_{01}$ is a unit in $R$.

  W.l.o.g. we can assume $u_{01}= 1$. We then have $u_{0l} = u_{1l}$ in $T_{0l}\cap T_{1l} = T_l$
  and we take $s_l = u_{0l} = u_{1l}$.
\end{proof}

\begin{corollary}
  $\Pic(\bP^n) = \Z$.
\end{corollary}

We can then strengthen this result, with the same reasoning as in Proposition \ref{Matthias}.

\begin{theorem}\label{Matthias2}
  The map $\KR\times\Z\rightarrow (\bP^n\rightarrow \KR)$
  which associates to $l_0,d$ the map $x\mapsto l_0\otimes \OO(d)(x)$ is an equivalence.
\end{theorem}

We deduce from this a characterisation of the maps $\bP^n\rightarrow\bP^m$.% using Lemma \ref{hom}.

\begin{corollary}\label{map}
  A map $\bP^n\rightarrow\bP^m$ is given by $m+1$ homogeneous polynomials $p = (p_0,\dots,p_m)$ on $R^{n+1}$
  of the same   degree $d$ such that $x\neq 0$ implies $p(x)\neq 0$.
\end{corollary}

\begin{proof}
Write $T_n(l)$ for $l^{n+1}\setminus\{0\}$. We have $\bP^n = \Sigma_{l:\KR}T_n(l)$ and so
$$
\bP^n\rightarrow\bP^m = \sum_{s:\bP^n\rightarrow \KR}\prod_{x:\bP^n}T_m(s~x)
$$
Using Theorem \ref{Matthias2}, this is equal to
$$
\sum_{l_0:\KR}\sum_{d:\Z}\prod_{l:\KR}T_n(l)\rightarrow T_m(l_0\otimes l^{\otimes d})
$$
and, as for Lemma \ref{hom}, this is the set of tuples of $m+1$ polynomials in $R[X_0,\dots,X_n]$ homogenenous
of degree $d$, sending $x\neq 0$ to $p(x)\neq 0$, and quotiented by proportionality.
\end{proof}

We deduce the characterisation of $\Aut(\bP^n)$. This is a
remarkable result, since the automorphisms are in this framework only bijections of sets.

\begin{corollary}
  $\Aut(\bP^n)$ is $\PGL_{n+1}$.
\end{corollary}

We also have the following application of computation of cohomology groups \cite{draft}.

\begin{corollary}
A function $\bP^n\rightarrow\bP^m$ is constant if $n>m$.
\end{corollary}

\begin{proof}
We proved in \cite{cech-draft} that cohomology groups can be computed as Cech cohomology for any
finite open acyclic covering and used this to prove $H^n(\bP^n,\OO(-n-1))=R$.
By \Cref{map}, a map $\bP^n\rightarrow\bP^m$ is given by $m+1$ non zero polynomials
$p(x) = (p_0(x),\dots,p_m(x))$ homogeneous of the same degree $d\geqslant 0$ and such that $x\neq 0$ implies $p(x)\neq 0$.
This means that $\bP^n$ is covered by $m+1$ open subsets $U_i(x)$ defined by $p_i(x)\neq 0$.
I claim that we should have $d=0$.

 If $q(x)$ is a non zero homogeneous polynomial of degree $d>0$, the open $q(x)\neq 0$ defines an {\em affine}
 and hence acyclic \cite{cech-draft}, open subset of $\bP^n$ (see e.g. Exercise 3.5 in \cite{Hartshorne}).
 It follows that the covering $U_0,\dots,U_m$ is acyclic if $d>0$. But this contradicts $H^n(\bP^n,\OO(-n-1))=R$.

 Hence $d=0$ and the map is constant.
\end{proof}


\section[Automorphism group of projective space]{Another proof of $\Aut(\bP^n) = \PGL_{n+1}$}
In this section, we present an alternative way to prove that   $\Aut(\bP^n)$ is $\PGL_{n+1}$.

It is direct \cite{seminormal} to show that a projective module of rank $1$ over a polynomial
ring $K[X_1,\dots,X_n]$ is free, where $K$ is a discrete field. (The result actually holds in general
for rank $n$, this is the famous Serre's Problem \cite{Lam}.) Interpreting this proof dynamically
\cite{lombardi-quitte}, it follows that if $A$ is now an arbitrary commutative ring, and $M$ a
projective module of rank $1$ over $A[X_1,\dots,X_n]$, it is possible to build a binary tree, where
the root is $A$, and each node is an ring extension $i:A\rightarrow B$, and a branch is obtained
by forcing an element $a$ in $A$ to be invertible $B\rightarrow B[1/i(a)]$ or zero $B\rightarrow B/(i(a))$,
and each leaf is such that $M\otimes B[X_1,\dots,X_n]$ is free.

For the ring $R$, we have $a\neq 0\rightarrow \exists_x ax=1$ and hence $\neg \neg (a=0\vee \exists_x a x=1)$.
Hence we have the following result.

\begin{lemma}\label{notnot}
  A projective module of rank $1$ over $R[X_1,\dots,X_n]$ is not not free. Equivalently, 
a line bundle on $Sp(R[X_1,\dots,X_n]) = R^n$ is not not trivial.
\end{lemma}

This generalizes Corollary 3.3.6 of \cite{draft}.


%% \begin{proof}
%%   Let $L$ be a line bundle $R^n\rightarrow\KR$. As we saw above, the $R[X_1,\dots,X_n]$-module $M = \prod_{u:\Spec(A)}L(u)$,
%%   which is projective of rank $1$,   is presented by an idempotent square matrix $P$, and this module is free
%%   exactly if we can find a column vector $x$ and a line vector $y$ such that $xy = I-P$. Using
%%   \cite{seminormal}, this is satisfied if $A$ is a gcd domain. Since $R$ is not not a discrete
%%   field, in the sense that we have $\forall_{x:R}\neg\neg (x=0 \vee \exists_y (xy = 1))$, we conclude
%%   that we have $\neg\neg \exists_{x~y}xy = I-P$, and hence the $A$-module  $M$
%%   is not not free, in the sense $\neg\neg \exists_{m_0:M}\forall_{m:M}\exists!_{a:A} m = am_0$
%%   (or, equivalently $\neg\neg M = A$, where the equality is equality of $A$-modules).
%% \end{proof}

Similarly, using the fact that maps $\bP^n\rightarrow \bP^m$ are given by $m+1$ homogeneous polynomials in $K[X_0,\dots,X_n]$
over a discrete field $K$, we deduce.

\begin{lemma}\label{weakmap}
  Given $\varphi:\bP^n\rightarrow  \bP^m$, we have not not there exists
  $m+1$ homogeneous polynomials $p = (p_0,\dots,p_m)$ on $R^{n+1}$
  of the same   degree $d$ such that $x\neq 0$ implies $p(x)\neq 0$.
\end{lemma}

%% \begin{proof}
%%   As in the previous Section, let $T$ be the ring of polynomials $u = \Sigma_p u(p)X^p$ with
%% $X^p = X_0^{p_0}\dots X_n^{p_n}$ with $\Sigma p_i = 0$. We write $T_l$ for the subring
%% of $T$ which contains only monomials $X^p$ with $p_i\geqslant 0$ if $i\neq l$
%% and $T_{lm}$ the subring of $T$ 
%% which contains only monomials $X^p$ with $p_i\geqslant 0$ if $i\neq l$ and $i\neq m$.

%%   We view $\varphi(u)$ as a line in $R^{m+1}$. 
%%   If $U_i = Sp(T_i)$ is the affine open subset of
%%   $(x_0:\dots:x_n)$ in $\bP^n$ such that $x_i\neq 0$, we have by the previous Lemma that
%%   $\prod_{u:U_i}\varphi(u)$ is not not a free $T_i$-module and so we not not have
%%   an element $\psi_i:\prod_{u:U_i}R^{m+1}-0$ such that $\psi_i(u)$ generates $\varphi(u)$ for $u:U_i$.
%%   We get then $m+1$ elements $(p_{i0},\dots,p_{im})$ of $T_i$ such that
%%   $\psi_i(u) = (p_{i0}(u),\dots,p_{im}(u))$.

%%   For $u$ in $U_i\cap U_j$ the line $\varphi(u)$ is generated both by $\psi_i(u)$ and $\psi_j(u)$
%%   and we not not have $t_{ij}$ invertible in $T_i[X_i/X_j] = T_j[X_j/X_i] = T_{ij}$ such that
%%   $t_{ij}(u)\psi_i(u) = \psi_j(u)$. By Proposition \ref{units} (which has a much simpler proof in the case
%%   of a discrete field), we have not not $d$ and $s_i$ invertible in $T_i$ such that
%%   $t_{ij} = X_j^d s_j/X_i^d s_i$.
%%   We then not not
%%   get $m+1$ homogeneous polynomials $p_{il}X_i^ds_i = p_{jl}X_j^d s_j = p_l$ for $l = 0,\dots,m$
%%   such that $\varphi(u) = (p_0(u):\dots:p_m(u))$.
%% \end{proof}

 We can then give a proof that   $\Aut(\bP^n)$ is $\PGL_{n+1}$ without relying on Horrocks' Theorem.

We see the element of $\bP^n$ as lines in $R^{n+1}$.
If $p_0,\dots,p_n$ are $n+1$ points of $\bP^n$, we can define the open proposition that
the points $p_0,\dots,p_n$ are independent
by the fact that for any choice of non zero elements $v_0,\dots,v_n$ in respectively $p_0,\dots,p_n$
the determinant of the $(n+1)\times (n+1)$ matrix $v_0\dots v_n$ is $\neq 0$.

\begin{corollary}
    $\Aut(\bP^n)$ is $\PGL_{n+1}$.
\end{corollary}

\begin{proof}
  Let $\varphi$ be an element of $\Aut(\bP^n)$.
  Let $u_0,\dots,u_n$ be the vectors $(1,0,\dots,0), \dots, (0,\dots,0,1)$
  and $u$ be the vector $(1,1,\dots,1)$.

  We can find a $(n+1)\times (n+1)$ matrix $A$ such that $\varphi(R u_i) = R (Au_i)$
  and $\varphi(R u) = R (A u)$. By Lemma \ref{weakmap}, we have not not $\varphi$ is represented by $A$,
  that is $\neg\neg \forall_{v\neq 0}\varphi(R v) = R (Av)$. If follows that $\neg \neg det(A) \neq 0$
  and hence $\det(A)\neq 0$ and $A$ is in $GL_{n+1}$.

  In this way, we reduce the problem of showing that if $\neg\neg \forall_{v\neq 0}\varphi (R v) = R v$
  then $\varphi$ is the identity. But then we have not not $\varphi(U_i)\subseteq U_i$
  and so $\varphi(U_i)\subseteq U_i$ for the standard $n+1$ affine charts $U_i$ of $\bP^n$, and we can
  conclude as in \cite{Demazure}, III, 4.6.
\end{proof}


\section[Picard group of projective space (geometric)]{A geometric proof of $\Pic(\bP^n)=\Z$}
\label{geometric-proof}
A geometric property of $\bP^n$:

\begin{lemma}\label{constant-functions-Pn-minus-points}
  Let $n>1$ and $p\neq q$ be points of $\bP^n$, then all functions
  \begin{center}
  \begin{enumerate}[(i)]
  \item $\bP^n\setminus\{p\}\to \Z$
  \item $\bP^n\setminus\{p,q\}\to \Z$
  \item $\bP^n\setminus\{p\}\to R$
  \item $\bP^n\setminus\{p,q\}\to R$
  \end{enumerate}
  \end{center}
  are constant.
\end{lemma}

\begin{proof}
  We start with (iv).
  Let $f:\bP^n\setminus\{p,q\}\to R$.
  For the charts $U_0=\{[x_0:\dots:x_n]\mid x_0\neq 0\}$ and $U_1=\{[x_0:\dots:x_n]\mid x_0\neq 0\}$, we can assume $p\in U_0, p\notin U_1$ and $q\in U_1, q\notin U_0$.
  Then $f_{|U_0\setminus\{p\}}$ can be extended to $U_0$ by \Cref{ext} and an analogous extension exisits on $U_1$.
  These extensions glue with $f$ to a function $\widetilde{f}:\bP^n\to R$ which agrees with $f$ on $\bP^n\setminus\{p,q\}$.
  By \Cref{const}, $\widetilde{f}$ is constant and therefore $f$ is constant.
  This carries over to functions $\bP^n\setminus\{p,q\}\to \mathrm{Bool}$ since $\mathrm{Bool}\subseteq R$ and thus also to any $\bP^n\setminus\{p,q\}\to \Z$,
  which shows (ii).
  (i) and (iii) follow from (ii) and (iv).
\end{proof}
We proceed by extendending \Cref{Matthias} to subspaces of $\bP^n$ which can be constructed like $\bP^1$:

\begin{lemma}\label{line-bundle-on-line}
  Let $M\subseteq R^{n+1}$ be a submodule with $\|M=R^2\|$.
  Then $\Gr(1,M)\subseteq \bP^n$ and the map
  \begin{align*}
    \Z\times \KR &\to (\Gr(1,M)\to \KR) &\\
    (d,l_0) &\mapsto (L\mapsto L^{\otimes d}) &\text{ for $d\geq 0$} \\
    (d,l_0) &\mapsto (L\mapsto \Hom_{\Mod{R}}(L^{\otimes d},R)) &\text{ for $d< 0$} 
  \end{align*}
  is an equivalence.
\end{lemma}

\begin{proof}
  We prove a proposition, so we have an $R$-linear isomorphism $\phi:R^2\to M$ and for each $d:\Z$, we get a commutative triangle:
  \begin{center}
  \begin{tikzcd}
    \Gr(1,M)\ar[rr,"\OO(d)"] && \KR \\
    & \Gr(1,R^2)\ar[lu,"\phi"]\ar[ur,swap,"\OO(d)"] &
  \end{tikzcd}
\end{center}
by restricting $\phi$ to each line in $\Gr(1,R^2)$.
This shows that the map from \Cref{Matthias} and from the statement are equal as maps to $(V:\Mod{R})\times \|V=R^2\| \times (V\to\KR))$,
which proves the claim.
\end{proof}

\begin{theorem}
  The map
  \begin{align*}
    \Z\times \KR &\to (\bP^n\to \KR) \\
    (d,l_0) &\mapsto (x\mapsto l_0\otimes \OO(d)(x))
  \end{align*}
  is an equivalence.
\end{theorem}

\begin{proof}
  It is enough to show that the map is surjective, by the same reasoning as in the proof of \Cref{Matthias}.
  Let $L:\bP^n\to \KR$. First we determine the degree of $L$.
  Let $p\neq q$ be points in $\bP^n$ and $M\subseteq R^{n+1}$ be the span of $p$ and $q$ as submodules of $R^{n+1}$.
  Then $\|M=R^2\|$ and we can use the inverse $i$ of the map in \Cref{line-bundle-on-line} to define $d\colonequiv \pi_1(i(L_{|\Gr(1,M)}))$.
  The integer $d$ is independent of the choice of $p$ and $q$:
  If we let $p$ vary, we get a function of type $\bP^n\setminus\{q\}\to \Z$ which is constant by \Cref{constant-functions-Pn-minus-points}.
  The same applies for $q$
  and the two subsets $\bP^n\setminus\{p\}$ and $\bP^n\setminus\{q\}$ cover $\bP^n$.

  In the following we consider only $L$ such that $d$ as constructed above is $0$.
  This means that on each line $\Gr(1,M)$, $L$ will be constant.
  So for $p,x:\bP^n$,
  and $x\neq p$ we can construct an equality $P_x:L(x)=L(p)$  by restricting $L$ to $\Gr(1,\langle x,p\rangle)$ and applying \Cref{line-bundle-on-line}.
  So we have $P:(x:\bP^n\setminus\{p\})\to L(x)=L(p)$ and for $q\neq p$ we can construct
  $Q:(y:\bP^n\setminus\{q\})\to L(y)=L(q)$ analogously.
  
  The claim follows if we show that $L$ is constant on all of $\bP^n$.
  Since, overall, we show the proposition that the map from the statement merely has a preimage,
  we can assume $a:L(p)=R^1$ and $b:L(q)=R^1$ and get:
  \[ \left((x:\bP^n\setminus\{p,q\})\mapsto a^{-1}P_x^{-1}Q_xb\right) : \bP^n\setminus\{p,q\} \to R^\times\]
  which is constantly $\lambda$ by \Cref{constant-functions-Pn-minus-points}.
  So $P$ and $Q$ can be corrected using $\lambda,a$ and $b$ to yield a global proof of constancy of $L$.
\end{proof}

\newpage

\section*{Appendix 1: Horrock's Theorem}

We present an alternative constructive proof of the the following special case of the {\em affine}
Horrocks Theorem \cite{Lam}, V.2, for a commutative ring $A$. We
essentially follow Nashier-Nichols' Proof of Horrocks Theorem, as presented in \cite{Lam}, IV.5.

\begin{lemma}\label{Horrocks}
  If an ideal of $A[X]$ divides a principal ideal $(f)$ with $f$ monic then it is itself a principal ideal.
\end{lemma}

Let $L$ and $M$ be such that $L\cdot M = (f)$. We can then write $f = \Sigma u_iv_i$ with $u_i$ in $L$ and
$v_i$ in $M$. Using $f$ monic, we then have $L = (u_1,\dots,u_n)$ and $M = (v_1,\dots,v_n)$.
The strategy of the proof is to build comaximal monoids $S_1,\dots,S_l$ in $A$ \cite{lombardi-quitte},
XIV.1, such that $L$ is generated by a monic polynomial in each $A_{S_j}[X]$.

\subsection{Formal computation of gcd}

%We start by describing a general technique introduced in \cite{lombardi-quitte}.

If we have a list $u_1,\dots,u_n$ of polynomials over a field we can compute the gcd of this list
$(g) = (u_1,\dots,u_n)$ and $g$ is either $0$ (in the case where all the polynomials $u_1,\dots,u_n$ are $0$)
or a monic polynomial.

In general, if we are over an arbitrary ring $A$, we can interpret this computation formally as
follows. We build a binary tree of root $A$. 
To each branch is associated a pair of finite sets $I;U$
of elements in $A$ and the finite
extension $A(I;U) = A_{M(U)}/(I)$ of $A$, where $M(U)$ is the multiplicative monoid generated by the
elements in $U$. Intuitively, we force the elements in $I$
to be $0$ and the elements in $U$ to be invertible. If we are at such a node $A(I;U)$, and $a$ is
an element of $A$, we can open two new branches by forcing $a$ to be $0$, getting the pair $I,a;U$,
or to be invertible, getting the pair $I;U,a$.

Corresponding to the formal computation of the gcd, we get a binary tree where we have at each leaf
a ring $A(I;U)$ and a polynomial $g$ in $A_{M(U)}[X]$, which is monic or $0$, and
such that $(g) = (u_1,\dots,u_n)$ in $A(I;U)[X]$.

We have assumed that the ideal $(u_1,\dots,u_n)$ in $A[X]$
contains a monic polynomial, so we can only have $g = 0$ if $1=0$ in $A(I;U)$ and we can replace
$g$ by $1$. So in the case where the ideal $(u_1,\dots,u_n)$ in $A[X]$ contains a monic
polynomial, we can assume that on each leach we have a {\em monic} polynomial.

Like in \cite{lombardi-quitte}, XIV.1, we can also associate to each branch
the multiplicative monoid $S(U;I)$ generated by the elements of $U$
and $1 + (I)$. In the ring $A_{S(U;I)}$ we force the elements in $U$ to be invertible {\em and} the elements
in $I$ to be in the Jacobson radical \cite{lombardi-quitte}.
If we do this for each branch, we get a list of monoids $S_1,\dots,S_l$
that are {\em comaximal} \cite{lombardi-quitte}: if $s_i$ in $S_i$ then $1 = (s_1,\dots,s_l)$.

\subsection{Application to Horrocks' Theorem}

We assume $f = \Sigma u_iv_i$ and $fp_{ij} = u_iv_j$ with $\Sigma p_{ii} = 1$
in $A[X]$. The goal is to build comaximal monoids $S_1,\dots,S_l$ with $(u_1,\dots,u_n)$ principal
and generated by a monic polynomial in $A_{S_j}[X]$.

Note that $(u_1,\dots,u_n)$ contains the monic polynomial $f$.

We first build a binary tree which corresponds to the formal computation of the gcd of
$u_1,\dots,u_n$ as described above. For each branch $I;U$ we have a monic polynomial
$g$ in $A_{M(U)}[X]$, such that $(u_1,\dots,u_n) = (g)$ in $A(I;U)[X]$.

We can then state two Lemmas.

The first Lemma has the same proof as in Lam's presentation
of Nashier-Nichols' Proof of Horrocks Theorem, Lemma IV.5.1, \cite{Lam}.

\begin{lemma}
  Let $R$ be a ring with an ideal $J$ contained in the Jacobson radical
  and $L$ an ideal of $R[X]$ which contains a monic polynomial. We consider
  the reduction mod. $J$
  $$\pi: R[X]\mapsto (R/J)[X]$$
  Any monic polynomial of $\pi(I)$ can be lifted to a polynomial in $R[X]$.
\end{lemma}

 Using this Lemma, we get a monic polynomial $h$ in $(u_1,\dots,u_n)$ in $A_{S(U;I)}[X]$
 and such that $h$ generates $(u_1,\dots,u_n)$ mod. $(I)$.
 We can now use that $I$ is contained in the Jacobson radical of $A_{S(U;I)}$ and the
 following second Lemma, which corresponds to Proposition IV.5.2 of \cite{Lam},
 to conclude that we actually have $(h) = (u_1,\dots,u_n)$ in $A_{S(U;I)}[X]$.

\begin{lemma}
  Let $R$ a ring, $J$ an ideal of $R$ contained in the Jacobson radical of $R$. If
  we have $L\cdot M = (f)$ with $f$ monic in $R[X]$, and $L$ contains a monic polynomial
  $h$ such that $L = (h)$ in $(R/J)[X]$ then $L = (h)$ in $R[X]$.
\end{lemma}

\begin{proof}
  Since $L$ contains $L\cap J$ and $L\cdot M = (f)$ with $f$ regular, we can find $K$
  such that $L\cdot K = L\cap J$.
  We then have $L\cdot K = 0$ mod. $J$ and hence $K = 0$ mod. $J$ since $L$ contains $f$
  which is monic.
  This means $L\cap J = L\cdot J$. Then we have $L = (h) + L\cdot J$.
  The result then follows from the fact that $h$ is monic and from Nakayama's Lemma, as in Lam \cite{Lam}:
  the module $P = L/(h)$ is a finitely generated module over $R$ and satisfies
  $P\subseteq JP$ and $J$ is contained in the Jacobson radical of $R$, so $P = 0$ by Nakayama's Lemma.
\end{proof}

\begin{corollary}
  We can find comaximal elements $s_1,\dots,s_l$ such that $(u_1,\dots,u_n)$ is principal and generated by a
  monic polynomial in each $A_{s_j}[X]$. Since these monic polynomials are uniquely determined
  we can patch these generators and get that $(u_1,\dots,u_n)$ is principal in $A[X]$\footnote{If $A$ is not
  connected, the generator of $(u_1,\dots,u_n)$ may not be monic: if $e(1-e)=1$ then the ideal $(eX+(1-e))$
  divides the ideal $(X)$.}.
\end{corollary}


\section*{Appendix 2: Quillen Patching}

We reproduce the argument in Quillen's paper \cite{Quillen}, as simplified in \cite{lombardi-quitte}.
This technique of Quillen Patching has been replaced by the equivalence in Proposition \ref{Matthias}.

If $P$ and $Q$ are two idempotent matrix of the same size, let us write $P\simeq Q$ for expressing that $P$ and $Q$ presents
the same projective module (which means that there are similar, which is in this case is the same as being equivalent).

If we have a projective module on $A[X]$, presented by a matrix $P(X)$, this module is extended
precisely when we have $P(X)\simeq P(0)$.

\begin{lemma}
  If $S$ is a multiplicative monoid of $A$ and $P(X)\simeq P(0)$ on $A_S[X]$ then there exists
  $s$ in $S$ such that $P(X+sY)\simeq P(X)$ in $A[X]$.
\end{lemma}

\begin{lemma}
  The set of $s$ in $A$ such that $P(X+sY)\simeq P(X)$ is an ideal of $A$.
\end{lemma}

\begin{corollary}
  If we have $M$ projective module of $A[X]$ and $S_1,\dots,S_n$ comaximal multiplicative monoids of $A$
  such that each $M\otimes_{A[X]} A_{S_i}[X]$ is extended from $A_{S_i}$ then $M$ is extended from $A$.
\end{corollary}

Let us reformulate in synthetic term this result. Let $A$ be a f.p. $R$-algebra and $L:\Spec(A)\rightarrow B\Gm^{\A^1}$.
Then $L$ corresponds to a projective module of rank $1$ on $A[X]$. We can form
$$T(x) = \prod_{r:R}L~x~r = L~x~0$$
and $\|T(x)\|$ expresses that $L~x$ defines a trivial line bundle on $\A^1 = \Spec(R[X])$.
It is extended exactly when we have
$\|{\prod_{x:\Spec(A)}T(x)}\|$. We can then use Zariski local choice to state.

\begin{proposition}\label{c2}
  We have the implication $(\prod_{x:\Spec(A)}\|T(x)\|)\rightarrow \|\prod_{x:\Spec(A)}T(x)\|$.
\end{proposition}

\newpage

\section*{Appendix 3: Classical argument}

We reproduce a message of Brian Conrad in MathOverflow \cite{conrad-mathoverflow-16324}.

\medskip

``We know that the Picard group of projective $(n-1)$-space over a field $k$ is $\Z$
generated by $\OO(1)$.
This underlies the proof that the automorphism group of such a projective space is $\PGL_n(k)$.
But what is the automorphism group of $\bP^{n-1}(A)$ for a general ring $A$? Is it $\PGL_n(A)$?
It's a really important fact that the answer is yes.
But how to prove it? It's a shame that this isn't done in Hartshorne.

By an elementary localization, we may assume $A$ is local.
In this case we claim that $\Pic(\bP^{n-1}(A))$ is infinite cyclic generated by $\OO(1)$.
Since this line bundle has the known $A$-module of global sections,
it would give the desired result if true by the same argument as in the field case.
And since we know the Picard group over the residue field, we can twist
to get to the case when the line bundle is trivial on the special fiber. How to do it?

\medskip

 Step 0: The case when $A$ is a field. Done.

 \medskip

 Step 1: The case when $A$ is Artin local.
 This goes via induction on the length, the case of length $0$ being Step $0$
 and the induction resting on cohomological results for projective space over the residue field.

  \medskip

 Step 2: The case when $A$ is complete local noetherian ring. This goes
 using Step 1 and the theorem on formal functions (formal schemes in disguise).

  \medskip

 Step 3: The case when $A$ is local noetherian.
 This is faithfully flat descent from Step 2 applied over $A~\widehat{}$

 \medskip
 
 Step 4: The case when $A$ is local:
 descent from the noetherian local case in Step 3 via direct limit arguments.

\medskip
 
QED''


\printindex

\printbibliography

\end{document}
