% latexmk -pdf -pvc main.tex
\documentclass{../util/zariski-small}



\title{Projective Space in Synthetic Algebraic Geometry}

\begin{document}


\author{Felix Cherubini, Thierry Coquand, Matthias Hutzler and David Wärn}

\maketitle

\begin{abstract}
  Synthetic algebraic geometry is a new approach to algebraic geometry. It consists in using homotopy type theory extended with three axioms, together with the interpretation of these in a higher version of the Zariski topos, in order to do algebraic geometry internally to this topos.
  
  In this article, we will show basic properties of projective n-space $\bP^n$ in synthetic algebraic geometry.
  In particular, we show that the automorphism group of $\bP^n$ is $\PGL_{n+1}(R)$ and that the picard group is $\Z$.
  We will provide different proofs of the latter statement, where the most synthetic approach naturally leads to the refined statement that the type of line bundles on $\bP^n$ is the higher type $\Z\times K(R^\times,1)$, where $K(R^\times,1)$ is a delooping of the group of units of the internal base ring $R$.
\end{abstract}

\section*{A brief introduction to Synthetic Algebraic Geometry}

In mathematics, it is common practice to assume a fixed set theory, usually with the axiom of choice, as a common basis. While it is a great advantage to work in one common language and share a lot of the basic constructions, the dual approach of adapting the  ``base language'' to particular mathematical domains is sometimes more concise, provides a new perspective and encourages new proof techniques which would be hard to find otherwise.
We use the word ``synthetic'' to indicate that the latter approach is used,
as it was used by Kock and Lawvere to describe developement of mathematics internal to certain categories \cite{lawvere-categorical-dynamics}, in particular toposes -- a program which dates back as far as 1967.

Already in the 70s in the same program, Anders Kock suggested to use the language of higher-order logic \cite{Church40} to describe the Zariski topos, the collection of sheaves for the Zariski topology \cite{Kock74,kockreyes}, which is the first occurence of synthetic algebraic geometry.
Kock's approach allowed for a more suggestive and geometrical description of schemes.
There is in particular a ``generic local ring'' $R$, which, as a sheaf, associates to any algebra $A$ its underlying set and, as described in \cite{kockreyes}, the projective space $\bP^n$ is then the set of lines in $R^{n+1}$.

Just using category theory is not the same as reasoning synthetically -- for the latter the goal is usually to derive results exclusively in one system,
as Kock and Lawvere did with differential geometry in his work.
The distinction with just using an abstraction like categories is important, since the translation from the synthetic language and back can become cumbersome -- although it is still the goal to derive statements about ordinary mathematical objects in the end.

Starting with Kock and Lawvere's work, more differential geometry was developed synthetically \cite{kock-sdg} along with a study of the models of the theory \cite{moerdijk-reyes}.
One basic axiom of the theory, called the Kock-Lawvere axiom, allows for reasoning with nilpotent infinitesimals. Our version of synthetic algebraic geometry use a generalisation of this axiom called the duality axiom. Let us now describe the Kock-Lawvere axiom.

The Kock-Lawvere axiom is added to a basic language which can be interpreted in good enough categories (for example toposes), more precisely we need basic objects like $\emptyset$, $\{\ast\}$ and $\N$ as well natural constructions like $A\times B$ or $A^B$ for objects $A$, $B$, which all behave as expected. We also need predicates $P(x)$ for elements $x:A$ so we can form subobjects like $\{x:A\mid P(x)\}$.
In this language, we assume there is a fixed ring $R$, which can be thought of as the real numbers. We define $\D(1)=\{x\in R\mid x^2=0\}$ to be the set of all square-zero elements of $R$, then the Kock-Lawvere axiom gives us a bijection
\[ e : R \times R  \to R^{\D(1)} \]
which commutes with evaluation at $0$ and projection to the first factor.
The intuition is that $\D(1)$ is so small that any function on it is linear and therefore determined by its value and its derivative at $0\in\D(1)$.
With this axiom, the derivative at $0:R$ of a function $f : R \to R$ may then be defined as $\pi_2(e^{-1}(f_{\vert \D(1)}))$. This is the start of a convenient development of differential calculus, which doesn't require any further structures on $R$ or other objects. This is the core of the synthetic method: we can work with these differential spaces as if they were sets.

To give an example, the tangent bundle of a manifold $M$ can be defined as $M^{\D(1)}$ and vector fields as sections of the map $M^{\D(1)}\to M$ evaluating at $0$. Then it is easy to see that a vector field is the same as a map $\zeta:\D(1)\to M^M$ with $\zeta(0)=\id_M$, which can be interpreted as an infinitesimal transformation of the identity map. This style of reasoning with spaces as if they were sets is also central in current synthetic algebraic geometry. 

The Kock-Lawvere axiom above and many of the axioms used in synthetic reasoning are incompatible with the law of excluded middle (LEM) and therefore also with the axiom of choice (AC). Indeed they tend to imply all maps are well-behaved (for example all maps are differentiable in the case of the Kock-Lawvere axiom), which contradicts LEM. It is however a recurring phenomenon that restricted versions of LEM and AC are compatible with synthetic languages. A very basic example is that equality of natural numbers is decidable, meaning that two natural numbers are either equal or not equal. 

%TODO: Hugo is here in the rereading

The use of nilpotent elements to capture infinitesimal quantities as mentioned above was inspired by the Grothendieck school of algebraic geometry and Anders Kock also worked with an extended axiom \cite{Kock74,kockreyes} suitable for synthetic algebraic geometry, where the role of $\D(1)$ above can be taken by any finitely presented affine scheme. In 2017 Ingo Blechschmidt finished his doctoral thesis in which he noticed a property holding internally in the Zariski-topos, which he called synthetic quasi-coherence -- this was a more general and internal version of what Kock used. In 2018, David Jaz Myers\footnote{Myers' never published on the subject, but communicated his ideas to Felix Cherubini and in talks to a larger audience \cite{myers-talk1,myers-talk2}.} started to work with a specialization of Blechschmidt's synthetic quasi-coherence and used homotopy type theory as a base language, which is the standard in synthetic algebraic geometry now and we will highlight some implications below. Myers' specialized axiom is what we now call \emph{duality axiom}.

To state the duality axiom we need the general concept behind the space $\D(1)$, which is spaces that are the common zeros of some system of polynomial equations over $R$. Such a system can be encoded representation independent by a finitely presented $R$-algebra, i.e.\ an $R$-algebra $A$ which is of the form $R[X_1,\dots,X_n]/(P_1,\dots,P_l)$ for some numbers $n,l$ and polynomials $P_i\in R[X_1,\dots,X_n]$.
Then the zero set of the system is given by the type $\Hom_{\Alg{R}}(A,R)$ of $R$-algebra homomorphisms from $A$ to the base ring, which we denote by $\Spec A$.
Now the duality axiom states that $\Spec$ is the inverse to exponentiating with $R$, i.e.\ for all 
finitely presented $R$-algebras $A$ the following is an isomorphism:
\[ (a\mapsto (\varphi\mapsto \varphi(a))) : A\to R^{\Spec A}\rlap{.}\]

Using homotopy type theory as a language for synthetic algebraic geometry is, in addition to convenience, also a language for synthetic homotopy theory.
So instead of the usual practice in algebraic topology to provide model spaces using point-set topology, one can start directly at the level of homotopy types and instead of implementing their higher structure with Kan complexes, there are rules which do not mention any implementation.
The rules of homotopy type theory allow to work with the basic objects of the theory, types, in very much the same way as one would work with sets in traditional mathematics -- with the clear exception of the law of excluded middle and the axiom of choice - although the former and restricted versions of the latter can be assumed.
Both can be seen as stating something about the spatial structure. The law of excluded middle allows us to find a complement of each subset of a given set A, which exposes A as a coproduct.
This is not true in topology, for example, $\R$ is not the coproduct of the topological subspaces $\{0\}$ and $\R/\{0\}$.
The axiom of choice states that any surjection has a section. This is also not true in topology and would trivialize all cohomology.
Thus, constructive reasoning in the sense of not using these two axioms is a necessity if we want to use spatial collections in the same way we use sets.
In synthetic algebraic geometry, we work inside homotopy type theory and remind readers of this by using the notation $x:X$ which can often be thought of as $x\in X$.
\cite{shulman-logic-of-spaces} is a more detailed introduction to homotopy type theory for a general mathematical audience.


One of the main advantages of using specifically homotopy type theory and not a different internal language,
is that it is possible to make cohomological computations, using homotopy type theory for synthetic homotopical reasoning.
This means that we are mixing two synthetic approaches, combining their advantages,
which rests on the possibility of interpreting homotopy type theory in higher toposes \cite{shulman2019all} and not just the higher topos of $\infty$-groupoids.
The general idea of using homotopy type theory to combine some kind of synthetic, spatial reasoning with synthetic homotopy theory, goes back at least to 2014, to Mike Shulman and Urs Schreiber \cite{Schreiber_2014}.
Schreiber suggested to the HoTT community at various occasions to make use of HoTT as the internal language of higher toposes, where specifities of the topos are accessed in the language via modalities.
This approach was shown to be quite effective and intuitive in Shulman's \cite{shulman-Brouwer-fixed-point} work on mixing synthetic homotopy theory in the form of HoTT and a synthetic approach to topology using a triple of modalities -- a structure called cohesion by Lawvere \cite{Lawvere2007}.

One of Schreiber's motivation was to make use of the modern perspective on cohomology, which in a higher topos can be realized as the connected components of a space of maps. This can be mimicked in HoTT, like follows: Let $X$ be a type and $A$ an abelian group and $n:\N$, then
\[ H^n(X,A):=\| X\to K(A,n) \|_0\]
is the $n$-th cohomology group of $X$ with coefficients in $A$, where $\|\_\|_0$ is the $0$-truncation, an operation which turns a type with possibly non-trivial higher identity types into a set -- a type with trivial higher structure. The type $K(A,n)$ is the $n$-th Eilenberg MacLane space, which can always be constructed for any abelian group $A$ and comes with an isomorphism $\Omega^n(K(A,n))\simeq A$.
This definition of cohomology groups allows using the synthetic homotopy theory to reason about cohomology, which had been already done successfully at the time for the cohomology of homotopy types, like spheres and finite CW-complexes, but works as well to study $0$-types.
While this internal version of cohomology does not agree with the external version mentioned above --
even the type is wrong; it is a sheaf of groups instead of a single one and, indexed by an internal natural number instead of an external one -- internal cohomology turned out to be quite useful in practice.

In 2022, trying to use this approach to calculate cohomology groups in synthetic algebraic geometry led to the discovery of what is now called Zariski-local choice \cite{draft},
which is an additional axiom that holds in the higher Zariski-topos.
It is a weakening of the axiom of choice which can be formulated as: For any surjective map $f:X\to Y$, there exists a section, i.e.\ a map $s:Y\to X$ such that $f\circ s=\id_Y$.
Zariski-local choice also states the existence of a section, but only Zariski-local and only for surjections into an affine scheme: For any surjection $f:E\to \Spec A$,
there exists a Zariski-cover $U_1,\dots,U_n$ of $\Spec A$ and maps $s_i:U_i\to E$ such that $f(s_i(x))=x$ for all $x\in U_i$.

In homotopy type theory, we use the propositional truncation $\|\_\|$ to define surjections and more generally what we mean with ``exists''.
Propositional truncation turns an arbitrary type $A$ into a type $\|A\|$ with the property $x=y$ for all $x,y:\|A\|$.
Types with this property are called propositions or (-1)-types in homotopy type theory.
Using a univalent universe of types $\mathcal U$ we have that surjection into a type $A$ are the same as type families $F:A\to \mathcal U$, such that we have $\|F(x)\|$ for all $x: A$.
Using type families instead of maps allows us to drop the condition that the maps we get are sections, since we can express it using dependent function types and we arrive at the formulation of Zariski-local choice given below in the list of axioms.
In this instance and many others, homotopy type theory provides a lot of convenience when working very formally, which is an advantage in formalization of synthetic algebraic geometry.

In total, apart from homotopy type theory and a fixed commutative ring $R$ we use in synthetic algebraic geometry the following three axioms -- we will provide some explanation for the first one below:

\begin{center}
\begin{axiom}[Locality]%
  \label{loc-axiom}
  $R$ is a local ring, i.e.\ $1\neq 0$ and whenever $x+y$ is invertible $x$ is invertible or $y$ is invertible.
\end{axiom}

\begin{axiom}[Duality]%
  \label{duality-axiom}
  For any finitely presented $R$-algebra $A$, the homomorphism
  \[ a \mapsto (\varphi\mapsto \varphi(a)) : A \to (\Spec A \to R)\]
  is an isomorphism of $R$-algebras.
\end{axiom}

\begin{axiom}[Zariski-local choice]%
  \label{Z-choice-axiom}
  Let $A$ be a finitely presented $R$-algebra
  and let $B : \Spec A \to \mU$ be a family of inhabited types.
  Then there exists a Zariski-cover $U_1,\dots,U_n\subseteq \Spec A$
  together with dependent functions $s_i : (x : U_i)\to B(x)$.
\end{axiom}
\end{center}

With the Kock-Lawvere axiom, we introduced the first historic predecessor of the duality axiom as a starting point for convenient infinitesimal computations,
while this is also possible in synthetic algebraic geometry, the general duality axiom has a lot of surprising consequences.
In line with classical algebraic geometry, it shows that we have the usual anti-equivalence between finitely presented $R$-algebras and affine schemes of finite presentation over $R$.
What is more surprising, is the consequence that all functions $R\to R$ are polynomials and that it has implications on the properties of the base ring $R$.
For example, for all $x:R$, $x$ is invertible if and only if we have $x\neq 0$.
Duality also implies that affine schemes can only have bounded maps to the natural numbers.

Surprisingly, the Zariski-local choice axiom was also usable to solve problems which have no obvious connection to cohomology.
For example, it admits a proof that pointwise open subsets of an affine scheme are the same as subsets which are given by unions of non-vanishing sets of functions on the scheme.
In more detail, we say a proposition $P$ is open, if there are a natural number $n$ and elements $r_1,\dots,r_n$ of the base ring $R$,
such that $P$ is equivalent to the proposition $r_1\neq 0 \vee\dots\vee r_n\neq 0$.
Then we call a subset $U$ of a type $X$ open, if the proposition $x\in U$ is open for all $x:X$.
Using Zariski-local choice, these pointwise existing ring elements can be turned into locally existing functions.
For an affine scheme $X$ it is even the case, that an open subset in the pointwise sense, is a union of non-vanishing sets $D(f_i)$ of global functions $f_i:\Spec A \to R$.
An analogous result holds also for closed propositions, which are propositions of the form $r_1=0\wedge\dots\wedge r_n=0$ for $r_i:R$ and vanishing sets of functions on affine schemes.

The connection between pointwise and local openness is important to make the synthetic definition of a scheme work well:
A scheme is a type $X$, that merely has a finite open cover by affine schemes.
To produce interesting examples, it is necessary to use the locality axiom.
This is related to the Zariski topology and ensures that classical examples of Zariski covers can be reproduced.
One central example is projective space, which can be defined as the quotient of $R^{n+1}/\{0\}$ by the action of $R^\times$ by scaling.
A cover of this type is given by sets of equivalence classes of the form $\{[x_0:\cdots:x_n] \vert x_i\neq 0 \}$, which is clearly open by the pointwise definition.
To see that it is a cover, one has to note that for $x:R^{n+1}$, $x\neq 0$ is equivalent to one of the entries $x_i$ being different from 0. In synthetic algebraic geometry, this is the case for the base ring $R$ and the proof uses that $R$ is a local ring.


\section*{Content of this article}
Grothendieck advocated for a functor of points approach to schemes early on in his
project of foundation of algebraic geometry (see the introduction of \cite{EGAI}).
In this approach, a scheme is defined as a special kind of (covariant) set valued functor
on the category of commutative rings. This functor should in particular
be a sheaf w.r.t.\ the Zariski topology. As a typical example, the projective space $\bP^n$
is the functor, which to a ring $A$,
associates the set of finitely presented sub-modules of  rank $1$ of
$A^{n+1}$, which are direct factors \cite{Demazure,Eisenbud,Jantzen}.

In the 70s, Anders Kock suggested to use the language of higher-order logic \cite{Church40}
to describe the Zariski topos, the collection of sheaves for the Zariski topology \cite{Kock74,kockreyes}.
This allows for
a more suggestive and geometrical description of schemes, that can now be seen as a special kind
of types satisfying some properties. Morphisms of schemes in this setting are just general maps.
There is in particular a ``generic
local ring'' $R$, which associates to $A$ its underlying set. As described in \cite{kockreyes}
the projective space $\bP^n$ is then the set of lines in $R^{n+1}$.

A natural question is if we can show in this setting that the automorphism group of $\bP^n$
is  $\PGL_{n+1}(R)$.
More generally, can we show that any map $\bP^n\rightarrow \bP^m$ is given by $m+1$ homogeneous
polynomials of same degree in $n+1$ variables?
From this, it is possible to deduce the corresponding result about $\bP^n$ defined as
a functor of points (but the maps are now {\em natural transformations}) or about $\bP^n$ defined
as a scheme (but the maps are now {\em maps of schemes}).
While this result is a basic text book in the case of projective space over a field, the general
case is more subtle.
(This general result, though fundamental, is not in \cite{Hartshorne} for instance.)
One goal of this paper is to present such a proof.

In \cite{draft}, we presented an axiomatisation of the Zariski {\em higher topos} \cite{lurie-htt},
using instead of the language of higher-order logic the language of dependent type theory
with univalence \cite{hott}. The first axiom is that we have a local ring $R$. We then define
an affine scheme to be a type of the form $\Sp(A) = \Hom_{\Alg{R}}(A,R)$ for some finitely presented
$R$-algebra $A$. The second axiom, inspired from the work of Ingo Blechschmidt \cite{ingo-thesis},
states that the evaluation map $A\rightarrow R^{\Sp(A)}$ is a bijection. The last axiom states
that each $\Sp(A)$ satisfies some form of local choice \cite{draft}. We can then define a notion
of {\em open} proposition, with the corresponding notion of open subset, and define a scheme as a type
covered by a finite number of open subsets that are affine schemes. In particular, we define
$\bP^n$ as in \cite{kockreyes} and show that it is a scheme.
In this setting, dependent type theory with univalence extended with these 3 axioms,
we show the above result about maps between $\bP^n$ and $\bP^m$ and the result about automorphisms of $\bP^n$.

Interestingly, though these results are
about the Zariski $1$-topos, the proof makes use of types that
are not (homotopy) sets (in the sense of \cite{hott}),
since it proceeds in characterizing $\bP^n\rightarrow\KR$, where $\KR$ is the delooping
(thus a type which is not a set) of the multiplicative group of units of $R$.
More technically, we also use such higher types as an alternative to the technique
of Quillen patching \cite{Quillen,lombardi-quitte,Lam}.



%% Schemes as special kind of sheaf for Zariski topos.

%% Even nicer in a type theoretic framework

%% Anders Kock property of Zariski topos.

%% Zariski topos higher logic

%% Definition of $\bP^n$ as a set of lines in $R^{n+1}$ coincides with the definition
%% of projective as functor of points (Demazure? Eisenbud?)

%% ``Geometric'' definition

%% Meyers, Blechschmidt use of type theory with univalence

%% Axiomatisation of the Zariski (higher) topos

%% A scheme is defined as a type satisfying some property and a map of schemes is {\em any} function
%% between the corresponding types



\paragraph{Acknowledgements.}
We thank Thierry Coquand for discussions on the topic and in particular for explaining a proof of \Cref{extend-from-image} to us.
We thank Marc Nieper-Wißkirchen for a discussion which led to the explanation in \Cref{remark-sym-dual}.
Work on this article was supported by the ForCUTT project, ERC advanced grant 101053291.

\section[Definition of projective space and some linear algebra]{Definition of $\bP^n$ and some linear algebra}
We follow the notations and setting for Synthetic Algebraic Geometry \cite{draft}.
In particular, $R$ denotes the generic local ring and $R^\times$ is the multiplicative group of units of $R$.

In Synthetic Algebraic Geometry, a scheme is defined as a set satisfying some property \cite[Definiton 5.1.1]{draft}. In particular
the projective space $\bP^n$ can be defined to be the quotient of $R^{n+1}\setminus\{0\}$ by the
equivalence relation $a\sim b$ which expresses that $a$ and $b$ are proportional, i.e.\ $a_ib_j=a_jb_i$. This proportionality is a closed proposition which is equal to $\Sigma_{r:R^\times}ar = b$.
We can then prove \cite[Theorem 6.1.5]{draft} that this set is a scheme. This definition goes back to \cite{Kock74}.

 In this setting, a map of schemes is simply an arbitrary set theoretic map. An application of this work is to show
 that the maps $\bP^n\rightarrow \bP^m$ are given by $m+1$ homogeneous polynomials of the same degree in $n+1$ variables.

\medskip


There is another definition of $\bP^n$ which uses ``higher'' notions. Let $\KR= K(R^\times,1)$ be the delooping
of $R^\times$. It can be defined as the type of lines $\Sigma_{M:\Mod{R}}\|{M=R^1}\|$. Over $\KR$ we have the
family of sets
$$E_n(l) = l^{n+1}\setminus\{0\}$$
Note that we use the same notation for an element $l : \KR$,
its underlying $R$-module and its underlying set.
An equivalent definition of $\bP^n$ is then
$$
\bP^n = \sum_{l:\KR}E_n(l)
$$
This is a general construction of the \emph{homotopy} quotient of a type by a group action --
the type in this case is $E(R^1)=R^{n+1}\setminus\{0\}$ and the action of $R^\times=\GL(1,R)=\Omega (\KR,R^1)$ is given by transporting in $E(R^1)$ along a loop $l:\Omega (\KR,R^1)$.
One can calculate, that this transport and therefore the action corresponds to scalar multiplication.
Since this is a free group action, the homotopy quotient, given by the sigma type above, will be a set.
This is part of the homotopy typetheoretic view on group theory and group actions, which is explained in detail in \cite{Sym}.

We will use the following constructions of $\bP^n$ and the identifications between them given below:
\begin{remark}\label{identification-Pn}
  Projective $n$-space $\bP^n$ is given by the following equivalent constructions of which we prefer
  the first in this article:
  \begin{center}
  
    \begin{enumerate}[(i)]
    \item $\sum_{l:\KR}E_n(l)$
    \item The set-quotient $R^{n+1}\setminus\{0\}/R^\times$, where $R^\times$ acts on non-zero vectors in $R^{n+1}$ by multiplication.
    \item For any $k$ and $R$-module $V$ we define the \emph{Grassmannian}
      \[ \Gr(k,V)\colonequiv \{ U\subseteq V \mid \text{$U$ is an $R$-submodule and $\|U=R^k\|$ }\} \rlap{.}\]
      Projective $n$-space is then $\Gr(1,R^{n+1})$.
    \end{enumerate}
    
  \end{center}
  We use the following, well-defined identifications:
  \begin{center}
  
  \begin{enumerate}
  \item[] (i)$\to$(iii): Map $(l,s)$ to $R\cdot (u s_0,\dots, u s_n)$ where $u:l=R^1$
  \item[] (iii)$\to $(i): Map $L\subseteq R^{n+1}$ to $(L, x)$ for a non-zero $x\in L$
  \item[] (ii)$\leftrightarrow$ (iii): A line through a non-zero $x:R^{n+1}$
          is identified with $[x]:R^{n+1}\setminus\{0\}/R^\times$
  \end{enumerate}
    
  \end{center}
\end{remark}

We construct the standard line bundles $\OO(d)$ for all $d\in\Z$,
which are classically known as \emph{Serre's twisting sheaves} on $\bP^n$ as follows:

\begin{definition}
  For $d:\Z$, the line bundle $\OO(d):\bP^n\to \KR$ is given by $\OO(d)(l,s) = l^{\otimes d}$
  and the following definition of $l^{\otimes d}$ by cases:
  \begin{enumerate}[(i)]
  \item $d \geqslant 0$: $l^{\otimes d}$ using the tensor product of $R$-modules
  \item $d < 0$: $(l^{\vee})^{-d}$, where $l^{\vee}\colonequiv\Hom_{\Mod{R}}(l,R^1)$ is the dual of $l$.
  \end{enumerate}
\end{definition}

This definition of $\OO(d)$ agrees with \cite{draft}[Definition 6.3.2] where $\OO(-1)$
is given on $\Gr(1,R^{n+1})$ by mapping submodules of $R^{n+1}$ to $\KR$.
Using the identification of $\bP^n$ from \Cref{identification-Pn} we can give the following explicit equality:

\begin{remark}
  We have a commutative triangle:
  \begin{center}
    \begin{tikzcd}
        \sum_{l:\KR} E_n(l)\ar[rr]\ar[dr,swap,"\OO(1)"] && R^{n+1}\setminus\{0\}/R^\times\ar[ld,"\OO(1)"] \\
                  & \KR &
    \end{tikzcd}
  \end{center}

by the isomorphism given for $(l,s)$ by mapping $x:l$ to $r(u s_0,\dots, u s_n)\mapsto r(u x)$ for some isomorphism $u:l\cong R^1$.
\end{remark}

\medskip

 Connected to this definition of $\bP^n$, we will prove some equalities in the following.
 To prove these equalities, we will make use of the following lemma, which holds in synthetic algebraic geometry:
 
\begin{lemma}\label{invariant-implies-homogenous}
  Let $n,d:\N$ and $\alpha:R^n\to R$ be a map such that
  \[\alpha(\lambda x)=\lambda^d\alpha(x)\]
  then $\alpha$ is a homogenous polynomial of degree $d$.
\end{lemma}

\begin{proof}
  By duality, any map $\alpha:R^n\to R$ is a polynomial.
  To see it is homogenous of degree $d$, let us first note that any $P:R[\lambda]$ with $P(\lambda)=\lambda^d P(1)$
  for all $\lambda:R^\times$ also satisfies this equation for all $\lambda : R$ and is therefore homogenous of degree $d$.
  Then for $\alpha'_x:R[\lambda]$ given by $\alpha'_x(\lambda)\colonequiv \alpha(\lambda\cdot x)$
  we have $\alpha'_x(\lambda)=\lambda^d \alpha'_x(1)$. This means any coeffiecent of $\alpha'_x$
  of degree different from $d$ is 0. Since this means every monomial appearing in $\alpha$,
  which is not of degree $d$, is zero for all $x$ and therefore 0.   
\end{proof}

\begin{proposition}\label{end}
  $$\prod_{l:\KR}l^n\rightarrow l \;\;\;=\;\;\; \Hom(R^n,R)$$
\end{proposition}

\begin{proof}
We rewrite $\Hom(R^n,R)$, the set of $R$-module morphism, as
$$
\sum_{\alpha:R^n\rightarrow R}\prod_{\lambda:R^\times}\prod_{x:R^n}\alpha(\lambda x) = \lambda \alpha(x)
$$
using \Cref{invariant-implies-homogenous} with $d=1$.

\medskip


It is then a general fact that if we have a pointed connected groupoid $(A,a)$ and a family of
sets $T(x)$ for $x:A$, then $\prod_{x:A}T(x)$ is the set of fixed points of $T(a)$ for the $\Omega(A,a)$-action: If we have $f:\prod_{x:A}T(x)$, then for any $g:\Omega(A,a)$ we have $g(f_a)=f_a$ by depedent application of $f$ to $g$. And if $z:T(a)$ is a fixed point and $x:A$, the transport along a $p:a=x$ is independent of $p$, so we can construct a $z':T(x)$.
\end{proof}

We will use the following remark, proved in \cite{draft}[Remark 6.2.5].

\begin{lemma}\label{ext}
  Any map $R^{n+1}\setminus\{0\}\rightarrow R$ can be uniquely extended to a map $R^{n+1}\rightarrow R$ for $n>0$.
\end{lemma}

We will also use the following proposition, already noticed in \cite{draft}.

\begin{proposition}\label{const}
  Any map from $\bP^n$ to $R$ is constant.
\end{proposition}

\begin{proof}
  Since $\bP^n$ is a quotient of $R^{n+1}\setminus\{0\}$, the set $\bP^n\rightarrow R$ is
  the set of maps $\alpha:R^{n+1}\setminus\{0\}\rightarrow R$
  such that $\alpha(\lambda x) = \alpha(x)$ for all $\lambda$ in $R^\times$.
  These are exactly the constant maps
  using \Cref{ext} and \Cref{invariant-implies-homogenous} with $d=0$.
\end{proof}

\begin{proposition}\label{aut}
  For all $n:\N$ we have:
$$\prod_{l:\KR}E_n(l)\rightarrow E_n(l) \;\;=\;\; \GL_{n+1}$$
\end{proposition}

\begin{proof}
  For $n=0$, this is the direct computation that a Laurent-polynomial $\alpha:(R[X,1/X])^\times$ which satisfies
  $\alpha(\lambda x)=\lambda \alpha(x)$ is $\lambda\alpha(1)$ where $\alpha(1):R^\times=\GL_1$.
  
  \medskip
  
  For $n>0$, the proposition follows from two remarks.

  The first remark is that maps $E_n(R)\to E_n(R)$, which are invariant under the induced $\KR$ action, are linear.
  To prove this remark, we first map from $E_n(l)\to E_n(l)$ to $E_n(l)\to l^{n+1}$ by composing with the inclusion.
  Maps of the latter kind can be uniquely extended to maps $l^{n+1}\to l^{n+1}$, since by 
  \Cref{ext} the restriction map
$$
(l^{n+1}\rightarrow l)\rightarrow ((l^{n+1}\setminus\{0\})\rightarrow l)
$$
is a bijection for $n>0$ and all $l:\KR$.

\medskip

The second remark is that a linear map $u:R^{m}\rightarrow R^{m}$ such that
$$
x\neq 0~\rightarrow~u(x)\neq 0
$$
is exactly an element of $\GL_{m}$.

We show this by induction on $m$. For $m=1$ we have $u(1)\neq 0$ iff $u(1)$ invertible.

For $m>1$, we look at $u(e_1) = \Sigma \alpha_ie_i$ with $e_1,\dots,e_m$ basis of $R^m$.
We have that some $\alpha_j$ is invertible.
By composing $u$ with an element in $\GL_m$, we can then
assume that $u(e_1) = e_1+v_1$ and $u(e_i) = v_i$, for $i>1$, with $v_1,\dots,v_m$ in $Re_2+\dots+Re_m$.
We can then conclude by induction.
\end{proof}

We can generalize \Cref{end}
and get a result related to \Cref{aut} as follows.
 
\begin{lemma}\label{hom}
  \begin{enumerate}[(i)]
    \item
      \[  \prod_{l:\KR}l^n\rightarrow l^{\otimes d} \;\;=\;\; (R[X_1, \dots, X_n])_d \]
      That is,
      every element of the left-hand side is given by
      a unique homogeneous polynomial of degree $d$ in $n$ variables.
    \item
      An element in
      $$\prod_{l:\KR}E_n(l)\rightarrow E_m(l^{\otimes d})$$
      is given by $m+1$ homogeneous polynomials $p = (p_0,\dots,p_m)$ of degree $d$ such that
      $x\neq 0$ implies $p(x)\neq 0$.
  \end{enumerate}
\end{lemma}

\begin{proof}
We show the first item. Following \cite{Sym} again, this product is the set of maps $\alpha:R^n\rightarrow R^{\otimes d}$
which are invariant by the $R^\times$-action which in this case acts by mapping $\alpha$ to $r^d\alpha(r^{-1} x)$ for each $r:R^\times$.
So by \Cref{invariant-implies-homogenous} these are exactly the maps given by homogeneous polynomials of degree $d$.
\end{proof}


%\section{Horrocks Theorem}
%\input{horrock.tex}

\section{Line bundles on affine schemes}
\input{line-bundles.tex}

\section{Application of the Veronese embedding}
\input{veronese}

\section[Picard group of projective space]{Picard group of $\bP^1$}
%% The goal of this note is to show that $\Pic(\bP^n) = \Z$ in the setting of Synthetic Algebraic
%% Geometry \cite{draft}. We actually present a strengthening of this result,
%% which in particular states the equivalence
%% $$(\bP^1\to{\KR})\simeq (\Z\times \KR)$$
%% %We will also present Matthias' strengthened version of this statement, which states that the
%% %map $\Z\times \KR\rightarrow {\KR}^{\bP^1},~p,l\mapsto (x\mapsto l\times \OO(p)(x))$
%% %is an equivalence.

%% One application is that $\Aut(\bP^n)$ is $\PGL_{n+1}$.

%% For the case $n=1$,
%% we follow the proof of Horrock's Lemma as presented in Lam's book on Serre's problem \cite{Lam}
%% on projective modules\footnote{This argument is different from the one presented in Lombardi-Quitt\'e \cite{lombardi-quitte}; we instead give a constructive version of the proof of Nashier-Nichols \cite{Nashier}.}.
%% For the general case, we don't follow the Quillen patching technique
%% presented in the 1976 paper \cite{Quillen}, but instead present an argument which uses
%% our description of ${\bP^1}\to{\KR}$.
%% We then explain how we can deduce that $\Aut(\bP^n)$ is $\PGL_{n+1}$.

%%  One point of this work is to show that all these results can be proven axiomatically in the
%%  setting of univalent type theory with the 3 axioms described in \cite{draft}.
%%  ALREADY IN THE INTRODUCTION

% The following result also holds for a general \emph{connected}\footnote{If $e(1-e) = 0$ then $e=0$ or $e=1$.} ring,
% without assuming a finite presentation. 

 \begin{lemma}\label{stand}
   Let $A$ be a \emph{connected}\footnote{If $e(1-e) = 0$ then $e=0$ or $e=1$.} ring, then
   an invertible element of $A[X,1/X]$ can be written $X^N\Sigma a_nX^n$ with $N$ in $\Z$
   and $a_0$ unit and $a_n$ nilpotent if $n\neq 0$.
 \end{lemma}

 \begin{proof}
   Let $P = \sum_i a_iX^i:A[X,1/X]$ be invertible.
   The result is clear if $A$ is an integral domain. Let $B(A)$ is the constructible spectrum of $A$
   with the two generating maps $D(a)$ and $V(a)$ for $a$ in $A$ \cite{lombardi-quitte}. The argument for an integral
   domain, looking at $D(a)$ as $a\neq 0$ and $V(a)$ as $a =0$, shows that we have $\vee_i D(a_i) = 1$
   and $D(a_ia_j) = 0$ for $i\neq j$. Since $A$ is connected, this implies that exactly one $a_i$ is a unit,
   and all the other coefficient are nilpotent.
   %% By duality we can view the coefficients as functions $a_i,b_i:\Spec(A)\to R$.
   %% For all $x:\Spec(A)$, we get an invertible $P_x:R[X,1/X]$ by evaluating the coefficients of $P$ at $x$.
   %% Then $P_x\cdot Q_x=1$ and in particular $1=\sum_{i+j=0}a_ib_j$, so by locality of $R$, we have $i$ such that $a_i$ is invertible.
   %% Without loss of generality, we assume $i=0$ and want to show $\neg\neg (a_j=0)$ for $j\neq 0$.
   %% Since we prove a negated proposition, we can assume that we have $l,k$ minimal with $a_l$ and $b_k$ invertible.
   %% Then we must have $k+l=0$ because we would have $0=a_lb_k$ otherwise.
   %% $k$ was minimal, so it is $0$ and $l$ is $0$ as well.
   %% The same reasoning applies for a maximal choice of $k$.
 \end{proof}
 
 Using this Lemma we deduce the following, for $A$ connected ring.

\begin{lemma}\label{nilpotent}
  Any invertible element of $A[X,1/X]$ can be written uniquely as a product
  $uX^l(1+a)(1+b)$ with $l$ in $\Z$, $u$ in $A^{\times}$ and $a$ (resp. $b$)
  polynomial in $XA[X]$ (resp. $1/XA[1/X]$) with only nilpotent coefficients.
\end{lemma}

\begin{proof}
  Write $\Sigma v_nX^n$ the invertible element of $A[X,1/X]$.
  W.l.o.g. we can assume, by the previous Lemma, that the polynomial is of the form $v_0 + \Sigma v_nX^n$ with
  all $v_n,~n\neq 0$ nilpotent.
  We let $J$ be the ideal generated by these nilpotent elements.
  We have some $N$ such that $J^N = 0$.
  
  We first multiply by the inverse of $v_0 + \Sigma_{n<0}v_nX^n$, making all coefficients of
  $X^n,~n<0$ in $J^2$.
  We keep doing this until all these elements are $0$.
  %We then do the same, killing all coefficients of $X^n$ for $n>0$.
  We have then written the invertible polynomials on the form $u(1+a)(1+b)$ with $u$ unit in $A$.

  Such a decomposition is unique: if we have $(1+a)(1+b)$ in $A^{\times}$ with $a = \Sigma_{n> 0}a_nX^n$
  and $b = \Sigma_{n<0}b_nX^n$ then we have $a_n = 0$ for $n>0$ and $b_n = 0$ for $n<0$.
\end{proof}

\begin{corollary}\label{Pic1}
  We have $\prod_{L:\bP^1\rightarrow \KR}\Sigma_{p:\Z}\|L = \OO(p)\|$
\end{corollary}

\begin{proof}
A line bundle $L([x_0,x_1])$ on $\bP^1$ is trivial on each of the affine charts $x_0\neq 0$ and $x_1\neq 0$ by Corollary \ref{c1}, so
it is characterised by an invertible Laurent polynomial on $R$, and the result follows from Lemma \ref{nilpotent}.
\end{proof}

We can then state the following strengthening.

\begin{proposition}\label{Matthias}
  The map $\KR\times\Z\rightarrow (\bP^1\rightarrow \KR)$
  which associates to $(l_0,d)$ the map $x\mapsto l_0\otimes \OO(d)(x)$ is an equivalence.
\end{proposition}

\begin{proof}
  Corollary \ref{Pic1} shows that this map is surjective.
  So we can conclude by showing that the map is also an embedding.
  For $(l,d),(l',d'):\KR\times\Z$ let us first consider the case $d=d'$. 
  Then we merely have $(l,d)=(\ast,d)$ and $(l',d')=(\ast,d)$,
  so it is enough to note that the induced map on loop spaces based at $(\ast,d)$ is an equivalence by \Cref{const}.
  Now let $d\neq d'$. To conclude we have to show $\OO(k)$ is different from $\OO(0)$ for $k\neq 0$.
  It is enough to show that for $k>0$ the bundle $\OO(k)$ has at least two linear independent sections,
  since we know $\OO(0)$ only has constant sections by \Cref{const}.
  This follows from the fact that $\OO(k)(x)$ is $\Hom_{\Mod{R}}(Rx^{\otimes k},R)$ and has all projections as sections.
  %% which we can naturally identify with $\Hom_{\Mod{R}}(R(x_0^k,x_1^k),R)$ for $x=(x_0,x_1)$.
  %% Then $s_0(r\cdot (x_0^k,x_1^k))\colonequiv r\cdot x_o^k$ and $s_1(r\cdot (x_0^k,x_1^k))\colonequiv r\cdot x_1^k$ define two sections
  %% which are independent since $s_0([0:1])=0\neq s_1([0:1])$ and $s_1([1:0])=0\neq s_1([1:0])$.
  %TC: I think it is clearer not so say too much here. 
\end{proof}

 It is a curious remark that $\KR\rightarrow \KR$ is also equivalent
 to $\KR\times \Hom_{\mathrm{Group}}(R^\times,R^\times) = \KR\times\Z$.

\begin{corollary}\label{Matthias1}
  We have $\prod_{L:\bP^1\rightarrow \KR}\prod_{x:R}L([1:x]) = L([0:1])$.
\end{corollary}

\begin{proof}
  By the equivalence in \Cref{Matthias}, we have
  \[ \prod_{L:\bP^1\to \KR} \,\prod_{x : \bP^1}  L(x)=l_0\otimes \OO(d)(x) \]
  for some $(l_0,d)$ corresponding to $L$.
  $\OO(d)([0:1])$ can be identified with $R^1$ and $\OO(d)$ is trivial on $R$,
  so we have $L([1:x])=l_0=L([0:1])$ for all $x:R$.
\end{proof}

\section{Line bundles on $\bP^n$}
We will prove $\Pic(\bP^n)=\Z$ and a strengthening thereof in this section by mostly algebraic means.
In \Cref{geometric-proof} we will give a shorter geometric proof.

We can now reformulate Quillen's argument for Theorem 2' \cite{Quillen} in our setting.

\begin{proposition}\label{trivial}
  For all $V:\bP^n\rightarrow \KR$ we have ${\prod_{s:R^n}V([1:s]) = V([0:1:0:\cdots :0])}$.
\end{proposition}

\begin{proof}
  We define $L:R^{n-1}\rightarrow (\bP^1 \to \KR)$ by $L~t~[x_0:x_1] = V([x_0:x_1:x_0t])$.
  Let $s=(s_1,\dots,s_{n}):R^{n}$. We apply Corollary \ref{Matthias1} and we get
  \[
   V([1:s]) = L~(s_2,\dots,s_n)~[1:s_1] = L~(s_2,\dots,s_n)~[0:1] = V([0:1:0:\cdots :0])
   \rlap{.}
  \]
\end{proof}

 Note that the use of Corollary \ref{Matthias1} replaces the use of the ``Quillen patching''
 \cite{lombardi-quitte} introduced in \cite{Quillen}.

\medskip

Let $T$ be the ring of polynomials $u = \Sigma_p u(p)X^p$ with
$X^p = X_0^{p_0}\dots X_n^{p_n}$ with $\Sigma p_i = 0$. We write $T_l$ for the subring
of $T$ which contains only monomials $X^p$ with $p_i\geqslant 0$ if $i\neq l$
and $T_{lm}$ the subring of $T$ 
which contains only monomials $X^p$ with $p_i\geqslant 0$ if $i\neq l$ and $i\neq m$.

Note that $T_l$ is the polynomial ring $T_l = R[X_0/X_l,\dots,X_n/X_l]$.

A line bundle on $\bP^n$ is given by compatible line bundles on each $\Spec(T_l)$.

By \Cref{trivial}, a line bundle on $\bP^n$ is trivial on each $\Spec(T_l)$.
So it is determined by $t_{ij}$ invertible in $T_i[X_i/X_j] = T_j[X_j/X_i] = T_{ij}$
such that $t_{ik} = t_{ij}t_{jk}$ and $t_{ii} = 1$. 
Using \Cref{stand} we can assume without loss of generality, that
$t_{ij} = (X_i/X_j)^{N_{ij}} u_{ij}$, for some $N_{ij}$ in $\Z$, where $u_{ij}(p)$ is invertible for $p = 0$
and all other coefficients $u_{ij}(p)$ for $p\neq 0$
are nilpotent. By looking at the relation  $t_{ik} = t_{ij}t_{jk}$ when we quotient by nilpotent elements, we see that
$N_{ij} = N$ does not depend on $i,j$.
The result $\Pic(\bP^n) = \Z$ will be a consequence of the following result, proved in the Appendix.

\begin{proposition}\label{units}
  There exists $s_i$ invertible in $T_i$ such that $u_{ij} = s_i/s_j$ 
\end{proposition}

\begin{corollary}
  $\Pic(\bP^n) = \Z$.
\end{corollary}

We can then strengthen this result, with the same reasoning as in Proposition \ref{Matthias}.

\begin{theorem}\label{Matthias2}
  The map $\KR\times\Z\rightarrow (\bP^n\rightarrow \KR)$
  which associates to $l_0,d$ the map $x\mapsto l_0\otimes \OO(d)(x)$ is an equivalence.
\end{theorem}

We deduce from this a characterisation of the maps $\bP^n\rightarrow\bP^m$.% using Lemma \ref{hom}.

\begin{corollary}\label{map}
  A map $\bP^n\rightarrow\bP^m$ is given by $m+1$ homogeneous polynomials $p = (p_0,\dots,p_m)$ on $R^{n+1}$
  of the same   degree $d$ such that $x\neq 0$ implies $p(x)\neq 0$.
\end{corollary}

\begin{proof}
Write $T_n(l)$ for $l^{n+1}\setminus\{0\}$. We have $\bP^n = \Sigma_{l:\KR}T_n(l)$ and so
$$
\bP^n\rightarrow\bP^m = \sum_{s:\bP^n\rightarrow \KR}\prod_{x:\bP^n}T_m(s~x)
$$
Using Theorem \ref{Matthias2}, this is equal to
$$
\sum_{l_0:\KR}\sum_{d:\Z}\prod_{l:\KR}T_n(l)\rightarrow T_m(l_0\otimes l^{\otimes d})
$$
and, as for Lemma \ref{hom}, this is the set of tuples of $m+1$ polynomials in $R[X_0,\dots,X_n]$ homogenenous
of degree $d$, sending $x\neq 0$ to $p(x)\neq 0$, and quotiented by proportionality.
\end{proof}

We deduce the characterisation of $\Aut(\bP^n)$. This is a
remarkable result, since the automorphisms are in this framework only bijections of sets.

\begin{corollary}
  $\Aut(\bP^n)$ is $\PGL_{n+1}$.
\end{corollary}

We also have the following application of computation of cohomology groups \cite{draft}.

\begin{corollary}
A function $\bP^n\rightarrow\bP^m$ is constant if $n>m$.
\end{corollary}

\begin{proof}
We proved in \cite{cech-draft} that cohomology groups can be computed as Cech cohomology for any
finite open acyclic covering and used this to prove $H^n(\bP^n,\OO(-n-1))=R$.
By \Cref{map}, a map $\bP^n\rightarrow\bP^m$ is given by $m+1$ non zero polynomials
$p(x) = (p_0(x),\dots,p_m(x))$ homogeneous of the same degree $d\geqslant 0$ and such that $x\neq 0$ implies $p(x)\neq 0$.
This means that $\bP^n$ is covered by $m+1$ open subsets $U_i(x)$ defined by $p_i(x)\neq 0$.
I claim that we should have $d=0$.

 If $q(x)$ is a non zero homogeneous polynomial of degree $d>0$, the open $q(x)\neq 0$ defines an {\em affine}
 and hence acyclic \cite{cech-draft}, open subset of $\bP^n$ (see e.g. Exercise 3.5 in \cite{Hartshorne}).
 It follows that the covering $U_0,\dots,U_m$ is acyclic if $d>0$. But this contradicts $H^n(\bP^n,\OO(-n-1))=R$.

 Hence $d=0$ and the map is constant.
\end{proof}


\section[Automorphism group of projective space]{Another proof of $\Aut(\bP^n) = \PGL_{n+1}$}
\input{alt.tex}

\section[Picard group of projective space (geometric)]{A geometric proof of $\Pic(\bP^n)=\Z$}
\label{geometric-proof}
\input{geometric-proof.tex}

\newpage

\section*{Appendix 1: Horrock's Theorem}

We present an alternative constructive proof of the the following special case of the {\em affine}
Horrocks Theorem \cite{Lam}, V.2, for a commutative ring $A$. We
essentially follow Nashier-Nichols' Proof of Horrocks Theorem, as presented in \cite{Lam}, IV.5.

\begin{lemma}\label{Horrocks}
  If an ideal of $A[X]$ divides a principal ideal $(f)$ with $f$ monic then it is itself a principal ideal.
\end{lemma}

Let $L$ and $M$ be such that $L\cdot M = (f)$. We can then write $f = \Sigma u_iv_i$ with $u_i$ in $L$ and
$v_i$ in $M$. Using $f$ monic, we then have $L = (u_1,\dots,u_n)$ and $M = (v_1,\dots,v_n)$.
The strategy of the proof is to build comaximal monoids $S_1,\dots,S_l$ in $A$ \cite{lombardi-quitte},
XIV.1, such that $L$ is generated by a monic polynomial in each $A_{S_j}[X]$.

\subsection{Formal computation of gcd}

%We start by describing a general technique introduced in \cite{lombardi-quitte}.

 If we have a list $u_1,\dots,u_n$ of polynomials over a field we can compute the gcd of this list
$(g) = (u_1,\dots,u_n)$ and $g$ is either $0$ (in the case where all the polynomials $u_1,\dots,u_n$ are $0$)
 or a monic polynomial. More generally, over a local ring $A$ which is residually discrete of maximal
 ideal $m$, we can compute a polynomial $g$ in $A[X]$ such that 
 $(g) = (u_1,\dots,u_n)$ modulo $m$
 and $g$ is either $0$ (in the case where all the polynomials $u_1,\dots,u_n$ are $0$ modulo $m$)
 or a monic polynomial.

In general, if we are over an arbitrary ring $A$, we can interpret this computation formally as
follows (\cite{lombardi-quitte}, XIV.1). We build a binary tree of root $A$, where at each node,
we intuitively force an element of $A$ to be in an approximation of a maximal ideal
or to be invertible modulo this approximation.
To each branch is associated a pair of finite sets $I;U$
of elements in $A$ and we associate the monoid $S(I;U) = M(U) + (I)$ where $M(U)$ is the multiplicative
monoid generated by $U$ and $(I)$ the ideal generated by $I$.

Corresponding to the formal computation of the gcd, we get a binary tree where we have at each leaf
a ring $A_{S(I;U)}$ and a polynomial $g$ in $A_{S(I;U)}[X]$, which is monic or $0$, and
such that $(g) = (u_1,\dots,u_n)$ modulo the ideal generated by $I$ in $A_{S(I;U)}$.

%% We have assumed that the ideal $(u_1,\dots,u_n)$ in $A[X]$
%% contains a monic polynomial, so we can only have $g = 0$ if $1=0$ in $A(I;U)$ and we can replace
%% $g$ by $1$. So in the case where the ideal $(u_1,\dots,u_n)$ in $A[X]$ contains a monic
%% polynomial, we can assume that on each leach we have a {\em monic} polynomial.

%% Like in \cite{lombardi-quitte}, XIV.1, we can also associate to each branch
%% the multiplicative monoid $S(I;U) = M(U) + (I)$.
%% In the ring $A_{S(I;U)}$ we force the elements in $U$ to be invertible {\em and} the elements
%% in $I$ to be in the Jacobson radical \cite{lombardi-quitte}.
If we do this for each branch, we get a list of monoids $S_1,\dots,S_l$
that are {\em comaximal} (\cite{lombardi-quitte}, XIV.1): if $s_i$ in $S_i$ then $1 = (s_1,\dots,s_l)$.

\subsection{Application to Horrocks' Theorem}

We assume $f = \Sigma u_iv_i$ and $fp_{ij} = u_iv_j$ with $\Sigma p_{ii} = 1$
in $A[X]$. The goal is to build comaximal monoids $S_1,\dots,S_l$ with $(u_1,\dots,u_n)$ principal
and generated by a monic polynomial in $A_{S_j}[X]$.

Note that $(u_1,\dots,u_n)$ contains the monic polynomial $f$.

We first build a binary tree which corresponds to the formal computation of the gcd of
$u_1,\dots,u_n$ as described above. For each branch $I;U$ we have a monic polynomial
$g$ in $A_{S(I;U)}[X]$, which belongs to $(u_1,\dots,u_n)$, and
such that $(u_1,\dots,u_n) = (g)$ modulo\footnote{Since $(u_1,\dots,u_n)$
contains a monic polynomial, the case where $g=0$ can only happen if $0$ belongs to in $S(I;U)$ and in this
case, we can replace $g$ by $1$.} the ideal generated by $I$.

The next Lemma is Lemma IV.5.1, \cite{Lam} in Lam's presentation of
Nashier-Nichols' Proof of Horrocks Theorem.

\begin{lemma}
  Let $R$ be a ring with an ideal $J$ contained in the Jacobson radical
  and $L$ an ideal of $R[X]$ which contains a monic polynomial. We consider
  the reduction modulo $J$
  $$\pi: R[X]\mapsto (R/J)[X]$$
  Any monic polynomial of $\pi(L)$ can be lifted to a monic polynomial in $R[X]$.
\end{lemma}

 Using this Lemma, we get a monic polynomial $h$ in $(u_1,\dots,u_n)$ in $A_{S(I;U)}[X]$
 and such that $h$ generates $(u_1,\dots,u_n)$ modulo $(I)$.
 We can now use that $I$ is contained in the Jacobson radical of $A_{S(I;U)}$ and the
 following second Lemma, which corresponds to Proposition IV.5.2 of \cite{Lam},
 to conclude that we actually have $(h) = (u_1,\dots,u_n)$ in $A_{S(I;U)}[X]$.

\begin{lemma}
  Let $R$ a ring, $J$ an ideal of $R$ contained in the Jacobson radical of $R$. If
  we have $L\cdot M = (f)$ with $f$ monic in $R[X]$, and $L$ contains a monic polynomial
  $h$ such that $L = (h)$ in $(R/J)[X]$ then $L = (h)$ in $R[X]$.
\end{lemma}

\begin{proof}
  Since $L$ contains $L\cap J$ and $L\cdot M = (f)$ with $f$ regular (since $f$ is monic),
  we can find $K$   such that $L\cdot K = L\cap J$.
  We then have $L\cdot K = 0$ modulo $J$ and hence $K = 0$ modulo $J$ since $L$ contains $f$
  which is monic.
  This means $L\cap J = L\cdot J$. Then we have $L = (h) + L\cdot J$.
  The result then follows from the fact that $h$ is monic and from Nakayama's Lemma, as in Lam \cite{Lam}:
  the module $P = L/(h)$ is a finitely generated module over $R$ and satisfies
  $P\subseteq JP$ and $J$ is contained in the Jacobson radical of $R$, so $P = 0$ by Nakayama's Lemma.
\end{proof}

\begin{corollary}
  We can find comaximal elements $s_1,\dots,s_l$ such that $(u_1,\dots,u_n)$ is principal and generated by a
  monic polynomial in each $A_{s_j}[X]$. Since these monic polynomials are uniquely determined
  we can patch these generators and get that $(u_1,\dots,u_n)$ is principal in $A[X]$\footnote{If $A$ is not
  connected, the generator of $(u_1,\dots,u_n)$ may not be monic: if $e(1-e)=1$ then the ideal $(eX+(1-e))$
  divides the ideal $(X)$.}.
\end{corollary}

\newpage

\section*{Appendix 2: proof of Proposition \ref{units}}

\begin{proof}
%  (We follow essentially David's argument.)
  Each $u_{ij}$ is such that $u_{ij}(p)$ unit for $p=0$ and
  all $u_{ij}(p)$ nilpotent for $p\neq 0$.

  Like in the proof of \Cref{nilpotent}, we can change $u_{01}$ so that
  we have $u_{01}(p) = 0$ if $p\neq 0$ and $p_0\geqslant 0$ or $p_1\geqslant 0$ by multiplying $u_{01}$ by a unit in $T_0$ and
  a unit in $T_1$. Let us show for instance how to force $u_{01}(p) = 0$ if $p\neq 0$ and $p_1\geqslant 0$ by multiplying $u_{01}$
  by a unit in $T_0$. Let $M$ be the ideal generated by $u_{01}(p)$ for $p\neq 0$, which is a nilpotent ideal. If we
  multiply $u_{01}$ by $u_{01}(0) - \Sigma_{p_1\geqslant 0} u_{01}(p)$
  we change $u_{01}$ to $u'_{01}$ where all $u_{01}'(p)$, for $p_1\geqslant 0$ and $p\neq 0$, are in $M^2$. We iterate this process
  and since $M$ is nilpotent, we force $u_{01}(p) = 0$ or $p\neq 0$ and $p_1\geqslant 0$.

  We can thus assume that $u_{01}(p) = 0$ if $p\neq 0$ and $p_0\geqslant 0$ or $p_1\geqslant 0$.
  
  We claim then that, in this case, $u_{01}$ has to be a unit. For this we show that $u_{01}(p) = 0$
  if $p_l>0$ for each $l\neq 0,1$. 
  This is obtained by looking at the relation $u_{01}= u_{0l}u_{l1}$ that can be rewritten as
  $$u_{01}(p) = u_{0l}(p)u_{l1}(0) + u_{0l}(0)u_{l1}(p) + \Sigma_{q+r = p, q\neq 0, r\neq 0}u_{0l}(q)u_{l1}(r)$$
  or
  $$u_{0l}(p)u_{l1}(0) = - u_{01}(p) + u_{0l}(0)u_{l1}(p) + \Sigma_{q+r = p, q\neq 0, r\neq 0}u_{0l}(q)u_{l1}(r)\leqno{(1)}$$
  (recall that $u_{l1}(0)$ is a unit).

  Let $L$ be the ideal generated by
  coefficients $u_{0l}(p)$ and $u_{1l}(p)$ with $p_l>0$ and $I$
  the ideal generated by all nilpotent coefficients of $u_{0l}$ and $u_{l1}$.

  Thanks to the form of $u_{01}$ we will show that $L\subseteq LI$ and so $L=0$ by Nakayama.
  For this we use $(1)$ and show that $u_{0l}(p)$ is in $LI$.

  Since $p_l>0$, we have $u_{0l}(p) = 0$ if $p_0\geqslant 0$, since then there exists $i\neq 0, i\neq l$ such that $p_i>0$
  (since $\Sigma p = 0$)  and hence we can assume $p_0<0$.
  
  We also have $u_{0l}(p)$ if $p_1<0$ since $u_{0l}$ is in $T_{0l}$ and we can assume $p_1\geqslant 0$ as well.
  
  This implies $u_{l1}(p) = 0$ (since $p_0<0$ and $u_{l1}$ in $T_{l1}$)
  and $u_{01}(p) = 0$ (since $p_0<0$ and $0\leqslant p_1$).
  
  We get thus
  $$u_{0l}(p)u_{l1}(0) = - \Sigma_{q+r = p, q\neq 0, r\neq 0}u_{0l}(q)u_{l1}(r)$$
  and each member in the sum $u_{0l}(q)u_{l1}(r)$ is in $IL$ since $q_l+r_l = p_l>0$ and hence $q_l>0$ or $r_l>0$.

  We thus deduce $L=0$ by Nakayama. We get, for $p_l>0$
  $$u_{01}(p) = u_{0l}(p)u_{l1}(0) + u_{0l}(0)u_{l1}(p)$$
  and if $p_0<0$ and $p_1<0$ we have $u_{0l}(p) = u_{l1}(p) = 0$.

  This implies that all coefficients $u_{01}(p)$ such that $p_l>0$ are $0$.

  Since this holds for each $l>1$ we have that $u_{01}$ is a unit in $R$.

  W.l.o.g. we can assume $u_{01}= 1$. We then have $u_{0l} = u_{1l}$ in $T_{0l}\cap T_{1l} = T_l$
  and we take $s_l = u_{0l} = u_{1l}$.
\end{proof}

\newpage


\section*{Appendix 3: Quillen Patching}

We reproduce the argument in Quillen's paper \cite{Quillen}, as simplified in \cite{lombardi-quitte}.
This technique of Quillen Patching has been replaced by the equivalence in Proposition \ref{Matthias}.

If $P$ and $Q$ are two idempotent matrix of the same size, let us write $P\simeq Q$ for expressing that $P$ and $Q$ presents
the same projective module (which means that there are similar, which is in this case is the same as being equivalent).

If we have a projective module on $A[X]$, presented by a matrix $P(X)$, this module is extended
precisely when we have $P(X)\simeq P(0)$.

\begin{lemma}
  If $S$ is a multiplicative monoid of $A$ and $P(X)\simeq P(0)$ on $A_S[X]$ then there exists
  $s$ in $S$ such that $P(X+sY)\simeq P(X)$ in $A[X]$.
\end{lemma}

\begin{lemma}
  The set of $s$ in $A$ such that $P(X+sY)\simeq P(X)$ is an ideal of $A$.
\end{lemma}

\begin{corollary}
  If we have $M$ projective module of $A[X]$ and $S_1,\dots,S_n$ comaximal multiplicative monoids of $A$
  such that each $M\otimes_{A[X]} A_{S_i}[X]$ is extended from $A_{S_i}$ then $M$ is extended from $A$.
\end{corollary}

Let us reformulate in synthetic term this result. Let $A$ be a f.p. $R$-algebra and $L:\Spec(A)\rightarrow B\Gm^{\A^1}$.
Then $L$ corresponds to a projective module of rank $1$ on $A[X]$. We can form
$$T(x) = \prod_{r:R}L~x~r = L~x~0$$
and $\|T(x)\|$ expresses that $L~x$ defines a trivial line bundle on $\A^1 = \Spec(R[X])$.
It is extended exactly when we have
$\|{\prod_{x:\Spec(A)}T(x)}\|$. We can then use Zariski local choice to state.

\begin{proposition}\label{c2}
  We have the implication $(\prod_{x:\Spec(A)}\|T(x)\|)\rightarrow \|\prod_{x:\Spec(A)}T(x)\|$.
\end{proposition}

\newpage

\section*{Appendix 4: Classical argument}

We reproduce a message of Brian Conrad in MathOverflow \cite{conrad-mathoverflow-16324}.

\medskip

``We know that the Picard group of projective $(n-1)$-space over a field $k$ is $\Z$
generated by $\OO(1)$.
This underlies the proof that the automorphism group of such a projective space is $\PGL_n(k)$.
But what is the automorphism group of $\bP^{n-1}(A)$ for a general ring $A$? Is it $\PGL_n(A)$?
It's a really important fact that the answer is yes.
But how to prove it? It's a shame that this isn't done in Hartshorne.

By an elementary localization, we may assume $A$ is local.
In this case we claim that $\Pic(\bP^{n-1}(A))$ is infinite cyclic generated by $\OO(1)$.
Since this line bundle has the known $A$-module of global sections,
it would give the desired result if true by the same argument as in the field case.
And since we know the Picard group over the residue field, we can twist
to get to the case when the line bundle is trivial on the special fiber. How to do it?

\medskip

 Step 0: The case when $A$ is a field. Done.

 \medskip

 Step 1: The case when $A$ is Artin local.
 This goes via induction on the length, the case of length $0$ being Step $0$
 and the induction resting on cohomological results for projective space over the residue field.

  \medskip

 Step 2: The case when $A$ is complete local noetherian ring. This goes
 using Step 1 and the theorem on formal functions (formal schemes in disguise).

  \medskip

 Step 3: The case when $A$ is local noetherian.
 This is faithfully flat descent from Step 2 applied over $A~\widehat{}$

 \medskip
 
 Step 4: The case when $A$ is local:
 descent from the noetherian local case in Step 3 via direct limit arguments.

\medskip
 
QED''


\printindex

\printbibliography

\end{document}
