
In this section, let $X$ be a type, $U_1,\dots,U_n\subseteq X$ open subtypes that cover $X$
and $\mathcal F:X\to \AbGroup$ a dependent abelian group on $X$.
We start by repeating the classical definition of \v{C}ech-Cohomology groups for a given cover.

\begin{definition}%
  \label{chech-complex}
  \begin{enumerate}[(a)]
  \item \index{$\mathcal F(U)$} For open $U\subseteq X$, we use the notation
    \[
      \mathcal F(U)\colonequiv \prod_{x:U}\mathcal F_x\rlap{.}
    \]
  \item For $s:\mathcal F(U)$ and open $V\subseteq U$ we use the notation $s\colonequiv s_{|V} \colonequiv (x:V)\mapsto s_x$.
  \item \index{$U_{i_1\dots i_l}$}For a selection of indices $i_1,...,i_l:\{1,\dots,n\}$, we use the notation
    \[
      U_{i_1\dots i_l}\colonequiv U_{i_1}\cap\dots\cap U_{i_l}\rlap{.}
    \]
  \item For a list of indices $i_1,\dots,i_l$, let $i_1,\dots,\hat{i_t},\dots,i_l$ be the same list with the $t$-th element removed.
  \item For $k:\Z$, the $k$-th \notion{Čech-boundary operator}\index{$\partial^k$} is the homomorphism
    \[
      \partial^k:\bigoplus_{i_0,\dots,i_k}\mathcal F(U_{i_0\dots i_k})\to \bigoplus_{i_0,\dots,i_{k+1}}\mathcal F(U_{i_0\dots i_{k+1}})
    \]
    given by $\partial^k(s)\colonequiv (l_0,\dots,l_{k+1}) \mapsto \sum_{j=0}^k (-1)^j s_{l_0,\dots,\hat{l_j},\dots,l_k|U_{l_0,\dots,l_{k+1}}}$.
  \item The $k$-th \notion{Čech-Cohomology group} for the cover $U_1,\dots,U_n$ with coefficients in $\mathcal F$ is
    \[
      \check{H}^k(\{U\},\mathcal F)\colonequiv \ker\partial^{k} / \im(\partial^{k-1})\rlap{.}
    \]
  \end{enumerate}
\end{definition}

\begin{definition}
  \label{sheaf-cech}
  Let $\{U\}_{i:I}$ be a finite collection of open subtypes of $X$ and $\mathcal F : X\to \AbGroup$.
  Let $I_x^k\colonequiv (i_0, \dots, i_k:I)\times U_{i_0\dots i_k}(x)$ for $k:\N$ and $i^k:(x:X)\times I^k_x\to X$ be the first projection.
  Then the dependent abelian group\index{$\check{\mathcal F}$}
  \[ \check{\mathcal F}^k\colonequiv (x:X)\mapsto \mathcal F_x^{I_x^k} \equiv i^k_\ast {i^k}^\ast\mathcal F \]
  is called the $k$-th \notion{\v{C}ech-sheaf} of $\mathcal F$.
\end{definition}

\begin{remark}
  \label{cech-sheaf-pi-direct-sum}
  \begin{enumerate}[(a)]
  \item The functor $\prod:\mathcal A \to \AbGroup$ is additive.
  \item Let $X$ be a type covered by $\{U\}_{i:I}$ and $\mathcal F : X\to \AbGroup$.
    Then 
    \[
      \prod_{x:X} \check{\mathcal F}^k_x=\bigoplus_{i_0,\dots,i_k}\mathcal F(U_{i_0\dots i_k})
      \rlap{.}
    \]
  \end{enumerate}
\end{remark}

\begin{proof}
  \begin{enumerate}[(a)]
  \item The finite biproducts in $\mathcal A$ are in particular finite products, which commute with $\prod$.
  \item \begin{align*}
          \prod_{x:X} \check{\mathcal F}^k_x &= \prod_{x:X} \mathcal F_x^{I^k_x} \\
                                             &= \prod_{x:X} ((i_0, \dots, i_k:I)\times U_{i_0\dots i_k}(x))\to \mathcal F_x \\
                                             &= \prod_{x:X} \prod_{i_0, \dots, i_k:I} \mathcal F_x^{U_{i_0\dots i_k}(x)} \\
                                             &= \prod_{i_0, \dots, i_k:I} \prod_{x:X} \mathcal F_x^{U_{i_0\dots i_k}(x)} \\
                                             &= \bigoplus_{i_0,\dots,i_k}\mathcal F(U_{i_0\dots i_k})\rlap{.}
        \end{align*}
  \end{enumerate}
\end{proof}

\begin{definition}
  \begin{enumerate}[(a)]
  \item A cover $\{U\}=U_1,\dots,U_n$ is called \notion{acyclic} for $\mathcal F$
    if for all $k:\N$ and $i_0,\dots,i_k$, we have that the higher (non Čech) cohomology groups are trivial:
    \[
      \forall l>0. H^l(U_{i_0,\dots,i_k},\mathcal F)=0\rlap{.}
    \]
  \item A cover $\{U\}=U_1,\dots,U_n$ is called \notion{\v{C}ech-trivializing} \rednote{Better names welcome!} for $\mathcal F$
    if for all $k\geq 0$ and all indices $i_o,\dots,i_k:I$ we have $H^1(U_{i_0\dots i_k},\mathcal F)=0$ and $H^1(U_{i_0\dots i_k}(x),\mathcal F_x)=0$ for all $x:X$.
  \end{enumerate}
\end{definition}

\begin{theorem}%
  \label{cech-les}
  Let $X$ be covered by a finite $\{U\}$ and let 
  \[ 0\to \mathcal F\to \mathcal G\to \mathcal H\to 0\]
  be a short exact sequence of depedent abelian groups on $X$.
  If $\{U\}$ is \v{C}ech-trivializing for $\mathcal F$,
  then a long exact sequence of \v{C}ech-Cohomology groups is induced:
  \begin{center}
    \begin{tikzcd}
      &\dots\ar[r] & \check{H}^{k-1}(\{U\},\mathcal H)\ar[dll] \\
      \check{H}^k(\{U\},\mathcal F)\ar[r] & \check{H}^k(\{U\},\mathcal G)\ar[r] & \check{H}^k(\{U\},\mathcal H)\ar[dll] \\
      \check{H}^{k+1}(\{U\},\mathcal F)\ar[r] & \dots &
    \end{tikzcd}
  \end{center}
\end{theorem}

\begin{proof}
  The cover is \v{C}ech-trivializing for $\mathcal F$,
  so $H^1(I_x^k,\mathcal F)=\bigoplus_{i_0,\dots,i_k} H^1(U_{i_0\dots i_k}(x),\mathcal F)=0$.
  Using the long exact sequence for Eilenberg-MacLane Cohomology \Cref{EM-les},
  this means that for all $x:X$, the sequence
  \[
    0\to {\mathcal F}^{I^k_x}\to {\mathcal G}^{I^k_x}\to {\mathcal H}^{I^k_x}\to 0
  \]
  is exact, which implies exactness of all sequences:
  \[
    0\to \check{\mathcal F^k}\to \check{\mathcal G}^k\to \check{\mathcal H}^k\to 0
    \rlap{.}
  \]
  The cover is \v{C}ech-trivializing for $\mathcal F$, so using \Cref{cohomologically-trivial-fibers} and \Cref{cohomology-copropduct-direct-sum} we have
  \begin{align*}
    H^1(X,\check{\mathcal F^k})&=H^1(X,i_\ast^k {i^k}^\ast\mathcal F) \\
                               &=H^1((x:X)\times (i_0,\dots,i_k)\times U_{i_0\dots i_k}(x),{i^k}^\ast\mathcal F) \\
                               &=\bigoplus_{i_0\dots i_k} H^1(U_{i_0\dots i_k},\mathcal F) \\
                               &=0    
  \end{align*}
  This implies, by the long exact sequence for non-\v{C}ech cohomology, that applying $\prod$ preserves the exactness.
  So by \Cref{cech-sheaf-pi-direct-sum}, we have short exact sequences:
  \[
  0\to \bigoplus_{i_0,\dots,i_k}\mathcal F(U_{i_0\dots i_k}) \to \bigoplus_{i_0,\dots,i_k}\mathcal G(U_{i_0\dots i_k}) \to \bigoplus_{i_0,\dots,i_k}\mathcal H(U_{i_0\dots i_k})\to 0
  \]
  which assemble to a short exact sequence of chain complexes.
  By homological algebra \cite[\href{https://stacks.math.columbia.edu/tag/0117}{Tag 0117}]{stacks-project}, this induces the desired long exact sequence of the cohomology groups of these complexes.
\end{proof}

A very specific consequence we will need for the proof that \v{C}ech cohomology is a universal $\partial$-functor:

\begin{corollary}%
  \label{chech-coefficient-sum}
  Let $X$ have a cover $\{U\}$ with the same properties as in the theorem with respect to all $\mathcal F_1,\dots,\mathcal F_n$.
  Then, all higher \v{C}ech-Cohomology groups of $\bigoplus_{i}\mathcal F_i$ vanish,
  if they vanish for all the $\mathcal F_i$.
\end{corollary}

Some properties of cohomology are a lot easier to check for \v{C}ech Cohomology than for Eilenberg-MacLane Cohomology,
like the following commutativity with direct sums.

\begin{lemma}
  \label{cech-commute-direct-sum}
  Let $I$ be a decidable set, $\mathcal F_i:X\to \AbGroup$ for each $i$ and $\{U\}$ a cover of $X$.
  The canonical map induces an equality for all $k:\N$:
  \[
    \check{H}^k(\{U\},\bigoplus_{i:I} \mathcal F_i)=\bigoplus_{i:I}\check{H}^k(\{U\},\mathcal F_i)
  \]
\end{lemma}

\begin{proof}
  This follows from commutativity of direct sums with kernels, images and quotients.
\end{proof}

There is the slight variant of \v{C}ech-Cohomology, where only ordered tuples of indices appear, which is sometimes helpful for computations.
The resulting cohomology groups are the same, which can be shown on the level of \emph{abelian groups indexed over finite sets of indices},
following \cite{soergel-sheaf-cohomology}[1.4].
The notation introduced below will not be used in the rest of the article.

\begin{definition}
  Let $E$ be a type, $\mathrm{Fin}(E)\colonequiv \{A\subseteq E \mid \exists \phi:\mathrm{Fin}(n)\to E.\mathrm{im}(\phi)=A \}$
  and $M$ be a functor from $\mathrm{Fin}(E)$ with ``$\subseteq$'' to abelian groups. We write $M(e_0,\dots,e_q)\colonequiv M(\{e_0,\dots,e_q\})$.
  For $q\in \Z$, let
  \[
    C^qM\colonequiv \prod_{e_0,\dots,e_q:E^{q+1}}M(e_0,\dots,e_q)
  \]
  be the \emph{unordered \v{C}ech-complex} for $M$, with a differential defined as above.
  Let $C^q_{\mathrm{alt}}M$ be the subcomplex of \emph{alternating} cochains, i.e.\ elements $s:C^qM$ such that
  for all $0\leq i<q$, we have $s(e_0,\dots,e_i,e_{i+1},\dots,e_q)=-s(e_0,\dots,e_{i+1},e_i,\dots,e_q)$ and $s(e_0,\dots,e_i,e_{i+1},\dots,e_q)=0$
  whenever $e_i=e_{i+1}$.

  For $q\in \Z$, let
  \[
    C^q_<M\colonequiv \prod_{e_0<\dots <e_q:E^{q+1}}M(e_0,\dots,e_q)
  \]
  be the \emph{ordered \v{C}ech-complex} for $M$, with a differential defined as above and
  $C^q_{\times}M$ be the kernel of the restriction map $C^qM\to C^q_<M$.
\end{definition}

\begin{lemma}
  \label{ordered-cech-equivalence}
  Let $M$ be as in the definition above, for the sequence:
  \begin{center}
    \begin{tikzcd}
      C^\bullet_{\mathrm{alt}} M\ar[r,"f"] & C^\bullet M\ar[r,"g"] & C^\bullet_<M
    \end{tikzcd}
  \end{center}
  \begin{enumerate}[(i)]
  \item The three complexes are isomorphic in non-positive degrees.
  \item\label{iso-complexes} $g\circ f$ is an isomorphims of complexes.
  \item\label{skyscraper-group-exact} If there is $e:E$ such that $M(A)\to M(A\cup \{e\})$ is an isomorphism for all $A:\mathrm{Fin}(E)$,
    then the three complexes are exact.
  \item\label{finite-ordered-cech} If $E$ is finite, then $f$ and $g$ induce isomorhisms on cohomology.
  \end{enumerate}
\end{lemma}

\begin{proof}
  \begin{enumerate}[(i)]
  \item Clear.
  \item Clear.
  \item We start by showing that $C^\bullet M$ is exact.
    Let $q\geq -1$, for $s:C^q M$ we define $\delta:C^{q+1}M\to C^q M$ by
    \[
      (\delta s)(e_0,\dots,e_q)\colonequiv M(A\subseteq A\cup\{e\})^{-1} s(e,e_0,\dots,e_q)
    \]
    then one can compute $\delta d+d\delta=\mathrm{id}$, so $C^\bullet M$ is exact.
    Since $\delta$ preserves alternating cochains, this also proves that $C_{\mathrm{alt}}^\bullet M$ is exact.
    By \ref{iso-complexes}, $C_{\mathrm{alt}}^\bullet M\cong C_<^\bullet M$, so $C_<^\bullet M$ is exact as well.
  \item It is enough to show that $C_\times^\bullet M$ is exact.
    For $N:\AbGroup$ and $F:\mathrm{Fin}(E)$, a functor $\mathrm{Fin}(E)\to \AbGroup$ is given by $N_F(B)\colonequiv N^{B\subseteq F}$.
    By case distinction on $F=\emptyset$ and using \ref{skyscraper-group-exact}, we see that $C_\times^\bullet N_F$ is exact.
    We have an exact sequence of functors $\mathrm{Fin}(E)$:
    \[
      M\to \prod_{A:\mathrm{Fin}(E)}(M(A))_A\to \mathrm{cok}
    \]
    By applying $C_\times^\bullet$ to this sequence and commuting $\prod$ with $C_\times^\bullet$,
    we get a sequence
    \[
      C_\times^\bullet M\to \prod_{A:\mathrm{Fin}(E)}C_\times^\bullet (M(A))_A\to C_\times^\bullet \mathrm{cok}
    \]
    which is still exact by finiteness of $\mathrm{Fin}(E)$.
    By applying the long exact cohomology sequence, we get $H^n(C_\times^\bullet \mathrm{cok})\cong H^{n+1}(C_\times^\bullet M)$.
    If we also apply this for $M=\mathrm{cok}$, this is enough to conclude.
  \end{enumerate}
\end{proof}

\Cref{ordered-cech-equivalence}\ref{finite-ordered-cech} could be strengthened for non-finite index sets $E$
such that the first cohomology on $\mathrm{Fin}(E)$ with some coefficients which could be extracted from the proof vanishes.
For clarity of the statement and for a lack of application, this generalization was omitted.