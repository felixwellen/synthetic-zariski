\subsection{First (non-abelian) cohomology vanish for Stone spaces}
In this section, we generalize the cohomological results in Section 5 of \cite{synthetic-stone-duality} from Abelian groups to general groups.

\begin{remark}
  Let $G$ be a group,
  we denote $K(G,n)$ for the $n$'th delooping of $G$.
  To be precise, $K(G,0) = G$ and 
  $K(G,n+1)$ is a pointed type with only one point
  with loop space $K(G,n)$, 
  and such that \cite{davidw23}.
  To be more precise, we have $*:K(G,n+1)$, 
  $\prod_{x : K(G,n+1)} \propTrunc{x = *}$ and the type
  $*=*$ is cannonically equal to $G$. 
  The first delooping of $G$ is denoted $BG$. 
  \\
  For $X$ a type and $G: X \to Group$ 
  we define $H^n(X,G) = ||\prod_{x : X} K(G(x),n)||_0$.
\end{remark} 

\begin{definition}
Assume given a type $X$ with $Y(x)$ a type and $G(x)$ a group depending on $x:X$. We define its its short \v{C}ech sequence $\check{C}(X,Y,G)$ as:
\[\prod_{x:X}G(x)^{Y(x)} \overset{\delta_0}{\longrightarrow} \prod_{x:X}G(x)^{Y(x)^2} \overset{\delta_1}{\longrightarrow} \prod_{x:X}G(x)^{Y(x)^3} \]
with:
\[\delta_0(\alpha)_x(y_1,y_2) = \alpha_x(y_1)^{-1}\cdot\alpha_x(y_2)\]
\[\delta_0(\beta)_x(y_1,y_2,y_3) = \beta_x(y_2,y_3)\cdot\beta_x(y_1,y_3)^{-1}\cdot\beta_x(y_1,y_2)\]
We have that $\delta_1(\delta_0(\alpha)))_x(y_1,y_2,y_3) = 1$.
\end{definition}

First we give a couple of auxiliary results on \v{C}ech short sequence, justifying there usefulness.

\begin{lemma}\label{cover-with-section-exact-cech}
Assume given a type $X$ with $Y(x)$ a type and $G(x)$ a group depending on $x:X$. If we merely have a section $t:\prod_{x:X}Y(x)$, then $\check{C}(X,Y,G)$ is exact.
\end{lemma}

\begin{proof}
  Lemma 5.5 of \cite{synthetic-stone-duality} goes trough 

  \rednote{Check:}
  Assume given a cocycle $\beta: \prod_{x : X} A_x^{T_x^2}$. 
  Define $\alpha :\prod_{x : X} A_x ^{T_x}$ by 
  $\alpha_x(u) = \beta_x(t_x,u)$. Then 
  $$
    \delta_0(\alpha)_x(u,v) = 
    \alpha_x(u)^{-1}\cdot\alpha_x(v) =
    \beta_x(t_x,u)^{-1} \cdot \beta(t_x,v)
  $$
  As $\beta$ is a cocycle, we have that 
  $$\beta_x(u,v) \cdot \beta_x(t_x,v)^{-1} \cdot \beta_x(t_x,u) = 1,$$
  hence
  $$\beta_x(t_x,v)^{-1} \cdot \beta_x(t_x,u) = \beta_x(u,v)^{-1},$$
  and taking inverses then gives that 
  $$\beta_x(t_x,u)^{-1} \cdot \beta_x(t_x,v) = \beta_x(u,v),$$
  from which we conclude that 
  $\delta_0(\alpha)_x(u,v) = \beta_x(u,v)$, so $\beta$ is a coboundary. 
  Hence the sequence is exact. 
\end{proof}

\begin{lemma}\label{cocyle-being-coboundary}
Assume given a set $X$ with $Y(x)$ an inhabited set depending on $x:X$. Moreover, assume given $G(x)$ a group depending on $x:X$ with:
\[\phi:\prod_{x:X}BG(x)\]
and:
\[\psi:\prod_{x:X}(\phi(x)=*)^{Y(x)}\]
If $\check{C}(X,Y,G)$ is exact then $\phi=*$.
\end{lemma}

\begin{proof}
  For $y_1, y_2 : T(x)$, we have 
  $\psi_x(y_1) ^{-1} \cdot \psi_x(y_2) : * = *$, 
  so we can define 
\[\beta : \prod_{x:X} G(x)^{T(x)^2}\]
by having $\beta_x(y_1,y_2) = \psi_x(y_1)^{-1} \cdot \psi_x(y_2)$. It is easy to check that it is a cocycle in the given sequence (meaning $\delta_1(\beta) = 1$), so that by exactness there is $\alpha:\prod_{x:X}G(x)^{Y(x)}$ such that for all $x:X$ and $y_1,y_2:Y(x)$ we have that:
\[\beta_x(y_1,y_2)=\alpha_x(y_1)^{-1}\cdot \alpha_x(y_2)\]
Then we define:
\[\psi' : \prod_{x:X}(\phi(x)=*)^{Y(x)}\]
\[\psi'_x(y) = \psi_x(y)\cdot \alpha_x(y)^{-1}\]
so that for all $x:X$ and $y_1,y_2:Y(x)$ we have that:
\begin{eqnarray}
\psi'_x(y_1)^{-1}\cdot\psi'_x(y_2) &=& \alpha_x(y_1) \cdot \psi_x(y_1)^{-1}\cdot \psi_x(y_2)\cdot \alpha_x(y_2)^{-1}\nonumber\\
 &=&  \alpha_x(y_1) \cdot \beta_x(y_1,y_2)\cdot \alpha_x(y_2)^{-1} \nonumber\\
 &=&  \alpha_x(y_1) \cdot\alpha_x(y_1)^{-1} \cdot\alpha_x(y_2)\cdot \alpha_x(y_2)^{-1} \nonumber\\ 
 &=& 1 \nonumber
 \end{eqnarray}
This means that $\psi'_x(y_1) = \psi'(y_2)$, so that $\psi'$ factors through $X$, giving a proof of $\phi=*$.
\end{proof}

Now our key idea is to use finite approximation of Stone space.

\begin{lemma}\label{finite-approximation-stone-surjection-cohomology}
Assume given a Stone space and $T:S\to \Stone$ such that $\prod_{x:S}\propTrunc{T(x)}$. Then there exists a sequence $T_k: S\to \Stone$ with $\prod_{x:S}T_{k+1}(x)\to T_k(x)$ for $k:\N$ such that:
\begin{itemize} 
\item For all $x:S$ we have $\mathrm{lim}_kT_k(x) = T(x)$.
\item For all $k:\N$ we have a section in $\prod_{x:S}T_k(x)$.
%such that $T_k$ are Stone.%, $p_k$
\end{itemize}
\end{lemma}

%\begin{proof}
%From \cite{foundation}, we get a tower of finite types $(S_k)_{k:\N}$ with a dependent tower of merely inhabited finite types $(Q_k:S_k\to \Type)_{k:\N}$ over it such that $\mathrm{lim}_kQ_k = S$ and given $x:S$ we have that $\lim_kQ_k(x_k) = T(x)$. Since $S_k$ is finite we merely have sections in $\prod_{x:S_k}Q_k(x)$ and then defining $T_k : S\to \Stone$ as $T_k(x) = Q_k(x_k)$ works.
%\end{proof}
\begin{proof}
  By assumption, the projection 
  $\pi: \sum_{x:S} T(x) \to S$ is surjective.
  By Theorem 4.18 of \cite{synthetic-stone-duality}, the domain is Stone.
  Hence by Remark 3.4 of \cite{synthetic-stone-duality}, 
  we can write $\pi$ as limit of surjections 
  $\pi_k : Q_k\twoheadrightarrow S_k$
  between finite sets.
  As taking fibers and limits commute, we have that 
  $T(x)$ is the limit of $\pi_k^{-1}(x|_k)=:T_k(x)$, 
  which is a closed subset of a finite set, hence Stone by 
  Theorem 3.11 of \cite{synthetic-stone-duality}. 
  Furthermore, surjections of finite sets have sections
  $s_k: S_k \hookrightarrow Q_k$, 
  giving the required section 
  $\lambda x . s_k(x|_k) : \prod_{x :S} T_k(x)$. 
\end{proof}

\begin{lemma}\label{short-cech-exact}
Assume given $S:\Stone$ and $T:S\to \Stone$ such that $\prod_{x:S}\propTrunc{T(x)}$ with $G:S\to \mathrm{Grp}_\ODisc$. Then $\check{C}(S,T,G)$ is exact.
\end{lemma}

\begin{proof}
We use \Cref{finite-approximation-stone-surjection-cohomology} to get a sequence of $T_k:S\to \Stone$ having sections such that $\mathrm{lim}_kT_k(x) = T(x)$ for all $x:S$, since they have section by \Cref{cover-with-section-exact-cech} we have that $\check{S,T_k,G}$ is exact for all $k:\N$, we just need to prove $\mathrm{colim}_k\check{C}(S,T,G) = \check{C}(S,T,G)$ to conclude, as a sequential colimit of exact sequence is exact. But we have that:
\begin{eqnarray}
\mathrm{colim}_k \prod_{x:S}G(x)^{T_k(x)^l} &=& \prod_{x:S}\mathrm{colim}_k G(x)^{T_k(x)^l}\nonumber\\
&=& \prod_{x:S}G(x)^{\mathrm{lim}_kT_k(x)^l}\nonumber\\
&=& \prod_{x:S}G(x)^{T(x)^l}\nonumber
\end{eqnarray}
where the first equality comes from \Cref{scott-continuity-right} and the second from \Cref{scott-continuity-left}.
\end{proof}

\begin{lemma}\label{local-choice-delooping-trick}
  Let $S$ be Stone and $G: S\to \mathrm{Grp}_\ODisc$. 
  Then there merely exists some family $(T_x)_{x:S}$ of 
  merely inhabited Stone spaces such that for all 
  $\phi:\prod_{s:S}BG(x)$, we have 
  $\prod_{x:S}(\phi_x = *)^{T_x}$.
\end{lemma}
\begin{proof}
  Let $S$ be Stone and $G: S\to \mathrm{Grp}_\ODisc$. 
  Then for any $\phi:\prod_{s:S}BG(x)$, 
  we have that:
  $\prod_{x:S}\propTrunc{\phi_x=*}$. 
  Thus by local choice, 
  there exists some $T:\Stone$ with surjection $q : T \twoheadrightarrow S$, 
  such that $\prod_{t : T} \phi_{qt} = *$. 
  Writing $T_x$ for the fiber of $q$ over $x:S$, 
  we have that $\prod_{x : S} (T_x \to \phi_x = *)$.
  By surjectivity each ${T_x}$ is merely inhabited. 
\end{proof} 

Now we just have to assemble the pieces.

\begin{lemma}\label{vanishing-1-cohomology-stone}
Let $S$ be Stone and $G: S\to \mathrm{Grp}_\ODisc$. Then:
\[H^1(S,G) = 0\]
\end{lemma}

\begin{proof}
  Assume $\phi:\prod_{s:S}BG(x)$,
  by \Cref{local-choice-delooping-trick}, 
  there merely exists some family $(T_x)_{x:S}$ of inhabited Stone spaces with 
  $\prod_{x:S}||T_x||$ and  
  $\alpha:\prod_{x:S}(\phi(x)=*)^{T(x)}.$
In order to conclude by applying \Cref{cocyle-being-coboundary}, it is enough to prove that the \v{C}ech sequence $\check{S,T,G}$ is exact. But this is \Cref{short-cech-exact}.
\end{proof}

\subsection{Stone spaces are acyclic}
The goal of this section is to show that for $S:\Stone$, we have 
$H^n(S,A) = 0$ for any $A:S\to \mathrm{Ab}_\ODisc, n>0$. 
We will actually show something stronger, 
namely that delooping commutes with $\Stone$-indexed $\prod$-types.
This acyclity will be used as an intermediate step. 
To be more specific, we will show in 
\Cref{eilenberg-exponentials} that the following map is an equivalence for all $S,A,n$:
\begin{definition}
  Let $S:\Stone,A:S\to \mathrm{Ab}_\ODisc,n : \N$, we denote $ev_n^{S,A}$ for the canonical map:
  \[ev_n^{S,A} : K(\prod_{x:S}A_x,n)\to \prod_{x:S}K(A_x,n)\]
\end{definition}
\begin{lemma}\label{ev-eq-acyclic}
  For any $S,A$ and $n\geq 1$, if $ev_n^{S,A}$ is an equivalence, then $H^n(S,A) = 0$. 
\end{lemma}
\begin{proof}
  By definition
  $H^n(S,A) = || \prod_{x : S} K(A_x,n) ||_0$. 
  If $ev_n^{S,A}$ is an equivalence, it follows that the latter equals
  $|| K(\prod_{x : S} A_x , n) ||_0$, 
  which is $0$ if $n\geq 1$. 
\end{proof}
\begin{lemma}\label{ev-embedding}
  For any $S,A,n$,
  if $ev_n^{S,A}$ is an equivalence, then 
  $ev_{n+1}^{S,A}$ is an embedding. 
\end{lemma}
\begin{proof}
  Note that the source is pointed and connected. 
  So it is enough to prove the induced map:
  \[\Omega(K(\prod_{x:S}A_x,n+1)) \to \Omega(\prod_{x:S}K(A_x,n+1))\]
  is an equivalence, but this map is equal to the canonical map:
  \[ev_n^{S,A} : K(\prod_{x:S}A_x,n) \to \prod_{x:S}K(A_x,n),\]
  which was an equivalence by the assumption. 
\end{proof} 
\begin{lemma}\label{acyclic-ev-surjective}
  For $n\geq 1$, if $H^n(S,A) = 0$, then $ev_{n}^{S,A}$ is a surjection. 
\end{lemma}
\begin{proof}
  Note that for $n\geq 1$, 
  $K(\prod_{x:S}A_x,n)$ is $0$-connected. 
  Therefore, it's sufficient to show that 
  $\prod_{x : S} K(A_x,n)$ is also $0$-connected, 
  which means exactly that $H^{n}(S,A) = || \prod_{x : S} K(A_x,n) ||_0 = 0$. 
\end{proof} 


\rednote{The following should be standard, we need a reference}
\begin{lemma}\label{ExactSequenceDelooping}
  Let $A \to B \to C\to 0$ be an exact sequence. 
  Then we get an exact sequence 
  $$
  \cdots \to K(C, k-1) \to K(A,k) \to K(B,k) \to K(C,k) \to K(A,k+1) \to \cdots 
  $$
\end{lemma}
\begin{proof}
Note that $B \to C$ induces a map 
$K(B,n) \to K(C,n)$. 
Following definition 8.4.3 of \cite{hott}, we get the 
following long exact sequence in $k$:
  $$
  \cdots \to \pi_{k+1}(C) \to 
  \pi_{k}(F) \to \pi_{k}(K(B,n)) \to \pi_{k}(K(C,n)) \to 
  \pi_{k-1}(F) \to \cdots ,
  $$
where $F$ is the fiber of $K(B,k) \to K(C,n)$. 
Taking $k =n-1$, we have that 
$$\pi_n(F) \to \pi_{n}K(B,n) \to \pi_nK(C,n) \to \pi_{n-1} (F) \to \pi_{n-1}K(B,n)$$
is exact. 
Now we use that for any group $G$, we have $\pi_{n-1}(K(G,n)) = 0, \pi_n(K(G,n)) = G$. 
Thus $\pi_n(F) \to B \to C \to \pi_{n-1}(F) \to 0$ is exact. 
As $B \to C \to 0$ was exact, $B \to C$ is surjective, 
so the image of $C \to \pi_{n-1}(F)$ is the same as that of 
$B \to \pi_{n-1}(F)$, and that image is $0$. 
But since the kernel of $\pi_{n-1}(F) \to 0$ is all of $\pi_{n-1}(F)$, 
we must have $\pi_{n-1}(F) = 0$.
Thus $F$ is $n-1$-connected. 
Also $F$ is the fiber of a map of $n$-types, which is an $n$-type. 
So $F$ is an $n-1$-connected $n$-type.
Thus $\pi_n(F)$ is a set and equals $A$. 
Therefore $F = K(A,n)$. 
Furthermore, the fiber of $K(A,n) \to K(B,n)$ should be $\Omega K(C,n) = K(C,n-1)$. 
We conclude that we have an exact sequence of the form
$$\cdots\to 
  K(A,k) \to K(B, k) \to K(C,k)
$$
but then also our first exact sequence also tells us that 
there is a fiber sequence 
$$\cdots\to 
  K(C,k-1) \to 
  K(A,k) \to K(B, k) \to K(C,k) \to
  K(A,k+1) \to \cdots
$$
\end{proof}

\begin{theorem}\label{eilenberg-exponentials}
Assume given $S:\Stone$ and $A:S\to \mathrm{Ab}_\ODisc$, then for all $n:\N$ we have that the canonical map:
\[ev_n^{S,A} : K(\prod_{x:S}A_x,n)\to \prod_{x:S}K(A_x,n)\]
is an equivalence.
\end{theorem}

\begin{proof}
  We will use induction on $n$ to show that for all 
  $S:\Stone,
  A: S \to \mathrm{Ab}_\ODisc$, 
  the map $ev_n^{S,A}$ is an equivalence
  If $n=0$, this is the identity map. 
  Now assume for some $n:\N$ that for all $S,A$ we have
  $ev_n^{S,A}$ is an equivalence. 
  We will show that for all $S,A$ we also have $ev_{n+1}^{S,A}$ is an equivalence. 
  By the induction hypothesis and \Cref{ev-embedding}, 
  $ev_{n+1}^{S,A}$ is an embedding. 
  By Theorem 4.6.3 of \cite{hott}, 
  we need only to show that $ev_{n+1}^{S,A}$ is surjective.
  By \Cref{acyclic-ev-surjective}, it's sufficient to show that $H^{n+1}(S,A) = 0$. 
  For $n=0$ this is the abelian version of \Cref{vanishing-1-cohomology-stone}. 
  We may thus assume $n\geq 1$ for the rest of this proof. 

  By \Cref{local-choice-delooping-trick}, 
  there merely exists some family $(T_x)_{x:S}$ of inhabited Stone spaces such that 
  for any $\alpha : \prod_{s : S} K(A_s , n+1)$, we have 
  $\prod_{x:S} (\alpha_x = *)^{T_x}$. 
For $x:S$, denote $L_x$ for $A_x^{T_x}/A_x$, 
where the quotient is over the image of the constant map.
As $A_x^{T_x} \to L_x$ is surjective, we get an exact sequence 
$A_x \to A_x^{T_x} \to L_x \to 0$. 
By \Cref{ExactSequenceDelooping}, we get an exact sequence
$$\cdots\to 
  K(L_x,k-1) \to 
  K(A_x,k) \to K(A_x^{T_x}, k) \to K(L_x,k) \to
  K(A_x,k+1) \to \cdots
$$
Now as $\prod$ preserves exact sequences,
we also get an exact sequence 
$$\cdots\to 
\prod_{x:S}K(L_x,k-1) \to 
 \prod_{x:S} K(A_x,k) \to \prod_{x:S}K(A_x^{T_x}, k) \to \prod_{x:S}K(L_x,k) \to
 \prod_{x:S} K(A_x,k+1) \to \cdots
$$
  Consider the decomposition of 
  $\prod_{x:S}K(A_x,n+1) \to \prod_{x:S} K(A_x,n+1)^{T_x}$ given by:
  $$
    {\prod_{x:S} K(A_x,n+1)} \to \prod_{x:S}  {K(A_x^{T_x},n+1)} \hookrightarrow \prod_{x:S}  {K(A_x,n+1)^{T_x}}
  $$
  where the second map is composition with 
  $ev_{n+1}^{T_x,\lambda t . A_x}$, 
  which is an embedding by the induction hypothesis and \Cref{ev-embedding}. 
  By assumption on $T$, the composite has $0$ as image, 
  and thus the first also has $0$ as image. 
  As this image equals the kernel of $\prod_{x:S} K(A_x^{T_x},k) \to \prod_{x:S}K(L_x,k)$ 
  by exactness, it follows that that the set truncation of the latter map is injective. 

  As set truncation also preserves exact sequences (Theorem 8.4.6 of \cite{hott}), 
  we also get the following exact sequence:
  $$
  \cdots\to H^{k-1}(S,L) \to 
  H^{k}(S,A)\to H^k(S,\lambda x . A_x^{T_x}) \to H^k(S,L)\to
  H^{k+1}(S,A) \to \cdots
  $$
  By induction hypothesis and \Cref{ev-eq-acyclic}, $H^n(S,L)=0$, and 
  we have the following exact subsequence 
  $$ 0 \to H^{n+1}(S,A) \to H^{n+1}(S, \lambda x . A_x^{T_x}) 
  \hookrightarrow H^{n+1}(S,L).$$
  Now as the last map is injective, it's kernel is $0$, 
  so the image of $H^{n+1}(S,A) \to H^{n+1}(S,\lambda x . A_x^{T_x})$ is $0$, 
  which means the kernel of the same map is the whole set, 
  but that should also be the image of $0 \to H^{n+1}(S,A)$, which has only $*$ in the image. 
  Thus $H^{n+1}(S,A) = 0$. 
\end{proof}

\begin{corollary}\label{vanishing-cohomology-stone}
Let $S$ be Stone and $A:S\to\mathrm{Ab}_\ODisc$. Then for all $n>0$ we have that:
\[H^n(S,A) = 0\]
\end{corollary}


\subsection{\v{C}ech cohomology}

\begin{definition}
A \v{C}ech cover for a type $X$ consists of a surjective map:
\[f:S\to X\]
where $S$ is Stone and for all $x:X$ the fiber $S_x$ of $f$ over $x$ is Stone.
\end{definition}

Next lemma show that cohomology interact well with \v{C}ech cover. It will be used later to prove that cohomology and \v{C}ech cohomology agree.

\begin{lemma}\label{inductive-definition-cohomology}
Assume given $X$ a type with a \v{C}ech cover:
\[f:S\to X\]
as well as $A:X\to \mathrm{Ab}_\ODisc$.

For all $n\geq 1$ we have an exact sequence:
\[H^{n-1}(X,x\mapsto A_x^{S_x}) \to H^{n-1}(X,L)\to H^n(X,A)\to 0\]
natural in $A$, where $L_x = A_x^{S_x}/A_x$.
\end{lemma}

\begin{proof}
Using the long exact sequence associated to:
\[0\to A_x\to A_x^{S_x}\to L_x\to 0\]
by \Cref{vanishing-cohomology-stone} it is enough to prove that for all $n\geq 1$ we have:
\[H^n(x:X,A_x^{S_x}) = 0\]
But by \Cref{eilenberg-exponentials} we have that:
\[\prod_{x:X}K(A_x^{S_x},n) = \prod_{x:S}K(A_x,n) = K(\prod_{x:S}A_x,n)\]
\end{proof}

\begin{definition}\label{cech-sequence-definition}
Assume given a \v{C}ech cover
\[f:S\to X\]
and $A:X\to\mathrm{Ab}_\ODisc$.

Then we define the \v{C}ech complex by:
\[\prod_{x:X}A_x^{S_x} \to \prod_{x:X}A_x^{S_x\times S_x} \to \cdots\]
with the boundary maps defined as expected, that is:
\[\delta(\alpha)(x,u_0,\cdots,u_n) = \sum_{i=0}^n (-1)^i\alpha(x,u_0,\cdots,\hat{u_i},\cdots,u_n)\]
Then the \v{C}ech cohomology:
\[\check{H}^k(x:X,A_x)\]
is defined as the $k$-th homology group of the \v{C}ech complex.
\end{definition}

\begin{lemma}\label{long-exact-cech-cohomology}
Assume given a \v{C}ech cover:
\[f:S\to X\]
If we are given a short exact sequence of overtly discrete abelian group:
\[0\to A_x\to B_x\to C_x\to 0\]
depending on $x:X$, there is a long exact sequence of \v{C}ech cohomology groups:
\[\check{H}^0(X,A) \to\check{H}^0(X,B) \to\check{H}^0(X,C) \to\check{H}^1(X,A) \to\check{H}^1(X,B) \to\check{H}^1(X,C) \to\cdots \]
Moreover this long exact sequence is natural in the short exact sequence.
\end{lemma}

\begin{proof}
We just use the fact that all elements $\sum_{x:X} T_x^{k+1}$ in the \v{C}ech complex are Stone spaces, so a short exact sequence of overtly discrete abelian group induces a short exact of \v{C}ech complexes by \Cref{vanishing-cohomology-stone}.
\end{proof}

\begin{lemma}\label{inductive-definition-cech-cohomology}
Assume given a \v{C}ech cover:
\[f:S\to X\]
and $A:X\to\mathrm{Ab}_\ODisc$.

For all $n\geq 1$ we have an exact sequence:
\[\check{H}^{n-1}(X,x\mapsto A_x^{S_x}) \to \check{H}^{n-1}(X,L)\to \check{H}^n(X,A)\to 0\]
natural in $A$ where $L_x=A_x^{S_x}/A_x$.
\end{lemma}

\begin{proof}
By \Cref{long-exact-cech-cohomology}, it is enough to prove $\check{H}^n(X,x\mapsto A_x^{S_x}) = 0$ 
\rednote{maybe we could have this as an auxiliary lemma, as it does not use anything on $X$, $S$ and $A$? It can be found in Foundation I think.}
for all $n\geq 1$. Indeed assume given:
\[\alpha : \prod_{x:X} S_x^{n+1}\to S_x\to A_x\]
such that $\delta(\alpha) = 0$, i.e. for all $x:X$ and $u_0,\cdots, u_{n+1},v:S_x$ we have that:
\[\sum_{i=0}^{n+1}(-1)^i\alpha(x,u_0,\cdots,  \hat{u_i},\cdots ,u_{n+1},v) = 0\]
Then we define:
\[\beta : \prod_{x:X} S_x^{n}\to S_x\to A_x\]
\[\beta(x,u_0,\cdots,u_{n-1},v) = (-1)^n\alpha(x,u_0,\cdots,u_{n-1},v,v)\]
and then:
\[\delta(\beta)(x,u_0,\cdots,u_{n},v) = (-1)^n\sum_{i=0}^{n} (-1)^i \alpha(x,u_0,\cdots,\hat{u_i},\cdots,u_n,v,v) \]
\[= \alpha(x,u_0,\cdots,u_n,v)\]
\end{proof}

\begin{theorem}
Assume given a \v{C}ech cover:
\[f:S\to X\]
and $A:X\to\mathrm{Ab}_\ODisc$.

Then we have a natural isomorphism:
\[H^n(X,A) = \check{H}^n(X,A)\]
\end{theorem}

\begin{proof}
We proceed by induction on $n$. For $n=0$ we need to prove that maps:
\[\alpha:\prod_{s:S}A_{f(s)}\]
such that whenever $f(s)=f(t)$ we have that $\alpha(s) = \alpha(t)$ are naturally isomorphic to:
\[\prod_{x:X}A_x\]
This is immediate.

For the inductive step we use \Cref{inductive-definition-cohomology} and \Cref{inductive-definition-cech-cohomology}. Naturality comes from the naturality of the exact sequences.
\end{proof}


\subsection{The unit interval is acyclic}

\begin{proposition}\label{vanishing-cohomology-interval}
For all $A$ overtly discrete and all $k$ we have that:
\[H^k(\mathbb{I},A) = 0\]
\end{proposition}

\begin{proof}
TODO
\end{proof}
