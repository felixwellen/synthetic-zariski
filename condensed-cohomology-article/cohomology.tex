In this part we will use result from the previous section to compute some cohomology. In Sections \ref{vanishing-first-stone} and \ref{vanishing-stone} we will prove the vanishing of the cohomology of Stone spaces with overtly discrete coefficients (\Cref{vanishing-cohomology-stone}). In \Cref{cech-sheaf-agree-section} we use this vanishing to prove that \v{C}ech cohomology agree with the usual one for compact Hausdorff spaces  with overtly discrete coefficients (\Cref{cech-and-sheaf-agree}). The main point is that \v{C}ech cohomology is often simpler to compute. Indeed in \Cref{interval-acyclic} we apply this to proving the vanishing of the cohomology of the interval with overtly discrete coefficients (\Cref{cohomology-I}). Recall that by \Cref{overtly-discrete-countably-presented}, overtly discrete coefficient is the same as countably presented coefficients.



\subsection{Vanishing of the first (non-abelian) cohomology for Stone spaces}
\label{vanishing-first-stone}

In this section we will prove that given $S$ a Stone space and $G(x)$ an overtly discrete (non-abelian) group depending on $x:S$, we have that $H^1(S,G)=0$. Recall that by \Cref{overtly-discrete-countably-presented}, we have that a group is overtly discrete if and only if it is countably presented.

\rednote{TODO actually I don't think that $\Pi_{x:X}G(x)^{Y(x)}$ is normal in $\Pi_{x:X}G(x)^{Y(x)^2}$, so this is maybe not the correct definition? Or it is probably right for our purpose, as we are only interested in saying that all cocycle are cohomologous to 1? I think it is allright for vanishing, I doubt it is fine for computing $H^1(C,G)$ for $C$ compact Hausdorff.}

\begin{definition}
  Assume given a type $X$ with $Y(x)$ a type and $G(x)$ a group both depending on $x:X$. We define its short \v{C}ech sequence $\check{C}(X,Y,G)$ as:
  \[\Pi_{x:X}G(x)^{Y(x)} \overset{\delta_0}{\longrightarrow} \Pi_{x:X}G(x)^{Y(x)^2} \overset{\delta_1}{\longrightarrow} \Pi_{x:X}G(x)^{Y(x)^3} \]
  with:
  \[\delta_0(\alpha)_x(y_0,y_1) = \alpha_x(y_0)^{-1}\cdot\alpha_x(y_1)\]
  \[\delta_1(\beta)_x(y_0,y_1,y_2) = \beta_x(y_1,y_2)\cdot\beta_x(y_0,y_2)^{-1}\cdot\beta_x(y_0,y_1)\]
  We have that $\delta_1(\delta_0(\alpha)))_x(y_0,y_1,y_2) = 1$.
\end{definition}

First we give a couple of auxiliary results on \v{C}ech short sequences, justifying their usefulness.

\begin{lemma}\label{cover-with-section-exact-cech}
Assume given a type $X$ with $Y(x)$ a type and $G(x)$ a group both depending on $x:X$. If we merely have $\Pi_{x:X}Y(x)$, then $\check{C}(X,Y,G)$ is exact.
\end{lemma}

\begin{proof}
  The proof of Lemma 5.5 of \cite{synthetic-stone-duality} goes through.
%PRECISE CHECK
  %Assume given a cocycle $\beta: \Pi_{x : X} A_x^{T_x^2}$. 
  %Define $\alpha :\Pi_{x : X} A_x ^{T_x}$ by 
  %$\alpha_x(u) = \beta_x(t_x,u)$. Then 
  %$$
  %  \delta_0(\alpha)_x(u,v) = 
  %  \alpha_x(u)^{-1}\cdot\alpha_x(v) =
  %  \beta_x(t_x,u)^{-1} \cdot \beta(t_x,v)
  %$$
  %As $\beta$ is a cocycle, we have that 
  %$$\beta_x(u,v) \cdot \beta_x(t_x,v)^{-1} \cdot \beta_x(t_x,u) = 1,$$
  %hence
  %$$\beta_x(t_x,v)^{-1} \cdot \beta_x(t_x,u) = \beta_x(u,v)^{-1},$$
  %and taking inverses then gives that 
  %$$\beta_x(t_x,u)^{-1} \cdot \beta_x(t_x,v) = \beta_x(u,v),$$
  %from which we conclude that 
  %$\delta_0(\alpha)_x(u,v) = \beta_x(u,v)$, so $\beta$ is a coboundary. 
  %Hence the sequence is exact. 
\end{proof}

\begin{lemma}\label{cocyle-being-coboundary}
Assume given a set $X$ with $Y(x)$ an inhabited set and $G(x)$ a group both depending on $x:X$. Assume given $\phi:\Pi_{x:X}BG(x)$ and $\psi:\Pi_{x:X}(\phi(x)=*)^{Y(x)}$. If $\check{C}(X,Y,G)$ is exact then $\phi=*$.
\end{lemma}

\begin{proof}
 Identifying $* =_{BG(x)} *$ and $G(x)$, we define $\beta : \Pi_{x:X} G(x)^{T(x)^2}$ by:
 \[\beta_x(y_1,y_2) = \psi_x(y_0)^{-1} \cdot \psi_x(y_1)\] 
 It is easy to check that it is a cocycle in the given sequence (meaning $\delta_1(\beta) = 1$), so that by exactness it is a coboundary, i.e. there exists some $\alpha:\Pi_{x:X}G(x)^{Y(x)}$ such that for all $x:X$ and $y_0,y_1:Y(x)$ we have that:
\[\beta_x(y_0,y_1)=\alpha_x(y_0)^{-1}\cdot \alpha_x(y_1)\]
Then we define $\psi' : \Pi_{x:X}(\phi(x)=*)^{Y(x)}$ by:
\[\psi'_x(y) = \psi_x(y)\cdot \alpha_x(y)^{-1}\]
so that for all $x:X$ and $y_0,y_1:Y(x)$ we have that:
\begin{eqnarray}
\psi'_x(y_0)^{-1}\cdot\psi'_x(y_1) &=& \alpha_x(y_0) \cdot \psi_x(y_0)^{-1}\cdot \psi_x(y_1)\cdot \alpha_x(y_1)^{-1}\nonumber\\
 &=&  \alpha_x(y_0) \cdot \beta_x(y_0,y_1)\cdot \alpha_x(y_1)^{-1} \nonumber\\
 &=&  \alpha_x(y_0) \cdot\alpha_x(y_0)^{-1} \cdot\alpha_x(y_1)\cdot \alpha_x(y_1)^{-1} \nonumber\\ 
 &=& 1 \nonumber
 \end{eqnarray}
This means that $\psi'_x(y_0) = \psi'_x(y_1)$, so that $\psi'$ factors through $X$, giving a proof of $\phi=*$.
\end{proof}

\begin{lemma}\label{short-cech-exact}
Assume given $S$ a Stone space with $T(x)$ an inhabited Stone space and $G(x)$ an overtly discrete group both depending on $x:S$. Then $\check{C}(S,T,G)$ is exact.
\end{lemma}

\begin{proof}
We use \Cref{finite-approximation-stone-surjection-cohomology} to get a sequence of Stone spaces $(T_k(x))_{k:\N}$ depending on $x:S$ such that $\mathrm{lim}_kT_k(x) = T(x)$ for all $x:S$ and we merely have $\Pi_{x:S}T_k(x)$ for all $k:\N$. By \Cref{cover-with-section-exact-cech} we have that $\check{C}(S,T_k,G)$ is exact for all $k:\N$. As a sequential colimit of exact sequences is exact, we just need to prove $\mathrm{colim}_k\check{C}(S,T,G) = \check{C}(S,T,G)$ to conclude. But we have that:
\begin{eqnarray}
\mathrm{colim}_k \Pi_{x:S}G(x)^{T_k(x)^l} &=& \Pi_{x:S}\mathrm{colim}_k (G(x)^{T_k(x)^l})\nonumber\\
&=& \Pi_{x:S}G(x)^{\mathrm{lim}_kT_k(x)^l}\nonumber\\
&=& \Pi_{x:S}G(x)^{T(x)^l}\nonumber
\end{eqnarray}
where the first equality comes from \Cref{scott-continuity-right} and the second from \Cref{scott-continuity-left}.
\end{proof}

%\begin{lemma}\label{local-choice-delooping-trick}
%  Let $S$ be Compact Hausdorff and $G: S\to \mathrm{Grp}_\ODisc$, and 
%  $\phi:\Pi_{s:S}BG(x)$.
%  Then there merely exists some family $(T_x)_{x:S}$ of 
%  merely inhabited Stone spaces such that 
%  $\Pi_{x:S}(\phi_x = *)^{T_x}$.
%\end{lemma}

%\begin{proof}
%  Let $S$ be Stone and $G: S\to \mathrm{Grp}_\ODisc$. 
%  By definition of $BG$, we have that 
%  $\Pi_{x:S}\propTrunc{\phi_x=*}$. 
%  Thus by local choice, 
%  there merely exists some $T:\Stone$ with surjection $q : T \twoheadrightarrow S$, 
%  such that $\Pi_{t : T} \phi_{qt} = *$. 
%  Writing $T_x$ for the fiber of $q$ over $x:S$, 
%  we have that $\Pi_{x : S} (T_x \to \phi_x = *)$.
%  By surjectivity each ${T_x}$ is merely inhabited. 
%\end{proof} 

Now we just have to assemble the pieces.

\begin{proposition}\label{vanishing-1-cohomology-stone}
Assume given $S$ a Stone space with $G(x)$ an overtly discrete type depending on $x:S$. Then $H^1(S,G) = 0$
\end{proposition}

\begin{proof}
  Assume $\phi:\Pi_{s:S}BG(x)$,
  %by \Cref{local-choice-delooping-trick}, 
  by local choice
  there merely exists some family $(T_x)_{x:S}$ of inhabited Stone spaces with  
  $\alpha:\Pi_{x:S}(\phi(x)=*)^{T(x)}.$
In order to conclude by applying \Cref{cocyle-being-coboundary}, it is enough to prove that the \v{C}ech sequence $\check{C}(S,T,G)$ is exact. But this is \Cref{short-cech-exact}.
\end{proof}



\subsection{Stone spaces are acyclic}
\label{vanishing-stone}

The goal of this section is to show that given $S$ a Stone space with $A(x)$ an overly discrete abelian group depending on $x:S$, we have that $H^n(S,A) = 0$ for all $n\geq 1$. The case $n=1$ is an immediate consequence of \Cref{vanishing-1-cohomology-stone}. To prove the general case, we proceed by induction on $n$. For this induction to go through, we have to strenghten the result, so that we actually prove that deloopings of overtly discrete abelian group commutes with Stone-indexed $\Pi$-types.
%This acyclity will be used as an intermediate step. 

To be more specific, we will show in 
\Cref{eilenberg-exponentials} that the following maps $\epsilon_n^{S,A}$ is an equivalence:
\begin{definition}
  Assume given $S$ a Stone space with $A(x)$ an overtly discrete abelian group depending on $x:S$ and $n:\N$. We write $\epsilon_n^{S,A}$ for the canonical map from $K(\Pi_{x:S}A_x,n)$ to $\Pi_{x:S}K(A_x,n)$.
\end{definition}

\rednote{TODO is next lemma really necessary? We would need to change the definition to be able to state it.}
\begin{lemma}\label{ev-eq-acyclic}
  Assume given $X$ a type, $A(x)$ an abelian group depending on $x:X$ and $n\geq 1$. If $\epsilon_n^{S,A}$ is an equivalence, then $H^n(S,A) = 0$. 
\end{lemma}

\begin{proof}
  By definition
  $H^n(S,A) = || \Pi_{x : S} K(A_x,n) ||_0$. 
  If $\epsilon_n^{S,A}$ is an equivalence, it follows that the latter equals
  $|| K(\Pi_{x : S} A_x , n) ||_0$, 
  which is $0$ if $n\geq 1$. 
\end{proof}

\begin{lemma}\label{ev-embedding}
  Assume given $S,A,n$,
  if $\epsilon_n^{S,A}$ is an equivalence, then 
  $\epsilon_{n+1}^{S,A}$ is an embedding. 
\end{lemma}

\begin{proof}
  Note that the source is pointed and connected. 
  So it is enough to prove the induced map:
  \[\Omega(K(\Pi_{x:S}A_x,n+1)) \to \Omega(\Pi_{x:S}K(A_x,n+1))\]
  is an equivalence, but this map is equal to the canonical map:
  \[\epsilon_n^{S,A} : K(\Pi_{x:S}A_x,n) \to \Pi_{x:S}K(A_x,n),\]
  which was an equivalence by the assumption. 
\end{proof} 
\begin{lemma}\label{acyclic-ev-surjective}
  For $n\geq 1$, if $H^n(S,A) = 0$, then $\epsilon_{n}^{S,A}$ is a surjection. 
\end{lemma}
\begin{proof}
  Note that for $n\geq 1$, 
  $K(\Pi_{x:S}A_x,n)$ is $0$-connected. 
  Therefore, it's sufficient to show that 
  $\Pi_{x : S} K(A_x,n)$ is also $0$-connected, 
  which means exactly that $H^{n}(S,A) = || \Pi_{x : S} K(A_x,n) ||_0 = 0$. 
\end{proof} 


\rednote{TODO I think we should save ourselves some trouble by just using the long exact cohomology sequence.} Next lemma is quite standard, although we did not find any Homotopy Type Theory reference.

\begin{lemma}\label{ExactSequenceDelooping}
  Let $0\to A \to B \to C\to 0$ be an exact sequence of abelian groups.
  Then we get a fiber sequence 
  $$
  \cdots \to K(C, k-1) \to K(A,k) \to K(B,k) \to K(C,k) \to K(A,k+1) \to \cdots 
  $$
\end{lemma}

\begin{proof}
Note that $B \to C$ induces a pointed map 
$K(B,n) \to K(C,n)$, let $F$ be its fiber. It is a pointed $n$-type. 
Following Theorem 8.4.6 from \cite{hott}, we get the 
following exact sequence:
  $$
  \cdots \to \pi_{k+1}(K(C,n)) \to 
  \pi_{k}(F) \to \pi_{k}(K(B,n)) \to \pi_{k}(K(C,n)) \to 
  \pi_{k-1}(F) \to \cdots ,
  $$
  Form this plus the map $\pi_n(K(B,n))\to \pi_n(K(C,n))$ being surjective, we conclude that $F$ is $n-1$-connected and that $\pi_n(F) = A$. By unicity of deloopings we get $F=K(A,n)$. We can then splice the short fiber sequences $K(A,n)\to K(B,n) \to K(C,n)$ together using Lemma 8.4.4 from \cite{hott}.
  
%where $F$ is the fiber of $K(B,k) \to K(C,n)$. 
%Taking $k =n-1$, we have that 
%$$\pi_n(F) \to \pi_{n}K(B,n) \to \pi_nK(C,n) \to \pi_{n-1} (F) \to \pi_{n-1}K(B,n)$$
%is exact. 
%Now we use that for any group $G$, we have $\pi_{n-1}(K(G,n)) = 0, \pi_n(K(G,n)) = G$. 
%Thus $\pi_n(F) \to B \to C \to \pi_{n-1}(F) \to 0$ is a fiber sequence. 
%As $B \to C \to 0$ was exact, $B \to C$ is surjective, 
%so the image of $C \to \pi_{n-1}(F)$ is the same as that of 
%But since the kernel of $\pi_{n-1}(F) \to 0$ is all of $\pi_{n-1}(F)$, 
%we must have $\pi_{n-1}(F) = 0$.
%Thus $F$ is $n-1$-connected. 
%Also $F$ is the fiber of a map of $n$-types, which is an $n$-type. 
%So $F$ is an $n-1$-connected $n$-type.
%Thus $\pi_n(F)$ is a set and equals $A$. 
%Therefore $F = K(A,n)$. 
%Furthermore, the fiber of $K(A,n) \to K(B,n)$ should be $\Omega K(C,n) = K(C,n-1)$. 
%We conclude that we have a fiber sequence of the form
%$$\cdots\to 
%  K(A,k) \to K(B, k) \to K(C,k)
%$$
%but then also our first exact sequence also tells us that 
%there is a fiber sequence 
%$$\cdots\to 
%  K(C,k-1) \to 
%  K(A,k) \to K(B, k) \to K(C,k) \to
%  K(A,k+1) \to \cdots
%$$
\end{proof}

\begin{theorem}\label{eilenberg-exponentials}
Assume given $S:\Stone$ and $A:S\to \mathrm{Ab}_\ODisc$, then for all $n:\N$ we have that the canonical map:
\[\epsilon_n^{S,A} : K(\Pi_{x:S}A_x,n)\to \Pi_{x:S}K(A_x,n)\]
is an equivalence.
\end{theorem}

\begin{proof}
  We will use induction on $n$ to show that for all 
  $S:\Stone,
  A: S \to \mathrm{Ab}_\ODisc$, 
  the map $\epsilon_n^{S,A}$ is an equivalence
  If $n=0$, this is the identity map. 
  Now assume for some $n:\N$ that for all $S,A$ we have
  $\epsilon_n^{S,A}$ is an equivalence. 
  We will show that for all $S,A$ we also have $\epsilon_{n+1}^{S,A}$ is an equivalence. 
  By the induction hypothesis and \Cref{ev-embedding}, 
  $\epsilon_{n+1}^{S,A}$ is an embedding. 
  By Theorem 4.6.3 of \cite{hott}, 
  we need only to show that $\epsilon_{n+1}^{S,A}$ is surjective.
  By \Cref{acyclic-ev-surjective}, it's sufficient to show that $H^{n+1}(S,A) = 0$. 
  For $n=0$ this is the abelian version of \Cref{vanishing-1-cohomology-stone}. 
  We may thus assume $n\geq 1$ for the rest of this proof. 

  Let $\alpha : \Pi_{s : S} K(A_s , n+1)$, 
  we need to show that merely $\alpha = *$. 
  By \Cref{local-choice-delooping-trick}, 
  there merely exists some family $(T_x)_{x:S}$ of inhabited Stone spaces such that 
  $\Pi_{x:S} (\alpha_x = *)^{T_x}$. 
  For $x:S$, denote $L_x$ for $A_x^{T_x}/A_x$, 
  where the quotient is over the image of the constant map.
  As $A_x^{T_x} \to L_x$ is surjective, we get an exact sequence 
  $0\to A_x \to A_x^{T_x} \to L_x \to 0$. 
  By \Cref{ExactSequenceDelooping}, we get a fiber sequence
  $$\cdots\to 
    K(L_x,k-1) \to 
    K(A_x,k) \to K(A_x^{T_x}, k) \to K(L_x,k) \to
    K(A_x,k+1) \to \cdots
  $$
  Now as $\prod$ preserves fiber sequences we also get a fiber sequence 
  $$
    \cdots\to 
    \Pi_{x:S}K(L_x,k-1) \to 
    \Pi_{x:S} K(A_x,k) \to \Pi_{x:S}K(A_x^{T_x}, k) \to \Pi_{x:S}K(L_x,k) \to
    \Pi_{x:S} K(A_x,k+1) \to \cdots
  $$
  Consider the decomposition of 
  $\Pi_{x:S}K(A_x,n+1) \to \Pi_{x:S} K(A_x,n+1)^{T_x}$ given by:
  $$
    {\Pi_{x:S} K(A_x,n+1)} \to \Pi_{x:S}  {K(A_x^{T_x},n+1)} \hookrightarrow \Pi_{x:S}  {K(A_x,n+1)^{T_x}}
  $$
  where the second map is composition with 
  $\epsilon_{n+1}^{T_x,\lambda t . A_x}$, 
  which is an embedding by the induction hypothesis and \Cref{ev-embedding}. 
  By assumption on $T$, the composite sends $\alpha$ to $*$,
  and thus the first map also sends $\alpha$ to $*$.
  Thus $\alpha$ is in the kernel of $\Pi_{x:S}K(A_x,n+1) \to K(A_x^{T_x},n+1)$.
  By exactness $\alpha$ is in the image $\Pi_{x:S} K(L_x,n) \to \Pi_{x:S}K(A_x,n+1)$. 
  As $n\geq 1$, we can use the induction hypothesis and \Cref{ev-eq-acyclic} to see that $H^n(S,L)=0$.
  Thus any $\beta:\Pi_{x:S}K(L_x,n)$ merely equals $*$, 
  so their image merely equals $*$ in $\Pi_{x:S}K(A_x,n+1)$. 
  Thus we merely have $\alpha = *$, as required. 
\end{proof}

\begin{corollary}\label{vanishing-cohomology-stone}
Let $S$ be Stone and $A:S\to\mathrm{Ab}_\ODisc$. Then for all $n>0$ we have that:
\[H^n(S,A) = 0\]
\end{corollary}



\subsection{\v{C}ech cohomology}
\label{cech-sheaf-agree-section}

\rednote{TODO We also want that $\check{H}^1(C,S,G) = H^1(C,S)$ for $G$ non-abelian.}

\rednote{
\begin{theorem}\label{non-abelian-cech-and-sheaf-agree}
Non-abelian \v{C}ech and regular cohomology agree.
\end{theorem}
}




\begin{definition}
A \v{C}ech cover for a type $X$ consists of a surjective map:
\[f:S\to X\]
where $S$ is Stone and for all $x:X$ the fiber $S_x$ of $f$ over $x$ is Stone.
\end{definition}


Next lemma show that cohomology interact well with \v{C}ech cover. It will be used later to prove that cohomology and \v{C}ech cohomology agree.

\begin{lemma}\label{inductive-definition-cohomology}
Assume given $X$ a type with a \v{C}ech cover:
\[f:S\to X\]
as well as $A:X\to \mathrm{Ab}_\ODisc$.

For all $n\geq 1$ we have an exact sequence:
\[H^{n-1}(X,\lambda x.A_x^{S_x}) \to H^{n-1}(X,L)\to H^n(X,A)\to 0\]
natural in $A$, where $L_x = A_x^{S_x}/A_x$.
\end{lemma}

\begin{proof}
  Applying \Cref{ExactSequenceDelooping} to the exact sequence
  \[A_x\to A_x^{S_x}\to L_x\to 0,\]
  and the fact that $\prod$ and set truncation commute with fiber sequences, 
  we get a fiber sequence of the form 
  \[ \cdots \to 
    H^{n-1}(X,\lambda x.A_x^{S_x}) \to 
    H^{n-1}(X,L)\to 
    H^n(X,A)\to 
    H^n(X,\lambda x .A_x^{S_x}) \to \cdots\]
    We will show that 
    $H^n(X,\lambda x .A_x^{S_x})= 0$. 
    By \Cref{eilenberg-exponentials}, 
    we have $K(A_x^{S_x},n) = \Pi_{t:S_x}K(A_x,n)$. Thus
    $$
      H^n(X,\lambda x .A_x^{S_x})= 
      ||\Pi_{x:X}K(A_x^{S_x},n)||_0 = 
      ||\Pi_{x:X}\Pi_{t:S_x}K(A_x,n)||_0 = 
      ||\Pi_{s : S} K(A_{fs},n)||_0 = 
      H^n(S, \lambda s .A_{fs})
    $$
    and as $n\geq 1$, by \Cref{vanishing-cohomology-stone},
    the latter is $0$ as required. 
\end{proof}

Next definition will be used for \v{C}ech cover.

\begin{definition}
Assume given a type $X$ with $Y(x)$ a type and $A(x)$ an abelian group depending on $x:X$. We define its its \v{C}ech sequence $\check{C}(X,Y,A)$ as:
\[\Pi_{x:X}A(x)^{Y(x)} \overset{\delta_0}{\longrightarrow} \Pi_{x:X}G(x)^{A(x)^2} \overset{\delta_1}{\longrightarrow} \Pi_{x:X}A(x)^{Y(x)^3} \overset{\delta_3}{\longrightarrow} \cdots \]
with:
\[\delta_n(\alpha)_x(y_0,\cdots,y_n) = \Sigma_{i=0}^n (-1)^i \alpha(y_0,\hdots,\hat{x_i},\hdots,y_n)\]
We have that $\delta_{n+1} \circ \delta_n = 0$.

We define its $n$-th \v{C}ech cohomology group $\check{H}^n(X,Y,A)$ as the $n$-th cohomology of this chain complex.
\end{definition}

We prove a generic result about \v{C}ech sequence.

\begin{lemma}\label{cech-coefficient-lemma}
Assume given a type $X$ with $Y(x)$ a type and $A(x)$ an abelian group depending on $x:X$. Then $\check{H}(X,Y,x\mapsto A^{Y(x)})=0$.
\end{lemma}

\begin{proof}
Indeed assume given:
\[\alpha : \Pi_{x:X} Y(x)^{n+1}\to A(x)^{Y(x)}\]
such that $\delta(\alpha) = 0$, i.e. for all $x:X$ and $u_0,\cdots, u_{n+1},v:Y(x)$ we have that:
\[\Sigma_{i=0}^{n+1}(-1)^i\alpha(x,u_0,\cdots,  \widehat{u_i},\cdots ,u_{n+1},v) = 0\]
Then we define:
\[\beta : \Pi_{x:X} Y(x)^{n}\to A(x)^{Y(x)}\]
\[\beta(x,u_0,\cdots,u_{n-1},v) = (-1)^n\alpha(x,u_0,\cdots,u_{n-1},v,v)\]
and then:
\[\delta(\beta)(x,u_0,\cdots,u_{n},v) = (-1)^n\Sigma_{i=0}^{n} (-1)^i \alpha(x,u_0,\cdots,\hat{u_i},\cdots,u_n,v,v) \]
\[= \alpha(x,u_0,\cdots,u_n,v)\]
so $\alpha$ is indeed a coboundary.
\end{proof}

\begin{lemma}\label{long-exact-cech-cohomology}
Assume given a \v{C}ech cover:
\[f:S\to X\]
If we are given a short exact sequence of overtly discrete abelian group:
\[0\to A_x\to B_x\to C_x\to 0\]
depending on $x:X$, there is a long exact sequence of \v{C}ech cohomology groups:
\[\check{H}^0(X,A) \to\check{H}^0(X,B) \to\check{H}^0(X,C) \to\check{H}^1(X,A) \to\check{H}^1(X,B) \to\check{H}^1(X,C) \to\cdots \]
Moreover this long exact sequence is natural in the short exact sequence.
\end{lemma}

\begin{proof}
We just use the fact that all elements $\Sigma_{x:X} T_x^{k+1}$ in the \v{C}ech complex are Stone spaces, so a short exact sequence of overtly discrete abelian group induces a short exact of \v{C}ech complexes by \Cref{vanishing-cohomology-stone}.
\end{proof}

\begin{lemma}\label{inductive-definition-cech-cohomology}
Assume given a \v{C}ech cover:
\[f:S\to X\]
and $A:X\to\mathrm{Ab}_\ODisc$.

For all $n\geq 1$ we have an exact sequence:
\[\check{H}^{n-1}(X,x\mapsto A_x^{S_x}) \to \check{H}^{n-1}(X,L)\to \check{H}^n(X,A)\to 0\]
natural in $A$ where $L_x=A_x^{S_x}/A_x$.
\end{lemma}

\begin{proof}
By \Cref{long-exact-cech-cohomology}, it is enough to prove $\check{H}^n(X,S,x\mapsto A_x^{S_x}) = 0$ 
for all $n\geq 1$. This is \Cref{cech-coefficient-lemma}
\end{proof}

\begin{theorem}\label{cech-and-sheaf-agree}
Assume given a \v{C}ech cover:
\[f:S\to X\]
and $A:X\to\mathrm{Ab}_\ODisc$.

Then we have a natural isomorphism:
\[H^n(X,A) = \check{H}^n(X,A)\]
\end{theorem}

\begin{proof}
We proceed by induction on $n$. For $n=0$ we need to prove that maps:
\[\alpha:\Pi_{s:S}A_{f(s)}\]
such that whenever $f(s)=f(t)$ we have that $\alpha(s) = \alpha(t)$ are naturally isomorphic to:
\[\Pi_{x:X}A_x\]
This is immediate.

For the inductive step we use \Cref{inductive-definition-cohomology} and \Cref{inductive-definition-cech-cohomology}. Naturality comes from the naturality of the exact sequences.
\end{proof}


\subsection{The unit interval is acyclic}
\label{interval-acyclic}

This section closely follows Section 5.4 in \cite{foundation-synthtetic-stone-duality}.

\begin{remark}\label{description-Cn-simn}
  Recall that 
  there is a binary relation $\sim_n$ on $2^n=:\I_n$ such that 
  $(2^n,\sim_n)$ is equivalent to  $(\Fin(2^n),\lambda x,y.\ |x-y|\leq 1)$
  and for $\alpha,\beta:2^\N$ we have $(cs(\alpha) = cs(\beta)) \leftrightarrow 
  \left(\forall_{n:\N}\alpha|_n \sim_n \beta|_n\right)$. 
\end{remark}

We define $\I_n^{\sim k} = \{x_1,\hdots,x_n:\I_n\ |\ \forall(i,j).\, x_i\sim_n x_j\}$.

\begin{lemma}\label{interval-connected}
Given $I$ a finite type, any map from $\I$ to $I$ is constant.
\end{lemma}

\begin{proof}
We consider the map $p : 2^\N\to\I\to I$, by \Cref{factorisation-stone-finite} it factors through $2^n$ for some $n:\N$. 

Let us prove that for all $k:\N$, if the map factors through $2^{k+1}$ as $p_{k+1}$ then it factors through $2^k$. Consider $\alpha:2^k$, then since $p$ factors through $\I$ we have that $p(\alpha,1,0,0,\hdots) = p(\alpha,0,1,1,\hdots)$ so that $p_{k+1}(\alpha,0) = p_{k+1}(\alpha,1)$, and $p_{k+1}$ indeed factors through $2^k$, therefore so does $p$.

From this we get that $p$ factors through $2^0$, so it is indeed constant.
\end{proof}


\begin{lemma}\label{non-abelian-exact-sequence}
Given $n:\N$ and $G$ a group, we have an exact sequence:
\[G^{\I_n} \overset{\delta_0}{\longrightarrow} G^{\I_n^{\sim 2}} \overset{\delta_1}{\longrightarrow} G^{\I_n^{\sim 2}} \]
  where:
  \begin{eqnarray}
  \delta_0(\alpha)(i_0,i_1) = \alpha(i_0)^{-1}\cdot\alpha(i_1)\nonumber\\
  \delta_1(\beta)(i_0,i_1,i_2) = \beta(i_1,i_2)\cdot\beta(i_0,i_2)^{-1}\cdot\beta(i_0,i_1)\nonumber
  \end{eqnarray}
\end{lemma}

\begin{proof}
TODO
\end{proof}

\begin{lemma}\label{Cn-exact-sequence}
Given $n:\N$ and $A$ an abelian group, we have an exact sequence:
\[A^{\I_n} \overset{d_0}{\longrightarrow} A^{\I_n^{\sim 2}} \overset{d_1}{\longrightarrow} A^{\I_n^{\sim3}} \overset{d_3}{\longrightarrow} \cdots\]
where:
\begin{eqnarray}
 d_k(\alpha)(x_0,\hdots,x_k) &=& \Sigma_{i=0}^k (-1)^i\alpha(x_0,\hdots,\widehat{x_i},\hdots,x_k)\nonumber
\end{eqnarray}
\end{lemma}

\begin{proof}
This follows from the computations given in Stack Project Tag 01FG. \rednote{TODO we need more? Maybe Serre DAC 20, proposition 2?}
\end{proof}

\begin{theorem}\label{cohomology-I}
The unit interval is acyclic, in the sense that:

\begin{enumerate}[(i)]
\item Given $I$ an overtly discrete type, any map in $\I\to I$ is constant.
\item Given $G$ an overtly discrete group, we have that $H^1(\I,G) = 0$.
\item Given $A$ an overtly discrete abelian group, we have that $H^n(\I,A) = 0$ for $n\geq 1$.
\end{enumerate}
\end{theorem}

\begin{proof}
We prove each point as follows:

\begin{enumerate}[(i)]

\item Given a map $p:\I\to I$, we write $I=\mathrm{colim}_jI_j$ with $I_j$ finite. By \Cref{scott-continuity-chaus} we know that $p$ factors through $I_k$ for some $k$, but by \Cref{interval-connected} any map from $\I$ to $I_k$ is constant so $p$ is indeed constant.

\item Consider $cs:2^\N\to\I$ and the associated \v{C}ech cover $T$ of $\I$ defined by: 
\[T(x) = \Sigma_{y:2^\N} (x=_\I cs(y))\]
Then we have that $\mathrm{lim}_n\I_n^{\sim l} = \Sigma_{x:\I} T(x)^l$. By \Cref{non-abelian-exact-sequence} and stability of exactness under sequential colimit, we have an exact sequences:
\[ \mathrm{colim}_n G^{\I_n} \to \mathrm{colim}_n G^{\I_n^{\sim2}}\to \mathrm{colim}_n G^{\I_n^{\sim3}}\]
By \cref{scott-continuity-left} this sequence is equivalent to
\[\Pi_{x:\I}G^{T(x)} \to  \Pi_{x:\mathbb{I}}G^{T(x)^2} \to  \Pi_{x:\mathbb{I}}G^{T(x)^3}\]
So it being exact implies that $\check{H}^n(\I,T,G) = 0$ for $n\geq 1$.
We conclude by \Cref{non-abelian-cech-and-sheaf-agree}.

\item Same as point (ii) using \Cref{Cn-exact-sequence} instead of \ref{non-abelian-exact-sequence} and \Cref{cech-and-sheaf-agree} instead of \ref{non-abelian-cech-and-sheaf-agree}.

\end{enumerate}
\end{proof}

