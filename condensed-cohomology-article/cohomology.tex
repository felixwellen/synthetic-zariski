\rednote{TODO update section summary}

In this part we will use result from the previous section to compute some cohomology. In Sections \ref{vanishing-first-stone} and \ref{vanishing-stone} we will prove the vanishing of the cohomology of Stone spaces with overtly discrete coefficients (\Cref{vanishing-cohomology-stone}). In \Cref{cech-sheaf-agree-section} we use this vanishing to prove that \v{C}ech cohomology agree with the usual one for compact Hausdorff spaces  with overtly discrete coefficients (\Cref{cech-and-sheaf-agree}). The main point is that \v{C}ech cohomology is often simpler to compute. Indeed in \Cref{interval-acyclic} we apply this to proving the vanishing of the cohomology of the interval with overtly discrete coefficients (\Cref{cohomology-I}). Recall that by \Cref{overtly-discrete-countably-presented}, overtly discrete coefficient is the same as countably presented coefficients.


\subsection{Non-abelian \v{C}ech cohomology}

\rednote{TODO Are $X$ and $Y$ really an arbitrary type or did I use they were sets at some point?}

First a little bit on torsors. We define torsors using right actions.

\begin{definition}
Assume given $G$ a group and $\alpha$ a $G$-torsor with $x,y:\alpha$. Then $x^{-1}\cdot y$ is defined as the only $g:G$ such that $x\cdot g = y$. 
\end{definition}

Note that $(x^{-1}\cdot y)\cdot (y^{-1}\cdot z) = x^{-1}\cdot z$. Now we define non-abelian \v{C}ech cohomology

\begin{definition}
Assume given $X$ a type, $Y(x)$ an inhabited type and $G(x)$ a group both depending on $x:X$.
\begin{itemize}
\item A cocycle is a $\delta:\Pi_{x:X}G(x)^{Y(x)^2}$ such that for all $x:X$ and $y_0,y_1,y_2:Y(x)$ we have $\delta_x(y_0,y_1)\cdot \delta_x(y_1,y_2) = \delta_x(y_0,y_2)$.
\item Two cocycles $\delta$ and $\delta'$ are called cohomologous if there exists $\gamma:\Pi_{x:X}G(x)^{Y(x)}$ such that for all $x:X$ and $y_0,y_1:Y(x)$ we have $\gamma_x(y_0)\cdot\delta_x(y_0,y_1) = \delta'_x(y_0,y_1)\cdot\gamma_x(y_1)$.
\end{itemize}
\end{definition}

Being cohomologous in an equivalence relation. 

\begin{definition}
Assume given $X$ a type, $Y(x)$ an inhabited type and $G(x)$ a group both depending on $x:X$. The \v{C}ech non-abelian cohomology of $X$ with coefficient in $G$ is defined as cocycles modulo being cohomologous. It is denoted by $\check{H}^1(X,Y,G)$.
\end{definition}

We prove a first easy lemma.

\begin{lemma}\label{first-cech-has-section-vanish}
Assume given $X$ a type, $Y(x)$ an inhabited type and $G(x)$ a group both depending on $x:X$. If we merely have $\Pi_{x:X}Y(x)$ then $\check{H}^1(X,Y,G)$ is a singleton.
\end{lemma}

\begin{proof}
Assume given $\delta:\Pi_{x:X}G(x)^{Y(x)^2}$ a cocycle, we need to show it is cohomologous to $1$. It is enough to consider $\gamma:\Pi_{x:X}G(x)^{Y(x)}$ defined by $\gamma_x(y) = \delta_x(t_x,y)$.
\end{proof}

Our main goal is to show that \v{C}ech cohomology $\check{H}^1(X,Y,G)$ corresponds to $G$-torsors over $X$ that are trivialised by $Y$. More precisely:

\begin{definition}
Assume given $X$ a type, $Y(x)$ an inhabited type and $G(x)$ a group both depending on $x:X$. We define $H^1(X,Y,G)$ as $\{\alpha:H^1(X,G)\ |\ \pi_Y^*(\alpha) = 0 \}$ where $\pi_Y$ is the projection $\Sigma_{x:X}Y(x)\to X$.
\end{definition}

Our goal is then to show $H^1(X,Y,G) = \check{H}^1(X,Y,G)$.

\begin{lemma}\label{phi-well-defined}
Assume given $X$ a type, $Y(x)$ an inhabited type and $G(x)$ a group both depending on $x:X$. Given $\alpha:\Pi_{x:X}BG(x)$ and $\beta:\Pi_{x:X}\alpha(x)^{Y(x)}$, we define $\phi(\alpha,\beta):\Pi_{x:X}G(x)^{Y(x)^2}$ by $\phi(\alpha,\beta)_x(y_0,y_1) = \beta_x(y_0)^{-1}\cdot\beta_x(y_1)$. Then we have that:

\begin{enumerate}[(i)]
\item $\phi(\alpha,\beta)$ is a cocycle.
\item Given another $\beta':\Pi_{x:X}\alpha(x)^{Y(x)}$, we have that $\phi(\alpha,\beta)$ and $\phi(\alpha,\beta')$ are cohomologous.
\item We have an induced map $\Phi : H^1(X,Y,G)\to \check{H}^1(X,Y,G)$ defined by $\Phi(\alpha) = \phi(\alpha,\beta)$ for any $\beta:\Pi_{x:X}\alpha(x)^{Y(x)}$.
\end{enumerate}
\end{lemma}

\begin{proof}
As follows:

\begin{enumerate}[(i)]
\item Clear.
\item To show that $\phi(\alpha,\beta)$ and $\phi(\alpha,\beta')$ are cohomologous, we consider $\gamma = \lambda x,y.\, \beta'_x(y)^{-1}\cdot\beta_x(y)$. Given $x:X$ and $y_0,y_1:Y(x)$ we indeed have:
\[\phi(\alpha,\beta')_x(y_0,y_1)\cdot\gamma_x(y_1) = (\beta'_x(y_0)^{-1}\cdot \beta'_x(y_1))\cdot (\beta'_x(y_1)^{-1}\cdot\beta_x(y_1) = \beta'_x(y_0)^{-1}\cdot\beta_x(y_1)\]
\[\gamma_x(y_0)\cdot\phi(\alpha,\beta)_x(y_0,y_1) = (\beta'_x(y_0)^{-1}\cdot \beta_x(y_0))\cdot(\beta_x(y_0)^{-1}\cdot\beta_x(y_1)) = \beta'_x(y_0)^{-1}\cdot\beta_x(y_1)\]
\item Since $\alpha$ is trivialised by $\pi_Y$, we know there there exists such a $\beta$. By (ii) the class of $\psi(\alpha,\beta)$ in $\check{H}^1(X,Y,G)$ does not depend on the choice of such a $\beta$.
\end{enumerate}
\end{proof}

\begin{lemma}\label{psi-well-defined}
Assume given $Y$ an inhabited type and $G$ a group, as well as $\delta:G^{Y^2}$ a cocycle. %such that for all $y_0,y_1,y_2:Y$ we have $\delta(y_0,y_1)\cdot\delta(y_1,y_2) = \delta(y_0,y_2)$. 
Then we define $\psi(\delta) = \{\gamma:G^Y\ |\ \forall y_0,y_1.\, \gamma(y_0)\cdot\gamma(y_1)^{-1} = \delta(y_0,y_1)\}$. Then we have that:

\begin{enumerate}[(i)] 
\item We have a $H:\psi(\delta)^Y$ defined by $H(y) = \lambda y'.\, \delta(y',y)$.
\item The type $\psi(\delta)$ is a $G$-torsor.
\item Given $\delta$ and $\delta'$ cocycles with $\iota:G^Y$ such that for all $y_0,y_1:Y$ we have $\iota(y_0)\cdot\delta(y_0,y_1) = \delta(y_0,y_1)\cdot\iota(y_1)$, we have $\psi(\delta)=\psi(\delta')$.
\item We have an induced map $\Psi : \check{H}^1(X,Y,G)\to H^1(X,Y,G)$ defined by $\Psi(\delta) = \lambda x.\, \psi(\delta_x)$.
\end{enumerate}
\end{lemma}

\begin{proof}
We prove this as follows:

\begin{enumerate}[(i)]
\item We just need to check that for all $\bar{y},y_0,y_1$ we indeed have $\delta(y_0,\bar{y})\cdot\delta(y_1,\bar{y})^{-1} = \delta(y_0,y_1)$, but this follows from $\delta$ being a cocycle.

\item We define the action of $g:G$ on $\gamma:\psi(\delta)$ by $\gamma\cdot g = \lambda y.\, \gamma(y)\cdot g$.

It is clear that the action is well-defined and free, now let us check it is transitive. Assume $\gamma,\gamma':\psi(\delta)$, then we can check that $\gamma(y)^{-1}\cdot\gamma'(y)$ does not depend on $y$, indeed $\gamma(y_0)\cdot\gamma(y_1)^{-1} = \delta_x(y_0,y_1) = \gamma'(y_0)\cdot\gamma'(y_1)^{-1}$. Since $Y$ is inhabited this gives us an element $g:G$ such that $\gamma(y)\cdot g = \gamma'(y)$ for all $y:Y$, which is precisely what we want.

The fact that $\propTrunc{\psi(\delta)}$ follows directly from $Y$ inhabited and (i).

\item The map in $\psi(\delta)\to \psi(\delta')$ sending $\gamma$ to $\lambda y.\, \iota(y)\cdot\gamma(y)$ is clearly an equivalence of torsors.

\item Given (ii) and (iii) it is clear that we get an element in $H^1(X,G)$. From $(i)$ we see that we have $\Pi_{x:X}\psi(\delta_x)^{Y(x)}$, so that $\lambda x.\, \psi(\delta_x)$ is indeed trivialised by $\pi$.
\end{enumerate}
\end{proof}

\begin{proposition}\label{non-abelian-chech-is-sheaf-generic}
Assume given $X$ a type, $Y(x)$ an inhabited type and $G(x)$ a group both depending on $x:X$. Then $H^1(X,Y,G) = \check{H}^1(X,Y,G)$.
\end{proposition}

\begin{proof}
First we check that $\Psi\circ\Phi$ is the identity. It is enough to assume $\alpha:\Pi_{x:X}BG(x)$ with $\beta:\Pi_{x:X}\alpha(x)^{Y(x)}$ and prove that for all $x:X$ we have 
\[\alpha(x) = \psi(\psi(\alpha,\beta)_x)\]
where
\[\psi(\phi_x(\alpha,\beta)) = \{\gamma:G(x)^{Y(x)}\ |\ \forall y_0,y_1.\, \gamma(y_0)\cdot\gamma(y_1)^{-1} = \beta_x(y_0)^{-1}\cdot\beta_x(y_1)\}\]

We define $f:\alpha(x)\to \psi(\psi(\alpha,\beta)_x)$ by $f(e) = \lambda y.\, \beta_x(y)^{-1}\cdot e$. This is well defined as for all $y_0,y_1:Y(x)$ we have that:
\[f(p,y_0)\cdot f(p,y_1)^{-1} = \beta_x(y_0)^{-1}\cdot e \cdot (\beta_x(y_1)^{-1}\cdot e)^{-1} = \beta_x(y_0)^{-1}\cdot\beta_x(y_1)\]
Morever this map clearly commutes with the group action so it is enough to show it is an equivalence of type to conclude.

In the other direction, assume given $\gamma:\psi(\psi(\alpha,\beta)_x)$. Then we have that $\beta_x(y)\cdot\gamma(y):\alpha(x)$ does not depend on the choice of $y:Y(x)$ by the hypothesis on $\gamma$, since $Y(x)$ is inhabited this defines an element $g(\gamma):\alpha(x)$.

Next we check that $g(f(e)) = e$, since we want to prove a proposition we can assume $y:Y(x)$, then we need to prove that $\beta_x(y)\cdot (\beta_x(y)^{-1}\cdot e) = e$ which is clear. Finally to check $f(g(\gamma)) = \gamma$ we need to prove that $\beta_x(y)^{-1}\cdot \beta_x(y) \cdot \gamma(y) = \gamma(y)$ which is also obvious (we used the fact that $g(\gamma)$ can be computed using any $y:Y(x)$).

Now we check that $\Phi\circ \Psi$ is the identity. It is enough to prove that given a cocycle $\delta$, for all $x:X$ with $y_0,y_1:Y(x)$ we have that $H_x(y_0)^{-1}\cdot H_x(y_1) = \delta_x(y_0,y_1)$ where $H_x : Y(x)\to \psi(\delta_x)$ is defined in \Cref{psi-well-defined}, point (ii). But recall that $H_x(\bar{y}) = \lambda y.\delta_x(y,\bar{y})$ so we need to check that $(\lambda y.\delta_x(y,y_0))\cdot\delta_x(y_0,y_1) = \lambda y.\delta_x(y,y_1)$ which is clear as the action is defined pointwise.
\end{proof}

\begin{corollary}\label{non-abelian-chech-is-sheaf-generic-ayclic-cover}
Assume given $X$ a type, $Y(x)$ an inhabited type and $G(x)$ a group both depending on $x:X$. If $H^1(\Sigma_{x:X}Y(x), G\circ\pi) = 0$, then $H^1(X,G) = \check{H}^1(X,Y,G)$.
\end{corollary}


\subsection{Non-abelian cohomology for Stone spaces and compact Hausdorff spaces}

\begin{lemma}\label{first-cech-scott-continuity}
Assume given $S$ a Stone space with $(T_k(x))_{k:\N}$ a tower of Stone spaces and $G(x)$ an overly discrete group both depending on $x:S$. Write $T(x) =\mathrm{lim}_kT_k(x)$. Then $\check{H}^1(S,T,G) = \mathrm{colim}_k \check{H}^1(S,T_k,G)$.
\end{lemma}

\begin{proof}
TODO
\end{proof}

\begin{lemma}\label{cech-vanishing-first-stone}
Assume given $S$ a Stone space with $T(x)$ an inhabited Stone spaces and $G(x)$ an overtly discrete group both depending on $x:X$. Then $\check{H}^1(S,T,G) = 0$.
\end{lemma}

\begin{proof}
By \Cref{finite-approximation-stone-surjection-cohomology} we have $(T_k(x))_{k:\N}$ a tower of Stone spaces depending on $x:S$ such that for all $x:S$ we have that $\mathrm{lim}_kT_k(x) = 0$ and we merely have $\Pi_{x:S}T_k(x)$ for all $k:\N$.

Then by \Cref{first-cech-scott-continuity} we have that $\check{H}^1(S,T,G) = \mathrm{colim}_k\check{H}^1(S,T_k,G)$, and by \Cref{first-cech-has-section-vanish} we have that $\check{H}^1(S,T_k,G) = 0$ for all $k:\N$ so we can conclude.
\end{proof}

\begin{proposition}\label{vanishing-first-stone}
Assume given $S$ a Stone space with $G(x)$ an overtly discrete group depending in $x:X$. Then $H^1(S,G) = 0$.
\end{proposition}

\begin{proof}
Let us consider $\alpha:H^1(S,G)$. By local choice there is $T(x)$ an inhabited Stone space depending on $x:S$ such that $\pi^*(\alpha) = 0$, so that $\alpha$ is in $H^1(S,T,G)$. But by \Cref{non-abelian-chech-is-sheaf-generic} we have $H^1(S,T,G) = \check{H}^1(S,T,G)$ and by \Cref{cech-vanishing-first-stone} we have $\check{H}^1(S,T,G) = 0$, so that $H^1(S,T,G) = 0$ and $\alpha=0$.
\end{proof}

\begin{definition}
A \v{C}ech cover for a set $X$ consists of $S(x)$ an inhabited Stone spaces depending on $x:X$ such that $\Sigma_{x:X}S(x)$ is a Stone space.
\end{definition}

\begin{remark}
A set $X$ merely has a \v{C}ech cover if and only is it is a compact Hausdorff space. This is because given $X$ a type with a \v{C}ech cover $S$, given $x:X$ and $y:S$ we have that $x=[y]$ if and only if $y\in f^{-1}(x)$ and any Stone subspace of a Stone space is closed.
\end{remark}

\begin{proposition}\label{non-abelian-cech-and-sheaf-agree}
Assume given $X$ a compact Hausdorff space with a \v{C}ech cover $S$ and $G(x)$ an overtly discrete group depending on $x:X$. Then $H^1(X,G) = \check{H}^1(X,S,G)$.
\end{proposition}

\begin{proof}
From \Cref{non-abelian-chech-is-sheaf-generic-ayclic-cover} and \Cref{vanishing-first-stone}.
\end{proof}



\subsection{Vanishing of higher cohomology for Stone spaces}
\label{vanishing-stone}

The goal of this section is to show that given $S$ a Stone space with $A(x)$ an overly discrete abelian group depending on $x:S$, we have that $H^n(S,A) = 0$ for all $n\geq 1$. The case $n=1$ is an immediate consequence of \Cref{vanishing-1-cohomology-stone}. To prove the general case, we proceed by induction on $n$. For this induction to go through, we have to strenghten the result, so that we actually prove that deloopings of overtly discrete abelian group commutes with Stone-indexed $\Pi$-types.
%This acyclity will be used as an intermediate step. 

To be more specific, we will show in 
\Cref{eilenberg-exponentials} that the following maps $\epsilon_n^{S,A}$ is an equivalence:
\begin{definition}
  Assume given $S$ a Stone space with $A(x)$ an overtly discrete abelian group depending on $x:S$ and $n:\N$. We write $\epsilon_n^{S,A}$ for the canonical map from $K(\Pi_{x:S}A_x,n)$ to $\Pi_{x:S}K(A_x,n)$.
\end{definition}

\rednote{TODO is next lemma really necessary? We would need to change the definition to be able to state it.}
\begin{lemma}\label{ev-eq-acyclic}
  Assume given $X$ a type, $A(x)$ an abelian group depending on $x:X$ and $n\geq 1$. If $\epsilon_n^{S,A}$ is an equivalence, then $H^n(S,A) = 0$. 
\end{lemma}

\begin{proof}
  By definition
  $H^n(S,A) = || \Pi_{x : S} K(A_x,n) ||_0$. 
  If $\epsilon_n^{S,A}$ is an equivalence, it follows that the latter equals
  $|| K(\Pi_{x : S} A_x , n) ||_0$, 
  which is $0$ if $n\geq 1$. 
\end{proof}

\begin{lemma}\label{ev-embedding}
  Assume given $S,A,n$,
  if $\epsilon_n^{S,A}$ is an equivalence, then 
  $\epsilon_{n+1}^{S,A}$ is an embedding. 
\end{lemma}

\begin{proof}
  Note that the source is pointed and connected. 
  So it is enough to prove the induced map:
  \[\Omega(K(\Pi_{x:S}A_x,n+1)) \to \Omega(\Pi_{x:S}K(A_x,n+1))\]
  is an equivalence, but this map is equal to the canonical map:
  \[\epsilon_n^{S,A} : K(\Pi_{x:S}A_x,n) \to \Pi_{x:S}K(A_x,n),\]
  which was an equivalence by the assumption. 
\end{proof} 
\begin{lemma}\label{acyclic-ev-surjective}
  For $n\geq 1$, if $H^n(S,A) = 0$, then $\epsilon_{n}^{S,A}$ is a surjection. 
\end{lemma}
\begin{proof}
  Note that for $n\geq 1$, 
  $K(\Pi_{x:S}A_x,n)$ is $0$-connected. 
  Therefore, it's sufficient to show that 
  $\Pi_{x : S} K(A_x,n)$ is also $0$-connected, 
  which means exactly that $H^{n}(S,A) = || \Pi_{x : S} K(A_x,n) ||_0 = 0$. 
\end{proof} 


\rednote{TODO I think we should save ourselves some trouble by just using the long exact cohomology sequence.} Next lemma is quite standard, although we did not find any Homotopy Type Theory reference.

\begin{lemma}\label{ExactSequenceDelooping}
  Let $0\to A \to B \to C\to 0$ be an exact sequence of abelian groups.
  Then we get a fiber sequence 
  $$
  \cdots \to K(C, k-1) \to K(A,k) \to K(B,k) \to K(C,k) \to K(A,k+1) \to \cdots 
  $$
\end{lemma}

\begin{proof}
Note that $B \to C$ induces a pointed map 
$K(B,n) \to K(C,n)$, let $F$ be its fiber. It is a pointed $n$-type. 
Following Theorem 8.4.6 from \cite{hott}, we get the 
following exact sequence:
  $$
  \cdots \to \pi_{k+1}(K(C,n)) \to 
  \pi_{k}(F) \to \pi_{k}(K(B,n)) \to \pi_{k}(K(C,n)) \to 
  \pi_{k-1}(F) \to \cdots ,
  $$
  Form this plus the map $\pi_n(K(B,n))\to \pi_n(K(C,n))$ being surjective, we conclude that $F$ is $n-1$-connected and that $\pi_n(F) = A$. By unicity of deloopings we get $F=K(A,n)$. We can then splice the short fiber sequences $K(A,n)\to K(B,n) \to K(C,n)$ together using Lemma 8.4.4 from \cite{hott}.
  
%where $F$ is the fiber of $K(B,k) \to K(C,n)$. 
%Taking $k =n-1$, we have that 
%$$\pi_n(F) \to \pi_{n}K(B,n) \to \pi_nK(C,n) \to \pi_{n-1} (F) \to \pi_{n-1}K(B,n)$$
%is exact. 
%Now we use that for any group $G$, we have $\pi_{n-1}(K(G,n)) = 0, \pi_n(K(G,n)) = G$. 
%Thus $\pi_n(F) \to B \to C \to \pi_{n-1}(F) \to 0$ is a fiber sequence. 
%As $B \to C \to 0$ was exact, $B \to C$ is surjective, 
%so the image of $C \to \pi_{n-1}(F)$ is the same as that of 
%But since the kernel of $\pi_{n-1}(F) \to 0$ is all of $\pi_{n-1}(F)$, 
%we must have $\pi_{n-1}(F) = 0$.
%Thus $F$ is $n-1$-connected. 
%Also $F$ is the fiber of a map of $n$-types, which is an $n$-type. 
%So $F$ is an $n-1$-connected $n$-type.
%Thus $\pi_n(F)$ is a set and equals $A$. 
%Therefore $F = K(A,n)$. 
%Furthermore, the fiber of $K(A,n) \to K(B,n)$ should be $\Omega K(C,n) = K(C,n-1)$. 
%We conclude that we have a fiber sequence of the form
%$$\cdots\to 
%  K(A,k) \to K(B, k) \to K(C,k)
%$$
%but then also our first exact sequence also tells us that 
%there is a fiber sequence 
%$$\cdots\to 
%  K(C,k-1) \to 
%  K(A,k) \to K(B, k) \to K(C,k) \to
%  K(A,k+1) \to \cdots
%$$
\end{proof}

\begin{theorem}\label{eilenberg-exponentials}
Assume given $S:\Stone$ and $A:S\to \mathrm{Ab}_\ODisc$, then for all $n:\N$ we have that the canonical map:
\[\epsilon_n^{S,A} : K(\Pi_{x:S}A_x,n)\to \Pi_{x:S}K(A_x,n)\]
is an equivalence.
\end{theorem}

\begin{proof}
  We will use induction on $n$ to show that for all 
  $S:\Stone,
  A: S \to \mathrm{Ab}_\ODisc$, 
  the map $\epsilon_n^{S,A}$ is an equivalence
  If $n=0$, this is the identity map. 
  Now assume for some $n:\N$ that for all $S,A$ we have
  $\epsilon_n^{S,A}$ is an equivalence. 
  We will show that for all $S,A$ we also have $\epsilon_{n+1}^{S,A}$ is an equivalence. 
  By the induction hypothesis and \Cref{ev-embedding}, 
  $\epsilon_{n+1}^{S,A}$ is an embedding. 
  By Theorem 4.6.3 of \cite{hott}, 
  we need only to show that $\epsilon_{n+1}^{S,A}$ is surjective.
  By \Cref{acyclic-ev-surjective}, it's sufficient to show that $H^{n+1}(S,A) = 0$. 
  For $n=0$ this is the abelian version of \Cref{vanishing-1-cohomology-stone}. 
  We may thus assume $n\geq 1$ for the rest of this proof. 

  Let $\alpha : \Pi_{s : S} K(A_s , n+1)$, 
  we need to show that merely $\alpha = *$. 
  By \Cref{local-choice-delooping-trick}, 
  there merely exists some family $(T_x)_{x:S}$ of inhabited Stone spaces such that 
  $\Pi_{x:S} (\alpha_x = *)^{T_x}$. 
  For $x:S$, denote $L_x$ for $A_x^{T_x}/A_x$, 
  where the quotient is over the image of the constant map.
  As $A_x^{T_x} \to L_x$ is surjective, we get an exact sequence 
  $0\to A_x \to A_x^{T_x} \to L_x \to 0$. 
  By \Cref{ExactSequenceDelooping}, we get a fiber sequence
  $$\cdots\to 
    K(L_x,k-1) \to 
    K(A_x,k) \to K(A_x^{T_x}, k) \to K(L_x,k) \to
    K(A_x,k+1) \to \cdots
  $$
  Now as $\prod$ preserves fiber sequences we also get a fiber sequence 
  $$
    \cdots\to 
    \Pi_{x:S}K(L_x,k-1) \to 
    \Pi_{x:S} K(A_x,k) \to \Pi_{x:S}K(A_x^{T_x}, k) \to \Pi_{x:S}K(L_x,k) \to
    \Pi_{x:S} K(A_x,k+1) \to \cdots
  $$
  Consider the decomposition of 
  $\Pi_{x:S}K(A_x,n+1) \to \Pi_{x:S} K(A_x,n+1)^{T_x}$ given by:
  $$
    {\Pi_{x:S} K(A_x,n+1)} \to \Pi_{x:S}  {K(A_x^{T_x},n+1)} \hookrightarrow \Pi_{x:S}  {K(A_x,n+1)^{T_x}}
  $$
  where the second map is composition with 
  $\epsilon_{n+1}^{T_x,\lambda t . A_x}$, 
  which is an embedding by the induction hypothesis and \Cref{ev-embedding}. 
  By assumption on $T$, the composite sends $\alpha$ to $*$,
  and thus the first map also sends $\alpha$ to $*$.
  Thus $\alpha$ is in the kernel of $\Pi_{x:S}K(A_x,n+1) \to K(A_x^{T_x},n+1)$.
  By exactness $\alpha$ is in the image $\Pi_{x:S} K(L_x,n) \to \Pi_{x:S}K(A_x,n+1)$. 
  As $n\geq 1$, we can use the induction hypothesis and \Cref{ev-eq-acyclic} to see that $H^n(S,L)=0$.
  Thus any $\beta:\Pi_{x:S}K(L_x,n)$ merely equals $*$, 
  so their image merely equals $*$ in $\Pi_{x:S}K(A_x,n+1)$. 
  Thus we merely have $\alpha = *$, as required. 
\end{proof}

\begin{corollary}\label{vanishing-cohomology-stone}
Let $S$ be Stone and $A:S\to\mathrm{Ab}_\ODisc$. Then for all $n>0$ we have that:
\[H^n(S,A) = 0\]
\end{corollary}



\subsection{\v{C}ech cohomology for compact Hausdorff spaces}
\label{cech-sheaf-agree-section}

Next lemma show that cohomology interact well with \v{C}ech cover. It will be used later to prove that cohomology and \v{C}ech cohomology agree.

\begin{lemma}\label{inductive-definition-cohomology}
Assume given $X$ a type with a \v{C}ech cover $S$ and $A(x)$ an overtly discrete abelian group depending on $x:X$. Then for all $n:\N$ we have an exact sequence:
\[H^{n}(X,\lambda x.A(x)^{S(x)}) \to H^{n}(X,L)\to H^{n+1}(X,A)\to 0\]
natural in $A$, where $L(x) = A(x)^{S(x)}/A(x)$.
\end{lemma}

\begin{proof}
From the long exact cohomology sequence we just need to prove that $H^{n+1}(X,\lambda x.A(x)^{S(x)}) = 0$. But by \Cref{eilenberg-exponentials} we have that $\Pi_{x:X}K(A(x)^{S(x)},n+1) = \Pi_{x:X}K(A(x),n+1)^{S(x)}$ so that $H^{n+1}(X,\lambda x.A(x)^{S(x)}) = H^{n+1}(\Sigma_{x:X}S(x),A\circ \pi_S)$ so it is $0$ by \Cref{vanishing-cohomology-stone}.
\end{proof}

Next definition will be used for \v{C}ech cover.

\begin{definition}
Assume given a type $X$ with $Y(x)$ an inhabited type and $A(x)$ an abelian group both depending on $x:X$. We define its its \v{C}ech sequence $\check{C}(X,Y,A)$ as:
\[\Pi_{x:X}A(x)^{Y(x)} \overset{\delta_0}{\longrightarrow} \Pi_{x:X}A(x)^{Y(x)^2} \overset{\delta_1}{\longrightarrow} \Pi_{x:X}A(x)^{Y(x)^3} \overset{\delta_3}{\longrightarrow} \cdots \]
with:
\[\delta_n(\alpha)_x(y_0,\cdots,y_n) = \Sigma_{i=0}^n (-1)^i \alpha(y_0,\hdots,\hat{x_i},\hdots,y_n)\]
We have that $\delta_{n+1} \circ \delta_n = 0$.

We define its $n$-th \v{C}ech cohomology group $\check{H}^n(X,Y,A)$ as the $n$-th cohomology group of this chain complex.
\end{definition}

We prove a generic result about \v{C}ech sequence.

\begin{lemma}\label{cech-coefficient-lemma}
Assume given a type $X$ with $Y(x)$ a type and $A(x)$ an abelian group both depending on $x:X$. Then $\check{H}^n(X,Y,\lambda x.\, A(x)^{Y(x)})=0$ for all $n\geq 1$.
\end{lemma}

\begin{proof}
Indeed assume given:
\[\alpha : \Pi_{x:X} Y(x)^{n+1}\to A(x)^{Y(x)}\]
such that $\delta(\alpha) = 0$, i.e. for all $x:X$ and $y_0,\hdots, y_{n+1},y:Y(x)$ we have that:
\[\Sigma_{i=0}^{n+1}(-1)^i\alpha(x,y_0,\hdots,\widehat{y_i},\hdots ,y_{n+1},y) = 0\]
Then we define:
\[\beta : \Pi_{x:X} Y(x)^{n}\to A(x)^{Y(x)}\]
\[\beta(x,y_0,\hdots,y_{n-1},y) = (-1)^n\alpha(x,y_0,\hdots,y_{n-1},y,y)\]
and then:
\[\delta(\beta)(x,y_0,\hdots,y_{n},y) = (-1)^n\Sigma_{i=0}^{n} (-1)^i \alpha(x,y_0,\hdots,\hat{y_i},\hdots,y_n,y,y) \]
\[= \alpha(x,y_0,\hdots,y_n,y)\]
so $\delta(\alpha) = 0$.
\end{proof}

\begin{lemma}\label{long-exact-cech-cohomology}
Assume given $X$ a type with $S$ a \v{C}ech cover and a short exact sequence of overtly discrete abelian group:
\[0\to A(x)\to B(x)\to C(x)\to 0\]
depending on $x:X$. Then there is a long exact sequence of \v{C}ech cohomology groups:
\[\check{H}^0(X,A) \to\check{H}^0(X,B) \to\check{H}^0(X,C) \to\check{H}^1(X,A) \to\check{H}^1(X,B) \to\check{H}^1(X,C) \to\cdots \]
Moreover this long exact sequence is natural in the short exact sequence.
\end{lemma}

\begin{proof}
We just use the fact that all elements $\Sigma_{x:X} T_x^{k+1}$ in the \v{C}ech complex are Stone spaces, so a short exact sequence of overtly discrete abelian group induces a short exact sequence of \v{C}ech complexes by \Cref{vanishing-cohomology-stone}, and then we use the long exact cohomology sequence induced by such a short exact sequence of complexes.
\end{proof}

\begin{lemma}\label{inductive-definition-cech-cohomology}
Assume given $X$ a type with $S$ a \v{C}ech cover and $A(x)$ an overtly discrete abelian group depending on $x:X$.

Then for all $n:\N$ we have an exact sequence:
\[\check{H}^{n}(X,\lambda x.\, A(x)^{S(x)}) \to \check{H}^{n}(X,L)\to \check{H}^{n+1}(X,A)\to 0\]
natural in $A$ where $L(x)=A(x)^{S(x)}/A(x)$.
\end{lemma}

\begin{proof}
By \Cref{long-exact-cech-cohomology}, it is enough to prove $\check{H}^{n+1}(X,\lambda x.\, A(x)^{S(x)}) = 0$. This is \Cref{cech-coefficient-lemma}.
\end{proof}

\begin{theorem}\label{cech-and-sheaf-agree}
Assume given $X$ a type with $S$ a \v{C}ech cover and $A(x)$ an overtly discrete abelian group depending on $x:X$.

Then we have a natural isomorphism $H^n(X,A) = \check{H}^n(X,S,A)$
\end{theorem}

\begin{proof}
We proceed by induction on $n$. For $n=0$ we need to prove that the type of maps:
\[\{\alpha:\Pi_{x:X}A(x)^{S(x)}\ |\ \forall x,s_0,s_1.\, \alpha_x(s_0)=\alpha_x(s_1)\}\]
is naturally isomorphic to $\Pi_{x:X}A(x)$. This is immediate from $S(x)$ being inhabited sets.

For the inductive step we use \Cref{inductive-definition-cohomology} and \Cref{inductive-definition-cech-cohomology}. Naturality comes from the naturality of the long exact sequences.
\end{proof}



\subsection{The unit interval is acyclic}
\label{interval-acyclic}

This section extends Section 5.4 in \cite{foundation-synthtetic-stone-duality}.

\begin{remark}\label{description-Cn-simn}
  Recall that 
  there is a binary relation $\sim_n$ on $2^n=:\I_n$ such that 
  $(2^n,\sim_n)$ is equivalent to  $(\Fin(2^n),\lambda x,y.\ |x-y|\leq 1)$
  and for $\alpha,\beta:2^\N$ we have $(cs(\alpha) = cs(\beta)) \leftrightarrow 
  \left(\forall_{n:\N}\alpha|_n \sim_n \beta|_n\right)$. 
\end{remark}

We define $\I_n^{\sim k} = \{x_1,\hdots,x_n:\I_n\ |\ \forall(i,j).\, x_i\sim_n x_j\}$.

\begin{lemma}\label{interval-connected}
Given $I$ a finite type, any map from $\I$ to $I$ is constant.
\end{lemma}

\begin{proof}
We consider the map $p : 2^\N\to\I\to I$, by \Cref{factorisation-stone-finite} it factors through $2^n$ for some $n:\N$. 

Let us prove that for all $k:\N$, if the map factors through $2^{k+1}$ as $p_{k+1}$ then it factors through $2^k$. Consider $\alpha:2^k$, then since $p$ factors through $\I$ we have that $p(\alpha,1,0,0,\hdots) = p(\alpha,0,1,1,\hdots)$ so that $p_{k+1}(\alpha,0) = p_{k+1}(\alpha,1)$, and $p_{k+1}$ indeed factors through $2^k$, therefore so does $p$.

From this we get that $p$ factors through $2^0$, so it is indeed constant.
\end{proof}

\rednote{Should be rephrased properly, not talking about exact sequence}

\begin{lemma}\label{non-abelian-exact-sequence}
Given $n:\N$ and $G$ a group, we have an exact sequence:
\[G^{\I_n} \overset{\delta_0}{\longrightarrow} G^{\I_n^{\sim 2}} \overset{\delta_1}{\longrightarrow} G^{\I_n^{\sim 2}} \]
  where:
  \begin{eqnarray}
  \delta_0(\alpha)(i_0,i_1) = \alpha(i_0)^{-1}\cdot\alpha(i_1)\nonumber\\
  \delta_1(\beta)(i_0,i_1,i_2) = \beta(i_1,i_2)\cdot\beta(i_0,i_2)^{-1}\cdot\beta(i_0,i_1)\nonumber
  \end{eqnarray}
\end{lemma}

\begin{proof}
TODO
\end{proof}

\begin{lemma}\label{Cn-exact-sequence}
Given $n:\N$ and $A$ an abelian group, we have an exact sequence:
\[A^{\I_n} \overset{d_0}{\longrightarrow} A^{\I_n^{\sim 2}} \overset{d_1}{\longrightarrow} A^{\I_n^{\sim3}} \overset{d_3}{\longrightarrow} \cdots\]
where:
\begin{eqnarray}
 d_k(\alpha)(x_0,\hdots,x_k) &=& \Sigma_{i=0}^k (-1)^i\alpha(x_0,\hdots,\widehat{x_i},\hdots,x_k)\nonumber
\end{eqnarray}
\end{lemma}

\begin{proof}
This follows from the computations given in Stack Project Tag 01FG. \rednote{TODO we need more? Maybe Serre DAC 20, proposition 2?}
\end{proof}

\begin{theorem}\label{cohomology-I}
The unit interval is acyclic, in the sense that:

\begin{enumerate}[(i)]
\item Given $I$ an overtly discrete type, any map in $\I\to I$ is constant.
\item Given $G$ an overtly discrete group, we have that $H^1(\I,G) = 0$.
\item Given $A$ an overtly discrete abelian group, we have that $H^n(\I,A) = 0$ for $n\geq 1$.
\end{enumerate}
\end{theorem}

\begin{proof}
We prove each point as follows:

\begin{enumerate}[(i)]

\item Given a map $p:\I\to I$, we write $I=\mathrm{colim}_jI_j$ with $I_j$ finite. By \Cref{scott-continuity-chaus} we know that $p$ factors through $I_k$ for some $k$, but by \Cref{interval-connected} any map from $\I$ to $I_k$ is constant so $p$ is indeed constant.

\item Consider $cs:2^\N\to\I$ and the associated \v{C}ech cover $T$ of $\I$ defined by: 
\[T(x) = \Sigma_{y:2^\N} (x=_\I cs(y))\]
Then we have that $\mathrm{lim}_n\I_n^{\sim l} = \Sigma_{x:\I} T(x)^l$. By \Cref{non-abelian-exact-sequence} and stability of exactness under sequential colimit, we have an exact sequences:
\[ \mathrm{colim}_n G^{\I_n} \to \mathrm{colim}_n G^{\I_n^{\sim2}}\to \mathrm{colim}_n G^{\I_n^{\sim3}}\]
By \cref{scott-continuity-left} this sequence is equivalent to
\[\Pi_{x:\I}G^{T(x)} \to  \Pi_{x:\mathbb{I}}G^{T(x)^2} \to  \Pi_{x:\mathbb{I}}G^{T(x)^3}\]
So it being exact implies that $\check{H}^n(\I,T,G) = 0$ for $n\geq 1$.
We conclude by \Cref{non-abelian-cech-and-sheaf-agree}.

\item Same as point (ii) using \Cref{Cn-exact-sequence} instead of \ref{non-abelian-exact-sequence} and \Cref{cech-and-sheaf-agree} instead of \ref{non-abelian-cech-and-sheaf-agree}.

\end{enumerate}
\end{proof}

We give an example of application, which justifies why we spend all the effort on the non-abelian case. Recall that a type $X$ is $\I$-local if the map $X\to X^\I$ is an equivalence.

\begin{lemma}\label{delooping-I-local}
We have the following:

\begin{enumerate}[(i)]
\item Any overtly discrete type is $\I$-local.
\item Given $G$ an overtly discrete group, we have that $BG$ is $\I$-local.
\item Given $A$ an overtly discrete abelian group and $n:\N$, we have that $K(A,n)$ is overtly discrete.
\end{enumerate}
\end{lemma}

\begin{proof}
As follows:

\begin{enumerate}[(i)]
\item From \Cref{cohomology-I}(i).
\item Identity types in $BG$ are $G$-torsors, therefore they are $\I$-local by (i), so the map $BG\to BG^\I$ is an embedding. It is surjective by \Cref{cohomology-I}(ii).
\item By induction on $n:\N$, the case $n=0$ is point (i). Assume it holds for $n\geq 0$, then identity types in $K(A,n+1)$ are $K(A,n)$-torsors, therefore they are $\I$-local by induction hypothesis, so the map $K(A,n+1)\to K(A,n+1)^\I$ is an embedding. It is surjective by \Cref{cohomology-I}(iii).
\end{enumerate} 
\end{proof}

\begin{proposition}
Assume given $X$ a type such that:

\begin{itemize}
\item $X$ is Postnikov complete (i.e. $\mathrm{lim}_k\propTrunc{X}_k = X$).
\item $\propTrunc{X}_0$ is overtly discrete.
\item For all $x:X$ and $n\geq 1$, we have that $\pi_n(X,x)$ is overtly discrete.
\end{itemize}
Then $X$ is $\I$-local.
\end{proposition}

\begin{proof}
By Postnikov completion it is enough to prove that $\propTrunc{X}_n$ is $\I$-local for all $n:\N$. We proceed by induction on $n:\N$. 

For $n=0$ this is \Cref{delooping-I-local}(i).

For $n=1$ we have a map $\propTrunc{X}_1\to \propTrunc{X}_0$, from the $n=0$ case we just need to prove its fibers are $\I$-local. But they merely are of the form $B\pi_1(X,x)$ for some $x:X$, so we conclude by \Cref{delooping-I-local}(ii). 

Assuming this holds for $n\geq 1$, we have a map $\propTrunc{X}_{n+1}\to \propTrunc{X}_n$. By the induction hypothesis we just need to prove its fibers are $\I$-local to conclude that $\propTrunc{X}_{n+1}$ is $\I$-local. But the fibers are merely of the form $K(\pi_{n+1}(X,x),n+1)$ for some $x:X$, so we conclude by \Cref{delooping-I-local}(iii). 
\end{proof}


