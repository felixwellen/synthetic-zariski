In cohomology, instead of the \v Cech complex, sometimes the alternating and ordered variants are used. There is a standard proof that the alternating and ordered \v Cech cohomology agree with the standard \v Cech cohomology. 
In this note, we sketch out the standard proof (as in for example \cite{stacks-project}{Section 01FG}). We do not claim these results translate to HoTT in general, but we sketch out why our application is still correct. 

\begin{definition}
  Let $q: S \twoheadrightarrow X$ be a \v Cech cover and $A: X \to Ab$. 
  For $n:\N$ and $\alpha: \prod_{x:X} A_x^{(q_x)^n}$, we call $\alpha$ \textbf{alternating} if for any $x:X$, any $n$-tuple $(s_i)_{1\leq i \leq n}:(q_x)^n$ and any $n$-permutation $\sigma: Fin_n = Fin_n$, we have that 
  $$\alpha_x((s_{\sigma(i)})_{1\leq i \leq n}) = sgn(\sigma) \cdot \alpha_x( (s_{i})_{1\leq i \leq n}).$$
  In this case, we write $\mathrm{alt}(\alpha)$. 
\end{definition} 
\begin{remark} 
  To check whether a function is alternating, it's sufficient to do so for $2$-permutations.
\end{remark}
\begin{lemma}
  If we restrict the groups in the \v Cech sequence to the group of alternating functions in $\prod_{x:X} A_x^{(q_x)^n}$, the resulting sequence is still exact. 
\end{lemma} 
\begin{proof}
  
\end{proof}
\begin{definition}
  We call the resulting sequence the alternating Cech sequence.
\end{definition}

\begin{definition} Let $q: S \twoheadrightarrow X$ be a \v Cech cover and $A: X \to Ab$. Let $<$ be a strict order on $S$. 
  We define an $n$-tuple $(s_i)_{1 \leq i \leq n} : (q_x)^n$ to be \textbf{ordered} iff $s_1 < s_1 < \cdots < s_n$. 
  We write $\mathrm{ord}(s)$ if this is the case. 
\end{definition} 
\begin{lemma}
  If we restrict the functions in the \v Cech sequence to only work on ordered fibers, the resulting sequence is still exact.  
\end{lemma}














\begin{definition}
  We define the relations $<,\leq$ on $2^\N$ as follows:
  Given $\alpha,\beta:2^\N$, define for each $n:\N$ the relation 
  expressing that $n$ is the first index where $\alpha,\beta$ differ and 
  $\alpha_n< \beta_n$. So 
  $$\alpha<_n\beta:= 
    (\alpha_n < \beta_n)  \wedge 
    (\forall_{k<n} \alpha_k=\beta_k)
  $$
  Then define $\alpha<\beta:= \exists_{n:\N} \alpha<_n \beta$.
  We also define $\alpha \leq \beta := \neg (\beta < \alpha)$.
\end{definition}
\begin{remark}
  One can check that $<_k$ is decidable for all $k$, 
  hence $<$ is open and $\leq$ is closed. 
  One can also check that both are transitive, and $\leq$ is symmetric. 
  Finally, one can show that if $\alpha<_k\beta$, then $k$ is unique with this proprty and $\neg (\beta<_l \alpha)$ for any $l:\N$.
  It follows in particular that $<$ is antisymmetric. 
\end{remark}
\begin{lemma}
  $\leq$ defined above is linear if and only if LLPO holds. 
\end{lemma}
\begin{proof}
  To show that LLPO implies linearity, let $\alpha,\beta:2^\N$. 
  Define $\gamma:2^\N$ as follows:
  $$
  \gamma_{2k} = 
  \begin{cases}
    1 \text{ if } \alpha <_k \beta_k
    0 \text{ otherwise} 
  \end{cases}, 
  \gamma_{2k+1} = 
  \begin{cases}
    1 \text{ if } \beta <_l \alpha_l
    0 \text{ otherwise} 
  \end{cases}, 
  $$
  Note that by the uniqueness discussed in the above remark, $\gamma$ hits $1$ at most once, hence lives in $\N_\infty\subseteq 2^\N$. 
  Using LLPO, we can deduce that $\gamma$ is $0$ on all evens or all odds. 
  Now note that if $\gamma_{2n} = 0$ for all $n$, then $\beta \leq \alpha$, 
  and if $\gamma_{2n+1} = 0$ for all $n$, then $\alpha \leq \beta$. 

  For the converse direction, let $\gamma:\N_\infty$. 
  Define $\alpha,\beta:2^\N$ with $\alpha_n = \gamma_{2n}, \beta_n = \gamma_{2n+1}$. If $\alpha\leq \beta$, then $\gamma_{2n} = 0$ for all $n:\N$, and if $\beta\leq \alpha$, then $\gamma_{2n+1} = 0$ for all $n:\N$. 
\end{proof}
\begin{remark}
  We conjecture the existence of any linear $(\leq)$-order on Cantor space implies LLPO. 
\end{remark}
\begin{corollary}
  Let $S$ be a Stone space. Then there merely exists a linear order $\leq$ on $S$. 
\end{corollary}
\begin{proof}
  Note that every Stone space can merely be seen as subset of $2^\N$, so it's sufficient to define a linear order or $2^\N$, which we have by the above. 
\end{proof} 




