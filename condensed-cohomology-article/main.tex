% latexmk -pdf -pvc main.tex
\documentclass{../util/zariski}
\newtheorem{goal}[theorem]{Goal}


\title{Cohomology in Synthetic Stone Duality}

\begin{document}

\author{Felix Cherubini, Thierry Coquand, Freek Geerligs and Hugo Moeneclaey}

\maketitle

\begin{abstract}
Peter Scholze and Dustin Clausen \cite{Scholze} introduced light condensed sets, defined as sheaves on the site of light profinite sets. They can be used as an alternative to topological spaces. 
Synthetic Stone duality is an extension of homotopy type theory by four axioms, which was introduced in \cite{synthetic-stone-duality}. In this article, it was proven that $H^1(S,\Z) = 0$ for $S$ a Stone space, that $H^1(X,\Z)$ for $X$ compact Hausdorff can be computed using \v{C}ech cohomology and that $H^1(\mathbb{I},\Z) = 0$ where $\mathbb{I}$ is the unit interval. In this article we extend these results to higher cohomology groups with non-constant countably presented abelian groups as coefficients. Those are synthetic analogues of results from Roy Dyckhoff \cite{dyckhoff76,dyckhoff76-1}.

A key step is proving Barton and Commelin's condensed type theory axioms \cite{TODO}. \rednote{TODO probably should say more actually it doesn't even explain the link between SSD and light condensed sets...}.
\end{abstract}

\tableofcontents

\section{Introduction}
Grothendieck advocated for a functor of points approach to schemes early in his
project of foundation of algebraic geometry (see the introduction of \cite{EGAIV1}).
In this approach, a scheme is defined as a special kind of (covariant) set valued functor
on the category of finitely presented commutative ring. This functor should in particular
be a sheaf w.r.t.\ the Zariski topology. As a typical example, the projective space $\bP^n$
is the functor, which to a ring $A$ associates the set of finitely presented sub-modules of $A^{n+1}$, which are direct factors \cite{Demazure,Eisenbud,Jantzen}.

In the 70s, Anders Kock suggested to use the language of higher-order logic \cite{Church40}
to describe the Zariski topos, the collection of sheaves for the Zariski topology \cite{Kock74,kockreyes}.
This allows for
a more suggestive and geometrical description of schemes, that has now seen as special kind
of types satisfying some properties and a map of schemes in this setting is just any map.
There is in particular a ``generic
local ring'' $R$, which associates to $A$ its underlying set. As described in \cite{kockreyes}
the projective space $\bP^n$ is then the set of lines in $R^{n+1}$.

A natural question is if we can show in this setting that the automorphism group of $\bP^n$
is  $PGL_{n+1}(R)$.
More generally, can we show that any map $\bP^n\rightarrow \bP^m$ is given by $m+1$ homogeneous
polynomials of same degree in $n+1$ variables?
It is possible from this to deduce the corresponding result about $\bP^n$ as defined
as functor of points (but the maps are now {\em natural transformations}) or about $\bP^n$ as
defined as a scheme (but the maps are now {\em maps of schemes}).
(This result, though fundamental, is surprisingly not in \cite{Hartshorne}.)
One goal of this paper is to present such a proof.

In \cite{draft}, we presented an axiomatisation of the Zariski {\em higher topos} \cite{lurie-htt},
using instead of the language of higher-order logic the language of dependent type theory
with univalence \cite{hott}. The first axiom is that we have a local ring $R$. We define
then an affine scheme to be a type of the form $Sp(A) = Hom_{R-alg}(A,R)$ for some finitely presented
$R$-algebra $A$. The second axiom, inspired from the work of Ingo Blechschmidt \cite{ingo-thesis},
states that the evaluation map $A\rightarrow R^{Sp(A)}$ is a bijection. The last axiom states
that each $Sp(A)$ satisfies some form of local choice \cite{draft}. We can then define a notion
of {\em open} proposition, with the corresponding notion of open subset, and define a scheme as a type
covered by a finite number of open subsets that are affine schemes. We can then in particular define
$\bP^n$ as in \cite{kockreyes} and show that it is a scheme.
We show in this setting, dependent type theory with univalence extended with these 3 axioms,
the above result about maps between $\bP^n$ and $\bP^m$ and the result about automorphisms of $\bP^n$.

Interestingly, though these results are
about the Zariski $1$-topos, the proof makes use of types that
are not (homotopy) sets (in the sense of \cite{hott}),
since it proceeds in characterizing $\bP^n\rightarrow\KR$, where $\KR$ is the delooping
(thus a type which is not a set) of the multiplicative group of units of $R$.
More technically, we also use such higher types as an alternative of the technique
of Quillen patching \cite{Quillen,lombardi-quitte,Lam}.



%% Schemes as special kind of sheaf for Zariski topos.

%% Even nicer in a type theoretic framework

%% Anders Kock property of Zariski topos.

%% Zariski topos higher logic

%% Definition of $\bP^n$ as a set of lines in $R^{n+1}$ coincides with the definition
%% of projective as functor of points (Demazure? Eisenbud?)

%% ``Geometric'' definition

%% Meyers, Blechschmidt use of type theory with univalence

%% Axiomatisation of the Zariski (higher) topos

%% A scheme is defined as a type satisfying some property and a map of schemes is {\em any} function
%% between the corresponding types




\section{Scott continuity}
The goal of this section is to prove the axioms of Barton-Commelin's condensed type theory in synthetic Stone duality. Sections \ref{stone-spaces} and \ref{overtly-discrete-types} contain generalities on Stone spaces and overtly discrete types, \Cref{scott-continuity-for-cohomology} contains the restricted versions of Barton-Commelin's axioms \Cref{TODO} needed for our cohomoloogy computations (so it only contains results about Stone spaces). Finally \Cref{barton-commelin} contains the proof of the axioms proper, which essentially means it extends results from \Cref{scott-continuity-for-cohomology} to compact Hausdorff spaces. This section can be skipped by readers only interested in the cohomology results.



\subsection{Stone spaces}
\label{stone-spaces}

We recall generalities about Stone spaces. Most results come from \cite{synthetic-stone-duality}. 

\begin{definition}
A Stone space is a type that is merely a sequential limit of finite types.
\end{definition}

\begin{remark}
This is equivalent to the definition of Stone spaces as spectrum of countably presented booolean algebras given in \cite{synthetic-stone-duality} by Lemmas 3.1 and 3.2 from \cite{synthetic-stone-duality}. We use this alternative definition to emphasis the duality between Stone spaces and overtly discrete types which will be defined in \Cref{overtly-discrete-types} as sequential colimits of finite types.
\end{remark}

\begin{lemma}
A proposition is a Stone space if and only if it is closed.
\end{lemma}

\begin{proof}
Corollary 3.9 of \cite{synthetic-stone-duality}.
\end{proof}

\begin{proposition}
We have the following:

\begin{enumerate}[(i)]
\item Finite types are Stone spaces.
\item Stone spaces are stable under identity types and and sigma types.
\item Stone spaces are stable under sequential limits.
  %\rednote{Not in \cite{synthetic-stone-duality}, is it a version of the quarter-plane lemma?}
%\item Stone spaces have local choice.
 % \rednote{Axiom 3 of \cite{synthetic-stone-duality}}
\end{enumerate}
\end{proposition}

\begin{proof}
As follows:

\begin{enumerate}[(i)]
\item Clear.
\item Lemma 3.10 and Theorem 4.18 of \cite{synthetic-stone-duality}.
\item From duality and the fact that a sequential colimit of a countably presented boolean algebra is countably presented.
%\item This is the local choice axiom.
\end{enumerate}
\end{proof}

Next we give two compactness results for Stone spaces. The first one is used as a definition of compactness in synthetic topology \cite{TODO}, while the second one is closer to the usual topological formulation of compactness.

\begin{lemma}\label{compact-hausdorff-compact}
Assume given $S$ Stone and $U\subset S$ an open subtype. Then $U=S$ is open.
\end{lemma}

\begin{proof}
This is Corollary 4.4 of \cite{synthetic-stone-duality}.
%We show that $\neg(U=S)$ is closed. To do this it is enough to show that:
%\[\neg(U=S) \leftrightarrow \exists (x:S).\neg U_x\]
%but we know that $\exists (x:S).\neg U_x$ is closed and therefore $\neg\neg$-stable, we have that $U_x$ is $\neg\neg$-stable as well and we can conclude from that since we always have:
%\[\neg(\forall(x:S).\ \neg\neg U_x) \to \neg\neg(\exists(x:S).\ \neg U_x)\]
\end{proof}

\begin{lemma}\label{compact-hausforff-countable-cover}
Assume given $S:\Stone$ and $(U_i)_{i:\N}$ open subsets of $S$ such that:
\[U_0\subset U_1 \subset \cdots\]
%If $\forall(x:S).\exists(i:\N). x\in U_i$, then $\exists(i:\N). \forall(x:S).x\in U_i$.
If $\cup_i U_i = S$, there there is a $k$ such that $U_k=S$
\end{lemma}

\begin{proof}
By local choice we have a surjective map $f: T \to S$
such that $\forall(x:T).\Sigma (i:\N). f(x)\in U_i$. By boundedness there is $k:\N$ such that $\forall(x:T). f(x)\in U_k$
and then we conclude by surjectivity.
\end{proof}

Next we give a finite approximation result for surjections between Stone spaces, which will be key in showing the vanishing of the first cohomology groups of Stone spaces.

\begin{lemma}\label{finite-approximation-stone-surjection-cohomology}
Assume given a $S:\Stone$ and $T:S\to \Stone$ such that $\Pi_{x:S}\propTrunc{T(x)}$. Then there exists a family of $T_k: S\to \Stone$ for $k:\N$ with maps $d_k:\Pi_{x:S}T_{k+1}(x)\to T_k(x)$ such that:
\begin{itemize} 
\item For all $x:S$ we have $\mathrm{lim}_kT_k(x) = T(x)$.
\item For all $k:\N$ there exists a section of $\Pi_{x:S}T_k(x)$.
\end{itemize}
\end{lemma}

\begin{proof}
  By assumption, the projection 
  $\pi: \sum_{x:S} T(x) \to S$ is surjective.
  %By Theorem 4.18 of \cite{synthetic-stone-duality}, the domain is Stone.
  By Remark 3.4 of \cite{synthetic-stone-duality}, 
  we can write $\pi$ as limit of surjections 
  $\pi_k : Q_k\twoheadrightarrow S_k$
  between finite sets.
  As taking fibers and limits commute, we have that 
  $T(x)$ is the limit of $\pi_k^{-1}(x|_k)=:T_k(x)$, 
  which is a closed subset of a finite set, hence Stone. 
  %by Theorem 3.11 of \cite{synthetic-stone-duality}. 
  Furthermore, surjections of finite sets merely have sections
  $s_k: S_k \hookrightarrow Q_k$, 
  giving the required sections
  $\lambda x . s_k(x|_k) : \Pi_{x :S} T_k(x)$. 
\end{proof}



\subsection{Overtly discrete types}
\label{overtly-discrete-types}

Here we give generalities on overtly discrete types. There definition is in some sense dual to that of Stone spaces.

\begin{definition}
A type is overtly discrete if it merely is a sequential colimit of finite types.
\end{definition}

\begin{definition}
A type $X$ is countable if there merely exists a decidable subset of $\N$ equal to $X$.
\end{definition}

Next characterisation is helpful to get intuition about overtly discrete types, although we will almost never use it.

\begin{lemma}\label{overtly-discrete-colimit-finite}
Let $X$ be a type, the following are equivalent:
\begin{enumerate}[(i)]
\item $X$ is overtly discrete.
\item $X$ is a quotient of a countable type by an open equivalence relation.
\end{enumerate}
\end{lemma}

\begin{proof}
This is Lemma 2.12 of \cite{synthetic-stone-duality}.
%\begin{itemize}
%\item (i) implies (ii). Assume $X$ is of the form
%\[X  = (\Sigma_\N D)/R\]
%with $D$ decidable and $R$ open. Using choice for $\Sigma_\N D$ we get:
%\[\alpha : (\Sigma_\N D) \to (\Sigma_\N D)\to 2^\N\]
%such that:
%\[R(x,y) = \exists_{k:\N} \alpha(x,y,k) = 1\]
%Then we define:
%\[X_n = (\Sigma_{\mathrm{Fin}(n)} D) / L\]
%\[L(x,y) = \exists_{k:\mathrm{Fin}(n)} \alpha(x,y,k) = 1\]
%We have that $X_n$ is a finite type as it is a decidable quotient of a decidable subset of a finite type. Moreover:
%\[\mathrm{colim}_n X_n = X\]
%as sequential colimits commute with quotients by equivalence relations.
%\item (ii) implies (i). Indeed consider a sequential colimit of:
%\[f_k : \mathrm{Fin}(l_k) \to \mathrm{Fin}(l_{k+1})\]
%Then:
%\[\mathrm{colim}_k \mathrm{Fin}(l_k)  =  \left(\sum_{k:\N} \mathrm{Fin}(l_k)\right) / L\]
%where $L$ is the equivalence relation generated by $(k,x) \sim (k+1,f_k(x))$. But $\sum_{k:\N} \mathrm{Fin}(l_k)$ is countable and the equivalence relation generated by a decidable relation on such a type is open.
%\end{itemize}
\end{proof}

\begin{lemma}
A proposition is overtly discrete if and only if it is open.
\end{lemma}

\begin{proof}
This is Lemma 2.8 of \cite{synthetic-stone-duality}.
\end{proof}

\begin{proposition}\label{overtly-discrete-closure}
We have the following:

\begin{enumerate}[(i)]
\item Finite types are overtly discrete.
\item Overtly discrete types are stable under identity types and sigma types.
\item Overtly discrete types are stable under quotients by equivalence relation with value in overtly discrete types.
\item Overtly discrete type are stable under sequential colimits.
\item Overtly discrete types have local choice.
\end{enumerate}
\end{proposition}

\begin{proof}
As follows:

\begin{enumerate}[(i)]
\item Clear.
\item This is Lemma 2.7 of \cite{synthetic-stone-duality}.
%For stability under identity types, we use that sequential colimits commutes with identity types. 
%For stability under sigma types, sequential colimits commutes with sigma so that by (iii) it is enough to show that overtly discrete types are stable under finite coproduct. But sequential colimits commute with finite coproducts.
\item This follows from \Cref{overtly-discrete-colimit-finite}.

\item This is Lemma 2.6 of \cite{synthetic-stone-duality}.
%Assume given a tower of sequential colimits of finite types. By using dependent choice with \cref{presentation-maps-overtly-discrete} repeatedly, we get a quarter plane of finite types:
%\begin{center}
%\begin{tikzcd}
%F_{0,0}\ar[d]\ar[r] & F_{0,1}\ar[d]\ar[r] & \cdots \\
%F_{1,0}\ar[d]\ar[r] & F_{1,1}\ar[d]\ar[r] & \cdots \\
%\vdots & \vdots & \ddots\\
%\end{tikzcd}
%\end{center}
%which colimit is the colimit of the assumed tower. Then we just use \cref{colimit-quarter-diagonal} to conclude that this colimit is overtly discrete.

\item By \cref{overtly-discrete-colimit-finite}, we have a cover of any overtly discrete type by a countable type, which is an overtly discrete type that has choice.
\end{enumerate}
\end{proof}



\subsection{Scott continuity for Stone spaces}
\label{scott-continuity-for-cohomology}

Here we prove crucial results for our cohomology computations, namely Propositions \ref{scott-continuity-right}, \ref{scott-continuity-left} and \ref{tychonov-dual-stone}. They are then summed up nicely as \Cref{eine-scott-continuity-stone}, which is a fully dependent version of the following: 

\begin{goal}[Non-dependent version]
Given a tower of Stone spaces $(S_k)_{k:\N}$ and a tower of overtly discrete types $(I_j)_{j:\N}$, we have that the canonical map $\mathrm{colim}_{k,j}(S_k\to I_j) \to (\mathrm{lim}_k S_k\to \mathrm{colim}_j I_j)$ is an equivalence.
\end{goal}

First we focus on the case where $S_k$ is constant, and the target is dependent. We need an auxiliary lemma.

\begin{lemma}\label{overtly-discrete-union-open}
Given a tower of overtly discrete types $(I_i)_{i:\N}$, we have that $\mathrm{Im}(I_i)$ is open in $\mathrm{colim}_i I_i$ for any $i:\N$. 
\end{lemma}

\begin{proof}
Writing $I$ for $\mathrm{colim}_i I_i$, for any $y:I$, we have that $y\in \mathrm{Im}(I_i)$ is by definition $\exists(x:I_i).\, x=_{I}y$,
which is the propositional truncation of an overtly discrete type and therefore open.
\end{proof}

Now we show that maps factoring through the image of a layer of a dependent tower of overtly discrete type over a Stone space actually factors through one of the (possibly higher) layer.

\begin{lemma}\label{factorisation-image-true-factorisation}
Assume given $S$ Stone and for each $x:S$ a tower of overtly discrete types $(I_i(x))_{i:\N}$. Assume given $f:\Pi_{x:S}\mathrm{colim}_i I_i(x)$ which factors through $\mathrm{Im}(I_i)$ for some $i:\N$. Then $f$ factors through $I_j$ for some $j\geq i$.
\end{lemma}

\begin{proof}
By local choice there exists a surjective map $p:T\to S$
with $g:\Pi_{x:T} I_i(p(x))$ such that $[g(x)] =_{\mathrm{colim}_i I_i(x)} f(p(x))$ for all $x:T$.
Then we have that:
\[\forall(x,y:T).\, p(x)=p(y) \to [g(x)] =_{\mathrm{colim}_i I_i(x)} [g(y)]\]
so that:
\[\forall(x,y:T).\, p(x)=p(y) \to \exists(i:\N). [g(x)] =_{I_i(x)} [g(y)]\]
Since $\Sigma(x,y:T). p(x)=p(y)$ is Stone we can apply \Cref{compact-hausforff-countable-cover} to get a $j:\N$ such that:
\[\forall(x,y:T).\, p(x)=p(y) \to [g(x)] =_{I_j(x)} [g(y)]\]
which gives a factorisation of $f$ through $I_j$.
\end{proof}

We can conclude in the case where the source is constant.

\begin{proposition}\label{scott-continuity-right}
Assume given $S$ Stone and for each $x:S$ a tower of overtly discrete types $(I_i(x))_{i:\N}$. Then the canonical map $\mathrm{colim}_i (\Pi_{x:S} I_i(x)) \to \Pi_{x:S} \mathrm{colim}_i I_i(x)$ is an equivalence.
\end{proposition}

\begin{proof}
First we check the canonical map is injective. Given $f,g:\Pi_{x:S} I_i(x)$ such that:
\[\forall(x:S).\, [f(x)] =_{\mathrm{colim}_iI_i(x)} [g(x)]\]
Then we have that:
\[\forall(x:S).\, \exists(i:\N).\, f(x) =_{I_i(x)} g(x)\]
so that by \Cref{compact-hausforff-countable-cover} we have that there exist a $j:\N$ such that:
\[\forall(x:S). [f(x)] =_{I_j(x)} [g(x)]\]
which precisely means that $f=g$ in $\mathrm{colim}_i \Pi_{x:S} I_i(x)$.

Now we check that it is surjective. Given a map $f: \Pi_{x:S} \mathrm{colim}_i I_i(x)$ we know that:
\[\forall(x:S).\, \exists(i:\N).\, f(x)\in \mathrm{Im}(I_i(x))\]
but $f(x)\in \mathrm{Im}(I_i(x))$ is open by \Cref{overtly-discrete-union-open} so that by \Cref{compact-hausforff-countable-cover} we have that there exists $j:\N$ such that:
\[\forall(x:S).\, f(x)\in \mathrm{Im}(I_j(x))\]
This precisely means that $f$ factors through $\mathrm{Im}(I_j)$. We conclude by \Cref{factorisation-image-true-factorisation}.
\end{proof}

Now we focus on the case where target is constant. If the target is finite this follows from duality.

\begin{lemma}\label{factorisation-stone-finite}
Given a tower of Stone spaces $(S_k)_{k:\N}$ and $I$ a finite type, we have that the canonical map $\mathrm{colim}_k (S_k\to I)\to \left(\mathrm{lim}_kS_k\to I\right)$ is an equivalence.
\end{lemma}

\begin{proof}
We have that:
\begin{eqnarray}
\mathrm{colim}_k (S_k\to I) &=& \mathrm{colim}_k \Hom(2^I, 2^{S_k})\nonumber\\
&=& \Hom(2^I, \mathrm{colim}_k 2^{S_k})\nonumber\\
&=& \Spec(\mathrm{colim}_k 2^{S_k})\to \Spec(2^I) \nonumber\\
&=& \mathrm{lim}_kS_k\to I\nonumber
\end{eqnarray}
Where the second line comes from the fact that $2^I$ is finitely presented. We omit the verification that this is indeed the canonical map.
\end{proof}

Now we extend to the case where the target is an arbitrary overtly discrete type.

\begin{proposition}\label{scott-continuity-left}
Given a tower of Stone spaces $(S_k)_{k:\N}$ and $I$ an overtly discrete type, we have that the canonical map $\mathrm{colim}_k (S_k\to I)\to \left(\mathrm{lim}_kS_k\to I\right)$ is an equivalence.
\end{proposition}

\begin{proof}
There exists a tower of finite types $(I_i)_{i:\N}$ so that $I = \mathrm{colim}_iI_i$. By the non-dependent version of \Cref{scott-continuity-right} together with \Cref{factorisation-stone-finite} we have that:
\begin{eqnarray}
\mathrm{colim}_k(S_k\to \mathrm{colim}_iI_i) &=& \mathrm{colim}_{k,i}(S_k\to I_i)\nonumber\\
&=& \mathrm{colim}_i(\mathrm{lim}_kS_k\to I_i)\nonumber\\
&=& \mathrm{lim}_kS_k\to \mathrm{colim}_iI_i\nonumber
\end{eqnarray}
We omit the verification that this is indeed the canonical map.
\end{proof}

Now we try to prove the dual of Tychonov for Stone spaces, i.e. that a product of overtly discrete types indexed by a Stone space is itself overtly discrete. We need an auxiliary lemma.

\begin{lemma}\label{tychonov-dual-auxiliary}
Assume given $p:T\to S$ a surjective map between Stone spaces with $I(x)$ an overtly discrete type depending on $x:S$. If $\Pi_{x:T}I(p(x))$ is overtly discrete then so is $\Pi_{x:S}I(x)$.
\end{lemma}

\begin{proof}
Since the map is surjective we have an embedding $\Pi_{x:S}I(x)\subset \Pi_{x:T}I(p(x))$. But the fiber over $g:\Pi_{x:T}I(p(x))$ is:
\[\forall (x,y:T).\, p(x)=p(y) \to g(x)=g(y)\]
which is open by \Cref{compact-hausdorff-compact}. Therefore $\Pi_{x:S}I(x)$ is an open subtype of an overtly discrete type, so it is itself overtly discrete.
\end{proof}

\begin{proposition}[Tychonov's Dual]\label{tychonov-dual-stone}
Assume given $S$ Stone with $I(x)$ an overtly discrete type depending on $x:S$. Then $\Pi_{x:S}I(x)$ is overtly discrete.
\end{proposition}

\begin{proof}
By local choice and \Cref{tychonov-dual-auxiliary} we can assume given tower of finite types $(\mathrm{Fin}(l_{k,x}))_{k:\N}$
such that $I(x) = \mathrm{colim}_k\, \mathrm{Fin}(l_{k,x})$ for all $x:S$.
Then using \Cref{scott-continuity-right} and the fact that overtly discrete types are stable under sequential colimits, it is enough to prove that $\Pi_{x:S} \mathrm{Fin}(l_{k,x})$ is overtly discrete. 

Using boundedness we have that there exists $n:\N$ such that $\Pi_{x:S} \mathrm{Fin}(l_{k,x}) = \Pi_{i<n}(S_i\to \mathrm{Fin}(i))$ where $S_i = \{x:S\ |\ l_{k,x} = i\}$. Then we conclude by \Cref{factorisation-stone-finite} and the fact that overtly discrete types are closed under finite products.
\end{proof}

So far we have two versions of Scott continuity (Propositions \ref{scott-continuity-right} and \ref{scott-continuity-left}), neither clearly implying the other. In the rest of this section we give a common generalisation, suggested to us by Reid Barton. It is not used for the cohomology results. We start by an auxiliary definition.

\begin{definition}\label{category-scott-continuity}
We have a category ${\mathcal C}_\Stone$ defined by:

\begin{itemize}
\item An object consists of $S:\Stone$ and $I:S\to \ODisc$.
\item A morphism from $(S,I)$ to $(T,J)$ consists of $p:T\to S$ with $q:\Pi_{x:T}(I(p(x))\to J(x))$.
\end{itemize}
\end{definition}

The arrows are defined such that we have a covariant functor $\Pi:{\mathcal C}\to \ODisc$. By \Cref{tychonov-dual-stone} it indeed takes value in $\ODisc$. Now we prove the general version of Scott continuity for Stone spaces, which comes from the two given versions plus some reasoning on colimits.

\begin{theorem}[Scott continuity for Stone spaces]\label{eine-scott-continuity-stone}
The functor $\Pi : {\mathcal C}_\Stone \to \ODisc$ commutes with sequential colimits.
\end{theorem}

\begin{proof}
We assume given a tower in ${\mathcal C}_\Stone$, i.e. we assume a tower:
\[S_0 \overset{p_0}{\leftarrow} S_1 \overset{p_0}{\leftarrow} S_2 \overset{p_2}{\leftarrow}\cdots \]
in $\Stone$ with for all $k:\N$ a dependent type $I_k:S_k\to \ODisc$ and $q_k : \Pi_{x:S_{k+1}}I_k(p_k(x))\to I_{k+1}(x)$.
We want to prove that the canonical map $\mathrm{colim}_i (\Pi_{x:S_i}I_i(x)) \to \Pi_{x:\mathrm{lim}_kS_k}\mathrm{colim}_i I_i(x_i)$ is an equivalence. 

By general reasoning on colimits and Propositions \ref{scott-continuity-right} and \ref{scott-continuity-left}, we get the following string of equivalences:
\begin{eqnarray}
\mathrm{colim}_i (\Pi_{x:S_i}I_i(x)) &=& \mathrm{colim}_{i,k\geq i} \Pi_{x:S_k} I_i(x_i)\nonumber\\
&=& \mathrm{colim}_{i,k\geq i} \Pi_{x:S_i}\Pi_{y:S_k, [y]=x} \to I_i(x) \nonumber\\
&=& \mathrm{colim}_i\Pi_{x:S_i} \mathrm{colim}_{k\geq i} \Pi_{y:S_k, [y]=x} I_i(x)\nonumber\\
&=& \mathrm{colim}_i\Pi_{x:S_i} \Pi_{y:\mathrm{lim}_{k\geq i} S_k,[y]=x} I_i(x)\nonumber\\
&=& \mathrm{colim}_i\Pi_{x:\mathrm{lim}_{k} S_k} I_i([x])\nonumber\\
&=& \Pi_{x:\mathrm{lim}_{k} S_k} \mathrm{colim}_i I_i(x)\nonumber
\end{eqnarray}
%We know that the source is equivalent to:
%\[\mathrm{colim}_{i,k\geq i} \Pi_{x:S_k} I_i(x_i)\]
%which is the same as:
%\[\mathrm{colim}_{i,k\geq i} \Pi_{x:S_i}\Pi_{y:S_k} y_k=x \to I_i(x)\]
%which by \Cref{scott-continuity-right} is in turn equal to:
%\[\mathrm{colim}_i\Pi_{x:S_i} \mathrm{colim}_{k\geq i} \Pi_{y:S_k} y_i = x \to I_i(x)\]
%which by \Cref{scott-continuity-left} is equal to:
%\[\mathrm{colim}_i\Pi_{x:S_i} \Pi_{y:\mathrm{lim}_{k\geq i} S_k} y_i=x \to I_i(x)\]
%which is immediately seen as:
%\[\mathrm{colim}_i\Pi_{x:\mathrm{lim}_{k} S_k} I_i(x_i)\]
%which by \Cref{scott-continuity-right} is equal to:
%\[\Pi_{x:\mathrm{lim}_{k} S_k} \mathrm{colim}_i I_i(x_k)\]
We omit the verification that this is indeed the canonical map.
\end{proof}

\begin{remark}
Scott continuity has the immediate consequence that given a sequence of Stone spaces $(S_k)_{k:\N}$ and $I$ overtly discrete, for any map in $\left(\Pi_{k:\N} S_k\right) \to I$ merely factors through $\Pi_{k<n} S_k$ for some $n:\N$. This justifies the name Scott continuity.
\end{remark}


\subsection{Barton-Commelin axioms}
\label{barton-commelin}

We already have proven all the results needed for our cohomology results. In this section we prove the additional results needed to get all of Barton and Commelin's condensed type theory axioms. \Cref{overtly-discrete-closure} already give all the axioms assumed about overtly discrete types. Stone spaces are not closed under quotients so we have no hope of getting the same result for them, so we consider compact Hausdorff spaces which are precisely the quotients of Stone spaces by closed equivalence relations. We recall the definition.

\begin{definition}
A type $X$ is a compact Hausdorff space of its identity types are closed and there exists a Stone space $X$ with a surjection $S\to X$.
\end{definition}

We need an auxiliary lemma to prove compact Hausdorff spaces closed under sequential limits.

\begin{lemma}\label{sequential-limit-Hausdorff}
Assume given a tower $(C_k)_{k:\N}$ of compact Hausdorff spaces. Then there exists a tower $(S_k)_{k:\N}$ of Stone spaces with maps:

\begin{center}
\begin{tikzcd}
\cdots \ar[r]& S_1\ar[r]\ar[d] & S_0 \ar[d]\\
\cdots \ar[r] & C_1\ar[r] & C_0 \\
\end{tikzcd}
\end{center}

such that the map:
\[S_0\to C_0\]
is surjective and for all $n:\N$ the induced map:
\[S_{n+1} \to C_{n+1}\times_{C_n} S_n\]
is surjective.

This implies that the induced map:
\[\lim_kS_k \to \lim_k C_k\]
is surjective.
\end{lemma}

\begin{proof}
By definition of a Compact Hausdorff type, we can merely find a Stone space $S_0$ with a surjection:
\[S_0\to C_0\]
Using dependent choice, it is enough to show that we can merely extend such a tower $(S_k)_{k\leq n}$ to $(S_k)_{k\leq n+1}$.

We choose a Stone space $T$ and a surjection:
\[T \to C_{n+1} \]
Then we define $S_{n+1}$ by the following pullback square:
\begin{center}
\begin{tikzcd}
S_{n+1}\ar[d]\ar[r] & C_{n+1}\times_{C_n} S_n\ar[r]  \ar[d]& S_n\ar[d]\\
T \ar[r] & C_{n+1}\ar[r]  & C_n
\end{tikzcd}
\end{center}
And we see the map:
\[S_{n+1} \to C_{n+1}\times_{C_n} S_n\] 
surjective as it is a pullback of the map:
\[T\to C_{n+1}\]

This implies that the map:
\[\lim_kS_k \to \lim_k C_k\]
is surjective by dependent choice.
\end{proof}

\begin{theorem}
We have the following:
\begin{enumerate}[(i)]
\item Finite types are compact Hausdorff spaces.
\item Compact Hausdorff spaces are stable under identity types and and sigma types.
  \rednote{Consequence of Corollary 3.9, and Lemma 4.11 of \cite{synthetic-stone-duality}}
\item Compact Hausdorff spaces are stable under quotients by equivalence relation with value in compact Hausdorff spaces.
  \rednote{One needs that compact Hausdorff propositions are closed, which follows from Lemma 4.17. }
\item Compact Hausdorff spaces are stable under sequential limits.
\item Compact Hausdorff space have local choice.
\end{enumerate}
\end{theorem}

\begin{proof}
We proceed as follows:
\begin{enumerate}[(i)]
\item For identity types this is because closed propositions are compact Hausdorff. For sigma this is \rednote{TODO reference}

\item Clear from the definition.

\item We use \cref{sequential-limit-Hausdorff} and the fact that closed proposition are stable under sequential limits.

\item From the fact that Stone spaces have local choice.
\end{enumerate}
\end{proof}

Next we need to prove Tychonov and its dual for compact Hausdorff spaces.

\begin{proposition}[Tychonov]
Assume given $I$ overtly discrete and $C(i)$ compact Hausdorff depending on $i:I$. Then:
\[\Pi_{i:I}C_i\]
is compact Hausdorff.
\end{proposition}

\begin{proof}
We can assume $I = \mathrm{colim}_{k:\N}\, I_k$ with $I_k$ finite. Then:
\[\Pi_{i:I}C_i = \Pi_{i:\mathrm{colim}_{k:\N}\, I_k} C_i = \lim_{k:\N}\, \Pi_{i:I_k}C(\iota_k(i))\]
and we can conclude using that compact Hausdorff spaces are stable under sequential limits and finite products.
\end{proof}

\begin{proposition}[Tychonov's dual]
Assume given $C$ compact Hausdorff and $I(x)$ overtly discrete depending on $x:C$. Then:
\[\Pi_{x:C}I(x)\]
is overtly discrete.
\end{proposition}

\begin{proof}
We consider a surjection $p:S\to C$ with $S$ Stone. Then
\[\Pi_{x:C}I(x) = \{f:\Pi_{x:S}I(p(x))\ |\ \forall(x,y:S).\, p(x)=p(y)\to f(x)=f(y)\}\]
But $\forall(x,y:S).\, p(x)=p(y)\to f(x)=f(y)$ is open by \Cref{compact-hausdorff-compact} and $prod_{x:S}I(p(x))$ is overtly discrete by \Cref{tychonov-dual-stone}, so we can conclude.
\end{proof}

Finally we extend Scott continuity to compact Hausdorff spaces. We write ${\mathcal C}_\CHaus$ for ${\mathcal C}_\Stone$ as given in \Cref{category-scott-continuity} with Stone spaces replaced by compact Hausdroff spaces. As for Tychonov's dual, this just involves reasoning on quotients.

\begin{theorem}[Scott Continuity]
The functor:
\[\Pi : {\mathcal C}_\CHaus \to \ODisc\]
commutes with sequential colimits.
\end{theorem}

\begin{proof}
Assume given $(C_i,I_i)_{i:\N}$ a tower in $\mathcal C$. By \Cref{sequential-limit-Hausdorff} we can consider $(S_i)_{i:\N}$ a tower of Stone spaces with $p_k:S_k\to C_k$ giving a level-wise surjection to the tower $(C_i)_{i:\N}$. We write $S=\mathrm{lim}_iS_i$, $C=\mathrm{lim}_iC_i$, for $x:C$ we have $I(x) = \mathrm{colim}_iI_i(x_i)$ and finally $p =\mathrm{lim}_ip_i$. Then the canonical map:
\[\mathrm{colim}_i\, (\Pi_{x:C_i}I_i(x) ) \to \Pi_{x:C}I(x)\]
is equal to the canonical map:

\[ \left\{ \iota_i(f_i) : \mathrm{colim}_i\, \Pi_{x:S_i}\, I_i(p_i(x))\ |\ \mathrm{colim}_{j>i}\ Q_j(f_i)\right\} \to \left\{f:\Pi_{x:S}\, I(p(x))\ |\ Q(f)\right\} \]
where we defined:
\begin{eqnarray}
 Q_j(f_i) &=& \Pi_{(x,y:S_j,p_j(x)=p_j(y))}\, \iota_j(f_i(\pi_i(x)))=\iota_j(f_i(\pi_i(y)))\nonumber\\
Q(f) &=& \Pi_{(x,y:S,p(x)=p(y))}\, f(x)=f(y)\nonumber
\end{eqnarray}

Then by \Cref{eine-scott-continuity-stone} we have that the canonical map
\[\epsilon : \mathrm{colim}_i \Pi_{x:S_i}\,I_i(p_i(x)) \to \Pi_{x:S}\, I(p(x))\]
is an equivalence, by \Cref{eine-scott-continuity-stone} again we have that the canonical map
\[\mathrm{colim}_{j>i}Q_j(f_i) \to Q(\epsilon(\iota_i(f_i)))\]
is an equivalence so we can conclude.
\end{proof}





\section{Cohomology}
In non-synthetic algebraic geometry,
the structure sheaf~$\mathcal{O}_X$ is part of the data constituting a scheme~$X$.
In our internal setting,
the scheme $X$ is just a set without any additional data,
but when we want to consider the structure sheaf as an object in its own right,
then we can represent it by the trivial bundle
that assings to every point $x : X$ the set $R$.
Indeed, for an affine scheme $X = \Spec A$,
taking the sections of this bundle over a basic open $D(f) \subseteq X$
\[ (\prod_{x : D(f)} R) = (D(f) \to R) = A[f^{-1}] \]
yields the localizations of the ring $A$
expected from the structure sheaf $\mathcal{O}_X$.
More generally,
instead of sheaves of abelian groups, $\mathcal{O}_X$-modules, etc.,
we will consider bundels of abelian groups, $R$-modules, etc.,
in the form of maps from $X$ to the respective type of algebraic structures.

\subsection{Quasi-coherent bundles}

This subsection is still experimental.

Sometimes we want to ``apply'' a bundle to a subtype,
like sheaves can be evaluated on open subspaces
and introduce the common notation ``$M(U)$'' for that below.
It is, however, not justified to expect, that this application
and the corresponding theory of ``sheaves'' is ``the same'' as the external one,
since the definition below, uses the internal hom ``$\prod$''
-- where the corresponding external construction, would be the set of continuous sections of a bundle.

\begin{definition}
  \index{$M(U)$}
  Let $X$ be a type and $M:X\to \Mod{R}$ a dependent module.
  Let $U\subseteq X$ be any subtype.
  \begin{enumerate}[(a)]
  \item We write:
    \[
      M(U)\colonequiv \prod_{x:U}M_x
      \rlap{.}
    \]
  \item With pointwise structure, $U\to R$ is an $R$-algebra
    and $M(U)$ is a $(U\to R)$-module.
  \end{enumerate}
\end{definition}

Somewhat surprisingly, localization of modules $M(U)$
can be done pointwise:

\begin{lemma}[using \axiomref{loc}, \axiomref{sqc}, \axiomref{Z-choice}]%
  \label{module-bundle-localization-pointwise}
  Let $X$ be a scheme and $M:X\to \Mod{R}$ a dependent module.
  Let $U=\Spec A\subseteq X$ be open affine.
  Let $f:A$.
  \begin{enumerate}[(a)]
  \item There is a morphism
    \[
      M(U)_f\to \prod_{x:U}(M_x)_{f(x)}
      \rlap{.}
    \]
  \item Let $g,h:M(U)_f$. Then $g=h$ if and only if
    \[
      \prod_{x:U}g(x)=_{(M_x)_{f(x)}}h(x)
      \rlap{.}
    \]
  \item The morphism in (a) is an equivalence, i.e.
    \[
      M(U)_f=\prod_{x:U}(M_x)_{f(x)}
      \rlap{.}
    \]
  \end{enumerate}
\end{lemma}

\begin{proof}
  \begin{enumerate}[(a)]
  \item We have to show, that the map
    \[
      \frac{m}{f^k}\mapsto\left(x\mapsto \frac{m(x)}{f(x)^k}\right)
    \]
    is well-defined. So let $\frac{m}{f^k}=\frac{m'}{f^{k'}}$,
    i.e. let there be an $l:\N$ such that $f^l(mf^{k'}-m'f^k)=0$.
    But then we can choose the same $l:\N$ for each $x:U$
    and apply the equation to each $x:U$.
  \item The forward direction was treated in (a).
    So let $g,h:M(U)_f$ such that $p:\prod_{x:U}g(x)=_{(M_x)_{f(x)}}h(x)$.
    Let $m_g,m_h:\prod_{x:U} M_x$ and $k_g,k_h:\N$ such that
    \[
      g=\frac{m_g}{f^{k_g}} \quad\text{and}\quad h=\frac{m_h}{f^{k_h}}
      \rlap{.}
    \]
    From $p$ we know $\prod_{x:U}\exists_{k_x:\N}f(x)^{k_x}(m_g(x)f(x)^{k_h}-m_h(x)f(x)^{k_g})=0$.
    By \cref{strengthened-boundedness},
    we find one $k : \N$ with
    \[
      \prod_{x:U}f(x)^{k}(m_g(x)f(x)^{k_h}-m_h(x)f(x)^{k_g})=0
    \]
    --- which shows $g=h$.
  \item The map in (a) is injective by (b);
    it remains to show that it is surjective.
    So let $\varphi:\prod_{x:U}(M_x)_{f(x)}$ and
    note that
    \[
      \prod_{x:U}
      \exists_{k_x:\N,m_x:M_x}
      \varphi(x)=\frac{m_x}{f(x)^{k_x}}
      \rlap{.}
    \]
    By \cref{strengthened-boundedness} and \axiomref{Z-choice},
    we get $k:\N$, coprime $a_1,\dots,a_l:A$ and $m_i:(x : D(a_i))\to M_x$
    such that for each $i$ and $x:D(a_i)$ we have
    \[
      \varphi(x)=\frac{m_i(x)}{f(x)^{k}}
      \rlap{.}
    \]
    The problem is now to construct a global $m:(x:U)\to M_x$ from the $m_i$.
    We have
    \[
        \prod_{x:D(a_ia_j)}\frac{m_i(x)}{f(x)^k}=\varphi(x)=\frac{m_j(x)}{f(x)^k}
    \]
    meaning there is pointwise an exponent $t_x:\N$,
    such that $f(x)^{t_x}m_i(x)=f(x)^{t_x}m_j(x)$.
    By \cref{strengthened-boundedness},
    we can find a single $t:\N$ with this property and define
    \[
      \tilde{m}_i(x) \colonequiv f(x)^t m_i(x)
      \rlap{.}
    \]
    Then we have $\tilde{m}_i(x)=\tilde{m}_j(x)$ on all intersections $D(a_i)\cap D(a_j)$,
    which is what we need to get a global $m:(x:U)\to M_x$ from \cref{kraus-glueing}.
    Since $\varphi(x)=\frac{f(x)^t m_i(x)}{f(x)^{t+k}}=\frac{\tilde{m}_i(x)}{f(x)^{t+k}}$
    for all $i$ and $x : D(a_i)$,
    we have found a preimage of $\varphi$ in $M(U)_f$.
  \end{enumerate}
\end{proof}

We will need the following algebraic lemma:

\begin{lemma}%
  \label{localization-to-module-if-non-zero}
  Let $M$ be an $R$-module and $f:R$,
  then there is an $R$-linear map
  \[
    M_f\to M^{D(f)}
    \rlap{.}
  \]
\end{lemma}

\begin{proof}
  Let $x\equiv \frac{m}{f^k}:M_f$ and $p:D(f)$.
  Then $f$ is invertible, so we have
  \[
    x\equiv \frac{m}{f^k}=\frac{f^{-k}m}{1}
  \]
  and mapping $x$ to $f^{-k}m$ is an $R$-linear map.
  
\end{proof}

\begin{lemma}[using \axiomref{sqc}, \axiomref{loc}, \axiomref{Z-choice}]%
  \label{localization-to-restriction}                    
  Let $X$ be a scheme, $M:X\to\Mod{R}$, $U=\Spec A\subseteq X$ open and $f:A$.
  Then there is an $R$-linear map
  \[
    M(U)_f \to M(D(f)) 
    \rlap{.}
  \]
\end{lemma}

\begin{proof}
  Combining \cref{module-bundle-localization-pointwise}
  and pointwise application of \cref{localization-to-module-if-non-zero} we get
  \[
    M(U)_f=\left(\prod_{x:U}(M_x)_{f(x)}\right)\to \left(\prod_{x:U}(M_x)^{D(f(x))}\right)
    =\left(\prod_{x:D(f)}M_x\right)
    =M(D(f))
  \]
\end{proof}

The following is an experimental definition,
which might be suitable
to mimic the external notion of quasi-coherent $\mathcal O_X$-module sheaves.

\begin{definition}%
  \label{quasi-coherent-bundle}
  Let $X$ be a scheme.
  A dependent module $M:X\to \Mod{R}$ is \notion{quasi-coherent},
  if for all $x:X$ and $f:R$,
  the canonical map from \cref{localization-to-module-if-non-zero} is an equivalence:
  \[
    (M_x)_f\simeq M_x^{D(f)}
    \rlap{.}
  \]
\end{definition}

An immediate consequence is, that
quasi coherent dependent modules have
the property that ``restricting is the same as localizing'':

\begin{lemma}[using \axiomref{sqc}, \axiomref{loc}, \axiomref{Z-choice}]
  Let $X$ be a scheme and $M:X\to \Mod{R}$ quasi-coherent,
  then for all open affine $U=\Spec A\subseteq X$ and $f:A$
  the canonical morphism
  \[
    M(U)_f\to M(D(f))
  \]
  is an equivalence.
\end{lemma}

\begin{proof}
  By construction of the canonical map from \cref{localization-to-restriction}.
\end{proof}

Let us look at an example.

\begin{proposition}
  \label{fp-algebra-bundle-is-quasi-coherent}
  Let $X$ be a scheme and $C:X\to \Alg{R}_{fp}$.
  Then $C$, as a bundle of $R$-modules, is quasi coherent.
\end{proposition}

\begin{proof}
  Then for any $f:R$ and $x:X$, using \cref{algebra-valued-functions-on-affine}, we have
  \[
    (C_x)_f=C_x\otimes_R R_f=(\Spec R_f \to C_x)=(D(f)\to C_x)={C_x}^{D(f)}
    \rlap{.}
  \]
\end{proof}

\begin{proposition}[using \axiomref{loc}, \axiomref{sqc}, \axiomref{Z-choice}]
  Not every $R$-module is quasi-coherent
  in the sense of \cref{quasi-coherent-bundle}.
\end{proposition}

\begin{proof}
  We construct a family of $R$-modules,
  parametrized by the elements of $R$,
  and deduce a contradiction from the assumption that
  all modules of this family are quasi-coherent.

  Given an element $f : R$,
  the $R$-module we want to consider is
  the countable product
  \[ M(f) \colonequiv \prod_{n : \N} R/(f^n) \rlap{.} \]
  If $f \neq 0$ then $M(f) = 0$
  (using \cref{non-zero-invertible}).
  This implies that the $R$-module $M(f)^{f \neq 0}$
  is trivial:
  any function $f \neq 0 \to M(f)$ can only assign the value $0$
  to any of the at most one witnesses of $f \neq 0$.
  If $M(f)$ is quasi-coherent,
  then this means that $M(f)_f$ is also trivial.
  Noting that
  $M(f)$ is not only an $R$-module
  but even an $R$-algebra in a natural way,
  we have
  \begin{align*}
    M(f)_f = 0
    &\;\Leftrightarrow\;
    \exists k : \N.\; \text{$f^k = 0$ in $M(f)$} \\
    &\;\Leftrightarrow\;
    \exists k : \N.\; \forall n : \N.\; f^k \in (f^n) \subseteq R \\
    &\;\Leftrightarrow\;
    \exists k : \N.\; f^k \in (f^{k + 1}) \subseteq R
    \rlap{.}
  \end{align*}

  In summary,
  if the module $M(f)$ is quasi-coherent
  for every $f : R$,
  then the ring $R$ is zero-dimensional
  in the sense of \cref{zero-dimensional-ring}.
  But this is not the case,
  as we saw in \cref{R-not-zero-dimensional}.
\end{proof}

\begin{lemma}[using \axiomref{sqc}, \axiomref{loc}, \axiomref{Z-choice}]%
  \label{weakly-quasi-coherent-pi}
  Let $X$ be an affine scheme and $M_x$ a weakly quasi-coherent $R$-module for any $x:X$,
  then
  \[
    \prod_{x:X}M_x
  \]
  is weakly quasi-coherent.
\end{lemma}

\begin{proof}
  TODO
\end{proof}

Quasi-coherent dependent modules turn out to have very good properties,
which are to be expected from what is known about their external counterparts.
We will show below, that quasi coherence is preserved by the following constructions:

\begin{definition}
  \label{pullback-push-forward}
  Let $X,Y$ be types and $f:X\to Y$ be a map.
  \begin{enumerate}[(a)]
  \item \index{$f^*M$} For any dependent module $N:Y\to\Mod{R}$,
    the \notion{pullback} or \notion{inverse image} is the dependent module
    \[
      f^*N\colonequiv (x:X) \mapsto M_{f(x)}\rlap{.}
    \]
  \item \index{$f_*M$} For any dependent module $M:X\to\Mod{R}$,
    the \notion{push-forward} or \notion{direct image} is the dependent module
    \[
      f_*M\colonequiv (y:Y) \mapsto \prod_{x:\fib_f(y)}M_{\pi_1(x)}\rlap{.}
    \]
  \end{enumerate}
\end{definition}

\begin{theorem}[using \axiomref{sqc}, \axiomref{loc}, \axiomref{Z-choice}]%
  \label{pullback-push-forward-qcoh}
  Let $X,Y$ be schemes and $f:X\to Y$ be a map.
  \begin{enumerate}[(a)]
  \item For any quasi-coherent dependent module $N:Y\to\Mod{R}$,
    the inverse image $f^*N$ is quasi-coherent.
  \item For any dependent module $M:X\to\Mod{R}$,
    the direct image $f_*M$ is quasi-coherent.
  \end{enumerate}
\end{theorem}

\begin{proof}
  \begin{enumerate}[(a)]
  \item There is nothing to do, when we use the pointwise definition of quasi-coherence. 
  \item TODO, Ideas:

    Show that the dependent product of modules is a module.
    Show that this product preserves qcoh, if the index type is a scheme.
    Use that the fiber of a scheme morphism is a scheme.
  \end{enumerate}
\end{proof}

\subsection{Finitely presented bundles}

We now investigate the relationship between bundles of $R$-modules on $X = \Spec A$
and $A$-modules.

\begin{proposition}
  Let $A$ be a finitely presented $R$-algebra.
  There is an adjunction
  \[ \begin{tikzcd}[row sep=tiny]
    M \ar[r, mapsto] & {(M \otimes x)}_{x : \Spec A} \\
    \Mod{A} \ar[r, shift left=2] \ar[r, phantom, "\rotatebox{90}{$\vdash$}"] &
    \Mod{R}^{\Spec A} \ar[l, shift left=2] \\
    \prod_{x : \Spec A} N_x & N \ar[l, mapsto]
  \end{tikzcd} \]
  between the category of $A$-modules
  and the category of bundles of $R$-modules on $\Spec A$.
\end{proposition}

\begin{theorem}%
  \label{fp-module}
  Let $X=\Spec(A)$ be affine and
  let a bundle of finitely presented $R$-modules $M : X\to \fpMod{R}$ be given.
  Then the $A$-module
  \[ \tilde{M}\coloneqq\prod_{x:X}M_x \]
  is finitely presented and for any $x:X$ the $R$-module $\tilde{M}\otimes_A R$ is $M_x$.
  Under this correspondence, localizing $\tilde{M}$ at $f:A$ corresponds to restricting $M$ to $D(f)$.
\end{theorem}

\subsection{Cohomology on affine schemes}

\begin{definition}%
  \label{torsor}
  Let $X$ be a type and $A:X\to \AbGroup$ a map to the type of abelian groups.
  For $x:X$ let $T_x$ be a set with an $A_x$ action.
  \begin{enumerate}[(a)]
  \item $T$ is an \notion{$A$-pseudotorsor}, if the action is free and transitive for all $x:X$.
  \item $T$ is an \notion{$A$-torsor}, if it is an $A$-pseudotorsor and
    \[ \prod_{x:X} \| T_x \| \rlap{.}\]
  \item We write $\Tors{A}(X)$ for the type of $A$-torsors on $X$.
  \end{enumerate}
\end{definition}

Torsors on a point are a concrete implementaion of first deloopings:

\begin{definition}
  \label{delooping}
  Let $n:\N$.
  A $n$-th \notion{delooping}\index{$K(A,n)$} of an abelian group $A$,
  is a pointed, $(n-1)$-connected, $n$-truncated type $K(A,n)$,
  such that $\Omega^nK(A,n)=_{\AbGroup}A$.
\end{definition}

For any abelian group and any $n$, a delooping $K(A,n)$ exists by \cite{licata-finster}.
Deloopings can be used to represent cohomology groups by mapping spaces.
This is usually done in homotopy type theory to study higher inductive types, such as spheres and CW-complexes,
but the same approach works for internally representing sheaf cohomology,
which is the intent of the following definition:

\begin{definition}
  \label{cohomology}
  Let $X$ be a type and $\mathcal F:X\to\AbGroup$ a dependent abelian group.
  The $k$-th cohomology group of $X$ with coefficients in $\mathcal F$ is
  \[
    H^k(X,\mathcal F)\colonequiv \left\|\prod_{x:X}K(\mathcal F,k)\right\|_0\rlap{.}
  \]
\end{definition}


The following is an explicit formulation of the fact, that the Čech-Complex for an
$\mathcal{O}_X$-module sheaf on $X=\Spec(A)$ given by an $A$-module $M$ is exact in degree 1.
\begin{lemma}%
  \label{H1-algebra}
  Let $M$ be a module over a commutative ring $A$, $F_1,\dots,F_l$ a coprime system on $A$
  and for $i,j\in\{1,\dots,l\}$, let $s_{ij} : F_i^{-1} F_j^{-1} M$ such that:
  \[ s_{jk}-s_{ik}+s_{ij}=0 \rlap{.}\]
  Then there are $u_i:F_i^{-1}M$ such that $s_{ij}=u_j - u_i$.
\end{lemma}

\begin{proof}
  Let $s_{ij}=\frac{m_{ij}}{f_i f_j}$ with $m_{ij}:M$, $f_i:F_i$ and $f_j:F_j$ such that:
  \[ f_i\cdot m_{jk}-f_j\cdot m_{ik}+f_k\cdot m_{ij}=0 \rlap{.}\]
  Let $r_i$ such that $\sum r_i f_i =1$.
  Then for
  \[ u_i \coloneqq -\sum_{k=1}^l\frac{r_k}{f_i}m_{ik} \]
  we have:
  \begin{align*}
      u_j-u_i &= -\sum_{k=1}^l\frac{r_k}{f_j}m_{jk} + \sum_{k=1}^l\frac{r_k}{f_i}m_{ik} \\
              &= -\sum_{k=1}^l\frac{r_k}{f_j f_i}f_i m_{jk} + \sum_{k=1}^l\frac{r_k}{f_i f_j} f_j m_{ik} \\
              &= \sum_{k=1}^l\frac{r_k}{f_j f_i}(-f_i m_{jk} + f_j m_{ik}) \\
              &= \sum_{k=1}^l\frac{r_k}{f_j f_i}f_k m_{ij} \\
              &= \frac{m_{ij}}{f_i f_j}
  \end{align*}
  \ %
\end{proof}

\begin{theorem}[using \axiomref{Z-choice}]%
  \label{H1-fp-module-affine-trivial}
  For any affine scheme $X=\Spec(A)$ and coefficients $M: X\to \fpMod{R}$, we have
  \[ H^1(X,M)=0 \rlap{.} \]
\end{theorem}
\begin{proof}
  We need to show, that any $M$-torsor $T$ on $X$ is merely equal to the trivial torsor $M$,
  or equivalently show the existence of a section of $T$.
  We have
  \[ \prod_{x:X}\| T_x \|\]
  and therefore, by (\axiomref{Z-choice}),
  there merely are $f_1,\dots,f_l:A$,
  such that the $U_i\coloneqq \Spec(A_{f_i})$ cover $X$ and
  there are local sections
  \[ s_i:\prod_{x:U_i}T_x\]
  of $T$. Our goal is to construct a matching family from the $s_i$.
  On intersections, let $t_{ij}\coloneqq s_i-s_j$ be the difference, so $t_{ij}:(x : U_i\cap U_j) \to M_x$.
  By \cref{fp-module} equivalently, we have $t_{ij}:\tilde{M}_{f_i f_j}$.
  Since the $t_{ij}$ were defined as differences,
  the condition in \cref{H1-algebra} is satisfied and we get
  $u_i:\tilde{M}_{f_i}$, such that $t_{ij}=u_i-u_j$.
  So we merely have a matching family $\tilde{s}_i\coloneqq s_i-u_i$ and therefore, using Lemma \ref{kraus-glueing} merely a section of $T$.
\end{proof}

A similar result is provable for $H^2(X,M)$ and we expect that $H^n(X,M)$ holds, at least for any external $n$.

\subsection{Čech-Cohomology}

In this section, let $X$ be a type, $U_1,\dots,U_n\subseteq X$ open subtypes that cover $X$
and $\mathcal F:X\to \AbGroup$ a dependent abelian group on $X$.
We start by repeating the classical definition of Chech-Cohomology groups for a given cover.

\begin{definition}%
  \label{chech-complex}
  \begin{enumerate}[(a)]
  \item \index{$\mathcal F(U)$} For open $U\subseteq X$, we use the notation
    \[
      \mathcal F(U)\colonequiv \prod_{x:U}\mathcal F_x\rlap{.}
    \]
  \item For $s:\mathcal F(U)$ and open $V\subseteq U$ we use the notation $s\colonequiv s_{|V} \colonequiv (x:V)\mapsto s_x$.
  \item \index{$U_{i_1\dots i_l}$}For a selection of indices $i_1,...,i_l:\{1,\dots,n\}$, we use the notation
    \[
      U_{i_1\dots i_l}\colonequiv U_{i_1}\cap\dots\cap U_{i_l}\rlap{.}
    \]
  \item For a list of indices $i_1,\dots,i_l$, let $i_1,\dots,\hat{i_t},\dots,i_l$ be the same list with the $t$-th element removed.
  \item For $k:\Z$, the $k$-th \notion{Čech-boundary operator}\index{$\partial^k$} is the homomorphism
    \[
      \partial^k:\bigoplus_{i_0,\dots,i_k}\mathcal F(U_{i_0\dots i_k})\to \bigoplus_{i_0,\dots,i_{k+1}}\mathcal F(U_{i_0\dots i_{k+1}})
    \]
    given by $\partial^k(s)\colonequiv (l_0,\dots,l_{k+1}) \mapsto \sum_{j=0}^k (-1)^j s_{l_0,\dots,\hat{l_j},\dots,l_k|U_{l_0,\dots,l_{k+1}}}$.
  \item The $k$-th \notion{Čech-Cohomology group} for the cover $U_1,\dots,U_n$ with coefficients in $\mathcal F$ is
    \[
      \check{H}^k(\{U\},\mathcal F)\colonequiv \ker\partial^{k} / \im(\partial^{k-1})\rlap{.}
    \]
  \end{enumerate}
\end{definition}

\begin{definition}
  The cover $U_1,\dots,U_n$ is called \notion{acyclic} for $\mathcal F$,
  if for all $k:\N$ and $i_0,\dots,i_k$, we have that the higher (non Čech) cohomology groups are trivial:
  \[
    \forall l>0. H^l(U_{i_0,\dots,i_k},\mathcal F)=0\rlap{.}
  \]
\end{definition}

\begin{example}
  If $X$ is a scheme, $U_1,\dots,U_n$ a cover by affine open subtypes and $\mathcal F$ pointwise a finitely presented $R$-module,
  then $U_1,\dots,U_n$ is acyclic for $\mathcal F$ by \cref{H1-fp-module-affine-trivial}.
\end{example}

\begin{theorem}[using \axiomref{Z-choice}]%
  If $U_1,\dots,U_n$ is an acyclic cover for $\mathcal F$, then
  \[
    \check{H}^1(\{U\},\mathcal F)=H^1(X,\mathcal F)\rlap{.}
  \]
\end{theorem}

\begin{proof}
  Let $\pi$ be the projection map
  \[
    \pi :
    \left(
      \sum_{T:\Tors{\mathcal F}(X)}\prod_{i}\prod_{x:U_i}T_x
    \right)
    \to \Tors{\mathcal F}(X)\rlap{.}
  \]
  Let us abbreviate the left hand side with $T(\mathcal F,U)$.
  Since the cover is acyclic, $\pi$ is surjective.
  There is a map $\iota$ into the kernel of $\partial^1$ (\cref{chech-complex} (e)):
  \[
    \iota \colonequiv
    (T,t) \mapsto (i,j\mapsto t_i - t_j) :
    T(\mathcal F,U)
    \to
    \ker(\partial^1)
    \subseteq
    \bigoplus_{i,j}\mathcal F(U_{ij})\rlap{.}
  \]
  We will now show, that $\iota$ is an embedding and therefore also, that its domain is a set.
  Let $(T,t),(T',t'):T(\mathcal F,U)$ such that $\iota((T,t))=\iota((T',t'))$,
  i.e. for all $i,j$ we have $t_i-t_j=t'_i-t'_j$.
  The latter shows the well-definedness (needed to apply \cref{kraus-glueing})
  of a global map $T\simeq T'$, given by sending $t_i(x)$ to $t'_i(x)$
  for all $i$ and $x$.

  The map $\iota$ is also a surjection and therefore an isomorphism:
  Let $s:\ker(\partial^1)$.
  Then we can contruct a torsor,
  by starting with the trivial torsor on each $U_i$.
  We use \cref{kraus-glueing-1-type} to get a torsor
  with the identification given by the $s_{ij}$
  where the cocycle condition holds because $s$ is in the kernel.

  Realizing, that $\im(\partial^0)$ corresponds to the subtype of $T(\mathcal F,U)$ of trivial torsors,
  we arrive at the following diagram:
  \begin{center}
    \begin{tikzcd}
      & \Tors{\mathcal F}(X)\ar[r,->>] & H^1(X,\mathcal F) \\
      \sum_{T:T(\mathcal F,U)}\|\pi_1(T)=\mathcal F\|\ar[r,hook] & T(\mathcal F,U)\ar[u,->>]\ar[d,equal] & \\
      \im{\partial^0}\ar[r,hook]\ar[u,equal] & \ker{\partial^1}\ar[r,->>] & \check{H}^1(\{U\},\mathcal F)
    \end{tikzcd}
  \end{center}
  By \cref{MISSING},
  the composed map $T(\mathcal F,U)\to H^1(X,\mathcal F)$ is a homomorphism
  and therefore by \cref{surjective-abgroup-hom-is-cokernel} a cokernel.
  So the two cohomology groups are equal, since they are cokernels of the same diagram.
\end{proof}


\printbibliography

\end{document}
