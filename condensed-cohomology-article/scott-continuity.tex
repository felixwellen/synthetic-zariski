The goal of this section is to prove the axioms of Barton-Commelin's condensed type theory in synthetic Stone duality. Sections \ref{stone-spaces} and \ref{overtly-discrete-types} contain generalities on Stone spaces and overtly discrete types, \Cref{scott-continuity-for-cohomology} contains the restricted versions of Barton-Commelin's axioms \Cref{TODO} needed for our cohomoloogy computations (so it only contains results about Stone spaces). Finally \Cref{barton-commelin} contains the proof of the axioms proper, which essentially means it extends results from \Cref{scott-continuity-for-cohomology} to compact Hausdorff spaces. This section can be skipped by readers only interested in the cohomology results.



\subsection{Stone spaces}
\label{stone-spaces}

We recall generalities about Stone spaces. Most results come from \cite{synthetic-stone-duality}. 

\begin{definition}
A Stone space is a type that is merely a sequential limit of finite types.
\end{definition}

\begin{remark}
This is equivalent to the definition of Stone spaces as spectrum of countably presented booolean algebras given in \cite{synthetic-stone-duality} by Lemmas 3.1 and 3.2 from \cite{synthetic-stone-duality}. We use this alternative definition to emphasis the duality between Stone spaces and overtly discrete types which will be defined in \Cref{overtly-discrete-types} as sequential colimits of finite types.
\end{remark}

\begin{lemma}
A proposition is a Stone space if and only if it is closed.
\end{lemma}

\begin{proof}
Corollary 3.9 of \cite{synthetic-stone-duality}.
\end{proof}

\begin{proposition}
We have the following:

\begin{enumerate}[(i)]
\item Finite types are Stone spaces.
\item Stone spaces are stable under identity types and and sigma types.
\item Stone spaces are stable under sequential limits.
  %\rednote{Not in \cite{synthetic-stone-duality}, is it a version of the quarter-plane lemma?}
%\item Stone spaces have local choice.
 % \rednote{Axiom 3 of \cite{synthetic-stone-duality}}
\end{enumerate}
\end{proposition}

\begin{proof}
As follows:

\begin{enumerate}[(i)]
\item Clear.
\item Lemma 3.10 and Theorem 4.18 of \cite{synthetic-stone-duality}.
\item From duality and the fact that a sequential colimit of a countably presented boolean algebra is countably presented.
%\item This is the local choice axiom.
\end{enumerate}
\end{proof}

Next we give two compactness results for Stone spaces. The first one is used as a definition of compactness in synthetic topology \cite{TODO}, while the second one is closer to the usual topological formulation of compactness.

\begin{lemma}\label{compact-hausdorff-compact}
Assume given $S$ Stone and $U\subset S$ an open subtype. Then $U=S$ is open.
\end{lemma}

\begin{proof}
This is Corollary 4.4 of \cite{synthetic-stone-duality}.
%We show that $\neg(U=S)$ is closed. To do this it is enough to show that:
%\[\neg(U=S) \leftrightarrow \exists (x:S).\neg U_x\]
%but we know that $\exists (x:S).\neg U_x$ is closed and therefore $\neg\neg$-stable, we have that $U_x$ is $\neg\neg$-stable as well and we can conclude from that since we always have:
%\[\neg(\forall(x:S).\ \neg\neg U_x) \to \neg\neg(\exists(x:S).\ \neg U_x)\]
\end{proof}

\begin{lemma}\label{compact-hausforff-countable-cover}
Assume given $S:\Stone$ and $(U_i)_{i:\N}$ open subsets of $S$ such that:
\[U_0\subset U_1 \subset \cdots\]
%If $\forall(x:S).\exists(i:\N). x\in U_i$, then $\exists(i:\N). \forall(x:S).x\in U_i$.
If $\cup_i U_i = S$, there there is a $k$ such that $U_k=S$
\end{lemma}

\begin{proof}
By local choice we have a surjective map $f: T \to S$
such that $\forall(x:T).\Sigma (i:\N). f(x)\in U_i$. By boundedness there is $k:\N$ such that $\forall(x:T). f(x)\in U_k$
and then we conclude by surjectivity.
\end{proof}

Next we give a finite approximation result for surjections between Stone spaces, which will be key in showing the vanishing of the first cohomology groups of Stone spaces.

\begin{lemma}\label{finite-approximation-stone-surjection-cohomology}
Assume given a $S:\Stone$ and $T:S\to \Stone$ such that $\Pi_{x:S}\propTrunc{T(x)}$. Then there exists a family of $T_k: S\to \Stone$ for $k:\N$ with maps $d_k:\Pi_{x:S}T_{k+1}(x)\to T_k(x)$ such that:
\begin{itemize} 
\item For all $x:S$ we have $\mathrm{lim}_kT_k(x) = T(x)$.
\item For all $k:\N$ there exists a section of $\Pi_{x:S}T_k(x)$.
\end{itemize}
\end{lemma}

\begin{proof}
  By assumption, the projection 
  $\pi: \sum_{x:S} T(x) \to S$ is surjective.
  %By Theorem 4.18 of \cite{synthetic-stone-duality}, the domain is Stone.
  By Remark 3.4 of \cite{synthetic-stone-duality}, 
  we can write $\pi$ as limit of surjections 
  $\pi_k : Q_k\twoheadrightarrow S_k$
  between finite sets.
  As taking fibers and limits commute, we have that 
  $T(x)$ is the limit of $\pi_k^{-1}(x|_k)=:T_k(x)$, 
  which is a closed subset of a finite set, hence Stone. 
  %by Theorem 3.11 of \cite{synthetic-stone-duality}. 
  Furthermore, surjections of finite sets merely have sections
  $s_k: S_k \hookrightarrow Q_k$, 
  giving the required sections
  $\lambda x . s_k(x|_k) : \Pi_{x :S} T_k(x)$. 
\end{proof}



\subsection{Overtly discrete types}
\label{overtly-discrete-types}

Here we give generalities on overtly discrete types. There definition is in some sense dual to that of Stone spaces.

\begin{definition}
A type is overtly discrete if it merely is a sequential colimit of finite types.
\end{definition}

\begin{definition}
A type $X$ is countable if there merely exists a decidable subset of $\N$ equal to $X$.
\end{definition}

Next characterisation is helpful to get intuition about overtly discrete types, although we will almost never use it.

\begin{lemma}\label{overtly-discrete-colimit-finite}
Let $X$ be a type, the following are equivalent:
\begin{enumerate}[(i)]
\item $X$ is overtly discrete.
\item $X$ is a quotient of a countable type by an open equivalence relation.
\end{enumerate}
\end{lemma}

\begin{proof}
This is Lemma 2.12 of \cite{synthetic-stone-duality}.
%\begin{itemize}
%\item (i) implies (ii). Assume $X$ is of the form
%\[X  = (\Sigma_\N D)/R\]
%with $D$ decidable and $R$ open. Using choice for $\Sigma_\N D$ we get:
%\[\alpha : (\Sigma_\N D) \to (\Sigma_\N D)\to 2^\N\]
%such that:
%\[R(x,y) = \exists_{k:\N} \alpha(x,y,k) = 1\]
%Then we define:
%\[X_n = (\Sigma_{\mathrm{Fin}(n)} D) / L\]
%\[L(x,y) = \exists_{k:\mathrm{Fin}(n)} \alpha(x,y,k) = 1\]
%We have that $X_n$ is a finite type as it is a decidable quotient of a decidable subset of a finite type. Moreover:
%\[\mathrm{colim}_n X_n = X\]
%as sequential colimits commute with quotients by equivalence relations.
%\item (ii) implies (i). Indeed consider a sequential colimit of:
%\[f_k : \mathrm{Fin}(l_k) \to \mathrm{Fin}(l_{k+1})\]
%Then:
%\[\mathrm{colim}_k \mathrm{Fin}(l_k)  =  \left(\sum_{k:\N} \mathrm{Fin}(l_k)\right) / L\]
%where $L$ is the equivalence relation generated by $(k,x) \sim (k+1,f_k(x))$. But $\sum_{k:\N} \mathrm{Fin}(l_k)$ is countable and the equivalence relation generated by a decidable relation on such a type is open.
%\end{itemize}
\end{proof}

\begin{lemma}
A proposition is overtly discrete if and only if it is open.
\end{lemma}

\begin{proof}
This is Lemma 2.8 of \cite{synthetic-stone-duality}.
\end{proof}

\begin{proposition}\label{overtly-discrete-closure}
We have the following:

\begin{enumerate}[(i)]
\item Finite types are overtly discrete.
\item Overtly discrete types are stable under identity types and sigma types.
\item Overtly discrete types are stable under quotients by equivalence relation with value in overtly discrete types.
\item Overtly discrete type are stable under sequential colimits.
\item Overtly discrete types have local choice.
\end{enumerate}
\end{proposition}

\begin{proof}
As follows:

\begin{enumerate}[(i)]
\item Clear.
\item This is Lemma 2.7 of \cite{synthetic-stone-duality}.
%For stability under identity types, we use that sequential colimits commutes with identity types. 
%For stability under sigma types, sequential colimits commutes with sigma so that by (iii) it is enough to show that overtly discrete types are stable under finite coproduct. But sequential colimits commute with finite coproducts.
\item This follows from \Cref{overtly-discrete-colimit-finite}.

\item This is Lemma 2.6 of \cite{synthetic-stone-duality}.
%Assume given a tower of sequential colimits of finite types. By using dependent choice with \cref{presentation-maps-overtly-discrete} repeatedly, we get a quarter plane of finite types:
%\begin{center}
%\begin{tikzcd}
%F_{0,0}\ar[d]\ar[r] & F_{0,1}\ar[d]\ar[r] & \cdots \\
%F_{1,0}\ar[d]\ar[r] & F_{1,1}\ar[d]\ar[r] & \cdots \\
%\vdots & \vdots & \ddots\\
%\end{tikzcd}
%\end{center}
%which colimit is the colimit of the assumed tower. Then we just use \cref{colimit-quarter-diagonal} to conclude that this colimit is overtly discrete.

\item By \cref{overtly-discrete-colimit-finite}, we have a cover of any overtly discrete type by a countable type, which is an overtly discrete type that has choice.
\end{enumerate}
\end{proof}



\subsection{Scott continuity for Stone spaces}
\label{scott-continuity-for-cohomology}

Here we prove crucial results for our cohomology computations, namely Propositions \ref{scott-continuity-right}, \ref{scott-continuity-left} and \ref{tychonov-dual-stone}. They are then summed up nicely as \Cref{eine-scott-continuity-stone}, which is a fully dependent version of the following: 

\begin{goal}[Non-dependent version]
Given a tower of Stone spaces $(S_k)_{k:\N}$ and a tower of overtly discrete types $(I_j)_{j:\N}$, we have that the canonical map $\mathrm{colim}_{k,j}(S_k\to I_j) \to (\mathrm{lim}_k S_k\to \mathrm{colim}_j I_j)$ is an equivalence.
\end{goal}

First we focus on the case where $S_k$ is constant, and the target is dependent. We need an auxiliary lemma.

\begin{lemma}\label{overtly-discrete-union-open}
Given a tower of overtly discrete types $(I_i)_{i:\N}$, we have that $\mathrm{Im}(I_i)$ is open in $\mathrm{colim}_i I_i$ for any $i:\N$. 
\end{lemma}

\begin{proof}
Writing $I$ for $\mathrm{colim}_i I_i$, for any $y:I$, we have that $y\in \mathrm{Im}(I_i)$ is by definition $\exists(x:I_i).\, x=_{I}y$,
which is the propositional truncation of an overtly discrete type and therefore open.
\end{proof}

Now we show that maps factoring through the image of a layer of a dependent tower of overtly discrete type over a Stone space actually factors through one of the (possibly higher) layer.

\begin{lemma}\label{factorisation-image-true-factorisation}
Assume given $S$ Stone and for each $x:S$ a tower of overtly discrete types $(I_i(x))_{i:\N}$. Assume given $f:\Pi_{x:S}\mathrm{colim}_i I_i(x)$ which factors through $\mathrm{Im}(I_i)$ for some $i:\N$. Then $f$ factors through $I_j$ for some $j\geq i$.
\end{lemma}

\begin{proof}
By local choice there exists a surjective map $p:T\to S$
with $g:\Pi_{x:T} I_i(p(x))$ such that $[g(x)] =_{\mathrm{colim}_i I_i(x)} f(p(x))$ for all $x:T$.
Then we have that:
\[\forall(x,y:T).\, p(x)=p(y) \to [g(x)] =_{\mathrm{colim}_i I_i(x)} [g(y)]\]
so that:
\[\forall(x,y:T).\, p(x)=p(y) \to \exists(i:\N). [g(x)] =_{I_i(x)} [g(y)]\]
Since $\Sigma(x,y:T). p(x)=p(y)$ is Stone we can apply \Cref{compact-hausforff-countable-cover} to get a $j:\N$ such that:
\[\forall(x,y:T).\, p(x)=p(y) \to [g(x)] =_{I_j(x)} [g(y)]\]
which gives a factorisation of $f$ through $I_j$.
\end{proof}

We can conclude in the case where the source is constant.

\begin{proposition}\label{scott-continuity-right}
Assume given $S$ Stone and for each $x:S$ a tower of overtly discrete types $(I_i(x))_{i:\N}$. Then the canonical map $\mathrm{colim}_i (\Pi_{x:S} I_i(x)) \to \Pi_{x:S} \mathrm{colim}_i I_i(x)$ is an equivalence.
\end{proposition}

\begin{proof}
First we check the canonical map is injective. Given $f,g:\Pi_{x:S} I_i(x)$ such that:
\[\forall(x:S).\, [f(x)] =_{\mathrm{colim}_iI_i(x)} [g(x)]\]
Then we have that:
\[\forall(x:S).\, \exists(i:\N).\, f(x) =_{I_i(x)} g(x)\]
so that by \Cref{compact-hausforff-countable-cover} we have that there exist a $j:\N$ such that:
\[\forall(x:S). [f(x)] =_{I_j(x)} [g(x)]\]
which precisely means that $f=g$ in $\mathrm{colim}_i \Pi_{x:S} I_i(x)$.

Now we check that it is surjective. Given a map $f: \Pi_{x:S} \mathrm{colim}_i I_i(x)$ we know that:
\[\forall(x:S).\, \exists(i:\N).\, f(x)\in \mathrm{Im}(I_i(x))\]
but $f(x)\in \mathrm{Im}(I_i(x))$ is open by \Cref{overtly-discrete-union-open} so that by \Cref{compact-hausforff-countable-cover} we have that there exists $j:\N$ such that:
\[\forall(x:S).\, f(x)\in \mathrm{Im}(I_j(x))\]
This precisely means that $f$ factors through $\mathrm{Im}(I_j)$. We conclude by \Cref{factorisation-image-true-factorisation}.
\end{proof}

Now we focus on the case where target is constant. If the target is finite this follows from duality.

\begin{lemma}\label{factorisation-stone-finite}
Given a tower of Stone spaces $(S_k)_{k:\N}$ and $I$ a finite type, we have that the canonical map $\mathrm{colim}_k (S_k\to I)\to \left(\mathrm{lim}_kS_k\to I\right)$ is an equivalence.
\end{lemma}

\begin{proof}
We have that:
\begin{eqnarray}
\mathrm{colim}_k (S_k\to I) &=& \mathrm{colim}_k \Hom(2^I, 2^{S_k})\nonumber\\
&=& \Hom(2^I, \mathrm{colim}_k 2^{S_k})\nonumber\\
&=& \Spec(\mathrm{colim}_k 2^{S_k})\to \Spec(2^I) \nonumber\\
&=& \mathrm{lim}_kS_k\to I\nonumber
\end{eqnarray}
Where the second line comes from the fact that $2^I$ is finitely presented. We omit the verification that this is indeed the canonical map.
\end{proof}

Now we extend to the case where the target is an arbitrary overtly discrete type.

\begin{proposition}\label{scott-continuity-left}
Given a tower of Stone spaces $(S_k)_{k:\N}$ and $I$ an overtly discrete type, we have that the canonical map $\mathrm{colim}_k (S_k\to I)\to \left(\mathrm{lim}_kS_k\to I\right)$ is an equivalence.
\end{proposition}

\begin{proof}
There exists a tower of finite types $(I_i)_{i:\N}$ so that $I = \mathrm{colim}_iI_i$. By the non-dependent version of \Cref{scott-continuity-right} together with \Cref{factorisation-stone-finite} we have that:
\begin{eqnarray}
\mathrm{colim}_k(S_k\to \mathrm{colim}_iI_i) &=& \mathrm{colim}_{k,i}(S_k\to I_i)\nonumber\\
&=& \mathrm{colim}_i(\mathrm{lim}_kS_k\to I_i)\nonumber\\
&=& \mathrm{lim}_kS_k\to \mathrm{colim}_iI_i\nonumber
\end{eqnarray}
We omit the verification that this is indeed the canonical map.
\end{proof}

Now we try to prove the dual of Tychonov for Stone spaces, i.e. that a product of overtly discrete types indexed by a Stone space is itself overtly discrete. We need an auxiliary lemma.

\begin{lemma}\label{tychonov-dual-auxiliary}
Assume given $p:T\to S$ a surjective map between Stone spaces with $I(x)$ an overtly discrete type depending on $x:S$. If $\Pi_{x:T}I(p(x))$ is overtly discrete then so is $\Pi_{x:S}I(x)$.
\end{lemma}

\begin{proof}
Since the map is surjective we have an embedding $\Pi_{x:S}I(x)\subset \Pi_{x:T}I(p(x))$. But the fiber over $g:\Pi_{x:T}I(p(x))$ is:
\[\forall (x,y:T).\, p(x)=p(y) \to g(x)=g(y)\]
which is open by \Cref{compact-hausdorff-compact}. Therefore $\Pi_{x:S}I(x)$ is an open subtype of an overtly discrete type, so it is itself overtly discrete.
\end{proof}

\begin{proposition}[Tychonov's Dual]\label{tychonov-dual-stone}
Assume given $S$ Stone with $I(x)$ an overtly discrete type depending on $x:S$. Then $\Pi_{x:S}I(x)$ is overtly discrete.
\end{proposition}

\begin{proof}
By local choice and \Cref{tychonov-dual-auxiliary} we can assume given tower of finite types $(\mathrm{Fin}(l_{k,x}))_{k:\N}$
such that $I(x) = \mathrm{colim}_k\, \mathrm{Fin}(l_{k,x})$ for all $x:S$.
Then using \Cref{scott-continuity-right} and the fact that overtly discrete types are stable under sequential colimits, it is enough to prove that $\Pi_{x:S} \mathrm{Fin}(l_{k,x})$ is overtly discrete. 

Using boundedness we have that there exists $n:\N$ such that $\Pi_{x:S} \mathrm{Fin}(l_{k,x}) = \Pi_{i<n}(S_i\to \mathrm{Fin}(i))$ where $S_i = \{x:S\ |\ l_{k,x} = i\}$. Then we conclude by \Cref{factorisation-stone-finite} and the fact that overtly discrete types are closed under finite products.
\end{proof}

So far we have two versions of Scott continuity (Propositions \ref{scott-continuity-right} and \ref{scott-continuity-left}), neither clearly implying the other. In the rest of this section we give a common generalisation, suggested to us by Reid Barton. It is not used for the cohomology results. We start by an auxiliary definition.

\begin{definition}\label{category-scott-continuity}
We have a category ${\mathcal C}_\Stone$ defined by:

\begin{itemize}
\item An object consists of $S:\Stone$ and $I:S\to \ODisc$.
\item A morphism from $(S,I)$ to $(T,J)$ consists of $p:T\to S$ with $q:\Pi_{x:T}(I(p(x))\to J(x))$.
\end{itemize}
\end{definition}

The arrows are defined such that we have a covariant functor $\Pi:{\mathcal C}\to \ODisc$. By \Cref{tychonov-dual-stone} it indeed takes value in $\ODisc$. Now we prove the general version of Scott continuity for Stone spaces, which comes from the two given versions plus some reasoning on colimits.

\begin{theorem}[Scott continuity for Stone spaces]\label{eine-scott-continuity-stone}
The functor $\Pi : {\mathcal C}_\Stone \to \ODisc$ commutes with sequential colimits.
\end{theorem}

\begin{proof}
We assume given a tower in ${\mathcal C}_\Stone$, i.e. we assume a tower:
\[S_0 \overset{p_0}{\leftarrow} S_1 \overset{p_0}{\leftarrow} S_2 \overset{p_2}{\leftarrow}\cdots \]
in $\Stone$ with for all $k:\N$ a dependent type $I_k:S_k\to \ODisc$ and $q_k : \Pi_{x:S_{k+1}}I_k(p_k(x))\to I_{k+1}(x)$.
We want to prove that the canonical map $\mathrm{colim}_i (\Pi_{x:S_i}I_i(x)) \to \Pi_{x:\mathrm{lim}_kS_k}\mathrm{colim}_i I_i(x_i)$ is an equivalence. 

By general reasoning on colimits and Propositions \ref{scott-continuity-right} and \ref{scott-continuity-left}, we get the following string of equivalences:
\begin{eqnarray}
\mathrm{colim}_i (\Pi_{x:S_i}I_i(x)) &=& \mathrm{colim}_{i,k\geq i} \Pi_{x:S_k} I_i(x_i)\nonumber\\
&=& \mathrm{colim}_{i,k\geq i} \Pi_{x:S_i}\Pi_{y:S_k, [y]=x} \to I_i(x) \nonumber\\
&=& \mathrm{colim}_i\Pi_{x:S_i} \mathrm{colim}_{k\geq i} \Pi_{y:S_k, [y]=x} I_i(x)\nonumber\\
&=& \mathrm{colim}_i\Pi_{x:S_i} \Pi_{y:\mathrm{lim}_{k\geq i} S_k,[y]=x} I_i(x)\nonumber\\
&=& \mathrm{colim}_i\Pi_{x:\mathrm{lim}_{k} S_k} I_i([x])\nonumber\\
&=& \Pi_{x:\mathrm{lim}_{k} S_k} \mathrm{colim}_i I_i(x)\nonumber
\end{eqnarray}
%We know that the source is equivalent to:
%\[\mathrm{colim}_{i,k\geq i} \Pi_{x:S_k} I_i(x_i)\]
%which is the same as:
%\[\mathrm{colim}_{i,k\geq i} \Pi_{x:S_i}\Pi_{y:S_k} y_k=x \to I_i(x)\]
%which by \Cref{scott-continuity-right} is in turn equal to:
%\[\mathrm{colim}_i\Pi_{x:S_i} \mathrm{colim}_{k\geq i} \Pi_{y:S_k} y_i = x \to I_i(x)\]
%which by \Cref{scott-continuity-left} is equal to:
%\[\mathrm{colim}_i\Pi_{x:S_i} \Pi_{y:\mathrm{lim}_{k\geq i} S_k} y_i=x \to I_i(x)\]
%which is immediately seen as:
%\[\mathrm{colim}_i\Pi_{x:\mathrm{lim}_{k} S_k} I_i(x_i)\]
%which by \Cref{scott-continuity-right} is equal to:
%\[\Pi_{x:\mathrm{lim}_{k} S_k} \mathrm{colim}_i I_i(x_k)\]
We omit the verification that this is indeed the canonical map.
\end{proof}

\begin{remark}
Scott continuity has the immediate consequence that given a sequence of Stone spaces $(S_k)_{k:\N}$ and $I$ overtly discrete, for any map in $\left(\Pi_{k:\N} S_k\right) \to I$ merely factors through $\Pi_{k<n} S_k$ for some $n:\N$. This justifies the name Scott continuity.
\end{remark}


\subsection{Barton-Commelin axioms}
\label{barton-commelin}

We already have proven all the results needed for our cohomology results. In this section we prove the additional results needed to get all of Barton and Commelin's condensed type theory axioms. \Cref{overtly-discrete-closure} already give all the axioms assumed about overtly discrete types. Stone spaces are not closed under quotients so we have no hope of getting the same result for them, so we consider compact Hausdorff spaces which are precisely the quotients of Stone spaces by closed equivalence relations. We recall the definition.

\begin{definition}
A type $X$ is a compact Hausdorff space of its identity types are closed and there exists a Stone space $X$ with a surjection $S\to X$.
\end{definition}

We need an auxiliary lemma to prove compact Hausdorff spaces closed under sequential limits.

\begin{lemma}\label{sequential-limit-Hausdorff}
Assume given a tower $(C_k)_{k:\N}$ of compact Hausdorff spaces. Then there exists a tower $(S_k)_{k:\N}$ of Stone spaces with maps:

\begin{center}
\begin{tikzcd}
\cdots \ar[r]& S_1\ar[r]\ar[d] & S_0 \ar[d]\\
\cdots \ar[r] & C_1\ar[r] & C_0 \\
\end{tikzcd}
\end{center}

such that the map:
\[S_0\to C_0\]
is surjective and for all $n:\N$ the induced map:
\[S_{n+1} \to C_{n+1}\times_{C_n} S_n\]
is surjective.

This implies that the induced map:
\[\lim_kS_k \to \lim_k C_k\]
is surjective.
\end{lemma}

\begin{proof}
By definition of a Compact Hausdorff type, we can merely find a Stone space $S_0$ with a surjection:
\[S_0\to C_0\]
Using dependent choice, it is enough to show that we can merely extend such a tower $(S_k)_{k\leq n}$ to $(S_k)_{k\leq n+1}$.

We choose a Stone space $T$ and a surjection:
\[T \to C_{n+1} \]
Then we define $S_{n+1}$ by the following pullback square:
\begin{center}
\begin{tikzcd}
S_{n+1}\ar[d]\ar[r] & C_{n+1}\times_{C_n} S_n\ar[r]  \ar[d]& S_n\ar[d]\\
T \ar[r] & C_{n+1}\ar[r]  & C_n
\end{tikzcd}
\end{center}
And we see the map:
\[S_{n+1} \to C_{n+1}\times_{C_n} S_n\] 
surjective as it is a pullback of the map:
\[T\to C_{n+1}\]

This implies that the map:
\[\lim_kS_k \to \lim_k C_k\]
is surjective by dependent choice.
\end{proof}

\begin{theorem}
We have the following:
\begin{enumerate}[(i)]
\item Finite types are compact Hausdorff spaces.
\item Compact Hausdorff spaces are stable under identity types and and sigma types.
  \rednote{Consequence of Corollary 3.9, and Lemma 4.11 of \cite{synthetic-stone-duality}}
\item Compact Hausdorff spaces are stable under quotients by equivalence relation with value in compact Hausdorff spaces.
  \rednote{One needs that compact Hausdorff propositions are closed, which follows from Lemma 4.17. }
\item Compact Hausdorff spaces are stable under sequential limits.
\item Compact Hausdorff space have local choice.
\end{enumerate}
\end{theorem}

\begin{proof}
We proceed as follows:
\begin{enumerate}[(i)]
\item For identity types this is because closed propositions are compact Hausdorff. For sigma this is \rednote{TODO reference}

\item Clear from the definition.

\item We use \cref{sequential-limit-Hausdorff} and the fact that closed proposition are stable under sequential limits.

\item From the fact that Stone spaces have local choice.
\end{enumerate}
\end{proof}

Next we need to prove Tychonov and its dual for compact Hausdorff spaces.

\begin{proposition}[Tychonov]
Assume given $I$ overtly discrete and $C(i)$ compact Hausdorff depending on $i:I$. Then:
\[\Pi_{i:I}C_i\]
is compact Hausdorff.
\end{proposition}

\begin{proof}
We can assume $I = \mathrm{colim}_{k:\N}\, I_k$ with $I_k$ finite. Then:
\[\Pi_{i:I}C_i = \Pi_{i:\mathrm{colim}_{k:\N}\, I_k} C_i = \lim_{k:\N}\, \Pi_{i:I_k}C(\iota_k(i))\]
and we can conclude using that compact Hausdorff spaces are stable under sequential limits and finite products.
\end{proof}

\begin{proposition}[Tychonov's dual]
Assume given $C$ compact Hausdorff and $I(x)$ overtly discrete depending on $x:C$. Then:
\[\Pi_{x:C}I(x)\]
is overtly discrete.
\end{proposition}

\begin{proof}
We consider a surjection $p:S\to C$ with $S$ Stone. Then
\[\Pi_{x:C}I(x) = \{f:\Pi_{x:S}I(p(x))\ |\ \forall(x,y:S).\, p(x)=p(y)\to f(x)=f(y)\}\]
But $\forall(x,y:S).\, p(x)=p(y)\to f(x)=f(y)$ is open by \Cref{compact-hausdorff-compact} and $prod_{x:S}I(p(x))$ is overtly discrete by \Cref{tychonov-dual-stone}, so we can conclude.
\end{proof}

Finally we extend Scott continuity to compact Hausdorff spaces. We write ${\mathcal C}_\CHaus$ for ${\mathcal C}_\Stone$ as given in \Cref{category-scott-continuity} with Stone spaces replaced by compact Hausdroff spaces. As for Tychonov's dual, this just involves reasoning on quotients.

\begin{theorem}[Scott Continuity]
The functor:
\[\Pi : {\mathcal C}_\CHaus \to \ODisc\]
commutes with sequential colimits.
\end{theorem}

\begin{proof}
Assume given $(C_i,I_i)_{i:\N}$ a tower in $\mathcal C$. By \Cref{sequential-limit-Hausdorff} we can consider $(S_i)_{i:\N}$ a tower of Stone spaces with $p_k:S_k\to C_k$ giving a level-wise surjection to the tower $(C_i)_{i:\N}$. We write $S=\mathrm{lim}_iS_i$, $C=\mathrm{lim}_iC_i$, for $x:C$ we have $I(x) = \mathrm{colim}_iI_i(x_i)$ and finally $p =\mathrm{lim}_ip_i$. Then the canonical map:
\[\mathrm{colim}_i\, (\Pi_{x:C_i}I_i(x) ) \to \Pi_{x:C}I(x)\]
is equal to the canonical map:

\[ \left\{ \iota_i(f_i) : \mathrm{colim}_i\, \Pi_{x:S_i}\, I_i(p_i(x))\ |\ \mathrm{colim}_{j>i}\ Q_j(f_i)\right\} \to \left\{f:\Pi_{x:S}\, I(p(x))\ |\ Q(f)\right\} \]
where we defined:
\begin{eqnarray}
 Q_j(f_i) &=& \Pi_{(x,y:S_j,p_j(x)=p_j(y))}\, \iota_j(f_i(\pi_i(x)))=\iota_j(f_i(\pi_i(y)))\nonumber\\
Q(f) &=& \Pi_{(x,y:S,p(x)=p(y))}\, f(x)=f(y)\nonumber
\end{eqnarray}

Then by \Cref{eine-scott-continuity-stone} we have that the canonical map
\[\epsilon : \mathrm{colim}_i \Pi_{x:S_i}\,I_i(p_i(x)) \to \Pi_{x:S}\, I(p(x))\]
is an equivalence, by \Cref{eine-scott-continuity-stone} again we have that the canonical map
\[\mathrm{colim}_{j>i}Q_j(f_i) \to Q(\epsilon(\iota_i(f_i)))\]
is an equivalence so we can conclude.
\end{proof}



