\subsection{Stone spaces}

Recall that Stone spaces are defined as spectrum of c.p. boolean algebra, or equivalently as sequential limits of finite types.
\rednote{Lemmas 3.1 and 3.2 of \cite{synthetic-stone-duality}}

\begin{lemma}
A proposition is a Stone space if and only if it is closed.
\rednote{Corollary 3.8 of \cite{synthetic-stone-duality}}
\end{lemma}

\begin{proof}
The key point is that the propositional truncation of a stone space is a closed proposition.
\end{proof}

\begin{theorem}
We have the following:
\begin{enumerate}[(i)]
\item Finite types are Stone spaces.
\item Stone spaces are stable under identity types and and sigma types.
  \rednote{Lemma 3.10, Theorem 4.18 of \cite{synthetic-stone-duality}}
\item Stone spaces are stable under sequential limits.
  \rednote{Not in \cite{synthetic-stone-duality}, 
  is it a version of the quarter-plane lemma?}
\item Stone spaces have local choice.
  \rednote{Axiom 3 of \cite{synthetic-stone-duality}}
\end{enumerate}
\end{theorem}

\begin{proof}
\rednote{TODO precise reference} 
\end{proof}

We give two compactness results for Stone spaces:

\begin{lemma}\label{compact-hausdorff-compact}
Assume given $S$ Stone and $U\subset S$ an open subtype. Then $U=S$ is open.
\rednote{Corollary 4.4 of \cite{synthetic-stone-duality}, although the proof is left implicit there.}
\end{lemma}

\begin{proof}
We show that $\neg(U=S)$ is closed. To do this it is enough to show that:
\[\neg(U=S) \leftrightarrow \exists (x:S).\neg U_x\]
but we know that $\exists (x:S).\neg U_x$ is closed and therefore $\neg\neg$-stable, we have that $U_x$ is $\neg\neg$-stable as well and we can conclude from that since we always have:
\[\neg(\forall(x:S).\ \neg\neg U_x) \to \neg\neg(\exists(x:S).\ \neg U_x)\]
\end{proof}

\begin{lemma}\label{compact-hausforff-countable-cover}
Assume given $S$ Stone and open subsets of $S$ as follows:
\[U_0\subset U_1 \subset \cdots\]
If:
\[\forall(x:S).\exists(i:\N). x\in U_i\]
then:
\[\exists(i:\N). \forall(x:S).x\in U_i\]
\end{lemma}

\begin{proof}
By local choice we have a surjective map:
\[ f: T \to S\]
such that:
\[\forall(x:T).\Sigma (i:\N). f(x)\in U_i\]
By boundedness there is $k:\N$ such that:
\[\forall(x:T). f(x)\in U_k\]
and then we conclude by surjectivity.
\end{proof}

Next we give a finite approximation results for surjections between Stone spaces, which will be key in showing the vanishing of the first cohomology groups of Stone spaces.

\begin{lemma}\label{finite-approximation-stone-surjection-cohomology}
Assume given a Stone space and $T:S\to \Stone$ such that $\prod_{x:S}\propTrunc{T(x)}$. Then there exists a sequence $T_k: S\to \Stone$ with $\prod_{x:S}T_{k+1}(x)\to T_k(x)$ for $k:\N$ such that:
\begin{itemize} 
\item For all $x:S$ we have $\mathrm{lim}_kT_k(x) = T(x)$.
\item For all $k:\N$ we have a section in $\prod_{x:S}T_k(x)$.
\end{itemize}
\end{lemma}

\begin{proof}
  By assumption, the projection 
  $\pi: \sum_{x:S} T(x) \to S$ is surjective.
  By Theorem 4.18 of \cite{synthetic-stone-duality}, the domain is Stone.
  Hence by Remark 3.4 of \cite{synthetic-stone-duality}, 
  we can write $\pi$ as limit of surjections 
  $\pi_k : Q_k\twoheadrightarrow S_k$
  between finite sets.
  As taking fibers and limits commute, we have that 
  $T(x)$ is the limit of $\pi_k^{-1}(x|_k)=:T_k(x)$, 
  which is a closed subset of a finite set, hence Stone by 
  Theorem 3.11 of \cite{synthetic-stone-duality}. 
  Furthermore, surjections of finite sets have sections
  $s_k: S_k \hookrightarrow Q_k$, 
  giving the required section 
  $\lambda x . s_k(x|_k) : \prod_{x :S} T_k(x)$. 
\end{proof}


\subsection{Overtly discrete types}


\begin{definition}
A type $X$ is countable if there merely exists a decidable subset of $\N$ equal to $X$.
\end{definition}

\begin{definition}
A type is overtly discrete if it is a sequential colimit of finite types.
\end{definition}

\begin{lemma}\label{overtly-discrete-colimit-finite}
Let $X$ be a type, the following are equivalent:
\begin{enumerate}[(i)]
\item $X$ is overtly discrete.
\item $X$ is a quotient of a countable type by an open equivalence relation.
\end{enumerate}
\rednote{Lemma 2.12 of \cite{synthetic-stone-duality}}
\end{lemma}

\begin{proof}
\begin{itemize}
\item (i) implies (ii). Assume $X$ is of the form
\[X  = (\Sigma_\N D)/R\]
with $D$ decidable and $R$ open. Using choice for $\Sigma_\N D$ we get:
\[\alpha : (\Sigma_\N D) \to (\Sigma_\N D)\to 2^\N\]
such that:
\[R(x,y) = \exists_{k:\N} \alpha(x,y,k) = 1\]
Then we define:
\[X_n = (\Sigma_{\mathrm{Fin}(n)} D) / L\]
\[L(x,y) = \exists_{k:\mathrm{Fin}(n)} \alpha(x,y,k) = 1\]
We have that $X_n$ is a finite type as it is a decidable quotient of a decidable subset of a finite type. Moreover:
\[\mathrm{colim}_n X_n = X\]
as sequential colimits commute with quotients by equivalence relations.
\item (ii) implies (i). Indeed consider a sequential colimit of:
\[f_k : \mathrm{Fin}(l_k) \to \mathrm{Fin}(l_{k+1})\]
Then:
\[\mathrm{colim}_k \mathrm{Fin}(l_k)  =  \left(\sum_{k:\N} \mathrm{Fin}(l_k)\right) / L\]
where $L$ is the equivalence relation generated by $(k,x) \sim (k+1,f_k(x))$. But $\sum_{k:\N} \mathrm{Fin}(l_k)$ is countable and the equivalence relation generated by a decidable relation on such a type is open.
\end{itemize}
\end{proof}

\begin{remark}
A proposition is overtly discrete if and only if it is open.
\rednote{Lemma 2.8 of \cite{synthetic-stone-duality}.}
\end{remark}

\begin{lemma}\label{presentation-maps-overtly-discrete}
Assume given a tower of finite types:
\[F_0\to F_1\to \cdots\]
and $J$ overtly discrete with a map:
\[f:\mathrm{colim}_iF_i\to J\]
Then there merely exists a tower of finite types:
\[G_0\to G_1\to \cdots\]
with a map of towers:
\begin{center}
\begin{tikzcd}
F_0\ar[d]\ar[r] & F_1\ar[d]\ar[r] & \cdots \\
G_0\ar[r] & G_1\ar[r] & \cdots \\
\end{tikzcd}
\end{center}
such that:
\[\mathrm{colim}_jG_j = J\]
and the induced map:
\[\mathrm{colim}_iF_i \to \mathrm{colim}_jG_j\]
is equal to $f$.
\end{lemma}

\begin{proof}
We can assume $J = \mathrm{colim}_jG_j$. Then for all $k:\N$ the map:
\[F_k\to \mathrm{colim}_iF_i \to \mathrm{colim}_jG_j\]
merely factors through $G_{l}$ for some arbitrary large $l$. Using dependent choice we merely have a sequence $(l_k)_{k:\N}$ (that we can assume strictly increasing) such that for all $k:\N$ we have that:
\[F_k\to \mathrm{colim}_iF_i \to \mathrm{colim}_jG_j\]
factors through $G_{l_k}$. This gives what we want.
\end{proof}

\begin{lemma}\label{colimit-quarter-diagonal}
Assume given a quarter plane of commutative squares:
\begin{center}
\begin{tikzcd}
A_{0,0}\ar[d]\ar[r] & A_{0,1}\ar[d]\ar[r] & \cdots \\
A_{1,0}\ar[d]\ar[r] & A_{1,1}\ar[d]\ar[r] & \cdots \\
\vdots & \vdots & \ddots\\
\end{tikzcd}
\end{center}
where all the $A_{i,j}$ are sets. Then we have that:
\[\mathrm{colim}_{i,j} A_{i,j} = \mathrm{colim}_iA_{i,i}\]
\end{lemma}

\begin{proof}
We define a map in:
\[\mathrm{colim}_{i,j} A_{i,j} \to \mathrm{colim}_iA_{i,i}\]
by sending $x:A_{i,j}$ to its image in $A_{\max(i,j),\max(i,j)}$.

Given $x:A_{i,j}$ we need to check that it is send to the same element in $\mathrm{colim}_iA_{i,i}$ as its image in $A_{i+1,j}$. If $i<j$ this is immediate. If $i\geq j$ then we just need to check that the maps:
\[A_{i,j} \to A_{i+1,j} \to A_{i+1,i+1}\]
and:
\[A_{i,j}\to A_{i,i}\to A_{i+1,i+1}\]
are equal, which is straightforward.

Checking that this definition respects commutation of squares is vacuous as $\mathrm{colim}_iA_{i,i}$ is a set.

Now this map is clearly left inverse to the canonical map:
\[ \mathrm{colim}_iA_{i,i}\to\mathrm{colim}_{i,j} A_{i,j} \]
we just need to check that it is right inverse to conclude. Given $x:A_{i,j}$ we need to check that it is equal to its image in $A_{\max(i,j),\max(i,j)}$ in $\mathrm{colim}_{i,j} A_{i,j}$, which holds by definition of colimits. Since $\mathrm{colim}_{i,j} A_{i,j}$ is a set (as a sequential colimit of sequential colimits of sets), there is nothing else to check.
\end{proof}

\begin{remark}
We conjecture the previous lemma holds even when the $A_{i,j}$ are not sets.
\end{remark}

\begin{theorem}\label{overtly-discrete-closure}
We have the following:
\begin{enumerate}[(i)]
\item Finite types are overtly discrete.
\item Overtly discrete types are stable under identity types and sigma types.
  \rednote{Lemma 2.7 of  \cite{synthetic-stone-duality}.}
\item Overtly discrete types are stable under quotients by equivalence relation with value in overtly discrete types.
\item Overtly discrete type are stable under sequential colimits.
\item Overtly discrete types have local choice.
\end{enumerate}
\end{theorem}

\begin{proof}
\begin{enumerate}[(i)]
\item For stability under identity types, we use that sequential colimits commutes with identity types. 

For stability under sigma types, sequential colimits commutes with sigma so that by (iii) it is enough to show that overtly discrete types are stable under finite coproduct. But sequential colimits commute with finite coproducts.

\item Clear from the alternative description in \cref{overtly-discrete-colimit-finite}.

\item Assume given a tower of sequential colimits of finite types. By using dependent choice with \cref{presentation-maps-overtly-discrete} repeatedly, we get a quarter plane of finite types:
\begin{center}
\begin{tikzcd}
F_{0,0}\ar[d]\ar[r] & F_{0,1}\ar[d]\ar[r] & \cdots \\
F_{1,0}\ar[d]\ar[r] & F_{1,1}\ar[d]\ar[r] & \cdots \\
\vdots & \vdots & \ddots\\
\end{tikzcd}
\end{center}
which colimit is the colimit of the assumed tower. Then we just use \cref{colimit-quarter-diagonal} to conclude that this colimit is overtly discrete.

\item By \cref{overtly-discrete-colimit-finite}, we have a cover of any overtly discrete type by a countable type, which is an overtly discrete type that has choice.
\end{enumerate}
\end{proof}

\begin{remark}
(ii) implies that the propositional truncation of an overtly discrete type is open.

(iii) implies that overtly discrete types are closed under countable coproducts.
\end{remark}



\subsection{Scott continuity for Stone spaces}

\begin{lemma}\label{factorisation-stone-finite}
For any tower of Stone space $(S_k)_{k:\N}$ and $l:\N$, we have that:
\[\left(\mathrm{lim}_kS_k\to \mathrm{Fin}(l)\right) \simeq \mathrm{colim}_k (S_k\to \mathrm{Fin}(l))\]
\end{lemma}

\begin{proof}
We have that:
\[\left(\mathrm{lim}_kS_k\to \mathrm{Fin}(l)\right) \]
\[\simeq \Hom(2^l, \mathrm{colim}_k 2^{S_k})\]
\[\simeq \mathrm{colim}_k \Hom(2^l, 2^{S_k}) \]
\[\simeq \mathrm{colim}_k (S_k\to \mathrm{Fin}(l))\]
Where the third line comes from the fact that $2^l$ is finitely presented.
\end{proof}
\begin{corollary}
  For any overtly discrete space $A$, and any tower of Stone spaces $(S_k)_{k:\N}$, we have that \[\left(\mathrm{lim}_kS_k\to A\right) \simeq \mathrm{colim}_k (S_k\to A)\]
\end{corollary}
\begin{proof}
  Consider $A$ as colimit of finite types $A_i$, by the above we have 
  $$ \mathrm{lim}_kS_k \to A \simeq 
  \mathrm{lim}_kS_k \to \mathrm{colim}_i S_i \simeq 
  \mathrm{colim}_{i} (\mathrm{lim}_kS_k \to A_i) \simeq 
  \mathrm{colim}_{i} (\mathrm{colim}_k (S_k \to A_i) \simeq 
  \mathrm{colim}_{k} (\mathrm{colim}_i (S_k \to A_i) \simeq 
  \mathrm{colim}_{k} ((S_k \to A) \simeq 
  $$
    
\end{proof}

\begin{lemma}\label{overtly-discrete-union-open}
Assume $(I_i)_{i:\N}$ a tower of overtly discrete types. Then for any $i:\N$ we have that $\mathrm{Im}(I_i)$ is open in $\mathrm{colim}_i I_i$. 
\end{lemma}

\begin{proof}
For $y:\mathrm{colim}_iI_i$, we have that $y\in \mathrm{Im}(I_i)$ is:
\[\exists(x:I_i). x=_{\mathrm{colim}_iI_i}y\]
which is the propositional truncation of an overtly discrete type and therefore open.
\end{proof}

\begin{lemma}\label{factorisation-image-true-factorisation}
Assume given $S$ Stone and for each $x:S$ a tower:
\[I_0(x)\to I_1(x)\to \cdots\]
of overtly discrete types. Given:
\[f:\Pi_{x:S}\mathrm{colim}_i I_i(x)\]
such that there exists $i:\N$ such that $f$ factors through $\mathrm{Im}(I_i)$. Then there exists $j\geq i$ such that $f$ factors through $I_j$.
\end{lemma}

\begin{proof}
By local choice there exists a surjective map:
\[p:T\to S\]
with:
\[g:\Pi_{x:T} I_i(p(x))\]
such that:
\[\forall(x:T). g(x) =_{\mathrm{colim}_i I_i(x)} f(p(x))\]
Then we have that:
\[\forall(x,y:T). p(x)=p(y) \to g(x) =_{\mathrm{colim}_i I_i(x)} g(y)\]
so that:
\[\forall(x,y:T). p(x)=p(y) \to \exists(i:\N). g(x) =_{I_i(x)} g(y)\]
Since $\Sigma(x,y:T). p(x)=p(y)$ is Stone we can apply \Cref{compact-hausforff-countable-cover} to get a $j:\N$ such that:
\[\forall(x,y:T). p(x)=p(y) \to g(x) =_{I_j(x)} g(y)\]
which gives a factorisation of $f$ through $I_j$.
\end{proof}

\begin{proposition}\label{scott-continuity-right}
Assume given $S:\Stone$ and for each $x:S$ a tower:
\[I_0(x)\to I_1(x)\to \cdots\]
of overtly discrete types. Then the canonical map:
\[\mathrm{colim}_i \Pi_{x:S} I_i(x) \to \Pi_{x:S} \mathrm{colim}_i I_i(x) \]
is an equivalence.
\end{proposition}

\begin{proof}
First we check the canonical map is injective. Given $f,g:\Pi_{x:S} I_i(x)$ such that:
\[\forall(x:S).  f(x) =_{\mathrm{colim}_iI_i(x)} g(x)\]
Then we have that:
\[\forall(x:S).  \exists(i:\N). f(x) =_{I_i(x)} g(x)\]
so that by \Cref{compact-hausforff-countable-cover} we have that:
\[\exists(j:\N). \forall(x:S). f(x) =_{I_j(x)} g(x)\]
which precisely means that $f=g$ in $\mathrm{colim}_i \Pi_{x:S} I_i(x)$.

Now we check that it is surjective. Given a map:
\[f: \Pi_{x:S} \mathrm{colim}_i I_i(x)\]
we know that:
\[\forall(x:S). \exists(i:\N). f(x)\in \mathrm{Im}(I_i(x))\]
but $f(x)\in \mathrm{Im}(I_i(x))$ is open by \Cref{overtly-discrete-union-open} so that by \Cref{compact-hausforff-countable-cover} we have that:
\[\exists(i:\N). \forall(x:S).  f(x)\in \mathrm{Im}(I_i(x))\]
which precisely mean that $f$ factors through $\mathrm{Im}(I_i)$ for some $i$. We conclude by \Cref{factorisation-image-true-factorisation}.
\end{proof}

\begin{proposition}\label{scott-continuity-left}
Assume $(S_k)_{k:\N}$ a tower of Stone spaces, and let $I$ be an over. Then the canonical morphism:
\[\mathrm{colim}_k(S_k\to I) \to (\mathrm{lim}_kS_k\to I)\]
is an equivalence.
\end{proposition}

\begin{proof}
There exists a tower of finite types $(I_i)_{i:\N}$ so that $I = \mathrm{colim}_iI_i$. By the non-dependent version of \Cref{scott-continuity-right} together with the fact that $\mathrm{lim}_kS_k$ is a Stone space, it is enough to prove the result for each $I_i$, i.e. that:
\[\mathrm{colim}_k(S_k\to I_i) \to (\mathrm{lim}_kS_k\to I_i)\]
But this is \Cref{factorisation-stone-finite}.
\end{proof}

\begin{remark}
A consequence of Scott continuity is that given a family $(S_k)_{k:\N}$ and $I$ overtly discrete, for any map:
\[f: \left(\Pi_{k:\N} S_k\right) \to I\]
there merely exists $n:\N$ such that $f$ factors through:
\[\Pi_{k:\mathrm{Fin}(n)} S_k\]
which justifies the name.
\end{remark}

Now we try to prove Tychinov's dual for Stone spaces, i.e. that $\Pi:{\mathcal C}_\Stone
\to \Type$ takes value in $\ODisc$.

\begin{lemma}\label{tychonov-dual-auxiliary}
Assume $p:T\to S$ a surjective map with $S,T:\Stone$. Given $I_x$ overtly discrete depending on $x:S$, if:
\[\prod_{x:T}I_{p(x)}\]
is overtly discrete then so is:
\[\prod_{x:S}I_x\]
\end{lemma}

\begin{proof}
Since the map is surjective we have an embedding:
\[\prod_{x:S}I_x\subset \prod_{x:T}I_{p(x)}\]
But the fiber over $g:\prod_{x:T}I_{p(x)}$ is:
\[\forall (x,y:T). p(x)=p(y) \to g(x)=g(y)\]
which is open by \Cref{compact-hausdorff-compact}.
\end{proof}

\begin{theorem}[Tychonov's dual]\label{tychonov-dual-stone}
Assume given $S$ Stone and $I_x$ overtly discrete depending on $x:S$. Then:
\[\prod_{x:S}I_x\]
is overtly discrete.
\end{theorem}

\begin{proof}
By local choice and \Cref{tychonov-dual-auxiliary} we can assume that $C$ is Stone (so we denote $C$ by $S$) and that we have towers of finite types $(I_k(x)))_{k:\N}$ such that for all $x:S$ we have that:
\[I(x) = \mathrm{colim}_k\, \mathrm{Fin}(l_{k,x})\]
Then using \Cref{scott-continuity-right} and the fact that overtly discrete types are stable under sequential colimits, it is enough to prove that we have that:
\[\prod_{x:S} \mathrm{Fin}(l_{k,x})\]
is overtly discrete. 

Using boundedness, the fact that:
\[S_n = \Sigma_{x:S} I_{k,x} = n\]
is Stone and that overtly discrete types are stable under finite products, we just have to prove that for any $l:\N$ and $S$ Stone we have that:
\[S\to \mathrm{Fin}(l)\]
is overtly discrete. We can conclude using that \Cref{factorisation-stone-finite} and that Stone are limits of finite types.
\end{proof}

So far we have two versions of Scott continuity (\Cref{scott-continuity-right} and \Cref{scott-continuity-left}), neither clearly implying the other. It is not used for the cohomology results. We give a common generalisation, suggested to me (Hugo) by Reid Barton.

\begin{definition}\label{category-scott-continuity}
We have a category ${\mathcal C}_\Stone$ defined by:
\begin{itemize}
\item An object consists of $X:\Stone$ and $I:X\to \ODisc$.
\item A morphism from $(X,I)$ to $(Y,J)$ consists of $f:Y\to X$ with $\forall y:Y.\ I_{f(x)}\to J_x$.
\end{itemize}
We consider ${\mathcal C}_\Stone$ the full subcategory where $X$ is in $\Stone$.
\end{definition}

The arrows are oriented such that we have a covariant functor $\Pi:{\mathcal C}\to \ODisc$. It takes value in $\ODisc$ by \Cref{tychonov-dual-stone} First we prove the general version for Stone spaces. It just comes from the two versions and some reasoning and colimits.

\begin{proposition}\label{eine-scott-continuity-stone}
The functor:
\[\Pi : {\mathcal C}_\Stone \to \ODisc\]
commutes with sequential colimits.
\end{proposition}

\begin{proof}
We assume given a tower in $\C$, that is we assume a tower:
\[S_0 \overset{p_0}{\leftarrow} S_1 \overset{p_0}{\leftarrow} S_2 \overset{p_2}{\leftarrow}\cdots \]
in $\Stone$ with for all $k:\N$ a dependent type $I_k:S_k\to \ODisc$ and:
\[q_k : \Pi_{x:S_{k+1}}I_k(p_k(x))\to I_{k+1}(x)\]
We want to prove the canonical map:
\[\mathrm{colim}_i (\Pi_{x:S_i}I_i(x)) \to \Pi_{x\mathrm{lim}_kS_k}\mathrm{colim}_i I_i(x_i))\]
is an equivalence. 

By general reasoning on colimits, we know that the source is equivalent to:
\[\mathrm{colim}_{i,k\geq i} \Pi_{x:S_k} I_i(x_i)\]
which is the same as:
\[\mathrm{colim}_{i,k\geq i} \Pi_{x:S_i}\Pi_{y:S_k} y_k=x \to I_i(x)\]
which by \Cref{scott-continuity-right} is in turn equal to:
\[\mathrm{colim}_i\Pi_{x:C_i} \mathrm{colim}_{k\geq i} \Pi_{y:S_k} y_i = x \to I_i(x)\]
which by \Cref{scott-continuity-left} is equal to:
\[\mathrm{colim}_i\Pi_{x:S_i} \Pi_{y:\mathrm{lim}_{k\geq i} S_k} y_i=x \to I_i(x)\]
which is immediately seen as:
\[\mathrm{colim}_i\Pi_{x:\mathrm{lim}_{k} S_k} I_i(x_i)\]
which by \Cref{scott-continuity-right} is equal to:
\[\Pi_{x:\mathrm{lim}_{k} S_k} \mathrm{colim}_i I_i(x_k)\]

We omit the checking that this is indeed the canonical map.
\end{proof}



\subsection{Barton Commelin axioms}

We already have proven all the results needed for our cohomology results. In this section we prove the additional results needed to get all of Barton and Commelin's condensed type theory axioms. \Cref{overtly-discrete-closure} already give all the axioms assumed about overtly discrete types. Stone spaces are not closed under quotients so we have no hope of getting the same result for them, so we consider compact Hausdorff spaces which are precisely the quotients of Stone spaces by closed equivalence relations. We recall the definition.

\begin{definition}
A type $X$ is a compact Hausdorff space of its identity types are closed and there exists a Stone space $X$ with a surjection $S\to X$.
\end{definition}

We need an auxiliary lemma to prove compact Hausdorff spaces closed under sequential limits.

\begin{lemma}\label{sequential-limit-Hausdorff}
Assume given a tower $(C_k)_{k:\N}$ of compact Hausdorff spaces. Then there exists a tower $(S_k)_{k:\N}$ of Stone spaces with maps:

\begin{center}
\begin{tikzcd}
\cdots \ar[r]& S_1\ar[r]\ar[d] & S_0 \ar[d]\\
\cdots \ar[r] & C_1\ar[r] & C_0 \\
\end{tikzcd}
\end{center}

such that the map:
\[S_0\to C_0\]
is surjective and for all $n:\N$ the induced map:
\[S_{n+1} \to C_{n+1}\times_{C_n} S_n\]
is surjective.

This implies that the induced map:
\[\lim_kS_k \to \lim_k C_k\]
is surjective.
\end{lemma}

\begin{proof}
By definition of a Compact Hausdorff type, we can merely find a Stone space $S_0$ with a surjection:
\[S_0\to C_0\]
Using dependent choice, it is enough to show that we can merely extend such a tower $(S_k)_{k\leq n}$ to $(S_k)_{k\leq n+1}$.

We choose a Stone space $T$ and a surjection:
\[T \to C_{n+1} \]
Then we define $S_{n+1}$ by the following pullback square:
\begin{center}
\begin{tikzcd}
S_{n+1}\ar[d]\ar[r] & C_{n+1}\times_{C_n} S_n\ar[r]  \ar[d]& S_n\ar[d]\\
T \ar[r] & C_{n+1}\ar[r]  & C_n
\end{tikzcd}
\end{center}
And we see the map:
\[S_{n+1} \to C_{n+1}\times_{C_n} S_n\] 
surjective as it is a pullback of the map:
\[T\to C_{n+1}\]

This implies that the map:
\[\lim_kS_k \to \lim_k C_k\]
is surjective by dependent choice.
\end{proof}

\begin{theorem}
We have the following:
\begin{enumerate}[(i)]
\item Finite types are compact Hausdorff spaces.
\item Compact Hausdorff spaces are stable under identity types and and sigma types.
  \rednote{Consequence of Corollary 3.9, and Lemma 4.11 of \cite{synthetic-stone-duality}}
\item Compact Hausdorff spaces are stable under quotients by equivalence relation with value in compact Hausdorff spaces.
  \rednote{One needs that compact Hausdorff propositions are closed, which follows from Lemma 4.17. }
\item Compact Hausdorff spaces are stable under sequential limits.
\item Compact Hausdorff space have local choice.
\end{enumerate}
\end{theorem}

\begin{proof}
We proceed as follows:
\begin{enumerate}[(i)]
\item For identity types this is because closed propositions are compact Hausdorff. For sigma this is \rednote{TODO reference}

\item Clear from the definition.

\item We use \cref{sequential-limit-Hausdorff} and the fact that closed proposition are stable under sequential limits.

\item From the fact that Stone spaces have local choice.
\end{enumerate}
\end{proof}

Next we need to prove Tychonov and its dual for compact Hausdorff spaces.

\begin{proposition}[Tychonov]
Assume given $I$ overtly discrete and $C(i)$ compact Hausdorff depending on $i:I$. Then:
\[\prod_{i:I}C_i\]
is compact Hausdorff.
\end{proposition}

\begin{proof}
We can assume $I = \mathrm{colim}_{k:\N}\, I_k$ with $I_k$ finite. Then:
\[\prod_{i:I}C_i = \prod_{i:\mathrm{colim}_{k:\N}\, I_k} C_i = \lim_{k:\N}\, \prod_{i:I_k}C(\iota_k(i))\]
and we can conclude using that compact Hausdorff spaces are stable under sequential limits and finite products.
\end{proof}

\begin{proposition}[Tychonov's dual]
Assume given $C$ compact Hausdorff and $I(x)$ overtly discrete depending on $x:C$. Then:
\[\prod_{x:C}I(x)\]
is overtly discrete.
\end{proposition}

\begin{proof}
We consider a surjection $p:S\to C$ with $S$ Stone. Then
\[\prod_{x:C}I(x) = \{f:\prod_{x:S}I(p(x))\ |\ \forall(x,y:S).\, p(x)=p(y)\to f(x)=f(y)\}\]
But $\forall(x,y:S).\, p(x)=p(y)\to f(x)=f(y)$ is open by \Cref{compact-hausdorff-compact} and $prod_{x:S}I(p(x))$ is overtly discrete by \Cref{tychonov-dual-stone}, so we can conclude.
\end{proof}

Finally we extend Scott continuity to compact Hausdorff spaces. We write ${\mathcal C}_\CHaus$ for ${\mathcal C}_\Stone$ as given in \Cref{category-scott-continuity} with Stone spaces replaced by compact Hausdroff spaces. As for Tychonov's dual, this just involves reasoning on quotients.

\begin{theorem}[Scott Continuity]
The functor:
\[\Pi : {\mathcal C}_\CHaus \to \ODisc\]
commutes with sequential colimits.
\end{theorem}

\begin{proof}
Assume given $(C_i,I_i)_{i:\N}$ a tower in $\mathcal C$. By \Cref{sequential-limit-Hausdorff} we can consider $(S_i)_{i:\N}$ a tower of Stone spaces with $p_k:S_k\to C_k$ giving a level-wise surjection to the tower $(C_i)_{i:\N}$. We write $S=\mathrm{lim}_iS_i$, $C=\mathrm{lim}_iC_i$, for $x:C$ we have $I(x) = \mathrm{colim}_iI_i(x_i)$ and finally $p =\mathrm{lim}_ip_i$. Then the canonical map:
\[\mathrm{colim}_i\, (\Pi_{x:C_i}I_i(x) ) \to \Pi_{x:C}I(x)\]
is equal to the canonical map:

\[ \left\{ \iota_i(f_i) : \mathrm{colim}_i\, \Pi_{x:S_i}\, I_i(p_i(x))\ |\ \mathrm{colim}_{j>i}\ Q_j(f_i)\right\} \to \left\{f:\Pi_{x:S}\, I(p(x))\ |\ Q(f)\right\} \]
where we defined:
\begin{eqnarray}
 Q_j(f_i) &=& \Pi_{(x,y:S_j,p_j(x)=p_j(y))}\, \iota_j(f_i(\pi_i(x)))=\iota_j(f_i(\pi_i(y)))\nonumber\\
Q(f) &=& \Pi_{(x,y:S,p(x)=p(y))}\, f(x)=f(y)\nonumber
\end{eqnarray}

Then by \Cref{eine-scott-continuity-stone} we have that the canonical map
\[\epsilon : \mathrm{colim}_i \prod_{x:S_i}\,I_i(p_i(x)) \to \prod_{x:S}\, I(p(x))\]
is an equivalence, by \Cref{eine-scott-continuity-stone} again we have that the canonical map
\[Q(f) \to P(\epsilon(f))\]
is an equivalence so we can conclude.
\end{proof}



