\subsection{Scott continuity}

\begin{lemma}\label{factorisation-stone-finite}
For any tower of Stone space $(S_k)_{k:\N}$ and $l:\N$, we have that:
\[\left(\mathrm{lim}_kS_k\to \mathrm{Fin}(l)\right) \simeq \mathrm{colim}_k (S_k\to \mathrm{Fin}(l))\]
\end{lemma}

\begin{proof}
We have that:
\[\left(\mathrm{lim}_kS_k\to \mathrm{Fin}(l)\right) \]
\[\simeq \Hom(2^l, \mathrm{colim}_k 2^{S_k})\]
\[\simeq \mathrm{colim}_k \Hom(2^l, 2^{S_k}) \]
\[\simeq \mathrm{colim}_k (S_k\to \mathrm{Fin}(l))\]
Where the third line comes from the fact that $2^l$ is finitely presented.
\end{proof}

\begin{lemma}\label{overtly-discrete-union-open}
Assume $(I_i)_{i:\N}$ a tower of overtly discrete types. Then for any $i:\N$ we have that $\mathrm{Im}(I_i)$ is open in $\mathrm{colim}_i I_i$. 
\end{lemma}

\begin{proof}
For $y:\mathrm{colim}_iI_i$, we have that $y\in \mathrm{Im}(I_i)$ is:
\[\exists(x:I_i). x=_{\mathrm{colim}_iI_i}y\]
which is the propositional truncation of an overtly discrete type and therefore open.
\end{proof}

\begin{lemma}\label{factorisation-image-true-factorisation}
Assume given $S$ Stone and for each $x:S$ a tower:
\[I_0(x)\to I_1(x)\to \cdots\]
of overtly discrete types. Given:
\[f:\Pi_{x:S}\mathrm{colim}_i I_i(x)\]
such that there exists $i:\N$ such that $f$ factors through $\mathrm{Im}(I_i)$. Then there exists $j\geq i$ such that $f$ factors through $I_j$.
\end{lemma}

\begin{proof}
By local choice there exists a surjective map:
\[p:T\to S\]
with:
\[g:\Pi_{x:T} I_i(p(x))\]
such that:
\[\forall(x:T). g(x) =_{\mathrm{colim}_i I_i(x)} f(p(x))\]
Then we have that:
\[\forall(x,y:T). p(x)=p(y) \to g(x) =_{\mathrm{colim}_i I_i(x)} g(y)\]
so that:
\[\forall(x,y:T). p(x)=p(y) \to \exists(i:\N). g(x) =_{I_i(x)} g(y)\]
Since $\Sigma(x,y:T). p(x)=p(y)$ is Stone we can apply \Cref{compact-hausforff-countable-cover} to get a $j:\N$ such that:
\[\forall(x,y:T). p(x)=p(y) \to g(x) =_{I_j(x)} g(y)\]
which gives a factorisation of $f$ through $I_j$.
\end{proof}

\begin{proposition}\label{scott-continuity-right}
Assume given $S:\Stone$ and for each $x:S$ a tower:
\[I_0(x)\to I_1(x)\to \cdots\]
of overtly discrete types. Then the canonical map:
\[\mathrm{colim}_i \Pi_{x:S} I_i(x) \to \Pi_{x:S} \mathrm{colim}_i I_i(x) \]
is an equivalence.
\end{proposition}

\begin{proof}
First we check the canonical map is injective. Given $f,g:\Pi_{x:S} I_i(x)$ such that:
\[\forall(x:S).  f(x) =_{\mathrm{colim}_iI_i(x)} g(x)\]
Then we have that:
\[\forall(x:S).  \exists(i:\N). f(x) =_{I_i(x)} g(x)\]
so that by \Cref{compact-hausforff-countable-cover} we have that:
\[\exists(j:\N). \forall(x:S). f(x) =_{I_j(x)} g(x)\]
which precisely means that $f=g$ in $\mathrm{colim}_i \Pi_{x:S} I_i(x)$.

Now we check that it is surjective. Given a map:
\[f: \Pi_{x:S} \mathrm{colim}_i I_i(x)\]
we know that:
\[\forall(x:S). \exists(i:\N). f(x)\in \mathrm{Im}(I_i(x))\]
but $f(x)\in \mathrm{Im}(I_i(x))$ is open by \Cref{overtly-discrete-union-open} so that by \Cref{compact-hausforff-countable-cover} we have that:
\[\exists(i:\N). \forall(x:S).  f(x)\in \mathrm{Im}(I_i(x))\]
which precisely mean that $f$ factors through $\mathrm{Im}(I_i)$ for some $i$. We conclude by \Cref{factorisation-image-true-factorisation}.
\end{proof}

\begin{proposition}\label{scott-continuity-left}
Assume $(S_k)_{k:\N}$ a tower of Stone spaces, and let $I$ be an over. Then the canonical morphism:
\[\mathrm{colim}_k(S_k\to I) \to (\mathrm{lim}_kS_k\to I)\]
is an equivalence.
\end{proposition}

\begin{proof}
There exists a tower of finite types $(I_i)_{i:\N}$ so that $I = \mathrm{colim}_iI_i$. By the non-dependent version of \Cref{scott-continuity-right} together with the fact that $\mathrm{lim}_kS_k$ is a Stone space, it is enough to prove the result for each $I_i$, i.e. that:
\[\mathrm{colim}_k(S_k\to I_i) \to (\mathrm{lim}_kS_k\to I_i)\]
But this is \Cref{factorisation-stone-finite}.
\end{proof}

\begin{remark}
A consequence of Scott continuity is that given a family $(S_k)_{k:\N}$ and $I$ overtly discrete, for any map:
\[f: \left(\Pi_{k:\N} S_k\right) \to I\]
there merely exists $n:\N$ such that $f$ factors through:
\[\Pi_{k:\mathrm{Fin}(n)} S_k\]
which justifies the name.
\end{remark}


\subsection{Tychonov and its dual}

\rednote{Here we prove the version for compact Hausdorff as it is not more complicated than the Stone version. I think the Stone version is enough for our purpose.}

\begin{theorem}[Tychonov]
Assume given $I$ overtly discrete and $C(i)$ compact Hausdorff depending on $i:I$. Then:
\[\prod_{i:I}C_i\]
is compact Hausdorff.
\end{theorem}

\begin{proof}
We can assume $I = \mathrm{colim}_{k:\N}\, I_k$ with $I_k$ finite. Then:
\[\prod_{i:I}C_i = \prod_{i:\mathrm{colim}_{k:\N}\, I_k} C_i = \lim_{k:\N}\, \prod_{i:I_k}C(\iota_k(i))\]
and we can conclude using that compact Hausdorff spaces are stable under sequential limits and finite products.
\end{proof}

\begin{lemma}\label{tychonov-dual-auxiliary}
Assume $p:C\to D$ a surjective map with $C$ and $D$ compact Hausdorff. Given $I_x$ overtly discrete depending on $x:D$, if:
\[\prod_{x:C}I_{p(x)}\]
is overtly discrete then so is:
\[\prod_{x:D}I_x\]
\end{lemma}

\begin{proof}
Since the map is surjective we have an embedding:
\[\prod_{x:D}I_x\subset \prod_{x:C}I_{p(x)}\]
But the fiber over $g:\prod_{x:C}I_{p(x)}$ is:
\[\forall (x,y:C). p(x)=p(y) \to g(x)=g(y)\]
which is open by \Cref{compact-hausdorff-compact}.
\end{proof}

\begin{theorem}[Tychonov's dual]\label{tychonov-dual}
Assume given $C$ compact Hausdorff and $I_x$ overtly discrete depending on $x:C$. Then:
\[\prod_{x:C}I_x\]
is overtly discrete.
\end{theorem}

\begin{proof}
By local choice and \Cref{tychonov-dual-auxiliary} we can assume that $C$ is Stone (so we denote $C$ by $S$) and that we have towers of finite types $(I_k(x)))_{k:\N}$ such that for all $x:S$ we have that:
\[I(x) = \mathrm{colim}_k\, \mathrm{Fin}(l_k(x))\]
Then using \Cref{scott-continuity-right} and the fact that overtly discrete types are stable under sequential colimits, it is enough to prove that we have that:
\[\prod_{x:S} \mathrm{Fin}(l_k(x))\]
is overtly discrete. 

Using boundedness, the fact that overtly discrete types are stable under finite coproducts, we just have to prove that for any $n:\N$ we have that:
\[S_n \to \mathrm{Fin}(n)\]
where $S_n = \Sigma_{x:S} (l_k(x) = n)$ is Stone. We can conclude using \Cref{factorisation-stone-finite} and that Stone are limits of finite types.
\end{proof}


\subsection{The general version of Scott continuity}

So far we have two versions of Scott continuity (\Cref{scott-continuity-right} and \Cref{scott-continuity-left}), neither clearly implying the other. We give a common generalisation, suggested to me (Hugo) by Reid Barton.

\begin{definition}
We have a category $\mathcal C$ defined by:
\begin{itemize}
\item An object consists of $X:\CHaus$ and $I:X\to \ODisc$.
\item A morphism from $(X,I)$ to $(Y,J)$ consists of $f:Y\to X$ with $\forall y:Y.\ I_{f(x)}\to J_x$.
\end{itemize}
We consider ${\mathcal C}_\Stone$ the full subcategory where $X$ is in $\Stone$.
\end{definition}

The arrows are oriented such that we have a covariant functor $\Pi:{\mathcal C}\to \ODisc$. First we prove the general version for Stone spaces. It just comes from the two versions and some reasoning and colimits.

\begin{proposition}\label{eine-scott-continuity-stone}
The functor:
\[\Pi : {\mathcal C}_\Stone \to \ODisc\]
commutes with sequential colimits.
\end{proposition}

\begin{proof}
We assume given a tower in $\C$, that is we assume a tower:
\[S_0 \overset{p_0}{\leftarrow} S_1 \overset{p_0}{\leftarrow} S_2 \overset{p_2}{\leftarrow}\cdots \]
in $\Stone$ with for all $k:\N$ a dependent type $I_k:S_k\to \ODisc$ and:
\[q_k : \Pi_{x:S_{k+1}}I_k(p_k(x))\to I_{k+1}(x)\]
We want to prove the canonical map:
\[\mathrm{colim}_i (\Pi_{x:S_i}I_i(x)) \to \Pi_{x\mathrm{lim}_kS_k}\mathrm{colim}_i I_i(x_i))\]
is an equivalence. 

By general reasoning on colimits, we know that the source is equivalent to:
\[\mathrm{colim}_{i,k\geq i} \Pi_{x:S_k} I_i(x_i)\]
which is the same as:
\[\mathrm{colim}_{i,k\geq i} \Pi_{x:S_i}\Pi_{y:S_k} y_k=x \to I_i(x)\]
which by \Cref{scott-continuity-right} is in turn equal to:
\[\mathrm{colim}_i\Pi_{x:C_i} \mathrm{colim}_{k\geq i} \Pi_{y:S_k} y_i = x \to I_i(x)\]
which by \Cref{scott-continuity-left} is equal to:
\[\mathrm{colim}_i\Pi_{x:S_i} \Pi_{y:\mathrm{lim}_{k\geq i} S_k} y_i=x \to I_i(x)\]
which is immediately seen as:
\[\mathrm{colim}_i\Pi_{x:\mathrm{lim}_{k} S_k} I_i(x_i)\]
which by \Cref{scott-continuity-right} is equal to:
\[\Pi_{x:\mathrm{lim}_{k} S_k} \mathrm{colim}_i I_i(x_k)\]

We omit the checking that this is indeed the canonical map.
\end{proof}

Then we extend it to compact Hausdorff spaces. This just comes from reasoning on quotients.

\begin{theorem}[Scott Continuity]
The functor:
\[\Pi : {\mathcal C} \to \ODisc\]
commutes with sequential colimits.
\end{theorem}

\begin{proof}
Assume given $(C_i,I_i)_{i:\N}$ a tower in $\mathcal C$. Consider $(S_i)_{i:\N}$ a tower of Stone spaces with $p_k:S_k\to C_k$ giving a level-wise surjection to the tower $(C_i)_{i:\N}$. We write $S=\mathrm{lim}_iS_i$, $C=\mathrm{lim}_iC_i$, for $x:C$ we have $I(x) = \mathrm{colim}_iI_i(x_i)$ and finally $p =\mathrm{lim}_ip_i$. Then the canonical map:
\[\mathrm{colim}_i\, (\Pi_{x:C_i}I_i(x) ) \to \Pi_{x:C}I(x)\]
is equal to the canonical map:

\[ \left\{ \iota_i(f_i) : \mathrm{colim}_i\, \Pi_{x:S_i}\, I_i(p_i(x))\ |\ \mathrm{colim}_{j>i}\ Q_j(f_i)\right\} \to \left\{f:\Pi_{x:S}\, I(p(x))\ |\ Q(f)\right\} \]
where we defined:
\begin{eqnarray}
 Q_j(f_i) &=& \Pi_{(x,y:S_j,p_j(x)=p_j(y))}\, \iota_j(f_i(\pi_i(x)))=\iota_j(f_i(\pi_i(y)))\nonumber\\
Q(f) &=& \Pi_{(x,y:S,p(x)=p(y))}\, f(x)=f(y)\nonumber
\end{eqnarray}

Then by \Cref{eine-scott-continuity-stone} we have that the canonical map
\[\epsilon : \mathrm{colim}_i \prod_{x:S_i}\,I_i(p_i(x)) \to \prod_{x:S}\, I(p(x))\]
is an equivalence, by \Cref{eine-scott-continuity-stone} again we have that the canonical map
\[\mathrm{colim}_{j>i}Q_j(f_i) \to Q(\epsilon(\iota_i(f_i)))\]
is an equivalence so we can conclude.
\end{proof}

