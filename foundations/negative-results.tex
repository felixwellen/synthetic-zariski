Here we collect some results of
the theory developed from the axioms
(\axiomref{loc}), (\axiomref{sqc}) and (\axiomref{Z-choice})
that are of a negative nature
and primarily serve the purpose of counterexamples.

We adopt the following definition from
\cite[Section IV.8]{lombardi-quitte}.

\begin{definition}%
  \label{zero-dimensional-ring}
  A ring $A$ is \notion{zero-dimensional}
  if for all $x : A$
  there exists $a : A$ and $k : \N$
  such that $x^k = a x^{k + 1}$.
\end{definition}

\begin{lemma}[using \axiomref{loc}, \axiomref{sqc}, \axiomref{Z-choice}]%
  \label{R-not-zero-dimensional}
  The ring $\A$ is not zero-dimensional.
\end{lemma}

\begin{proof}
  Assume that $\A$ is zero-dimensional,
  so for every $f : \A$ there merely is some $k : \N$ with $f^k \in (f^{k + 1})$.
  We note that $\A = \A^1$ is an affine scheme and
  that if $f^k \in (f^{k + 1})$,
  then we also have $f^{k'} \in (f^{k' + 1})$ for every $k' \geq k$.
  This means that we can apply \Cref{strengthened-boundedness}
  and merely obtain a number $K : \N$
  such that $f^K \in (f^{K + 1})$ for all $f : \A$.
  In particular, $f^{K + 1} = 0$ implies $f^K = 0$,
  so the canonical map
  $\Spec \A[X]/(X^K) \to \Spec \A[X]/(X^{K + 1})$
  is a bijection.
  But this is a contradiction,
  since the homomorphism $\A[X]/(X^{K + 1}) \to \A[X]/(X^K)$
  is not an isomorphism.
\end{proof}

\begin{example}[using \axiomref{loc}, \axiomref{sqc}, \axiomref{Z-choice}]%
  \label{non-existence-of-roots}
  It is not the case that
  every monic polynomial $f : \A[X]$ with $\deg f \geq 1$ has a root.
  More specifically,
  if $U \subseteq \A^1$ is an open subset
  with the property that
  the polynomial $X^2 - a : \A[X]$ merely has a root
  for every $a : U$,
  then $U = \emptyset$.
\end{example}

\begin{proof}
  Let $U \subseteq \A^1$ be as in the statement.
  Since we want to show $U = \emptyset$,
  we can assume a given element $a_0 : U$
  and now have to derive a contradiction.
  By \axiomref{Z-choice},
  there exists in particular a basic open $D(f) \subseteq \A^1$
  with $a_0 \in D(f)$
  and a function $g : D(f) \to \A$
  such that ${(g(x))}^2 = x$ for all $x : D(f)$.
  By \axiomref{sqc},
  this corresponds to an element $\frac{p}{f^n} : \A[X]_f$
  with ${(\frac{p}{f^n})}^2 = X : \A[X]_f$.
  We use \Cref{polynomial-with-regular-value-is-regular}
  together with the fact that $f(a_0)$ is invertible
  to get that $f : \A[X]$ is regular,
  and therefore $p^2 = f^{2n}X : \A[X]$.
  Considering this equation over $\A^{\mathrm{red}} = \A/\sqrt{(0)}$ instead,
  we can show by induction that all coefficients of $p$ and of $f^n$ are nilpotent,
  which contradicts the invertibility of $f(a_0)$.
\end{proof}

\begin{remark}
  \Cref{non-existence-of-roots} shows that
  the axioms we are using here
  are incompatible with a natural axiom that is true
  for the structure sheaf of the big étale topos,
  namely that $\A$ admits roots for unramifiable monic polynomials.
  The polynomial $X^2 - a$ is even separable for invertible $a$,
  assuming that $2$ is invertible in $\A$.
  To get rid of this last assumption,
  we can use the fact that either $2$ or $3$ is invertible in the local ring $\A$
  and observe that the proof of \Cref{non-existence-of-roots}
  works just the same for $X^3 - a$.
\end{remark}

We now give two different proofs that not all $\A$-modules are weakly quasi-coherent
in the sense of \Cref{weakly-quasi-coherent-module}.
The first shows that the map
\[ M_f \to M^{D(f)} \]
is not always surjective,
the second shows that it is not always injective.

\begin{proposition}[using \axiomref{loc}, \axiomref{sqc}, \axiomref{Z-choice}]%
  \label{RN-non-wqc}
  The $\A$-module $\A^\N$ is not weakly quasi-coherent
  (in the sense of \Cref{weakly-quasi-coherent-module}).
\end{proposition}

\begin{proof}
  For $f : \A$,
  we have ${(\A^{\N})}^{D(f)} = {(\A^{D(f)})}^\N = {(\A_f)}^\N$,
  so the question is whether the canonical map
  \[ {(\A^\N)}_f \to {(\A_f)}^\N \]
  is an equivalence.
  If it is,
  for a fixed $f : \A$,
  then the sequence $(1, \frac{1}{f}, \frac{1}{f^2}, \dots)$
  has a preimage,
  so there is an $n : \N$ such that
  for all $k : \N$,
  $\frac{a_k}{f^n} = \frac{1}{f^k}$ in $\A_f$
  for some $a_k : \A$.
  In particular, $\frac{a_{n+1}}{f^n} = \frac{1}{f^{n+1}}$ in $\A_f$
  and therefore $a_{n+1} f^{n+1+\ell} = f^{n+\ell}$ in $\A$ for some $\ell : \N$.
  This shows that $\A$ is zero-dimensional
  (\Cref{zero-dimensional-ring})
  if $\A^\N$ is weakly quasi-coherent.
  So we are done by \Cref{R-not-zero-dimensional}.
\end{proof}

\begin{proposition}[using \axiomref{loc}, \axiomref{sqc}, \axiomref{Z-choice}]%
  \label{non-wqc-module-family}
  The implication
  \[ M^{D(f)} = 0 \quad\\Aightarrow\quad M_f = 0 \]
  does not hold for all $\A$-modules $M$ and $f : \A$.
  In particular,
  the map $M_f \to M^{D(f)}$ from \Cref{weakly-quasi-coherent-module}
  is not always injective.
\end{proposition}

\begin{proof}
  Assume that the implication always holds.
  We construct a family of $\A$-modules,
  parametrized by the elements of $\A$,
  and deduce a contradiction from the assumption
  applied to the $\A$-modules in this family.

  Given an element $f : \A$,
  the $\A$-module we want to consider is
  the countable product
  \[ M(f) \colonequiv \prod_{n : \N} \A/(f^n) \rlap{.} \]
  If $f \neq 0$ then $M(f) = 0$
  (using \Cref{non-zero-invertible}).
  This implies that the $\A$-module $M(f)^{f \neq 0}$
  is trivial:
  any function $f \neq 0 \to M(f)$ can only assign the value $0$
  to any of the at most one witnesses of $f \neq 0$.
  By assumption, this implies that $M(f)_f$ is also trivial.
  Noting that
  $M(f)$ is not only an $\A$-module
  but even an $\A$-algebra in a natural way,
  we have
  \begin{align*}
    M(f)_f = 0
    &\;\Leftrightarrow\;
    \exists k : \N.\; \text{$f^k = 0$ in $M(f)$} \\
    &\;\Leftrightarrow\;
    \exists k : \N.\; \forall n : \N.\; f^k \in (f^n) \subseteq \A \\
    &\;\Leftrightarrow\;
    \exists k : \N.\; f^k \in (f^{k + 1}) \subseteq \A
    \rlap{.}
  \end{align*}

  In summary,
  our assumption implies that the ring $\A$ is zero-dimensional
  (in the sense of \Cref{zero-dimensional-ring}).
  But this is not the case,
  as we saw in \Cref{R-not-zero-dimensional}.
\end{proof}

\begin{example}[using \axiomref{loc}, \axiomref{sqc}]
  It is not the case that
  for any pair of lines $L, L' \subseteq \bP^2$,
  the $\A$-algebra $\A^{L \cap L'}$ is
  as an $\A$-module free of rank $1$.
\end{example}

\begin{proof}
  The $\A$-algebra $\A^{L \cap L'}$ is free of rank $1$
  if and only if the structure homomorphism
  $\varphi : \A \to \A^{L \cap L'}$ is bijective.
  We will show that it is not even always injective.

  Consider the lines
  \[ L = \{\, [x : y : z] : \bP^2 \mid z = 0 \,\} \]
  and
  \[ L' = \{\, [x : y : z] : \bP^2 \mid \varepsilon x + \delta y + z = 0 \,\}
     \rlap{,} \]
  where $\varepsilon$ and $\delta$ are elements of $\A$
  with $\varepsilon^2 = \delta^2 = 0$.
  Consider the element $\varphi(\epsilon \delta) : \A^{L \cap L'}$,
  which is the constant function $L \cap L' \to \A$
  with value $\varepsilon \delta$.
  For any point $[x : y : z] : L \cap L'$,
  we have $z = 0$ and $\varepsilon x + \delta y = 0$.
  But also, by definition of $\bP^3$,
  we have $(x, y, z) \neq 0 : \A^3$,
  so one of $x, y$ must be invertible.
  This implies $\delta \divides \varepsilon$ or $\varepsilon \divides \delta$,
  and in both cases we can conclude $\varepsilon \delta = 0$.
  Thus, $\varphi(\epsilon \delta) = 0 : \A^{L \cap L'}$.

  If $\varphi$ was always injective
  then this would imply $\varepsilon \delta = 0$
  for any $\varepsilon, \delta : \A$
  with $\varepsilon^2 = \delta^2 = 0$.
  In other words, the inclusion
  \[ \Spec \A[X, Y]/(X^2, Y^2, XY) \hookrightarrow \Spec \A[X, Y]/(X^2, Y^2) \]
  would be a bijection.
  But the corresponding $\A$-algebra homomorphism is not an isomorphism.
\end{proof}
