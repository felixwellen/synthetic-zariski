\subsection{Statement of the axioms}%
\label{statement-of-axioms}

We always assume there is a fixed commutative ring $\A$.
In addition, we assume the following three axioms about $\A$,
which were already mentioned in the introduction,
but we will indicate which of these axioms are used to prove each statement
by listing their shorthands.

\begin{axiom}[Loc]%
  \label{loc}\index{Loc}
  $\A$ is a local ring (\Cref{local-ring}).
\end{axiom}

\begin{axiom}[SQC]%
  \label{sqc}\index{sqc}
  For any finitely presented $\A$-algebra $A$, the homomorphism
  \[ a \mapsto (\varphi\mapsto \varphi(a)) : A \to (\Spec A \to \A)\]
  is an isomorphism of $\A$-algebras.
\end{axiom}

\begin{axiom}[Z-choice]%
  \label{Z-choice}\index{Z-choice}
  Let $A$ be a finitely presented $\A$-algebra
  and let $B : \Spec A \to \mU$ be a family of inhabited types.
  Then there merely exist unimodular $f_1, \dots, f_n : A$
  together with dependent functions $s_i : \Pi_{x : D(f_i)} B(x)$.
  As a formula\footnote{Using the notation from \Cref{unimodular}}:
  \[ (\Pi_{x : \Spec A} \propTrunc{B(x)}) \to
     \propTrunc{ ((f_1,\dots,f_n):\Um(A)) \times
      \Pi_i \Pi_{x : D(f_i)} B(x) }
     \rlap{.}
  \]
\end{axiom}

\subsection{First consequences}

Let us draw some first conclusions from the axiom (\axiomref{sqc}),
in combination with (\axiomref{loc}) where needed.

\begin{proposition}[using \axiomref{sqc}]%
  \label{spec-embedding}
  For all finitely presented $\A$-algebras $A$ and $B$ we have an equivalence
  \[
    f\mapsto \Spec f : \Hom_{\AAlg}(A,B) = (\Spec B \to \Spec A)
    \rlap{.}
  \]
\end{proposition}

\begin{proof}
  By \Cref{algebra-from-affine-scheme}, we have a natural equivalence
  \[
    X\to \Spec (\A^X)
  \]
  and by \axiomref{sqc}, the natural map
  \[
    A\to \A^{\Spec A}
  \]
  is an equivalence.
  We therefore have a contravariant equivalence between
  the category of finitely presented $\A$-algebras
  and the category of affine schemes.
  In particular, $\Spec$ is an embedding.
\end{proof}

An important consequence of \axiomref{sqc}, which may be called \notion{weak nullstellensatz}:

\begin{proposition}[using \axiomref{loc}, \axiomref{sqc}]%
  \label{weak-nullstellensatz}
  If $A$ is a finitely presented $\A$-algebra,
  then we have $\Spec A=\emptyset$ if and only if $A=0$.
\end{proposition}

\begin{proof}
  If $\Spec A = \emptyset$
  then $A = \A^{\Spec A} = \A^\emptyset = 0$
  by (\axiomref{sqc}).
  If $A = 0$
  then there are no homomorphisms $A \to \A$
  since $1 \neq 0$ in $\A$ by (\axiomref{loc}).
\end{proof}

For example, this weak nullstellensatz suffices
to prove the following properties of the ring $\A$,
which were already proven in
\cite{ingo-thesis}[Section 18.4].

\begin{proposition}[using \axiomref{loc}, \axiomref{sqc}]%
  \label{nilpotence-double-negation}\label{non-zero-invertible}\label{generalized-field-property}
  
  \begin{enumerate}[(a)]
  \item An element $x:\A$ is invertible,
    if and only if $x\neq 0$.
  \item A vector $x:\A^n$ is non-zero,
    if and only if one of its entries is invertible.
  \item An element $x:\A$ is nilpotent,
    if and only if $\neg \neg (x=0)$.
  \end{enumerate}
\end{proposition}

\begin{proof}
  Part (a) is the special case $n = 1$ of (b).
  For (b),
  consider the $\A$-algebra $A \colonequiv \A/(x_1, \dots, x_n)$.
  Then the set $\Spec A \equiv \Hom_{\AAlg}(A, \A)$
  is a proposition (that is, it has at most one element),
  and, more precisely, it is equivalent to the proposition $x = 0$.
  By \Cref{weak-nullstellensatz},
  the negation of this proposition is equivalent to $A = 0$
  and thus to $(x_1, \dots, x_n) = \A$.
  Using (\axiomref{loc}),
  this is the case if and only if one of the $x_i$ is invertible.

  For (c),
  we instead consider the algebra $A \colonequiv \A_x \equiv \A[\frac{1}{x}]$.
  Here we have $A = 0$ if and only if $x$ is nilpotent,
  while $\Spec A$ is the proposition $\inv(x)$.
  Thus, we can finish by \Cref{weak-nullstellensatz},
  together with part (a) to go from $\lnot \inv(x)$ to $\lnot \lnot (x = 0)$.
\end{proof}

The following lemma,
which is a variant of \cite{ingo-thesis}[Proposition 18.32],
shows that $\A$ is in a weak sense algebraically closed.
See \Cref{non-existence-of-roots} for a refutation of
a stronger formulation of algebraic closure of~$\A$.

\begin{lemma}[using \axiomref{loc}, \axiomref{sqc}]%
  \label{polynomials-notnot-decompose}
  Let $f : \A[X]$ be a polynomial.
  Then it is not not the case that:
  either $f = 0$ or
  $f = \alpha \cdot {(X - a_1)}^{e_1} \dots {(X - a_n)}^{e_n}$
  for some $\alpha : \A^\times$,
  $e_i \geq 1$ and pairwise distinct $a_i : \A$.
\end{lemma}

\begin{proof}
  Let $f : \A[X]$ be given.
  Since our goal is a proposition,
  we can assume we have a bound $n$ on the degree of $f$,
  so
  \[ f = \sum_{i = 0}^n c_i X^i \rlap{.} \]
  Since our goal is even double-negation stable,
  we can assume $c_n = 0 \lor c_n \neq 0$
  and by induction $f = 0$ (in which case we are done)
  or $c_n \neq 0$.
  If $n = 0$ we are done,
  setting $\alpha \colonequiv c_0$.
  Otherwise,
  $f$ is not invertible (using $0 \neq 1$ by (\axiomref{loc})),
  so $\A[X]/(f) \neq 0$,
  which by (\axiomref{sqc}) means that
  $\Spec(\A[X]/(f)) = \{ x : \A \mid f(x) = 0 \}$
  is not empty.
  Using the double-negation stability of our goal again,
  we can assume $f(a) = 0$ for some $a : \A$
  and factor $f = (X - a_1) f_{n - 1}$.
  By induction, we get $f = \alpha \cdot (X - a_1) \dots (X - a_n)$.
  Finally, we decide each of the finitely many propositions $a_i = a_j$,
  which we can assume is possible
  because our goal is still double-negation stable,
  to get the desired form
  $f = \alpha \cdot {(X - \widetilde{a}_1)}^{e_1} \dots {(X - \widetilde{a}_n)}^{e_n}$
  with distinct $\widetilde{a}_i$.
\end{proof}
