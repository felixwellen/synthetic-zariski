% latexmk -pdf -pvc main.tex
% latexmk -pdf -pvc -interaction=nonstopmode main.tex
\documentclass{../util/zariski}

\title{A Foundation for Synthetic Algebraic Geometry}
\author{Felix Cherubini, Thierry Coquand and Matthias Hutzler}

\begin{document}

\maketitle

\begin{center}
  \color{purple}
  \large{Press CRTL+F5 to clear cached versions}
  \large{(if you are viewing these notes online)}
\end{center}

\begin{abstract}
  This is a foundation for algebraic geometry, developed internal to the Zariski topos, building on the work of Kock and Blechschmidt (\cite{kock-sdg}[I.12], \cite{ingo-thesis}).
  The Zariski topos consists of sheaves on the site opposite to the category of finitely presented algebras over a fixed ring, with the Zariski topology, i.e.\ generating covers are given by localization maps $A\to A_{f_1}$ for finitely many elements $f_1,\dots,f_n$ that generate the ideal $(1)=A\subseteq A$.
  We use homotopy type theory together with three axioms as the internal language of a (higher) Zariski topos.

  One of our main contributions is the use of higher types -- in the homotopical sense -- to define and reason about cohomology.
  Actually computing cohomology groups, seems to need a principle along the lines of our ``Zariski local choice'' axiom,
  which we justify as well as the other axioms using a cubical model of homotopy type theory.
\end{abstract}

\tableofcontents

\section*{Introduction}
This draft is empty so far.

\section*{Formalization}
There is a related formalization project, which, at the time of writing,
contains the construction of projective $n$-space $\bP^n$ as a scheme.
The code may be found here:
\begin{center}
  \url{https://github.com/felixwellen/synthetic-geometry}
\end{center}
It makes extensive use of the algebra part of the cubical-agda library:
\begin{center}
  \url{https://github.com/agda/cubical}
\end{center}
-- which contains many contributions, in particular,
on finitely presented algebras and related concepts,
which where made in the scope of that project.

\section*{Acknowlegements}
The idea to use the topological characterization of stone spaces as totally disconnected, compact Hausdorff spaces to prove \Cref{stone-sigma-closed} was explained to us by Martín Escardó.
We profited a lot from a discussion with Reid Barton and Johann Commelin. 
David Wärn noticed that Markov's principle (\Cref{MarkovPrinciple}) holds. 
At TYPES 2024, we had an interesting discussion with Bas Spitters on the topic of the article.
Work on this article was supported by the ForCUTT project, ERC advanced grant 101053291.


\section{Preliminaries}

We will use \cite{draft}[Lemma 4.2.11]:

\begin{lemma}
  \label{closed-implies-open-to-or}
  Let $C$ be a closed proposition and $U$ be an open proposition,
  then $C\to U$ is equivalent to $\neg C \vee U$.
\end{lemma}

We will also need:

\begin{lemma}
  \label{commute-open-in-closed}
  Let $X$ be a scheme, $C\subseteq X$ a closed subtype and $U\subseteq C$ open.
  Then there is an open $\tilde{U}\subseteq X$ such that $\tilde{U}\cap C = U$.
\end{lemma}



\section{Axioms}
\subsection{Statement of the axioms}
We always assume there is a commutative ring $R$.
Sometimes we will assume $R$ has additional properties, or, more generally,
axioms hold that involve $R$.
We will always mention which of these axioms are needed to prove each statement,
by listing the shorthands introduced in the axioms below.

\begin{axiom}[Loc]%
  \label{loc}\index{Loc}
  $R$ is a local ring.
\end{axiom}

\begin{axiom}[SQC]%
  \label{sqc}\index{sqc}
  For any finitely presented $R$-algebra $A$, the homomorphism
  \[ a \mapsto (\varphi\mapsto \varphi(a)) : A \to (\Spec A \to R)\]
  is an isomorphism of $R$-algebras.
\end{axiom}

\begin{axiom}[Z-choice]%
  \label{Z-choice}\index{Z-choice}
  Let $A$ be a finitely presented $R$-algebra
  and let $B : \Spec A \to \mU$ be a family of inhabited types.
  Then there merely exists
  a finite list of coprime elements $f_1, \dots, f_n \in A$
  together with dependent functions $s_i : \Pi_{x : D(f_i)} B(x)$.
  As a formula:
  \[ (\Pi_{x : \Spec A} \propTrunc{B(x)}) \to
     \propTrunc{ \Sigma_{n : \N} \Sigma_{f_1, \dots, f_n : A}
      ((f_1, \dots, f_n) = (1)) \times
      \Pi_i \Pi_{x : D(f_i)} B(x) }
     \rlap{.}
  \]
\end{axiom}

\subsection{First consequences}

\begin{proposition}[using \axiomref{sqc}]%
  \label{spec-embedding}
  For all finitely presented $R$-algebras $A$ and $B$ we have
  \[ (\Spec B \to \Spec A) =\Hom_{\Alg{R}}(A,B)\]
  -- where the equality is induced by exponentiation with $R$.
\end{proposition}

\begin{proof}
  TODO
\end{proof}

An important consequence, which may be called \notion{weak nullstellensatz}:

\begin{proposition}[using \axiomref{loc}, \axiomref{sqc}]%
  \label{weak-nullstellensatz}
  If $A$ is a finitely presented $R$-algebra,
  then we have $\Spec A=\emptyset$ if and only if $A=0$.
\end{proposition}

\begin{proof}
  If $\Spec A = \emptyset$
  then $A = R^{\Spec A} = R^\emptyset = 0$
  by (\axiomref{sqc}).
  If $A = 0$
  then there are no homomorphisms $A \to R$
  since $1 \neq 0$ in $R$ by (\axiomref{loc}).
\end{proof}

The following are originally proven in 
\cite{ingo-thesis}[Section 18]:

\begin{proposition}[using \axiomref{sqc}, \axiomref{loc}]%
  \label{nilpotence-double-negation}\label{non-zero-invertible}\label{generalized-field-property}
  
  \begin{enumerate}[(a)]
  \item An element $x:R$ is nilpotent,
    if and only if $\neg \neg (x=0)$.
  \item An element $x:R$ is invertible,
    if and only if $x\neq 0$.
  \item A vector $x:R^n$ is non-zero,
    if and only if one of its entries is invertible.
  \end{enumerate}
\end{proposition}


\section{Affine schemes}
We only talk about affine schemes of finite type, i.e. schemes of the form $\Spec A$,
where $A$ is a finitely presented algebra.

\begin{definition}%
  A type $X$ is \notion{(qc-)affine},
  if the there is a finitely presented $R$-algebra $A$, such that $X=\Spec A$. 
\end{definition}

\begin{proposition}%
  Let $X$ be a type.
  The type of all finitely presented $R$-algebra $A$, such that $X=\Spec A$, is a proposition.
\end{proposition}

\subsection{Affine-open subtypes}

When we write ``$\Spec A$'' we implicitly assume $A$ is a finitely presented $R$-algebra.

\begin{definition}%
  Let $X=\Spec A$.
  A \notion{(affine) standard open} is the subtype $D(f):X\to\Prop$
  for some $f:\Spec A\to R$ given by $D(f)(x)\colonequiv f(x)\neq 0$.
\end{definition}

\begin{definition}%
  \label{def:affine-open}
  Let $X=\Spec A$.
  A subtype $U:X\to\Prop$ is called \notion{affine-open},
  if one of the following logically equivalent statements holds:
  \begin{enumerate}[(i)]%
  \item $U$ is the union of finitely many affine standard opens.
  \item There are $f_1,\dots,f_n:A$ such that
    \[U(x) \Leftrightarrow \exists_{i} f_i(x)\neq 0 \]
  \end{enumerate}
\end{definition}

We sometimes write $D(f_1, \dots, f_n) \coloneq D(f_1) \cup \dots \cup D(f_n)$
for a finite union of standard opens.
Note that in general, affine-open subtypes do not need to be affine
-- this is why we use the dash ``-''.

We will introduce a more general definition of open subtype in \cref{def:qc-open}
and show in \cref{thm:qc-open-affine-open}, that the two notions agree on affine schemes.

Affine-openness is transitive in the following sense:

\begin{lemma}%
  \label{lem:affine-open-trans}
  Let $X=\Spec A$ and $D(f)\subseteq X$ be a standard open.
  Any affine-open subtype $U$ of $D(f)$ is also affine-open in $X$.
\end{lemma}

\begin{proof}
  It is enough to show the statement for $U=D(g)$, $g:A_f$.
  Then
  \[ g=\frac{h}{f^k}\rlap{.}\]
  Now $D(hf)$ is an affine-open in $X$,
  that coincides with $U$: \\
  Let $x:X$, then $(hf)(x)$ is invertible, if and only if both $h(x)$ and $f(x)$ are invertible.
  The latter means $x:D(f)$, so we can interpret $x$ as a homorphism from $A_f$ to $R$.
  Then $x:D(g)$ means $x(g)$ is invertible, which is equivalent to $x(h)$ being invertible,
  since $x(f)^k$ is invertible anyway.
\end{proof}

\begin{lemma}[using \axiomref{sqc}]%
  \label{lem:standard-open-empty}
  Let $X=\Spec A$ be an affine scheme and $D(f)\subseteq X$ a standard open,
  then $D(f)=\emptyset$, if and only if, $f$ is nilpotent.
\end{lemma}

\begin{proof}
  Since $D(f)=\Spec A_f$, by \cref{Weak-Nullstellensatz}, we know $D(f)=\emptyset$,
  if and only if, $A_f=0$.
  The latter is equivalent to $f$ being nilpotent.
\end{proof}

More generally:

\begin{lemma}[using \axiomref{sqc}]%
  Let $A$ be a finitely presented $R$-algebra
  and let $f, g_1, \dots, g_n \in A$.
  Then we have $D(f) \subseteq D(g_1, \dots, g_n)$
  as subsets of $\Spec A$
  if and only if $f \in \sqrt{(g_1, \dots, g_n)}$.
\end{lemma}

\begin{proof}
  Since $D(g_1, \dots, g_n) = \{\, x \in \Spec A \mid x \notin V(g_1, \dots, g_n) \,\}$,
  the inclusion $D(f) \subseteq D(g_1, \dots, g_n)$
  can also be written as
  $D(f) \cap V(g_1, \dots, g_n) = \varnothing$, that is,
  $\Spec((A/(g_1, \dots, g_n))[f^{-1}]) = \varnothing$.
  By (\axiomref{sqc})
  this means that the finitely presented $R$-algebra $(A/(g_1, \dots, g_n))[f^{-1}]$
  is zero.
  And this is the case if and only if $f$ is nilpotent in $A/(g_1, \dots, g_n)$,
  that is, if $f \in \sqrt{(g_1, \dots, g_n)}$, as stated.
\end{proof}

\subsection{Fiber products}

\begin{lemma}[using \axiomref{sqc}]%
  \label{lem:affine-fiber-product}
  Let $X=\Spec A,Y=\Spec B$ and $Z=\Spec C$ be affine schemes
  with maps $f:X\to Z$, $g:Y\to Z$.
  Then the pullback of this diagram is an affine scheme given by $\Spec (A\otimes_C B)$.
\end{lemma}

\begin{proof}
  The maps $f:X\to Z$, $g:Y\to Z$ are induced by $R$-algebra homomorphisms $f^*:A\to R$ and $g^*:B\to R$.
  Let
  \[ (h,k,p) : \Spec A \times_{\Spec C} \Spec B \]
  with $p:h\circ f^*=k\circ g^* $.
  This defines a $R$-cocone on the diagram
  \[
    \begin{tikzcd}
      A & C\ar[r,"g^*"]\ar[l,"f^*",swap] & B
    \end{tikzcd}
  \]
  Since $A\otimes_C B$ is a pushout in $R$-algebras,
  there is a unique $R$-algebra homomorphism $A\otimes_C B \to R$ corresponding to $(h,k,p)$.
\end{proof}

\subsection{Boundedness of functions to $\N$}

\begin{theorem}[using (\axiomref{loc}), (\axiomref{sqc})]%
  \label{thm:boundedness}
  Let $A$ be a finitely presented $R$-algebra.
  Then every function $f : \Spec A \to \N$ is bounded:
  \[ \Pi_{f : \Spec A \to \N} \propTrunc{\Sigma_{n : \N} \Pi_{x : \Spec A} f(x) \le n}
     \rlap{.} \]
\end{theorem}

\begin{proof}
  Let $f : \Spec A \to \N$ be given.
  We compose this function with the embedding
  \[ \begin{tikzcd}[row sep=0mm]
    \N \ar[r, "\iota"] & R[X] \ar[r, phantom, "{=}"] & (\bA^1 \to R) \\
    n \ar[r, mapsto] & X^n
  \end{tikzcd} \]
  (it is an embedding since $1 \neq 0 \in R$ by (\axiomref{loc}))
  to obtain $\widetilde{f} : \Spec A \to (\bA^1 \to R)$
  and its transpose $g : \Spec A \times \bA^1 \to R$.
  Since $\Spec A \times \bA^1 = \Spec (A \otimes R[X]) = \Spec A[X]$
  (see~\ref{MISSING}),
  we can regard $g$ as an element of $A[X]$.
  For $x \in \Spec A$,
  the polynomial $\iota(f(x)) \in R[X]$ is then obtained from the polynomial~$g$
  by applying the $R$-algebra homomorphism $x : A \to R$ to all coefficients.
  This makes it clear that
  we have a common bound on the degrees of the polynomials $\iota(f(x))$,
  in other words,
  the function $f$ is bounded.
\end{proof}


\section{Topology of schemes}
\label{topology-of-schemes}
Analogous to synthetic algebraic geometry,
we use pointwise and local definitions of open subsets,
which agree if we assume a corresponding choice axiom.


\section{Schemes}
In \cref{schemes} we defined a scheme to be a type $X$ such that $X$
may be covered by finitely many open affine subtypes.
In this section, we will present general properies of schemes and a couple of common constructions for schemes.


\subsection{General Properties}

\begin{lemma}[using \axiomref{sqc}, \axiomref{loc}]%
  \label{intersection-of-all-opens}
  Let $X$ be a scheme and $x:X$, then for all $y:X$ the proposition
  \[ \prod_{U:X\to \Open}U(x)\to U(y) \]
  is equivalent to $\neg\neg (x=y)$.
\end{lemma}

\begin{proof}
  By \cref{open-union-intersection},
  open proposition are always double-negation stable,
  which settles one implication.
  For the implication
  \[ \left(\prod_{U:X\to \Open}U(x)\to U(y)\right) \Rightarrow \neg\neg (x=y) \]
  we can assume that $x$ and $y$ are both inside an open affine $U$
  and use that the statement holds for affine schemes by \cref{affine-intersection-of-all-opens}.
\end{proof}

\subsection{Glueing}

\begin{proposition}[using \axiomref{loc}, \axiomref{sqc}, \axiomref{Z-choice}]%
  Let $X,Y$ be schemes and $f:U\to X$, $g:U\to Y$ be embeddings with open images in $X$ and $Y$,
  then the pushout of $f$ and $g$ is a scheme.
\end{proposition}

\begin{proof}
  As we noted in \cref{MISSING}, such a pushout is always 0-truncated.
  Let $U_1,\dots,U_n$ be a cover of $X$ and $V_1,\dots,V_m$ be a cover of $Y$.
  By \cref{qc-open-trans}, $U_i\cap U$ is open in $Y$,
  so we can use (large) pushout-recursion to construct a subtype $\tilde{U_i}$,
  which is open in the pushout and restricts to $U_i$ on $X$ and $U_i\cap U$ on $Y$.
  Symetrically we define $\tilde{V_i}$ and in total get an open finite cover of the pushout.
  The pieces of this new cover are equivalent to their counterparts in the covers of $X$ and $Y$,
  so they are affine as well.
\end{proof}

\subsection{Subschemes}

\begin{definition}
  Let $X$ be a scheme.
  A \notion{subscheme} of $X$ is a subtype $Y:X\to\Prop$,
  such that $\sum Y$ is a scheme.
\end{definition}

\begin{proposition}[using \axiomref{loc}, \axiomref{sqc}, \axiomref{Z-choice}]%
  \label{open-subscheme}
  Any open subtype of a scheme is a scheme.
\end{proposition}

\begin{proof}
  Using \cref{qc-open-affine-open}.
\end{proof}

\begin{proposition}[using \axiomref{sqc}, \axiomref{loc}, \axiomref{Z-choice}]%
  \label{closed-subscheme}
  Any closed subtype $A:X\to \Prop$ of a scheme $X$ is a scheme.
\end{proposition}

\begin{proof}
  Any open subtype of $X$ is also open in $A$.
  So it is enough to show,
  that any affine open $U_i$ of $X$,
  has affine intersection with $A$.
  But $U_i\cap A$ is closed in $U_i$ and therefore affine by \cref{closed-subtype-affine}.
\end{proof}

We can extend the operation from \cref{affine-n-th-inf-neighborhood}
to schemes:

\begin{definition}[using \axiomref{sqc}, \axiomref{loc}, \axiomref{Z-choice}]%
  Let $X$ be a scheme and $C\subseteq X$ a closed subscheme.
  Then $C^n$ is the closed subscheme of $X$,
  defined locally as in \cref{affine-n-th-inf-neighborhood}.
\end{definition}

\begin{proof}
  We need the axioms to locally get ideals that generate the closed subscheme.
  We need to show that the construction can be done locally,
  but this is the case, since for any open affine $U$,
  $(C\cap U)^n\subseteq \neg\neg (C\cap U)\subseteq U$ by \cref{MISSING}.
\end{proof}

\begin{lemma}[using \axiomref{sqc}, \axiomref{loc}, \axiomref{Z-choice}]%
  \label{dense-closed-n-th-neighborhood}
  Let $U\subseteq X$ be a dense open subtype of a scheme.
  For any closed subtype $V$ containing $U$,
  there merely is an $n:\N$, such that $V^n=X$.
\end{lemma}

\begin{proof}
  It is enough to do the construction for an open affine $W=\Spec A$,
  where $V\cap U=\Spec A/(f_1,\dots,f_n)$ and
  $U=D(g_1,\dots,g_l)$.
  By \cref{dense-is-jointly-nilregular} we can assume the $g_1,\dots,g_l$
  are jointly nilregular.
  For any $f_i$ we know $f_i\cdot g_j$ is nilpotent, since
  \[ \neg D(f_ig_j)=\neg\{x:W\mid f_ig_i(x)\text{ invertible}\} =\emptyset\rlap{,}\]
  since if $f_ig_j(x)$ is invertible, then $g_j(x)$ is invertible, but then, we are in
  $U$ and $f_i(x)$ has to be zero, which contradicts its invertibility.

  By the joint nilregularity of the $g_j$, $f_i$ is nilpotent,
  so $f_i^n=0$ and $V^n=W$.
\end{proof}

In the situation of a clopen subset, we get the classical equality:

\begin{lemma}[using \axiomref{sqc}, \axiomref{loc}, \axiomref{Z-choice}]%
  \label{clopen-dense-is-all}
  Let $U\subseteq X$ be a dense open and closed subtype of a scheme,
  then $U=X$
\end{lemma}

\begin{proof}
  By \cref{dense-closed-n-th-neighborhood},
  $U^n=X$.
  By \cref{affine-n-th-inf-neighborhood-formal}, we have
  \[
    X\subseteq U^n\subseteq \neg\neg U = U\rlap{.}
  \]
\end{proof}

\subsection{Equality types}

\begin{lemma}%
  \label{affine-equality-closed}
  Let $X$ be an affine scheme and $x,y:X$,
  then $x=_Xy$ is an affine scheme
  and $((x,y):X\times X)\mapsto x=_Xy$
  is a closed subtype of $X\times X$.
\end{lemma}

\begin{proof}
  Any affine scheme is merely embedded into $\A^n$ for some $n:\N$.
  The proposition $x=y$ for elements $x,y:\A^n$ is equivalent to $x-y=0$,
  which is equivalent to all entries of this vector being zero.
  The latter is a closed proposition.
\end{proof}

\begin{proposition}[using \axiomref{sqc}, \axiomref{loc}, \axiomref{Z-choice}]%
  \label{equality-scheme}
  Let $X$ be a scheme.
  The equality type $x=_Xy$ is a scheme for all $x,y:X$.
\end{proposition}

\begin{proof}
  Let $x,y:X$ and
  $U\subseteq X$ be an affine open containing $x$.
  Then $U(y)\wedge x=y$ is equivalent to $x=y$, so it is enough to show that $U(y)\wedge x=y$ is a scheme.
  As a open subscheme of the point, $U(y)$ is a scheme and $(x:U(y))\mapsto x=y$ defines a closed subtype by \cref{affine-equality-closed}.
  But this closed subtype is a scheme by \cref{closed-subscheme}.
\end{proof}

\subsection{Dependent sums}

\begin{theorem}[using \axiomref{loc}, \axiomref{sqc}, \axiomref{Z-choice}]%
  \label{sigma-scheme}
  Let $X$ be a scheme and for any $x:X$, let $Y_x$ be a scheme.
  Then the dependent sum
  \[ \left((x:X)\times Y_x\right)\equiv \sum_{x:X}Y_x\]
  is a scheme.
\end{theorem}

\begin{proof}
  We start with an affine $X=\Spec A$ and $Y_x=\Spec B_x$.
  Locally on $U_i = D(f_i)$, for a Zariski-cover $f_1,\dots,f_l$ of $X$,
  we have $B_x=\Spec R[X_1,\dots,X_{n_i}]/(g_{i,x,1},\dots,g_{i,x,m_i})$
  with polynomials $g_{i,x,j}$.
  In other words, $B_x$ is the closed subtype of $\A^{n_i}$
  where the functions $g_{i,x,1},\dots,g_{i,x,m_i}$ vanish.
  By \cref{affine-fiber-product}, the product
  \[ V_i\colonequiv U_i\times \A^{n_i}\]
  is affine.
  The type $(x:U_i)\times \Spec B_x\subseteq V_i$ is affine,
  since it is the zero set of the functions
  \[ ((x,y):V_i)\mapsto g_{i,x,j}(y) \]
  Furthermore, $W_i\colonequiv (x:U_i)\times \Spec B_x$
  is open in $(x:X)\times Y_x$,
  since $W_i(x)$ is equivalent to $U_i(\pi_1(x))$,
  which is an open proposition.

  This settles the affine case.
  We will now assume, that
  $X$ and all $Y_x$ are general schemes.
  We pass again to a cover of $X$ by affine open $U_1,\dots,U_n$.
  We can choose the latter cover,
  such that for each $i$ and $x:U_i$, the $Y_{\pi_1(x)}$
  are covered by $l_i$ many open affine pieces $V_{i,x,1},\dots,V_{i,x,l_i}$
  (by \cref{boundedness}).
  Then $W_{i,j}\colonequiv(x:U_i)\times V_{i,x,j}$ is affine by what we established above.
  It is also open.
  To see this, let $(x,y):((x:X)\times Y_x)$.
  We want to show, that $(x,y)$ being in $W_{i,j}$ is an open proposition.
  We have to be a bit careful, since the open proposition
  $V_{i,x,j}$ is only defined, for $x:U_i$.
  So the proposition we are after is $(z:U_i(x,y))\times V_{i,z,j}(y)$.
  But this proposition is open by \cref{qc-open-sigma-closed}.
\end{proof}

\begin{corollary}
  \label{scheme-map-classification}
  Let $X$ be a scheme.
  For any other scheme $Y$ and any map $f:Y\to X$,
  the fiber map
  $(x:X)\mapsto \fib_f(x)$
  has values in the type of schemes $\Sch$.
  Mapping maps of schemes to their fiber maps,
  is an equivalence of types
  \[ \left(\sum_{Y:\Sch}(Y\to X)\right)\simeq (X\to \Sch)\rlap{.}\]
\end{corollary}

\begin{proof}
  By univalence, there is an equivalence
  \[ \left(\sum_{Y:\Type}(Y\to X)\right)\simeq (X\to \Type)\rlap{.} \]
  From left to right, the equivalence is given by turning a $f:Y\to X$ into $x\mapsto \fib_f(x)$,
  from right to left is given by taking the depedent sum.
  So we just have to note, that both constructions preserve schemes.
  From left to right, this is \cref{fiber-product-scheme}, from right to left,
  this is \cref{sigma-scheme}.
\end{proof}

Subschemes are classified by propositional schemes:

\begin{corollary}
  Let $X$ be a scheme.
  $Y:X\to\Prop$ is a subscheme,
  if and only if $Y_x$ is a scheme for all $x:X$.
\end{corollary}

\begin{proof}
  Restriction of \cref{scheme-map-classification}.
\end{proof}

\subsection{Pullbacks of Schemes}

In this section, we will show in two different ways,
that the pullback of a cospan of schemes is a scheme.
The first poof is very short and reuses what we proved about equality and sigma-types,
the second proof is more direct, uses the proof of the affine case \cref{affine-fiber-product}
and is along the lines of what one might find in an algebraic geometry textbook.

\begin{theorem}[using \axiomref{loc}, \axiomref{sqc}, \axiomref{Z-choice}]%
  \label{fiber-product-scheme}
  Let
  \[
    \begin{tikzcd}
      X\ar[r,"f"] & Z & Y\ar[l,swap,"g"]
    \end{tikzcd}
  \]
  be schemes, then the \notion{pullback} $X\times_Z Y$ is also a scheme.
\end{theorem}

\begin{proof}
  The type $X\times_Z Y$ is given as the following, interated dependent sum:
  \[ \sum_{x:X}\sum_{y:Y}f(x)=g(y)\rlap{.}\]
  The innermost type, $f(x)=g(y)$
  is the equality type in the scheme $Z$ and by \cref{equality-scheme} a scheme.
  By applying \cref{sigma-scheme} twice, we prove that the itereated dependent sum is a scheme.
\end{proof}

We conclude with a construction, analogous to the classical treatment:

\begin{proof}[alternative proof of \cref{fiber-product-scheme}]
  Let $W_1,\dots,W_n$ be a finite affine cover of $Z$.
  The preimages of $W_i$ under $f$ and $g$ are open
  and covered by fintely many affine open $U_{ik}$ and $V_{ij}$ by \cref{open-subscheme}.
  This leads to the following diagram:
  \begin{center}
    \begin{tikzcd}
      X\times_Z Y\ar[rrr]\ar[ddd] & & & Y\ar[ddd] & \\
      & P_{ij}\ar[hook,lu]\ar[rrr,crossing over] &&& V_{ij}\ar[hook,lu]\ar[ddd] \\
      &&&& \\
      X\ar[rrr] & & & Z & \\
      & U_{i}\ar[rrr]\ar[hook,lu]\ar[from=uuu,crossing over] & & & W_i\ar[hook,lu]
    \end{tikzcd}
  \end{center}
  where the front and bottom square are pullbacks by definition.
  By pullback-pasting, the top is also a pullback,
  so all diagonal maps are embeddings.
  
  $P_{ij}$ is open, since it is a preimage of $V_{ij}$ (\cref{preimage-open}),
  which is open in $Y$ by \cref{qc-open-trans}.
  It remains to show, that the $P_{ij}$ cover $X\times_Z Y$ and that $P_{ij}$ is a scheme.
  Let $x:X\times_Z Y$.
  For the image $w$ of $x$ in $W$, there merely is an $i$ such that $w$ is in $W_i$.
  The image of $x$ in $V_i$ merely lies in some $V_{ij}$,
  so $x$ is in $P_{ij}$.

  We proceed by showing that $P_{ij}$ is a scheme.
  Let $U_{ik}$ be a part of the finite affine cover of $U_i$.
  We repeat part of what we just did:
  \begin{center}
    \begin{tikzcd}
      P_{ij}\ar[rrr]\ar[ddd] & & & U_i\ar[ddd] & \\
      & P_{ijk}\ar[hook,lu]\ar[rrr,crossing over] &&& U_{ik}\ar[hook,lu]\ar[ddd] \\
      &&&& \\
      V_{ij}\ar[rrr] & & & W_i & \\
      & V_{ij}\ar[rrr]\ar[equal,lu]\ar[from=uuu,crossing over] & & & W_i\ar[equal,lu]
    \end{tikzcd}
  \end{center}

  So by \cref{affine-fiber-product}, $P_{ijk}$ is affine.
  Repetition of the above shows, that the $P_{ijk}$ are open and cover $P_{ij}$.
\end{proof}


\subsection{Line bundles}

\begin{definition}%
  Let $X$ be a type.
  A \notion{line bundle} is a map $\mathcal L : X\to \Mod{R}$,
  such that
  \[ \prod_{x:X} \propTrunc{\mathcal L_x=_{\Mod{R}}R} \rlap{.}\]
  The \notion{trivial line bundle} on $X$ is the line bundle
  $X \to \Mod{R}, x \mapsto R$,
  and when we say that a line bundle $\mathcal{L}$ is trivial
  we mean that $\mathcal{L}$ is equal to the trivial line bundle,
  or equivalently $\propTrunc{\prod_{x:X} \mathcal L_x=_{\Mod{R}}R}$.
\end{definition}

\begin{lemma}[using \axiomref{loc}, \axiomref{sqc}]%
  \label{polynomials-notnot-decompose}
  Let $f : R[X]$ be a polynomial.
  Then it is not not the case that:
  either $f = 0$ or
  $f = \alpha \cdot {(X - a_1)}^{e_1} \dots {(X - a_n)}^{e_n}$
  for some $\alpha : R^\times$,
  $e_i : \N$ and pairwise distinct $a_i : R$.
\end{lemma}

\begin{proof}
  Let $f : R[X]$ be given.
  Since our goal is a proposition,
  we can assume we have a bound $n$ on the degree of $f$,
  so
  \[ f = \sum_{i = 0}^n c_i X^i \rlap{.} \]
  Since our goal is even double-negation stable,
  we can assume $c_n = 0 \lor c_n \neq 0$
  and by induction $f = 0$ (in which case we are done)
  or $c_n \neq 0$.
  If $n = 0$ we are done,
  setting $\alpha \colonequiv c_0$.
  Otherwise,
  $f$ is not invertible (using $0 \neq 1$ by (\axiomref{loc})),
  so $R[X]/(f) \neq 0$,
  which by (\axiomref{sqc}) means that
  $\Spec(R[X]/(f)) = \{ x : R \mid f(x) = 0 \}$
  is not empty.
  Using the double-negation stability of our goal again,
  we can assume $f(a) = 0$ for some $a : R$
  and factor $f = (X - a_1) f_{n - 1}$.
  By induction, we get $f = \alpha \cdot (X - a_1) \dots (X - a_n)$.
  Finally, we decide each of the finitely many propositions $a_i = a_j$,
  which we can assume is possible
  because our goal is still double-negation stable,
  to get the desired form
  $f = \alpha \cdot {(X - \widetilde{a}_1)}^{e_1} \dots {(X - \widetilde{a}_n)}^{e_n}$
  with distinct $\widetilde{a}_i$.
\end{proof}

% \begin{lemma}[using \axiomref{loc}, \axiomref{sqc}]
%   For any $f, g : R[X]$
%   there does not not exist a polynomial $\gcd(f, g) : R[X]$
%   with $(f, g) = (\gcd(f, g)) \subseteq R[X]$.
% \end{lemma}
%
% \begin{proof}
%   TODO
% \end{proof}

\begin{lemma}[using \axiomref{loc}, \axiomref{sqc}, \axiomref{Z-choice}]
  For every open subset $U : \A^1 \to \Prop$ of $\A^1$
  we have not not:
  either $U = \emptyset$
  or $U = \A^1 \setminus \{ a_1, \dots, a_n \}$
  for pairwise distinct numbers $a_1, \dots, a_n : R$.
\end{lemma}

\begin{proof}
  For $U = D(f)$,
  this is just \cref{polynomials-notnot-decompose}.
  In general we have $U = D(f_1) \cup \dots \cup D(f_n)$
  by \cref{qc-open-affine-open},
  so we do not not get
  (that $U = \emptyset$ or)
  a list of elements $a_1, \dots, a_n : R$
  such that $U = \A^1 \setminus \{ a_1, \dots, a_n \}$.
  Then we can not not get rid of any duplicates in the list.
\end{proof}

\begin{lemma}[using \axiomref{loc}, \axiomref{sqc}, \axiomref{Z-choice}]
  Let $U, V : \A^1 \to \Prop$ be two open subsets
  and let $f : U \cap V \to R^\times$ be a function.
  Then there do not not exist functions
  $g : U \to R^\times$ and
  $h : V \to R^\times$
  such that $f(x) = g(x)h(x)$ for all $x : U \cap V$.
\end{lemma}

\begin{proof}
  TODO
\end{proof}

\begin{theorem}[using \axiomref{loc}, \axiomref{sqc}, \axiomref{Z-choice}]
  Every line bundle on $\A^1$ is not not trivial.
\end{theorem}

\begin{proof}
  \dots
\end{proof}

In classical algebraic geometry,
there is the concept of a \notion{generic section} of a line bundle.
Informally, the generic sections have the smallest possible vanishing set.
The following definition corresponds to this notion:

\begin{definition}%
  \label{regular-section}
  Let $X$ be a type and $\mathcal L:X\to \Mod{R}$ a line bundle.
  A section
  \[ s:\prod_{x:X}\mathcal L_x \]
  is \notion{regular}, there merely is a trivializing affine cover $U_1=\Spec A_1,\dots,U_n=\Spec A_n$
  of $\mathcal L$, such that each trivialized restriction
  \[ s_i:\Spec A_i\to R \]
  is a regular element (\cref{regular-element}) of $(\Spec A_i\to R) = A_i$.
\end{definition}

\begin{lemma}%
  \label{regular-zariski-local}
  Let $s:\Spec A\to R$.
  $s$ being regular is Zariski-local, i.e.
  for all Zariski-covers $U_1,\dots,U_n$ of $\Spec A$,
  $s$ is regular, if and only if it is regular on all $U_i$.
\end{lemma}

\begin{proof}
  It is enough to check this for a localization at $f:A$.
  Let
  \[ \frac{s}{1}\cdot\frac{g}{f^k}=0\rlap{.} \]
  then $f^lsg=0$, which implies $f^lg=0$ by regularity of $s$ and therefore $\frac{g}{f^l}=0$.
\end{proof}

\begin{proposition}%
  The choice of trivializing cover in \cref{regular-section}
  is irrelevant.
\end{proposition}

\begin{proof}
  By \cref{regular-zariski-local}.
\end{proof}

From a line bundle together with a regular section,
we can produce a closed subtype of a special kind:

\begin{definition}%
  Let $X$ be a scheme.
  A \notion{regular closed subtype} of $X$ is a closed subtype
  $C:X\to \Prop$, such that there merely is an affine open cover $U_1=\Spec A_1,\dots,U_n=\Spec A_n$,
  and $C\cap U_i$ is $V(f_i)$ for a regular $f_i:A_i$.
\end{definition}

\begin{lemma}%
  Let $f,g:A$, $f$ be regular and $V(f)=V(g)$,
  then $g$ is regular and there is a unique unit $\alpha:A^\times$, such that $\alpha f=g$.
\end{lemma}

\begin{proof}
  $V(f)=V(g)$ implies there are $\alpha,\beta:A$ such that
  $\alpha f = g$ and $\beta g = f$.
  But then: $f=\beta g=\beta\alpha f$.
  So by regularity of $f$, $\beta\alpha=1$.
  By \cref{units-products-regular}, units are regular and products of regular elements are regular,
  so $g$ is regular.
  Uniqueness of $\alpha$ follows from regularity.
\end{proof}

\begin{theorem}[using \axiomref{Z-choice}]%
  Let $X$ be a scheme.
  For any regular closed subscheme $C$,
  there is a line bundle with regular section $(\mathcal L,s)$ on $X$,
  such that $C=V(s)$.
\end{theorem}

\begin{proof}
  Let $U_1=\Spec A_1,\dots,U_n=\Spec A_n$ be a cover by standard  affine opens such that we have
  regular $f_i$ with $C\cap U_i=V(f_i)$. 
  We define $\mathcal L$ to be the trivial line bundle $\_\mapsto R$ on each $U_i$
  and by giving automorphisms on the intersections $U_i\cap U_j\colonequiv U_{ij}=\Spec A_{ij}$.
  On $U_{ij}$, $C$ is given by $V(\frac{f_i}{1})$ and $V(\frac{f_j}{1})$ which are both regular.
  Therefore, there is a unit $\alpha:A_{ij}^\times$ such that $\alpha\frac{f_i}{1}=\frac{f_j}{1}$,
  which we can also view as a map $U_{ij}\to R^\times$ and since $R^\times$
  is equivalent to the automorphism group of $R$ as an $R$-module,
  this provides the identetification we need to construct $\mathcal L$.
  Under the identification, the local regular sections are identified, so we get a global section $s$ of $\mathcal L$,
  which is locally regular.
\end{proof}



\section{Projective space}

We give two definitions of projective space, which differ only in size.

\begin{definition}
  \begin{enumerate}
  \item An $n$-dimensional $R$-\notion{vector space} is an $R$-module $V$,
    such that $\| V = R^n \|$. 
  \item We write $\Vect{R}{n}$ for the type of these vector spaces and $V\setminus\{0\}$ for the type
    \[ \sum_{x:V}x\neq 0\]
  \item A \notion{vector bundle} on a type $X$ is a map $V:X\to \Vect{R}{n}$. 
  \end{enumerate}
\end{definition}

The following defines projective space as the space of lines in a vector space.
This is a large type.
We will see below, that there is also a small definition of the same type.

\begin{definition}
  \begin{enumerate}
  \item   A \notion{line} in a $R$-vector space $V$ is a subtype $\mathcal L:V\to \Prop$,
    such that there exists an $x:V\setminus\{0\}$ with
    \[ \prod_{y:V}\left(\mathcal L (y) \Leftrightarrow \exists c:R.y=c\cdot x\right)\]
  \item The space of all lines in a fixed $n$-dimensional vector space $V$ is the projectivization of $V$:
    \[ \bP(V)\colonequiv \sum_{\mathcal L:V\to \Prop} \mathcal L \text{ is a line}  \]
  \item \notion{Projective $n$-space} is the projectivization of $\bA^{n+1}$,
    $\bP^n \colonequiv \bP(\bA^{n+1})$.
  \end{enumerate}
\end{definition}

Lines are closed subschemes (not defined yet):

\begin{proposition}
  For any line $\mathcal L : \bA^{2}\to \Prop$, there is a degree one polynomial $P\in R[X_0,X_1]$ such that
  for all $x:\bA^{2}$, $\mathcal L(x)$ is equivalent to $P(x)=0$.
\end{proposition}
\begin{proof}
  Take $P$ to be the polynomial, given by inner product 
\end{proof}

\ignore{
\begin{definition}
  Let $n:\NN$. Projective $n$-space is the type given 
\end{definition}
}

\begin{theorem}[\axiomref{sqc},\axiomref{loc}]
  $\bP^n$ is a scheme.
\end{theorem}

\begin{proof}
  \dots
\end{proof}

\begin{lemma}
  All functions $\bP^1 \to R$ are constant.
\end{lemma}

\begin{proof}
  \dots
\end{proof}

\begin{lemma}[using (\axiomref{sqc}), (\axiomref{loc})]
  Let $p \neq q \in \bP^n$ be given.
  Then there exists a map $f : \bP^1 \to \bP^n$
  such that $f([0 : 1]) = p$, $f([1 : 0]) = q$.
\end{lemma}

\begin{proof}
  What we want to prove is a proposition,
  so we can assume chosen $a, b \in \bA^{n+1} \setminus \{0\}$
  with $p = [a]$, $q = [b]$.
  Then we set
  \[ f([x, y]) \colonequiv [xa + yb] \rlap{.}\]
  Let us check that $xa + yb \neq 0$.
  By \dots,
  we have that $x$ or $y$ is invertible
  and both $a$ and $b$ have at least one invertible entry.
  If $xa = - yb$
  then it follows that $x$ and $y$ are both invertible
  and therefore $a$ and $b$ would be linearly equivalent,
  contradicting the assumption $p \neq q$.
  Of course $f$ is also well-defined
  with respect to linear equivalence in the pair $(x, y)$.
\end{proof}

\begin{lemma}
  Let $n \geq 1$.
  For every point $p \in \bP^n$,
  we have $p \neq [1 : 0 : 0 : \dots]$
  or $p \neq [0 : 1 : 0 : \dots]$.
\end{lemma}

\begin{proof}
  Let $p = [a]$ with $a \in \bA^{n+1} \setminus \{0\}$.
  By \dots,
  there is an $i \in \{0, \dots, n\}$ with $a_i \neq 0$.
  If $i = 0$ then $p \neq [0 : 1 : 0 : \dots]$,
  if $i \geq 1$ then $p \neq [1 : 0 : 0 : \dots]$.
\end{proof}

\begin{theorem}
  All functions $\bP^n \to R$ are constant,
  that is,
  \[ H^0(\bP^n, R) \colonequiv (\bP^n \to R) = R \rlap{.} \]
\end{theorem}

\begin{proof}
  \dots
\end{proof}


\section{Bundles and cohomology}
In non-synthetic algebraic geometry,
the structure sheaf~$\mathcal{O}_X$ is part of the data constituting a scheme~$X$.
In our internal setting,
the scheme $X$ is just a set without any additional data,
but when we want to consider the structure sheaf as an object in its own right,
then we can represent it by the trivial bundle
that assings to every point $x : X$ the set $R$.
Indeed, for an affine scheme $X = \Spec A$,
taking the sections of this bundle over a basic open $D(f) \subseteq X$
\[ (\prod_{x : D(f)} R) = (D(f) \to R) = A[f^{-1}] \]
yields the localizations of the ring $A$
expected from the structure sheaf $\mathcal{O}_X$.
More generally,
instead of sheaves of abelian groups, $\mathcal{O}_X$-modules, etc.,
we will consider bundels of abelian groups, $R$-modules, etc.,
in the form of maps from $X$ to the respective type of algebraic structures.

\subsection{Quasi-coherent bundles}

This subsection is still experimental.

Sometimes we want to ``apply'' a bundle to a subtype,
like sheaves can be evaluated on open subspaces
and introduce the common notation ``$M(U)$'' for that below.
It is, however, not justified to expect, that this application
and the corresponding theory of ``sheaves'' is ``the same'' as the external one,
since the definition below, uses the internal hom ``$\prod$''
-- where the corresponding external construction, would be the set of continuous sections of a bundle.

\begin{definition}
  \index{$M(U)$}
  Let $X$ be a type and $M:X\to \Mod{R}$ a dependent module.
  Let $U\subseteq X$ be any subtype.
  \begin{enumerate}[(a)]
  \item We write:
    \[
      M(U)\colonequiv \prod_{x:U}M_x
      \rlap{.}
    \]
  \item With pointwise structure, $U\to R$ is an $R$-algebra
    and $M(U)$ is a $(U\to R)$-module.
  \end{enumerate}
\end{definition}

Somewhat surprisingly, localization of modules $M(U)$
can be done pointwise:

\begin{lemma}[using \axiomref{loc}, \axiomref{sqc}, \axiomref{Z-choice}]%
  \label{module-bundle-localization-pointwise}
  Let $X$ be a scheme and $M:X\to \Mod{R}$ a dependent module.
  Let $U=\Spec A\subseteq X$ be open affine.
  Let $f:A$.
  \begin{enumerate}[(a)]
  \item There is a morphism
    \[
      M(U)_f\to \prod_{x:U}(M_x)_{f(x)}
      \rlap{.}
    \]
  \item Let $g,h:M(U)_f$. Then $g=h$ if and only if
    \[
      \prod_{x:U}g(x)=_{(M_x)_{f(x)}}h(x)
      \rlap{.}
    \]
  \item The morphism in (a) is an equivalence, i.e.
    \[
      M(U)_f=\prod_{x:U}(M_x)_{f(x)}
      \rlap{.}
    \]
  \end{enumerate}
\end{lemma}

\begin{proof}
  \begin{enumerate}[(a)]
  \item We have to show, that the map
    \[
      \frac{m}{f^k}\mapsto\left(x\mapsto \frac{m(x)}{f(x)^k}\right)
    \]
    is well-defined. So let $\frac{m}{f^k}=\frac{m'}{f^{k'}}$,
    i.e. let there be an $l:\N$ such that $f^l(mf^{k'}-m'f^k)=0$.
    But then we can choose the same $l:\N$ for each $x:U$
    and apply the equation to each $x:U$.
  \item The forward direction was treated in (a).
    So let $g,h:M(U)_f$ such that $p:\prod_{x:U}g(x)=_{(M_x)_{f(x)}}h(x)$.
    Let $m_g,m_h:\prod_{x:U} M_x$ and $k_g,k_h:\N$ such that
    \[
      g=\frac{m_g}{f^{k_g}} \quad\text{and}\quad h=\frac{m_h}{f^{k_h}}
      \rlap{.}
    \]
    From $p$ we know $\prod_{x:U}\exists_{k_x:\N}f(x)^{k_x}(m_g(x)f(x)^{k_h}-m_h(x)f(x)^{k_g})=0$.
    By \cref{strengthened-boundedness},
    we find one $k : \N$ with
    \[
      \prod_{x:U}f(x)^{k}(m_g(x)f(x)^{k_h}-m_h(x)f(x)^{k_g})=0
    \]
    --- which shows $g=h$.
  \item The map in (a) is injective by (b);
    it remains to show that it is surjective.
    So let $\varphi:\prod_{x:U}(M_x)_{f(x)}$ and
    note that
    \[
      \prod_{x:U}
      \exists_{k_x:\N,m_x:M_x}
      \varphi(x)=\frac{m_x}{f(x)^{k_x}}
      \rlap{.}
    \]
    By \cref{strengthened-boundedness} and \axiomref{Z-choice},
    we get $k:\N$, coprime $a_1,\dots,a_l:A$ and $m_i:(x : D(a_i))\to M_x$
    such that for each $i$ and $x:D(a_i)$ we have
    \[
      \varphi(x)=\frac{m_i(x)}{f(x)^{k}}
      \rlap{.}
    \]
    The problem is now to construct a global $m:(x:U)\to M_x$ from the $m_i$.
    We have
    \[
        \prod_{x:D(a_ia_j)}\frac{m_i(x)}{f(x)^k}=\varphi(x)=\frac{m_j(x)}{f(x)^k}
    \]
    meaning there is pointwise an exponent $t_x:\N$,
    such that $f(x)^{t_x}m_i(x)=f(x)^{t_x}m_j(x)$.
    By \cref{strengthened-boundedness},
    we can find a single $t:\N$ with this property and define
    \[
      \tilde{m}_i(x) \colonequiv f(x)^t m_i(x)
      \rlap{.}
    \]
    Then we have $\tilde{m}_i(x)=\tilde{m}_j(x)$ on all intersections $D(a_i)\cap D(a_j)$,
    which is what we need to get a global $m:(x:U)\to M_x$ from \cref{kraus-glueing}.
    Since $\varphi(x)=\frac{f(x)^t m_i(x)}{f(x)^{t+k}}=\frac{\tilde{m}_i(x)}{f(x)^{t+k}}$
    for all $i$ and $x : D(a_i)$,
    we have found a preimage of $\varphi$ in $M(U)_f$.
  \end{enumerate}
\end{proof}

We will need the following algebraic lemma:

\begin{lemma}%
  \label{localization-to-module-if-non-zero}
  Let $M$ be an $R$-module and $f:R$,
  then there is an $R$-linear map
  \[
    M_f\to M^{D(f)}
    \rlap{.}
  \]
\end{lemma}

\begin{proof}
  Let $x\equiv \frac{m}{f^k}:M_f$ and $p:D(f)$.
  Then $f$ is invertible, so we have
  \[
    x\equiv \frac{m}{f^k}=\frac{f^{-k}m}{1}
  \]
  and mapping $x$ to $f^{-k}m$ is an $R$-linear map.
  
\end{proof}

\begin{lemma}[using \axiomref{sqc}, \axiomref{loc}, \axiomref{Z-choice}]%
  \label{localization-to-restriction}                    
  Let $X$ be a scheme, $M:X\to\Mod{R}$, $U=\Spec A\subseteq X$ open and $f:A$.
  Then there is an $R$-linear map
  \[
    M(U)_f \to M(D(f)) 
    \rlap{.}
  \]
\end{lemma}

\begin{proof}
  Combining \cref{module-bundle-localization-pointwise}
  and pointwise application of \cref{localization-to-module-if-non-zero} we get
  \[
    M(U)_f=\left(\prod_{x:U}(M_x)_{f(x)}\right)\to \left(\prod_{x:U}(M_x)^{D(f(x))}\right)
    =\left(\prod_{x:D(f)}M_x\right)
    =M(D(f))
  \]
\end{proof}

The following is an experimental definition,
which might be suitable
to mimic the external notion of quasi-coherent $\mathcal O_X$-module sheaves.

\begin{definition}%
  \label{quasi-coherent-bundle}
  Let $X$ be a scheme.
  A dependent module $M:X\to \Mod{R}$ is \notion{quasi-coherent},
  if for all $x:X$ and $f:R$,
  the canonical map from \cref{localization-to-module-if-non-zero} is an equivalence:
  \[
    (M_x)_f\simeq M_x^{D(f)}
    \rlap{.}
  \]
\end{definition}

An immediate consequence is, that
quasi coherent dependent modules have
the property that ``restricting is the same as localizing'':

\begin{lemma}[using \axiomref{sqc}, \axiomref{loc}, \axiomref{Z-choice}]
  Let $X$ be a scheme and $M:X\to \Mod{R}$ quasi-coherent,
  then for all open affine $U=\Spec A\subseteq X$ and $f:A$
  the canonical morphism
  \[
    M(U)_f\to M(D(f))
  \]
  is an equivalence.
\end{lemma}

\begin{proof}
  By construction of the canonical map from \cref{localization-to-restriction}.
\end{proof}

Let us look at an example.

\begin{proposition}
  \label{fp-algebra-bundle-is-quasi-coherent}
  Let $X$ be a scheme and $C:X\to \Alg{R}_{fp}$.
  Then $C$, as a bundle of $R$-modules, is quasi coherent.
\end{proposition}

\begin{proof}
  Then for any $f:R$ and $x:X$, using \cref{algebra-valued-functions-on-affine}, we have
  \[
    (C_x)_f=C_x\otimes_R R_f=(\Spec R_f \to C_x)=(D(f)\to C_x)={C_x}^{D(f)}
    \rlap{.}
  \]
\end{proof}

\begin{proposition}[using \axiomref{loc}, \axiomref{sqc}, \axiomref{Z-choice}]
  Not every $R$-module is quasi-coherent
  in the sense of \cref{quasi-coherent-bundle}.
\end{proposition}

\begin{proof}
  We construct a family of $R$-modules,
  parametrized by the elements of $R$,
  and deduce a contradiction from the assumption that
  all modules of this family are quasi-coherent.

  Given an element $f : R$,
  the $R$-module we want to consider is
  the countable product
  \[ M(f) \colonequiv \prod_{n : \N} R/(f^n) \rlap{.} \]
  If $f \neq 0$ then $M(f) = 0$
  (using \cref{non-zero-invertible}).
  This implies that the $R$-module $M(f)^{f \neq 0}$
  is trivial:
  any function $f \neq 0 \to M(f)$ can only assign the value $0$
  to any of the at most one witnesses of $f \neq 0$.
  If $M(f)$ is quasi-coherent,
  then this means that $M(f)_f$ is also trivial.
  Noting that
  $M(f)$ is not only an $R$-module
  but even an $R$-algebra in a natural way,
  we have
  \begin{align*}
    M(f)_f = 0
    &\;\Leftrightarrow\;
    \exists k : \N.\; \text{$f^k = 0$ in $M(f)$} \\
    &\;\Leftrightarrow\;
    \exists k : \N.\; \forall n : \N.\; f^k \in (f^n) \subseteq R \\
    &\;\Leftrightarrow\;
    \exists k : \N.\; f^k \in (f^{k + 1}) \subseteq R
    \rlap{.}
  \end{align*}

  In summary,
  if the module $M(f)$ is quasi-coherent
  for every $f : R$,
  then the ring $R$ is zero-dimensional
  in the sense of \cref{zero-dimensional-ring}.
  But this is not the case,
  as we saw in \cref{R-not-zero-dimensional}.
\end{proof}

\begin{lemma}[using \axiomref{sqc}, \axiomref{loc}, \axiomref{Z-choice}]%
  \label{weakly-quasi-coherent-pi}
  Let $X$ be an affine scheme and $M_x$ a weakly quasi-coherent $R$-module for any $x:X$,
  then
  \[
    \prod_{x:X}M_x
  \]
  is weakly quasi-coherent.
\end{lemma}

\begin{proof}
  TODO
\end{proof}

Quasi-coherent dependent modules turn out to have very good properties,
which are to be expected from what is known about their external counterparts.
We will show below, that quasi coherence is preserved by the following constructions:

\begin{definition}
  \label{pullback-push-forward}
  Let $X,Y$ be types and $f:X\to Y$ be a map.
  \begin{enumerate}[(a)]
  \item \index{$f^*M$} For any dependent module $N:Y\to\Mod{R}$,
    the \notion{pullback} or \notion{inverse image} is the dependent module
    \[
      f^*N\colonequiv (x:X) \mapsto M_{f(x)}\rlap{.}
    \]
  \item \index{$f_*M$} For any dependent module $M:X\to\Mod{R}$,
    the \notion{push-forward} or \notion{direct image} is the dependent module
    \[
      f_*M\colonequiv (y:Y) \mapsto \prod_{x:\fib_f(y)}M_{\pi_1(x)}\rlap{.}
    \]
  \end{enumerate}
\end{definition}

\begin{theorem}[using \axiomref{sqc}, \axiomref{loc}, \axiomref{Z-choice}]%
  \label{pullback-push-forward-qcoh}
  Let $X,Y$ be schemes and $f:X\to Y$ be a map.
  \begin{enumerate}[(a)]
  \item For any quasi-coherent dependent module $N:Y\to\Mod{R}$,
    the inverse image $f^*N$ is quasi-coherent.
  \item For any dependent module $M:X\to\Mod{R}$,
    the direct image $f_*M$ is quasi-coherent.
  \end{enumerate}
\end{theorem}

\begin{proof}
  \begin{enumerate}[(a)]
  \item There is nothing to do, when we use the pointwise definition of quasi-coherence. 
  \item TODO, Ideas:

    Show that the dependent product of modules is a module.
    Show that this product preserves qcoh, if the index type is a scheme.
    Use that the fiber of a scheme morphism is a scheme.
  \end{enumerate}
\end{proof}

\subsection{Finitely presented bundles}

We now investigate the relationship between bundles of $R$-modules on $X = \Spec A$
and $A$-modules.

\begin{proposition}
  Let $A$ be a finitely presented $R$-algebra.
  There is an adjunction
  \[ \begin{tikzcd}[row sep=tiny]
    M \ar[r, mapsto] & {(M \otimes x)}_{x : \Spec A} \\
    \Mod{A} \ar[r, shift left=2] \ar[r, phantom, "\rotatebox{90}{$\vdash$}"] &
    \Mod{R}^{\Spec A} \ar[l, shift left=2] \\
    \prod_{x : \Spec A} N_x & N \ar[l, mapsto]
  \end{tikzcd} \]
  between the category of $A$-modules
  and the category of bundles of $R$-modules on $\Spec A$.
\end{proposition}

\begin{theorem}%
  \label{fp-module}
  Let $X=\Spec(A)$ be affine and
  let a bundle of finitely presented $R$-modules $M : X\to \fpMod{R}$ be given.
  Then the $A$-module
  \[ \tilde{M}\coloneqq\prod_{x:X}M_x \]
  is finitely presented and for any $x:X$ the $R$-module $\tilde{M}\otimes_A R$ is $M_x$.
  Under this correspondence, localizing $\tilde{M}$ at $f:A$ corresponds to restricting $M$ to $D(f)$.
\end{theorem}

\subsection{Cohomology on affine schemes}

\begin{definition}%
  \label{torsor}
  Let $X$ be a type and $A:X\to \AbGroup$ a map to the type of abelian groups.
  For $x:X$ let $T_x$ be a set with an $A_x$ action.
  \begin{enumerate}[(a)]
  \item $T$ is an \notion{$A$-pseudotorsor}, if the action is free and transitive for all $x:X$.
  \item $T$ is an \notion{$A$-torsor}, if it is an $A$-pseudotorsor and
    \[ \prod_{x:X} \| T_x \| \rlap{.}\]
  \item We write $\Tors{A}(X)$ for the type of $A$-torsors on $X$.
  \end{enumerate}
\end{definition}

Torsors on a point are a concrete implementaion of first deloopings:

\begin{definition}
  \label{delooping}
  Let $n:\N$.
  A $n$-th \notion{delooping}\index{$K(A,n)$} of an abelian group $A$,
  is a pointed, $(n-1)$-connected, $n$-truncated type $K(A,n)$,
  such that $\Omega^nK(A,n)=_{\AbGroup}A$.
\end{definition}

For any abelian group and any $n$, a delooping $K(A,n)$ exists by \cite{licata-finster}.
Deloopings can be used to represent cohomology groups by mapping spaces.
This is usually done in homotopy type theory to study higher inductive types, such as spheres and CW-complexes,
but the same approach works for internally representing sheaf cohomology,
which is the intent of the following definition:

\begin{definition}
  \label{cohomology}
  Let $X$ be a type and $\mathcal F:X\to\AbGroup$ a dependent abelian group.
  The $k$-th cohomology group of $X$ with coefficients in $\mathcal F$ is
  \[
    H^k(X,\mathcal F)\colonequiv \left\|\prod_{x:X}K(\mathcal F,k)\right\|_0\rlap{.}
  \]
\end{definition}


The following is an explicit formulation of the fact, that the Čech-Complex for an
$\mathcal{O}_X$-module sheaf on $X=\Spec(A)$ given by an $A$-module $M$ is exact in degree 1.
\begin{lemma}%
  \label{H1-algebra}
  Let $M$ be a module over a commutative ring $A$, $F_1,\dots,F_l$ a coprime system on $A$
  and for $i,j\in\{1,\dots,l\}$, let $s_{ij} : F_i^{-1} F_j^{-1} M$ such that:
  \[ s_{jk}-s_{ik}+s_{ij}=0 \rlap{.}\]
  Then there are $u_i:F_i^{-1}M$ such that $s_{ij}=u_j - u_i$.
\end{lemma}

\begin{proof}
  Let $s_{ij}=\frac{m_{ij}}{f_i f_j}$ with $m_{ij}:M$, $f_i:F_i$ and $f_j:F_j$ such that:
  \[ f_i\cdot m_{jk}-f_j\cdot m_{ik}+f_k\cdot m_{ij}=0 \rlap{.}\]
  Let $r_i$ such that $\sum r_i f_i =1$.
  Then for
  \[ u_i \coloneqq -\sum_{k=1}^l\frac{r_k}{f_i}m_{ik} \]
  we have:
  \begin{align*}
      u_j-u_i &= -\sum_{k=1}^l\frac{r_k}{f_j}m_{jk} + \sum_{k=1}^l\frac{r_k}{f_i}m_{ik} \\
              &= -\sum_{k=1}^l\frac{r_k}{f_j f_i}f_i m_{jk} + \sum_{k=1}^l\frac{r_k}{f_i f_j} f_j m_{ik} \\
              &= \sum_{k=1}^l\frac{r_k}{f_j f_i}(-f_i m_{jk} + f_j m_{ik}) \\
              &= \sum_{k=1}^l\frac{r_k}{f_j f_i}f_k m_{ij} \\
              &= \frac{m_{ij}}{f_i f_j}
  \end{align*}
  \ %
\end{proof}

\begin{theorem}[using \axiomref{Z-choice}]%
  \label{H1-fp-module-affine-trivial}
  For any affine scheme $X=\Spec(A)$ and coefficients $M: X\to \fpMod{R}$, we have
  \[ H^1(X,M)=0 \rlap{.} \]
\end{theorem}
\begin{proof}
  We need to show, that any $M$-torsor $T$ on $X$ is merely equal to the trivial torsor $M$,
  or equivalently show the existence of a section of $T$.
  We have
  \[ \prod_{x:X}\| T_x \|\]
  and therefore, by (\axiomref{Z-choice}),
  there merely are $f_1,\dots,f_l:A$,
  such that the $U_i\coloneqq \Spec(A_{f_i})$ cover $X$ and
  there are local sections
  \[ s_i:\prod_{x:U_i}T_x\]
  of $T$. Our goal is to construct a matching family from the $s_i$.
  On intersections, let $t_{ij}\coloneqq s_i-s_j$ be the difference, so $t_{ij}:(x : U_i\cap U_j) \to M_x$.
  By \cref{fp-module} equivalently, we have $t_{ij}:\tilde{M}_{f_i f_j}$.
  Since the $t_{ij}$ were defined as differences,
  the condition in \cref{H1-algebra} is satisfied and we get
  $u_i:\tilde{M}_{f_i}$, such that $t_{ij}=u_i-u_j$.
  So we merely have a matching family $\tilde{s}_i\coloneqq s_i-u_i$ and therefore, using Lemma \ref{kraus-glueing} merely a section of $T$.
\end{proof}

A similar result is provable for $H^2(X,M)$ and we expect that $H^n(X,M)$ holds, at least for any external $n$.

\subsection{Čech-Cohomology}

In this section, let $X$ be a type, $U_1,\dots,U_n\subseteq X$ open subtypes that cover $X$
and $\mathcal F:X\to \AbGroup$ a dependent abelian group on $X$.
We start by repeating the classical definition of Chech-Cohomology groups for a given cover.

\begin{definition}%
  \label{chech-complex}
  \begin{enumerate}[(a)]
  \item \index{$\mathcal F(U)$} For open $U\subseteq X$, we use the notation
    \[
      \mathcal F(U)\colonequiv \prod_{x:U}\mathcal F_x\rlap{.}
    \]
  \item For $s:\mathcal F(U)$ and open $V\subseteq U$ we use the notation $s\colonequiv s_{|V} \colonequiv (x:V)\mapsto s_x$.
  \item \index{$U_{i_1\dots i_l}$}For a selection of indices $i_1,...,i_l:\{1,\dots,n\}$, we use the notation
    \[
      U_{i_1\dots i_l}\colonequiv U_{i_1}\cap\dots\cap U_{i_l}\rlap{.}
    \]
  \item For a list of indices $i_1,\dots,i_l$, let $i_1,\dots,\hat{i_t},\dots,i_l$ be the same list with the $t$-th element removed.
  \item For $k:\Z$, the $k$-th \notion{Čech-boundary operator}\index{$\partial^k$} is the homomorphism
    \[
      \partial^k:\bigoplus_{i_0,\dots,i_k}\mathcal F(U_{i_0\dots i_k})\to \bigoplus_{i_0,\dots,i_{k+1}}\mathcal F(U_{i_0\dots i_{k+1}})
    \]
    given by $\partial^k(s)\colonequiv (l_0,\dots,l_{k+1}) \mapsto \sum_{j=0}^k (-1)^j s_{l_0,\dots,\hat{l_j},\dots,l_k|U_{l_0,\dots,l_{k+1}}}$.
  \item The $k$-th \notion{Čech-Cohomology group} for the cover $U_1,\dots,U_n$ with coefficients in $\mathcal F$ is
    \[
      \check{H}^k(\{U\},\mathcal F)\colonequiv \ker\partial^{k} / \im(\partial^{k-1})\rlap{.}
    \]
  \end{enumerate}
\end{definition}

\begin{definition}
  The cover $U_1,\dots,U_n$ is called \notion{acyclic} for $\mathcal F$,
  if for all $k:\N$ and $i_0,\dots,i_k$, we have that the higher (non Čech) cohomology groups are trivial:
  \[
    \forall l>0. H^l(U_{i_0,\dots,i_k},\mathcal F)=0\rlap{.}
  \]
\end{definition}

\begin{example}
  If $X$ is a scheme, $U_1,\dots,U_n$ a cover by affine open subtypes and $\mathcal F$ pointwise a finitely presented $R$-module,
  then $U_1,\dots,U_n$ is acyclic for $\mathcal F$ by \cref{H1-fp-module-affine-trivial}.
\end{example}

\begin{theorem}[using \axiomref{Z-choice}]%
  If $U_1,\dots,U_n$ is an acyclic cover for $\mathcal F$, then
  \[
    \check{H}^1(\{U\},\mathcal F)=H^1(X,\mathcal F)\rlap{.}
  \]
\end{theorem}

\begin{proof}
  Let $\pi$ be the projection map
  \[
    \pi :
    \left(
      \sum_{T:\Tors{\mathcal F}(X)}\prod_{i}\prod_{x:U_i}T_x
    \right)
    \to \Tors{\mathcal F}(X)\rlap{.}
  \]
  Let us abbreviate the left hand side with $T(\mathcal F,U)$.
  Since the cover is acyclic, $\pi$ is surjective.
  There is a map $\iota$ into the kernel of $\partial^1$ (\cref{chech-complex} (e)):
  \[
    \iota \colonequiv
    (T,t) \mapsto (i,j\mapsto t_i - t_j) :
    T(\mathcal F,U)
    \to
    \ker(\partial^1)
    \subseteq
    \bigoplus_{i,j}\mathcal F(U_{ij})\rlap{.}
  \]
  We will now show, that $\iota$ is an embedding and therefore also, that its domain is a set.
  Let $(T,t),(T',t'):T(\mathcal F,U)$ such that $\iota((T,t))=\iota((T',t'))$,
  i.e. for all $i,j$ we have $t_i-t_j=t'_i-t'_j$.
  The latter shows the well-definedness (needed to apply \cref{kraus-glueing})
  of a global map $T\simeq T'$, given by sending $t_i(x)$ to $t'_i(x)$
  for all $i$ and $x$.

  The map $\iota$ is also a surjection and therefore an isomorphism:
  Let $s:\ker(\partial^1)$.
  Then we can contruct a torsor,
  by starting with the trivial torsor on each $U_i$.
  We use \cref{kraus-glueing-1-type} to get a torsor
  with the identification given by the $s_{ij}$
  where the cocycle condition holds because $s$ is in the kernel.

  Realizing, that $\im(\partial^0)$ corresponds to the subtype of $T(\mathcal F,U)$ of trivial torsors,
  we arrive at the following diagram:
  \begin{center}
    \begin{tikzcd}
      & \Tors{\mathcal F}(X)\ar[r,->>] & H^1(X,\mathcal F) \\
      \sum_{T:T(\mathcal F,U)}\|\pi_1(T)=\mathcal F\|\ar[r,hook] & T(\mathcal F,U)\ar[u,->>]\ar[d,equal] & \\
      \im{\partial^0}\ar[r,hook]\ar[u,equal] & \ker{\partial^1}\ar[r,->>] & \check{H}^1(\{U\},\mathcal F)
    \end{tikzcd}
  \end{center}
  By \cref{MISSING},
  the composed map $T(\mathcal F,U)\to H^1(X,\mathcal F)$ is a homomorphism
  and therefore by \cref{surjective-abgroup-hom-is-cokernel} a cokernel.
  So the two cohomology groups are equal, since they are cokernels of the same diagram.
\end{proof}


\section{Type Theoretic justification of axioms}
\newcommand{\inc}{\mathsf{inc}}
\newcommand{\inl}{\mathsf{inl}}
\newcommand{\inr}{\mathsf{inr}}
\newcommand{\idd}{\mathsf{id}}
%\newcommand{\UU}{\mathcal{U}}
\newcommand{\II}{\mathbf{I}}
\newcommand{\nats}{\mathbb{N}}


\newcommand{\ext}{\mathsf{ext}}
\newcommand{\patch}{\mathsf{patch}}
\newcommand{\cov}{\mathsf{cov}}
\newcommand{\isSheaf}{\mathsf{isSheaf}}
\newcommand{\isIso}{\mathsf{isIso}}
\newcommand{\Fib}{\mathsf{Fib}}

\newcommand{\Typp}{\mathsf{Type}}
\newcommand{\Elem}{\mathsf{Elem}}
\newcommand{\Cont}{\mathsf{Cont}}

\newcommand{\BB}{\square}
\newcommand{\CC}{\mathcal{C}}
\newcommand{\UU}{\mathcal{U}}
\newcommand{\WW}{\mathcal{W}}
\newcommand{\VV}{\mathcal{V}}

In this section, we present a model of the 3 axioms stated in \Cref{statement-of-axioms}.
This model is best described as an \emph{internal} model
of a presheaf model. The first part can then be described purely syntactically, starting from any model
of 4 other axioms that are valid in a suitable \emph{presheaf} model. We obtain then the sheaf model by defining
a family of open left exact modalities, and the new model is the model of types that are modal for all these modalities.
This method works both in a $1$-topos framework and for models of univalent type theory.
Throughout this section, we use the words \emph{internal} and \emph{external} relative to the model satisfying the 4 axioms below or state explicitly to which model they refer.

\subsection{Internal sheaf model}

\subsubsection{Axioms for the presheaf model}

We start from 4 axioms. The 3 first axioms can be seen as variation of our 3 axioms for synthetic algebraic geometric.

\begin{enumerate}[(1)]
\item $R$ is a ring,
\item for any f.p.\ $R$-algebra $A$, the canonical map $A\rightarrow R^{\Spec(A)}$ is an equivalence
\item for any f.p.\ $R$-algebra $A$, the set $\Spec(A)$ satisfies choice, which can be formulated as
  the fact that for any family of types $P(x)$ for $x:\Spec(A)$ there is a map
  $(\Pi_{x:\Spec(A)}\norm{P(x)})\rightarrow \norm{\Pi_{x:\Spec(A)}P(x)}$.
\item for any f.p.\ $R$-algebra $A$, the diagonal map $\nats\rightarrow\nats^{\Spec(A)}$ is an equivalence.
\end{enumerate}

As before, $\Spec(A)$ denotes the type of $R$-algebra maps from $A$ to $R$, and
if $r$ is in $R$, we write $D(r)$ for the proposition $\Spec(R_r)$.

Note that the first axiom does not require
$R$ to be local, and the third axiom states that $\Spec(A)$ satisfies \emph{choice} and not only Zariski local choice,
for any f.p. $R$-algebra $A$.


\subsubsection{Justification of the axioms for the presheaf model}

\newcommand{\FP}{\mathsf{FP}}

We justify briefly the second axiom (synthetic quasi-coherence). This justification will be done
in a $1$-topos setting, but exactly the same argument holds in the setting of presheaf models of
univalent type theory, since it only involves strict presheaves. A similar direct verification holds
for the other axioms.

We work with presheaves on the opposite of the category of finitely presented $k$-algebras. We write
$L,M,N,\dots$ for such objects, and $f,g,h,\dots$ for the morphisms. A presheaf $F$ on this category is given
by a collection of sets $F(L)$ with restriction maps $F(L)\rightarrow F(M),~u\mapsto f u$ for
$f:L\rightarrow M$ satisfying the usual uniformity conditions.
The ring $R$ is interpreted as the presheaf given by $R(L)\colonequiv L$.

We first introduce the presheaf $\FP$ of {\em finite presentations}. This is internally the type
$$
\Sigma_{n:\nats}\Sigma_{m:\nats}R[X_1,\dots,X_n]^m
$$
which is interpreted by $\FP(L) = \Sigma_{n:\nats}\Sigma_{m:\nats}L[X_1,\dots,X_n]^m$.
If $\xi = (n,m,q_1,\dots,q_m)\in\FP(L)$ is such a presentation, we build a natural extension
$\iota:L\rightarrow L_{\xi} = L[X_1,\dots,X_n]/(q_1,\dots,q_m)$ where the system $q_1 = \dots = q_m = 0$
has a solution $s_{\xi}$. Furthermore, if we have another extension $f:L\rightarrow M$
and a solution $s\in M^n$ of this system in $M$, there exists a unique map $i(f,s):L_{\xi}\rightarrow M$
such that $i(f,s) s_{\xi} = s$ and $i(f,s)\circ \iota = f$.
Note that $i(\iota,s_{\xi}) = \id$.

\medskip

Internally, we have a map $A:\FP\rightarrow R\mathsf{-alg}(\UU_0)$, which to any presentation
$\xi = (n,m,q_1,\dots,q_m)$ associates the $R$-algebra $A(\xi) = L[X_1,\dots,X_n]/(q_1,\dots,q_m)$.
This corresponds externally to the presheaf on the category of elements of $\FP$ defined
by $A(L,\xi) = L_{\xi}$.

Internally, we have a map $\Spec(A):\FP\rightarrow \UU_0$, defined by $\Spec(A)(\xi) = Hom(A(\xi),R)$.
We can replace it by the isomorphic map which to $\xi = (n,m,q_1,\dots,q_m)$ associates the set
$S(\xi)$ of solutions of the system $q_1=\dots=q_m= 0$ in $R^n$.
Externally, this corresponds to the presheaf on the category of elements of $\FP$ so that
$\Spec(A)(L,n,m,q_1,\dots,q_m)$ is the set of solutions of the system $q_1=\dots=q_m=0$ in $L^n$.

\medskip

We now define externally two inverse maps $\varphi:A(\xi)\rightarrow R^{\Spec(A(\xi))}$ and
$\psi:R^{\Spec(A(\xi))}\rightarrow A(\xi)$.

\medskip

Notice first that $R^{\Spec(A)}(L,\xi)$, for $\xi = (n,m,q_1,\dots,q_m)$,
is the set of families of elements $l_{f,s}:M$ indexed by $f:L\rightarrow M$
and $s:M^n$ a solution of $fq_1 = \dots = fq_m=0$, satisfying the uniformity condition
$g(l_{f,s}) = l_{(g\circ f),gs}$ for $g:M\rightarrow N$.

\medskip

For $u$ in $A(L,\xi) = L_{\xi}$ we define $\varphi~u$ in $R^{\Spec(A)}(L,\xi)$ by
$$
(\varphi~u)_{f,s} = i(f,s)~u
$$
and for $l$ in $R^{\Spec(A)}(L,\xi)$ we define $\psi~l$ in $A(L,\xi) = L_{\xi}$ by
$$
\psi~ l = l_{\iota,s_{\xi}}
$$
These maps are natural, and one can check
$$
\psi~(\varphi~u) = (\varphi~u)_{\iota,s_{\xi}} = i(\iota,s_{\xi})~u = u
$$
and
$$
(\varphi~(\psi~l))_{f,s} = i(f,s)~(\psi~l) = i(f,s)~l_{\iota,s_{\xi}} = l_{(i(f,s)\circ \iota),(i(f,s)~s_{\xi})} = l_{f,s}
$$
which shows that $\varphi$ and $\xi$ are inverse natural transformations.

Furthermore, the map $\varphi$ is the external version of the canonical map $A(\xi)\rightarrow R^{\Spec(A(\xi))}$.
The fact that this map is an isomorphism is an (internally) equivalent statement of the second axiom.



\subsubsection{Sheaf model obtained by localisation from the presheaf model}

We define now a family of propositions. As before, if $A$ is a ring, we let $\Um(A)$ be the type of unimodular sequences
(\Cref{unimodular})
$f_1,\dots,f_n$ in $A$, i.e.\ such that $(1) = (f_1,\dots,f_n)$. To any element $\vec{r} = r_1,\dots,r_n$
in $\Um(R)$ we associate
the proposition $D(\vec{r}) = D(r_1)\vee\dots\vee D(r_n)$. If $\vec{r}$ is the empty sequence then
$D(\vec{r})$ is the proposition $1 =_R 0$. %For $n=0$, we get the proposition $1=_R 0$.

  Starting from any model of dependent type theory with univalence satisfying the 4 axioms above, we build a new
  model of univalent type theory by considering the types $T$ that are modal for all modalities defined by the propositions
  $D(\vec{r})$, i.e.\ such that all diagonal maps $T\rightarrow T^{D(\vec{r})}$ are equivalences.
  This new model is called the \emph{sheaf model}.

    This way of building a new sheaf model can be described purely syntactically, as in \cite{Quirin16}. In \cite{CRS21}, we extend
    this interpretation to cover inductive data types. In particular, we describe there the sheafification $\nats_S$ of the type
    of natural numbers with the unit map $\eta:\nats\rightarrow\nats_S$. 

    A similar description can be done starting with the $1$-presheaf model. In this case, we use for the propositional truncation of a
    presheaf $A$ the image of the canonical map $A\rightarrow 1$. We however get a model of type theory {\em without} universes when we
    consider modal types.

    \begin{proposition}\label{modal}
      The ring $R$ is modal. It follows that any f.p.\ $R$-algebra is modal.
    \end{proposition}

    \begin{proof}
      If $r_1,\dots,r_n$ is in $\Um(R)$, we build a patch function $R^{D(r_1,\dots,r_n)}\rightarrow R$.
      Any element $u:R^{D(r_1,\dots,r_n)}$ gives a compatible family of elements $u_i:R^{D(r_i)}$, hence
      a compatible family of elements in $R_{r_i}$ by quasi-coherence. But then it follows from local-global
      principle \cite{lombardi-quitte}, that we can patch this family to a unique element of $R$.
      
      If $A$ is a f.p.\ $R$-algebra, then $A$ is isomorphic to $R^{\Spec(A)}$ and hence is modal.
    \end{proof}

    \begin{proposition}
      In this new sheaf model, $\perp_S$ is $1 =_R 0$.
    \end{proposition}

    \begin{proof}
      The proposition $1=_R0$ is modal by the previous proposition.
      If $T$ is modal, all diagonal maps $T\rightarrow T^{D(\vec{r})}$ are equivalences. For the empty sequence $\vec{r}$
      we have that $D(\vec{r})$ is $\perp$, and the empty sequence is unimodular exactly when $1 =_R 0$. So $1=_R0$
      implies that $T$ and $T^{\perp}$ are equivalent, and so implies that $T$ is contractible. By extensionality,
      we get that $(1=_R0)\rightarrow T$ is contractible when $T$ is modal.
    \end{proof}
    
    \begin{lemma}\label{Um}
      For any f.p.\ $R$-algebra $A$, we have $\Um(R)^{\Spec(A)} = \Um(A)$.
    \end{lemma}

    \begin{proof}
      Note that the fact that $r_1,\dots,r_n$ is unimodular is expressed by
      $$\norm{\Sigma_{s_1,\dots,s_n:R}r_1s_1+\dots+r_ns_n = 1}$$
      and we can use these axioms 2 and 3 to get
      $$\norm{\Sigma_{s_1,\dots,s_n:R}r_1s_1+\dots+r_ns_n = 1}^{\Spec(A)} = \norm{\Sigma_{v_1,\dots,v_n:A}\Pi_{x:\Spec(A)}r_1v_1(x)+\dots+r_nv_n(x) = 1}$$
      The result follows then from this and axiom 4.
    \end{proof}      
      %, that $A$ is quasi-coherent.
%      \rednote{I think it follows from axioms 4,2 and 3.}


%%     If $A$ is a ring, a fundamental system of orthogonal idempotents $e_1,\dots,e_p$ of $A$ is a sequence of 
%%     idempotent elements satisying $e_1+\dots+e_p = 1$ and $e_ie_j = 0$ if $i\neq j$. We then have a partition
%%     of $\Spec(A)$ into open subsets $\Spec(A_{e_i})$.

%%     \begin{lemma}\label{nats}
%%       For any function $u:\nats^{\Spec(A)}$ there exists a fundamental system of orthogonal idempotents $e_1,\dots,e_p$, and corresponding
%%       numbers $n_1,\dots,n_p$ such that $u$ is constant and equal to $n_i$ on $\Spec(A_{e_i})$.
%%     \end{lemma}

%%     We write $Um(A)$ for $\Sigma_{n:\nats}\Um(A)$.

%%     \begin{corollary}
%%       If $A$ is a f.p. ring, then any function in $Um(R)^{\Spec(A)}$ is given by
%%       a fundamental system of orthogonal idempotents $e_1,\dots,e_p$, and corresponding elements in $Um(A_{e_1}),\dots,Um(A_{e_p})$.
%%     \end{corollary}
    
    For an f.p.\ $R$-algebra $A$, we can define the type of presentations $Pr_{n,m}(A)$ as the type $A[X_1,\dots,X_n]^m$.
    Each element in $Pr_{n,m}(A)$ defines an
    f.p.\ $A$-algebra. Since $Pr_{n,m}(A)$ is a modal type since $A$ is f.p., the type of presentations $Pr_{n,m}(A)_S$ in the sheaf model
    defined for $n$ and $m$ in $\nats_S$ will be such that $Pr_{\eta p,\eta q}(A)_S = Pr_{p,q}(A)$ \cite{CRS21}.
%    We have $Pr(R)^{\Spec(A)} = Pr(A)$. \rednote{Def of Pr missing}
    
    \begin{lemma}\label{propsheaf}
      If $P$ is a proposition, then the sheafification of $P$ is
      $$\norm{\Sigma_{(r_1,\dots,r_n):\Um(R)}P^{D(r_1,\dots,r_n)}}$$
    \end{lemma}
    
    \begin{proof}
      If $Q$ is a modal proposition and $P\rightarrow Q$ we have
      $$\norm{\Sigma_{(r_1,\dots,r_n):\Um(R)}P^{D(r_1,\dots,r_n)}}\rightarrow Q$$
      since
      $P^{D(r_1,\dots,r_n)}\rightarrow Q^{D(r_1,\dots,r_n)}$ and $Q^{D(r_1,\dots,r_n)}\rightarrow Q$.
      It is thus enough to show that
      $$P_0 = \norm{\Sigma_{(r_1,\dots,r_n):\Um(R)}P^{D(r_1,\dots,r_n)}}$$
      is modal.
      If $s_1,\dots,s_m$ is in $\Um(R)$ we show $P_0^{D(s_1,\dots,s_m)}\rightarrow P_0$. This follows
      from $\Um(R)^{D(r)} = \Um(R_r)$, Lemma \ref{Um}.
    \end{proof}
    

    \begin{proposition}\label{norm}
      For any modal type $T$, the proposition $\norm{T}_S$ is
      $$\norm{\Sigma_{(r_1,\dots,r_n):\Um(R)}T^{D(r_1)}\times\dots\times T^{D(r_n)}}$$
    \end{proposition}
    
    \begin{proof}
      It follows from Lemma \ref{propsheaf} that the proposition $\norm{T}_S$ is
      $$\norm{\Sigma_{(r_1,\dots,r_n):\Um(R)}\norm{T}^{D(r_1,\dots,r_n)}} = \norm{\Sigma_{(r_1,\dots,r_n):\Um(R)}\norm{T}^{D(r_1)}\times\dots\times\norm{T}^{D(r_n)}}$$
      and we get the result using the fact that choice holds for each $D(r_i)$, so that
      \[\norm{T}^{D(r_1)}\times\dots\times\norm{T}^{D(r_n)} = \norm{T^{D(r_1)}}\times\dots\times\norm{T^{D(r_n)}} =
        \norm{T^{D(r_1)}\times\dots\times T^{D(r_n)}}\]
    \end{proof}
    
    \begin{proposition}
      In the sheaf model, $R$ is a local ring.
    \end{proposition}

    \begin{proof}
      This follows from \Cref{norm} and Lemma \ref{Um}.
    \end{proof}

    \begin{lemma}\label{localfp}
      If $A$ is a $R$-algebra which is modal and there exists $r_1,\dots,r_n$ in $\Um(R)$ such that each
      $A^{D(r_i)}$ is a f.p.\ $R_{r_i}$-algebra, then $A$ is a f.p.\ $R$-algebra.
    \end{lemma}
    
    \begin{proof}
      Using the local-global principles presented in \cite{lombardi-quitte}, we can patch together the f.p.\ $R_{r_i}$-algebra
      to a global f.p.\ $R$-algebra. This f.p.\ $R$-algebra is modal by Proposition \ref{modal}, and is locally equal to $A$
      and hence equal to $A$ since $A$ is modal.
    \end{proof}

    \begin{corollary}
      The type of f.p.\ $R$-algebras is modal and is the type of f.p.\ $R$-algebras in the sheaf model.
    \end{corollary}

    \begin{proof}
          For any $R$-algebra $A$, we can form a type $\Phi(n,m,A)$ expressing that $A$ has a presentation for some $v:Pr_{n,m}(R)$,
    as the type stating that there is some map $\alpha:R[X_1,\dots,X_n]\rightarrow A$ and that $(A,\alpha)$ is universal such that
    $\alpha$ is $0$ on all elements of $v$. We can also look at this type $\Phi(n,m,A)_S$ in the sheaf model. Using the translation
    from \cite{Quirin16,CRS21}, we see that the type $\Phi(\eta n,\eta m,A)_S$ is exactly the type stating that $A$ is presented by
    some $v:Pr_{n,m}(A)$ among the modal $R$-algebras. This is actually equivalent to $\Phi(n,m,A)$ since any f.p. $R$-algebra is modal.

     If $A$ is a modal $R$-algebra which is f.p. in the sense of the sheaf model, this means that we have
     $$\norm{\Sigma_{n:\nats_S}\Sigma_{m:\nats_S}\Phi(n,m,A)_S}_S$$
     This is equivalent to
     $$\norm{\Sigma_{n:\nats}\Sigma_{m:\nats}\Phi(\eta n,\eta m,A)_S}_S$$
     which in turn is equivalent to
     $$\norm{\Sigma_{n:\nats}\Sigma_{m:\nats}\Phi(n,m,A)}_S$$
     Using Lemma \ref{localfp} and Proposition \ref{norm}, this is equivalent to $\norm{\Sigma_{n:\nats}\Sigma_{m:\nats}\Phi(n,m,A)}$.
    \end{proof}

     Note that the type of f.p. $R$-algebra is universe independent.

    \begin{proposition}
      For any f.p.\ $R$-algebra $A$, the type $\Spec(A)$ is modal and satisfies the axiom of Zariski local choice in
      the sheaf model.
    \end{proposition}
    
    \begin{proof}
      Let $P(x)$ be a family of types over $x:\Spec(A)$ and assume $\Pi_{x:\Spec(A)}\norm{P(x)}_S$. By Proposition \ref{norm},
      this means $\Pi_{x:\Spec(A)}\norm{\Sigma_{(r_1,\dots,r_n):Um}P(x)^{D(r_1)}\times\dots\times P(x)^{D(r_n)}}$. The result follows
      then from choice over $\Spec(A)$ and Lemma \ref{Um}.
    \end{proof}      

%It is then natural to ask how the global section operation behaves for this model, and we show that
%it satisfies a property similar to Zariski local choice
%\rednote{We do not understand what the global section operation has to do with Zariski choice}. 

    \subsection{Presheaf models of univalence}

    We recall first how to build presheaf models of univalence \cite{CCHM,survey},
    and presheaf models satisfying the 3 axioms of the previous section.

The constructive models of univalence are presheaf models parametrised by an interval object $\II$
(presheaf with two global distinct elements $0$ and $1$ and which is tiny) and a classifier object
$\Phi$ for cofibrations. The model is then obtained as an internal model of type theory inside the
presheaf model. For this, we define $C:U\rightarrow U$, uniform in the universe $U$, operation
closed by dependent products, sums and such that $C(\Sigma_{X:U}X)$ holds. It further satisfies, for $A:U^{\II}$, the transport principle
$$
(\Pi_{i:\II}C(Ai))\rightarrow (A0\rightarrow A1)
$$
We get then a model of univalence by interpreting a type as a presheaf $A$ together with an element
of $C(A)$.

 This is over a base category $\BB$.
 
 If we have another category $\CC$, we automatically get a new model of univalent type theory by
 changing $\BB$ to $\BB\times\CC$.

 A particular case is if $\CC$ is the opposite of the category of f.p.\@ $k$-algebras, where $k$ is a
 fixed commutative ring.

 We have the presheaf $R$ defined by $R(J,A) = Hom(k[X],A)$ where $J$ is an object of $\BB$ and $A$ is an object of $\CC$.

  The presheaf $\Gm$ is defined by $\Gm(J,A) = Hom(k[X,1/X],A) = A^{\times}$, the set of invertible elements of $A$.

\subsection{Propositional truncation}

    We start by giving a simpler interpretation of propositional truncation. This will simplify
    the proof of the validity of choice in the presheaf model.

    We work in the presheaf model over a base category $\BB$ which interprets univalent type theory,
    with a presheaf $\Phi$ of cofibrations. The interpretation of the propositional
    truncation $\norm{T}$ {\em does not} require the use of the interval $\II$.

    We recall that in the models, to be contractible can be formulated as having an operation
    $\ext(\psi,v)$ which extends any partial element $v$ of extent $\psi$ to a total element.

    The (new) remark is then that to be a (h)proposition can be formulated as having instead
    an operation $\ext(u,\psi,v)$ which, now {\em given}
    an element $u$, extends any partial element $v$ of extent $\psi$ to a total element.

\medskip    

Propositional truncation is defined as follows. An element of $\norm{T}$ is either of the form
$\inc(a)$ with $a$ in $T$, or of the form $\ext(u,\psi,v)$ where $u$ is in $\norm{T}$ and $\psi$
in $\Phi$ and $v$ a partial element of extent $\psi$.

In this definition, the special constructor $\ext$ is a ``constructor with restrictions'' which
satisfies $\ext(u,\psi,v) = v$ on the extent $\psi$ \cite{CoquandHM18}.

\subsection{Choice}

We prove choice in the presheaf model: if $A$ is a f.p.\@ algebra over $R$ then we have a map
$$
l:(\Pi_{x:\Spec(A)}\norm{P})\rightarrow \norm{\Pi_{x:\Spec(A)}P}
$$

For defining the map $l$, we define $l(v)$ by induction on $v$.
The element $v$ is in $(\Pi_{x:\Spec(A)}\norm{P})(B)$, which can be seen as
an element of $\norm{P}(A)$. If it is $\inc(u)$ we associate $\inc(u)$ and 
if it is $\ext(u,\psi,v)$ the image is $\ext(l(u),\psi,l(v))$.

\subsection{$1$-topos model}

For any small category $\CC$ we can form the presheaf model of type theory over the base category $\CC$ \cite{hofmann,huber-phd-thesis}.
%\rednote{Reference to Hoffmann/Simon's thesis?}.

\medskip

We look at the special case where $\CC$ is the opposite of the category of finitely presented $k$-algebras for a fixed
ring $k$.

    In this model we have a presheaf $R(A) = Hom(k[X],A)$ which has a ring structure.

    In the {\em presheaf} model, we can check that we have $\neg\neg (0=_R 1)$. Indeed, at any stage $A$ we have
    a map $\alpha:A\rightarrow 0$ to the trivial f.p. algebra $0$, and $0 =_R 1$ is valid at the stage $0$.

    The previous internal description of the sheaf model applies as well in the $1$-topos setting.

    \medskip

    However the type of modal types in a given universe is not modal in this $1$-topos setting. This problem can actually be seen as a
    motivation for introducing the notion of stacks, and is solved when we start from a constructive model of univalence.

    \subsection{Some properties of the sheaf model}

    \subsubsection{Quasi-coherence}

A module $M$ in the sheaf model defined at stage $A$, where $A$ is a f.p.\@ $k$-algebra, is given by a sheaf over the category
of elements of $A$. It is thus given by a family of modules $M(B,\alpha)$, for $\alpha:A\rightarrow B$, and restriction maps
$M(B,\alpha)\rightarrow M(C,\gamma\alpha)$ for $\gamma:B\rightarrow C$. In general this family is not determined by
its value $M_A = M(A,\idd_A)$ at $A,\idd_A$.
The next proposition expresses internally in the sheaf model, when a module has this property.
This characterisation is due to Blechschmidt \cite{ingo-thesis}.

\begin{proposition}
  $M$ is internally quasi-coherent\footnote{In the sense that the canonical map $M\otimes A\rightarrow M^{\Spec(A)}$ is an isomorphism for any
  f.p. $R$-algebra $A$.} iff we have $M(B,\alpha) = M_A\otimes_A B$ and the restriction map for
  $\gamma:B\rightarrow C$ is $M_A\otimes_A\gamma$.
\end{proposition}

    \subsubsection{Projective space}

We have defined $\bP^n$ to be the set of lines in $V = R^{n+1}$, so we have
$$
\bP^n ~=~ \Sigma_{L:V\rightarrow \Omega}[\exists_{v:V}\neg (v = 0)\wedge L = R v]
$$
The following was noticed in \cite{kockreyes}.

\begin{proposition}
  $\bP^n(A)$ is the set of submodules of $A^{n+1}$ factor direct in $A^{n+1}$ and of rank $1$.
\end{proposition}

\begin{proof}
  $\bP^n$ is the set of pairs $L,0$ where $L:\Omega^V(A)$ satisfies the proposition $\exists_{v:V}\neg (v = 0)\wedge L = Rv$ at stage
  $A$. This condition implies that $L$ is a quasicoherent submodule of $R^{n+1}$ defined at stage $A$.
  It is thus determined by its value $L(A,\idd_A) = L_A$.

  Furthermore, the condition also implies that $L_A$ is locally free of rank $1$. By local-global principle \cite{lombardi-quitte},
  $L_A$ is finitely generated. We can then apply Theorem 5.14 of
  \cite{lombardi-quitte} to deduce that $L_A$ is factor direct in $A^{n+1}$ and of rank $1$.
\end{proof}

One point in this argument was to notice that the condition
$$
\exists_{v:V}\neg (v = 0)\wedge L = R v
$$
implies that $L$ is quasi-coherent. This would be direct in presence of univalence, since we would have then $L = R$ as a $R$-module
and $R$ is quasi-coherent. But it can also be proved without univalence by transport along isomorphism: a $R$-module which is
isomorphic to a quasi-coherent module is itself quasi-coherent.


\subsection{Global sections and Zariski global choice}

We let $\Box T$ the type of global sections of a globally defined sheaf $T$.
If $c = r_1,\dots,r_n$ is in $\Um(R)$ we let $\Box_c T$ be the type $\Box T^{D(r_1)}\times\dots\times\Box T^{D(r_n)}$.

Using these notations, we can state the principle of Zariski global choice
$$
(\Box \norm{T})\leftrightarrow \norm{\Sigma_{c:\Um(k)}\Box_c T}
$$

This principle is valid in the present model.

Using this principle, we can show that $\Box K(\Gm,1)$ is equal to the type of projective modules of rank $1$ over $k$
and that each $\Box K(R,n)$ for $n>0$ is contractible.
                                                                                  
%This should work over $\bP^n$ as well.

 


\appendix

\section{Negative results}
Here we collect some results of
the theory developed from the axioms
(\axiomref{loc}), (\axiomref{sqc}) and (\axiomref{Z-choice})
that are of a negative nature
and primarily serve the purpose of counterexamples.

We adopt the following definition from
\cite[Section IV.8]{lombardi-quitte}.

\begin{definition}%
  \label{zero-dimensional-ring}
  A ring $A$ is \notion{zero-dimensional}
  if for all $x : A$
  there exists $a : A$ and $k : \N$
  such that $x^k = a x^{k + 1}$.
\end{definition}

\begin{lemma}[using \axiomref{loc}, \axiomref{sqc}, \axiomref{Z-choice}]%
  \label{R-not-zero-dimensional}
  The ring $R$ is not zero-dimensional.
\end{lemma}

\begin{proof}
  Assume that $R$ is zero-dimensional,
  so for every $f : R$ there merely is some $k : \N$ with $f^k \in (f^{k + 1})$.
  We note that $R = \A^1$ is an affine scheme and
  that if $f^k \in (f^{k + 1})$,
  then we also have $f^{k'} \in (f^{k' + 1})$ for every $k' \geq k$.
  This means that we can apply \Cref{strengthened-boundedness}
  and merely obtain a number $K : \N$
  such that $f^K \in (f^{K + 1})$ for all $f : R$.
  In particular, $f^{K + 1} = 0$ implies $f^K = 0$,
  so the canonical map
  $\Spec R[X]/(X^K) \to \Spec R[X]/(X^{K + 1})$
  is a bijection.
  But this is a contradiction,
  since the homomorphism $R[X]/(X^{K + 1}) \to R[X]/(X^K)$
  is not an isomorphism.
\end{proof}

\begin{example}[using \axiomref{loc}, \axiomref{sqc}, \axiomref{Z-choice}]%
  \label{non-existence-of-roots}
  It is not the case that
  every monic polynomial $f : R[X]$ with $\deg f \geq 1$ has a root.
  More specifically,
  if $U \subseteq \A^1$ is an open subset
  with the property that
  the polynomial $X^2 - a : R[X]$ merely has a root
  for every $a : U$,
  then $U = \emptyset$.
\end{example}

\begin{proof}
  Let $U \subseteq \A^1$ be as in the statement.
  Since we want to show $U = \emptyset$,
  we can assume a given element $a_0 : U$
  and now have to derive a contradiction.
  By \axiomref{Z-choice},
  there exists in particular a basic open $D(f) \subseteq \A^1$
  with $a_0 \in D(f)$
  and a function $g : D(f) \to R$
  such that ${(g(x))}^2 = x$ for all $x : D(f)$.
  By \axiomref{sqc},
  this corresponds to an element $\frac{p}{f^n} : R[X]_f$
  with ${(\frac{p}{f^n})}^2 = X : R[X]_f$.
  We use \Cref{polynomial-with-regular-value-is-regular}
  together with the fact that $f(a_0)$ is invertible
  to get that $f : R[X]$ is regular,
  and therefore $p^2 = f^{2n}X : R[X]$.
  Considering this equation over $R^{\mathrm{red}} = R/\sqrt{(0)}$ instead,
  we can show by induction that all coefficients of $p$ and of $f^n$ are nilpotent,
  which contradicts the invertibility of $f(a_0)$.
\end{proof}

\begin{remark}
  \Cref{non-existence-of-roots} shows that
  the axioms we are using here
  are incompatible with a natural axiom that is true
  for the structure sheaf of the big étale topos,
  namely that $R$ admits roots for unramifiable monic polynomials.
  The polynomial $X^2 - a$ is even separable for invertible $a$,
  assuming that $2$ is invertible in $R$.
  To get rid of this last assumption,
  we can use the fact that either $2$ or $3$ is invertible in the local ring $R$
  and observe that the proof of \Cref{non-existence-of-roots}
  works just the same for $X^3 - a$.
\end{remark}

We now give two different proofs that not all $R$-modules are weakly quasi-coherent
in the sense of \Cref{weakly-quasi-coherent-module}.
The first shows that the map
\[ M_f \to M^{D(f)} \]
is not always surjective,
the second shows that it is not always injective.

\begin{proposition}[using \axiomref{loc}, \axiomref{sqc}, \axiomref{Z-choice}]%
  \label{RN-non-wqc}
  The $R$-module $R^\N$ is not weakly quasi-coherent
  (in the sense of \Cref{weakly-quasi-coherent-module}).
\end{proposition}

\begin{proof}
  For $f : R$,
  we have ${(R^{\N})}^{D(f)} = {(R^{D(f)})}^\N = {(R_f)}^\N$,
  so the question is whether the canonical map
  \[ {(R^\N)}_f \to {(R_f)}^\N \]
  is an equivalence.
  If it is,
  for a fixed $f : R$,
  then the sequence $(1, \frac{1}{f}, \frac{1}{f^2}, \dots)$
  has a preimage,
  so there is an $n : \N$ such that
  for all $k : \N$,
  $\frac{a_k}{f^n} = \frac{1}{f^k}$ in $R_f$
  for some $a_k : R$.
  In particular, $\frac{a_{n+1}}{f^n} = \frac{1}{f^{n+1}}$ in $R_f$
  and therefore $a_{n+1} f^{n+1+\ell} = f^{n+\ell}$ in $R$ for some $\ell : \N$.
  This shows that $R$ is zero-dimensional
  (\Cref{zero-dimensional-ring})
  if $R^\N$ is weakly quasi-coherent.
  So we are done by \Cref{R-not-zero-dimensional}.
\end{proof}

\begin{proposition}[using \axiomref{loc}, \axiomref{sqc}, \axiomref{Z-choice}]%
  \label{non-wqc-module-family}
  The implication
  \[ M^{D(f)} = 0 \quad\Rightarrow\quad M_f = 0 \]
  does not hold for all $R$-modules $M$ and $f : R$.
  In particular,
  the map $M_f \to M^{D(f)}$ from \Cref{weakly-quasi-coherent-module}
  is not always injective.
\end{proposition}

\begin{proof}
  Assume that the implication always holds.
  We construct a family of $R$-modules,
  parametrized by the elements of $R$,
  and deduce a contradiction from the assumption
  applied to the $R$-modules in this family.

  Given an element $f : R$,
  the $R$-module we want to consider is
  the countable product
  \[ M(f) \colonequiv \prod_{n : \N} R/(f^n) \rlap{.} \]
  If $f \neq 0$ then $M(f) = 0$
  (using \Cref{non-zero-invertible}).
  This implies that the $R$-module $M(f)^{f \neq 0}$
  is trivial:
  any function $f \neq 0 \to M(f)$ can only assign the value $0$
  to any of the at most one witnesses of $f \neq 0$.
  By assumption, this implies that $M(f)_f$ is also trivial.
  Noting that
  $M(f)$ is not only an $R$-module
  but even an $R$-algebra in a natural way,
  we have
  \begin{align*}
    M(f)_f = 0
    &\;\Leftrightarrow\;
    \exists k : \N.\; \text{$f^k = 0$ in $M(f)$} \\
    &\;\Leftrightarrow\;
    \exists k : \N.\; \forall n : \N.\; f^k \in (f^n) \subseteq R \\
    &\;\Leftrightarrow\;
    \exists k : \N.\; f^k \in (f^{k + 1}) \subseteq R
    \rlap{.}
  \end{align*}

  In summary,
  our assumption implies that the ring $R$ is zero-dimensional
  (in the sense of \Cref{zero-dimensional-ring}).
  But this is not the case,
  as we saw in \Cref{R-not-zero-dimensional}.
\end{proof}

\begin{example}[using \axiomref{loc}, \axiomref{sqc}]
  It is not the case that
  for any pair of lines $L, L' \subseteq \bP^2$,
  the $R$-algebra $R^{L \cap L'}$ is
  as an $R$-module free of rank $1$.
\end{example}

\begin{proof}
  The $R$-algebra $R^{L \cap L'}$ is free of rank $1$
  if and only if the structure homomorphism
  $\varphi : R \to R^{L \cap L'}$ is bijective.
  We will show that it is not even always injective.

  Consider the lines
  \[ L = \{\, [x : y : z] : \bP^2 \mid z = 0 \,\} \]
  and
  \[ L' = \{\, [x : y : z] : \bP^2 \mid \varepsilon x + \delta y + z = 0 \,\}
     \rlap{,} \]
  where $\varepsilon$ and $\delta$ are elements of $R$
  with $\varepsilon^2 = \delta^2 = 0$.
  Consider the element $\varphi(\epsilon \delta) : R^{L \cap L'}$,
  which is the constant function $L \cap L' \to R$
  with value $\varepsilon \delta$.
  For any point $[x : y : z] : L \cap L'$,
  we have $z = 0$ and $\varepsilon x + \delta y = 0$.
  But also, by definition of $\bP^3$,
  we have $(x, y, z) \neq 0 : R^3$,
  so one of $x, y$ must be invertible.
  This implies $\delta \divides \varepsilon$ or $\varepsilon \divides \delta$,
  and in both cases we can conclude $\varepsilon \delta = 0$.
  Thus, $\varphi(\epsilon \delta) = 0 : R^{L \cap L'}$.

  If $\varphi$ was always injective
  then this would imply $\varepsilon \delta = 0$
  for any $\varepsilon, \delta : R$
  with $\varepsilon^2 = \delta^2 = 0$.
  In other words, the inclusion
  \[ \Spec R[X, Y]/(X^2, Y^2, XY) \hookrightarrow \Spec R[X, Y]/(X^2, Y^2) \]
  would be a bijection.
  But the corresponding $R$-algebra homomorphism is not an isomorphism.
\end{proof}


\printindex

\printbibliography

\end{document}
