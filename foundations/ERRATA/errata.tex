%alg\documentclass[12pt,a4paper]{amsart}
\documentclass[10pt,a4paper]{article}
%\ifx\pdfpageheight\undefined\PassOptionsToPackage{dvips}{graphicx}\else%
%\PassOptionsToPackage{pdftex}{graphicx}
%\PassOptionsToPackage{pdftex}{color}
%\fi

%\usepackage{diagrams}

\usepackage{color}
\newcommand\coloremph[2][red]{\textcolor{#1}{\emph{#2}}}

\newcommand\greenemph[2][green]{\textcolor{#1}{\emph{#2}}}
\newcommand{\EMP}[1]{\emph{\textcolor{red}{#1}}}

%\usepackage[all]{xy}
\usepackage{url}
\usepackage{verbatim}
\usepackage{latexsym}
\usepackage{amssymb,amstext,amsmath,amsthm}
\usepackage{epsf}
\usepackage{epsfig}
% \usepackage{isolatin1}
\usepackage{a4wide}
\usepackage{verbatim}
\usepackage{proof}
\usepackage{latexsym}
%\usepackage{mytheorems}
\newtheorem{theorem}{Theorem}[section]
\newtheorem{corollary}{Corollary}[section]
\newtheorem{lemma}{Lemma}[section]
\newtheorem{proposition}{Proposition}[section]
\newcommand{\ras}{\twoheadrightarrow}

\usepackage{float}
\floatstyle{boxed}
\restylefloat{figure}


%%%%%%%%%
\def\oge{\leavevmode\raise
.3ex\hbox{$\scriptscriptstyle\langle\!\langle\,$}}
\def\feg{\leavevmode\raise
.3ex\hbox{$\scriptscriptstyle\,\rangle\!\rangle$}}

%%%%%%%%%


\newcommand\myfrac[2]{
 \begin{array}{c}
 #1 \\
 \hline \hline 
 #2
\end{array}}


\newcommand{\nats}{\mathbb{N}}

\newcommand{\Fin}[1]{T(#1)}

\newcommand{\ODisc}{\mathsf{ODisc}}
\newcommand{\ints}{\mathbb{Z}}
\newcommand{\rats}{\mathbb{Q}}
\newcommand{\FF}{\mathbb{F}}
\newcommand{\Stone}{\mathsf{Stone}}
\newcommand{\CHaus}{\mathsf{CHaus}}
\newcommand{\Open}{\mathsf{Open}}
\newcommand{\Closed}{\mathsf{Closed}}
\newcommand{\AbG}{\mathsf{Ab}}
\newcommand{\OAbG}{\mathsf{Ab_{ODisc}}}
\newcommand{\refl}{\mathsf{refl}}
\newcommand{\ra}{\rightarrow}
\newcommand{\Noo}{\nats_{\infty}}
\newcommand\norm[1]{\left\lVert #1 \right\rVert}
\newcommand\cHH{\check{H}}%\newcommand\cHH{\check{\mathrm{H}}}
\newcommand\disc{\mathsf{disc}}
\newcommand\Min{\mathsf{min}}
\newcommand\Tr{\mathsf{Tr}}
\newcommand\Nm{\mathsf{Nr}}
\newcommand\Nmr{\mathsf{Nrd}}
\newcommand\Hom{\mathsf{Hom}}


\begin{document}

\title{Central simple algebras over $\rats_{p,alg}$ are cyclic}

\author{}
\date{}
\maketitle

%\rightfooter{}

\section*{introduction}

We adapt the proof of Hasse, as presented by Weil \cite{Weil}, that central simple algebras
over $\rats_p$ are cyclic, to a proof,  {\em in a constructive framework}, that central simple algebras over $\rats_{p,alg}$ are
cyclic. For this, we rely on the preliminary work \cite{CLN1},
which provides the foundation of a constructive treatment of central simple algebras over
a {\em discrete field}. To simplify the notation, we write $K(p)$ for $\rats_{p,alg}$
and $Z(p)$ for $\ints_{p,alg}$. In a constructive setting, $K(p)$ and $Z(p)$ are {\em discrete}.
Following Weil, we show that a division algebra over $K(p)$ is cyclic, and we find $x$ and $y$
in $K(p)$ such that the algebra is $K(p)[x,y]$ and for some $n$ and $r$ prime with $n$
\begin{center}
  $x^{p^n-1} = 1~~~~~~~~~~yx = x^{p^r} y~~~~~~~~~y^n = p~~~~~~~~~(1)$
\end{center}

A corollary of our analysis is that it is decidable if a given central simple algebra over $K(p)$
has zero divisors or not. We do this by following ``dynamically'' Weil's argument, starting from
a given central simple algebra $A$: either we get a problem and find a divisor of $0$, or we find $x$ and $y$
satisfying the relations (1), and $A$ is a division algebra.

\section{Valuation}

Let $A$ be a finite dimensional algebra over $K(p)$, of basis $\alpha_1,\dots,\alpha_m$.
We can assume that all $\alpha_i$ are integral over $Z(p)$.
If $x$ in $A$, we can define the characteristic
polynomial of $x$ (the characteristic polynomial of multiplication by $x$) and the trace $\Tr(x)$
and the norm $\Nm(x)$.  We define $v(x) = v_p(\Nm(x))/m$.

We assume that $A$ is a division algebra. Let $x\neq 0$ be an element of $A$. Let $F$ be its
characteristic polynomial and $f$ be its minimal polynomial. Then $f$ divides $F$ and $F$ divides $f^m$.
Furthermore $f$ is irreducible since $A$ is a division algebra.

\begin{lemma}\label{main}
We have $v(x)\geqslant 0$ if, and only if, $x$ is integral over $Z(p)$, i.e.
all the coefficients $a$ of $F$ are in $Z(p)$. 
\end{lemma}

\begin{proof}
Otherwise there exists $h>0$ such that
$p^hF(X)$ is $\neq 0$ mod. $p$ and all its coefficients are in $Z(p)$.  We then have mod. $p$
and equality $p^hF(X) = x^lP(X)$ with $P(0)\neq 0$. We can then lift this decomposition  in $\rats_p[X]$
(Hensel's Lemma) and we get $p^hF(X) = G(X)H(X)$ with $G,H$ relatively prime. This contradicts
that $F$ divides $f^m$ and $f$ irreducible.
\end{proof}

We let then $E = \{x\in A~|~v(x)\geqslant 0\}$ and $P = \{x\in A~|~v(x)>0\}$. If $v(y)\leqslant v(x)$
with $y\neq 0$, we have $0\leqslant v(xy^{-1})$ and so $xy^{-1}$ is integral, and so is $1+xy^{-1}$.
Hence $v(y)\leqslant v(x+y)$. In general, we get $v(x+y)\geqslant \Min(v(x),v(y))$ and if $v(y)<v(x)$ then
$v(x+y) = v(y)$.
It follows that $E$ is a {\em subring} of $A$.




Let $\delta = \disc(\alpha_1,\dots,\alpha_m)$ be the determinant of $\Tr(\alpha_i\alpha_j)$. Since $A$ is a division algebra we
have $\delta \neq 0$. If $x = \Sigma x_i\alpha_i$ in $A$ is integral and $x_1\neq 0$, we can change
$\alpha_1$ to $x$ and we get that $\delta x_1^2 = \disc(x,\alpha_2,\dots,\alpha_n)$ is in $Z(p)$.
It follows that $(\delta x_1)^2$ and so $\delta x_1$ is in $Z(p)$. Similarly, we can prove that all $\delta x_i$ are in $Z(p)$.
Let $l = v_p(\delta)\geqslant 0$. It follows that we can find generators for the $Z(p)$-module $E$ by looking at
finitely many elements $\Sigma x_i a_i/p^l $, and we can find a $Z(p)$-basis $\omega_1,\dots,\omega_m$ of $E$.

The valuation $v$ on $A$ extends the valuation $v_p$ on $K(p)$. Since $\omega_1,\dots,\omega_m$ is
a $Z(p)$-basis of $E$, we have the following simple but key lemma.

\begin{lemma}\label{key}
  If $x = \Sigma x_i\omega_i$ with $x_i$ in $K(p)$ we have $v(x)\geqslant 0$ if, and only if,
  $v_p(x_i)\geqslant 0$ for all $i$.
\end{lemma}

\section{A division algebra is cyclic}

 We have $v(p) = 1$. We try to find $\omega = \Sigma y_i\omega_i$
 with $y_1,\dots, y_m$ in $\{0,\dots,p-1\}$ such that $0<v(\omega)<1$ and minimum.
 Each $z$ in $E$ can be written $z_0 + p z_1$ with the $z_0 = \Sigma z_{0i}\omega_i$ and $z_{0i}$
 in $\{0,\dots,p-1\}$. If $v(z_0) <1$ then $v(z) = v(z_0)$. If $1\leqslant v(z_0)$ then
 $v(z)\geqslant \Min(v(z_0),v(pz_1))\geqslant 1$. 
 So if we can find such $\omega$ we have $P = (\omega)$ and if we cannot find such $\omega$
 we have $P = (p)$ and we write $\omega = p$. We have $v(\omega) = 1/e = v(\Nmr(\omega))/n$
 and so $e$ divides $n$.

 Since $P$ is prime, $E/P$ is an integral extension of $K(p)/pK(p) = \FF_p$ and hence a division
 algebra since it is finite. It is finite since it is a vector space over $\FF_p$ generated by
 the equivalence classes of $\omega_1,\dots,\omega_m$. Let $p^f$ its cardinality.

 The basis $\omega_1,\dots,\omega_m$ is still a basis mod. $p$, hence $E/(p)$ has $p^m$ elements.
 We can write any element $a$ in $E$ on the form $a_0 + \dots + \omega^{e-1} a_{e-1}$, and this
 notation is unique for elements mod. $p$ if we select representative of elements mod. $P$.
 It follows that we have $p^m = (p^f)^e = p^{fe}$ and so $fe = m = n^2$.

 The field $E/P$ is an extension of $\FF_p$ of degree $f$. We choose $\epsilon$ which is an element of $E$
 of degree $f$ mod. $P$. The element $\epsilon$ is then of degree $\geqslant f$ over $Z(p)$. Hence we have
 $f\leqslant n$.

 Since we have $ef = n^2$ and $e$ divides $n$ and $f\leqslant n$, it follows that we have $e = f = n$.

 It follows that $\epsilon$ is of degree $n$. Furthermore, by \cite[Corollary 5.12]{ALP}, the local ring $Z(p)[\epsilon]$ is
 Henselien. Since $X^{p^n}-X$ is decomposable in linear factors mod. $p$, we can lift this decomposition
 and $X^{p^n}-X$ is decomposable in linear factor in $Z(p)[\epsilon]$. There is one factor which is
 irreducible of degree $n$ mod. $p$. We can thus assume that
 $\epsilon$ is of degree $n$ and satisfies $\epsilon^{p^n} = \epsilon$.

 We get in this way a list $\epsilon^i$ for $0\leqslant i<p^n$ of representatives of all elements of $E$ mod. $P$
 not in $P$. Among them, the elements $\epsilon^{jl}$ for $0< j<p$, with  for $l = (p^n-1)/p-1$,
 list all representative
 in $\{1,\dots,p-1\}\subseteq E$ mod. $p$.

 \begin{lemma}\label{unit}
 Any unit in $Z(p)$ is the norm of an element in $E$.   
 \end{lemma}
 
 \begin{proof}
   Let $e$ be such a unit. We have $e = \epsilon^{jl}$ mod. $p$ with $0<j<p$.
   We have $\Nm(\epsilon^j) = \epsilon^{jl}$ which is $e$ mod. $p$. The element $\epsilon^j$
   is root of an irreducible polynomial $X^n + \alpha_1 X^{n-1}+\dots+\alpha_n$. Since
   $X^n+\alpha_1 X^{n-1} +\dots +\alpha_{n-1}X + (-1)^ne$ is decomposable in $n$ linear factors in $Z(p)[\epsilon]$
   mod. $p$, and $Z(p)[\epsilon]$ is Henselian, it has a root in $Z(p)[\epsilon]$ which is an element of norm $e$.
 \end{proof}
 
 The automorphism $\psi : x\mapsto \omega x \omega^{-1}$ leaves $P$ invariant, hence we have
 $\psi(\epsilon) = \epsilon^{p^r}$ mod. $P$ for some $r$. Let $l_0$ the least $l$ such that $rl = 0$ mod. $n$.

 \begin{lemma}
   We can find $\alpha$ in $E\otimes \ints_p$ such that
   $(\omega+\omega^2 \alpha) \epsilon = \epsilon^{p^r}(\omega + \omega^2\alpha)$.
 \end{lemma}

 \begin{proof}
   We find $\alpha$ as a limit of sums $\gamma_n = \alpha_0 +\omega \alpha_1 + \omega^2\alpha_2+\dots + \omega^n\alpha_n$ by determining
   recursively $\alpha_0,\alpha_,\dots$ in $E$ in a recursive way such that
   $(\omega + \omega^2\gamma_n)\epsilon = \epsilon^{p^r} (\omega + \omega^2\gamma_n)$ mod. $\omega^{n+2}$.
   In this way, we get a Cauchy sequence w.r.t. the valuation $v$, and the conclusion follows
   from Lemma \ref{key}.
 \end{proof}
 
 \begin{corollary}
   We have find $\alpha$ in $E$ such that
   $(\omega+\omega^2 \alpha) \epsilon = \epsilon^{p^r}(\omega + \omega^2\alpha)$.
 \end{corollary}

 \begin{proof}
   The linear system in $\alpha$, viewing $A$ as a vector space over $K(p)$
   \begin{center}
     $(\omega+\omega^2 \alpha) \epsilon = \epsilon^{p^r}(\omega + \omega^2\alpha)$.
   \end{center}
   is not contradictory since it has a solution in $E\otimes \ints_p$. So we can find a solution
   $u_0$ and $u_1,\dots,u_l$ such that all solutions are of the form $u_0+\Sigma x_i u_i$ with
   $x_i$ in $K(p)$. In $A\otimes \rats_p$, all solutions are $u_0+\Sigma x_i u_i$ with $x_i$ in $\rats_p$.
   By the previous Lemma, we have a solution which satisfies $v(u_0+\Sigma x_i u_i)\geqslant 0$.
   If $v(x_i-y_i)$ big enough with $y_i$  in $K(p)$ we have $v(u_0+\Sigma y_iu_i)\geqslant 0$ and
   we take $\alpha = u_0+\Sigma y_iu_i$. 
 \end{proof}
 


 If we change $\omega$ to $\omega + \omega^2\alpha$, we have $\omega \epsilon = \epsilon^{p^r}\omega$.
 It follows that $\omega^{l_0}$ commutes with $\epsilon$ and $\omega$ and hence is in $\FF_p$.
 Since we have $v(\omega) = 1/n$ it follows that $\omega$ is of degree $n$ over $Z(p)$ and hence $l_0 = n$
 and $r$ is prime with $n$. Since $(\omega^n) = (p)$ we have $\omega^n = p e$ with $e$ unit in $Z(p)$.
 By Lemma \ref{unit}, the element $e$ is a norm of an element $u$ in $E$ and then $\omega u$
 satisfies $\omega u \epsilon = \epsilon^{p^r}\omega u$ and $(\omega u)^n = p$.

 We get in this way the following presentation of $A$ as a cyclic algebra over $K(p)$ with
 elements $x$ and $y$ such that $x$ is a primitive $p^n-1$ root of unity and
 $yx = x^{p^r}y$ with $r$ prime with $n$ and $y^n = p$.
 
 This provides a way of testing if a given central simple algebra $A$ over $K(p)$ is a division algebra or not.
 We follow dynamically this argument assuming that $A$ is a division algebra. Either we get in this way
 a zero divisor, or we find such $x$ and $y$ and we can conclude that $A$ is a division algebra
 from \cite{Wedderburn} and the following Lemma.

 \begin{lemma} 
   The elements $p,p^2,\dots,p^{n-1}$ are not norm of an element in $K(p)[x]$.
 \end{lemma}

 \begin{proof}
   Since $x$ is of degree $n$ mod. $p$, we get that $(p)$ is prime in $Z(p)[x]$ and then $Z(p)[x]$
   mod. $p$ is a finite field extension of $\FF_p$. If $\Nm(u) = p^i$ with $i>0$ then $u$ is in
   $Z(p)[x]$ by Lemma \ref{main}, and $u$ is $0$ mod. $p$, hence $i\geqslant n$.
 \end{proof}
 
 \section{Example}

 As an exemple, we take the quaternion algebra over $K(5)$, which is determined by the multiplication table
 \begin{center}
   $i^2 = j^2 = k^2 = -1~~~~~~~ij = -ji = k~~~~ik = -ki = -j~~~~~~jk = -kj = i$
 \end{center}
 We have a basis $1,i,j,k$ where all elements are integral over $Z(5)$.

 We assume that it is a division algebra. We should be able following Weil's argument to find
 automatically a contradiction.

 We compute $v(1) = v(i) = v(j) = v(k) = 0$ 

 The discriminant is a unit in $Z(5)$. Hence, $1,i,j,k$ is already an integral basis of $E$.
 Next, we find $\omega = 2+i$ such that $v(\omega) = 1/2$. We have $(2+i)^{-1} = 2/5 -1/5i$.
 The prime ideal $P$ should formed of all
 elements $a_0 + a_1i+a_2j+a_3k\neq 0$ such that $5$ divides $a_0^2 + a_1^2 + a_2^2 + a_3^3$.
 We should have $E/P$ finite field. However $(2+j)(2+i)^{-1} = (2+j)(2/5-1/5i) = 4/5 -2/5i + 2/5j +1/5k$
 is not integral and we get a contradiction, that we found in a finite way.

  \begin{thebibliography}{9}

 \bibitem[ALP]{ALP}
    M. E. Alonso, H. Lombardi and H. Perdry.
    \newblock{Elementary constructive theory of Henselian local rings.}
    \newblock{Mathematical Logic Quarterly 54, 253-271 (2008)}

  \bibitem[CLN2]{CLN2}
    Th. Coquand, H. Lombardi and S. Neuwirth.
    \newblock{Alg\`ebre d'Azumaya and Barr's Theorem.}
    \newblock{https://arxiv.org/html/2306.17679v2 (2025).}

  \bibitem[CLN1]{CLN1}
    Th. Coquand, H. Lombardi and S. Neuwirth.
    \newblock{Constructive basic theory of central simple algebras.}
    \newblock{https://arxiv.org/abs/2102.12775 (2023).}

   \bibitem[LQ]{LQ}
    H.\ Lombardi and C.\ Quitté.
    \newblock{\em Commutative Algebra. Constructive Methods.}
    \newblock{Springer  (2015).}
    
  \bibitem[Wedd]{Wedderburn}
    J.H.M. Wedderbrun.
    \newblock{A type of primitive algebra.}
    \newblock{Transactions of the Mathematical Society, Vol. 15 (1914).}

  \bibitem[Weil]{Weil}
    A. Weil.
    \newblock{Les corps gauches $p$-adiques.}
    \newblock{Le s\'eminaire de math\'ematiques, 1. Ann\'ee 1933-1934.}
    
    


 \end{thebibliography}


\end{document}     
                                                                                  











