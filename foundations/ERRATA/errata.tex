\documentclass[10pt,a4paper]{article}

\usepackage{color}
\newcommand\coloremph[2][red]{\textcolor{#1}{\emph{#2}}}

\newcommand\greenemph[2][green]{\textcolor{#1}{\emph{#2}}}
\newcommand{\EMP}[1]{\emph{\textcolor{red}{#1}}}

\usepackage{url}
\usepackage{verbatim}
\usepackage{latexsym}
\usepackage{amssymb,amstext,amsmath,mathtools,amsthm}
\usepackage{epsf}
\usepackage{epsfig}
\usepackage{a4wide}
\usepackage{verbatim}
\usepackage{proof}
\usepackage{latexsym}
\newtheorem{theorem}{Theorem}[section]
\newtheorem{corollary}{Corollary}[section]
\newtheorem{lemma}{Lemma}[section]
\newtheorem{proposition}{Proposition}[section]
\newcommand{\ras}{\twoheadrightarrow}

\usepackage{float}
\floatstyle{boxed}
\restylefloat{figure}

%%%%%%%%%
\def\oge{\leavevmode\raise
.3ex\hbox{$\scriptscriptstyle\langle\!\langle\,$}}
\def\feg{\leavevmode\raise
.3ex\hbox{$\scriptscriptstyle\,\rangle\!\rangle$}}

%%%%%%%%%

\newcommand\myfrac[2]{
 \begin{array}{c}
 #1 \\
 \hline \hline
 #2
\end{array}}


\newcommand{\nats}{\mathbb{N}}

\newcommand{\Fin}[1]{T(#1)}

\newcommand{\ODisc}{\mathsf{ODisc}}
\newcommand{\ints}{\mathbb{Z}}
\newcommand{\rats}{\mathbb{Q}}
\newcommand{\FF}{\mathbb{F}}
\newcommand{\Stone}{\mathsf{Stone}}
\newcommand{\CHaus}{\mathsf{CHaus}}
\newcommand{\Open}{\mathsf{Open}}
\newcommand{\Closed}{\mathsf{Closed}}
\newcommand{\AbG}{\mathsf{Ab}}
\newcommand{\OAbG}{\mathsf{Ab_{ODisc}}}
\newcommand{\refl}{\mathsf{refl}}
\newcommand{\ra}{\rightarrow}
\newcommand{\Noo}{\nats_{\infty}}
\newcommand\norm[1]{\left\lVert #1 \right\rVert}
\newcommand\cHH{\check{H}}%\newcommand\cHH{\check{\mathrm{H}}}
\newcommand\disc{\mathsf{disc}}
\newcommand\Min{\mathsf{min}}
\newcommand\Tr{\mathsf{Tr}}
\newcommand\Nm{\mathsf{Nr}}
\newcommand\Nmr{\mathsf{Nrd}}
\newcommand\Hom{\mathsf{Hom}}

\newcommand\Ty{\mathsf{Ty}}
\newcommand\El{\mathsf{El}}
\newcommand\p{\mathsf{p}}
\newcommand\q{\mathsf{q}}
\DeclareMathOperator\id{id}


\begin{document}

\title{Errata}

\author{Thierry Coquand, Jonas H\"ofer and Christian Sattler}
\date{}
\maketitle

%\rightfooter{}

\section*{introduction}

Section 8.1 the mistake is in building the presheaf model

It works in the 1-topos case

\section{Problem with quotient}

We work with cubical sets. If $E$ is a set we write $\Delta E$ constant cubical set.
Let $R$ be an equivalence relation on $E$. This defines a (strict) equivalence relation $\Delta R$
on the cubical set $\Delta E$. There is a projection map $\Delta E \rightarrow \Delta (E/R)$.


\begin{proposition}
  The projection map $\Delta E \rightarrow \Delta (E/R)$ defines a quotient of $\Delta E$
  by $\Delta R$ in the cubical set model if, and only if, $E\rightarrow E/R$ has section.
\end{proposition}
\begin{proof}
  Suppose first that $\eta \colon E \to E/R$ has a section $s \colon E/R \to E$.
  This means that we internally also have a section of $\Delta\eta \colon \Delta E \to \Delta(E/R)$ interpreted by $\Delta s$, implying in particular that $\Delta \eta$ is surjective in the internal sense.
  Furthermore, since paths in $\Delta (E/R)$ correspond to strict equality, we have that $\Delta R$ is equivalent to the kernel of $\Delta\eta$.
  These two properties already characterize the quotient~\cite[Theorem~18.2.3]{rijke2025intro}.

  Conversely, suppose that $\Delta\eta \colon \Delta E \to \Delta(E/R)$ has the universal property of the quotient internally.
  We view $E$ as a groupoid with objects $E$ and an invertible morphism between two objects exactly if they are related by $R$.
  The nerve of this groupoid $NE$ is a fibrant cubical set.
  This fibrant cubical set admits a map $p \colon \Delta E \to NE$ uniquely determined by the identity function $E \to E = (NE)([0])$.
  Furthermore, we have that the kernel of this map contains $\Delta R$ since we have an element of $\El(\Delta E. \Delta E. \Delta R, p(\q\p) = p(\q\p^2))$ exactly if for all $e_0, e_1 \in E$ satisfying $R(e_0, e_1)$ we have an element of $N(e_0, e_1)$ which is defined as the subsingleton $R(e_0, e_1)$.
  Hence, we get a map $\Delta(E/R) \to NE$ with $\overline p \circ \Delta\eta \sim p$.
  This means we have a map on points $\overline{p}_{[0]} \colon E/R \to E$ that we claim to be a section of \(\eta\).
  Externally, we have an element of $\El(\Delta E, \overline p (\Delta\eta\q) =_{NE} p\q)$ meaning that for each $e \in E$ we have $R(\overline p_{[0]}(\eta e), e)$ which implies that $\eta \overline{p}_{[0]} = \id_{E/R}$.
\end{proof}

\section{Case where $k$ is $\ints$ or a dicrete field}

 We have to use cobar.


A(p), R[X] is not described correctly
we don't clearly get that it has the desired universal property

We can even derive EM from


-problem general case

-if no quotient k = Z or k discrete field needs cobar

\section{An example where it works}

-distributive lattice


\bibliography{../../util/literature}
\bibliographystyle{plain}

\end{document}












