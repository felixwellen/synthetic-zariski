In this section we (Hugo Moeneclaey) work with condensed sets. It is very drafty, in particular I reuse many results from synthetic algebraic geometry freely, without checking they still hold.


\subsection{Presheaves}

We work in the presheaf topos satisfying the following:

\begin{itemize}
\item There is a boolean ring $B$.
\item For all c.p. boolean $B$-algebra $C$, the map:
\[C \to B^{\Spec(C)}\]
is an equivalence.
\item Affine scheme have choice.
\end{itemize}

We want to do two successive sheafification, the Zariski one and the fppf one. We could also do them in one go.


\subsection{Zariski sheaves}

\begin{definition}
The Zariski topology consists of finite sums:
\[f_1 \inv + \cdots + f_n \inv\]
where $f_1,\cdots,f_n:B$ such that:
\[(f_1,\cdots,f_n) = 1\]
\end{definition}

\begin{definition}
The disjoint topology consists of finite sums:
\[f_1 \inv + \cdots + f_n \inv\]
where $f_1,\cdots,f_n:B$ is a fundamental system of idempotent, i.e.
\[f_1+\cdots+f_n = 1\]
\[f_i^2 = f_i\]
\[f_if_j = 0\]
when $i\not=j$.
\end{definition}

\begin{lemma}
A type $X$ is a Zariski sheaf if and only if it is a disjoint sheaf.
\end{lemma}

\begin{proof}
The disjoint topology is included on the Zariski topology, so we just need to prove that:
\[f_1 \inv \lor \cdots \lor f_n \inv\]
is disjoint-contractible. So we need to prove the proposition assuming $f=0\lor f=1$ for finitely many $f:B$. This is straightforward.
\end{proof}

Next proposition is based on \cref{sheaves-internal}, which has not be proven in details yet.

\begin{proposition}
The Zariski topos satisfy the following:
\begin{itemize}
\item For all c.p. boolean ring $C$, the map:
\[C\to \mathbb{2}^{\Spec(C)}\]
is an equivalence.
\item For all c.p. boolean ring $C$, the type $\Spec(C)$ has choice.
\end{itemize}
\end{proposition}

\begin{proof}
We apply \cref{sheaves-internal}. 
\begin{itemize}
\item Any local boolean ring is $\mathbb{2}$, so the generic ring in the Zariski topos is $\mathbb{2}$, justifying the first item.
\item We have disjoint-local choice for affine schemes, let us check that it implies actual choice. To do this it is enough to check that disjoint-cover are equivalence, i.e. that given a fundamental system $f_1,\cdots,f_n$ of idempotent in $\mathbb{2}$ we have that:
\[f_1=1 + \cdots + f_n=1\]
is contractible. This is straightforward.
\end{itemize}
\end{proof}


\subsection{Fppf sheaves in the Zariski topos}

Now we work inside the Zariski topos (still with boolean algebras).

\begin{definition}
The fppf topology consists of $\Spec(B)$ for all fppf boolean algebra $B$.
\end{definition}

The checking that this is a topology is the same as in algebraic geometry.

\begin{lemma}\label{2-modules-flat}
Any $\mathbb{2}$-module is flat.
\end{lemma}

\begin{proof}
This is true for any discrete field, the idea is that any module is a filtered colimit of finitely presented modules, finitely presented modules over a discrete field are free therefore flat, and a filtered colimit of flat modules is flat.
\end{proof}

\begin{remark}
We think that any boolean algebra morphism $A\to B$ is flat. We were not able to prove this yet.
\end{remark}

\begin{lemma}
A boolean algebra $B$ is fppf if and only if $0\not=_B1$.
\end{lemma}

\begin{proof}
By \cref{2-modules-flat} we have that $B$ is always flat, so we have to prove that:
\[\prod_{M:\mathbb{2}-\mathrm{Mod}} B\otimes M = 0 \to M=0\]
if and only if:
\[0\not=_B1\]
The direct is clear, conversly if $0\not=_B1$ then:
\[\mathbb{2}\to B\]
is injective, but by \cref{2-modules-flat} we have that any $M$ is flat so that:
\[M\to M\otimes B\]
is injective which is enough to conclude.
\end{proof}

Once again next theorem relies on the unproven \cref{sheaves-internal}.

\begin{theorem}
The Fppf topos satisfy the following:
\begin{itemize}
\item For all c.p. $B$ such that $0\not=_B1$, we have that $\propTrunc{\Spec(B)}$
\item For all c.p. boolean ring $C$, the map:
\[C\to \mathbb{2}^{\Spec(C)}\]
is an equivalence.
\item For all c.p. boolean ring $C$, the type $\Spec(C)$ has representable choice.
\end{itemize}
\end{theorem}

\begin{lemma}
For all c.p. boolean algebras $A,B$, a map:
\[\Spec(B)\to \Spec(A)\]
is surjective if and only if the corresponding map:
\[A\to B\]
is injective.
\end{lemma}

\begin{proof}
Assume a surjective map:
\[\Spec(B)\to \Spec(A)\]
corresponding to:
\[f:A\to B\]
Then given $a,b:A$ such that $f(a)=f(b)$ we have that:
\[\prod_{x:\Spec(A)} x(a) = x(b)\]
indeed for any $x:\Spec(A)$ we want to prove a proposition, so we can assume $y:\Spec(B)$ such that $y\circ f = x$ and then $f(a)=f(b)$ allows us to conclude. Therefore $a=b$.

Conversely if the map is injective TODO
\end{proof}


\subsection{A strange property of the presheaf topos}

We work in the presheaf topos, with generic boolean algebra $B$.

\begin{lemma}
If $0\not=_B1$, then for all $x:B$ we have that $x=0\lor x=1$.
\end{lemma}

\begin{proof}
If $0\not=_B1$, for any $x:B$ we have that:
\[(x=0+x=1) \simeq (x=0\lor x=1)\]
but the left hand side is representable so it is a Zariski sheaf, and the right hand side is Zariski contractible, so that:
\[x=0\lor x=1\]
is contractible.
\end{proof}

\begin{remark}
We probably can make a much more reasonable argument from duality alone.
\end{remark}



\subsection{Dependent choice from presheaves to Zariski sheaves}


\subsection{Dependent choice from Zariski sheaves to Fppf sheaves}

