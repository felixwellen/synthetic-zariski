\paragraph{Exercise 1}{
Show that the canonical map $A[X]\rightarrow R[X]^{Sp(A)}$ is an isomorphim
}

\medskip

 Here are two applications:

\medskip

\paragraph{Exercise 2: {\bf Zariski lattice}}{

If $A$ is a ring, one defines (Joyal) the Zariski lattice $Zar(A)$ as the distributive lattice generated
by symbols $D(a)$ for $a$ in $A$ and relations:
\begin{eqnarray}
D(0) &=& 0 \nonumber\\
D(1) &=& 1 \nonumber\\
D(a+b) &\leq& D(a)\vee D(b)\nonumber\\
D(ab) &=& D(a)\wedge D(b) \nonumber
\end{eqnarray}

(This can be realized as the lattice of radicals of finitely generated ideals.)

Prove that $Zar(R)$, where $R$ is the generic local ring, is the set of open propositions, with $D(r)$ being
$r\neq 0$.

\medskip

If $A$ is a finitely presented $R$-algebra, prove that the canonical map $Zar(A)\rightarrow Zar(R)^{Sp(A)}$
is an isomorphism. Hint: use the surjective map $R[X]\rightarrow Zar(R),~\Sigma r_iX^i\mapsto D(r_0,\dots,r_n)$
and Zariski local choice.

}


\medskip

\paragraph{Exercise 3}{
  Give another proof that the functions $f:Sp(A)\to \N$ are continuous, where $A$ is
 a finitely presented $R$-algebra, by looking at the map $Sp(A)\rightarrow R[X],~x\mapsto X^{f(x)}$
 and the corresponding polynomial in $A[X]$.
 }