We present some sketches of synthetic linear algebra.

\begin{lemma}\label{inf-has-choice}
Let $X$ be an infinitesimal variety. Then choice over $X$ is valid:
for any type family $P$ with $(x : X) \to \propTrunc{P(x)}$,
we have $\propTrunc{(x : X) \to P(x)}$.
\end{lemma}
\begin{proof}
By Zariski local choice, it suffices to show that every Zariski cover
of $X \to R$ is trivial, that is that $X \to R$ is a local ring.
Indeed this is the case, since the evaluation $(X \to R) \to R$
reflects invertible elements.
\end{proof}

\begin{lemma}
In the category of finitely co-presented $R$-modules, every object is projective.
That is, if $M,N,L$ are finitely co-presented $R$-modules,
$f : M \to L$ is $R$-linear and $g : N \to L$ is $R$-linear and surjective,
then there merely exists $h : M \to N$ such that $f = g \circ h$.
\end{lemma}
\begin{proof}
We apply \cref{inf-has-choice} to obtain
$(m : \D(M^\star)) \to (n : N) \times g(n) = f(m)$.
That is, we have $h_0 : \D(M^\star) \to N$ such that $f = g \circ h_0$ on $\D(M)$.
Without loss of generality, $h_0(0) = 0$; otherwise we may replace $h_0$
by $h_0 - h_0(0)$, since in any case $g(h_0(0)) = f(0) = 0$.
Now $h_0$ lifts to a pointed map $\D(M^\star) \to_\pt \D(N^\star)$.
This corresponds to an $R$-linear map $N^\star \to M^\star$, and hence to
an $R$-linear map $M \to N$, as desired.
\end{proof}

In particular, since $R$ is a local ring, we have that if $M$ is finitely 
co-presented and finitely generated, then $M$ is free.

Note that for any $R$-linear map $f : M \to N$, we have
$(\coker(f))^\star = \ker(f^\star)$, where $f^\star : N^\star \to M^\star$ is
the dual map.
If $M$ and $N$ are finitely presented, then
so is $\coker(f)$,
and $\ker(f^\star)$ is finitely co-presented.
Hence if $M$ and $N$ are finitely presented, then 
$f$ is surjective iff $f^\star$ is injective.

\begin{definition}
For $M$ finitely presented, we say $\dim M \le n$ if
there merely exists a surjective linear map $R^n \to M$.

For $V$ finitely co-presented, we say $\dim V \le n$ if
there merely exists an injective linear map $V \to R^n$.
\end{definition}
It is direct  that there merely exists $n$ such that $\dim M \le n$ in
either case.
Note that as usual in constructive algebra, we do not define $\dim M$ as a natural
number, but only what it means to compare it with natural numbers.
To see that our notation is consistent, consider $M$ which is both
finitely presented and finitely co-presented.
In this case $M$ is free of some rank $k$, and we have that
$\dim M = k$, in the appropriate sense.
\begin{lemma}
We have $\dim M \le n$ iff $\dim M^\star \le n$.
\end{lemma}
\begin{proof}
Follows from $(R^n)^\star = R^n$ and the fact that $f$ is surjective if and only if
$f^\star$ is injective.
\end{proof}
\begin{lemma}\label{surjective-iff-nonzero-wedge}
A map $f : R^m \to R^n$ is surjective if and only if the induced map
$\bigwedge^n f : \bigwedge^n R^m \to \bigwedge^n R^n$ on the 
$n$th exterior power is non-zero.
In particular, this is an open proposition, asserting that a certain list of
$m \choose n$ numbers is nonzero.
\end{lemma}
\begin{proof}
For the forward implication, pick preimages for the basis vectors,
$f(u_i) = e_i$, 
and note $f(u_1 \wedge \ldots \wedge u_n) = e_1 \wedge \ldots \wedge e_n \ne 0$.
For the reverse implication, if $\bigwedge^n f$ is nonzero, then
there are $u_1,\ldots,u_n : R^m$ such that
$f(u_1\wedge\ldots\wedge u_n) = e_1 \wedge \ldots \wedge e_n$.
This means that $u_i$ determine a map $R^n \to R^m$ such that
the composite $R^n \to R^m \to R^n$ has invertible determinant and hence is surjective.
\end{proof}
\begin{lemma}\label{matrix-algorith}
For any linear map $f : R^m \to R^n$, it is not not the case that
$f$ is a composite $R^m \simeq R^r \oplus R^{m-r} \to R^r \to R^r \oplus R^n \simeq R^n$, 
where $r \le m,n$, the maps to and from $R^r$ are projections and inclusions
from and to a direct sum, and the outer isomorphisms are arbitrary.
\end{lemma}
\begin{proof}
Since we are proving a negated statement, we can pretend that $R$ is a discrete field.
In this case we follow a well-known matrix algorithm. % What is it called?
\end{proof}
In particular this shows that any finitely presented module is not not free.
\begin{lemma}
$\dim M \le n$ is an open proposition.
\end{lemma}
\begin{proof}
Let $M$ be represented as the cokernel of a map $f : R^k \to R^l$.
We claim that $\dim M \le n$ is equivalent to the assertion that
for some detachable subset $I \subseteq [l]$ of size at most $n$,
the composite $R^I \to R^l \to M$ is surjective.
Surjectivity of this map is equivalent to surjectivity of
$R^I \oplus R^k \to R^l$, which is an
open proposition by \cref{surjective-iff-nonzero-wedge}.
Since open propositions are closed under finite disjunction, it is enough
to prove our claim.
One direction is clear: if $R^I \to M$ is surjective, then $\dim M \le |I| \le n$.
Conversely, suppose $\dim M \le n$. Since our goal is to prove an
open proposition, and $M$ is not not free, we may assume $M$ is free
of rank $r \le n$. In this case, since we have a surection $R^l \to R^r$, 
we obtain a subset $I \subseteq [l]$
of size $r$ such that the composite $R^I \to R^l \to R^r$ is surjective,
as needed.
\end{proof}

From this proof, it is natural to define $\dim M \ge n$ as follows.
If $M$ is the cokernel of $f : R^k \to R^l$, then $\dim M \ge n+1$
iff $\bigwedge^l (\iota_I \oplus f) : \bigwedge^l (R^I \oplus R^k) \to \bigwedge^l R^l$
is zero for all $I \subseteq [l]$ of size less than $n$, with $\iota_I : R^I \to R^l$
the usual inclusion. This would make $\dim M \ge n$ a closed proposition,
whose negation is $\dim M \le n-1$. But with this definition, it is unclear
what $\dim M \ge n$ means for $M$ directly. For example, it does not imply that
there exists a surjection $M \to R^n$, even if $k = l = 2$ and $n = 1$.

\begin{definition}
For $X$ a scheme, we take $\dim X \le n$ to mean that
the set of $p : X$ with $\dim T_p X \le n$ is dense in $X$.
\end{definition}
