\subsection{Differential forms}

\begin{definition}
For any type $X$, the type $\Delta^k(X)$ of infinitesimal $k$-simplices in $X$ is defined by:
\[\Delta^k(X) = \{x_0,\cdots,x_n:X\ 
|\ \forall i,j.\ x_i\sim_1x_j\}\]
\end{definition}

\begin{definition}
An infinitesimal $k$-simplex $x_0,\cdots,x_n$ in $X$ is called degenerate if there merely exists $i\not= j$ such that $x_i=x_j$. 
\end{definition}

\begin{definition}
For any type $X$, the type $\Omega^k(X)$ of differential $k$-forms on $X$ consists of maps:
\[\omega:\Delta^k(X)\to R\]
which are zero on degenerate infinitesimal $k$-simplices.
\end{definition}

\subsection{Differential forms are alternating}

\begin{lemma}\label{first-order-functions}
For any f.p. $R$-algebra $A$, we have that $\Delta^1(\Spec(A))$ is the type of $x,y:\Spec(A)$ such that for all $f,g:A$ we have:
\[(f(x)-f(y))(g(x)-g(y))\]
\end{lemma}

\begin{proof}
We just need to prove that for $x,y:\Spec(A)$, the following are equivalent:
\begin{itemize}
\item We have that $x\sim_1y$.
\item For all $f,g:A$ we have:
\[(f(x)-f(y))(g(x)-g(y)) = 0\]
\end{itemize}
If $x\sim_1y$ then there is a f.g. ideal $I$ such that $I^2=0$ and $I=0\to x=y$. Then $I=0$ implies $f(x)-f(y)=0$ and $g(x)-g(y)=0$ so that both belong to $I$ and their product is zero.

Conversely we consider $f_1,\cdots,f_n$ generating $A$, then we consider the f.g. ideal:
\[(f_1(x)-f_1(y),\cdots,f_n(x)-f_n(y))\]
By hypothesis this ideal has square zero, and if it is null then $x=y$ so we indeed have that $x\sim_1y$. 
\end{proof}

This can surely be extended to $k$-simplices, although it is unpleasant to write down.

\begin{corollary}\label{one-simplices-spectrum}
For any f.p. $R$-algebra $A$, we have that $\Delta^1(\Spec(A))$ is the spectrum of $A\otimes A$ quotiented by:
\[(f\otimes 1 - 1\otimes f)(g\otimes 1 - 1\otimes g)\]
for all $f,g:A$. If $f_1,\cdots,f_n$ generate $A$, it is enough to quotient by:
\[(f_i\otimes 1 - 1\otimes f_i)(f_j\otimes 1 - 1\otimes f_j)\]
for all $i,j$.
\end{corollary}

\begin{proof}
The first part is just sqc. In the finitely generated case we check that:
\[(ff'\otimes 1 - 1\otimes ff')(g\otimes 1 - 1\otimes g)\]
belongs to the ideal generated by:
\[(f\otimes 1 - 1\otimes f)(g\otimes 1 - 1\otimes g)\]
and
\[(f'\otimes 1 - 1\otimes f')(g\otimes 1 - 1\otimes g)\]
\end{proof}

\begin{lemma}\label{one-form-alternating}
For any $\omega:\Omega^1(X)$ and $(x,y):\Delta^1(X)$ we have that:
\[\omega(x,y) = - \omega(y,x)\]
\end{lemma}

\begin{proof}
By \cref{one-simplices-spectrum} we know that there exists some $f_i,g_i:A$ such that for all $(x,y):\Delta^1(X)$ we have that:
\[\omega(x,y) = \sum_i f_i(x)g_i(y)\]
Then we need to check that for any $(x,y):\Delta^1(X)$ we have:
\[\omega(x,y) = - \omega(y,x)\]
But by \cref{first-order-functions}, for any $i$ we have that:
\[(f_i(x)-f_i(y))(g_i(x)-g_i(y)) = 0\]
so that we have that:
\[ \sum_i f_i(x)g_i(x)  -\sum_i f_i(x)g_i(y) - \sum_i f_i(y)g_i(x) + \sum_i f_i(y)g_i(y) = 0 \]
but since for any $x:\Spec(A)$ we have:
\[\sum_i f_i(x)g_i(x) = \omega(x,x) = 0\]
We can conclude that:
\[ \sum_i f_i(x)g_i(y) + \sum_i f_i(y)g_i(x) = 0\]
which is what we want.
\end{proof}

\begin{proposition}\label{differential-form-alternating}
Any $\omega:\Omega^k(X)$ is alternating, meaning that for any $x = (x_0,\cdots,x_n):\Delta^k(X)$ and $\sigma$ is a permutation of $n+1$ elements, we have that:
\[\omega(\sigma x) = \mathrm{sign}(\sigma)\omega(x) \]
\end{proposition}

\begin{proof}
It is enough to show this for the exchange $x_i$ and $x_j$. But by fixing $x_k$ for $k\not=i,j$, this is \cref{one-form-alternating} applied in the intersection of the first-order neighbourhood of $x_k$ for $k\not=i,j$.
\end{proof}

\subsection{Differential forms and the de Rham complex}

TODO in low dimension...

