\subsection{Definition}

\begin{example}
  Let us assume $2\neq 0$ in $R$ and look at the following square:
  \begin{center}
    \begin{tikzcd}
      1\ar[r]\ar[d] & \A^1\setminus\{0\}\ar[d,"x\mapsto x^2"] \\
      \D(1)\ar[r] & \A^1\setminus\{0\}
    \end{tikzcd}
  \end{center}
  As the bottom map, we choose the inclusion of $\{x:R^\times \mid (x-1)^2=0 \}$.
  Then, for any choice of the top map, there is a unique lift in this square:
  \begin{center}
    \begin{tikzcd}
      1\ar[r]\ar[d] & \A^1\setminus\{0\}\ar[d,"x\mapsto x^2"] \\
      \D(1)\ar[r]\ar[ru,dashed] & \A^1\setminus\{0\}
    \end{tikzcd}
  \end{center}
\end{example}

A non-finitely generated version of the following definition is usually used
to define \emph{formally étale} maps of schemes
\footnote{In \cite{EGAIV3}[§17], the definition of formally étale maps ranges over arbitrary ideals, but uses the same lifting condition as below.}
and then it is subsequently noted,
that formally étale maps of finite presentation are étale.
Since all of our schemes are of finite presentations, this should be a correct definition of étale morphism:

\begin{definition}
  \begin{enumerate}[(a)]
  \item   A map $f:X\to Y$ is \notion{formally étale},
    if for all finitely presented $R$-algebras $A$ and all finitely generated nilpotent ideals $N\subseteq A$,
    and all squares like below, there is a unique lift:
    \begin{center}
      \begin{tikzcd}
        \Spec (A/N)\ar[r]\ar[d,"\iota",swap] & X\ar[d,"f"] \\
        \Spec A \ar[r]\ar[ru,dashed] & Y
      \end{tikzcd}
    \end{center}
    -- where $\iota:\Spec (A/N)\to \Spec A$ is induced by the quotient map.
  \item A map $f:X\to Y$ of schemes is \notion{étale}, if it is formally étale.
  \end{enumerate}
\end{definition}

Another way to phrase our definition of formally étale maps would be to say,
that they are the maps with the right lifting property (\cite{modalities}[definition 1.45]) with respect to ``left`` maps of the form $\Spec (A/N)\to \Spec A$.
We will use some well known, general closure properties of left maps, starting with closure under composition:

\begin{lemma}%
\label{decomposition-nilpotent}
For $A$ an algebra and $N$ a nilpotent ideal in $A$, the map:
\[A
\to A/N
\] 
can be factored as maps of the form:
\[
B\to B/(b)
\]
where $b:B$ is such that $b^2=0$.
\end{lemma}

\begin{proof}
TODO
\end{proof}

\begin{lemma}
\label{equivalence-etale}
Having the right lifting property against the following class of maps is equivalent:
\begin{enumerate}[(i)]
\item Maps of the form $\Spec(A/N)\to \Spec(A)$ for $A$ f.g. algebra and $N$ a nilpotent ideal.
\item Maps of the form $P\to 1$ for $P$ a closed dense proposition.
\item Closed dense embeddings of types.
\item Maps of the form $(\epsilon=0)\to 1$ for $\epsilon:R$ such that $\epsilon^2=0$.
\item Maps of the form $\Spec(A/(a)) \to \Spec(A)$ for $A$ f.g. algebra and $a:A$ such that $a^2=0$.
\end{enumerate}
\end{lemma}

\begin{proof}
(i) implies (ii) because closed dense proposition are of the form $\Spec(R/N)$ for $N$ a nilpotent ideal.

(ii) implies (iii) because of \cref{pointwise-lift-is-enough}.

(iii) implies (iv) because $\epsilon=0$ is closed dense when $\epsilon$ is nilpotent.

(iv) implies (v) because of \cref{pointwise-lift-is-enough}, as the fiber of $\Spec(A/(a)) \to \Spec(A)$ over $x$ is $a(x)=0$.

(v) implies (i) because of \cref{decomposition-nilpotent}.
\end{proof}

\begin{lemma}
A map is formally étale if and only if all its fibers are formally étale.
\end{lemma}

\begin{proof}
By \cref{lifting-defined-fiberwise} and the characterisation (ii) from the previous lemma.
\end{proof}

From (ii) we even get that being formally étale is a lex modality, so we have the following:

\begin{proposition}
We have the following stability results:
\begin{itemize}
\item If $X$ is any type and for all $x:X$ we have a formally étale type $Y_x$, then:
\[\prod_{x:X}Y_x\]
is formally étale. 
\item  If $X$ is formally étale and for all $x:X$ we have a formally étale type $Y_x$, then:
\[\sum_{x:X}Y_x\]
is formally étale. 
\item If $X$ is formally étale then for all $x,y : X$ the type $x=y$ is formally étale.
\item The type of formally étale types is formally étale.
\end{itemize}
\end{proposition}

\begin{lemma}%
  \label{nilpotent-ideal-not-not-dense}
  Let $A$ be a finitely presented $R$-algebra and $N\subseteq A$ be finitely generated nilpotent.
  Then for $V\colonequiv \Spec (A/N)\subseteq \Spec A$ the following holds:
  \begin{enumerate}[(a)]
  \item For all $x:\Spec A$, $\neg\neg V(x)$.
  \item If $V=\emptyset$, then $\Spec A=\emptyset$.
  \end{enumerate}
\end{lemma}

\begin{proof}
  \begin{enumerate}[(a)]
  \item
    Let $x : \Spec A$ be given.
    The generators $n_1,\dots,n_l$ of $N$ are nilpotent functions,
    so in particular the elements $n_1(x), \dots, n_l(x)$ of $R$
    are not not zero.
    This means precisely $\neg\neg V(x)$.
  \item Assume $V=\emptyset$ and $x:\Spec A$.
        We want to show the $\neg\neg$-stable proposition $\emptyset$,
        so we can assume $V(x)$, which is a contradiction.
  \end{enumerate}
\end{proof}

\subsection{Examples}

\begin{proposition}%
  Let $P$ be a $\neg\neg$-stable proposition,
  then $P\to 1$ is formally étale.
\end{proposition}

\begin{proof}
  Direct application of \cref{nilpotent-ideal-not-not-dense}.
\end{proof}

\begin{proposition}%
  The map $\Bool\to 1$ is étale.
\end{proposition}

\begin{proof}
  We have to extend maps $f:\Spec (A/(a))\to \Bool$, with $a^2=0$.
  Since $\Bool\subseteq R$, the map $f$ yields an element $f:A/(a)$
  and we have a lift $\tilde{f}:A$ with $f=\tilde{f}+ab$.
  By \cref{nilpotent-ideal-not-not-dense},
  we have for any $x:\Spec A$, that $\neg\neg(\tilde{f}(x)=0)$ or $\neg\neg(\tilde{f}(x)=1)$.

  By Z-choice or computation, we find a $n:\N$,
  such that $\tilde{f}^n(x)=0$ or $\neg\neg(\tilde{f}^n(x)=1)$.
  With the map $1-\_:R\to R$, we can achieve the same for $1$.
\end{proof}

\begin{proposition}
TODO Stable by sums, finite types.
\end{proposition}

\begin{proposition}
TODO $\N$ is formally étale.
\end{proposition}

\begin{proposition}
TODO roots of a separable polynomial.
\end{proposition}

\begin{proposition}
TODO roots of polynomials with an invertible jacobian.
\end{proposition}

\subsection{Separated étale schemes have decidable equality}

Kind of outdated.

\begin{proposition}%
	\label{decidable-of-tangent}
	Let $X$ be an affine scheme and $a : X$ a point. Suppose the tangent space
	$T_a X$ has a unique point. Then for any $b : X$, equality $a = b$ is decidable.
\end{proposition}

\begin{proof}
	Given that $T_a X = \{0\}$, we also have $T^\star_a X = \mm_a / \mm_a^2 = 0$
	by \cref{maximal-cotangent}.
	So $\mm_a^2 = \mm_a$. By \cite[Lemma II.4.6]{lombardi-quitte}
	(proved using Nakayama's lemms, or the determinant trick), $\mm_a$ is generated
	by a single idempotent $e$ of $A$. We have $e(b)$ idempotent in $R$,
	so since $R$ is local it is either 0 or 1. We have $e(a) = 0$, so if $e(b) = 1$
	we have $a \ne b$. If $e(b) = 0$, then $a = b$ since $b$ is in the order 0
	neighborhood of $a$.
\end{proof}

\begin{proposition}%
	Let $X$ be a separated \'{e}tale scheme. Then $X$ has decidable equality.
\end{proposition}

\begin{proof}
	Let $a, b : X$. Since $X$ is separated, $a = b$ is closed.
	We claim it is also open.
	To see this, pick an affine open neighbourhood $U$ of $a$.
	Now $a = b$ is equivalent to $(b \in U) \wedge (a = b)$.
	Since open propositions are closed under $\Sigma$,
	it suffices to show that $a = b$ is open assuming $b \in U$.
	In this case we can apply \cref{decidable-of-tangent}.
\end{proof}

\subsection{Hensel lifting}

\begin{lemma}
  Let $f : \A^1 \to \A^1$ and
  $p : \A^1$ with $f(p)$ nilpotent and
  $f'(p)$ invertible.
  Then there exists $q : \A^1$
  with $\lnot \lnot (q = p)$
  and $f(q) = 0$.
\end{lemma}

\begin{proof}
  Let $f(p)^n = 0$, induct on $n$.
  If $n \leq 1$ we are done.
  Otherwise, let $p' = p - f(p)/f'(p)$.
  Then $\lnot \lnot (p' = p)$ and we have
  \[ f(p')
     = f(p) - f'(p) (f(p)/f'(p)) + r (f(p)/f'(p))^2
     = r (f(p)/f'(p))^2 \]
  for some $r : R$
  by Taylor expansion.
  Thus ${f(p')}^m = 0$ for $2m \geq n$
  and we are done by inductive hypothesis.
\end{proof}
