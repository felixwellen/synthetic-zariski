
A non-fintely generated version of the following definition is usually used to define \emph{formally étale} maps of schemes and then it is subsequently noted,
that formally étale maps of finite presentation are étale.
Since all of our schemes are of finite presentations, this should be a correct definition of étale morphism.

\begin{definition}
  \begin{enumerate}
  \item   A map $f:X\to Y$ is \notion{formally étale},
    if for all finitely presented $R$-algebras $A$ and all finitely generated nilpotent ideal $N\subseteq A$,
    and all squares like below, there is a unique lift:
    \begin{center}
      \begin{tikzcd}
        \Spec (A/N)\ar[r]\ar[d,"\iota",swap] & X\ar[d,"f"] \\
        \Spec A \ar[r]\ar[ru,dashed] & Y
      \end{tikzcd}
    \end{center}
    -- where $\iota:\Spec (A/N)\to \Spec A$ is induced by the quotient map.
  \item A map $f:X\to Y$ of schemes is \notion{étale}, if it is formally étale.
  \end{enumerate}
\end{definition}

\begin{proposition}%
  Let $P$ be a $\neg\neg$-stable proposition,
  then $P\to 1$ is formally étale.
\end{proposition}