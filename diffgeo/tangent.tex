\subsection{The Tangent Space}

We will use the concept of tangent spaces from synthetic differential geometry.
More concretely, we follow \cite{david-orbifolds}[Section 4]
on the subject, which also uses homotopy type theory as a basis.

\begin{definition}
  The \notion{first order disk}\index{$\D(n)$} of dimension $n$ is the type
  \[ \D(n)\colonequiv \{x:R^n \mid xx^T = 0 : R^{n \times n}\}\rlap{.}\]

  More generally, for an $R$-module $V$, the first order disk of $V$ is the type
  \[ \D(V) \colonequiv \{ f : V^\star \mid \forall v\,v'. f(v)f(v') = 0 \}, \]
  where $V^\star$ is the $R$-linear dual of $V$.
\end{definition}

Note that $\D(n)$ is equivalent to $\D(R^n)$.

\begin{definition}
  For $V$ an $R$-module, the associated square-zero extension of $R$ is the
  $R$-module $R \oplus V$ with multiplication given by
  \[ (r,v)(r',v') \coloneqq (rr', rv' + r'v).\]
\end{definition}
Note that if $V$ is a finitely presented $R$-module, then so is $R \oplus V$.
Hence in particular $R \oplus V$ is a finitely presented $R$ algebra in this case.

\begin{lemma}
  For any $R$-module $V$, we have an equivalence
  \[ \Spec(R \oplus V) \simeq \D(V)\]
\end{lemma}
\begin{proof}
An $R$-algebra morphism $R \oplus V \to R$ is determined by an $R$ linear map
$R \oplus V \to R$ that respects multiplication. Equivalently, this is an $R$-linear map
$f : V \to R$ such that $rr' + f(rv'+r'v) = rr' + r'f(v) + rf(v') + f(v)f(v')$.
By linearity, this condition amounts to $f(v) f(v') = 0$.
\end{proof}

Thus $\D(V)$ is an affine scheme whenever $V$ is finitely presented. In particular
$\D(n)$ is always an affine scheme. %
% TODO commented out for now, maybe bring the example back later
%
%\begin{example}[using \axiomref{sqc}]%
%  \label{example-tangent-bundle}
%  Let $S\colonequiv\Spec R[X,Y]/(X^2+Y^2-1)$.
%  We will see that $S^{\D(1)}$ is affine by the following calculation:
%  \begin{align*}
%    S^{\D(1)} &= \Spec R[\varepsilon] \to \Spec R[X,Y]/(X^2+Y^2-1) \\
%              &= \Hom_R(R[X,Y]/(X^2+Y^2-1), R[\varepsilon]) \\
%              &= \{{z,w:R[\varepsilon]} \mid z^2+w^2=1 \} \\
%              &= \{x+\varepsilon u, y+\varepsilon v:R+\varepsilon R \mid (x+\varepsilon u)^2+(y+\varepsilon v)^2=1 \} \\
%              &= \{x,y,u,v:R \mid x^2+y^2+\varepsilon (2xu+2yv)=1 \} \\
%              &=\{x,y,u,v:R \mid (x^2+y^2=1)\times(2xu+2yv=0)\} \\
%    &=\Spec R[X,Y,U,V]/(X^2+Y^2-1,2XU+2YV)
%  \end{align*}
%\end{example}
More generally than first order disks, we can consider infinitesimal varieties:

\begin{definition}
  \begin{enumerate}[(a)]
  \item A \notion{Weil algebra} over $R$ is a finitely presented $R$-algebra
    $W$ together with a homomorphism $\pi : W \to R$, such that the kernel of $\pi$ is a nilpotent ideal.
  \item An \notion{infinitesimal variety} is a pointed type $D$,
    such that $D = (\Spec W, \pi)$ for a Weil algebra $(W, \pi)$.
  \end{enumerate}
\end{definition}

Note that the kernel of $\pi$ is a finitely generated ideal,
as the kernel of a surjective homomorphism between finitely presented algebras.
To ask that $\ker \pi$ is nilpotent as an ideal
is therefore the same as to ask that each of its elements is nilpotent.

\begin{lemma}[using \axiomref{loc}, \axiomref{sqc}]%
  \label{characterization-infinitesimal-variety}
  A pointed affine scheme $(\Spec A, \pi)$
  is an infinitesimal variety
  if and only if,
  for every $x : \Spec A$ we have $\lnot \lnot (x = \pi)$.
\end{lemma}

\begin{proof}
  First note that we can choose generators
  $X_1,\ldots,X_n$ of $A$
  such that $\pi(X_i) = 0$ for all $i$
  (by replacing $X_i$ with $X_i - \pi(X_i)$ if necessary)
  and therefore $\ker \pi = (X_1, \dots, X_n)$.

  Assume $(\Spec A, \pi)$ is an infinitesimal variety.
  By (\axiomref{sqc}), this means that $(A, \pi)$ itself is a Weil algebra,
  so every $X_i$ is nilpotent in $A$.
  Now if $x : \Spec A$ is any homomorphism $A \to R$,
  then $x(X_i)$ is also nilpotent in $R$,
  meaning, by \cref{nilpotence-double-negation}, that $\lnot \lnot (x(X_i) = 0)$.
  Since we have this for all $i = 1, \dots, n$
  and double negation commutes with finite conjunctions,
  we have $\lnot \lnot (x = \pi)$.

  Now assume $\lnot \lnot (x = \pi)$ for all $x : \Spec A$.
  To show that $(A, \pi)$ is a Weil algebra,
  let $f : A$ be given with $\pi(f) = 0$.
  Then in particular we have $\lnot \lnot (x(f) = 0)$
  for every $x : \Spec A$.
  But this means $D(f) = 0$
  (using (\axiomref{loc}) for $\inv(x(f)) \Rightarrow x(f) \neq 0$),
  so $f$ is nilpotent by \cref{standard-open-empty}.
\end{proof}

The following lemma allows us to reduce
maps from infinitesimal varieties to schemes
to the affine case:

\begin{lemma}%
  \label{affine-opens-infinitesimal-closed}
  Let $X$ be a scheme, $V$ an infinitesimal variety and $p:X$.
  Then for all affine open $U\subseteq X$
  containing $p$, there is an equivalence
  of pointed mapping types:
  \[ V\to_\pt  (X, p) \quad\cong\quad V\to_\pt  (U, p) \]
\end{lemma}

\begin{proof}
  By \cref{characterization-infinitesimal-variety},
  all points in $V$ are not not equal,
  so all points in the image of a pointed map
  \[ V \to_\pt  (X, p) \]
  will be not not equal to $p$.
  Since $p \in U$ and open propositions are double-negation stable
  (\cref{open-union-intersection}),
  the image is contained in $U$
  and the map factors uniquely over $(U, p)$.
\end{proof}

\begin{definition}
  Let $X$ be a type.
  \begin{enumerate}[(a)]
  \item For $p:X$ a point in $X$.
    The \notion{tangent space}\index{$T_pX$} of $X$ at $p$, is the type
    \[ T_p X \colonequiv \{ t : \D(1) \to X \mid t(0)=p \}\rlap{.}\]
  \item The \notion{tangent bundle} of $X$ is the type $X^{\D(1)}$.
  \end{enumerate}
\end{definition}

Note that for any map $f : X \to Y$, we have a map 
$Df_p : T_p X \to T_{f(p)}Y$ given by $Df_p(t,x) = f(t(x))$.
This makes tangent spaces functorial.

\begin{definition}
For $A$ an $R$-algebra and $M$ an $A$-module, a \notion{derivation}
is an $R$-linear map $d : A \to M$ satisfying the Leibniz rule
\[ d(fg) = f \cdot dg + g \cdot df.\]
\end{definition}

\begin{lemma}
For $A$ an $R$-algebra and $V$ an $R$-module, we have an equivalence
between $R$-algebra maps $A \to R \oplus V$ and 
pairs $(p,d)$ where $p : A \to R$ is an $R$-algebra map, and
$d : A \to V$ is a derivation, where the $A$-module structure on $V$
is obtained by restricting scalars along $p$.
\end{lemma}
\begin{proof}
An $R$-algebra map is $A \to R \oplus V$ is given by an 
$R$-algebra map $p : A \to R$ and an $R$-linear map
$d : A \to V$ such that $a \mapsto (p(a), d(a))$ respects multiplication.
Since $p$ respects multiplication and $d$ is linear, this
amounts to the Leibniz rule.
\end{proof}

We transfer a result of Myers \cite{david-orbifolds}[Theorem 4.2.19] to schemes:

\begin{theorem}[using \axiomref{sqc}]%
  Let $X$ be a scheme and $p : X$ a point. Then $T_pX$ is an $R$-module.
\end{theorem}

We emphasize that this is an untruncated statement: we construct an
$R$-module structure on $T_p X$, as opposed to showing the mere existence of
an $R$-module structure.

\begin{proof}
  Following the proof of \cite{david-orbifolds}[Theorem 4.2.19],
  it is enough to show that any scheme is infinitesimal linear
  in the sense that
  \[
    \begin{tikzcd}
      X^{\D(n+m)}\ar[r]\ar[d] & X^{\D(n)}\ar[d] \\
      X^{\D(m)}\ar[r]         & X
    \end{tikzcd}
  \]
  is a pullback for all $n,m:\N$.
  This amounts to showing that for any point $p : X$, the map 
  \[
    \left(\D(n+m)\to_\pt X\right) \to 
    \left(\D(n)\to_\pt X\right) \times \left({\D(m)}\to_\pt X\right)
  \]
  is an equivalence.
  Since this statement is a proposition, we may assume given a
  an affine open $U = \Spec A$ containing $p$.
  By \cref{affine-opens-infinitesimal-closed}, we may assume $X = U$.
  In this case, $\D(n) \to_\pt X$ is equivalent to the type
  of derivations $A \to R^n$, where the $A$-module structure on $R^n$
  is obtained by restricting scalars along $p : A \to R$.
  It is clear that a derivation $A \to R^n \oplus R^m$
  is described by a pair of derivations $A \to R^n$, $A \to R^m$,
  as needed.
\end{proof}

We can characterise addition $+ : T_p X \to T_p X \to T_p X$ on
tangent spaces as follows. Given $f, g : T_p X$,
there is a unique $h : \D(2) \to X$ such that
$h(x,0) = f(x)$, $h(0,y) = g(y)$ for all $x, y : \D(1)$.
Then $(f + g)(x) = h(x,x)$.
The scalar action of $R$ on $T_p X$ is given by
$(rf)(x) = f(rx)$.
This makes it clear that $Df_p : T_p X \to T_p X$ is $R$-linear.

For an alternative characterisation, we note that for any 
map $q : X \to \A^1$, we have
$q((f+g)(x)) = q(f(x)) + q(g(x))$. This determines $f + g : \D(1) \to X$
uniquely in the case where points of $X$ are separated by functions
$X \to \A^1$ (e.g. if $X$ is an affine scheme).

\begin{lemma}
For any $p : \A^n$, the map
$R^n \to T_p \A^n$ given by
$v \mapsto x \mapsto p + xv$ is an isomorphism of $R$-modules.
\end{lemma}
\begin{proof}
It is direct that the map is $R$-linear. It remains to show it is an equivalence.
By considering each component separately, we may assume $n = 1$.
In this case $T_p \A^1$ corresponds by SQC to elements of $R \oplus R$
whose first component is zero.
We omit the verification that the map is the one we described.
\end{proof}

\begin{definition}
We say an $R$-module $V$ is \notion{finitely co-presented} 
if it can merely be represented as the kernel of some linear 
map $R^n \to R^m$ of finite free modules.

Equivalently, an $R$-module is finitely co-presented if it is the dual
of some finitely presented $R$ module.
\end{definition}

\begin{lemma}
Let $X$ be a scheme and $p : X$ a point. Then $T_p X$ is finitely
co-presented.

Explicitly, if $X$ is affine and given by the following pullback
diagram 
\[
\begin{tikzcd}
X \arrow[r] \arrow[d] \arrow [dr,phantom,"\lrcorner",very near start] 
  & 1 \arrow[d] \\
\A^n \arrow[r] & \A^m
\end{tikzcd}
\]
then $T_p X$ is given by a pullback diagram of $R$-modules
\[
\begin{tikzcd}
T_p X \arrow[r] \arrow[d] \arrow [dr,phantom,"\lrcorner",very near start] 
  & 1 \arrow[d] \\
R^n \arrow[r] & R^m
\end{tikzcd}
\]
\end{lemma}
\begin{proof}
If $X$ is a general scheme, we reduce to the affine case by picking an affine patch.
The affine case follows from the fact that pointed exponentiation with $\D(1)$
preserves pullback squares.
\end{proof}
Since finitely co-presented modules are affine schemes, it follows that any tangent space
of a scheme is an affine scheme. Since schemes are closed under sigma-types,
the tangent bundle of any scheme is again a scheme.

The following is a completely general algebraic fact about Taylor expansions of polynomials.
\begin{lemma}
For any map $f : \A^n \to \A^m$, the Jacobian of $f$ is an $n \times m$ matrix
describing a linear map $Jf_p : R^n \to R^m$, such that
for all $p : \A^n$, $x : \D(1)$, $v : R^n$, we have
\[ f(p + xv) = f(p) + x Jf_p v.\]
\end{lemma}

It follows that the derivative $R^n \to R^m$ of a map $\A^n \to \A^m$
at a point of $\A^n$ is given by the Jacobian matrix. In this way we can effectively
compute the tangent space of an affine scheme.
For example, since the Jacobian of a linear map is that same linear map,
we see that any tangent space of a finitely co-presented $R$-module
is naturally that same $R$-module.

\begin{lemma}\label{equivalence-module-infinitesimal}
Let $M$, $N$ be finitely presented modules. Then linear maps $M \to N$ correspond to
pointed maps $\D(N) \to_\pt \D(M)$. Explicitly, a linear map $g : M \to N$
corresponds to the pointed map $f \mapsto m \mapsto f(g(m))$.
\end{lemma}
\begin{proof}
Pointed maps $\D(N) \to_\pt \D(M)$ correspond to 
$R$-algebra maps $R \oplus M \to R \oplus N$ lifting the
projection $R \oplus M \to R$, and hence to derivations
$R \oplus M \to N$, where the $R \oplus M$-module structure on $M$ is obtained by
restricting scalars along the projection $R \oplus M \to R$.
The Leibniz rule amounts to $dr = 0$ for $r : R$, so we obtain all $R$-linear 
maps $M \to N$ in this way.
\end{proof}

\begin{lemma}\label{fp-duality}
Every finitely presented $R$-module is naturally isomorphic to its double dual.
Hence every finitely co-presented $R$-module is naturally isomorphic to its double dual.
Hence taking $R$-linear duals is a self-inverse contravariant equivalence of categories between
finitely presented and finitely co-presented $R$-modules.
\end{lemma}
Note that this is true for discrete fields but wildly false for general rings. 
For example $\Z/2\Z$ is a finitely presented $\Z$-module whose double dual is zero.
\begin{proof}
Let $M$ be a finitely presented $R$-module, and let $c : M^\star \to R$ be an element 
of the double dual.
We can restrict $c$ to a map $\D(M) \to_\pt R$. By SQC, this corresponds to
an element of $R \oplus M$ whose first component is zero.
It remains to check that this determines an equivalence $M^{\star\star} \simeq M$.
By construction, $c$ is sent to $m : M$ if and only if for all
$f : \D(M)$, $c(f) = f(m)$.
It remains to show that this condition is equivalent to the condition
that for all $f : M^\star$, $c(f) = f(m)$.
The reverse implication is clear. Thus let us consider the forward implication.
Suppose $f : M^\star$ and we want to show $c(f) = f(m)$ in $R$.
By SQC it suffices to show $x c(f) = xf(m)$ for all $x : \D(1)$.
   Indeed $xc(f) = c(xf) = xf(m)$ by linearity and using the fact
   that $xf : \D(M)$.
\end{proof}

\begin{lemma}\label{functions-on-module}
For $V$ a finitely co-presented $R$-module, $V$ is the spectrum of
the free $R$-algebra on the $R$-module $V^\star$, i.e. of the symmetrisation of
the tensor algebra on $V^\star$.
\end{lemma}
\begin{proof}
By universal properties, the relevant spectrum is equivalent to the type
of $R$-module maps $V^\star \to R$, i.e. to $V$.
\end{proof}
