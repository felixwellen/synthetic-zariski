
\rednote{It is not clear yet, if the general case is provable in the generality of SAG. The following is just a collection of results that might help.}

For $z:\bP^{19}$, let $C_z$ be the cubic given by using the components of $z$ as coefficients for a cubic form on $\bP^3$.
Then the type of lines in $C_z$ is projective. Let $\mathrm{Lines}(3)$ denote the type of lines in $\bP^3$ and
let \[Z\colonequiv\sum_{z:\bP^{19}}\sum_{l:\mathrm{Lines}(3)}l\subseteq C_z\rlap{.} \]
be the type of lines on cubics. As a closed subset of a product of projective spaces, \(Z\) is projective.
We have a span:
\begin{center}
\begin{tikzcd}
  & Z\ar[rd,"\alpha"]\ar[ld,"\beta",swap] & \\
  \bP^{19} & & \mathrm{Lines}(3)
\end{tikzcd}
\end{center}
where $\beta$ is the projection onto the first factor and $\alpha$ maps a tuple to its line.

\begin{lemma}
  The subtype $\{ z:\bP^{19} \;|\; C_z \;\text{is smooth} \}$ is open.
\end{lemma}

\begin{proof}
  We want to show that $C_z$ being smooth is an open proposition.
  Let $C_z = V(P)$. Then by virtue of corollary \ref{projective-jacobi-criterion}, smoothness of $C_z$ is equivalent to
  \[
    \prod_{x:C_z} J_xP \neq 0,
  \]
  which is open since $C_z$ is compact.
\end{proof}

\begin{lemma}
  For all \(l:\mathrm{Lines}(3)\) we have
  \[
  \|\fib_\alpha(l)=\bP^{15}\|
  \]
\end{lemma}

\begin{proof}
For two lines \(l,l^\prime:\mathrm{Lines}(3)\) there merely is \(T:\PGL_{4}(R)\) such that \(T(l)=l^\prime\)\rednote{TODO: prove}.
Any such \(T\) induces an equivalence of fibers \(\fib_\alpha(l)=\fib_\alpha(l^\prime)\).
So we can conclude by showing \(\fib_\alpha(l)=\bP^{15}\) for a fixed line \(l\).
Let \(l=V(X_0,X_1)\subseteq\bP^3\). Then \[\fib_\alpha(l)=\left(\sum_{z:\bP^{19}}\prod_{x_2,x_3:R}C_z(0,0,x_2,x_3)=0\right)=\bP^{15}\rlap{.}\]
\end{proof}

\begin{corollary}
  \label{topological-properties-lines-on-cubics}
\begin{center}
\begin{enumerate}[(a)]
\item \(\alpha\) is smooth.
\item \(Z\) is smooth and irreducible.
\item The complement of \(\im(\beta)\) is open and if it is non-empty, it is dense.
\end{enumerate}
\end{center}
\end{corollary}

\begin{proof}
\begin{enumerate}[(a)]
\item By definition of smooth in \cite{diffgeo-article}.
\item \(Z\) is smooth, since dependent sums of smooth types are smooth. Same for irreducible.
\item Since \(Z\) is projective and therefore compact, the proposition \((x\notin \im(\beta))=((z:Z)\to \beta(z)\neq x)\) is open. \(\bP^{19}\) is irreducible,
so any non-empty open is dense.
\end{enumerate}
\end{proof}

Now we will use the preparation from the example of the Fermat Curve, that the existence of a line is stable under infinitesimal variation, to show the following:

\begin{lemma}
  \(\im(\beta)\) has empty complement. So on any cubic, there not not is a line.
\end{lemma}

\begin{proof}
  \(\beta:Z\to \bP^{19}\) is a map from a compact scheme, so by \cite{proper-draft}[Remark 3.0.8],
  we merely have a closed \(V\subseteq \bP^{19}\) such that \(\im(\beta)\subseteq V\) and \(\neg V=\neg \im(\varphi)\).
  Now by the example in the previous section, there is \(z:Z\) and a chart \(D(X_i)\subseteq \bP^{19}\),
  such that \(\beta(z):D(X_i)\) and the formal neighbourhood around \(\beta(z)\) is contained in
  \(D(X_i)\cap V\). But then we already have \(D(X_i)\subseteq V\) and by \Cref{topological-properties-lines-on-cubics} (c) we have \(\neg V=\emptyset\).
\end{proof}

Over a smooth cubic, we can say more about the fibers.
We hope/expect that the following leads to a synthetic version of the classic result that the fibers of \(\beta\) over smooth cubics are finite sets.
We start by showing that \(\beta\) is unramified over smooth cubics, which together with smoothness will imply that \(\beta\) is étale over smooth cubics. Classically, it is possible to conclude that \(\beta\) is a finite morphism over smooth cubics, but we do not yet know a synthetic proof of this.

\begin{lemma}
  \(\beta\) is unramified over smooth cubics.
\end{lemma}

\begin{proof}
  Equivalently, we show that the type of lines on a smooth cubic has an étale identity type.
  So let \(C\) be a smooth cubic and \(\mathcal L,\mathcal L^\prime\) be lines on \(C\). By applying an automorphism of \(\bP^3\),
  we can assume that \(\mathcal L\) is the line \([t:s]\mapsto [0:0:t:s]\).
  
  Now we show that \(\mathcal L=\mathcal L^\prime\) is étale using the following characterization: For all \(\epsilon:R\) such that \(\epsilon^2=0\) and \(\mathcal L=\mathcal L^\prime\) if \(\epsilon=0\), we already have \(\mathcal L=\mathcal L^\prime\).
  In particular, we can assume that \(\mathcal L\) and \(\mathcal L^\prime\) are not not equal.
  Let \(\mathcal L^\prime\) be given by two different points \([a_0:a_1:a_2:a_3],[b_0:b_1:b_2:b_3]:\bP^3\), i.e.\ \[
    \mathcal L^\prime\colonequiv[t:s]\mapsto [a_0\cdot t+b_0\cdot s:\dots : a_3\cdot t+b_3\cdot s]
  \]
  When showing a negation, we can assume \(\mathcal L=\mathcal L^\prime\) and therefore
  \([0:0:1:0]\in\mathcal L^\prime\).
  This means that \(a_2\) or \(b_2\) has to be invertible.
  Without loss of generality, assume \(a_2=1\) and \(b_2=0\).
  We will now show \(b_3\) is invertible. So assume \(b_3=0\).
  We have \([0:0:0:1]=\mathcal L^\prime([t_0:s_0])\) for some \(t_0:s_0\),
  so \(t_0=0\) and \(a_3t_0=1\), which is a contradiction.
  So we can assume  \(b_3=1\) and \(a_3=0\) without loss of generality.
  So, in fact we can write \(\mathcal L^\prime\) as follows:
  \[
    \mathcal L^\prime\colonequiv[t:s]\mapsto [a_0t+b_0 s:a_1 t+b_1 s : t : s]
  \]
  with \(a_0,b_0,a_1,b_1:R\).

  Now we will show that for the ideal \(I\colonequiv(a_0,b_0,a_1,b_1) \) we have \(I^2=0\).
  We have \(\epsilon=0\) implies \(I=0 \), by using that the two points \([0:0:1:0]\) and \([0:0:0:1]\) lie on \(\mathcal L^\prime\) if \(\epsilon=0\). By \cite{draft}[Proposition 4.1.2], we get \(I\subseteq (\epsilon)\) and therefore \(I^2=0\).

  Now let \(C\) be given by a cubic form on \(\bP^3\) where for \(\alpha=(\alpha_0,\dots,\alpha_3):\N^4\), \(\sum\alpha_i=3\) the monomial \(X^\alpha=X_0^{\alpha_0}\cdots X_3^{\alpha_3}\) has coefficient \(c_\alpha:R\).
  By \(\mathcal L\subseteq C\) we know \(c_\alpha=0\) if \(\alpha_0=0\) and \(\alpha_1=0\).
  Now we plug \(\mathcal L^\prime\) into the cubic form:
  \begin{align*}
    &\sum_{\alpha:\N^4, \sum_\alpha=3}c_\alpha (a_0t+b_0s)^{\alpha_0}(a_1t+b_1s)^{\alpha_1}t^{\alpha_2}s^{\alpha_3}\\
    =&\sum_{\alpha:\N^4, \sum_\alpha=3,\alpha_0+\alpha_1=1}c_\alpha (a_0t+b_0s)^{\alpha_0}(a_1t+b_1s)^{\alpha_1}t^{\alpha_2}s^{\alpha_3} \\
    =& (a_0t+b_0s)X_0(c_{(1,0,2,0)}X_2^2+c_{(1,0,1,1)}X_2X_3+c_{(1,0,0,2)}X_3^2)+ \\
      &(a_1t+b_1s)X_1(c_{(0,1,2,0)}X_2^2+c_{(0,1,1,1)}X_2X_3+c_{(0,1,0,2)}X_3^2)
  \end{align*}
  \rednote{TODO: finish writing proof}
\end{proof}
