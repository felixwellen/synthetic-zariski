
\rednote{It is not clear yet, if the general case is provable in the generality of SAG. The following is just a collection of results that might help.}

For $z:\bP^{19}$, let $C_z$ be the cubic given by using the components of $z$ as coefficients for a cubic form on $\bP^3$.
Then the type of lines in $C_z$ is projective. Let $\mathrm{Lines}(3)$ denote the type of lines in $\bP^3$ and
let \[Z\colonequiv\sum_{z:\bP^{19}}\sum_{l:\mathrm{Lines}(3)}l\subseteq C_z\rlap{.} \]
be the type of lines on cubics. As a closed subset of a product of projective spaces, \(Z\) is projective.
We have a span:
\begin{center}
\begin{tikzcd}
  & Z\ar[rd,"\alpha"]\ar[ld,"\beta",swap] & \\
  \bP^{19} & & \mathrm{Lines}(3)
\end{tikzcd}
\end{center}
where $\beta$ is the projection onto the first factor and $\alpha$ maps a tuple to its line.

\begin{lemma}
  For all \(l:\mathrm{Lines}(3)\) we have
  \[
  \|\fib_\alpha(l)=\bP^{15}\|
  \]
\end{lemma}

\begin{proof}
For two lines \(l,l^\prime:\mathrm{Lines}(3)\) there merely is \(T:\PGL_{4}(R)\) such that \(T(l)=l^\prime\)\rednote{TODO: prove}.
Any such \(T\) induces an equivalence of fibers \(\fib_\alpha(l)=\fib_\alpha(l^\prime)\).
So we can conclude by showing \(\fib_\alpha(l)=\bP^{15}\) for a fixed line \(l\).
Let \(l=V(X_0,X_1)\subseteq\bP^3\). Then \[\fib_\alpha(l)=\left(\sum_{z:\bP^{19}}\prod_{x_2,x_3:R}C_z(0,0,x_2,x_3)=0\right)=\bP^{15}\rlap{.}\]
\end{proof}

\begin{corollary}
  \label{topological-properties-lines-on-cubics}
\begin{center}
\begin{enumerate}[(a)]
\item \(\alpha\) is smooth.
\item \(Z\) is smooth and irreducible.
\item The complement of \(\im(\beta)\) is open and if it is non-empty, it is dense.
\end{enumerate}
\end{center}
\end{corollary}

\begin{proof}
\begin{enumerate}[(a)]
\item By definition of smooth in \cite{diffgeo-article}.
\item \(Z\) is smooth, since dependent sums of smooth types are smooth. Same for irreducible.
\item Since \(Z\) is projective and therefore compact, the proposition \((x\notin \im(\beta))=((z:Z)\to \beta(z)\neq x)\) is open. \(\bP^{19}\) is irreducible,
so any non-empty open is dense.
\end{enumerate}
\end{proof}

Now we will use the preparation from the example of the Fermat Curve, that the existence of a line is stable under infinitesimal variation, to show the following:

\begin{lemma}
  \(\im(\beta)\) has empty complement. So on any cubic, there not not is a line.
\end{lemma}

\begin{proof}
  \(\beta:Z\to \bP^{19}\) is a map from a compact scheme, so by \cite{proper-draft}[Remark 2.0.2],
  we merely have a closed \(V\subseteq \bP^{19}\) such that \(\im(\beta)\subseteq V\) and \(\neg V=\neg \im(\varphi)\).
  Now by the example in the previous section, there is \(z:Z\) and a chart \(D(X_i)\subseteq \bP^{19}\),
  such that \(\beta(z):D(X_i)\) and the formal neighbourhood around \(\beta(z)\) is contained in
  \(D(X_i)\cap V\). But then we already have \(D(X_i)\subseteq V\) and by \Cref{topological-properties-lines-on-cubics} (c) we have \(\neg V=\emptyset\).
\end{proof}

We expect/hope, that for smooth cubics, this result can be strengethend to étale existence of lines,
since the fibers of \(\beta\) over smooth cubics are expected to be étale.
