We collect some useful results.

\begin{lemma}
  \label{homogenous-polynomial-zero}
  A homogenous polynomial $P:R[X_0,\dots,X_n]$ has vanishing set $\bP^n$,
  if and only if all coefficients of $P$ are zero.
\end{lemma}

\begin{proof}
  We look at $Q=P(1,X_1,\dots,X_n)$ assuming $V_{\bP^n}(P)=\bP^n$.
  So as a function $R^n\to R$, $Q$ agrees with the zero-function.
  Since we have $(R^n\to R)=R[X_1,\dots,X_n]$ by duality, we know that $Q$ must be the zero-polynomial.
  Since all coefficients of $P$ appear as coefficients of $Q$, $P$ is the zero-polynomial as well.
\end{proof}

\rednote{The following is wrong and we have to figure out what we actually need below}
\begin{lemma}
   Let $f : X \to Y$ be a map between schemes with X smooth and let $y:Y$ be arbitrary.
   Then the following are equivalent:
   \begin{enumerate}[(a)]
     \item The fiber $\fib_f(y)$ is smooth.
     \item For all $x : \fib_f(y)$, the induced map \[ d_xf : T_x(X) \to T_y(Y ) \] is surjective.
  \end{enumerate}
\end{lemma}

\begin{proof}
  Careful review of the proof of \cite[Corollary 4.1.1]{diffgeo-article} affords us the stated result.
\end{proof}


\begin{corollary}
  Let $X = \Spec R[x_1,...x_n]/(P)$ be the vanishing locus of a single polynomial $P \neq 0$.
  Then $X$ is smooth if and only if its Jacobian $J: R^n \to R$ is surjective (or equivalently, non-zero).
\end{corollary}

\begin{proof}
  Consider $P$ as a map $R^n \to R$.
  Then $X$ is the fiber of this map over $0$, which is smooth if and only if $d_xP : R^n \to R$ is surjective by the previous lemma.
\end{proof}

\begin{corollary}
  \label{projective-jacobi-criterion}
  Let $X = V(P) \subseteq \bP^n$ be the projective vanishing locus of a homogeneous polynomial $P \neq 0$.
  Then $X$ is smooth if and only if the projective Jacobian $J: R^{n+1} \to R$ is surjective.
\end{corollary}

\begin{proof}
  \rednote{TODO: add argument using affine cone and local character of smoothness}
\end{proof}
