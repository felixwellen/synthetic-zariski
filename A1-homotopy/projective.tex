\begin{lemma}
Any $\shape_{\A^1}$-connected map is surjective. In particular, for any type $X$ the map:
\[\eta_X:X\to \shape_{\A^1}X\]
is surjective.
\end{lemma}

\begin{proof}
This holds because any $\shape_{\A^1}$-connected type is merely inhabited, as any proposition is $\A^1$-local so that for any $X$ we have a map:
\[\shape_{\A^1}X \to \propTrunc{X}\]
\end{proof}

%\begin{lemma}\label{colimit-shape}
%For any colimit:
%\[\mathrm{colim}_{i:I} X_i\]
%that exists in HoTT (e.g. pushouts, sequential colimits), the map:
%\[\mathrm{colim}_{i:I} X_i \to \mathrm{colim}_{i:I} \shape_{\A^1}X_i\]
%is a $\shape_{\A^1}$-equivalence.
%\end{lemma}

%\begin{proof}
%For any $\A^1$-local type $Y$, we have that:
%\[(\mathrm{colim}_{i:I} \shape_{\A^1}X_i) \to Y \simeq \mathrm{lim}_{i:I} (\shape_{\A^1}X_i \to Y)\]
%\[\simeq \mathrm{lim}_{i:I} (X_i \to Y) \simeq (\mathrm{colim}_{i:I} X_i) \to Y\]
%which implies what we want.
%\end{proof}

\begin{lemma}\label{colimit-shape}
If colimits indexed by $I$ exists in HoTT (e.g. pushouts, sequential colimits, quotients of group actions), and we have a map of $I$-indexed diagram:
\[f_i : X_i \to Y_i\]
such that for all $i:I$ the map $f_i$ is a $\shape_{\A^1}$-equivalence, then the induced map: 
\[\mathrm{colim}_{i:I} X_i \to \mathrm{colim}_{i:I} Y_i\]
is a $\shape_{\A^1}$-equivalence.
\end{lemma}

\begin{proof}
%Same as \cref{colimit-shape}.
For any $\A^1$-local type $Z$, we have that:
\[(\mathrm{colim}_{i:I} Y_i) \to Z \simeq \mathrm{lim}_{i:I} (Y_i \to Z)\]
\[\simeq \mathrm{lim}_{i:I} (X_i \to Z) \simeq (\mathrm{colim}_{i:I} X_i) \to Z\]
which implies what we want.
\end{proof}

We often use this lemma with the $\shape_{\A^1}$-equivalences:
\[\eta_X : X \to \shape_{\A^1} X\]

\begin{proposition}
We have that:
\[\shape_{\A^1} \bP^1 \simeq \shape_{\A^1} \Susp(\A^\times)\]
\end{proposition}

\begin{proof}
We apply the \cref{colimit-shape} to the pushout diagram:
 \begin{center}
    \begin{tikzcd}
      \A^\times\ar[r]\ar[d] &  \A^1\ar[d] \\
      \A^1\ar[r] & \bP^1
    \end{tikzcd}
  \end{center}
  \end{proof}

\begin{lemma}\label{free-module-minus-point-shape}
For any natural number $n$ and any $V:B\A^\times$ we have a $\shape_{\A^1}$-equivalence:
\[V^{n}-\{0\} \to (V^\times)^{*n}\]
where $(V^\times)^{*n}$ is the $n$-th iterated join of $V^\times$.
\end{lemma}

\begin{proof}
It is clear for $n=0$ or $1$. Inductively we can apply \cref{colimit-shape} to the $\shape_{\A^1}$-equivalences from the pushout diagram:
 \begin{center}
    \begin{tikzcd}
      V^\times\times V^n-\{0\}\ar[r]\ar[d] & V\times V^n-\{0\}\ar[d] \\
      V^\times \times V^n \ar[r] & V^{n+1}-\{0\}
    \end{tikzcd}
  \end{center}
  to the pushout diagram:
   \begin{center}
    \begin{tikzcd}
      V^\times\times (V^\times)^{*n}\ar[r]\ar[d] & 1\times (V^\times)^{*n}\ar[d] \\
      V^\times \times 1\ar[r] & (V^\times)^{*n+1}
    \end{tikzcd}
  \end{center}
\end{proof}

\begin{proposition}
For any $n$ we have that:
\[\shape_{\A^1} \bP^n \simeq \shape_{\A^1} \sum_{V:B\A^\times} (V^\times)^{*n+1}\]
\end{proposition}

\begin{proof}
Using the fact that:
\[\bP^n = \sum_{V:B\A^\times} V^{n+1}-\{0\}\]
with \cref{colimit-shape} and \cref{free-module-minus-point-shape}.
\end{proof}

One might try to compute the previous type more explicitely using the following result:

\begin{lemma}
For any $n$ we have a pushout diagram:
 \begin{center}
    \begin{tikzcd}
      (R^\times)^{*n}\ar[r]\ar[d] & \sum_{V:B\A^\times} (V^\times)^{*n}\ar[d] \\
      1 \ar[r] & \sum_{V:B\A^\times} (V^\times)^{*n+1}
    \end{tikzcd}
  \end{center}
  where the top map is the natural inclusion.
\end{lemma}

\begin{proof}
TODO
\end{proof}

\begin{proposition}
We have that:
\[\shape_{\A^1} \bP^\infty \simeq \shape_{\A^1} B\A^\times\]
\end{proposition}

\begin{proof}
TODO
\end{proof}