
\begin{definition}
  A type $X$ is called \notion{$\A^1$-modal}, if all maps $\gamma:\A^1\to X$ factor uniquely over $1$:
  \begin{center}
    \begin{tikzcd}
      \A^1\ar[r,"\gamma"]\ar[d] & X \\
      1\ar[ru,dashed,"\exists!"]
    \end{tikzcd}
  \end{center}
\end{definition}

\begin{definition}
  Let $\shape_{\A^1}$ be the nullification modality at $\A^1$ and $\sigma_X:X\to \shape_{\A^1}X$ its unit at a type $X$. 
\end{definition}

As a consequence, $X$ is $\A^1$-modal, if and only if, $\shape_{\A^1}X=X$.

The following was observed by David Jaz Myers in 2018 for affine schemes of the form $\Spec (R[X]/P)$ for some special polynomials $P$.
We rediscovered this for a similar class of schemes by using surprising results on étale schemes.

\begin{proposition}
  Let $X$ be a type with decidable equality, then $X$ is $\A^1$-modal.
  In particular, every separated étale scheme is $\A^1$-modal.
\end{proposition}

\begin{proof}
  Let $\gamma:\A^1\to X$.
  Then $\gamma(0):X$, so we get $\tilde{\gamma}$ with:
  \begin{center}
    \begin{tikzcd}
      \A^1\ar[r,"\gamma"]\ar[d,equal] & X\ar[d,equal] & \\
      \A^1\ar[r,"\tilde{\gamma}"]\ar[d] & 1 + \left(\prod_{x:X}x\neq \gamma(0)\right)\ar[r] & 2 \\
      1\ar[ru,dashed]
    \end{tikzcd}
  \end{center}
  We get the factorization by connectedness of $\A^1$.
  By \cite{etale-draft}[Proposition 4.2.10] any separated étale scheme has decidable equality.
\end{proof}

\begin{lemma}
  \label{A1-paths-to-A1-prop}
  Let $X$ be a type such that
  \[
  \prod_{x,y:X} \sum_{\gamma:\A^1\to X} (\gamma(0)=x)\times (\gamma(1)=y)
  \]
  then $\shape X$ is a proposition.
\end{lemma}

\begin{proof}
  First, using the inverse of the map
  \[
    (\shape X) \cong  \shape X^{\A^1}
  \]
  to construct $(x,y:X)\to \sigma_X(x)=\sigma_X(y)$.
  By direct applications of the dependent universal property of a uniquely eliminating modality we conclude
  $(x,y:\shape X)\to x=y$.
\end{proof}

\begin{example}
  \label{basic-A1-connected-types}
  \begin{enumerate}[(a)]
  \item Let $(X,*)$ be a pointed type with a multiplicative left action of $R$,
    such that for all $x:X$, we have $0\cdot x=*$ and $1\cdot x=x$.
    Then $\shape X$ is a proposition by \Cref{A1-paths-to-A1-prop} and therefore contractible,
    using the maps
    \[
    \gamma_x\colonequiv (r:\A^1)\mapsto r\cdot x : \A^1\to X
    \rlap{.}
    \]
    This entails that $\D(n)$, $\D$ and all types with an $R$-module structure are $\A^1$-connected.
  \item For any pair of types $A,B:\mathcal U$, we have the maps
    \begin{align*}
      f_{A,B}&\colonequiv (x:\A^1)\mapsto A^{x=0}\times B^{x=1} \\
      g_{A,B}&\colonequiv (x:\A^1)\mapsto A\times (x=0) + B\times(x=1)
    \end{align*}
    both constructions imply with \Cref{A1-paths-to-A1-prop}
    that $\shape \mathcal U$ is a proposition and therefore contractible.
  \item For $A,B:\Mod{R}$ we can use the maps $f_{A,B}$ from above to construct a path
    where all values carry an $R$-module structure.
    Therefore $\shape \Mod{R}$ is constractbile as well.
    The same argument applies to all types of structured types closed under product
    and exponentiation with propositional affine schemes, e.g.\ $\Mod{R}_{wqc}$.
  \end{enumerate}
\end{example}

\begin{definition}
  Let $\A^\times\colonequiv \A^1\setminus\{0\}$.
\end{definition}

To describe $\shape_{\A^1}\A^\times$,
we will need a construction which is called coreduction, deRham stack, infinitesimal shape or crystalline modality. We will use yet another name:

\begin{definition}
  For any type $X$, let $\widetilde{X}$ denote the formally étale
  \footnote{See \cite{etale-draft}[Section 5.1]}
  replacement of $X$.
\end{definition}

\begin{proposition}
  $\shape_{\A^1}\A^\times=\widetilde{\A^\times}$.
\end{proposition}

\begin{proof}
  By \Cref{R-action-A1-connected}, the fibers of $\A^\times\to \widetilde{\A^\times}$ are $\shape_{\A^1}$-connected,
  so it is enough to show that $\widetilde{\A^\times}$ is $\shape_{\A^1}$-modal.
  Use Zariski choice in the situation
  \begin{center}
    \begin{tikzcd}
      & \A^\times\ar[d,->>] \\
      \A^1\ar[r,"\gamma"] & \widetilde{\A^\times} 
    \end{tikzcd}
  \end{center}
  to get $s_i:D(f_i)\to{\A^\times}$ with $\neg\neg (s_i=s_j)$ on intersections.
  Fix $x,y:\A^1$. We will show that $\neg\neg(\gamma(x)=\gamma(y))$.
  We can assume $s_i=s_j$, so the $s_i$ glue to a map $\A^1\to\A^\times$, which is a lift of $\gamma$.
  This lift is merely of the form $x\mapsto a_0+\sum_{i=1}^na_ix^i$ with $a_0\neq 0$ and nilpotent $a_i$ for $i>0$,
  which means $\neg\neg (\gamma(x)=a_0=\gamma(y))$.
  
  So the original map $\A^1\to \A^\times$ is weakly constant and therefore constant.
\end{proof}
