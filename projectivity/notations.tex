We write
\begin{equation*}
  \A \coloneqq \Spec R[T]
\end{equation*}
for the one-dimensional affine line, which is a ring object in the
category of (affine) schemes.  By the universal property of the polynomial
ring $R[X]$, the map
\begin{equation*}
  \A \to R, p \mapsto p(T)
\end{equation*}
is an equivalence of types, even rings, so $R$
inherits the structure of an affine scheme from $\A$.  Nevertheless,
we use $\A$ in contexts of an (affine) scheme representing a space and
$R$ in contexts of an algebra representing functions.

We write
\begin{equation*}
  \Gm \coloneqq \Spec R[T, T^{-1}]
\end{equation*}
for the multiplicative group, which is a group object in the category of (affine) schemes.  As we have an equivalence
\begin{equation*}
  \Gm \to R^\times, p \mapsto p(T)
\end{equation*}
by the universal property of the localization $R[T, T^{-1}]$, an
analogous comment applies for $R^\times$ and $\Gm$ as for $R$ and $\A$
above.

%%% Local Variables:
%%% mode: latex
%%% TeX-master: "main"
%%% End:
