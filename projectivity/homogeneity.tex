\subsection{Graded $R$-algebras}

\begin{definition}
  A \emph{graded} $R$-algebra $S$ is an $R$-algebra $S$ together with the datum of a direct sum decomposition
  \begin{equation*}
    S = \bigoplus_{n \in \Z} S_n
  \end{equation*}
  as an $R$-module such that $S_k \cdot S_\ell \subseteq S_{k + l}$
  for every $k$, $\ell \in \Z$.  We identify each $S_n$ for $n \in \Z$
  with its image in $S$.  The elements of $S_n$ are called \emph{homogeneous} of degree $n$.
\end{definition}

\begin{remark}
  The datum of a grading of an $R$-algebra $S$ thus gives us an
  essentially finite decomposition\footnote{That means there merely
    are $i$, $j :\Z$ such that $u_n = 0$ if $n < i$ or $n > j$.}
  $u = \sum_{n \in \Z} u_n$ for every $u \in S$ where each $u_n$ is
  homogeneous of degree $n$.  Furthermore, the decomposition of a
  homogeneous element $u$ is just $u = u$.
\end{remark}

\begin{proposition}\label{spectrum of graded algebra}
  Let $S$ be a graded finitely presented $R$-algebra and let
  $X \coloneqq \Spec S$.  For each $a \in \Gm$ and each $p \in X$ we
  define
  \begin{equation*}
    p \cdot a \coloneqq \left(u \mapsto \sum_{n \in \Z} p(u_n) a(T)^n\right) \in \Spec S.
  \end{equation*}
  This defines an operation of the multiplicative group $\Gm$ on $X$.
\end{proposition}

\begin{proof}
  That $p \cdot 1 = p$ for every $p \in X$ follows from
  $p(u) = \sum_{n \in \Z} p(u_n)\cdot 1^n$.  That
  $(p \cdot a) \cdot a' = p \cdot (a \cdot a')$ follows from the
  homogeneity of each $u_n$.
\end{proof}

\begin{theorem}
  The construction in \cref{spectrum of graded algebra} yields an
  identification between the type of graded finitely presented
  $R$-algebras and the type of affine schemes together with an action
  of the multiplicative group $\Gm$.
\end{theorem}

\begin{proof}
  We give the converse construction.  Let $X = \Spec S$ be an affine
  scheme together with the datum $X \times \Gm \to X$ of an action of
  the multiplicative group $\Gm$.  By synthetic quasi-coherence, the
  function $X \times \Gm \to X$ yields a homomorphism
  \begin{equation*}
    \alpha(T)\colon S \to S[T, T^{-1}] = R[T, T^{-1}] \otimes S, u \mapsto \sum_{n \in Z} u_n T^n
  \end{equation*}
  of $R$-algebras where the sum on the right hand side is essentially
  finite.  That $p \cdot 1 = p$ for all $p \in X$ is equivalent to
  $\alpha(1) = \id_S$, which, in turn, yields
  $u = \sum_{n \in \Z} u_n$.  That $(p \cdot a) \cdot a' = p \cdot (a \cdot a')$ is equivalent to
  $(\alpha(T) \otimes \id_{R[T, T^{-1}]}) \circ \alpha(T') = \alpha(T \cdot T')$, from which $\alpha(T) (u_n) = u_n T^n$ follows.
\end{proof}

\begin{remark}
  Let $S$ be a graded $R$-algebra.  The action of $\Gm$ on $\Spec S$ induces a natural action of $R^\times$ on the $R$-algebra of functions
  on $\Spec S$ given by
  \begin{equation*}
    R^\times \times R^{\Spec S} \to R^{\Spec S}, (f, t) \mapsto (p \mapsto f(p \cdot a))
  \end{equation*}
  where $a \in \Gm$ such that $a(T) = t$.
  Under the identification $R^{\Spec S} = S$ given by synthetic quasicoherence, this gives the action
  \begin{equation*}
    R^\times \times S \to S, (u, t) \mapsto \sum_{n \in \Z} u_n t^n
  \end{equation*}
  of $R^\times$ on the $R$-algebra $S$.
\end{remark}

\begin{remark}
  A finitely generated graded $R$-algebra $S$ is finitely generated by
  homogeneous elements: take the homogeneous components of each
  generator as new generators.  Thus, there is no difference between
  finitely generated graded $R$-algebras and \emph{homogeneously}
  finitely generated graded $R$-algebras.
\end{remark}

\begin{definition}
  A finitely presented graded $R$-algebra is \emph{homogeneously
    finitely presented} if its generators and relations can be chosen
  to be homogenous.
\end{definition}

From now on, finitely presented graded $R$-algebras a assumed to be
homogeneously finitely presented unless said otherwise.

\subsection{Vector spaces and their duals}

\begin{example}
  Let $M$ be a finitely presented $R$-module.  The symmetric algebra
  $\Sym^* M$ of $M$ over $R$ is naturally graded.  In particular, the
  affine scheme
  \begin{equation*}
    V \coloneqq M\spcheck \coloneqq \Spec \Sym^* M
  \end{equation*}
  carries a natural action by the multiplicative group $\Gm$.
\end{example}

\begin{remark}
  We write $M\spcheck$ as
  \begin{equation*}
    \Spec \Sym^* M = \Hom_{\Alg R}(\Sym^* M, R) = \Hom_{\Mod R}(M, R)
  \end{equation*}
  by the universal property of $\Sym^* M$.  In particular,
  $V = M\spcheck$, being the module dual of $M$, carries the structure
  of a (finitely copresented) $\A$-module.  Moreover, the natural
  action of $\Gm$ on the left hand side corresponds to scalar
  multiplication on the right hand side.
\end{remark}

\begin{definition}
  A \emph{vector space} is a finitely copresented $\A$-module.
\end{definition}

\begin{example}
  The space $\A$ is a vector space.
\end{example}

From the above, we can deduce that $M$ is reflexive in the following
sense:

\begin{theorem}
  Let $M$ be a finitely presented $R$-module.  Set
  \begin{equation*}
    M^{\vee\vee} \coloneqq \Hom_{\Mod A}(M\spcheck, \A).
  \end{equation*}
  The natural map
  \begin{equation*}
    M \to M^{\vee\vee}, f \mapsto (p \mapsto p(f))
  \end{equation*}
  is an isomorphism of $R$-modules.  In particular, $M$ is reflexive
  and the dual of every vector space, that is every finitely
  copresented $R$-module (which is always the dual of a finitely
  presented $R$-module), is finitely presented.
\end{theorem}

\begin{proof}
  The identification
  \begin{equation*}
    \Sym^* M \to R^{\Spec \Sym^* M} = R^{M\spcheck}
  \end{equation*}
  is an equivalence of $R$-algebras with an $R^\times$-action.  In
  particular, the homogeneous elements of degree $1$ on the left hand
  side correspond to the homogeneous elements of degree $1$ on the
  right hand side.  This yields an isomorphism
  \begin{equation*}
    \phi\colon M \to \Hom_{\Gm}(M\spcheck, \A) \coloneqq \{u\colon M\spcheck \to \A \mid \forall a \in \Gm\forall v\colon M. u(v a) = u(v) a\},
    v \mapsto (p \mapsto p(v))
  \end{equation*}
  of $R$-modules.  As the image of $\phi$ lies inside
  $M^{\vee\vee} \subseteq \Hom_{\Gm}(M\spcheck, \A)$ we actually have
  $M^{\vee\vee} = \Hom_{\Gm}(M\spcheck, \A)$ and the theorem is
  proven.
\end{proof}

\begin{remark}
  It follows from the above that for a vector space $V$, a
  ($1$-)homogenous map $V \to \A$ is already $\A$-linear (principle of
  microlinearity).
\end{remark}

%%% Local Variables:
%%% mode: latex
%%% TeX-master: "main"
%%% End:
