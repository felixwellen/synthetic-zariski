\subsection{The invariant ideal}

\begin{definition}
  A group (set) $G$ is a \emph{group of roots of unity} if there
  merely exists a $d : \N$ such $g^d = 1$ for all $g : G$.
\end{definition}

\begin{example}
  Let $d\colon \N$.  The group
  $\bmu_d \coloneqq \{t  : \Gm \mid t^d = 1\}$ of \emph{$d$-th
    roots of unity} is a group of roots of unity (and a group scheme).
\end{example}

\begin{definition}
  Let $X$ be a set and $G$ be a group (set).  An operation
  $X \times G \to X$ \emph{has isotropy groups of roots of unity} if
  and only if for all $p \in X$ the subgroup
  \begin{equation*}
    G_p \coloneqq \{g : G \mid x \cdot g = x\}
  \end{equation*}
  is a group of roots of unity.
\end{definition}

\begin{definition}\label{irrelevant ideal}
  Let $X \coloneqq \Spec S$ be an affine scheme with an action of the
  multiplicative group $\Gm$, that is $S$ is a (homogeneously)
  finitely presented graded $R$-algebra.  The ideal generated by
  $\bigoplus_{d \neq 0} S_d$ is the \emph{irrelevant ideal} $S^+$ of
  $S$.
\end{definition}

\begin{proposition}\label{invariant subscheme}
  Let $X \coloneqq \Spec S$ be as in the definition \cref{irrelevant ideal}.  Then
  \begin{equation*}
    X^{\Gm} \coloneqq \Spec S/S^+
  \end{equation*}
  is the largest closed subscheme of $X$ on which $\Gm$ acts trivially.
\end{proposition}

\begin{proof}
  As $S^+$ is generated by homogeneous elements, the $R$-algebra
  $S/S^+$ inherits its grading from $S$.  This grading is such that
  every element of $S/S^+$ is homogenous of degree zero, meaning that
  $\Gm$ acts trivially on $X^{\Gm} = \Spec S/S^+$.

  Let $Y = \Spec S/I \subseteq X$ any closed subscheme of $X$ on which
  $\Gm$ acts trivially.  In particular, $\Gm$ acts trivially on each
  $R$-valued function on $Y$, meaning that $\Gm$ acts trivially on
  $\Spec S/I$.  From this we deduce that every homogeneous element of
  $S$ of non-zero degree has to vanish modulo $I$, meaning that $S^+
  \subseteq I$.  This proves $Y \subset X^{\Gm}$.
\end{proof}

\begin{proposition}
  In the situation of \cref{irrelevant ideal}, the set
  \begin{equation*}
    X^\circ \coloneqq X \setminus X^{\Gm}
  \end{equation*}
  is a $\Gm$-invariant open subscheme of $X$ where the action of $\Gm$ has
  isotropy groups of roots of unity.
\end{proposition}

\begin{proof}
  Let $f_1$, \dots, $f_m$ be homogeneous generators of the irrelevant
  ideal $S^+$.  By homogeneity, each standard-open
  $D(f_i) \subseteq X$ is $\Gm$-invariant.  As
  $X^\circ = \bigcup_{i = 1}^m D(f_i)$, it follows that $X^\circ$ is a
  $\Gm$-invariant open subscheme of $X$.

  Let $p : X$ with $p \in X^\circ$.  We have to show that the isotropy group of $p$
  is a group of roots of unity: There exists an $f : S^+$ such that
  $f(p) \neq 0$.  We may assume that $f$ is homogeneous, say of degree
  $d$.  For each $g : G_p$, we have
  $f(p) = f(p \cdot g) = f(p) \cdot g^d$.  By invertibility of $f(p)$
  it follows that $g^d = 1$.
\end{proof}

\subsection{Invariant morphisms}

\begin{definition}
  Let $X$ be a set with an action $X \times G \to X$ of a group $G$.  Let $Y$ be another
  set.  A \emph{$G$-invariant map} (or just an \emph{invariant map})
  $\phi : X \to Y$ is a map such that
  \begin{equation*}
    \forall x: X \forall g: G. \phi(x \cdot g) = \phi(x).
  \end{equation*}
  The set of invariant maps $X \to Y$ is denoted by $\Hom(X, Y)^G$.
\end{definition}

\begin{definition}
  Let $X$ be a scheme with an action $X \times G \to X$ of a group scheme $G$.
  A \emph{(categorical) quotient} of $X$ by $G$ is a scheme $Y$
  together with a $G$-invariant map $\phi : X \to Y$ such that
  every invariant map $X \to Z$ into a scheme $Z$ factors uniquely
  over $\phi$.
\end{definition}

\subsection{The Proj-construction}

\begin{theorem}\label{proj construction}
  Let $X = \Spec S$ be an affine scheme together with an action of the
  multiplicative group $\Gm$, that is, $S$ is a (homogeneously)
  finitely presented graded $R$-algebra.  Then the categorical
  quotient
  \begin{equation*}
    \Proj S \coloneqq X^\circ/\Gm
  \end{equation*}
  exists.
\end{theorem}

We will make use of the following two lemmata to proof the theorem.

\begin{lemma}\label{proj construction lemma 1}
  Let $S$ be a homogeneously finitely presented graded ring such that
  there exists a unit $f\colon S^\times$ with $f \in S_d$ for some $d \neq
  0$.  Then, the map $I \mapsto \sqrt{SI}$ mapping each ideal of $S_0$
  to the radical ideal of $S$ generated by $I$ is a bijection from
  the finitely generated radical ideals of $S_0$ onto the homogeneously finitely
  generated radical ideals of $S$.
\end{lemma}

\begin{proof}[of \cref{proj construction lemma 1}]
  Let $I$ be a finitely generated radical ideal of $S_0$, generated by
  $f_1$, \dots, $f_m$.  Then $f_1$, \dots, $f_m$ are homogeneous generators of
  $\sqrt{SI}$ as a radical ideal of $S$.  Thus, for each finitely generated
  radical ideal $I$, its image $\sqrt{SI}$ is a homogeneously finitely
  generated radical ideal of $S$.  This proves well-definedness.

  Let $J$ be such an ideal of $S$ with
  generators $g_1$, \dots, $g_m$.  As these are generators of a
  radical ideal, we may replace each $g_i$ by its $d$-th power so that
  each generator is homogeneous of a degree that is divisible by $d$.
  It follows that there are exponents $n_1$, \dots, $n_m$ such that
  $g_1/g^{n_1}$, \dots, $g_m/g^{n_m}$ are homogeneous of degree zero,
  that is they lie in $S$.  They are still generators of $J$.  This
  proves surjectivity.

  Let $I$ and $I'$ be two finitely generated radical ideals of $S_0$
  with $\sqrt{SI} = \sqrt{SI'}$.  Let $f\colon S_0$ with $f \in I$.  Then $f \in \sqrt{SI}
  \cap S_0 = \sqrt{SI'} \cap S_0$.  In particular, $f^n \in SI'$ for
  some natural number $n$.  As $f^n \in S_0$, it follows that $f^n \in
  I'$.  As $I'$ is a radical ideal, we thus have $f \in I'$.  As $f$
  was arbitrary, this proves $I \subseteq I'$.  Analogously, $I'
  \subseteq I$.  This proves injectivity.
\end{proof}

\begin{lemma}\label{proj construction lemma 2}
  In the situation of \cref{proj construction, lemma 1}, the map
  $\pi\colon \Spec S \to \Spec S_0$ induced by the inclusion $S_0 \to
  S$ is a categorical quotient $X \to X/\Gm$.
\end{lemma}

\begin{remark}
  In the situation of the lemma, the irrelevant ideal is the unit
  ideal, thus $X = X^\circ$.  With the notation introduced in
  \cref{proj construction}, the statement of \cref{proj construction,
    lemma 2} can thus be phrased as
  \begin{equation*}
    \Proj S = \Spec S_0.
  \end{equation*}
\end{remark}

\begin{proof}[of \cref{proj construction lemma 2}]
  Let $\phi\colon \Spec S \to Z$ be an invariant morphism to a scheme $Z$.
  We have to show that $\phi$ factors uniquely over $\pi$.

  Let us first consider the case that $Z$ is affine, say $Z = \Spec C$
  for a finitely presented $R$-algebra $C$.  The morphism $\phi$ is
  induced by a homomorphism $C \to S$.  As $\phi$ is invariant, this
  morphism has to factor (uniquely) over $S_0 \subseteq S$.  Thus
  $\phi$ factors (uniquely) over $\pi$.

  \dots
\end{proof}

%%% Local Variables:
%%% mode: latex
%%% TeX-master: "main"
%%% End:
