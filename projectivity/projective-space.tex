\subsection{The invariant ideal}

\begin{definition}
  A group (set) $G$ is a \emph{group of roots of unity} if there
  merely exists a $d : \N$ such $g^d = 1$ for all $g : G$.
\end{definition}

\begin{example}
  Let $d\colon \N$.  The group
  $\bmu_d \coloneqq \{t  : \Gm \mid t^d = 1\}$ of \emph{$d$-th
    roots of unity} is a group of roots of unity (and a group scheme).
\end{example}

\begin{definition}
  Let $X$ be a set and $G$ be a group (set).  An operation
  $X \times G \to X$ \emph{has isotropy groups of roots of unity} if
  and only if for all $p \in X$ the subgroup
  \begin{equation*}
    G_p \coloneqq \{g : G \mid x \cdot g = x\}
  \end{equation*}
  is a group of roots of unity.
\end{definition}

\begin{definition}\label{irrelevant ideal}
  Let $X \coloneqq \Spec S$ be an affine scheme with an action of the
  multiplicative group $\Gm$, that is $S$ is a (homogeneously)
  finitely presented graded $R$-algebra.  The ideal generated by
  $\bigoplus_{d \neq 0} S_d$ is the \emph{irrelevant ideal} $S^+$ of
  $S$.
\end{definition}

\begin{proposition}\label{invariant subscheme}
  Let $X \coloneqq \Spec S$ be as in the definition \cref{irrelevant ideal}.  Then
  \begin{equation*}
    X^{\Gm} \coloneqq \Spec S/S^+
  \end{equation*}
  is the largest closed subscheme of $X$ on which $\Gm$ acts trivially.
\end{proposition}

\begin{proof}
  As $S^+$ is generated by homogeneous elements, the $R$-algebra
  $S/S^+$ inherits its grading from $S$.  This grading is such that
  every element of $S/S^+$ is homogenous of degree zero, meaning that
  $\Gm$ acts trivially on $X^{\Gm} = \Spec S/S^+$.

  Let $Y = \Spec S/I \subseteq X$ any closed subscheme of $X$ on which
  $\Gm$ acts trivially.  In particular, $\Gm$ acts trivially on each
  $R$-valued function on $Y$, meaning that $\Gm$ acts trivially on
  $\Spec S/I$.  From this we deduce that every homogeneous element of
  $S$ of non-zero degree has to vanish modulo $I$, meaning that $S^+
  \subseteq I$.  This proves $Y \subset X^{\Gm}$.
\end{proof}

\begin{definition}
  In the situation of \cref{irrelevant ideal}, a \emph{$\Gm$-invariant
    open subscheme} or just \emph{invariant open subscheme} of $X$ is
  an open subscheme of the affine scheme $\Spec S$ that is defined by
  a homogeneously finitely generated ideal of $S$.
\end{definition}

\begin{remark}
  If $f$ is homogeneous, each standard-open $D(f) \subseteq X$ is
  indeed $\Gm$-invariant in the sense that the action $X \times \Gm
  \to X$ restricts to an action $D(f) \times \Gm \to D(f)$.

  It follows that the action $X \times \Gm \to X$ restricts to all
  invariant open subschemes $U$.
\end{remark}

\begin{lemma}\label{finite invariant union}
  In the situation of \cref{irrelevant ideal}, a finite union of
  invariant open subschemes is again invariant open.
\end{lemma}

\begin{proof}
  This is clear by definition.
\end{proof}

\begin{lemma}\label{open subset of invariant open}
  In the situation of \cref{irrelevant ideal}, let $U$ be an invariant
  open subscheme and $f\colon U \to R$ a $\Gm$-invariant map that is,
  a map with
  \begin{equation*}
    \forall p: X \forall t: \Gm f(p \cdot t) = f(p).
  \end{equation*}
  Then $U \cap D(f)$ is again an invariant open subscheme.
\end{lemma}

\begin{proof}
  By precondition, $f$ is homogeneous of degree zero. Let $U =
  \bigcup_{i = 1}^n D(f_i)$ where each $f_i\colon U \to R$ is
  homogeneous (of some degree). Then
  \begin{equation*}
    U \cap D(f) = \bigcup_{i = 1}^n D(f f_i),
  \end{equation*}
  and each $f f_i$ is homogeneous (of some degree).
\end{proof}

\begin{proposition}
  In the situation of \cref{irrelevant ideal}, the set
  \begin{equation*}
    X^\circ \coloneqq X \setminus X^{\Gm}
  \end{equation*}
  is a $\Gm$-invariant open subscheme of $X$ where the action of $\Gm$ has
  isotropy groups of roots of unity.
\end{proposition}

\begin{proof}
  Let $f_1$, \dots, $f_m$ be homogeneous generators of the irrelevant
  ideal $S^+$.  As
  $X^\circ = \bigcup_{i = 1}^m D(f_i)$, it follows that $X^\circ$ is a
  $\Gm$-invariant open subscheme of $X$.

  Let $p : X$ with $p \in X^\circ$.  We have to show that the isotropy group of $p$
  is a group of roots of unity: There exists an $f : S^+$ such that
  $f(p) \neq 0$.  We may assume that $f$ is homogeneous, say of degree
  $d$.  For each $g : G_p$, we have
  $f(p) = f(p \cdot g) = f(p) \cdot g^d$.  By invertibility of $f(p)$
  it follows that $g^d = 1$.
\end{proof}

\subsection{Invariant maps}

\begin{definition}\label{invariant map}
  Let $X \coloneqq \Spec S$ be an affine scheme with an action of the
  multiplicative group $\Gm$.  A map $\phi\colon U \to Y$ from an
  invariant open subscheme $U$ of $X$ to a scheme $Y$ is a
  \emph{$\Gm$-invariant} or just \emph{invariant} map if
  \begin{equation*}
    \forall p : U \forall g : \Gm. \phi(pg) = \phi(p)
  \end{equation*}
  and if there merely exists
  an affine open cover $(V_j)_{j \in J}$ of $Y$ such that $\phi^{-1}(V_j)$
  is an invariant open subscheme of $U$ for each $j \in J$.
\end{definition}

\begin{proposition}\label{inverse invariant image}
  In the situation of \cref{invariant map}, let $V \subseteq Y$ be any
  open subscheme of $Y$.  Then $\phi^{-1}(V)$ is an invariant open
  subscheme of $U$.
\end{proposition}

\begin{proof}
  Assume that the proposition is true for open subschemes $V$ and $V'$
  of $Y$. Then it is also true for $V \cup V'$ because
  \begin{equation*}
    \phi^{-1}(V \cup V') = \phi^{-1}(V) \cup \phi^{-1}(V')
  \end{equation*}
  and because a (finite) union of open invariant subschemes is an open
  invariant subscheme by \cref{finite invariant union}.

  Assume that the proposition is true for an affine open subscheme $V$
  and let $f\colon V \to R$ be a map.  Then it is also true for $V
  \cap D(f)$ because
  \begin{equation*}
    \phi^{-1}(V \cap D(f)) = \phi^{-1}(V) \cap D(f \circ \phi)
  \end{equation*}
  and because $f \circ \phi\colon U \to R$ is $\Gm$-invariant; see
  \cref{open subset of invariant open}.

  This is enough to deduce the statement of the proposition to be
  proven from the validity for some affine open cover of $X$.
\end{proof}

\begin{definition}
  Let $X \coloneqq \Spec S$ be an affine scheme with an action of the
  multiplicative group $\Gm$.  Let $U$ be an invariant open subscheme.
  A \emph{(categorical) quotient} of $U$ by $\Gm$ is a scheme $Y$
  together with an invariant map $\phi : X \to Y$ such that
  every invariant map $X \to Z$ into a scheme $Z$ factors uniquely
  over $\phi$.
\end{definition}

\subsection{The Proj-construction}

\begin{theorem}\label{proj construction}
  Let $X = \Spec S$ be an affine scheme together with an action of the
  multiplicative group $\Gm$, that is, $S$ is a (homogeneously)
  finitely presented graded $R$-algebra.  Then the categorical
  quotient
  \begin{equation*}
    \Proj S \coloneqq X^\circ/\Gm
  \end{equation*}
  exists.
\end{theorem}

We will make use of the following two lemmata to proof the theorem.

\begin{lemma}\label{proj construction lemma 1}
  Let $S$ be a homogeneously finitely presented graded ring such that
  there exists a unit $f\colon S^\times$ with $f \in S_d$ for some $d \neq
  0$.  Then, the map $I \mapsto \sqrt{SI}$ mapping each ideal of $S_0$
  to the radical ideal of $S$ generated by $I$ is a bijection from
  the finitely generated radical ideals of $S_0$ onto the homogeneously finitely
  generated radical ideals of $S$.
\end{lemma}

\begin{proof}[of \cref{proj construction lemma 1}]
  Let $I$ be a finitely generated radical ideal of $S_0$, generated by
  $f_1$, \dots, $f_m$.  Then $f_1$, \dots, $f_m$ are homogeneous generators of
  $\sqrt{SI}$ as a radical ideal of $S$.  Thus, for each finitely generated
  radical ideal $I$, its image $\sqrt{SI}$ is a homogeneously finitely
  generated radical ideal of $S$.  This proves well-definedness.

  Let $J$ be such an ideal of $S$ with
  generators $g_1$, \dots, $g_m$.  As these are generators of a
  radical ideal, we may replace each $g_i$ by its $d$-th power so that
  each generator is homogeneous of a degree that is divisible by $d$.
  It follows that there are exponents $n_1$, \dots, $n_m$ such that
  $g_1/g^{n_1}$, \dots, $g_m/g^{n_m}$ are homogeneous of degree zero,
  that is they lie in $S$.  They are still generators of $J$.  This
  proves surjectivity.

  Let $I$ and $I'$ be two finitely generated radical ideals of $S_0$
  with $\sqrt{SI} = \sqrt{SI'}$.  Let $f\colon S_0$ with $f \in I$.  Then $f \in \sqrt{SI}
  \cap S_0 = \sqrt{SI'} \cap S_0$.  In particular, $f^n \in SI'$ for
  some natural number $n$.  As $f^n \in S_0$, it follows that $f^n \in
  I'$.  As $I'$ is a radical ideal, we thus have $f \in I'$.  As $f$
  was arbitrary, this proves $I \subseteq I'$.  Analogously, $I'
  \subseteq I$.  This proves injectivity.
\end{proof}

\begin{lemma}\label{proj construction lemma 2}
  In the situation of \cref{proj construction lemma 1}, the map
  $\pi\colon \Spec S \to \Spec S_0$ induced by the inclusion $S_0 \to
  S$ is a categorical quotient $X \to X/\Gm$.
\end{lemma}

\begin{remark}
  In the situation of the lemma, the irrelevant ideal is the unit
  ideal, thus $X = X^\circ$.  With the notation introduced in
  \cref{proj construction}, the statement of \cref{proj construction lemma 2} can thus be phrased as
  \begin{equation*}
    \Proj S = \Spec S_0.
  \end{equation*}
\end{remark}

\begin{proof}[of \cref{proj construction lemma 2}]
  Let $\phi\colon \Spec S \to Z$ be an invariant morphism to a scheme $Z$.
  We have to show that $\phi$ factors uniquely over $\pi$.

  Let us first consider the case that $Z$ is affine, say $Z = \Spec C$
  for a finitely presented $R$-algebra $C$.  The morphism $\phi$ is
  induced by a homomorphism $C \to S$.  As $\phi$ is invariant, this
  morphism has to factor (uniquely) over $S_0 \subseteq S$.  Thus
  $\phi$ factors (uniquely) over $\pi$.

  For general $Z$ consider an affine open cover $(W_i)_{i \in I}$.
  By \cref{inverse invariant image}, $\phi^{-1}(W_i)$ is
  an invariant open subset.  By \cref{proj construction lemma 1}, we
  can therefore assume that there are standard-open subsets $V_i$ of
  $\Spec S_0$ such that the $U_i \coloneqq \pi^{-1}(V_i)$ form a
  $(U_i)_{i \in I}$ of $X$ by invariant open subschemes of $U$ with
  $\phi(U_i) \subseteq W_i$ for each $i \in I$.

  By the affine case already covered (i.e.\ for $Z = W_i$), it follows
  that the canonical map $\pi_i\colon U_i \to V_i$ is the
  $\Gm$-quotient of the restriction $\phi_i\colon U_i \to W_i$ for all
  $i \in I$. In particular, there are unique morphisms $\psi_i\colon
  V_i \to Z$ with $\phi_i \circ \pi|_{U_i} = \phi_i$.  By (local) uniqueness
  and because the $V_i$ cover $\Spec S_0$, the maps $\psi_i$ glue to a
  morphism $\psi\colon \Spec S_0 \to Z$ with $\psi \circ \pi = \phi$.

  It remains to show that $\psi$ is unique with this property. This
  follows again from a local consideration by restricting to an affine
  cover in the target and \cref{proj construction lemma 2}.
\end{proof}

\begin{proof}[of \cref{proj construction}]
  We first construct $\Proj S$ as a scheme together with a map
  $\pi\colon X^\circ \to \Proj S$ and then show that it is a
  categorical quotient.

  Let $g_1$, \dots, $g_n$ be homogeneous generators of the irrelevant
  ideal $S^+$ of $S$.  Then $D(g_1)$, \dots, $D(g_n)$ cover
  $X^\circ$.  As the $g_i$ are homogeneous, the localizations
  $S[g_i^{-1}]$ inherit gradings.  Set $Y_i \coloneqq \Spec
  S[g_i^{-1}]_0$ and $Y_{i, j} \coloneqq \Spec[(g_i g_j)^{-1}]_0$.  We
  can view $Y_{i,j}$ as a standard-open affine subspace of $Y_i$.
  Moreover, there is a canonical identification $Y_{i,j} \sim
  Y_{j,i}$. Thus, we can glue the $Y_i$ to a scheme $Y = \Proj S$.

  The inclusion $S[g_i^{-1}]_0 \to S[g_i^{-1}$ induces a map
  $\pi_i\colon D(g_i) \to Y_i$.  By \cref{proj construction lemma 2},
  this map turns $Y_i$ into the categorical quotient $D(g_i)/\Gm$.
  The maps $\pi_i$ glue to a morphism $\pi\colon X^\circ \to Y$.

  It remains to show that $\pi$ fulfills the universal property of the
  categorical quotient.  For this, let $\phi\colon X^\circ \to Z$ be
  an invariant map into some scheme $Z$.  We have to show that a
  unique map $\psi\colon Y \to Z$ with $\phi = \psi \circ \pi$
  exists.  As $Y_i = D(g_i)/\Gm$, there exist unique maps
  $\psi_i\colon Y_i \to Z$ with $\phi|_{D(g_i)} = \psi_i \circ
  \pi_i$.  Because of (local) uniqueness, the maps $\phi_i$ glue to a
  map $\psi\colon Y \to Z$ with $\phi = \psi \circ \pi$.  The
  uniqueness of $\psi$ follows again by local uniqueness.
\end{proof}

%%% Local Variables:
%%% mode: latex
%%% TeX-master: "main"
%%% End:
