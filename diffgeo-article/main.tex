% latexmk -pdf -pvc main.tex
% latexmk -pdf -pvc -interaction=nonstopmode main.tex
\documentclass{../util/zariski}

\title{Differential Geometry of Synthetic Schemes}
\author{Felix Cherubini$^1$, Matthias Ritter$^1$, Hugo Moeneclaey$^1$ and David Wärn$^1$}
\date{$^1$ University of Gothenburg and Chalmers University of Technology }

\begin{document}

\maketitle

\begin{abstract}
  Synthetic algebraic geometry is a new approach to algebraic geometry. It consists in using homotopy type theory extended with three axioms, together with the interpretation of these in a higher version of the Zariski topos, in order to do algebraic geometry internally to this topos. In this article we make no essential use of the higher structure of types, so that we could alternatively use the traditional Zariski 1-topos.
  
  We give new synthetic definitions of étale, smooth and unramified maps between schemes.
  We prove the usual characterizations of these classes of maps in terms of injectivity, surjectivity and bijectivity of differentials. We also show that the tangent spaces of smooth schemes are finite free modules.
  Finally, we show that unramified, étale and smooth schemes can be understood very concretely, indeed they admit the expected local algebraic description.
\end{abstract}

\tableofcontents

\section*{Introduction}

In mathematics, it is common practice to assume a fixed set theory, usually with the axiom of choice, as a common basis. While it is a great advantage to work in one common language and share a lot of the basic constructions, the dual approach of adapting the  ``base language'' to particular mathematical domains is sometimes more concise, provides a new perspective and encourages new proof techniques which would be hard to find otherwise.
We use the word ``synthetic'' to indicate that the latter approach is used,
as it was used by Kock and Lawvere to describe developement of mathematics internal to certain categories \cite{lawvere-categorical-dynamics}, in particular toposes -- a program which dates back as far as 1967.

Already in the 70s in the same program, Anders Kock suggested to use the language of higher-order logic \cite{Church40} to describe the Zariski topos, the collection of sheaves for the Zariski topology \cite{Kock74,kockreyes}, which is the first occurence of synthetic algebraic geometry.
Kock's approach allowed for a more suggestive and geometrical description of schemes.
There is in particular a ``generic local ring'' $R$, which, as a sheaf, associates to any algebra $A$ its underlying set and, as described in \cite{kockreyes}, the projective space $\bP^n$ is then the set of lines in $R^{n+1}$.

Just using category theory is not the same as reasoning synthetically -- for the latter the goal is usually to derive results exclusively in one system,
as Kock and Lawvere did with differential geometry in his work.
The distinction with just using an abstraction like categories is important, since the translation from the synthetic language and back can become cumbersome -- although it is still the goal to derive statements about ordinary mathematical objects in the end.

Starting with Kock and Lawvere's work, more differential geometry was developed synthetically \cite{kock-sdg} along with a study of the models of the theory \cite{moerdijk-reyes}.
One basic axiom of the theory, called the Kock-Lawvere axiom, allows for reasoning with nilpotent infinitesimals. Our version of synthetic algebraic geometry use a generalisation of this axiom called the duality axiom. Let us now describe the Kock-Lawvere axiom.

The Kock-Lawvere axiom is added to a basic language which can be interpreted in good enough categories (for example toposes), more precisely we need basic objects like $\emptyset$, $\{\ast\}$ and $\N$ as well natural constructions like $A\times B$ or $A^B$ for objects $A$, $B$, which all behave as expected. We also need predicates $P(x)$ for elements $x:A$ so we can form subobjects like $\{x:A\mid P(x)\}$.
In this language, we assume there is a fixed ring $R$, which can be thought of as the real numbers. We define $\D(1)=\{x\in R\mid x^2=0\}$ to be the set of all square-zero elements of $R$, then the Kock-Lawvere axiom gives us a bijection
\[ e : R \times R  \to R^{\D(1)} \]
which commutes with evaluation at $0$ and projection to the first factor.
The intuition is that $\D(1)$ is so small that any function on it is linear and therefore determined by its value and its derivative at $0\in\D(1)$.
With this axiom, the derivative at $0:R$ of a function $f : R \to R$ may then be defined as $\pi_2(e^{-1}(f_{\vert \D(1)}))$. This is the start of a convenient development of differential calculus, which doesn't require any further structures on $R$ or other objects. This is the core of the synthetic method: we can work with these differential spaces as if they were sets.

To give an example, the tangent bundle of a manifold $M$ can be defined as $M^{\D(1)}$ and vector fields as sections of the map $M^{\D(1)}\to M$ evaluating at $0$. Then it is easy to see that a vector field is the same as a map $\zeta:\D(1)\to M^M$ with $\zeta(0)=\id_M$, which can be interpreted as an infinitesimal transformation of the identity map. This style of reasoning with spaces as if they were sets is also central in current synthetic algebraic geometry. 

The Kock-Lawvere axiom above and many of the axioms used in synthetic reasoning are incompatible with the law of excluded middle (LEM) and therefore also with the axiom of choice (AC). Indeed they tend to imply all maps are well-behaved (for example all maps are differentiable in the case of the Kock-Lawvere axiom), which contradicts LEM. It is however a recurring phenomenon that restricted versions of LEM and AC are compatible with synthetic languages. A very basic example is that equality of natural numbers is decidable, meaning that two natural numbers are either equal or not equal. 

%TODO: Hugo is here in the rereading

The use of nilpotent elements to capture infinitesimal quantities as mentioned above was inspired by the Grothendieck school of algebraic geometry and Anders Kock also worked with an extended axiom \cite{Kock74,kockreyes} suitable for synthetic algebraic geometry, where the role of $\D(1)$ above can be taken by any finitely presented affine scheme. In 2017 Ingo Blechschmidt finished his doctoral thesis in which he noticed a property holding internally in the Zariski-topos, which he called synthetic quasi-coherence -- this was a more general and internal version of what Kock used. In 2018, David Jaz Myers\footnote{Myers' never published on the subject, but communicated his ideas to Felix Cherubini and in talks to a larger audience \cite{myers-talk1,myers-talk2}.} started to work with a specialization of Blechschmidt's synthetic quasi-coherence and used homotopy type theory as a base language, which is the standard in synthetic algebraic geometry now and we will highlight some implications below. Myers' specialized axiom is what we now call \emph{duality axiom}.

To state the duality axiom we need the general concept behind the space $\D(1)$, which is spaces that are the common zeros of some system of polynomial equations over $R$. Such a system can be encoded representation independent by a finitely presented $R$-algebra, i.e.\ an $R$-algebra $A$ which is of the form $R[X_1,\dots,X_n]/(P_1,\dots,P_l)$ for some numbers $n,l$ and polynomials $P_i\in R[X_1,\dots,X_n]$.
Then the zero set of the system is given by the type $\Hom_{\Alg{R}}(A,R)$ of $R$-algebra homomorphisms from $A$ to the base ring, which we denote by $\Spec A$.
Now the duality axiom states that $\Spec$ is the inverse to exponentiating with $R$, i.e.\ for all 
finitely presented $R$-algebras $A$ the following is an isomorphism:
\[ (a\mapsto (\varphi\mapsto \varphi(a))) : A\to R^{\Spec A}\rlap{.}\]

Using homotopy type theory as a language for synthetic algebraic geometry is, in addition to convenience, also a language for synthetic homotopy theory.
So instead of the usual practice in algebraic topology to provide model spaces using point-set topology, one can start directly at the level of homotopy types and instead of implementing their higher structure with Kan complexes, there are rules which do not mention any implementation.
The rules of homotopy type theory allow to work with the basic objects of the theory, types, in very much the same way as one would work with sets in traditional mathematics -- with the clear exception of the law of excluded middle and the axiom of choice - although the former and restricted versions of the latter can be assumed.
Both can be seen as stating something about the spatial structure. The law of excluded middle allows us to find a complement of each subset of a given set A, which exposes A as a coproduct.
This is not true in topology, for example, $\R$ is not the coproduct of the topological subspaces $\{0\}$ and $\R/\{0\}$.
The axiom of choice states that any surjection has a section. This is also not true in topology and would trivialize all cohomology.
Thus, constructive reasoning in the sense of not using these two axioms is a necessity if we want to use spatial collections in the same way we use sets.
In synthetic algebraic geometry, we work inside homotopy type theory and remind readers of this by using the notation $x:X$ which can often be thought of as $x\in X$.
\cite{shulman-logic-of-spaces} is a more detailed introduction to homotopy type theory for a general mathematical audience.


One of the main advantages of using specifically homotopy type theory and not a different internal language,
is that it is possible to make cohomological computations, using homotopy type theory for synthetic homotopical reasoning.
This means that we are mixing two synthetic approaches, combining their advantages,
which rests on the possibility of interpreting homotopy type theory in higher toposes \cite{shulman2019all} and not just the higher topos of $\infty$-groupoids.
The general idea of using homotopy type theory to combine some kind of synthetic, spatial reasoning with synthetic homotopy theory, goes back at least to 2014, to Mike Shulman and Urs Schreiber \cite{Schreiber_2014}.
Schreiber suggested to the HoTT community at various occasions to make use of HoTT as the internal language of higher toposes, where specifities of the topos are accessed in the language via modalities.
This approach was shown to be quite effective and intuitive in Shulman's \cite{shulman-Brouwer-fixed-point} work on mixing synthetic homotopy theory in the form of HoTT and a synthetic approach to topology using a triple of modalities -- a structure called cohesion by Lawvere \cite{Lawvere2007}.

One of Schreiber's motivation was to make use of the modern perspective on cohomology, which in a higher topos can be realized as the connected components of a space of maps. This can be mimicked in HoTT, like follows: Let $X$ be a type and $A$ an abelian group and $n:\N$, then
\[ H^n(X,A):=\| X\to K(A,n) \|_0\]
is the $n$-th cohomology group of $X$ with coefficients in $A$, where $\|\_\|_0$ is the $0$-truncation, an operation which turns a type with possibly non-trivial higher identity types into a set -- a type with trivial higher structure. The type $K(A,n)$ is the $n$-th Eilenberg MacLane space, which can always be constructed for any abelian group $A$ and comes with an isomorphism $\Omega^n(K(A,n))\simeq A$.
This definition of cohomology groups allows using the synthetic homotopy theory to reason about cohomology, which had been already done successfully at the time for the cohomology of homotopy types, like spheres and finite CW-complexes, but works as well to study $0$-types.
While this internal version of cohomology does not agree with the external version mentioned above --
even the type is wrong; it is a sheaf of groups instead of a single one and, indexed by an internal natural number instead of an external one -- internal cohomology turned out to be quite useful in practice.

In 2022, trying to use this approach to calculate cohomology groups in synthetic algebraic geometry led to the discovery of what is now called Zariski-local choice \cite{draft},
which is an additional axiom that holds in the higher Zariski-topos.
It is a weakening of the axiom of choice which can be formulated as: For any surjective map $f:X\to Y$, there exists a section, i.e.\ a map $s:Y\to X$ such that $f\circ s=\id_Y$.
Zariski-local choice also states the existence of a section, but only Zariski-local and only for surjections into an affine scheme: For any surjection $f:E\to \Spec A$,
there exists a Zariski-cover $U_1,\dots,U_n$ of $\Spec A$ and maps $s_i:U_i\to E$ such that $f(s_i(x))=x$ for all $x\in U_i$.

In homotopy type theory, we use the propositional truncation $\|\_\|$ to define surjections and more generally what we mean with ``exists''.
Propositional truncation turns an arbitrary type $A$ into a type $\|A\|$ with the property $x=y$ for all $x,y:\|A\|$.
Types with this property are called propositions or (-1)-types in homotopy type theory.
Using a univalent universe of types $\mathcal U$ we have that surjection into a type $A$ are the same as type families $F:A\to \mathcal U$, such that we have $\|F(x)\|$ for all $x: A$.
Using type families instead of maps allows us to drop the condition that the maps we get are sections, since we can express it using dependent function types and we arrive at the formulation of Zariski-local choice given below in the list of axioms.
In this instance and many others, homotopy type theory provides a lot of convenience when working very formally, which is an advantage in formalization of synthetic algebraic geometry.

In total, apart from homotopy type theory and a fixed commutative ring $R$ we use in synthetic algebraic geometry the following three axioms -- we will provide some explanation for the first one below:

\begin{center}
\begin{axiom}[Locality]%
  \label{loc-axiom}
  $R$ is a local ring, i.e.\ $1\neq 0$ and whenever $x+y$ is invertible $x$ is invertible or $y$ is invertible.
\end{axiom}

\begin{axiom}[Duality]%
  \label{duality-axiom}
  For any finitely presented $R$-algebra $A$, the homomorphism
  \[ a \mapsto (\varphi\mapsto \varphi(a)) : A \to (\Spec A \to R)\]
  is an isomorphism of $R$-algebras.
\end{axiom}

\begin{axiom}[Zariski-local choice]%
  \label{Z-choice-axiom}
  Let $A$ be a finitely presented $R$-algebra
  and let $B : \Spec A \to \mU$ be a family of inhabited types.
  Then there exists a Zariski-cover $U_1,\dots,U_n\subseteq \Spec A$
  together with dependent functions $s_i : (x : U_i)\to B(x)$.
\end{axiom}
\end{center}

With the Kock-Lawvere axiom, we introduced the first historic predecessor of the duality axiom as a starting point for convenient infinitesimal computations,
while this is also possible in synthetic algebraic geometry, the general duality axiom has a lot of surprising consequences.
In line with classical algebraic geometry, it shows that we have the usual anti-equivalence between finitely presented $R$-algebras and affine schemes of finite presentation over $R$.
What is more surprising, is the consequence that all functions $R\to R$ are polynomials and that it has implications on the properties of the base ring $R$.
For example, for all $x:R$, $x$ is invertible if and only if we have $x\neq 0$.
Duality also implies that affine schemes can only have bounded maps to the natural numbers.

Surprisingly, the Zariski-local choice axiom was also usable to solve problems which have no obvious connection to cohomology.
For example, it admits a proof that pointwise open subsets of an affine scheme are the same as subsets which are given by unions of non-vanishing sets of functions on the scheme.
In more detail, we say a proposition $P$ is open, if there are a natural number $n$ and elements $r_1,\dots,r_n$ of the base ring $R$,
such that $P$ is equivalent to the proposition $r_1\neq 0 \vee\dots\vee r_n\neq 0$.
Then we call a subset $U$ of a type $X$ open, if the proposition $x\in U$ is open for all $x:X$.
Using Zariski-local choice, these pointwise existing ring elements can be turned into locally existing functions.
For an affine scheme $X$ it is even the case, that an open subset in the pointwise sense, is a union of non-vanishing sets $D(f_i)$ of global functions $f_i:\Spec A \to R$.
An analogous result holds also for closed propositions, which are propositions of the form $r_1=0\wedge\dots\wedge r_n=0$ for $r_i:R$ and vanishing sets of functions on affine schemes.

The connection between pointwise and local openness is important to make the synthetic definition of a scheme work well:
A scheme is a type $X$, that merely has a finite open cover by affine schemes.
To produce interesting examples, it is necessary to use the locality axiom.
This is related to the Zariski topology and ensures that classical examples of Zariski covers can be reproduced.
One central example is projective space, which can be defined as the quotient of $R^{n+1}/\{0\}$ by the action of $R^\times$ by scaling.
A cover of this type is given by sets of equivalence classes of the form $\{[x_0:\cdots:x_n] \vert x_i\neq 0 \}$, which is clearly open by the pointwise definition.
To see that it is a cover, one has to note that for $x:R^{n+1}$, $x\neq 0$ is equivalent to one of the entries $x_i$ being different from 0. In synthetic algebraic geometry, this is the case for the base ring $R$ and the proof uses that $R$ is a local ring.

\paragraph{Contribution and organization of the article. }
We give a novel synthetic definition of formally étale, smooth and unramified types, using what we call \emph{closed dense propositions} (\Cref{def-etale-closed-dense}). We then define étale (resp. smooth, unramified) schemes simply as schemes that happens to be formally étale (resp. smooth, unramified) as types. Étale (resp. smooth, unramified) maps between schemes are just defined as maps with étale (resp. smooth, unramified) fibers.
This is an instance of a general phenomenon in synthetic reasoning: concepts which are usually defined locally can be defined pointwise. %or on the level of propositions. DON'T KNOW WHAT IS MEANT?

While describing the infinitesimal structure of schemes in section 2, we also point out a curious discovery: there is a duality between finitely presented modules and finitely copresented modules over the internal base ring $R$ (\Cref{dual-of-fcop-fp,double-dual-identity}).
The latter notion of finitely copresented modules is not very prominent in algebra, but appears naturally in the study of tangent spaces of schemes (\Cref{tangent-finite-copresented}).

We show that the new definitions agree with straightforward translations of the classical concepts (\Cref{connection-to-ega-definition}) and provide some characterizations using tangent spaces:
a map between schemes is unramified if and only if it induces injections on tangent spaces (\Cref{unramified-map-characterisation}), and a map between smooth schemes is étale (resp. smooth) if and only if it induces isomorphisms (resp. surjections) on tangent spaces (\Cref{etale-schemes-iff-local-iso,smooth-schemes-iff-submersion}).

Finally, we show that unramified, étale and smooth schemes can be described very concretely in the expected way, via conditions on the polynomials locally describing such schemes (\Cref{unramified-iff-locally-std-unramified} and \Cref{standard-etale-are-etale,standard-smooth-is-smooth}). An important intermediate result for the characterization of smooth schemes is that their tangent spaces are finite free $R$-modules (\Cref{smooth-have-free-tangent}).

\paragraph{Acknowledgements. }
We thank Thierry Coquand for discussions on the topic and in particular for explaining a proof of \Cref{extend-from-image} to us.
We thank Marc Nieper-Wißkirchen for a discussion which led to the explanation in \Cref{remark-sym-dual}.
Work on this article was supported by the ForCUTT project, ERC advanced grant 101053291.

\section{Formally étale, unramified and smooth types}
In this section we will give our new synthetic definitions of formally étale, formally unramified and formally smooth types and maps. It is remarkable that these definitions work for any type or map rather than just scheme and map between them. Here we will derive consequences of these definition applicable for all types, as well as compare these definitions to the traditional ones.

\subsection{Definitions}

In \cite{draft} it is shown that elements of the base ring $R$ are nilpotent if and only if they are not not zero.
Both nilpotency and double negated equality have been used to describe infinitesimals and the following closed dense propositions can be viewed as closed subspaces of the point which are infinitesimally close to being the whole point:

\begin{definition}
A closed proposition is dense\index{closed dense proposition} if it is merely of the form:
\[r_1=0\land\cdots\land r_n=0\]
with $r_1,\cdots,r_n:R$ nilpotent.
\end{definition}

\begin{remark}
A closed proposition $P$ is closed dense if and only if $\neg\neg P$.
\end{remark}

From a traditional perspective, the inclusion $P\subseteq 1$ of a closed dense proposition into the point would be an infinitesimal extension.
In the following, we will use closed dense propositions to define synthetic analogs of notions which are traditionally defined using lifting properties against classes of infinitesimal extensions.
More details on the connection to traditional definitions will be given in \Cref{connection-to-ega-definition}.

\begin{definition}
  \label{def-etale-closed-dense}
A type $X$ is formally étale (resp. formally unramified, formally smooth)\index{formally étale}\index{formally unramified}\index{formally smooth} if for all closed dense proposition $P$ the map:
\[X\to X^P\]
is an equivalence (resp. an embedding, surjective).
\end{definition}

\begin{remark}
The map $X\to X^P$ is an equivalence (resp. an embedding, surjective) if and only if for any map $P\to X$ we have a unique (resp. at most one, merely one) dotted lift in:
\begin{center}
\begin{tikzcd}
P\ar[r]\ar[d] & X\\
1\ar[ru,dashed]& \\
\end{tikzcd}
\end{center}
\end{remark}

\begin{definition}
A map is said to be formally étale (resp. formally unramified, formally smooth) if its fibers are formally étale (resp. formally unramified, formally smooth).
\end{definition}

\begin{remark}
A type (or map) is formally étale if and only if it is formally unramified and formally smooth.
\end{remark}

\begin{lemma}\label{etale-unramified-smooth-square-zero}
A type $X$ is formally étale (resp. formally unramified, formally smooth) if and only if for all $\epsilon:R$ such that $\epsilon^2=0$, the map:
\[X\to X^{\epsilon=0}\]
is an equivalence (resp. an embedding, surjective).
\end{lemma}

\begin{proof}
The direct direction is obvious as $\epsilon=0$ is closed dense when $\epsilon^2=0$.

For the converse, assume $P=\Spec(R/N)$ a closed dense proposition. Then the map $R\to R/N$ with $N$ finitely generated nilpotent ideal can be decomposed as:
\[R\to A_1\to \cdots A_n = R/N\]
where $A_k$ is a quotient of $R$ by a finitely generated nilpotent ideal and:
\[A_k\to A_{k+1}\]
is of the form:
\[A\to A/(a)\]
for some $a:A$ with $a^2=0$.

We write $P_k = \Spec(A_k)$ and:
\[i_k:P_{k+1}\to P_k\] 
so that $\mathrm{fib}_{i_k}(x)$ is $a(x)=0$ where $a(x)^2=0$ holds.

Then by hypothesis we have that for all $k$ and $x:P_{k}$ the map:
\[X\to X^{\mathrm{fib}_{i_k}(x)}\]
is an equivalence (resp. an embedding, surjective). So the map:
\[X^{P_{k}} \to \prod_{x:P_{k}}X^{\mathrm{fib}_{i_k}(x)} = X^{P_{k+1}}\]
is an equivalence (resp. an embedding, surjective, where surjectivity uses $P_{k}$ having choice).
We conclude that the map:
\[X\to X^P\]
is an equivalence (resp. an embedding, surjective).
\end{proof}



\subsection{Stability results}

Being formally étale is a modality given as nullification at all dense closed propostions and therefore lex \cite[Corollary 3.12]{modalities}.
This means that we have the following:

\begin{proposition}
Formally étale types enjoy the following stability results:
\begin{itemize}
\item If $X$ is any type and for all $x:X$ we have $Y_x$ formally étale, then $\prod_{x:X}Y_x$ is formally étale. 
\item  If $X$ is formally étale and for all $x:X$ we have $Y_x$ formally étale, then $\sum_{x:X}Y_x$ is formally étale. 
\item If $X$ is formally étale then for all $x,y : X$ the type $x=y$ is formally étale.
\item The type of formally étale types is itself formally étale.
\end{itemize}
\end{proposition}

Formally unramified type are the separated types \cite[Definition 2.13]{localization} associated to formally étale types. This means:

\begin{lemma}
A type $X$ is formally unramified if and only if for all $x,y:X$ the type $x=y$ is formally étale.
\end{lemma}

By \cite[Lemma 2.15]{localization}, being formally unramified is a nullification modality as well. This means we have the following:

\begin{proposition}
Formally unramified types enjoy the following stability results:
\begin{itemize}
\item If $X$ is any type and for all $x:X$ we have $Y_x$ formally unramified, then $\prod_{x:X}Y_x$ is formally unramified. 
\item If $X$ is formally unramified and for all $x:X$ we have $Y_x$ formally unramified, then $\sum_{x:X}Y_x$ is formally unramified.
\end{itemize}
\end{proposition}

Being formally smooth is not a modality, indeed we will see it is not stable under identity types. Neverthless we have the following results:

\begin{lemma}\label{smooth-sigma-closed}
Formally smooth types enjoy the following stability results:
\begin{itemize}
\item If $X$ is any type satifying choice and for all $x:X$ we have $Y_x$ formally smooth, then $\prod_{x:X}Y_x$ is formally smooth.
\item If $X$ is a formally smooth type and for all $x:X$ we have $Y_x$ formally smooth, then $\sum_{x:X}Y_x$ is formally smooth.
\end{itemize}
\end{lemma}


\subsection{Type-theoretic examples}

The next proposition implies that open propositions, and therefore open embeddings, are formally étale.

\begin{lemma}\label{not-not-stable-prop-etale}
  Any $\neg\neg$-stable proposition is formally étale.
\end{lemma}

\begin{proof}
  Assume $U$ is a $\neg\neg$-stable proposition. For $U$ to be formally étale it is enough to check that $U^P\to U$ for all $P$ closed dense. This holds because for $P$ closed dense we have $\neg\neg P$.
  \end{proof}

Before proving the next lemma about closed formally étale propositions,
we will state and prove a general fact about closed propositions:

\begin{lemma}
  \label{square-zero-implies-zero-decidable}
  Let $I$ be a finitely generated ideal of $R$ such that $I^2=0$ implies $I=0$.
  Then the closed proposition $I=0$ is decidable.
\end{lemma}

\begin{proof}
  Let $I\subseteq R$ be a finitely generated ideal such that $I^2=0$ implies $I=0$.
  Since the other implication always holds, the propositions $I^2=0$ and $I=0$ are equivalent, so we have $I=I^2$.
  By Nakayama (see \cite[Lemma II.4.6]{lombardi-quitte}) there exists $e:R$ such that $eI = 0$ and $1-e\in I$.
  If $e$ is invertible then $I=0$, if $1-e$ in invertible then $I=R$.
\end{proof}

\begin{lemma}\label{closed-and-etale-decidable}
Any formally étale closed proposition is decidable.
\end{lemma} 

\begin{proof}
Given a formally étale closed proposition $P$, let us prove it is $\neg\neg$-stable. Indeed if $\neg\neg P$ then $P$ is closed dense so that $P\to P$ implies $P$ since $P$ is formally étale. 

Let $I$ be the finitely generated ideal in $R$ such that:
\[P\leftrightarrow I=0\]
We have that $I^2=0$ implies $\neg\neg (I=0)$ which implies $I=0$.
Then $P$ is decidable by \Cref{square-zero-implies-zero-decidable}. 
\end{proof}

\begin{proposition}\label{bool-is-etale}
  The type $\Bool$ is formally étale.
\end{proposition}

\begin{proof}
The identity types in $\Bool$ are decidable so $\Bool$ is formally unramified. Consider $\epsilon:R$ such that $\epsilon^2=0$ and a map:
\[\epsilon=0 \to \Bool\]
we want to merely factor it through $1$.

 Since $\Bool\subseteq R$, by duality the map gives $f:R/(\epsilon)$ such that $f^2=f$. Since $R/(\epsilon)$ is local we conclude that $f = 1$ or $f=0$ and so the map has constant value $0:\Bool$ or $1:\Bool$.
\end{proof}

\begin{remark}\label{finite-are-etale}
This means that formally étale (resp. formally unramified, formally smooth) types are stable by finite sums. In particular finite types are formally étale.
\end{remark}

%Not used anywhere, maybe remove TODO?
\begin{proposition}
The type $\N$ is formally étale.
\end{proposition}

\begin{proof}
Identity types in $\N$ are decidable so $\N$ is formally unramified, we want to show it is formally smooth. Assume given a map:
\[P\to \N\]
for $P$ a closed dense proposition, we want to show it merely factors through $1$. By boundedness the map merely factors through a finite type, which is formally étale by \Cref{finite-are-etale} so we can conclude.
\end{proof}

\begin{lemma}\label{prop-are-unramified}
Any proposition is formally unramified.
\end{lemma}

This means that any subtype of a formally unramified type is formally unramified.

\begin{remark}
  Given any lex modality, a type is separated if and only if it is a subtype of a modal type,
  so a type is formally unramified if and only if it is a subtype of a formally étale type.
\end{remark}

We also have the following surprising dual result, meaning that any quotient of a formally smooth type is formally smooth:

\begin{proposition}\label{smoothSurjective}
If $X$ is formally smooth and $p:X\twoheadrightarrow Y$ surjective, then $Y$ is formally smooth.
\end{proposition}

\begin{proof}
For any $P$ closed dense and any map $P\to Y$, consider the diagram:
 \begin{center}
      \begin{tikzcd}
      P \ar[rd,dashed]\ar[d]\ar[r]& Y\\
      1 \ar[r,dashed,swap,"x"]& X\ar[u,swap,"p",two heads]
      \end{tikzcd}
    \end{center} 
    By choice for closed propositions we merely get the dotted diagonal, and since $X$ is formally smooth we get the dotted $x$, and then $p(x)$ gives a lift.
\end{proof}


\subsection{Classical definitions, examples and counter-examples}

In this section we will show that our definition of étale, smooth and unramified maps between schemes is equivalent to the obvious internal version of the traditional definition. It is important to keep in mind that our schemes are always locally of finite presentation, so the following definition is sensible:

\begin{definition}
  An étale (resp.\ unramified, smooth)\index{étale}\index{smooth}\index{unramified} scheme is a scheme which is formally étale (resp.\ formally unramified, formally smooth) as a type.
  An étale (resp.\ unramified, smooth) map is a map between schemes which is formally étale (resp.\ formally unramified, formally smooth).
\end{definition}

A criterion very similar to next remark appears as the definition of a formally étale, unramified and smooth maps in \cite[§17]{EGAIV4},
except we restrict to finitely presented algebras and finitely generated ideals, as our schemes are assumed locally of finite presentation, and we ask the lift for smoothness to exists only \emph{Zariski-locally}, as suggested in \cite[\href{https://stacks.math.columbia.edu/tag/02GZ}{Tag 02GZ}]{stacks-project}.  It is not clear if this internal criterion corresponds to the external definitions.

\begin{remark}
  \label{connection-to-ega-definition}
  Let $f:X\to Y$ be a map between schemes.
  Then $f$ is étale (resp.\ unramified, smooth) if and only if there exists exactly one (resp.\ at most one, at least one \emph{Zariski-locally}) dotted lift in all squares of the form:
  \begin{center}
    \begin{tikzcd}
      \Spec(A/N)\ar[r,"t"]\ar[d] & X\ar[d,"f"] \\
      \Spec(A)\ar[ru,dashed]\ar[r,swap,"b"] & Y
    \end{tikzcd}    
  \end{center}
where $A$ is a finitely presented $R$-algebra, $N$ a finitely generated nilpotent ideal and the left map is induced by the quotient map $A\to A/N$. In the smooth case, we believe that it is possible to prove global existence of these lifts using cohmological methods.
\end{remark}

\begin{proof}
  The inclusion of a closed dense proposition $P$ into $1$ is a special case of left map in the remark, so we only need to show that étale,
  smooth and unramified maps satisfy the more general lifting property. For étale and unramified maps, we can just apply the lifting property for closed dense propositions for all points in $\Spec A$. So let $f:X\to Y$ be smooth. Then we merely have a lift for each point in $\Spec(A)$ and can apply Zariski-local choice to get the desired result.
\end{proof}

We conclude this section with a few examples and counter examples.

\begin{lemma}\label{An-is-smooth}
For all $k:\N$, we have that $\A^k$ is smooth.
\end{lemma}

\begin{proof}
  Let $P$ be a closed dense proposition and $N$ a nilpotent, finitely generated ideal such that $P=\Spec(R/N)$.
  Since $\Spec(R[X_1,\dots,X_k])=\A^k$, to prove $\A^k$ smooth we just need to find dotted lifts in:
  \begin{center}
    \begin{tikzcd}
      R/N & R[X_1,\dots,X_k]\ar[l]\ar[ld,dashed]  \\
      R\ar[u] & 
    \end{tikzcd}
  \end{center}
  This is easy using the universal property of $R[X_1,\dots,X_k]$.
\end{proof}

\begin{example}
The affine scheme $\Spec(R[X]/X^2)$ is not smooth.
\end{example}

\begin{proof}
If it were smooth, then for any $\epsilon$ with $\epsilon^3=0$ we would be able to prove $\epsilon^2=0$.
Indeed we would merely have a dotted lift in:
 \begin{center}
      \begin{tikzcd}
        R/(\epsilon^2)& R[X]/(X^2)\ar[l,swap,"\epsilon"]\ar[ld,dashed] \\
        R \ar[u]& 
      \end{tikzcd}
    \end{center}
    that is, an $r:R$ such that $(\epsilon+r\epsilon^2)^2=0$. Then $\epsilon^2=0$.
\end{proof}

\begin{example}
The affine scheme $\Spec(R[X,Y]/XY)$ is not smooth.
\end{example}

\begin{proof}
Again, we assume a lift for any $\epsilon$ with $\epsilon^3=0$:
 \begin{center}
   \begin{tikzcd}
     R/(\epsilon^2) & R[X,Y]/(XY)\ar[l]\ar[ld,dashed] \\
     R\ar[u] & 
   \end{tikzcd}
 \end{center}
 where the top map sends both $X$ and $Y$ to $\epsilon$. Then we have $r,r':R$ such that $(\epsilon+r\epsilon^2)(\epsilon+r'\epsilon^2)=0$ so that $\epsilon^2=0$.
\end{proof}

We will proof a generalization of the following example in \Cref{standard-etale-are-etale}.
The essential step is to improve a zero $g(y)=0$ up to some square-zero $\epsilon$ to an actual zero.

\begin{example}
Let $g$ be a polynomial in $R[X]$ such that for all $x:R$ we have that $g(x)=0$ implies $g'(x)\not=0$. Then $\Spec(R[X]/g)$ is étale.
\end{example}


\subsection{Being formally étale, unramified or smooth is Zariski local}

\begin{lemma}\label{etale-zariski-local}
Let $X$ be a type with $(U_i)_{i:I}$ a finite open cover of $X$. Then $X$ is formally étale (resp. formally unramified, formally smooth) if and only if all the $U_i$ are formally étale (resp. formally unramified, formally smooth).
\end{lemma}

\begin{proof}
First, we show this for formally unramified:
\begin{itemize}
\item Any subtype of a formally unramified type is formally unramified by \Cref{prop-are-unramified}, so $X$ formally unramified implies $U_i$ formally unramified.
\item Conversely, is each $U_i$ is formally unramified, then for all $x,y:X$ we need to prove $x=_Xy$ formally étale. But there exists $i:I$ such that $x\in U_i$ and then:
\[x=_Xy \leftrightarrow \sum_{y\in U_i} x=_{U_i}y \]
which is formally étale because open propositions are formally étale by \Cref{not-not-stable-prop-etale}.
\end{itemize}
Now for formally smooth:
\begin{itemize}
\item Open propositions are formally smooth by \Cref{not-not-stable-prop-etale} so that open subtypes of formally smooth types are formally smooth.
\item Conversely if each $U_i$ is formally smooth then $\Sigma_{i:I}U_i$ is formally smooth by \Cref{finite-are-etale}, so we can conclude that $X$ is formally smooth by applying \Cref{smoothSurjective} to the surjection $\Sigma_{i:I}U_i \twoheadrightarrow X$.
\end{itemize}
The result for formally étale immediately follows.
\end{proof}

\begin{corollary}
For all $k:\N$, the projective space $\bP^k$ is smooth.
\end{corollary}

\begin{proof}
By \Cref{etale-zariski-local} it is enough to check that $\A^k$ is smooth. This is \Cref{An-is-smooth}
\end{proof}






\section{Linear algebra and tangent spaces}

\subsection{Modules and infinitesimal disks}

\begin{definition}
Given $M$ a finitely presented $R$-module, we define an f.p. algebra structure on $R\oplus M$ by:
\[(r,m)(r',m') = (rr',rm'+r'm)\]
we define:
\[\D(M) = \Spec(R\oplus M)\]
It is pointed by the first projection which we denote $0$. 
\end{definition}

We write $\D(1)$ for $\D(R)$ so that:
\[\D(1) = \Spec(R[X]/(X^2)) = \{\epsilon:R\ |\ \epsilon^2=0\}\]

\begin{definition}
Assume given $M$ a f.p. module and $A$ an f.p. algebra with $x:\Spec(A)$. An $M$-derivation at $x$ is a morphism of modules:
\[d:A\to M\]
such that for all $a,b:A$ we have that:
\[d(ab) = a(x)d(b) + b(x)d(a)\]
\end{definition}

\begin{lemma}\label{tangent-are-derivation}
Assume given $M$ a f.p. module and $A$ an f.p. algebra with $x:\Spec(A)$. Pointed maps:
\[\D(M)\to_\pt (\Spec(A),x)\] 
corresponds to $M$-derivation at $x$.
\end{lemma}

\begin{proof}
Such a pointed map correpond to an algebra map:
\[f : A\to R\oplus M\]
where the composite with the first projection is $a$. This means that:
\[f(a) = (a(x),d(x))\]
for some module map $d:A\to M$. We can immediately see that $f$ being a map of algebra is equivalent to $d$ being a derivation.
\end{proof}

\begin{lemma}\label{equivalence-module-infinitesimal}
Let $M$, $N$ be finitely presented modules. Then linear maps $M \to N$ correspond to
pointed maps $\D(N) \to_\pt \D(M)$. 
%Explicitly, a linear map $g : M \to N$ corresponds to the pointed map $f \mapsto m \mapsto f(g(m))$.
\end{lemma}

\begin{proof}
By \cref{tangent-are-derivation} such a pointed map correspond to an $N$-derivations at $0:\D(M)$.

Such a derivation is a morphism of modules:
\[R\oplus M\to N\]
such that for all $(r,m),(r',m'):R\oplus M$ we have that:
\[d(rr',rm'+r'm) = rd(r',m')+r'd(r,m)\]
This is equivalent to $d(r,0) = 0$ for all $r : R$, so we obtain all $R$-linear 
maps $M \to N$ in this way.
\end{proof}

\begin{lemma}
Any finitely copresented module is projective.
\end{lemma}

\begin{proof}
\rednote{TODO, unsure how it connects to $N_1(x) = \D(T_x(X)^\star)$.}
\end{proof}

\begin{lemma}\label{neighborhood-tangent-correspondence-smooth}
A linear map between finitely copresented module:
\[f:M\to N\]
is surjective if and only if the corresponding pointed map:
\[\D(M^\star) \to \D(N^\star)\]
merely has a section preserving $0$.
\end{lemma}

\begin{proof}
By \cref{equivalence-module-infinitesimal} we know that:
\[\D(M^\star) \to \D(N^\star)\]
merely having a section preserving $0$ is equivalent to:
\[f:M\to N\]
merely having a section. But since any finitely copresented module is projective, this is equivalent to $f$ being surjective.
\end{proof}


\subsection{Tangent spaces}

\begin{definition}
Let $X$ be a type and let $x:X$, then we define the tangent space $T_x(X)$ of $X$ at $x$ by:
\[\{t:\D(1)\to X\ |\ t(0)=x\}\]
\end{definition}

\begin{definition}
Given $f:X\to Y$ and $x:X$ we have a map:
\[df_x : T_x(X)\to T_{f(x)}(Y)\]
induced by post-composition.
\end{definition}

\begin{lemma}\label{An-dimension-n}
For all $x:R^n$ we have $T_x(R^n) = R^n$.
\end{lemma}

\begin{proof}
Since $R^n$ is homogeneous we can assume $x=0$. By \cref{tangent-are-derivation} we know that $T_0(R^n)$ correspond to linear map:
\[R[X_1,\cdots,X_n] \to R\]
such that for all $P,Q$ we have:
\[d(PQ) = P(0)dQ + Q(0)dP\]
which is equivalent to $d(1) = 0$ and $d(X_iX_j) = 0$, so such a map us detemined by its image on the $X_i$ so it is equivalent to an element of $R^n$.
\end{proof}

\begin{lemma}\label{from-D1-to-D2}
Given $X$ a scheme with $x:X$ and $v,w:T_x(X)$, there exists a unique:
\[\psi_{v,w} : \D(2)\to_\pt X\]
such that for all $\epsilon:\D(1)$ we have that:
\[\psi_{v,w}(\epsilon,0) = v(\epsilon)\]
\[\psi_{v,w}(0,\epsilon) = w(\epsilon)\]
\end{lemma}

\begin{proof}
We can assume $X$ is affine. Then $\D(2)\to_\pt X$ is equivalent to the type of $R^2$-derivation at $x$, but giving an $M\oplus N$ derivation is equivalent to giving an $M$-derivation and an $N$-derivation. Checking the equalities is a routine computation.
\end{proof}

\begin{lemma}
For any scheme $X$ and $x:X$, we have that $T_x(X)$ is a module.
\end{lemma}

\begin{proof}
We define scalar multiplication by sending $v$ to $t\mapsto v(rt)$.

Then to define addition of $v,w:T_x(X)$, we have define:
\[(v+w)(\epsilon) = \psi_{v,w}(\epsilon,\epsilon)\]
where $\psi_{v,w}$ is defined in \cref{from-D1-to-D2}.

We omit the checking that this is a module structure.
\end{proof}

\begin{lemma}
For $f:X\to Y$ a map between schemes, for all $x:X$ the map $df_x$ is a map of $R$-module.
\end{lemma}

\begin{proof}
Commutation with scalar multiplication is immediate.

Commutation comes by applying uniqueness in \cref{from-D1-to-D2} to show:
\[f\circ \psi_{v,w} = \psi_{f\circ v,f\circ w}\]
\end{proof}

\begin{lemma}
\label{kernel-is-tangent-of-fibers}
For any map $f:X\to Y$ and $x:X$, we have that:
\[
\mathrm{Ker}(df_x) = T_{(x,\refl_{f(x)})}(\mathrm{fib}_f(f(x)))
\]
\end{lemma}

\begin{proof}
This holds because:
\[
(\mathrm{fib}_f(f(x)),(x,\refl_{f(x)}))
\]
is the pullback of:
\[
(X,x) \to (Y,f(y)) \leftarrow (1,*)
\]
in pointed types, applied using $(\mathbb{D}(1),0)$.
\end{proof}

\begin{lemma}
Let $X$ be a scheme with $x : X$. Then $T_x(X)$ is a finitely
co-presented $R$-module.
\end{lemma}

\begin{proof}
We cab assume $X$ affine. Then $X$ is the kernel of a map:
\[P:R^m\to R^n\]
so that for all $x:X$, by applying \cref{kernel-is-tangent-of-fibers} we know that we have $T_x(X)$ is the kernel of:
\[dP_x : T_x(R^m)\to T_0(R^n)\]
and we conclude by \cref{An-dimension-n}.
\end{proof}


\subsection{Infinitesimal neighborhood}

\begin{definition}
Let $X$ be a set with $x:X$. The first order neighborhood $N_1(x)$ is defined as the set of $y:X$ such that the exists a f.g. ideal $I$ such that $I^{2}=0$ and:
\[I=0 \to x=y\]
\end{definition}

\begin{lemma}\label{duality-infinitesimal-tangent}
Let $X$ be a scheme with $x:X$, then:
\[N_1(x) = \D(T_x(X)^\star)\]
\end{lemma}

\begin{proof}
\rednote{TODO}
\end{proof}


\subsection{Rank of matrices}

\begin{definition}
A matrix is said of rank $n$ if it has an invertible $n$-minor, and all its $n+1$-minor have determinant $0$.
\end{definition}

Having a rank is a property of matrices, and there is not rank function defined on all matrices.

\begin{lemma}\label{rank-bloc-matrix}
Assume given a matrix $M$ of rank $n$ decomposed into blocks:
\[M = \begin{pmatrix}
P & Q  \\
R & S \\
\end{pmatrix}\]
Such that $P$ is square of size $n$ and invertible. Then we have:
\[S = RP^{-1}Q\]
\end{lemma}

\begin{proof}
By columns manipulation the matrix is equivalent to:
\[M = \begin{pmatrix}
P & 0  \\
0 & S - RP^{-1}Q \\
\end{pmatrix}\]
but equivalent matrices have the same rank so $S=RP^{-1}Q$.
\end{proof}

\begin{definition}
Two matrices $M,N$ are said equivalent if there are invertible matrices $P,Q$ such that $M = PNQ$.
\end{definition}

It is clear that equivalent matrices have the same rank.

\begin{lemma}\label{rank-equivalent-definitions}
Assume given a matrix:
\[M : R^m\to R^k\]
Then the following are equivalent:
\begin{enumerate}[(i)]
\item $M$ has rank $n$.
\item The kernel of $M$ is equivalent to $R^{m-n}$.
\item The image of $M$ is equivalent to $R^n$.
\item $M$ is equivalent to the bloc matrix:
\[\begin{pmatrix}
I_n & (0)  \\
(0) & (0) \\
\end{pmatrix}\]
\end{enumerate}
\end{lemma}

\begin{proof}
\rednote{TODO, only (ii) implies (i) actually needed.}
\end{proof}







\section{Unramified schemes}
\label{unramified-characterisation}
In this short section we present characterisations of unramified schemes and unramified maps between them. The situation is significantly simpler than with smoothness and étaleness.

\subsection{Unramified schemes}

\begin{lemma}\label{unramified-affine-characterisation}
Let $X$ be an affine scheme, the following are equivalent:
\begin{enumerate}[(i)]
\item $X$ is unramified.
\item Identity types in $X$ are decidable.
\item For all $x:X$, we have that $T_x(X)=0$.
\end{enumerate}
\end{lemma}

\begin{proof}
(i) implies (ii): By \Cref{closed-and-etale-decidable}.

(ii) implies (i): Decidable propositions are formally étale.

(ii) implies (iii): Assume given $x:X$ with $t:T_x(X)$, then for all $\epsilon:\D(1)$ we have $\neg\neg(\epsilon = 0)$ so that we have $\neg\neg (t(\epsilon) = t(0))$ which implies $t(\epsilon) = t(0)$ since equality is assumed decidable. Therefore $t = 0$ in $T_x(X)$.

(iii) implies (i): Indeed given $\epsilon:R$ such that $\epsilon^2=0$, assume $x,y:X$ such that $\epsilon=0 \to x=y$. Then $x\in N_1(y)$ so that by \Cref{duality-infinitesimal-tangent} and $T_y(X)=0$ we conclude $x=y$.
\end{proof}

\begin{corollary}\label{unramified-scheme-characterisation}
Let $X$ be a scheme, the following are equivalent:
\begin{enumerate}[(i)]
\item $X$ is unramified.
\item Identity types in $X$ are open.
\item For all $x:X$, we have that $T_x(X)=0$.
\end{enumerate}
\end{corollary}

\begin{proof}
Assume $(U_i)_{i:I}$ a finite cover of $X$ by affine schemes. By \Cref{etale-zariski-local} we have that $X$ is formally unramified if and only if $U_i$ is formally unramified for all $i:I$.

(ii) implies (i). By \Cref{not-not-stable-prop-etale}.

(i) implies (iii). Indeed for all $x:X$ there exists $i:I$ such that $x\in U_i$, then $T_x(X) = T_x(U_i)$ and $T_x(U_i) = 0$ by \Cref{unramified-affine-characterisation}.

(iii) implies (ii). Assume $x,y:X$, then there exists $i:I$ such that $x\in U_i$ and:
\[x=_Xy \leftrightarrow \Sigma_{y\in U_i} x=_{U_i} y\]
By \Cref{unramified-affine-characterisation} we have that identity types in $U_i$ are decidable, so $x=_Xy$ is open.
\end{proof}

\subsection{Unramified morphisms between schemes}

Now we generalise this to maps between schemes.

\begin{proposition}\label{unramified-map-characterisation}
A map between schemes is unramified if and only if its differentials are injective. 
\end{proposition}

\begin{proof}
The map $df_x$ is injective if and only if its kernel is $0$. By \Cref{kernel-is-tangent-of-fibers}, this means that $df_x$ is injective for all $x:X$ if and only if:
\[
\prod_{x:X}T_{(x,\refl_{f(x)})}(\mathrm{fib}_f(f(x)))=0
\]
On the other hand having fibers with trivial tangent space is equivalent to:
\[
\prod_{y:Y}\prod_{x:X}\prod_{p:f(x)=y} T_{(x,p)}(\mathrm{fib}_f(y)) = 0
\]
Both are equivalent by path elimination on $p$.
\end{proof}


\subsection{Unramified schemes are locally standard}

\begin{definition}
A scheme is called standard unramified if it is of the form:
\[\Spec(R[X_1,\cdots,X_n]/P_1,\cdots,P_k)\]
with $k\geq n$ such that the determinant of:
\[\left( \frac{\partial P_i}{\partial X_j}\right)_{1\leq i,j\leq n}\]
is invertible in $R[X_1,\cdots,X_n]/P_1,\cdots,P_k$.
\end{definition}

%It is clear that standard unramified schemes have trivial tangent spaces, so that by \Cref{unramified-affine-characterisation} they are unramified. We have a local converse.

\begin{lemma}\label{standard-unramified-is-unramified}
A standard unramfied scheme is indeed unramified.
\end{lemma}

\begin{proof}
Given $X$ standard unramified, for all $x:X$ by \Cref{kernel-is-tangent-of-fibers} we have an exact sequence:
  \begin{center}
    \begin{tikzcd}
      0\ar[r] & T_x(X)\ar[r] & R^n\ar[r,"dP_x"] & R^k
    \end{tikzcd}
  \end{center}
  But since $dP_x$ is represented by the Jacobian matrix $\frac{\partial P_i}{\partial X_j}(x)$, the invertibility condition means $dP_x$ is injective and we can conclude.
\end{proof}

\begin{proposition}
  \label{unramified-iff-locally-std-unramified}
A scheme is unramified if and only if it has a cover by standard unramified schemes.
\end{proposition}

\begin{proof}
By \Cref{etale-zariski-local} and \Cref{standard-unramified-is-unramified}, we get the converse.

  For the direct implication, by \Cref{etale-zariski-local} it is enough to consider an affine scheme: 
  \[
  X=\Spec(R[X_1,\cdots,X_n]/P_1,\cdots,P_k)
  \]
 We reason as in \Cref{standard-unramified-is-unramified} to get that the Jacobian matrix $\left(\frac{\partial P_i}{\partial X_j}(x)\right)$ is invertible for all $x:X$, which means that $n\leq k$ and the Jacobians matrix has an invertible $n$-minor. We cover by principal open according to which $n$-minor is invertible and reorder variables and polynomials to get a cover by pieces if the form:
 \[\Spec(R[X_1,\cdots,X_n,Y]/P_1,\cdots,P_k,1-YQ(X))\]
 such that $\left(\frac{\partial P_i}{\partial X_j}(x)\right)_{1\leq i,j\leq n}$ is invertible for all $x:X$ such that $Q(x)\not=0$. If we reorder the quotienting ideal as $P_1,\hdots,P_n,1-YQ(X),P_{n+1},\hdots,P_k$ we get a standard unramified scheme.
\end{proof}




\section{Smooth and étale schemes}
In this section we will give characterizations similar to Section \ref{unramified-characterisation} for smooth and étale schemes.

\subsection{Smooth and étale maps between schemes}

Smooth maps between with a smooth smooth source are precisely submersions.

\begin{corollary}\label{smooth-schemes-iff-submersion}
Let $f:X\to Y$ be a map between schemes with $X$ smooth. Then the following are equivalent:
\begin{enumerate}[(i)] 
\item The map $f$ is smooth.
\item For all $x:X$, the induced map:
\[df : T_x(X)\to T_{f(x)}(Y)\]
is surjective.
\end{enumerate}
\end{corollary}

\begin{proof}
(i) implies (ii). Assume given $v:T_{f(x)}(Y)$, then for all $t:\D(1)$ we have a map:
\[t=0 \to \fib_f(v(t))\]
with constant value $x$. So since $f$ is smooth we merely have $w_t:\fib_f(v(t))$ such that $t=0$ implies $w_t=x$. We conclude using choice over $\D(1)$.

(ii) implies (i). Assume given $y:Y$ and $\epsilon:R$ such that $\epsilon^2=0$ and try to merely find a dotted lift in:
 \begin{center}
      \begin{tikzcd}
        \epsilon=0\ar[r,"\phi"]\ar[d] & \fib_f(y)\\
       1 \ar[ru,dashed] & \\
      \end{tikzcd}
    \end{center}
    Since $X$ is formally smooth we merely have an $x:X$ such that:
\[\prod_{p:\epsilon=0} \phi(p)=x\]
and therefore:
\[ \epsilon=0 \to y=f(x)\]
which means that $y\in N_1(f(x))$. 

We use \Cref{duality-infinitesimal-tangent} and \Cref{neighborhood-tangent-correspondence-smooth} to get that the map $N_1(x)\to N_1(f(x))$ induced by $f$ merely has a section $s$ sending $f(x)$ to $x$. Then $s(y):\fib_f(y)$ is such that for all $p:\epsilon=0$ we have that:
\[\phi(p) = x = s(f(x)) = s(y)\]
\end{proof}

\begin{corollary}\label{etale-schemes-iff-local-iso}
Let $f:X\to Y$ be a map between schemes with $X$ smooth. Then the following are equivalent:
\begin{enumerate}[(i)]
\item The map $f$ is étale. 
\item For all $x:X$, the induced map:
\[df : T_x(X)\to T_{f(x)}(Y)\]
is an iso.
\end{enumerate}
\end{corollary}

\begin{proof}
We apply \Cref{unramified-map-characterisation} and \Cref{smooth-schemes-iff-submersion}.
\end{proof}

\begin{remark}\label{smooth-maps-are-submersions}
In both previous results, we did not use the smoothness hypothesis on $X$ to prove (i) implies (ii).
\end{remark}

\subsection{Smooth schemes have free tangent spaces}

\begin{lemma}\label{smooth-implies-smooth-tangent}
Assume $X$ is a smooth scheme. Then for any $x:X$ the type $T_x(X)$ is smooth.
\end{lemma}

\begin{proof}
Consider $T(X) = X^{\mathbb{D}(1)}$ the tangent bundle of $X$. We have to prove that the map:
\[p:T(X)\to X\]
is formally smooth. Both source and target are schemes, and the source is formally smooth because $X$ is smooth and $\mathbb{D}(1)$ has choice. So by \Cref{smooth-schemes-iff-submersion} it is enough to prove that for all $x:X$ and $v:T_x(X)$ the induced map:
\[dp:T_{(x,v)}(T(X))\to T_x(X)\]
is surjective. 

Consider $u:T_x(X)$. By unpacking the definition of tangent spaces and computing $dp(w)$, we see that merely finding $w:T_{(x,v)}(T(X))$ such that $dp(w) = u$ means merely finding:
\[\phi : \mathbb{D}(1) \times \mathbb{D}(1) \to X\]
such that for all $t:\mathbb{D}(1)$ we have that:
\[\phi(0,t) = v(t)\]
\[\phi(t,0) = u(t)\]

But from \Cref{from-D1-to-D2} we know that there exists a unique:
\[\psi_{v,u} : \mathbb{D}(2)\to X\]
such that:
\[\psi_{v,u}(0,t) = v(t)\]
\[\psi_{v,u}(t,0) = u(t)\]

Then since $X$ is smooth and the fibers of:
\[\mathbb{D}(2) \to\mathbb{D}(1) \times \mathbb{D}(1) \]
are closed dense, we conclude from $\mathbb{D}(1) \times \mathbb{D}(1)$ having choice that there merely exists a lift of $\psi_{v,u}$ to $\mathbb{D}(1) \times \mathbb{D}(1)$, which gives us the $\phi$ we wanted.
\end{proof}

\begin{lemma}\label{smooth-kernel-decidable}
Assume given a linear map:
\[M:R^m\to R^n\] 
which has smooth kernel $K$. Then we can decide whether $M=0$.
\end{lemma}

\begin{proof}
Since $M=0$ is closed, by \Cref{not-not-stable-prop-etale} and \Cref{closed-and-etale-decidable} it is enough to prove that it is $\neg\neg$-stable to conclude that it is decidable. Assume $\neg\neg(M=0)$, then for any $x:R^m$ we have a dotted lift in:
 \begin{center}
      \begin{tikzcd}
        M=0\ar[d] \ar[r,"\_\, \mapsto x"] & K \\
       1 \ar[dashed,ru] &
      \end{tikzcd}
\end{center}
because $K$ is formally smooth, so that we merely have $y\in K$ such that: 
\[M=0\to x=y\]
which implies that $\neg\neg(x=y)$ since we assumed $\neg\neg(M=0)$.

Then considering a basis $(x_1,\cdots,x_n)$ of $R^m$, we get $(y_1,\cdots,y_n)$ such that for all $i$ we have that $y_i\in K$ and $\neg\neg(y_i=x_i)$. But then we have that $(y_1,\cdots,y_n)$ is infinitesimally close to a basis and that being a basis is an open proposition, so that $(y_1,\cdots,y_n)$ is a basis and $K=R^m$ so $M=0$.
\end{proof}

\begin{lemma}\label{smooth-corpresented-implies-free}
Assume that $K$ is a finitely copresented module that is also smooth. Then it is finite free.
\end{lemma}

\begin{proof}
Assume a finite copresentation:
\[0\to K\to R^m\overset{M}{\to} R^n\]
We proceed by induction on $m$. By \Cref{smooth-kernel-decidable} we can decide whether $M=0$ or not.
\begin{itemize}
\item If $M=0$ then $K=R^m$ and we can conclude.
\item If $M\not=0$ then we can find a non-zero coefficient in the matrix corresponding to $M$, and so up to base change it is of the form:
\[
\begin{pmatrix}
1 & \begin{matrix}0&\cdots & 0\end{matrix}  \\
\begin{matrix}0\\ \vdots\\ 0\end{matrix} & \widetilde{M} \\
\end{pmatrix}
\]
But then we know that the kernel of $M$ is isomorphic to the kernel of $\widetilde{M}$, and by applying the induction hypothesis we can conclude that it is finite free.
\end{itemize}
\end{proof}

\begin{proposition}\label{smooth-have-free-tangent}
Let $X$ be a smooth scheme. Then for any $x:X$ we have that $T_x(X)$ is finite free.
\end{proposition}

\begin{proof}
By \Cref{smooth-implies-smooth-tangent} we have that $T_x(X)$ is formally smooth, so that we can conclude by \Cref{smooth-corpresented-implies-free}.
\end{proof}

The dimension of $T_x(X)$ is called the dimension of $X$ at $x$. By boundedness any smooth scheme is a finite sum of smooth schemes of a fixed dimension.
We can turn this into a definition of dimension which works well in the case of smooth schemes:

\begin{definition}
  \label{definition-smooth-dim-n}
  A scheme is \notion{smooth of dimension $n$}, if it is smooth and all its tangent spaces are finite free of dimension $n$.
\end{definition}


\subsection{Standard étale and standard smooth schemes}

\begin{definition}
A standard smooth scheme of dimension $k$ is an affine scheme of the form:
\[\Spec\big(R[X_1,\cdots,X_n,Y_1,\cdots Y_{k}] / P_1,\cdots,P_n\big)\]
where the determinant of:
\[\left( \frac{\partial P_i}{\partial X_j}\right)_{1\leq i,j\leq n}\]
is invertible in $R[X_1,\cdots,X_n,Y_1,\cdots Y_{k}] / P_1,\cdots,P_n$.
\end{definition}

\begin{definition}
A standard smooth scheme of dimension $0$ is called a \notion{standard étale scheme}.
\end{definition}

\begin{remark}
  \label{standard-open-in-std-smooth}
  Let $X=\Spec\big(R[X_1,\cdots,X_n,Y_1,\cdots Y_{k}] / P_1,\cdots,P_n\big)$
  be as above but such that the determinant is
  only invertible in a localization $(R[X_1,\cdots,X_n,Y_1,\cdots Y_{k}] / P_1,\cdots,P_n)_G$ for some $G:X\to R$.
  Then $X$ is standard smooth of dimension $k$ since it is
  \[ \Spec\big(R[X_1,\cdots,X_n,X_{n+1},Y_1,\cdots,Y_k] / P_1,\cdots,P_n,GX_{n+1}-1\big)\]
  and we can compute
  \[\mathrm{det}(\mathrm{Jac}(P_1,\cdots,P_n,GX_{n+1}-1)) = \pm\mathrm{det}(\mathrm{Jac}(P_1,\cdots,P_n)) \cdot G\]
  which is invertible in $R[X_1,\cdots,X_n,X_{n+1},Y_1,\cdots,Y_k] / P_1,\cdots,P_n,GX_{n+1}-1$.
\end{remark}

\begin{lemma}\label{standard-etale-are-etale}
Standard étale schemes are étale.
\end{lemma}

\begin{proof}
Assume given a standard étale algebra:
\[R[X_1,\cdots,X_n]/P_1,\cdots,P_n\]
and write:
\[P:R^n\to R^n\]
for the map induced by $P_1,\cdots,P_n$.

Assume given $\epsilon:R$ such that $\epsilon^2=0$, we need to prove that there is a unique dotted lifting in:
  \begin{center}
      \begin{tikzcd}
       R/\epsilon & R[X_1,\cdots,X_n]/P_1,\cdots,P_n\ar[l,swap,"x"]\ar[dashed,ld] \\
       R\ar[u]&
      \end{tikzcd}
    \end{center}
This means that for all $x:R^n$ such that $P(x)=0$ mod $\epsilon$, there exists a unique $y:R^n$ such that:
\begin{itemize} 
\item We have $x=y$ mod $\epsilon$.
\item We have $P(y)=0$.
\end{itemize}

First we prove existence. For any $b:R^n$ we compute:
\[P(x+\epsilon b) = P(x) + \epsilon\ dP_x(b)\]
We have that $P(x)=0$ mod $\epsilon$, say $P(x) = \epsilon a$. Since $\neg\neg(P(x) = 0)$, we have that $dP_x$ is invertible. Then taking $b = -(dP_x)^{-1}(a)$ gives a lift $y=x+\epsilon b$ such that $P(y) = 0$.

Now we check unicity. Assume $y,y'$ two such lifts, then $y=y'$ mod $\epsilon$ and we have:
\[P(y) = P(y') + dP_{y'}(y-y')\]
and $P(y)=0$ and $P(y')=0$ so that:
\[dP_{y'}(y-y') = 0\]
But $dP_{y'}$ is invertible and we can conclude that $y=y'$.
\end{proof}

\begin{lemma}\label{standard-smooth-is-smooth}
Any standard smooth scheme of dimension $k$ is smooth of dimension $k$. %(\Cref{definition-smooth-dim-n}).
\end{lemma}

\begin{proof}
The fibers of the map:
\[\Spec\big(R[X_1,\cdots,X_n,Y_1,\cdots Y_{k}] / P_1,\cdots,P_n\big) \to \Spec(R[Y_1,\cdots Y_{k}])\]
are standard étale, so the map is étale by \Cref{standard-etale-are-etale}. Since:
\[\Spec(R[Y_1,\cdots Y_{k}]) = \A^k\]
is smooth by \Cref{An-is-smooth}, we can conclude it is smooth using \Cref{smooth-sigma-closed}. 

For the dimension we use \Cref{An-dimension-n} and \Cref{smooth-maps-are-submersions}.
\end{proof}



\subsection{Smooth schemes are locally standard smooth}

\begin{proposition}\label{smooth-are-locally-standard}
A scheme is smooth of dimension $k$ if and only if it has a finite open cover by standard smooth schemes of dimension $k$.
\end{proposition}

\begin{proof}
We can assume the scheme $X$ affine, say of the form:
\[X = \Spec(R[X_1,\cdots,X_m]/P_1,\cdots,P_l)\]

By \Cref{smooth-have-free-tangent}, for any $x:X$ we have that $dP_x$ has finite free kernel of rank $k$. Then by \Cref{rank-equivalent-definitions} we know that $dP_x$ has rank $n=m-k$ for every $x$.

We cover $X$ according to which $n$-minor is invertible, so that up to a rearranging of variables and polynomials and using \Cref{standard-open-in-std-smooth} we can assume that:
\[X = \Spec(R[X_1,\cdots,X_{q},Y_1,\cdots,Y_k]/P_1,\cdots,P_{q},Q_1,\cdots, Q_p)\]
with $q:=n+1$ and where $P_q$ is given by the construction in \Cref{standard-open-in-std-smooth} and we have:
\[d(P,Q)_{x,y} = \begin{pmatrix}
\left(\frac{\partial P}{\partial X}\right)_{x,y} & \left(\frac{\partial P}{\partial Y}\right)_{x,y} \\
\left(\frac{\partial Q}{\partial X}\right)_{x,y} & \left(\frac{\partial Q}{\partial Y}\right)_{x,y} \\
\end{pmatrix}\]
where we used the notation:
\[\left(\frac{\partial P}{\partial X}\right)_{x,y} = \begin{pmatrix}\left(\frac{\partial P_i}{\partial X_j}\right)_{x,y}\end{pmatrix}_{i,j}\]
so that $\frac{\partial P}{\partial X}_{x,y}$ is invertible of size $q$. Moreover by \Cref{rank-bloc-matrix} we get:
\[\left(\frac{\partial Q}{\partial Y}\right)_{x,y} = \left(\frac{\partial Q}{\partial X}\right)_{x,y}\left(\frac{\partial P}{\partial X}\right)_{x,y}^{-1} \left(\frac{\partial P}{\partial Y}\right)_{x,y} \]
which will be useful later.

Now we prove that for any $(x,y):R^{q+k}$ such that $P(x,y)=0$ it is decidable whether
\[Q(x,y)=0 \] 
Using \Cref{square-zero-implies-zero-decidable} this follow from:
\[(Q_1(x,y),\cdots,Q_p(x,y))^2=0 \to (Q_1(x,y),\cdots,Q_p(x,y))=0\]
Assuming $(Q_1(x,y),\cdots,Q_p(x,y))^2=0$, by smoothness there is a dotted lift in:
\begin{center}
      \begin{tikzcd}
        R/(Q_1(x,y),\cdots,Q_l(x,y)) & \Spec(R[X_1,\cdots,X_q,Y_1,\cdots,Y_k]/P_1,\cdots,P_q,Q_1,\cdots, Q_l)\ar[l,swap,"(x{,}y)"] \ar[dotted,ld,"(x{'}{,}y{'})"]\\
       R\ar[u] & \\
      \end{tikzcd}
\end{center}
Let us prove that $Q(x,y) = 0$. Indeed we have $(x,y) \sim_1 (x',y')$ so that we have:
\[P(x,y) = P(x',y')+ \left(\frac{\partial P}{\partial X}\right)_{x',y'}(x-x') + \left(\frac{\partial P}{\partial Y}\right)_{x',y'}(y-y') \]
\[Q(x,y) = Q(x',y')+ \left(\frac{\partial Q}{\partial X}\right)_{x',y'}(x-x') + \left(\frac{\partial Q}{\partial Y}\right)_{x',y'}(y-y') \]
Then we have $P(x,y) = 0$, $P(x',y')=0$ and $Q(x',y') = 0$. From the first equality we get:
\[x-x' =  -\left(\frac{\partial P}{\partial X}\right)_{x',y'}^{-1}\left(\frac{\partial P}{\partial Y}\right)_{x',y'}(y-y')\]
so that from the second we get:
\[Q(x,y) = -\left(\frac{\partial Q}{\partial X}\right)_{x',y'}\left(\frac{\partial P}{\partial X}\right)_{x',y'}^{-1}\left(\frac{\partial P}{\partial Y}\right)_{x',y'}(y-y') + \left(\frac{\partial Q}{\partial Y}\right)_{x',y'}(y-y')\]
so that $Q(x,y)=0$ as we have seen previously that:
\[\left(\frac{\partial Q}{\partial Y}\right)_{x',y'} = \left(\frac{\partial Q}{\partial X}\right)_{x',y'}\left(\frac{\partial P}{\partial X}\right)_{x',y'}^{-1} \left(\frac{\partial P}{\partial Y}\right)_{x',y'} \]
From the decidability of $Q(x,y)=0$ we get that $X$ is an open in $\Spec(R[X_1,\cdots,X_q,Y_1,\cdots,Y_k]/P_1,\cdots,P_q)$
so it is of the form $D(G_1,\cdots,G_r)$, and we have an open cover of our scheme by pieces of the form:
\[\Spec((R[X_1,\cdots,X_q,Y_1,\cdots,Y_k]/P_1,\cdots,P_q)_{G_i})\]
and we can conclude with \Cref{standard-open-in-std-smooth}.
\end{proof}

\begin{corollary}
A scheme is formally étale if and only if it has a cover by standard étale schemes.
\end{corollary}

\begin{proof}
By \Cref{unramified-scheme-characterisation} we know that a scheme is formally étale if and only if it is smooth of dimension $0$. Then we just apply \Cref{smooth-are-locally-standard}.
\end{proof}

% \begin{remark}
% An affine scheme with a cover by standard étale schemes is itself standard étale. TODO
% \end{remark}


\pagebreak
\appendix
\section{Appendix}
\subsection{Rank of matrices}

\begin{definition}
A matrix is said to have rank $\leq n$ if all its $n+1$-minors are zero. It is said to have rank $n$ if it has rank $\leq n$ and does not have rank $\leq n-1$.
\end{definition}

Having a rank is a property of matrices, as a rank function defined on all matrices would allow to e.g. decide if an $r:R$ is invertible.

\begin{lemma}\label{rank-bloc-matrix}
Assume given a matrix $M$ of rank $n$ decomposed into blocks:
\[M = \begin{pmatrix}
P & Q  \\
R & S \\
\end{pmatrix}\]
Such that $P$ is square of size $n$ and invertible. Then we have:
\[S = RP^{-1}Q\]
\end{lemma}

\begin{proof}
By columns manipulation the matrix is equivalent to:
\[M = \begin{pmatrix}
P & Q  \\
0 & S - RP^{-1}Q \\
\end{pmatrix}\]
but equivalent matrices have the same rank so $S=RP^{-1}Q$.
\end{proof}

\begin{lemma}\label{rank-equivalent-definitions}
If a linear map $R^m \to R^n$ given by multiplication with $M$
has finite free kernel of rank $k$, then $M$ has rank $m-k$.
\end{lemma}

\begin{proof}
  Let $a_1,\dots,a_{k}$ be a basis for the kernel of $M$ in $R^m$, which we complete into a basis of $R^m$ via $b_{k+1},\dots,b_m$.
  By completing $Mb_{k+1},\dots, Mb_m$ to a basis of $R^n$, we get a basis where $M$ is written as:
\[\begin{pmatrix}
I_{m-k} & 0  \\
0 & 0 \\
\end{pmatrix}\]
so that $M$ has rank $m-k$.
\end{proof}

%\begin{definition}
%Two matrices $M,N$ are said equivalent if there are invertible matrices $P,Q$ such that $M = PNQ$.
%\end{definition}

%It is clear that equivalent matrices have the same rank.

%\begin{lemma}\label{rank-equivalent-definitions}
%Assume given a matrix:
%\[M : R^m\to R^k\]
%Then the following are equivalent:
%\begin{enumerate}[(i)]
%\item $M$ has rank $n$.
%\item The kernel of $M$ is equivalent to $R^{m-n}$.
%\item The image of $M$ is equivalent to $R^n$.
%\item $M$ is equivalent to the bloc matrix:
%\[\begin{pmatrix}
%I_n & (0)  \\
%(0) & (0) \\
%\end{pmatrix}\]
%\end{enumerate}
%\end{lemma}

%\begin{proof}
%\end{proof}


\printindex

\printbibliography

\end{document}
