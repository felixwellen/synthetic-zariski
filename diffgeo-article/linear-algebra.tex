
\subsection{Modules and infinitesimal disks}
The most basic infinitesimal schemes are the first order neighbourhoods in affine n-space $R^n$. Their algebra of functions is $R^{n+1}$, which is an instance of the more general construction below.

For any $R$-module $M$, there is an $R$-algebra structure on $R\oplus M$ with multiplication given by:
\[(r,m)(r',m') = (rr',rm'+r'm)\]
Algebras of this form are called \emph{square zero extensions} of $R$, since products of the form $(0,m)(0,n)$ are zero.
By this property, for any $R$-linear map $\varphi:M\to N$ between modules $M,N$, the map $\mathrm{id}\oplus \varphi: R\oplus M\to R\oplus N$ is an $R$-algebra homomorphism. In particular, if $M$ is finitely presented, i.e.\ merely the cokernel of some $p:R^n\to R^m$ then $R\oplus M$ is the cokernel of a map between finitely presented algebras and therefore finitely presented as an algebra. 

\begin{definition}
  \index{$\D(M)$}
Given $M$ a finitely presented $R$-module, we define a finitely presented algebra structure on $R\oplus M$ as above and define:
\[\D(M) = \Spec(R\oplus M)\]
This is a pointed scheme by the first projection which we denote $0$
and the construction is functorial by the discussion above.
\end{definition}

We write $\D(n)$ for $\D(R^n)$ so that for example:
\[\D(1) = \Spec(R[X]/(X^2)) = \{\epsilon:R\ |\ \epsilon^2=0\}\]

\begin{definition}\label{derivation-pointwise}
Assume given $M$ a finitely presented $R$-module and $A$ a finitely presented $R$-algebra with $x:\Spec(A)$. An $M$-derivation at $x$ is a morphism of $R$-modules:
\[d:A\to M\]
such that for all $a,b:A$ we have that:
\[d(ab) = a(x)d(b) + b(x)d(a)\]
\end{definition}

\begin{lemma}\label{tangent-are-derivation}
Assume given $M$ a finitely presented module and $A$ a finitely presented algebra with $x:\Spec(A)$. Pointed maps:
\[\D(M)\to_\pt (\Spec(A),x)\] 
correspond to $M$-derivations at $x$.
\end{lemma}

\begin{proof}
Such a pointed map correponds to an algebra map:
\[f : A\to R\oplus M\]
where the composite with the first projection is $x$. This means that, for some module map $d:A\to M$ we have:
\[f(a) = (a(x),d(a))\]
We can immediately see that $f$ being a map of $R$-algebras is equivalent to $d$ being an $M$-derivation at $x$.
\end{proof}

\begin{lemma}\label{equivalence-module-infinitesimal}
Let $M$, $N$ be finitely presented modules. Then linear maps from $M$ to $N$ correspond to
pointed maps from $\D(N)$ to $\D(M)$. 
%Explicitly, a linear map $g : M \to N$ corresponds to the pointed map $f \mapsto m \mapsto f(g(m))$.
\end{lemma}

\begin{proof}
By \Cref{tangent-are-derivation} such a pointed map corresponds to an $N$-derivation at $0:\D(M)$.

Such a derivation is a morphism of modules:
\[d:R\oplus M\to N\]
such that for all $(r,m),(r',m'):R\oplus M$ we have that:
\[d(rr',rm'+r'm) = rd(r',m')+r'd(r,m)\]
This implies $d(r,0) = 0$ for all $r : R$, so such a map is entirely determined by the linear map $m\mapsto d(0,m)$. Conversly given a linear map $f:M\to N$ we can check that $(r,m)\mapsto f(m)$ is such a derivation, giving the correspondence.
\end{proof}


\subsection{Tangent spaces}
  \label{remark-sym-dual}

In traditional algebraic geometry, the tangent sheaf is defined as the dual of the cotangent sheaf which is given Zariski locally by the universal module of derivations. We could copy this approach synthetically, but contrary to the traditional picture, we can also define the tangent bundle directly and dualize to get the usual cotangent bundle. This mismatch with the traditional theory comes from the fact that the traditional dualization of $\mathcal O_X$-module sheaves is not the same as our dualization of $R$-module bundles.

We start with the definition of tangent spaces which is also used in synthetic differential geometry:

\begin{definition}
Let $X$ be a type with $x:X$, then we define the \notion{tangent space} $T_x(X)$ \index{$T_x(X)$} of $X$ at $x$ by:
\[T_x(X) = \{t:\D(1)\to X\ |\ t(0)=x\}\]
\end{definition}

\begin{definition}
Given $f:X\to Y$ and $x:X$ we have a map\index{$df_x$}:
\[df_x : T_x(X)\to T_{f(x)}(Y)\]
induced by post-composition.
\end{definition}

\begin{lemma}\label{An-dimension-n}
For all $x:R^n$ we have $T_x(R^n) = R^n$.
\end{lemma}

\begin{proof}
Since $R^n$ is homogeneous we can assume $x=0$. By \Cref{tangent-are-derivation} we know that $T_0(R^n)$ corresponds to the type of linear maps
\[R[X_1,\cdots,X_n] \to R\]
such that for all $P,Q$ we have:
\[d(PQ) = P(0)dQ + Q(0)dP\]
which is equivalent to $d(1) = 0$ and $d(X_iX_j) = 0$, so any such map is determined by its image on the $X_i$ so it is equivalent to an element of $R^n$.
\end{proof}

\begin{lemma}\label{from-D1-to-D2}
Given a scheme $X$ with $x:X$ and $v,w:T_x(X)$, there exists a unique:
\[\psi_{v,w} : \D(2)\to_\pt X\]
such that for all $\epsilon:\D(1)$ we have that:
\[\psi_{v,w}(\epsilon,0) = v(\epsilon)\]
\[\psi_{v,w}(0,\epsilon) = w(\epsilon)\]
\end{lemma}

\begin{proof}
We can assume $X$ is affine. Then $\D(2)\to_\pt X$ is equivalent to the type of $R^2$-derivations at $x$, but giving an $M\oplus N$-derivation is equivalent to giving an $M$-derivation and an $N$-derivation. Checking the equalities is a routine computation.
\end{proof}

\begin{lemma}
For any scheme $X$ and $x:X$, we have that $T_x(X)$ is a module.
\end{lemma}

\begin{proof}
We could give a conceptual proof similar to \cite[Theorem 4.2.19]{david-orbifolds}. Instead we give a more explicit proof with less technical prerequisites.

We define scalar multiplication by sending $v$ to $t\mapsto v(rt)$. Then for addition of $v,w:T_x(X)$, we define:
\[(v+w)(\epsilon) = \psi_{v,w}(\epsilon,\epsilon)\]
where $\psi_{v,w}$ is defined in \Cref{from-D1-to-D2}. We omit checking that this is a module structure.
\end{proof}

\begin{lemma}
For $f:X\to Y$ a map between schemes, for all $x:X$ the map $df_x$ is a map of $R$-modules.
\end{lemma}

\begin{proof}
Commutation with scalar multiplication is immediate. Commutation with addition comes by applying uniqueness from \Cref{from-D1-to-D2} to get:
\[f\circ \psi_{v,w} = \psi_{f\circ v,f\circ w}\]
\end{proof}

\begin{lemma}\label{kernel-is-tangent-of-fibers}
For any map $f:X\to Y$ and $x:X$, we have that:
\[
\mathrm{Ker}(df_x) = T_{(x,\refl_{f(x)})}(\mathrm{fib}_f(f(x)))
\]
\end{lemma}

\begin{proof}
Indeed we have that $\mathrm{fib}_f(f(x))$ pointed by $(x,\refl_{f(x)})$ is the pullback of:
\[
(X,x) \to (Y,f(x)) \leftarrow (1,*)
\]
in pointed types, and we conclude by exponentiating with $(\mathbb{D}(1),0)$.
\end{proof}

\begin{lemma}\label{tangent-finite-copresented}
Let $X$ be a scheme with $x : X$. Then $T_x(X)$ is a finitely
copresented $R$-module.
\end{lemma}

\begin{proof}
We can assume $X$ affine. For some map $P: R^m\to R^n$ we have $X=\mathrm{fib}_P(0)$.
By applying \Cref{kernel-is-tangent-of-fibers} we know that $T_x(X)$ is the kernel of $dP_x : T_x(R^m)\to T_0(R^n)$ for all $x:X$.
We conclude by \Cref{An-dimension-n}.
\end{proof}

\begin{corollary}
  \label{tangent-bundle-scheme}
  Let $X$ be a scheme, then the tangent bundle $X^{\mathbb{D}(1)}$ is a scheme.
\end{corollary}

\begin{proof}
  We give two proofs, the first uses \Cref{tangent-finite-copresented} and the second is a direct computation:
  \begin{enumerate}[(i)]
  \item Any finitely copresented modules is a scheme, indeed it is the set of common zeros of linear functions between finite free modules.
    So by \Cref{tangent-finite-copresented}, all tangent spaces $T_x(X)$ are schemes and:
    \[
      X^{\mathbb{D}(1)}=\sum_{x:X}T_x(X)
    \]
    is a dependent sum of schemes and therefore a scheme.
  \item Let $X$ be covered by open affine $U_1,\dots,U_n$ then $U_1^{\mathbb{D}(1)},\dots,U_n^{\mathbb{D}(1)}$ is an open cover of $X^{\mathbb{D}(1)}$. Indeed given $f:X^{\mathbb{D}(1)}$, by double negation stability of opens we have that $f\in U_i^{\mathbb{D}(1)}$ if and only if $f(0)\in U_i$.
    So we conclude by showing that for any affine $Y=\Spec R[X_1,\dots,X_n]/(f_1,\dots,f_l)$ the tangent bundle $Y^{\mathbb{D}(1)}$ is affine
    by direct computation:
    \begin{align*}
      Y^{\mathbb{D}(1)}&=\Hom_{\Alg{R}}(R[X_1,\dots,X_n]/(f_1,\dots,f_l), R\oplus \epsilon R) \\
                       &= \{(y_1,\dots,y_n):R\oplus \epsilon R \mid \forall i.\, f_i(y_1,\dots,y_n)=0\} \\
                       &= \{(x_1,\dots,x_n,d_1,\dots,d_n):R^{2n} \mid \forall i.\, f_i(x_1,\dots,x_n)=0\text{ and } \sum_j d_j\frac{\partial f_i}{\partial X_j}(x_1,\dots,x_n) =0\} 
    \end{align*}
  \end{enumerate}
\end{proof}

Now we want to define cotangent spaces.

\begin{definition}
  For $M$ an $R$-module, we denote its dual $\Hom_{R}(M,R)$ by $M^\star$.
\end{definition}

\begin{definition}
For $X$ a type with $x : X$, the \notion{cotangent space} of $X$ at $x$ is the dual $T_x^\star(X)$ of the tangent space $T_x(X)$.
\end{definition}

If $X$ is a scheme, then by \cref{tangent-finite-copresented} the cotangent spaces of $X$ are finitely presented.
We will not use the following definition and remark in the rest of this article, but we included them to show the connection with the traditional theory (see \cite[p. 172]{Hartshorne} or \cite[p. 573]{vakil}).

\begin{definition}
For $A$ an $R$-module, there is a universal derivation $d : A \to \Omega_{A/R}$. Elements of $\Omega_{A/R}$ are called \notion{Kähler differentials}.
\end{definition}

More precisely, $\Omega_{A/R}$ is generated as an $A$-module by symbols
$df$ for $f : A$, subject to the relations $d(r\cdot f) = r \cdot df$ for $r : R$ and
$d(fg) = f \cdot dg + g \cdot df$.
It can be seen that if $A$ is finitely presented as an $R$-algebra,
then $\Omega_{A/R}$ is finitely presented as an $A$-module.
Traditionally, the sheaf corresponding to $\Omega_{A/R}$ is the cotangent bundle.
Synthetically, it is enough to show this pointwise on $\Spec(A)$ by \cite[Theorem 8.2.3]{draft}.
To apply this theorem, we first turn $\Omega_{A/R}$ into an $R$-module bundle on $\Spec(A)$ by sending $x:\Spec A$ to $\Omega_{A/R,x}$ the type of $R$-derivations at $x$, as defined in \Cref{derivation-pointwise}. This agrees with tensoring $\Omega_{A/R}$ with $R$ using the evaluation at $x$, which is the general construction used in \cite[Theorem 8.2.3]{draft}.

\begin{remark}
For all $x:\Spec A$, we have $\Omega_{A/R,x} = T_x^\star(X)$ and therefore $\Omega_{A/R} = \prod_{x:\Spec A} T_x^\star(X)$.
\end{remark}

\begin{proof}
We need to show that for $x : X$,
   we have an isomorphism of $R$-modules:
   \[ T^\star_x(X) = \Omega_{A/R,x}=\Omega_{A/R} \otimes_A R\]
By \Cref{tangent-are-derivation} the tangent space $T_x(X)$ corresponds to derivations
$A \to R$, where the $A$-module structure on $R$ is obtained by evaluating at $x$.
By the universal property of Kähler differentials, these derivations correspond to $A$-module maps $\Omega_{A/R} \to R$, or equivalently to elements in $(\Omega_{A/R}\otimes_AR)^\star$. In \Cref{double-dual-identity}, we will see that $M^{\star\star} = M$ for finitely presented $R$-modules $M$, so we can conclude by dualizing.
\end{proof}

\subsection{Infinitesimal neighbourhoods}

\begin{definition}
Let $X$ be a set with $x:X$. The \notion{first order neighborhood} $N_1(x)$ \index{$N_1(x)$}is defined as the set of $y:X$ such that there exists a finitely generated ideal $I\subseteq R$ with $I^{2}=0$ and:
\[I=0 \to x=y\]
\end{definition}

\begin{lemma}\label{first-order-square-zero}
Assume $x,y:R^n$, then $x\in N_1(y)$ if and only if the ideal generated by the $x_i-y_i$ squares to zero.
\end{lemma}

\begin{proof}
Let us denote $I$ the ideal generated by the $x_i-y_i$ so that $x=y$ if and only if $I=0$. 

If $I^2=0$ then it is clear that $y\in N_1(x)$.

Conversely if $y\in N_1(x)$ then there is $J$ such that $J^2=0$ and $J=0 \to I=0$. Then by duality we have that $I\subset J$ so that $I^2=0$.
\end{proof}

\begin{lemma}\label{first-order-schemes}
Let $X$ be a scheme with $x:X$. Then $N_1(x)$ is an affine scheme. 
\end{lemma}

\begin{proof}
If $x\in U$ open in $X$, we have that $N_1(x)\subset U$ so that we can assume $X$ affine. This means $X$ is a closed subscheme $C\subset R^n$. Then by \Cref{first-order-square-zero}, we have that $N_1(x)$ is the type of $y:R^n$ such that $y\in C$ and for all $i,j$ we have that $(x_i-y_i)(x_j-y_j) = 0$, which is a closed subset of $C$ so it is an affine scheme.
\end{proof}

\begin{definition}
A pointed scheme $(X,*)$ is called a \notion{first order (infinitesimal) disk} if for all $x:X$ we have $x\in N_1(*)$.
\end{definition}

\begin{lemma}
  \label{N1-functor}
  $N_1$ extends to a functor from pointed schemes to first order disks.
\end{lemma}

\begin{proof}
It is clear that $N_1$ is functorial, as it is clear that $y\in N_1(x)$ implies $f(y)\in N_1(f(x))$ from the definition of $N_1$. Now we just need to check that for $X$ a scheme with $x:X$, we have that $(N_1(x),x)$ is a first order disk. Then $N_1(x)$ is a disk by \Cref{first-order-schemes}, and it is clear that the first order neighbourhood of $x$ in $N_1(x))$ is the whole type $N_1(x)$.
\end{proof}

\begin{lemma}\label{disk-are-infinitesimal}
A pointed scheme $(X,*)$ is a first order disk if and only if there exists a finitely presented module $M$ such that:
\[(X,*) = (\D(M),0)\]
\end{lemma}

\begin{proof}
First we check that for all $M$ finitely presented and $y:\D(M)$ we have that $y\in N_1(0)$. Let $m_1,\cdots, m_k$ be generators of $M$, then consider $d:M\to R$ induced by $y$, then $y=0$ if and only if $d=0$ and for all $i,j$ we have that:
\[d(m_i)d(m_j) = 0\]
This means that $I = (d(m_1),\cdots,d(m_k))$ has square $0$ and $I=0$ implies $y=0$ so that $y\in N_1(0)$.

For the converse we assume $X$ a first order disk, by \Cref{first-order-schemes} we have that $X$ is affine and pointed, up to translation we can assume $X$ is a closed subset $X\subset R^n$ pointed by $0$. Since $X$ is a first order disk we have that $X\subset N_1(0)$ and by \Cref{first-order-square-zero} we have $N_1(0) = \D(R^n)$.

This means there is a finitely generated ideal $J$ in $R\oplus R^n$ such that $X=\Spec(R\oplus R^n / J)$.
But $0$ corresponds to the first projection from $R\oplus R^n$, so that $0\in X$ means that if $(x,y)\in J$ then $x=0$, so that $J$ corresponds uniquely to a finitely generated sub-module $K$ of $R^n$ and:
\[X = \Spec(R\oplus (R^n/K)) = \D(R^n/K)\] 
\end{proof}

Now we want to study the duality between finitely presented and finitely copresented modules. While it is clear that the dual of a finite presentation yields a finite copresentation, the reverse is not true in general, but we will show  in \Cref{dual-of-fcop-fp} that it is a consequence of the duality axiom. First we need the following two extension results.

\begin{lemma}
  \label{extend-from-kernel}
  Let $M\subseteq R^n$ be the kernel of a linear map between finite free $R$-modules.
  Then any linear map $M\to R$ can be extended to $R^n$.
\end{lemma}

\begin{proof}
  First note that $M$ is affine of the form $\Spec(R[X_1,\dots,X_n]/(l_1,\dots,l_m))$ with $l_i$ linear.
  Let $L:M\to R$ be linear and $P:R^n\to R$ be given by taking a preimage under the quotient map $R[X_1,\dots,X_n]\to R[X_1,\dots,X_n]/(l_1,\dots,l_m)$,
  so we have $P_{\vert M}=L$.
  Let $P=\sum_{\sigma:\N^{\{1,\dots,n\}}}a_\sigma X_1^{\sigma(1)}\cdots X_n^{\sigma(n)}$.
  Now we can conclude by showing that the linear part of $P$
  \[
    K\colonequiv \sum_{\sigma:\N^{\{1,\dots,n\}}, \sum \sigma =1}a_\sigma X_1^{\sigma(1)}\cdots X_n^{\sigma(n)}
  \]
  extends $L$ as well, i.e.\ we will see $K_{\vert M}=L$.
  
  For all $x:M$ and $\lambda : R$ we have $L(\lambda x)=\lambda L(x)$ and therefore
  \[
    \sum_{\sigma:\N^{\{1,\dots,n\}}}\lambda^{\sum\sigma} a_\sigma x_1^{\sigma(1)}\cdots x_n^{\sigma(n)}=\lambda \sum_{\sigma:\N^{\{1,\dots,n\}}} a_\sigma x_1^{\sigma(1)}\cdots x_n^{\sigma(n)}
  \]
  By comparing coefficients as polynomials in $\lambda$, we have $\sum_{\sigma:\N^{\{1,\dots,n\}}, \sum \sigma \neq 1}a_\sigma x_1^{\sigma(1)}\cdots x_n^{\sigma(n)}=0$,
  which shows $K_{\vert M}=P_{\vert M}=L$.
\end{proof}

\begin{lemma}
  \label{extend-from-image}
  Let $\varphi:R^n\to R^m$ be $R$-linear, then any linear map $\mathrm{im}(\varphi)\to R$ on the image of $\varphi$ can be extended to $R^m$.
\end{lemma}

\begin{proof}\footnote{This proof is due to Thierry Coquand.}
  Let $(a_{i,j})$ be the coefficients of the matrix representing  $\varphi$ with respect to the standard basis, and let us denote the column $(a_{i,j})_{1\leq i\leq m}$ by $A_j$.
  Then the image of $\varphi$ is generated by these columns:
  \[
    \mathrm{im}(\varphi)=\left\{\Sigma_{j=1}^n x_jA_j\mid \forall_j.\, x_j:R\right\}
  \]
  Let $L:\mathrm{im}(\varphi)\to R$ be $R$-linear and $l_j\colonequiv L(A_j)$.
  Applying $L$ to a general element of $\mathrm{im}(\varphi)$ and using linearity yields the following implication:
  \[
    \sum_{j=1}^n x_jA_j = 0 \Rightarrow \sum_{j=1}^n x_jl_j = 0
  \]
  The left side being 0 means that $m$ linear polynomials $P_i(x_1,\dots,x_n)=\sum_{j=1}^n x_ja_{ij} $ vanish simultaneously.
  Let $Q(x_1,\dots,x_n)$ be the linear polynomial on the right side of the implication.
  Then by duality  the implication induces an inclusion of ideals $(Q)\subseteq (P_1,\dots,P_m)$ in $R[X_1,\dots,X_n]$.
  So there $b_i:R[X_1,\dots,X_n]$ such that
  \[
     Q = \sum_{i=1}^m b_iP_i
   \]
   By comparing coefficients it is clear that the $b_i$ can be chosen to be in $R$, which we now assume.
   
   We define a $R$-linear map $K:R^m\to R$ by $(y_1,\dots,y_m)\mapsto\sum_{i=1}^m b_iy_i$.
   $K$ extends $L$:
   \begin{eqnarray*}
     K\left(\sum_{j=1}^n x_jA_j\right) &=&\sum_{i=1}^m b_i\sum_{j=1}^n x_ja_{ij} \\
     &=& \sum_{i=1}^m b_iP_i(x_1,\dots,x_n) \\
     &=& Q(x_1,\dots,x_n) \\
     &=& \sum_{j=1}^n x_jl_j \\
     &=& L\left(\sum_{j=1}^n x_jA_j\right)
   \end{eqnarray*}
\end{proof}

\begin{lemma}
  \label{dual-of-fcop-fp}
  Let $M$ be finitely copresented, i.e.\ let there be an exact sequence
  \begin{center}
    \begin{tikzcd}
      M\ar[r,hook, "\varphi"] & R^n\ar[r,"P"] & R^m
    \end{tikzcd}
  \end{center}
  Then the dual of this sequence is exact as well and $\varphi^\star$ is surjective.
  In particular, $M^\star$ is finitely presented.
\end{lemma}

\begin{proof}
  Surjectivity of $\varphi^\star$ follows from \Cref{extend-from-kernel}.
  Linear maps $R^n\to R$ which vanish on $M$ factor through the image of $P$, so exactness at the middle of the dual sequence follows from \Cref{extend-from-image}.
\end{proof}

\begin{corollary}\label{double-dual-identity}
For any module $M$ finitely presented or finitely copresented, we have that $M^{\star\star}=M$.
\end{corollary}

\begin{lemma}\label{equivalence-modules-disks}
  The functor $M\mapsto \D(M^\star)$ from finitely copresented modules to first order disks is an equivalence with inverse $(X,x)\mapsto T_x(X)$.
\end{lemma}

\begin{proof}
It is fully faithful by \Cref{equivalence-module-infinitesimal} and essentially surjective by \Cref{disk-are-infinitesimal}. To check for the inverse it is enough to check that $T_0(\D(M^\star)) = M$. But by \Cref{equivalence-module-infinitesimal} we have that: 
\[T_0(\D(M^\star)) = \left(\D(1)\to_\pt \D(M^{\star})\right) = M^{\star\star}\] 
and we conclude by \Cref{double-dual-identity}.
\end{proof}

\begin{lemma}\label{duality-infinitesimal-tangent}
Let $X$ be a scheme with $x:X$, then we have that $N_1(x) = \D(T_x(X)^\star)$.
\end{lemma}

\begin{proof}
By \Cref{N1-functor} we have that $(N_1(x),x)$ is a first order disk. By \Cref{equivalence-modules-disks} it is enough to check that $T_x(N_1(x)) = T_0(\D(T_x(X)^\star))$. 

It is immediate that any map $f:\D(1)\to X$ uniquely factors through $N_1(f(0))$ so that $T_x(N_1(x)) = T_x(X)$, and we have that $T_0(\D(T_x(X)^\star)) = T_x(X)$ by \Cref{equivalence-modules-disks}.
\end{proof}


\subsection{Projectivity of finitely copresented modules}

Finitely copresented $R$-modules are projective objects in the category of finitely copresented $R$-modules, which means that all surjections between finitely copresented $R$-modules split.
%The results in this section will not be used in the rest of the article.

\begin{lemma}\label{tangent-copresented-modules}
Let $M$ be a finitely copresented module, then we have that $T_0(M) = M$.
\end{lemma}

\begin{proof}
We have that $M$ is the kernel of a linear map $P:R^m\to R^n$. By \Cref{kernel-is-tangent-of-fibers} we have that $T_0(M)$ is the kernel of:
\[dP_0:T_0(R^m)\to T_0(R^n)\]
but by \Cref{An-dimension-n} this is a map from $R^m$ to $R^n$, we omit the verification that $dP_0 = P$.
\end{proof}

\begin{lemma}
Assume given $M,N$ finitely copresented modules with a map $f:M\to N$. The following are equivalent:
\begin{enumerate}[(i)]
\item $f$ is surjective.
\item $f$ merely has a section.
\item The pointed map $\D(M^\star) \to \D(N^\star)$ corresponding to $f$ merely has a pointed section.
\end{enumerate}
\end{lemma}

\begin{proof}
By \Cref{equivalence-modules-disks} we know that (i) is equivalent to (ii). It is clear that (ii) implies (i).

Let us assume (i) and prove (iii). By \Cref{duality-infinitesimal-tangent} and \Cref{tangent-copresented-modules} we know that $\D(M^*)$ is the first order neighbourhood of $0$ in $M$, so that we have a commutative diagram:
\begin{center}
\begin{tikzcd}
\D(M^\star)\ar[r,hook]\ar[d] & M\ar[d,"f"]\\
\D(N^\star)\ar[r,hook,swap,"i"] & N\\
\end{tikzcd}
\end{center}
Since $\D(N^*)$ has choice and $f$ is surjective there is $g:\D(N^\star)\to M$ such that $f\circ g = i$. We know that $f(g(0))$, so by considering $g'=g-g(0)$ we have that $f\circ g' = i$ and $g'(0)$.
Then we can factor $g'$ through $\D(M^\star)$ as $N_1$ is functorial by \Cref{N1-functor}. This gives us a pointed section of the map $\D(M^\star) \to \D(N^\star)$.
\end{proof}

\begin{corollary}
Any finitely copresented module is projective in the category of finitely copresented modules.
\end{corollary}







