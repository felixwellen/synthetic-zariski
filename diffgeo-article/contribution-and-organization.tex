We give a novel synthetic definition of formally étale, smooth and unramified types, using what we call \emph{closed dense propositions} (\Cref{def-etale-closed-dense}). We then define étale (resp. smooth, unramified) schemes simply as schemes that happens to be formally étale (resp. smooth, unramified) when seen as types. Étale (resp. smooth, unramified) maps between schemes are defined as maps with étale (resp. smooth, unramified) fibers.
This is an instance of a general phenomenon in synthetic reasoning: concepts which are usually defined locally can be defined fiberwise. %or on the level of propositions. DON'T KNOW WHAT IS MEANT?

While describing the infinitesimal structure of schemes in section 2, we also point out a curious discovery: there is a duality between finitely presented modules and finitely copresented modules over the internal base ring $R$ (\Cref{dual-of-fcop-fp} and \Cref{double-dual-identity}).
The latter notion of finitely copresented modules is not very prominent in algebra, but appears naturally in the study of tangent spaces of schemes (\Cref{tangent-finite-copresented}).

We show that the new definitions agree with a straightforward translations of the classical concepts (\Cref{connection-to-ega-definition}) and provide some characterizations using tangent spaces:
a map between schemes is unramified if and only if it induces injections on tangent spaces (\Cref{unramified-map-characterisation}), and a map between smooth schemes is étale (resp. smooth) if and only if it induces isomorphisms (resp. surjections) on tangent spaces (\Cref{etale-schemes-iff-local-iso,smooth-schemes-iff-submersion}).

Finally, we show that unramified, étale and smooth schemes can be described very concretely in the expected way, via conditions on the polynomials locally describing such schemes (\Cref{unramified-iff-locally-std-unramified} and \Cref{standard-etale-are-etale,standard-smooth-is-smooth}). An important intermediate result for the characterization of smooth schemes is that their tangent spaces are finite free $R$-modules (\Cref{smooth-have-free-tangent}).
