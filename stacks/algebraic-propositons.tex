\subsection{Definition}

\begin{definition}
A Deligne-Mumford (resp. Artin) proposition is a Deligne-Mumford (resp. Artin) $(-1)$-stack.
\end{definition}

\begin{remark}
A proposition is Deligne-Mumford (resp. Artin) if and only is it has an étale (resp. smooth) atlas.
\end{remark}

We suspect any Artin propositon is in fact Deligne-Mumford.

\subsection{Fundamental theorem for algebraic propositions}

\begin{proposition}
A proposition is Deligne-Mumford (resp. Artin) if and only if it is merely of the form:
\[\propTrunc{\Spec(A)}\]
where $\propTrunc{\Spec(A)}$ implies that $\Spec(A)$ is formally étale (resp. formally smooth).
\end{proposition}

\begin{proof}
Let $P$ be such a proposition, then we merely have an étale (resp. smooth) atlas:
\[\Spec(A)\to P\]
Since this map is surjective it induces an equivalence:
\[\propTrunc{\Spec(A)} \simeq P\]
and the fact that:
\[\Spec(A)\to \propTrunc{\Spec(A)}\]
is étale (resp. smooth) is precisely the given condition. The converse if clear.
\end{proof}

\subsection{Examples}

\begin{lemma}
Any propositional scheme is a Deligne-Mumford proposition.
\end{lemma}

\begin{lemma}
Any étale (resp. smooth) scheme $X$, the type $\propTrunc{X}$ is a Deligne-Mumford (resp. Artin) proposition.
\end{lemma}

\begin{proof}
Immediate using the Zariski cover of $X$ by an étale (resp. smooth) affine scheme.
\end{proof}

\begin{lemma}
Any fomally étale (resp. formally smooth) Deligne-Mumford $\infty$-stack $X$, the type $\propTrunc{X}$ is a Deligne-Mumford (resp. Artin) proposition.
\end{lemma}

\subsection{Schemes are étale sheaves (obsolete)}

\begin{lemma}\label{constant-open-affine}
Assume given $A$ an f.p. algebra, then for any $h_1,\cdots,h_n:A$ such that the open $D(h_1,\cdots,h_n)$ is constant, for any $x:\Spec(A)$ we have that:
\[x\in D(h_1,\cdots,h_n) \leftrightarrow \exists i. h_i\ \mathrm{not\ nilpotent}\]
\end{lemma}

\begin{proof}
TODO
\end{proof}

\begin{lemma}
Any formally étale type $X$ is reduced, meaning that any closed dense embedding into $X$ not not has a section.
\end{lemma}

\begin{proof}
It is immediate that is actually has a section.
\end{proof}

\begin{corollary}\label{etale-affine-reduced}
For any affine étale scheme $\Spec(A)$we have that:
\[h\ \mathrm{nilpotent} \leftrightarrow \neg\neg h=0\]
\end{corollary}

\begin{proof}
The reverse implication is always true. If $h$ is nilpotent then for all $x:\Spec(A)$ is nilpotent, so that:
\[\prod_{x:\Spec(A)} \neg\neg h(x)=0\]
But then we have that $V(h)$ is a closed dense embedding, so it not not has a section, meaning $\neg\neg h = 0$.
\end{proof}

\subsection{Schemes are fppf sheaves}

This section should be moved to some file on sheaves at some point.

\begin{definition}
A type $X$ is called an fppf sheaf if for all monic polynomial $g:R[X]$, the type $X$ is $\propTrunc{\Spec(R[X]/g)}$-local.
\end{definition}

\begin{lemma}\label{fppf-subcanonical}
The type $R$ is $\propTrunc{\Spec(R[X]/g)}$-local for any monic polynomial $g$. 
\end{lemma}

\begin{proof}
See section on sheaves in diffgeo.
\end{proof}

\begin{lemma}\label{scheme-are-sheaf-from-affine}
Assume given a type $P$ such that:
\begin{itemize}
\item The type $R$ is $P$-local.
\item The type of open propositions is $P$-smooth.
\end{itemize}
Then any scheme is $P$-local.
\end{lemma}

\begin{proof}
Since $R$ is $P$-local we see that all affine schemes are $P$-local through stability under dependent products and identity types.

We check that any scheme $X$ is $P$-smooth. Assume given constant map:
\[f:P\to X\]
Take $(U_i)_{i:I}$ a finite cover of $X$ by affine scheme. Then for any $i:I$ we have that $f^{-1}(U_i)$ is a constant open in $P$, so since the type of open is $P$-smooth, we merely have an open proposition $V_i$ such that for all $x:P$, we have:
\[(x\in f^{-1}(U_i) )\leftrightarrow V_i\]
Since the $f^{-1}(U_i)$ cover $P$, we have that:
\[P\to \lor_{i:I} V_i\]
But open propositions are affine schemes and affine schemes are $P$-local, so we have:
\[ \lor_{i:I} V_i\]
Assume $k:I$ such that $V_k$ holds. Then $f^{-1}(U_k) = P$ and the map $f$ factors through the affine scheme $U_k$. Since affine schemes are $P$-local, we merely have a lift for $f$.

Now we conclude that any scheme is $P$-local by proving that its identity types are $P$-local. Indeed they are propositional schemes, so they are $P$-smooth propositions, so they are $P$-local.
\end{proof}

\begin{lemma}\label{roots-monic-proper}
For any monic $g:R[X]$, we have that $\Spec(R[X]/g)$ is compact, meaning that for any open $U$ in $\Spec(R[X]/g)$ the proposition:
\[\prod_{x:\Spec(R[X]/g)}U(x)\]
is open.
\end{lemma}

\begin{proof}
Assume that:
\[g=X^n+a_{n-1}X^{n-1}+\cdots+a_0\]
Then we consider the homogeneous polynomial:
\[f(X,Y) = X^n + a_{n-1}X^{n-1}Y+\cdots+a_0Y^n\]
We prove that:
\[\sum_{[x,y]:\bP^1}f(x,y) = 0\]
is equivalent to $\Spec(R[X]/g)$. Indeed for any $x,y:R$ such that $f(x,y)=0$, we have that $x\not=0$ implies $y\not=0$, so that $(x,y)\not=0$ implies $y\not=0$. Then:
\[\sum_{[x,y]:\bP^1}f(x,y) = 0\]
is equivalent to:
\[\sum_{x:R} f(x,1)=0\]
which is the type of roots of $g$. 

Now we conclude using the fact that 
\[\sum_{[x,y]:\bP^1}f(x,y) = 0\]
is closed in the compact scheme $\bP^1$, so that it is compact.
\end{proof}

\begin{proposition}\label{schemes-are-fppf-sheaves}
Any scheme is an fppf sheaf.
\end{proposition}

\begin{proof}
Assume given $g:R[X]$ monic, by \cref{scheme-are-sheaf-from-affine} it is enough to prove that $R$ is $\propTrunc{\Spec(R[X]/g)}$-local (this is \cref{fppf-subcanonical}) and that the type of open propositions is $\propTrunc{\Spec(R[X]/g)}$-smooth. 

Assume given a constant open $D(h_1,\cdots,h_n)$ in $\Spec(R[X]/g)$. Then for any $x:\Spec(R[x]/g)$ we have that:
\[x\in D(h_1,\cdots,h_n) \leftrightarrow \prod_{y:\Spec(R[x]/g)} y\in D(h_1,\cdots,h_n)\]
because the open $D(h_1,\cdots,h_n)$ is constant. But the right hand side is open by \cref{roots-monic-proper}, so we indeed have a lift.
\end{proof}

\subsection{Not all Deligne-Mumford propositions are schemes}

\begin{lemma}\label{no-roots-for-any-p}
If the map:
\[\psi:\A^1\to \A^1\]
\[\psi(x)=x^p\] 
is surjective on an open $U$, then this open is empty.
\end{lemma}

\begin{proof}
Assume an open $U\subset \A^1$ such that:
\[\psi: \psi^{-1}(U) \to U\]
is surjective. Assume $a\in U$, by Zarsiki-local choice we have $f,g:R[X]$ such that $a\in D(g)$ and we have:
\[\frac{f}{g^n}\]
inverse to $\psi$. Since $g(a)$ is invertible, we have that $g$ is regular, so that:
\[\frac{f^p}{g^{pn}} = X\]
implies that:
\[f^p = g^{pn}X\]
By induction we prove that all the coefficients of $f$ and $g$ are nilpotent, which contradicts $g(a)\not=0$.
\end{proof}

\begin{proposition}
Not all Deligne-Mumford propositions are schemes.
\end{proposition}

\begin{proof}
We have that $p\not=0$ for some prime $p$, because locality of the ring implies that $2\not=0$ or $3\not=0$. Then for all $a:R$ such that $a\not=0$, we consider the étale affine scheme:
\[E_a = \Spec(R[X]/X^p-a)\]
We see that the propositions $\propTrunc{E_a}$ are Deligne-Mumford. We assume that they are schemes and reach a contradiction. Indeed then $\propTrunc{E_a}$ would be an fppf sheaf by \cref{schemes-are-fppf-sheaves}, so it would be $\propTrunc{E_a}$-local, so it would be inhabited. This means that thet map:
\[\psi:R\to R\]
\[\psi(x)=x^p\]
would be surjective on $R^\times$. This contradicts \cref{no-roots-for-any-p}.
\end{proof}

A natural question to ask at this point is wether any Deligne-Mumford proposition that is an fppf sheaf is a scheme?
