\subsection{Definition}

\begin{definition}
A Deligne-Mumford (resp. Artin) proposition is a Deligne-Mumford (resp. Artin) $-1$-stack.
\end{definition}

\begin{remark}
A proposition is Deligne-Mumford (resp. Artin) if and only is it has an étale (resp. smooth) atlas.
\end{remark}

We suspect any Artin propositon is in fact Deligne-Mumford.

\subsection{Fundamental theorem for algebraic propositions}

\begin{proposition}
A proposition is Deligne-Mumford (resp. Artin) if and only if it is merely of the form:
\[\propTrunc{\Spec(A)}\]
where $\propTrunc{\Spec(A)}$ implies that $\Spec(A)$ is formally étale (resp. formally smooth).
\end{proposition}

\begin{proof}
Let $P$ be such a proposition, then we merely have an étale (resp. smooth) atlas:
\[\Spec(A)\to P\]
Since this map is surjective it induces an equivalence:
\[\propTrunc{\Spec(A)} \simeq P\]
and the fact that:
\[\Spec(A)\to \propTrunc{\Spec(A)}\]
is étale (resp. smooth) is precisely the given condition.
\end{proof}

\subsection{Examples}

\begin{remark}
Any propositional scheme is a Deligne-Mumford proposition.
\end{remark}

\begin{lemma}
Any étale (resp. smooth) scheme $X$ the type $\propTrunc{X}$ is a Deligne-Mumford (resp. Artin) proposition.
\end{lemma}

\begin{proof}
Immediate using the Zariski cover of $X$ by an étale (resp. smooth) affine scheme.
\end{proof}

\subsection{Schemes are étale sheaves}

This section should be moved to some file on sheaves at some point.

\begin{definition}
A type $X$ is called an étale sheaf if for all monic separable polynomial $g:R[X]$, the type $X$ is $\propTrunc{\Spec(R[X]/g)}$-local.
\end{definition}

\begin{lemma}\label{fppf-subcanonical}
The type $R$ is $\propTrunc{\Spec(R[X]/g)}$-local for any monic polynomial $g$. 
\end{lemma}

\begin{proof}
See section on sheaves in diffgeo.
\end{proof}

\begin{lemma}\label{scheme-are-sheaf-from-affine}
Assume given a type $P$ such that:
\begin{itemize}
\item The type $R$ is $P$-local.
\item The type of open proposition is $P$-smooth.
\end{itemize}
Then any scheme is $P$-local.
\end{lemma}

\begin{proof}
Since $R$ is $P$-local we see that all affine schemes are $P$ local through stability under products and identity types. 

We check that any scheme $X$ is $P$-smooth. TODO

Now we conclude that any scheme is $P$-local, indeed its identity types are schemes, so they are $P$-smooth propositions, so they are $P$-local and any scheme is $P$-unramified.
\end{proof}

\begin{lemma}\label{constant-open-affine}
Assume given $A$ an f.p. algebra, then for any $h_1,\cdots,h_n:A$ such that the open $D(h_1,\cdots,h_n)$ is constant, for any $x:\Spec(A)$ we have that:
\[x\in D(h_1,\cdots,h_n) \leftrightarrow \exists i. h_i\ \mathrm{not\ nilpotent}\]
\end{lemma}

\begin{proof}
TODO
\end{proof}

\begin{lemma}
Any formally étale type $X$ is reduced, meaning that any closed dense embedding into $X$ not not has a section.
\end{lemma}

\begin{proof}
It is immediate that is actually has a section.
\end{proof}

\begin{corollary}\label{etale-affine-reduced}
For any affine étale scheme $\Spec(A)$we have that:
\[h \mathrm{nilpotent} \leftrightarrow \neg\neg h=0\]
\end{corollary}

\begin{proof}
The reverse implication is always true. If $h$ is nilpotent then for all $x:\Spec(A)$ is nilpotent, so that:
\[\prod_{x:\Spec(A)} \neg\neg h(x)=0\]
But then we have that $V(h)$ is a closed dense embedding, so it not not has a section, meaning $\neg\neg h = 0$.
\end{proof}

\begin{proposition}\label{schemes-are-etale-sheaves}
Any scheme is an étale sheaf.
\end{proposition}

\begin{proof}
Assume given $g:R[X]$ monic, by \cref{scheme-are-sheaf-from-affine} it is enough to prove that $R$ is $\propTrunc{\Spec(R[X]/g)}$-local, this is \cref{fppf-subcanonical}, and that the type of open proposition is $\propTrunc{\Spec(R[X]/g)}$-smooth. Assume given a constant open $D(h_1,\cdots,h_n)$ in $\Spec(R[X]/g)$. By \cref{constant-open-affine}, for all $x:\Spec(R[X]/g)$ we have that:
\[x\in D(h_1,\cdots,h_n) \leftrightarrow \exists i. h_i\ \mathrm{not\ nilpotent}\] 
So it is enough to prove that:
\[\exists i. h_i\ \mathrm{not\ nilpotent}\] 
is open to conclude. But this is \cref{TODO}.

But by \cref{etale-affine-reduced} we have that $h_i$ not nilpotent iff and only $h_i\not=0$, which means that one of the coefficient in $h$ is not zero (as $g$ is monic), so it is indeed open.
\end{proof}

\begin{remark}
We conjecture that we even have that any scheme is an fppf sheaf. To prove this it would be enough to prove that for all monic $g:R[X]$ and $h:R[X]/g$, we have that $h$ not being nilpotent is an open proposition.
\end{remark}

\subsection{Not all Deligne-Mumford propositions are schemes}

\begin{lemma}\label{no-roots-for-any-p}
If the map:
\[\A^1\to \A^1\]
\[x\mapsto x^p\] 
is surjective on an open $U$, then this open is empty.
\end{lemma}

\begin{proof}
Consider $a\in U$, by Zarsiki choice we have $g:R$ such that $g\in D(f)$ and we have:
\[\frac{f}{g^n}\]
inverse to $x\mapsto x^p$. Since $g(a)$ is invertible, $g$ is regular, then:
\[\frac{f^p}{g^{pn}} = X\]
implies that:
\[f^p = g^{pn}X\]
By induction we prove that all the coefficients of $f$ and $g$ are nilpotent, which contradicts $g(a)\not=0$.
\end{proof}

\begin{proposition}
Not all Deligne-Mumford propositions are schemes.
\end{proposition}

\begin{proof}
We have that $p\not=0$ for some prime $p$ (we know this holds for $2$ or $3$). Then for all $a:R$ we consider the étale affine scheme:
\[X_a = \Spec(R[X]/X^p-a)\]
Then $\propTrunc{X_a}$ is a Deligne-Mumford proposition. We assume they are schemes and reach a contradiction. Indeed by \cref{schemes-are-etale-sheaves} then $\propTrunc{X_a}$ would be an étale sheaf, so it would be $\propTrunc{X_a}$-local, so it would be inhabited. This means that map:
\[R\to R\]
\[x\mapsto x^p\]
would be surjective. This is a contradiction by \cref{no-roots-for-any-p}.
\end{proof}

A natural question to ask at this point is wether any Deligne-Mumford proposition that is an fppf sheaf is indeed a scheme?

TODO
