This section is a draft toward a more proper definition of stacks, using étale sheaves. The main result so far is étale descent for algebraic stacks.

\subsection{Étale sheaves and algebraic stacks}

The following definition is local. It is supposed to be equivalent to the one using unramifiable polynomial, although we lack a proof at the moment.

\begin{definition}
A type $X$ is an étale sheaf if is is $\propTrunc{\Spec(A)}$-local for all $A$ fppf and étale algebra.
\end{definition}

\begin{remark}
We conjecture this is equivalent to being local against monic unramifiable polynomials having roots. With the given definition we do not have a proof that all schemes are étale sheaf, although this certainly should hold.
\end{remark}

\begin{definition}
A map $f:X\to Y$ is étale surjective if for all $y:Y$ we have:
\[ \mathrm{et}(\propTrunc{fib_f(y)})\]
\end{definition}

\begin{definition}
An étale sheaf $X$ has an étale atlas if there merely is an affine scheme $\Spec(A)$ and a map:
\[\Spec(A)\to X\]
that is formally étale and étale-surjective.
\end{definition}

\begin{definition}
An étale sheaf $X$ is an algebraic stack if:
\begin{itemize}
\item It has an étale atlas.
\item Coinductively, its identity types are algebraic stacks.
\end{itemize} 
\end{definition}

An algebraic $n$-stack is a $n$-type that is an algebraic stack. 

\begin{remark}
Note that an étale sheaf $X$ being algebraic stack could be defined without coinduction using:
\[S_n(X): \mathrm{Prop}\]
defined inductively on $n$ by:
\[S_0(X) = \mathrm{hasEtaleAtlas}(X)\]
\[S_{n+1}(X) = \forall(x,y:X).\ S_n(x=y)\]
and asking:
\[\forall(n:\N).\ S_n(X)\]
\end{remark}


\subsection{Descent for algebraic stacks}

Next lemma directly implies that Zariski cover are étale atlases, by taking $A=R$. 

\begin{lemma}\label{zariski-etale-atlas}
If $A$ is fppf and étale and we are given $f_1,\cdots,f_n:A$ such that $(f_1,\cdots,f_n)=1$, we have that:
\[A_{f_1}\times\cdots\times A_{f_n}\]
is fppf and étale.
\end{lemma}

\begin{proof}
TODO
\end{proof}

\begin{proposition}\label{hasEtaleAtlasEtale}
Let $X$ be an étale sheaf, then the property:
\[\mathrm{hasEtaleAtlas}(X)\]
is an étale sheaf.
\end{proposition}

\begin{proof}
Assume $A$ an fppf étale algebra such that:
\[\propTrunc{\Spec(A)} \to \mathrm{hasEtaleAtlas}(X)\]
We just need to prove that:
\[\mathrm{hasEtaleAtlas}(X)\]
Then we have:
\[\Spec(A) \to \mathrm{hasEtaleAtlas}(X)\]
and by Zariski local choice there merely is a Zariski cover $\Spec(A') \to \Spec(A)$ with for all $x:\Spec(A')$ an étale atlas:
\[f_x : \Spec(B_x)\to X\]
Then the induced map:
\[\sum_{x:\Spec(A')}\Spec(B_x)\to X\]
is an étale atlas for $X$, indeed its fiber over $z:X$ is:
\[\sum_{x:\Spec(A')}\sum_{y:\Spec(B_x)} f_x(y) = z\]
which is a sigma type of formally étale and étale inhabited types by \cref{zariski-etale-atlas}.
\end{proof}

\begin{corollary}\label{isAlgebraicStackEtale}
Let $X$ be an étale sheaf, then the property:
\[\mathrm{isAlgStack}(X)\]
is an étale sheaf.
\end{corollary}

\begin{proof}
Having $X$ an étale sheaf asserts that all iterated identity types in $X$ have étale atlases, but all these iterated identity types are étale sheaves so we can conclude using \cref{hasEtaleAtlasEtale}
\end{proof}

Next corollary asserts étale descent for algebraic stacks.

\begin{corollary}
The type of algebraic stack is an étale sheaf.
\end{corollary}

\begin{proof}
The type of étale sheaf is an étale sheaf, and an étale sheaf $X$ being an algebraic stack is an étale sheaf by \cref{isAlgebraicStackEtale}.
\end{proof}

\subsection{Algebraic stacks are stable by quotients}

Next lemma intuitively means that we can quotient algebraic stack by étale relation in the étale topos and get algebraic stack.

\begin{lemma}
Let $X$ be a type such that:
\begin{itemize}
\item Identity types in $X$ are algebraic stacks.
\item $X$ has an étale atlas.
\end{itemize}
Then $\mathrm{et}(X)$ is an algebraic stack.
\end{lemma}

\begin{proof}
We have $X$ étale-sheaf separated by hypothesis and étale sheafification is lex so that:
\[i:X\subset \mathrm{et}(X)\]
We denote the assumed atlas of $X$ by: 
\[f: \Spec(B)\to X\]
\begin{itemize}
\item We have that $\mathrm{et}(X)$ is an étale sheaf by definition.
\item We have to prove that:
\[\prod_{x:\mathrm{et}(X)} \mathrm{isFormallyEtale}(\fib_{i\circ f}(x))\]
As affine schemes are étale sheaves, so is $\fib_{i\circ f}(x)$, and so it being formally étale is an étale sheaf. Therefore it is enough to prove:
\[\prod_{x:X} \mathrm{isFormallyEtale}(\fib_{i\circ f}(i(x)))\]
But since $i$ is an embedding, we have that:
\[\fib_{i\circ f}(i(x)) = \fib_f(x)\]
and the fiber of $f$ is assumed formally étale.
\item We have to prove that:
\[\prod_{x,y:\mathrm{et}(X)} \mathrm{isAlgStack}(x=_{\mathrm{et}(X)}y)\]
but by \cref{hasEtaleAtlasEtale}, it is enough to prove that:
\[\prod_{x,y:X} \mathrm{isAlgStack}(i(x)=_{\mathrm{et}(X)}i(y))\]
But:
\[(i(x)=_{\mathrm{et}(X)}i(y)) \simeq \mathrm{et}(x=_Xy) \simeq (x=_Xy)\]
and $x=_Xy$ is assumed to be a algebraic stack.
\end{itemize}
\end{proof}

Next corollary assume that schemes are étale sheaves, which we have not proven yet.

\begin{corollary}
Let $X$ be a scheme and let $\sim$ be an equivalence relation on $X$ such that:
\begin{itemize}
\item For all $x,y:X$, we have that $x\sim y$ is a scheme.
\item For all $x:X$, the type:
\[\sum_{y:X}x\sim y\]
is formally étale.
\end{itemize}
Then $\mathrm{et}(X/\sim)$ is an algebraic set.
\end{corollary}

