\subsection{Definition and basic properties}

Usually, in some of the definitions below,
there is the additional condition that types are sheaves for the fppf topology.
We will just ignore this condition for now.

\begin{definition}
  A set $X$ is an \notion{algebraic space},
  if $x=_Xy$ is a scheme for all $x,y:X$ and
  there merely is a scheme $U$ with a formally étale surjection onto $X$.
\end{definition}

Such a map $U\to X$ is called a scheme cover for $X$. So any scheme is an algebraic space.

\begin{lemma}
Let $X$ be an algebraic space. Any function from $X$ to $\N$ is merely bounded.
\end{lemma}

\begin{proof}
  This holds for schemes.
  So given a surjective map $U\to X$ with $U$ a scheme, this holds for $X$.
\end{proof}

\begin{lemma}
Algebraic spaces have étale-local choice.
\end{lemma}

\begin{proof}
Immediate from the definition using the fact that Zariski-local choice implies étale-local choice.
\end{proof}

\subsection{Fundamental theorem of algebraic spaces}

In brief, algebraic spaces are quotients of schemes by étale equivalence relations.

\begin{lemma}\label{quotient-by-equivalence-relation}
Assume given a set $X$ then the map:
\[ \sum_{R:X\to X\to \Prop} R\ \mathrm{equivalence\ relation} \to \sum_{Y:\mathrm{Set}} \sum_{p:X\to Y} p\ \mathrm{surjective}\]
\[R \mapsto (X/R,[\_])\]
is an equivalence with inverse the map:
\[(Y,p) \mapsto \lambda x,y. p(x)=p(y)\]
\end{lemma}

\begin{proof}
TODO, plain HoTT, beware that we need to use the set-truncation to define the quotient.
\end{proof}

\begin{definition}
An equivalence relation $R$ on a type $X$ is called:
\begin{itemize}
\item Schematic if for all $x,y:X$ the proposition $R(x,y)$ is a scheme.
\item Étale if for any $x:X$ its fibers:
\[\sum_{x:X} R(x,y)\]
are formally étale.
\end{itemize}
\end{definition}

\begin{proposition}\label{fundamental-propriety-algebraic-spaces}
Assume given a set $X$, then the following types are equivalent:
\begin{itemize}
\item The type of schematic étale equivalence relation over $X$.
\item The type of sets $Y$ with schemes as identity types and a surjective formally étale map from $X$ to $Y$.
\end{itemize}
\end{proposition}

\begin{proof}
By the equivalence in \cref{quotient-by-equivalence-relation}, it is enough to check that:
\begin{itemize}
\item The identity types in $X/R$ are schemes if and only if the relation $R$ is schematic. For any $x,y:X$ we know that:
\[R(x,y) \simeq [x] =_{X/R}[y]\]
so the direct direction is immediate. For the converse we use that being a scheme is a proposition and that the map $[\_]:X\to X/R$ is surjective.
\item The fibers of: 
\[[\_]:X\to X/R\] 
are formally étale if and only if the relation $R$ is étale. For any $y:X$ we have that:
\[\sum_{x:X} R(x,y) \simeq \mathrm{fib}_{[\_]}([y])\]
so the direct direction is immediate. Here as well the converse follows from surjectivity of $[\_]$.
\end{itemize}
\end{proof}

\begin{theorem}
A type is an algebraic space if and only if it is merely the quotient of a scheme by a schematic étale equivalence relation.
\end{theorem}

\begin{proof}
This is a direct application of \cref{fundamental-propriety-algebraic-spaces}.
\end{proof}

\subsection{Stability for algebraic spaces}

\begin{lemma}\label{sum-etale-etale}
Assume given $X$ a type with $Y_x$ a type depending on $x:X$. If we have a formally étale surjection:
\[p:U\to X\]
and for all $u:U$ a formally étale surjection:
\[q_x : V_u \to Y_{p(u)}\]
then the induced map in:
\[\sum_{u:U} V_u \to \sum_{x:X}Y_x\]
is a formally étale surjection.
\end{lemma}

\begin{proof}
Formally étale merely inhabited types are closed under dependent sums, and fibers of the induced maps are dependent sums of fibers of $p$ and some $q_x$, so they are formally étale and merely inhabited.
\end{proof}

\begin{lemma}\label{algebraic-space-sum}
Algebraic spaces are stable by dependent sums.
\end{lemma}

\begin{proof}
Assume given an algebraic space $X$ and for all $x:X$ an algebraic space $Y_x$. Identity types in $\sum_{x:X} Y_x$ are schemes because schemes are stable by dependent sums. We need to merely find a scheme-cover for $\sum_{x:X} Y_x$. 

For any $x:X$ we merely have:
\[\sum_{V_x:\mathrm{Scheme}} \sum_{q_x: V_x \to Y_x} q_x\ \mathrm{étale\ surjection}\]
So by étale-local choice for $X$, there merely is a scheme $U$ with an étale surjection $p: U \to X$ such that we merely have:
\[\prod_{u:U} \sum_{V_u:\mathrm{Scheme}} \sum_{q_u: V_u \to Y_{p(u)}} q_u\ \mathrm{étale\ surjection}\]
Then we merely have a scheme $\sum_{u:U}V_u$ with an induced map:
\[\sum_{u:U}V_u \to \sum_{x:X} Y_x\]
which is étale surjective by \cref{sum-etale-etale}.
\end{proof}

\begin{lemma}\label{algebraic-space-identity}
Algebraic spaces are stable by identity types.
\end{lemma}

\begin{proof}
Because schemes are algebraic spaces.
\end{proof}

By \cref{algebraic-space-sum} and \cref{algebraic-space-identity}, algebraic spaces are stable by finite limits.

\begin{lemma}
Algebraic spaces are stable by quotients by schematic étale equivalence relations.
\end{lemma}

\begin{proof}
Assume given an algebraic space $X$. By \cref{fundamental-propriety-algebraic-spaces} it is enough to check that for any set $Y$ which identity types are schemes with a formally étale surjection:
\[p:X\to Y\]
we have that $Y$ is an algebraic space. Composing a scheme cover for $X$ with $p$ gives a scheme cover for $Y$.
\end{proof}

\subsection{Examples}

\begin{example}
The scheme $\A^1$ quotiented by the relation which identifies $x$ and $-x$ when $x\not=0$ is an algebraic space.
\end{example}

\begin{proof}
We need to show that the equivalence relation $E$ generated by $E(x,-x)$ when $x\not=0$ is schematic and étale. This equivalence relation is:
\[E(x,y) = (x=y) + (x\not=0 \land x=-y)\]
It is clearly schematic. To check that it is étale, for any $y:R$ we compute:
\[\sum_{x:X}(x=y) + (x\not=0 \land x=-y) \simeq 1 + (y\not=0)\]
which is indeed étale.
\end{proof}

\begin{example}
The scheme:
\[\sum_{x,y:R} xy=0\]
quotiented by the relation which identifies $(x,0)$ and $(0,x)$ when $x\not=0$ is an algebraic space.
\end{example}

\begin{proof}
We need to show that the equivalence relation $E$ generated by $E((x,0),(0,x))$ when $x\not=0$ is schematic and étale. This equivalence relation is:
\[E((x,y),(x',y')) = (x=x'\land y=y') + (x\not=0 \land x=y' \land x'=0)\]
as $x\not=0$ implies $y=0$ since $xy=0$. It is clearly schematic. To check that it is étale, for any $x',y':R$ such that $x'y'=0$ we compute:
\[\sum_{x,y:R} xy=0 \land E((x,y),(x',y')) \simeq 1+ (y'\not=0)\]
which is indeed étale.
\end{proof}

\subsection{Algebraic Spaces and group actions}

\begin{definition}
An action of a group $G$ on a type $X$ is free if for all $x,y:X$ the type:
\[\sum_{g:G}gx=y\]
 is a proposition.
\end{definition}

If $X$ is a set this is the same as asking that for all $x:X$ we have that $gx=x$ implies $g=1$.

\begin{lemma}
Let $G$ be an étale group scheme acting freely on an algebraic space $X$. Then:
\[x,y:X \mapsto \sum_{g:G}gx=_Xy\]
is a schematic étale equivalence relation.
\end{lemma}

\begin{proof}
The type:
\[\sum_{g:G}gx=_Xy\]
is a scheme because it is a dependent sum of schemes. For any $y:Y$ we have:
\[\sum_{x:X}\sum_{g:G}gx=y \simeq G\]
which is assumed étale.
\end{proof}

\begin{corollary}
Algebraic spaces are stabe by quotient by free action of étale group schemes. In particular quotient of schemes by free action of étale group scheme are algebraic spaces.
\end{corollary}

\begin{lemma}
Let $G$ be a finite group acting on an unramified scheme $X$, then $X/G$ is an algebraic space.
\end{lemma}

\begin{proof}
We are considering the quotient of $X$ by the equivalence relation:
\[R(x,y) = \exists (g:G). gx=_Xy\]
this is a schematic relation (even open) because $G$ is finite and identity types in $X$ are open propositions as $X$ is unramified. 

Now we need to show that for all $y:X$ the type:
\[\sum_{x:X}\exists (g:G). gx=_Xy\]
is formally étale:
\begin{itemize}
\item It is formally unramified because it is a subtype of $X$, which is assumed formally unramified.
\item It is formally smooth because we have a surjection:
\[G \simeq \sum_{x:X}\sum_{g:G} gx=_Xy \to \sum_{x:X}\exists (g:G). gx=_Xy\]
and $G$ is finite so it is smooth.
\end{itemize}
\end{proof}
