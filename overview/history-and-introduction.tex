
In mathematics, it is common practice to assume a fixed set theory, usually with the axiom of choice, as a common basis. While it is a great advantage to work in one common language and share a lot of the basic constructions, the dual approach of adapting the  ``base language'' to particular mathematical domains is sometimes more concise, provides a new perspective and encourages new proof techniques which would be hard to find otherwise.
We use the word ``synthetic'' to indicate that the latter approach is used,
as it was used by Kock and Lawvere to describe developement of mathematics internal to certain categories \cite{lawvere-categorical-dynamics}, in particular toposes -- a program which dates back as far as 1967.

Already in the 70s in the same program, Anders Kock suggested to use the language of higher-order logic \cite{Church40} to describe the Zariski topos, the collection of sheaves for the Zariski topology \cite{Kock74,kockreyes}, which is the first occurence of synthetic algebraic geometry.
Kock's approach allowed for a more suggestive and geometrical description of schemes.
There is in particular a ``generic local ring'' $R$, which, as a sheaf, associates to any algebra $A$ its underlying set and, as described in \cite{kockreyes}, the projective space $\bP^n$ is then the set of lines in $R^{n+1}$.

Just using category theory is not the same as reasoning synthetically -- for the latter the goal is usually to derive results exclusively in one system,
as Kock and Lawvere did with differential geometry in his work.
The distinction with just using an abstraction like categories is important, since the translation from the synthetic language and back can become cumbersome -- although it is still the goal to derive statements about ordinary mathematical objects in the end.

Starting with Kock and Lawvere's work, more differential geometry was developed synthetically \cite{kock-sdg} along with a study of the models of the theory \cite{moerdijk-reyes}.
One basic axiom of the theory, called the Kock-Lawvere axiom, allows for reasoning with nilpotent infinitesimals. Our version of synthetic algebraic geometry use a generalisation of this axiom called the duality axiom. Let us now describe the Kock-Lawvere axiom.

The Kock-Lawvere axiom is added to a basic language which can be interpreted in good enough categories (for example toposes), more precisely we need basic objects like $\emptyset$, $\{\ast\}$ and $\N$ as well natural constructions like $A\times B$ or $A^B$ for objects $A$, $B$, which all behave as expected. We also need predicates $P(x)$ for elements $x:A$ so we can form subobjects like $\{x:A\mid P(x)\}$.
In this language, we assume there is a fixed ring $R$, which can be thought of as the real numbers. We define $\D(1)=\{x\in R\mid x^2=0\}$ to be the set of all square-zero elements of $R$, then the Kock-Lawvere axiom gives us a bijection
\[ e : R \times R  \to R^{\D(1)} \]
which commutes with evaluation at $0$ and projection to the first factor.
The intuition is that $\D(1)$ is so small that any function on it is linear and therefore determined by its value and its derivative at $0\in\D(1)$.
With this axiom, the derivative at $0:R$ of a function $f : R \to R$ may then be defined as $\pi_2(e^{-1}(f_{\vert \D(1)}))$. This is the start of a convenient development of differential calculus, which doesn't require any further structures on $R$ or other objects. This is the core of the synthetic method: we can work with these differential spaces as if they were sets.

To give an example, the tangent bundle of a manifold $M$ can be defined as $M^{\D(1)}$ and vector fields as sections of the map $M^{\D(1)}\to M$ evaluating at $0$. Then it is easy to see that a vector field is the same as a map $\zeta:\D(1)\to M^M$ with $\zeta(0)=\id_M$, which can be interpreted as an infinitesimal transformation of the identity map. This style of reasoning with spaces as if they were sets is also central in current synthetic algebraic geometry. 

The Kock-Lawvere axiom above are incompatible with the law of excluded middle (LEM) and therefore also with the axiom of choice (AC). Indeed they imply all maps from say $R$ to $R$ are differentiable, which contradicts LEM. 
However restricted versions of LEM and AC are compatible with this axiom. A very basic example is that equality of natural numbers is decidable, meaning that two natural numbers are either equal or not equal. We will latter go back to why full LEM and AC tend to be incompatible with synthetic approach to various geometry.

%TODO: Hugo is here in the rereading

The use of nilpotent elements to capture infinitesimal quantities as mentioned above was inspired by the Grothendieck school of algebraic geometry and Anders Kock also worked with an extended axiom \cite{Kock74,kockreyes} suitable for synthetic algebraic geometry, where the role of $\D(1)$ above can be taken by any finitely presented affine scheme. In his 2017 doctoral thesis, Ingo Blechschmidt noticed a property holding internally in the Zariski-topos, which he called synthetic quasi-coherence. It is generalised and internalised version of what Kock used. In 2018, David Jaz Myers\footnote{Myers' never published on the subject, but communicated his ideas to Felix Cherubini and in talks to a larger audience \cite{myers-talk1,myers-talk2}.} started working with a specialization of Blechschmidt's synthetic quasi-coherence, which is what we now call \emph{duality axiom}.

To state the duality axiom we need to go from the space $\D(1)$ to spaces that are the common zeros of some finite system of polynomial equations over $R$. Such a space can be encoded independently of the choice of polynomials as a finitely presented $R$-algebra, i.e.\ an $R$-algebra $A$ which is of the form $R[X_1,\dots,X_n]/(P_1,\dots,P_l)$ for some numbers $n,l$ and polynomials $P_i\in R[X_1,\dots,X_n]$.
Then the set of roots of the system is given by the type $\Hom_{\Alg{R}}(A,R)$ of $R$-algebra homomorphisms from $A$ to the base ring. We denote this type by $\Spec A$.
Now the duality axiom states that $\Spec$ is the inverse to exponentiating with $R$, i.e.\ for all 
finitely presented $R$-algebras $A$ the following is an isomorphism:
\[ (a\mapsto (\varphi\mapsto \varphi(a))) : A\to R^{\Spec A}\rlap{.}\]

Myers used homotopy type theory as a base language, which is now the standard in synthetic algebraic geometry. Now we introduce homotopy type theory, in the next paragraphs we will explain how it fits with synthetic algebraic geometry. Homotopy type theory is a language for synthetic homotopy theory.
This means that when using it, we can think of the basic objects of the theory, that is types, directly as homotopy types. This should be contrasted with the usual practice in algebraic topology, which is to implement these homotopy types as topological spaces or Kan complexes.
So the rules of homotopy type theory allow to work with types in very much the same way as one would work with homotopy types in traditional mathematics. % -- with the clear exception of the law of excluded middle and the axiom of choice - although the former and restricted versions of the latter can be assumed.

On the other hand we also use homotopy type theory because it allows to reason synthetically about spaces, as plain type theory does. A key point is that we do not use the law of excluded middle (LEM) or the axiom of choice (AC), which are incompatible with types being interpreted as spaces. Indeed on one hand LEM allows us to find a complement of each subset of a given type $A$, which exposes $A$ as a coproduct.
This is not true for spaces, for example, $\R$ is not the coproduct of the topological subspaces $\{0\}$ and $\R/\{0\}$.
On the other hand AC states that any surjection has a section. This is also not true for any sensible notion of space, in particular it would trivialise all cohomology.
Thus, constructive reasoning in the sense of not using LEM and AC is a necessity if we want to types to be understood as having a spatial structure. It turns out that this is the only obstruction, so the rules of type theory allow to work with type as one would work with spaces in algebraic geometry.

In synthetic algebraic geometry, we work inside homotopy type theory so that types behave both as homotopy types and as spaces from algebraic geometry. This means that we are mixing two synthetic approaches, combining their advantages,
which rests on the possibility of interpreting homotopy type theory in any higher topos \cite{shulman2019all} and not just the higher topos of $\infty$-groupoids. More precisely we think of the higher topos of Zariski sheaves with value in homotopy type. %We remind readers of this by using the notation $x:X$ which can often be thought of as $x\in X$.
The general idea of using homotopy type theory to combine some kind of synthetic, spatial reasoning with synthetic homotopy theory, goes back at least to 2014, to Mike Shulman and Urs Schreiber \cite{Schreiber_2014}.
Schreiber suggested to the HoTT community at various occasions to make use of HoTT as the internal language of higher toposes, where specifities of the topos are accessed in the language via modalities.
This approach was shown to be quite effective and intuitive in Shulman's \cite{shulman-Brouwer-fixed-point} work on mixing synthetic homotopy theory in the form of HoTT and a synthetic approach to topology using a triple of modalities -- a structure called cohesion by Lawvere \cite{Lawvere2007}. A more detailed introduction to homotopy type theory for a general mathematical audience, with an emphasis on this mix of homotopical and spatial structure can be found in \cite{shulman-logic-of-spaces}.

One of the main advantages of using specifically homotopy type theory and not plain type theory, is using synthetic homotopical reasoning to make cohomological computations. Indeed one of Schreiber's motivation was to make use of the modern perspective on cohomology as the connected components of a space of maps in a higher topos. This can be mimicked in HoTT as follows: Given $X$ a type, $A$ an abelian group and $n:\N$, we define the $n$-th cohomology group of $X$ with coefficients in $A$ as
\[ H^n(X,A):=\| X\to K(A,n) \|_0\]
 where $\|\_\|_0$ is the $0$-truncation, an operation which turns any type into a $0$-type, that is a type with trivial higher structure. The type $K(A,n)$ is the $n$-th Eilenberg MacLane space, which can always be constructed for any abelian group $A$ and comes with an isomorphism $\Omega^n(K(A,n))\simeq A$.
With this definition of cohomology groups we can use synthetic homotopy theory to reason about cohomology, which had already been done successfully for the cohomology of homotopy types like spheres and finite cell complexes. It also works for the cohomology of $0$-types such as spaces in synthetic algebraic geometry.
This internal version of cohomology does not agree with the external version mentioned above, indeed it is a sheaf of groups instead of a single group, and it is indexed by an internal natural number instead of an external one. Nevertheless, internal cohomology turned out to be quite useful in practice.

In 2022, trying to use this approach to calculate cohomology groups in synthetic algebraic geometry led to the discovery of what is now called Zariski-local choice \cite{draft},
which is an additional axiom that holds in the higher Zariski-topos.
It is a weakening of the axiom of choice. In homotopy type theory, the axiom of choice can be formulated as follows: For any surjective map $f:X\to Y$, there exists a section, i.e.\ a map $s:Y\to X$ such that $f\circ s=\id_Y$.
Zariski-local choice also states the existence of a section, but only Zariski-locally and only for surjections into an affine scheme: For any surjection $f:E\to \Spec A$,
there exists a Zariski-cover $U_1,\dots,U_n$ of $\Spec A$ and maps $s_i:U_i\to E$ such that $f(s_i(x))=x$ for all $x\in U_i$.

In homotopy type theory, we use the propositional truncation $\|\_\|$ to define surjections and more generally what we mean by ``exists''.
Propositional truncation turns an arbitrary type $A$ into a type $\|A\|$ with the property that $x=y$ for all $x,y:\|A\|$.
Types with this property are called propositions or (-1)-types in homotopy type theory.
Using a univalent universe of types $\mathcal U$ we have that surjection into a type $A$ are the same as type families $F:A\to \mathcal U$, such that we have $\|F(x)\|$ for all $x: A$.
Using type families instead of maps allows us to drop the condition that the maps we get are sections, since we can express it using dependent function types and we arrive at the formulation of Zariski-local choice given below in the list of axioms.
In this instance and many others, homotopy type theory is much more convenient for formal reasoning, which is an advantage when formalizing synthetic algebraic geometry.

In total, the system we use for synthetic algebraic geometry consists of the extension of homotopy type theory postulating a fixed commutative ring $R$ satisfying these three axioms (see below for an explanation of the first one):

\begin{center}
\begin{axiom}[Locality]%
  \label{loc-axiom}
  $R$ is a local ring, i.e.\ $1\neq 0$ and whenever $x+y$ is invertible then $x$ is invertible or $y$ is invertible.
\end{axiom}

\begin{axiom}[Duality]%
  \label{duality-axiom}
  For any finitely presented $R$-algebra $A$, the homomorphism
  \[ a \mapsto (\varphi\mapsto \varphi(a)) : A \to (\Spec A \to R)\]
  is an isomorphism of $R$-algebras.
\end{axiom}

\begin{axiom}[Zariski-local choice]%
  \label{Z-choice-axiom}
  Let $A$ be a finitely presented $R$-algebra
  and let $B : \Spec A \to \mU$ be a family of inhabited types.
  Then there exists a Zariski-cover $U_1,\dots,U_n\subseteq \Spec A$
  together with dependent functions $s_i : (x : U_i)\to B(x)$.
\end{axiom}
\end{center}

As we explained above the duality axiom is a generalisation of the Kock-Lawvere axiom, which was used for convenient infinitesimal computations. It has a lot of consequences. In line with classical algebraic geometry, it shows that we have an anti-equivalence between finitely presented $R$-algebras and affine schemes of finite presentation over $R$.
More surprisingly, it implies that all functions in $R\to R$ are polynomials, as well as various the properties of the base ring $R$.
For example, for all $x:R$, we have that $x$ is invertible if and only if we have $x\neq 0$.
It also implies that any map from an affine schemes to the natural numbers is bounded.

Surprisingly, the Zariski-local choice axiom was also usable to solve problems which have no obvious connection to cohomology.
For example, it implies that pointwise open subsets of an affine scheme are the same as subsets which are given by unions of non-vanishing sets of functions on the scheme.
In more detail, we an open proposition is a proposition of the form $r_1\neq 0 \vee\dots\vee r_n\neq 0$ where $r_i:R$.
Then a subset $U$ of a type $X$  is called open if the proposition $x\in U$ is open for all $x:X$.
Given an open subset $U$ of $\Spec A$, using Zariski-local choice we turn these elements $r_1,\dots,r_n$ of the base ring into functions defined Zariski-locally on $\Spec A$.
We can then even prove that $U$ is a union of non-vanishing sets $D(f_i)$ of global functions $f_i:\Spec A \to R$.
An analogous result holds for closed propositionsand vanishing sets of functions on affine schemes, where closed propositions are propositions of the form $r_1=0\wedge\dots\wedge r_n=0$ where $r_i:R$.

This connection between pointwise and Zariski-local openness is crucial to make the synthetic definition of a scheme work well:
A scheme is a type $X$, that merely has a finite open cover by affine schemes.
To produce interesting examples, it is necessary to use the locality axiom.
This is related to the Zariski topology and ensures that classical examples of Zariski covers can be reproduced.
A central example is the projective spaces $\bP^n$, which can be defined as the quotients of $R^{n+1}/\{0\}$ by the action of $R^\times$ by scaling.
A cover of $\bP^n$ is given by sets of equivalence classes of the form $\{[x_0:\cdots:x_n] \vert x_i\neq 0 \}$, which is clearly open using the pointwise definition.
To see that it is a cover, one has to note that for $x:R^{n+1}$, we have that $x\neq 0$ is equivalent to one of the entries $x_i$ being different from $0$. In synthetic algebraic geometry, this is the case for the base ring $R$ and the proof uses that $R$ is a local ring.
