The goal of this section is to prove that projective space $\bP^n$ is compact,
a classical result of elimination theory.
We will first deal with the case $n = 1$, using algebraic methods, and
then deduce the general case.

The following lemma can be understood as a version of the Euclidean algorithm
for univariate polynomials in the abscence of decidable equality.
\begin{proposition}\label{euclid}
  Let $p_1, \ldots, p_n : R[X]$. Then we can find propositions
  $b_1,\ldots,b_r$, with each $b_i$ of the form $D(u) \wedge v_1 = \ldots = v_k = 0$,
  such that $\neg \neg (b_1 \vee \ldots \vee b_r)$,
  and for any $i$, we either have that $(p_1,\ldots,p_n) = 0$ if $b_i$ holds,
  or we have a natural $d$, such that if $b_i$ holds,
  then $(p_1,\ldots,p_n)$ is principal generated by a degree $d$ monic polynomial.
\end{proposition}
\begin{proof}[sketch]
  We suppose each $p_i$ is represented by a list of coefficients.
  If one of these lists is empty, we simply throw it away.
  If there is no $p_i$ left,
  then there is nothing left to prove: we take $r = 1$ and $b_1 = D(1)$,
  and note that $(p_1,\ldots,p_n) = 0$.
  Thus we may suppose $n \ge 1$, and take $i$ such that the formal degree
  of $p_i$ is the smallest. Let $u$ be the leading coefficient of $p_i$.
  We have $\neg \neg (D(u) \vee u = 0)$. 
  In either case, we can make progress: if $D(u)$ and $n = 1$,
  then $(p_1,\ldots,p_n)$ is $(p_i)$ with $p_i$ monic;
  if $D(u)$ and $n > 1$, then we can divide the other $p_j$ by $p_i$,
  decreasing their formal degrees;
  and if $u = 0$ then we can reduce the formal degree of $p_i$.
\end{proof}

\begin{lemma}\label{lorenzen}
  Let $A$ be an $R$-algebra, $J$ an ideal of $R$, and $x, y : A$ elements
  such that $xy = 1$. Suppose $1$ is in both $J[x]$ and $J[y]$.
  Then there is $m : J$  such that $m = 1$ in $A$.
\end{lemma}
\begin{proof}
  Write $1 = \sum_{i=0}^n a_ix^i = \sum_{j=0}^m b_j y^j$.
  Multiplying by $y^j$, we have
  $y^j = \sum_{i=0}^n a_j x^i y^j$.
  For $0 \le j \le m$, we have
  $x^i y^j \in \{y^m,y^{m-1},\ldots,y,1,x,\ldots,x^n\}$, since $xy = 1$.
  Hence $\langle y^m,\ldots,1,\ldots,x^n \rangle = J\langle y^m,\ldots,1,\ldots,x^n\rangle$.
  By Nakayama, there is $m : J$ such that $m \cdot 1 = 1$ in $A$.
\end{proof}

\begin{lemma}
  Let $U_1 : \bP^1\setminus\{0\} \to \Open$ and
  $U_2 : \bP^1\setminus\{\infty\} \to \Open$ be open subsets of the two affine patches
  of the projective line.
  Then there merely is an open proposition $\varphi$, such that
  if $(x : \bP^1\setminus\{0\}) \to U_1(x)$ and
  $(x : \bP^1\setminus\{\infty\}) \to U_2(x)$ both hold, then $\varphi$ also holds,
  and if $\varphi$ holds, then for all $x : \bP^1\setminus\{0,\infty\}$,
  $U_1(x) \vee U_2(x)$ holds.
\end{lemma}
\begin{proof}[incomplete]
  By \cite[Theorem 4.2.7]{draft}, we have $U_1 = D(p_1,\ldots,p_n)$ and
  $U_2 = D(q_1,\ldots,q_m)$, where $p_i \in R[X]$ and $q_i \in R[Y]$.
  By applying \cref{euclid} twice and combining the results, 
  we can find propositions
  $b_1,\ldots,b_r$ with each $b_i$ of the form
  $D(u) \wedge v_1 = \ldots = v_k = 0$,
  so that for each $i$, both $(p_1,\ldots,p_n)$ 
  and $(q_1,\ldots,q_m)$ are principal when $b_i$ holds
  (in the strong sense: we know what the degree will be even without knowing $b_i$).
 
  Consider now an $i$ such that if $b_i$ holds, then
  $(p_1,\ldots,p_n)$ and $(q_1,\ldots,q_m)$ are both the unit ideal.
  Write $b_i$ as $D(u) \wedge v_1 = \ldots = v_k = 0$.
  Since $(p_1,\ldots,p_n)$ is the unit ideal if $b_i$ holds,
  we have that $(R[X]/(p_1,\ldots,p_n))^{b_i} = R[X,u^{-1}]/(p_1,\ldots,p_n,v_1,\ldots,v_k)$
  is trivial.
  Thus $1 \in J[X]$ in $R[X,u^{-1}]/(p_1,\ldots,p_n)$
  where $J = \langle v_1,\ldots,v_k\rangle$.
  In $R[x,y,u^{-1}]/(xy-1,p_1(x),\ldots,p_n(x),q_1(y),\ldots,q_m(y))$,
  we thus have $1 \in J[x]$ and $1 \in J[y]$.
  By \cref{lorenzen}, we have $m' : J$ such that $m' = 1$ in
  this ring.
  This means we have $m : J$, $N : \N$ such that $u^N = m$
  in $R[x,y](xy-1,p_1(x),\ldots,q_m(y))$.
  We take $\phi$ to be the disjunction of
  $D(u^N-m)$ over all such $i$.
\end{proof}
