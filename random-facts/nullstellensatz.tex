In this section we prove that the Jacobson radical of any f.p. algebra $A$ is equal to its nilradical. The idea for this result as well as the proof for $A=R$ was given to me (Hugo Moeneclaey) by Max Zeuner.

\begin{definition}
Let $A$ be an $R$-algebra, then we define the nilradical $\mathrm{Nil}(A)$ of $A$ as the ideal of nilpotents in $A$.
\end{definition}

\begin{definition}
Let $A$ be an $R$-algebra, then we define the Jacobson radical $\mathrm{Jac}(A)$ of $A$ as the ideal of $a:A$ such that forall $b:A$ we have $1-ba$ invertible.
\end{definition}

\begin{lemma}\label{nilradical-in-jacobson}
For any $R$-algebra $A$ we have:
\[\mathrm{Nil}(A) \subset \mathrm{Jac}(A)\]
\end{lemma}

\begin{proof}
Because $1+x$ is invertible for $x$ nilpotent.
\end{proof}

\begin{lemma}\label{R-jacobson}
We have:
\[\mathrm{Nil}(R) = \mathrm{Jac}(R)\]
\end{lemma}

\begin{proof}
We have that:
\[\forall (y:R).\ 1-xy\ \mathrm{inv}\]
\[\Leftrightarrow \forall (y:R).\ \neg(xy = 1) \]
\[\Leftrightarrow \neg (\exists y:R.\ xy=1) \]
\[\Leftrightarrow \neg\neg (x=0) \]
\[\Leftrightarrow x\ \mathrm{nil} \]
\end{proof}

\begin{proposition}
For any f.p. algebra $A$, we have that:
\[\mathrm{Nil}(A) = \mathrm{Jac}(A)\]
\end{proposition}

\begin{proof}
Assume $a:A$. We have that $a$ is nilpotent if and only if:
\[\forall (x:\Spec(A)).\ a(x)\ \mathrm{nil}\]
Now by \cref{R-jacobson} this is equivalent to:
\[ \forall (x:\Spec(A))(y:R).\ 1 - a(x)y\ \mathrm{inv}\]
Which by considering $b$ to be the constant map with value $y$, is equivalent to:
\[ \forall (b:B)(x:\Spec(A)).\ 1 - a(x)b(x)\ \mathrm{inv}\]
which is the equivalent to:
\[\forall (b:B).\ 1 - ab\ \mathrm{inv}\]
\end{proof}
