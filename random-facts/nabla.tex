\begin{definition}
The $\nabla$-modality is the modality obtained by localising at the family of $x=0+x\not=0$ for $x:R$.
\end{definition}

\begin{lemma}\label{bot-nabla-modal}
The type $\bot$ is $\nabla$-modal.
\end{lemma}

\begin{proof}
Indeed for any $x:R$ we have $\neg\neg(x=0 + x\not=0)$ which is precisely what we want.
\end{proof}

\begin{remark}
It means any negated statement is $\nabla$-modal, e.g. $x:R$ being invertible or nilpotent.
\end{remark}

\begin{lemma}
The type $2$ is not $\nabla$-modal.
\end{lemma}

\begin{proof}
If $2$ is $\nabla$-modal, then for all $x:R$ we can define:
\[x=0+x\not=0\to 2\]
with value $0$ if $x=0$ and $1$ if $x\not=0$. This gives $b:2$ such that $x=0 \to b=0$ and $x\not=0\to b=1$ so that $b=0 \leftrightarrow \neg\neg(x=0)$ and $b=1\leftrightarrow x\not=0$. So we get $\prod_{x:R} \neg\neg(x=0)+x\not=0$, i.e. a map $f:R\to R$ such that if $\neg\neg(x=0)$ then $f(x)=0$ and if $x\not=0$ then $f(x)=1$. We get $P:R[X]$ such that $P(0)=0$, i.e. $P = XQ$, and $Q$ constant on $R^\times$. This is a contradiction.
\end{proof}

\begin{lemma}\label{nabla-closed-not-not}
Let $P$ be a closed proposition, then $\nabla P = \neg\neg P$
\end{lemma}

\begin{proof}
Assume $P$ closed with $P \leftrightarrow x_1=0\land\cdots\land x_n=0$. By \Cref{bot-nabla-modal} we have that $\nabla P\to \neg\neg P$. Now assume $\neg\neg P$, let us prove $\nabla P$. By hypothesis we can assume $x_1=0+x\not=0$. If $x_1\not=0$ this contradicts $\neg\neg P$. So $x_1=0$. Then we do the same with $x_2,\cdots,x_n$ and conclude $\nabla P$.
\end{proof}

\begin{lemma}\label{closed-not-nabla}
If a closed proposition is $\nabla$-modal then it is decidable.
\end{lemma}

\begin{proof}
Assume $P$ closed and $\nabla$-modal, by \Cref{nabla-closed-not-not} we have $\neg\neg P \leftrightarrow = P$, so $P$ is decidable by Nakayama.
\end{proof}

\begin{corollary}
The type $R$ is not $\nabla$-modal.
\end{corollary}

\begin{proof}
Otherwise its identity type would be decidable by \Cref{closed-not-nabla}.
\end{proof}

\begin{lemma}
Assume $x,y:R$ such that '$x$ nilpotent or $y$ nilpotent' is $\nabla$-modal, then it is equivalent to $xy$ nilpotent.
\end{lemma}

\begin{proof}
Let us prove $x$ nilpotent or $y$ nilpotent assuming $xy$ nilpotent. By hypothesis we can assume $y=0+y\not=0$. But if $y=0$ then $y$ is nilpotent, if $y\not=0$ then $xy$ nilpotent gives $z$ nilpotent. The converse is immediate.
\end{proof}