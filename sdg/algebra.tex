{
\newcommand{\EE}{\mathbf{E}}
\newcommand{\ev}{\operatorname{\mathrm{ev}}}

\subsection{Endomorphism Theories}
Let \(X\) be a type and \(n\) be a natural number.
In the following, we will write \(\EE_X(n)\) for the type of functions \(X^n\to X\).
It contains \(n\) canonical functions, the projections \(\pi^n_1,\dots,\pi^n_n : \EE_X(n)\).
Given any type \(Z\), the projection functions induce an equivalence
\[
  (Z \to X^n) \to (Z\to X)^n, \quad h \mapsto (\pi^n_1\circ h,\dots,\pi^n_n\circ h).
\]
We make implicit use of this equivalence by identifying an \(n\)-tuple of functions \(h_1,\dots,h_n:Z\to X\) with the unique induced function \(h : Z\to X^n\) which satisfies \(\pi^n_i \circ h = h_i\).

\begin{definition}
  Let \(X\) be a set.
  An \emph{\(\EE_X\)-algebra} consists of a set \(A\) and a function of type
  \[
    \interpret{-}\colonequiv\interpret{-}_A : \prod_{\{n:\N\}}\prod_{f:\EE_X(n)} A^n \to A
  \]
  such that \(\interpret{-}\) is compatible with projections, in that
  % \begin{align*}
  %   \prod_{n:\N}\prod_{i:\{1,\dots,n\}} \interpret{\pi^n_i} \underset{A^n \to A}{=} \pi^n_i
  % \end{align*}
  for each \(n:\N\) and each \(i : \{1,\dots,n\}\)
  \[\interpret{\pi^n_i} \underset{A^n \to A}{=} \pi^n_i,\]
  and with composition, in that for all \(m,n:\N\), \(f:\EE_X(n)\) and \(g_1,\dots,g_n : \EE_X(m)\)
  \[\interpret{f\circ (g_1,\dots,g_n)} \underset{A^m \to A}{=} \interpret{f}\circ (\interpret{g_1},\dots,\interpret{g_n}).\]
  The set \(A\) is called the \emph{carrier} of the \(\EE_X\)-algebra and the function \(\interpret{-}\) the \emph{interpretation function}.
  A \emph{morphism of \(\EE_X\)-algebras} is a map between the underlying carriers which commutes with the respective interpretation functions.
  Given two \(\EE_X\)-algebras \(A\) and \(B\), we write \(\Alg{\EE_X}(A,B)\) for the subset of maps \(A\to B\) which are morphisms of \(\EE_X\)-algebras.
  Here, we identify an algebra with its carrier, suppressing the interpretation function and compatibility equalities, as usual.
\end{definition}

\begin{example}
  The set \(X\) itself carries the structure of an \(\EE_X\)-algebra in a canonical fashion.
  More generally, for a type \(Y\), the set of functions \(Y \to X\) carries the structure of an \(\EE_X\)-algebra by setting
  \[\interpret{f}_{Y\to X} : (Y\to X)^n \to (Y\to X), \quad (s_1,\dots,s_n) \mapsto f\circ (s_1,\dots,s_n)\]
  for \(f:\EE_X(n)\), which is easily seen to be compatible with projections and composition.
  % Moreover, we write \(X_i\colonequiv\pi^n_i : F(n)\) for the \(n\) projection functions
  % We call this the \emph{free \(\EE_X\)-algebra} on \(n\) generators.
  % More specifically, the generators are given by the projection functions \(\pi^n_1,\dots,\pi^n_n : F(n)\)
\end{example}

The previous example allows us to view \(\EE_X(m)\) itself as an \(\EE_X\)-algebra for each natural number \(m\), by setting \(Y\equiv X^m\).
We distinguish this algebra from the set \(\EE_X(m)\)
In this case we write \(F(m)\colonequiv F_X(m)\colonequiv (X^m\to X)\) and denote the projection functions by \(X_1\colonequiv \pi^m_1, \dots X_m\colonequiv\pi^m_m : F(m)\).
The following lemma shows that \(F(m)\) is the \emph{free \(\EE_X\)-algebra} generated by \(m\) generators \(X_1,\dots,X_m\).
 % the \emph{free \(\EE_X\)-algebra} on \(m\) generators.

\begin{lemma}
  Let \(m\) be a natural number.
  For every \(\EE_X\)-algebra \(A\), the canonical map
  \[
    \Alg{\EE_X}(F(m),A) \to A^m, \varphi \mapsto (\varphi(X_1),\dots,\varphi(X_m))
  \]
  is an equivalence.
\end{lemma}
\begin{proof}
  Given an \(m\)-tuple \(\vec a\equiv (a_1,\dots,a_m) : A^n\) we define a map \(\ev_{\vec a} : F(m)\to A\) by setting
  \[\ev_{\vec a}(g) \colonequiv \interpret{g}_A(\vec a)\]
  for \(g:F(m)\).
  This is in fact a morphism of \(\EE_X\)-algebras.
  Namely, for every \(n:\N\), \(f:\EE_X(n)\) and \(g_1,\dots,g_n : F(m)\) we compute
  \begin{align*}
    \interpret{f}_A(\ev_{\vec a}(g_1),\dots,\ev_{\vec a}(g_n)) & = \interpret{f}_A(\interpret{g_1}_A(\vec a),\dots,\interpret{g_n}_A(\vec a)) \\
    & = \interpret{f\circ(g_1,\dots,g_n)}_A (\vec a)\\
    & = \interpret{\interpret{f}_{X^m\to X}(g_1,\dots,g_n)}_A (\vec a)\\
    & = \ev_{\vec a}(\interpret{f}_{X^m \to X}(g_1,\dots,g_n)).
  \end{align*}
  Note that \(\ev_{\vec a}(X_i) = a_i\), so \(\ev_{\vec a}\) lies in the fiber over \(\vec a\) of the map in question.
  Conversely, given a morphism \(\varphi : F(m)\to A\) of \(\EE_X\)-algebras, set \(a_i\colonequiv \varphi(X_i)\).
  Now, given any \(g:F(m)\), we have
  \begin{equation*}
    \ev_{\vec a}(g)
    \equiv \interpret{g}_A(\varphi(X_1),\dots,\varphi(X_m))
     = \varphi(\interpret{g}_{F(m)}(X_1,\dots,X_m))
     \equiv \varphi(g\circ (X_1,\dots,X_m))
     = \varphi(g\circ\id) \equiv \varphi(g).
  \end{equation*}
  This shows that mapping \(\vec a \mapsto \ev_{\vec a}\) defines an inverse to the map in question.
  % \begin{align*}
  %   \ev_{\vec a}(g) & \equiv \interpret{g}_A(\varphi(X_1),\dots,\varphi(X_m))\\
  %   & = \varphi(\interpret{g}_{F(m)}(X_1,\dots,X_m)) \\ 
  %   & \equiv \varphi(g\circ (X_1,\dots,X_m)) \\
  %   & = \varphi(g).
  % \end{align*}
\end{proof}

\begin{definition}
  An equivalence relation \(R\) on (the carrier of) an \(\EE_X\)-algebra \(A\) is called a \emph{congruence} if it is compatible with the interpretation function of \(A\) in the following sense:
  for every natural number \(n\), every operation \(f:\EE_X(n)\) and all \(a_1,\dots,a_n,b_1,\dots,b_n : A\) we have an implication
  \[
    R(a_1,b_1) \times \dots\times R(a_n,b_n) \to R(\interpret{f}(a_1,\dots,a_n),\interpret{f}(b_1,\dots,b_n)).
  \]
\end{definition}

\begin{lemma}
  Let \(R\) be an equivalence relation on the carrier of an \(\EE_X\)-algebra \(A\) and suppose that \(A/R\) is a quotient of \(A\) by \(R\) with quotient map \(q:A \to A/R\).
  If \(R\) is a congruence relation, there is a unique \(\EE_X\)-algebra structure on \(A/R\) such that \(q\) becomes a morphism of \(\EE_X\)-algebras.
\end{lemma}
\begin{proof}
\end{proof}


\subsection{Ringed Endomorphism Theories}

We now assume that our base set \(R\equiv X\) is a ring.
As usual, we will assume all rings to be commutative.


\begin{lemma}
  Every \(\EE_R\)-algebra \(A\) carries the structure of an \(R\)-algebra
\end{lemma}

Next:
\begin{itemize}
  \item Ringed endomorphism theories
  \item Ideals yield congruences
\end{itemize}

}
