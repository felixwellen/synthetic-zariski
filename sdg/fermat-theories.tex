{
\newcommand{\cat}[1]{\mathcal{#1}}
\newcommand{\theory}[1]{\mathbf{#1}}
\newcommand{\Psh}{\mathop{PSh}}
\newcommand{\Set}{\mathsf{Set}}
\newcommand{\TT}{\theory{T}}
\newcommand{\EE}{\theory{E}}
\newcommand{\Com}{\theory{Com}}
\newcommand{\diffquot}[2]{\frac{\Delta{#1}}{\Delta{#2}}}

Recall that an \emph{algebraic theory} \(\TT\) is a category whose underlying type of objects is the set \(\N\) of natural numbers, together with a choice of morphisms \(\pi^n_1,\dots,\pi^n_n : \TT(n,1)\) for each \(n : \N\) which exhibit \(n\) as an \(n\)-th power of \(1\) in \(\TT\).
Given an algebraic theory \(\TT\), a \emph{\(\TT\)-algebra object} in a category \(\cat C\) is a functor \(A:\TT\to\cat C\) which preserves finite powers, and a \emph{morphism} between two \(\TT\)-algebra objects in \(\cat C\) is a natural transformation.
The object \(A(1)\) is called the \emph{carrier} of the \(\TT\)-algebra object.
A \(\TT\)-algebra object in the category of sets (more precisely, in the category of \(\mU\)-small sets for a universe \(\mU\)) is simply called a \emph{\(\TT\)-algebra}.

\begin{remark}
  We also write \(\TT(n)\) for the set \(\TT(n,1)\) and refer to its terms as \emph{\(n\)-ary operations}.
  Since every object of an algebraic theory \(\TT\) is a finite power of \(1\), its universal property guarantees, that every morphism \(f:\TT(n,m)\) is uniquely determined by an \(m\)-tuple of \(n\)-ary operations \(f_1,\dots,f_m : \TT(n)\).
  Consequently, \(\TT\) is uniquely determined by the sets \(\TT(n)\) together with a composition operation which is associative and compatible with projection operations.
  Such a structure is also known as an \emph{(abstract) clone} of an algebraic theory.
\end{remark}

\begin{example}
Let \(R^0,R\equiv R^1,R^2,\dots\) be a sequence of objects of a category \(\cat C\) together with morphisms \(\pi^n_i : R^n \to R\) which exhibit \(R^n\) as an \(n\)-th power of \(R\).
We may assume that \(\pi^1_1 = \id_R\).
We define an algebraic theory \(\EE_R\) as follows: For its \(n\)-ary operations, set
\[\EE_R(n) \colonequiv \cat C(R^n,R).\]
The composition operation on \(\cat C\) then restricts to a composition operation on \(\EE_R\) which is associative and compatible with projections.
By the remark above, this defines an algebraic theory \(\EE_R\) called the \emph{endomorphism theory} of \(R\).
% It is easy to see that this datum uniquely specifies an algebraic theory \(\EE_R\) which we call the \emph{endomorphism theory} of \(R\).
Moreover, by mapping \(n\mapsto R^n\), there is a canonical fully faithful embedding \(\EE_R \hookrightarrow \cat C\) which allows us to regard \(\EE_R\) as the full subcategory of \(\cat C\) generated by the specified finite powers of \(R\).
\end{example}

\begin{remark}
  Every algebraic theory \(\TT\) is an endomorphism theory, namely \(\TT=\EE_1\) for \(1:\TT\).
\end{remark}

\begin{example}
  Consider the opposite category of the category of commutative rings and let us write \(\ell A\) for the formal dual of a commutative ring \(A\).
  The polynomial ring \(\Z[X]\) over the integers admits finite copowers (the formally dual concept to powers).
  Namely, we can realize its \(n\)-th copower as the polynomial ring \(\Z[X_1,\dots,X_n]\) in \(n\) variables with coprojections given by \(X\mapsto X_i\).
  Therefore, the endomorphism theory \(\EE_{\ell\Z[X]}\) on its formal dual exists.
  % This theory is called the \emph{theory of commutative rings} and denoted by \(\Com\).

  The carrier of an algebra of this theory admits the structure of a commutative ring.
  For example, the addition operation is specified by the image of the dual of the map \(X\mapsto X_1+X_2\).
  Conversely, every commutative ring \(A\) gives rise to an algebra of \(\EE_{\ell\Z[X]}\) in the category of sets because every polynomial \(f\in\Z[X_1,\dots,X_n]\) (that is, a ring morphism \(\Z[X]\to\Z[X_1,\dots,X_n]\)) induces a (polynomial) map \(A^n\to A, \vec{a}\mapsto f(\vec{a})\), and this is obviously compatible with projections.
  One can extend this correspondence to an equivalence of categories between the category of commutative rings and the category of algebras for \(\EE_{\ell\Z[X]}\).
  It is thus reasonable to call \(\Com\colonequiv\EE_{\ell\Z[X]}\) the \emph{theory of commutative rings}.
\end{example}

\begin{definition}
  Free algebras \(F_\TT(n)\).
\end{definition}

Given two algebraic theories \(\TT\) and \(\TT'\), a \emph{theory morphism} \(\tau:\TT\to\TT'\) is a functor between their underlying categories which is the identity on objects and preserves projections.
The pair \((\TT,\tau)\) is called an \emph{extension} of the theory.

To make the following definition more readable, let us introduce some notational conventions.
If we denote the projection operations \(\pi^n_i\) of an algebraic theory \(\TT\) by \(X_i\), leaving the arity \(n\) implicit, we may treat them as {variables} in the following sense:
By the universal property, we have \(\id_n = (X_1,\dots,X_n)\) in \(\TT(n,n)\).
Suppressing the composition symbol, any operation \(f : \TT(n)\) can thus be written as
\[f = f\circ \id_n = f \circ (X_1,\dots,X_n) = f(X_1,\dots,X_n) = f(\vec X),\]
where we employ vector notation \(\vec X \colonequiv (X_1,\dots,X_n)\) for the last equality.
Moreover, given morphisms \(g_1,\dots,g_n : \TT(m,n)\) we can resort to the usual substitution notation for composition, as is common in algebra:
\[f \circ (g_1,\dots,g_n) = f(g_1(X_1,\dots,X_m),\dots,g_n(X_1,\dots,X_m))\]
We also use different variable names instead of \(X_1,X_2,\dots\), for example \(X,Y,Z\).
Finally, assume that \(\TT\) is an extension of the theory of commutative rings, say \(\tau : \Com \to \TT\).
By abuse of notation, we suppress \(\tau\) and write \(0,+,-,1,\cdot\) for the usual ring operations inherited from \(\Com\).
Moreover, we use infix notation for addition and multiplication as is common in algebra.

\begin{definition}
  A \emph{Fermat theory} is an extension \(\TT\) of the theory \(\Com\) of commutative rings which satisfies the following property.
  For each \(f(X,\vec{Z}) : \TT(n+1)\) there exists a unique \(g(X,Y,\vec{Z}) : \TT(n+2)\) such that
  \[
    f(X,\vec{Z}) - f(Y,\vec{Z}) = (X-Y)\cdot g(X,Y,\vec{Z}).
  \]
  The operation \(g\) is called the \emph{difference quotient} or \emph{Fermat-Hadamard quotient} of \(f\) (with respect to the first argument).
  We also denote it by the symbol \(\diffquot{f}{X}\).
\end{definition}

Given an extension \(\tau:\Com\to\TT\) of the theory of commutative rings, a \(\TT\)-algebra canonically carries the structure of a commutative ring by precomposition with \(\tau\).
Accordingly, every morphism of \(\TT\)-algebras is already a homomorphism between the underlying commutative rings.
In particular, the free algebra \(R\colonequiv F_\TT(0)\) on zero generators is a commutative ring.
We call this the \emph{base ring} of the theory.
Since it is initial among all \(\TT\)-algebras, every \(\TT\)-algebra carries a canonical \(R\)-algebra structure.

\begin{proposition}
  Let \(A\) be an algebra over a Fermat theory \(\TT\) and \(I\) an ideal of the underlying ring of \(A\).
  There is a unique \(\TT\)-algebra structure on the quotient ring \(A/I\) such that the canonical ring morphism \(A \to A/I\) becomes a morphism of \(\TT\)-algebras which annihilates \(I\).
  Moreover, it is initial among all \(\TT\)-algebra morphisms \(A\to B\) which annihilate \(I\).
\end{proposition}
\begin{proof}
  See \cite[Proposition 1.2]{DK84} (check universal property?).
\end{proof}

\begin{definition}
  Finitely presented \(\TT\)-algebras.
\end{definition}

\begin{definition}
  Let \(\TT\) be an extension of the theory of commutative rings with base ring \(R\).
  The \emph{\(\TT\)-spectrum} of a finitely presented \(\TT\)-algebra \(A\) is the set of \(\TT\)-algebra homomorphisms from \(A\) to \(T\).
  \[\Spec_\TT(A) \colonequiv \Alg{\TT}(A,R)\]
\end{definition}

\begin{remark}
  More generally, given an arbitrary \(\TT\)-algebra \(A\) we can consider the \emph{slice theory} \(\TT_{A/}\) and define the spectrum of a \(\TT_{A/}\)-algebra.
  Note that \(\TT \equiv \TT_{R/}\).
  If \(\TT\) was a Fermat theory, the slice theory \(\TT_{A/}\) is a Fermat theory as well by \cite[Proposition 1.5']{DK84}.
\end{remark}

Since every \(\TT\)-algebra morphism \(\varphi : A \to A'\) is a \(\TT\)-algebra morphism \emph{under} the base ring \(R\), meaning it commutes with the unique maps from \(R\) to \(A\) and \(A'\),  it is in particular a homomorphism of \(R\)-algebras between the underlying rings.
Therefore, the \(\TT\)-spectrum of a \(\TT\)-algebra is a subset of its \(R\)-spectrum.

\begin{definition}
  A commutative ring object \(R\) is called a \emph{Fermat ring} if its endomorphism theory is a Fermat theory.
\end{definition}

}
