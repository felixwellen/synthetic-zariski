{
\newcommand{\cat}[1]{\mathcal{#1}}
\newcommand{\theory}[1]{\mathbf{#1}}
\newcommand{\Psh}{\mathop{PSh}}
\newcommand{\Set}{\mathsf{Set}}
\newcommand{\TT}{\theory{T}}
\newcommand{\EE}{\theory{E}}
\newcommand{\Com}{\theory{Com}}
\newcommand{\diffquot}[2]{\frac{\Delta{#1}}{\Delta{#2}}}

\begin{definition}
  Let \(\TT\) be an extension of the theory of commutative rings with base ring \(R\).
  The \emph{\(\TT\)-spectrum} of a finitely presented \(\TT\)-algebra \(A\) is the set of \(\TT\)-algebra homomorphisms from \(A\) to \(T\).
  \[\Spec_\TT(A) \colonequiv \Alg{\TT}(A,R)\]
\end{definition}

\begin{remark}
  More generally, given an arbitrary \(\TT\)-algebra \(A\) we can consider the \emph{slice theory} \(\TT_{A/}\) and define the spectrum of a \(\TT_{A/}\)-algebra.
  Note that \(\TT \equiv \TT_{R/}\).
  If \(\TT\) was a Fermat theory, the slice theory \(\TT_{A/}\) is a Fermat theory as well by \cite[Proposition 1.5']{DK84}.
\end{remark}

Since every \(\TT\)-algebra morphism \(\varphi : A \to A'\) is a \(\TT\)-algebra morphism \emph{under} the base ring \(R\), meaning it commutes with the unique maps from \(R\) to \(A\) and \(A'\),  it is in particular a homomorphism of \(R\)-algebras between the underlying rings.
Therefore, the \(\TT\)-spectrum of a \(\TT\)-algebra is a subset of its \(R\)-spectrum.

\begin{definition}
  A \emph{Fermat ring (object)} is a commutative ring object whose endomorphism theory is a Fermat theory.
\end{definition}

\begin{example}
  \begin{enumerate}
    \item
      The structure sheaf in the small Zariski topos of a scheme is a Fermat ring?
    \item
      The affine line in the big Zariski topos is a Fermat ring?
    \item
      The real line in the Dubuc topos is a Fermat ring?
  \end{enumerate}
\end{example}


We postulate the existence of a commutative ring \(R\) which satisfies the following axioms:
\begin{description}
  \item[\textbf{(Locality)}]
    The ring \(R\) is local.
  % \item[\textbf{(Differentiability)}]
  %   The ring \(R\) is a Fermat ring.
  %   We write \(\TT\) for its endomorphism theory.
  \item[\textbf{(Duality)}]
    For any finitely presented \(\TT\)-algebra \(A\), the canonical homomorphism
    \[A \to R^{\Spec_\TT(A)}, \quad a \mapsto (\varphi \mapsto \varphi(a))\]
    is an isomorphism of \(\TT\)-algebras.
\end{description}

Duality and locality allows us to prove a weak Nullstellensatz.

\begin{lemma}
  Let \(A\) be a finitely presented \(\EE_R\)-algebra.
  Then \(\Spec_{\EE_R}(A)=\emptyset\) if and only if \(A=0\).
\end{lemma}
\begin{proof}
  Like in synthetic algebraic geometry.
  The if direction uses locality of \(R\).
\end{proof}

As a consequence, we obtain that \(R\) is a field in the following sense.

\begin{lemma}
  An element \(r:R\) is invertible if and only if it does not vanish.
\end{lemma}
\begin{proof}
  In a nontrivial ring, invertible elements do not vanish.
  For the converse, suppose \(r\neq 0\) and consider the quotient \(\EE_R\)-algebra \(R/(r)\).
  Its \(\EE_R\)-spectrum is equivalent to the proposition \(r=0\) by the universal property of quotients and the fact that \(R\) is the free \(\EE_R\)-algebra on zero generators.
  Therefore, by the previous lemma, \(R/(r)\) is the zero ring which implies that \(r\) is invertible.
\end{proof}

This proof simply imitates the proof for the analogous statement in SAG.
Recall that another simple consequence of duality in SAG was that an element \(r:R\) is nilpotent if and only if \(\neg\neg(x=0)\) holds.
While the only-if direction is always true, a simple algebraic fact, the other implication does \emph{not} hold \rednote{Example?} in the context of \(\EE_R\)-duality.


}
