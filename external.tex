\begin{lemma}
  Let $(C, J)$ be a site,
  where the Grothendieck topology $J$ is subcanonical.
  Let
  \[ f : E \twoheadrightarrow \yo(c) \]
  be an epimorphism in $\Sh(C, J)$ with representable codomain.
  Then there is a $J$-cover $(c_i \to c)_{i \in I}$ of $c$
  such that for every $i$,
  the pullback of $f$ along $\yo(c_i) \to \yo(c)$
  is a split epimorphism.
  \[ \begin{tikzcd}
    E_i \ar[r] \ar[d, two heads, "f_i"] \ar[dr, phantom, very near start, "\lrcorner"] &
    E \ar[d, two heads, "f"] \\
    \yo(c_i) \ar[r] \ar[u, bend left, dashed]&
    \yo(c)
  \end{tikzcd} \]
\end{lemma}

\begin{proof}
  By the Yoneda lemma,
  an epimorphism $E \twoheadrightarrow \yo(c)$ is split
  if and only if
  the particular element $\id_c \in \yo(c)(c)$
  is in the image of the map $E(c) \to \yo(c)(c)$.
  Applying the usual characterization of epimorphisms of sheaves
  \cite[Corollary III.7.5]{MaclaneMoerdjik}
  to the element $\id_c \in \yo(c)(c)$
  shows that there is a $J$-cover ${(c_i \xrightarrow{g_i} c)}_{i \in I}$
  such that $f_{c_i}(e_i) = g_i \in \yo(c)(c_i)$
  for some $e_i \in E(c_i)$.
  But this means that $\id_{c_i}$ is in the image of ${(f_i)}_{c_i} : E_i(c_i) \to \yo(c_i)(c_i)$,
  as we can see by evaluating the pullback diagram at $c_i$.
\end{proof}
