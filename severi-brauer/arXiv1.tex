\documentclass[10pt,a4paper]{article}

\usepackage{url}
\usepackage{verbatim}
\usepackage{latexsym}
\usepackage{amssymb,amstext,amsmath,amsthm}
\usepackage{epsf}
\usepackage{epsfig}

\usepackage{tikz-cd}
\usepackage{tikz}
\usetikzlibrary{shapes.geometric}
% \usepackage{isolatin1}
\usepackage{a4wide}
\usepackage{verbatim}
\usepackage{proof}
\usepackage{latexsym}
%\usepackage{mytheorems}

\theoremstyle{definition}
    \newtheorem{theorem}{Theorem}[section]
    \newtheorem{lemma}[theorem]{Lemma}
    \newtheorem{definition}[theorem]{Definition}
    \newtheorem{conjecture}[theorem]{Conjecture}
    \newtheorem{remark}[theorem]{Remark}
    \newtheorem{example}[theorem]{Example}
    \newtheorem{corollary}[theorem]{Corollary}
    \newtheorem{proposition}[theorem]{Proposition}


\usepackage{hyperref}
%\usepackage{cleveref}

\newcommand{\Gr}{\mathrm{Gr}}

\DeclareMathOperator{\PGL}{PGL}
\DeclareMathOperator{\GL}{GL}


\DeclareMathOperator{\Spec}{Spec}

\setlength{\oddsidemargin}{0in} % so, left margin is 1in
\setlength{\textwidth}{6.27in} % so, right margin is 1in
\setlength{\topmargin}{0in} % so, top margin is 1in
\setlength{\headheight}{0in}
\setlength{\headsep}{0in}
\setlength{\textheight}{9.19in} % so, foot margin is 1.5in
\setlength{\footskip}{.8in}


% latexmk -pdf -pvc main.tex
% latexmk -pdf -pvc -interaction=nonstopmode main.tex
%\documentclass{../util/zariski}
\newcommand{\SB}{\mathrm{SB}}
\newcommand{\RI}{\mathrm{RI}}
\newcommand{\AZ}{\mathrm{AZ}}
\newcommand{\propTrunc}[1]{\lVert #1 \rVert}
\newcommand{\bP}{\mathbb{P}}
\newcommand{\N}{\mathbb{N}}
\newcommand{\fib}{\mathrm{fib}}
\newcommand{\Aut}{\mathrm{Aut}}

\usepackage{authblk}
\usepackage{doi}


\begin{document}

\title{Severi-Brauer Varieties in Synthetic Algebraic Geometry}

\author{Thierry Coquand}
\affil{Computer Science and Engineering Department,
University of Gothenburg, Sweden,
\url{coquand@chalmers.se}
}
\author{Hugo Moeneclaey}
\affil{Computer Science and Engineering Department,
University of Gothenburg, Sweden,
\url{hugomo@chalmers.se}
}
\date{\today}
\maketitle


\begin{abstract}
We work in synthetic algebraic geometry. It is an extension of homotopy type theory by three axioms, which can be interpreted in a higher version of the Zariski topos and therefore gives a novel approach to algebraic geometry where a Zariski sheaf of higher groupoids is represented simply as a type. In this context we give a proof of Ch\^atelet's result that a Severi-Brauer variety with a point is a projective space. We make crucial use of étale sheaves, which can be defined and worked with conveniently in this synthetic approach.
\end{abstract}

\tableofcontents


\section*{Introduction}

{F}ran\c cois {C}h\^atelet introduced the notion of Severi-Brauer variety in his 1944 PhD thesis
\cite{chatelet44}. One of his motivation was to provide
a generalisation of Poincarr\'e's result that a conic which has a rational point is isomorphic to $\bP^1(k)$.
He defined a Severi-Brauer variety as a variety which becomes isomorphic to a projective space after
a separable extension. He uses the characterisation of a central simple algebra over a field $k$ as
an algebra which becomes isomorphic to a matrix algebra $M_n(k)$ after a separable extension. He then noticed the fundamental
fact that $\bP^{n}(k)$ and $M_{n+1}(k)$ have the same automorphism group, namely $\PGL_{n+1}(k)$. He used this
to give a correspondence between Severi-Brauer varieties and central simple algebras, and as a corollary
obtained the following generalisation of Poincar\'e's result: a Severi-Brauer variety which has a rational point
is isomorphic to a projective space. This result and its proof are described in Serre's book on local fields \cite{serre62}, as well as in \cite{colliot88} and \cite{gille2017central}. The notion of central simple algebra over a field has been extended over an arbitrary ring under the name of Azumaya algebra \cite{azumaya51}, and the notion of Severi-Brauer variety has been extended over an arbitrary ring by Grothendieck \cite{grothendieck68}.

This article present a formulation and a proof of Ch\^atelet's result over an arbitrary commutative ring
in the setting of synthetic algebraic geometry \cite{draft}, using the results about projective
spaces already proved in this context \cite{sag-projective}. A key such result is the fact that any automorphism of the projective space is given by a homography. We also crucially rely on basic results about dependent type theory with univalence \cite{hott}
and modalities \cite{modalities}, as in our context \'etale sheafification can be defined
as a lex modality. 

Our formulation of Ch\^atelet's result is that given a lex modality $T$ such that schemes are $T$-modal and the type of finite free module is $T$-modal, for all $T$-modal type $X$ we have that:
\[\propTrunc{X=\bP^n}_{T} \to \propTrunc{X} \to \propTrunc{X=\bP^n}\]
where $\propTrunc{X=\bP^n}_{T}$ is the modal replacement of $\propTrunc{X=\bP^n}$. 

We also show that \'etale sheafification is an example of such a $T$.


\section{\'Etale sheaves}
\label{etale-sheaves}

In this section we define étale sheaves, prove that schemes are étale sheaves and prove that given a module $M$ that is an étale sheaf, the proposition ``$M$ is finite free'' is an étale sheaf.  This last statement is equivalent to the type of finite free modules being an étale sheaf, and we call it étale descent for finite free modules.


\subsection{Defining étale sheaves}

Our definition of étale sheaves crucially relies on monic unramifiable polynomials, which are defined in \cite{wraith79} and analysed in \cite{coqazumaya}. For the purpose of this article it is enough to know that monic unramifiable polynomials are non-constant, and that a monic polynomial of degree $n$ with $n$ roots in unramifiable if and only if it has a simple root.

If $P$ is a proposition, we say that a type $X$ is $P$-local if and only if the canonical map $X\to X^P$ is an equivalence.
Given a type $I$ and family of propositions $(P_i)_{i:I}$, the types that are $P_i$-local for all $i:I$ form a model of homotopy type theory, and we have an associated lex modality called the nullification modality \cite{modalities,Quirin16}.
Following \cite{wraith79}, we consider \'etale sheafification, which is the nullification at the family of propositions stating that monic unramifiable polynomials have roots.

\begin{definition}
A type $X$ is called an \'etale sheaf if for all $g:R[X]$ monic unramifiable, we have that $X$ is $(\exists(x:R).\, g(x)=0)$-local.
\end{definition}

As explained above, this means that étale sheaves form a model of homotopy type theory.

%This means that being an \'etale sheaf is a lex modality, as it is localisation at a family of propositions. 

\begin{remark}
  By \cite{wraith79} this should agree with the usual \'etale topology.
  We will never use the unramifiability assumption, we could just use non-constant monic polynomials instead.
\end{remark}


\subsection{Schemes are \'etale sheaves}

First we prove that $R$ is an étale sheaf, immediately implying that all affine schemes are étale sheaves.

\begin{lemma}\label{etale-subcanonical}
The type $R$ is an \'etale sheaf.
\end{lemma}

\begin{proof}
Let $g:R[X]$ be monic and let us write $S=\Spec(R[X]/g)$. We have a coequaliser in sets:
\[S\times S\rightrightarrows S \to \propTrunc{S}\]
Since $R$ is a set we get an equaliser:
\[R^{\propTrunc{S}} \to R^S\rightrightarrows R^{S\times S}\]
Therefore to conclude it is enough to prove that $R$ is the equaliser of:
\[R[X]/g \rightrightarrows R[X]/g \otimes R[X]/g\]
But since $g$ is monic we merely have an $n$ such that:
\[R[X]/g \simeq R^n\]
and it is clear that the equaliser of the corresponding maps $R^n\rightrightarrows R^n\otimes R^n$ (which are $x\mapsto x\otimes 1$ and $x\mapsto 1\otimes x$) is the submodule freely generated by $1$, so it is $R$.
\end{proof}

\begin{remark}\label{R-modal-subcanonical}
If $R$ is modal, then by general reasoning on modalities so is $\mathrm{Hom}(A,R)$ for any $R$-algebra $A$, so that every affine scheme is modal. By duality we also get that every finitely presented algebra is modal.
\end{remark}

Now we want to prove that every scheme is an étale sheaf. We will use freely the terminology and results of \cite{draft}, in particular a proposition is {\em open} if and only if it is merely of the form $r_1\neq 0\vee\dots\vee r_m\neq 0$ for some $r_1,\dots,r_m$ in $R$. We will use the following auxiliary result.

\begin{lemma}\label{scheme-are-sheaf-from-affine}
Assume given a proposition $P$ such that:
\begin{itemize}
\item The type $R$ is $P$-local.
\item Any open proposition is $P$-local.
\item The type of open propositions is $P$-local.
\end{itemize}
Then any scheme is $P$-local.
\end{lemma}

\begin{proof}
Since $R$ is $P$-local, all affine schemes are $P$-local as explained in Remark \ref{R-modal-subcanonical}.

We check that for all scheme $X$, any map $f:P\to X$ merely factors through $1$. Take $(U_i)_{i:I}$ a finite cover of $X$ by affine scheme. Then for any $i:I$ we have that $f^{-1}(U_i)$ is an open in $P$, meaning it is a map from $P$ to open propositions. But the type of open propositions is $P$-local, so we merely have an open proposition $V_i$ such that for all $x:P$ we have that:
\[(x\in f^{-1}(U_i) )\leftrightarrow V_i\]

Since the $f^{-1}(U_i)$ cover $P$, we have that $P\to \lor_{i:I} V_i$, but open propositions are assumed to be $P$-local, so we have $\lor_{i:I} V_i$.
Assume $k:I$ such that $V_k$ holds. Then $f^{-1}(U_k) = P$ and the map $f$ factors through the affine scheme $U_k$. Since affine schemes are $P$-local, we merely have a lift for $f$.

Now we conclude that any scheme is $P$-local by proving that its identity types are $P$-local. Indeed they are schemes and propositions, so the previous reasoning implies that they are $P$-local.
\end{proof}

Now we want to apply this. We need the type of roots of a monic polynomial to be compact. More precisely, we need:

\begin{lemma}\label{roots-monic-proper}
  Given $g$ is a monic polynomial and $h_1,\dots,h_m:R[X]$, the proposition:
  \[
  \forall(x:R).\, (g(x)=0\to h_1(x)\neq 0\vee\dots\vee h_m(x)\neq 0)
  \] 
  is open. It follows that, for any monic $g:R[X]$, and for any open $U$ in $\Spec(R[X]/g)$ the proposition:
  \[\Spec(R[X]/g) = U\]
is open.
\end{lemma}

\begin{proof}
  This follows from  IV-10-2 in \cite{lombardi-quitte}.
\end{proof}

We are now ready to conclude.

\begin{proposition}\label{scheme-is-etale-sheaf}
Any scheme is an \'etale sheaf.
\end{proposition}

\begin{proof}
Assume given $g:R[X]$ monic non-constant, we can apply Lemma \ref{scheme-are-sheaf-from-affine} because:
\begin{itemize}
\item The type $R$ is an \'etale sheaf by Lemma \ref{etale-subcanonical}.
\item Any open proposition $U$ is an \'etale sheaf because if $\propTrunc{\Spec(R[X]/g)}\to U$ then since $\neg\neg\Spec(R[X]/g)$ we have $\neg\neg U$, which implies $U$.
\item Since open propositions are \'etale sheaves, it is enough that any map $\propTrunc{\Spec(R[X]/g)}\to \mathrm{Open}$ merely factors through $1$. But given a constant open $U$ in $\Spec(R[X]/g)$, for any $x:\Spec(R[X]/g)$ we have that:
\[x\in U \leftrightarrow (\Spec(R[X]/g) = U)\]
The right hand side is open by Lemma \ref{roots-monic-proper}, giving the required lift. 
\end{itemize}
\end{proof}


\subsection{Étale descent for finite free modules}

It might be surprising to the reader familiar with traditional algebraic geometry that we get descent for finite free modules instead of finite projective module. This is because we work in the Zariski topos, where both notions agree:

\begin{lemma}\label{finite-projective-free}
Let $M$ be a finite projective module, then $M$ is finite free.
\end{lemma}

\begin{proof}
This is IX.2.2 in \cite{lombardi-quitte}, it crucially relies on $R$ being local.
\end{proof}

Now we gather a string of auxiliary results.

\begin{lemma}\label{fp-stable-etale-tensor}
  Let $A$ be an fppf algebra and let $M$ be an $R$-module. Then if $A\otimes M$ is finitely presented (resp.\ finite projective)
  as an $A$-module if and only if $M$ is finitely presented (resp.\ finite projective) as an $R$-module.
\end{lemma}

\begin{proof}
See VIII.6.7 in \cite{lombardi-quitte}.
\end{proof}

\begin{lemma}\label{fp-equivalent-pointwise}
  Given a finitely presented (resp.\ finite projective)
  $R$-module $M_x$ depending on $x:\Spec(A)$, we have that $\prod_{x:\Spec(A)}M_x$ is a finitely presented (resp.\ finite projective) $A$-module.
\end{lemma}

\begin{proof}
See Theorem 7.2.3 in \cite{draft}.
\end{proof}

\begin{lemma}\label{descent-sqc-etale}
Let $M$ be an $R$-module that is an \'etale sheaf such that we have the \'etale sheafification of ``$M$ is finitely presented'', then for any monic polynomial $g$ we have that:
\[R[X]/g\otimes M \simeq M^{\Spec(R[X]/g)}\]
\end{lemma}

\begin{proof}
We have that $R[X]/g\otimes M$ is merely equal to $M^n$ where $n$ is the degree of $g$, therefore it is an \'etale sheaf. But $M^{\Spec(R[X]/g)}$ is an \'etale sheaf as well, so when proving that the map:
\[R[X]/g\otimes M \to M^{\Spec(R[X]/g)}\]
is an equivalence, we can assume that $M$ is finitely presented. In this case we conclude by Theorem 7.2.3 in \cite{draft}.
\end{proof}

We are now ready to conclude.

\begin{proposition}\label{descent-finite-free}
For $M$ an $R$-module that is an \'etale sheaf, the proposition ``$M$ is a finite free'' is itself an \'etale sheaf.
\end{proposition}

\begin{proof}
By Lemma \ref{finite-projective-free} being finite free is equivalent to being finite projective, so we just need the result for ``$M$ is finite projective''

Assume a monic unramifiable polynomial $g$ such that:
\[\Spec(R[X]/g) \to \textrm{``$M$ is finite projective''}\]
By Lemma \ref{fp-equivalent-pointwise} we get that $M^{\Spec(R[X]/g)}$ is finite projective, but by Lemma \ref{descent-sqc-etale} we have that:
\[M^{\Spec(R[X]/g)} = R[X]/g\otimes M\] 
and we conclude using Lemma \ref{fp-stable-etale-tensor} and the fact that $R[X]/g$ is fppf.
\end{proof}

\begin{remark}
We sketch how Proposition \ref{descent-finite-free} relates to the traditional notion of étale descent, justifying the title of this section. 
A first remark is that it implies that the type of finite free modules is itself an étale sheaf. Now consider a monic separable polynomial $g$, so that $g$ is unramifiable and $\Spec(R[X]/g)$ is étale. Then $\Spec(R[X]/g)\to 1$ is an étale cover. But a descent data in finite free module for this cover is precisely a finite free module depending on 
$\propTrunc{\Spec(R[X]/g)}$, so that Proposition \ref{descent-finite-free} implies that giving such a descent data is equivalent to just giving a finite free module, which is a direct translation of what would be traditionally called étale descent.
\end{remark}



\section{Severi-Brauer varieties and Azumaya algebras}

From now on we assume a lex modality $T$ such that:
\begin{itemize}
\item Schemes are modal.
\item If $M$ is a modal $R$-module, then the proposition ``$M$ is finite free'' is modal.
\end{itemize}
Modal types are called sheaves, the modal replacement is called sheafification and we denote the sheafification of the propositional truncation of $X$ by $\propTrunc{X}_T$. Sheaves form a model of homotopy type theory \cite{modalities,Quirin16}.

We proved in Section \ref{etale-sheaves} that étale sheafification is such a modality, see Propositions \ref{scheme-is-etale-sheaf} and \ref{descent-finite-free}.


\subsection{Severi-Brauer varieties}

\begin{definition}
A type $X$ is called a Severi-Brauer variety of dimension $n$ if $X$ is a sheaf and $\propTrunc{X=\bP^n}_T$.
\end{definition}

We denote the type of Severi-Brauer varieties of dimension $n$ by $\SB_n$.

\begin{remark}
We will see later that every Severi-Brauer variety is a scheme. 
\end{remark}

Now we state the key result that the automorphism group of $\bP^n$ is $\PGL_{n+1}(R)$. Recall that $\PGL_{n+1}(R)$ is defined as the type of automorphism of $R^{n+1}$ (or equivalently of matrices in $M_{n+1}(R)$ with invertible determinant) modulo multiplication by an invertible $\lambda:R$.

\begin{proposition}\label{Aut-Pn-PGL}
The map:
\[\beta:\PGL_{n+1}(R)\to\Aut(\bP^n)\]
\[P\mapsto (X\mapsto PX)\]
is an equivalence.
\end{proposition}

\begin{proof}
This is the main result from \cite{sag-projective}.
\end{proof}

\begin{remark}\label{SB-is-delooping}
Consider $G$ a group, a delooping of $G$ inside sheaves is a pointed sheaf $(\mathrm{B}_TG,*)$ such that $\forall(x:\mathrm{B}_TG).\, \propTrunc{x=*}_T$ and $\Omega (\mathrm{B}_TG) = G$. This is precisely a delooping of $G$ inside the model of homotopy type theory where types are interpreted as sheaves. Then $\SB_n$ is a delooping of $PGL_{n+1}(R)$ inside sheaves.
\end{remark}

We want to give examples of Severi-Brauer varities. To do that we assume that $2\not=0$ and we take $T$ to be the étale sheafification for the rest of the subsection.

\begin{definition}
Given $a,b:R^\times$, we define the conic $C(a,b)$ as the set of $[x:y:z]:\bP^2$ such that $x^2=ay^2+bz^2$.
\end{definition}

\begin{lemma}\label{pointed-conics-projective}
Assume $a,b:R^\times$ such that $\propTrunc{C(a,b)}$, then $\propTrunc{C(a,b)=\bP^1}$.
\end{lemma}

\begin{proof}
Let us assume $x_0,y_0,z_0$ such that $x_0^2 = ay_0^2+bz_0^2$. We can assume $x_0\not=0$ without loss of generality by using $C(a,b) = C(\frac{1}{a},-\frac{b}{a})$ if $y_0\not=0$ and $C(a,b) = C(\frac{1}{b},-\frac{a}{b})$ if $z_0\not=0$. Then we can assume $x_0=1$, so that $ay_0^2 + bz_0^2 = 1$.

 Let us consider the map:
\[\psi:\bP^1\to \bP^2\]
\[[u:v] \mapsto [au^2+bv^2: y_0(au^2-bv^2) + 2buvz_0 : z_0(au^2-bv^2) - 2auvy_0]\]

We want to define $\phi$ inverse to $\psi$. Assume $[x:y:z]:\bP^2$ such that $x^2=ay^2+bz^2$. 

Let us proof that either $x+ay_0y+bz_0z$ or $x-ay_0y-bz_0z$ is invertible. It is enough to prove that either $x$ or $ay_0y+bz_0z$ is invertible, so we assume $x=0$ and $ay_0y+bz_0z=0$ and try to reach a contradiction. We have that $y$ or $z$ is invertible and $ay^2+bz^2=0$, so that $y$ and $z$ are invertible and $b = -a\frac{y^2}{z^2}$ and $y_0z=yz_0$. Moreover $y_0$ or $z_0$ is invertible. If say $z_0$ is invertible then $\frac{y}{z} = \frac{y_0}{z_0}$, so that $ay^2+bz^2=0$ implies $ay_0^2+bz_0^2=0$, which is a contradiction. 

If $x + ay_0y + bz_0z$ is invertible we define $\phi([x,y,z]) = [1:\frac{a(z_0y-y_0z)}{x + ay_0y + bz_0z}]$.

If $x - ay_0y - bz_0z$ is invertible we define $\phi([x,y,z]) = [\frac{b(z_0y-y_0z)}{x - ay_0y - bz_0z}:1]$.

This is well defined as if both are invertible then:
\[\frac{b(z_0y-y_0z)}{x - ay_0y - bz_0z}\times\frac{a(z_0y-y_0z)}{x + ay_0y + bz_0z} = \frac{ab(z_0y-y_0z)^2}{x^2 - (ay_0y + bz_0z)^2} = 1\]
because:
\[x^2 - (ay_0y + bz_0z)^2 = (ay^2+bz^2)(ay_0^2+bz_0^2) - (ay_0y + bz_0z)^2 = ab(z_0y-y_0z)^2\]
We omit the verification that this $\phi$ is indeed an inverse to $\psi$.
\end{proof}

\begin{remark}
We will extend this in Theorem \ref{chatelet-theorem}, which prove that any inhabited Severi-Brauer variety is a projective space. 
\end{remark}

\begin{lemma}\label{conic-one-split}
Assume given $b:R^\times$, then $\propTrunc{C(1,b) = \bP^1}$.
\end{lemma}

\begin{proof}
We just apply Lemma \ref{pointed-conics-projective} to the point $[1,1,0]:C(1,b)$.
\end{proof}

\begin{lemma}\label{conic-change-variable}
Given $a,b,u,v:R^\times$ we have that $C(a,b) = C(u^2a,v^2b)$.
\end{lemma}

\begin{proof}
Consider the change of variable $y\mapsto uy$ and $z\mapsto vz$.
\end{proof}

\begin{lemma}
Assume given $a,b:R^\times$, then $C(a,b)$ is a Severi-Brauer variety.
\end{lemma}

\begin{proof}
We have that $C(a,b)$ is a scheme so it is an étale, then we can assume $\sqrt{a}$ such that $\sqrt{a}^2 = a$ to prove $C(a,b)=\bP^1$. But by Lemma \ref{conic-change-variable} we have that $C(a,b) = C(\sqrt{a}^2,b) = C(1,b)$ and we conclude by Lemma \ref{conic-one-split}.
\end{proof}


\subsection{Azumaya algebras}

\begin{definition}
An Azumaya algebra of rank $n$ is a (non-commutative, unital) $R$-algebra $A$ such that its underlying type is a sheaf and $\propTrunc{A=M_{n+1}(R)}_T$.
\end{definition}

We write $\AZ_n$ for the type of Azumaya algebras of rank $n$.

\begin{lemma}\label{azumayas-are-finite-free}
Given $A:\AZ_n$, we have that $A$ is finite free as a module.
\end{lemma}

\begin{proof}
By hypothesis $A$ being finite free is modal so that $\propTrunc{A=M_{n+1}(R)}_T$ implies $A$ finite free.
\end{proof}

Now we want to prove that the automorphism group of $M_{n+1}(R)$ in $R$-algebras is also $\PGL_{n+1}(R)$. We start with an auxiliary result.

\begin{lemma}\label{fundamental-system-matrices}
Assume $e_{i,j}:M_{n+1}(R)$ for $0\leq i,j\leq n$ such that:
\[e_{i,j}e_{k,l} = \delta_{j,k}e_{i,l}\]
where $\delta_{j,k} = 1$ if $j=k$ and $0$ otherwise. Moreover assume:
\[e_{0,0}+\cdots+e_{n,n}=1\]
Then there exists $P:GL_{n+1}(R)$ such that:
\[e_{i,j} = PE_{i,j}P^{-1}\]
for all $0\leq i,j\leq n$, where the $E_{i,j}$ are the vectors of the canonical basis of $M_{n+1}(R)$.
\end{lemma}

\begin{proof}
We define $e_i = e_{i,i}$, then $e_0+\cdots+e_n = 1$, for all $i$ we have $e_i^2=e_i$ and for all $i\not=j$ we have that $e_ie_j=0$. From this we get:
\[R^{n+1} = V_0\oplus\cdots\oplus V_n\]
where $V_i = \{x\ |\ e_i(x)=x\}$ and $e_{i,j}:V_j\simeq V_i$.

As a direct summand of a free module we have that $V_0$ is projective, and since $V_0 = e_{0}(R^{n+1})$ we have that $V_0$ is finitely generated, so that by Lemma \ref{finite-projective-free} it is finite free. From $V_0^{n+1}=R^{n+1}$ we get that $\propTrunc{V_0=R}$, and therefore that $\propTrunc{V_i=R}$ for all $i$.

Then we choose $v_0$ generating $V_0$ and define $v_i = e_{i,0}(v_0)$ so that $v_i$ generates $V_i$. We get a basis $v_0,\hdots,v_n$ of $R^{n+1}$.

Let $u_0,\hdots,u_n$ be the canonical basis of $R^{n+1}$ and define $P:GL_{n+1}(R)$ by sending $u_i$ to $v_i$. Then for all $v_k$ we have that:
\[e_{i,j}v_k = PE_{i,j}P^{-1}v_k\]
so we can conclude.
\end{proof}

We could also use $M\mapsto PMP^{-1}$ in the next proposition, but our choice will be helpful later.

\begin{proposition}\label{Aut-MnR-PGL}
The map:
\[\alpha:\PGL_{n+1}(R)\to\Aut(M_{n+1}(R))\]
\[P\mapsto (M\mapsto (P^t)^{-1}MP^t)\]
is an equivalence.
\end{proposition}

\begin{proof}
It is clearly a group morphism. 

For injectivity we just need to check that if for all $M:M_{n+1}(R)$ we have $(P^t)^{-1}MP^t=M$ then there exists $\lambda\not=0$ such that $P=\lambda I_{n+1}$. We deduce this from $Pe_{i,j} = e_{i,j}P$ and $P$ invertible.

For surjectivity we can apply Lemma \ref{fundamental-system-matrices} to the $\sigma(E_{i,j})$ to get $Q:GL_{n+1}(R)$ such that:
\[\sigma(E_{i,j}) = QE_{i,j}Q^{-1}\]
so that for all $M:M_{n+1}(R)$ we have that $\sigma(M) = QMQ^{-1}$. Then $P = (Q^t)^{-1}$ works.
\end{proof}

\begin{remark}\label{AZ-is-delooping}
As in Remark \ref{SB-is-delooping}, we have that $\AZ_n$ is also a delooping of $PGL_{n+1}(R)$ inside sheaves.
\end{remark}

Now we give examples of Azumaya algebras. To do this we assume $2\not=0$ and we take $T$ to be the étale sheafification for the rest of the subsection.

\begin{definition}
Given $a,b:R^\times$, we define the quaternion algebra $Q(a,b)$ as the non-commutative algebra $R[i,j]/(i^2=a,j^2=b,ij=-ji)$.
\end{definition}

\begin{remark}
As an $R$-module, $Q(a,b)$ is free of dimension $4$ generated by $1,i,j,ij$.
\end{remark}

\begin{remark}
By the change of variable $i\mapsto j$ and $j\mapsto i$ we get $Q(a,b) = Q(b,a)$.
\end{remark}

\begin{lemma}\label{quaternion-split}
For all $b:R^\times$, we have that $Q(1,b) = M_2(R)$.
\end{lemma}

\begin{proof}
We send $i$ to:
\[I = \begin{pmatrix}
1 & 0\\
0 & -1\\
\end{pmatrix}\]
and $j$ to:
\[J = \begin{pmatrix}
0 & b\\
1 & 0\\
\end{pmatrix}\]
Then $ij$ is send to:
\[IJ = \begin{pmatrix}
0 & b\\
-1 & 0\\
\end{pmatrix}\]
It is easy to check this defines an algebra morphism, and since $1,I,J,IJ$ form a basis of $M_2(R)$ the map is an isomorphism.
\end{proof}

\begin{lemma}\label{quaternion-change-variable}
For all $a,b,u,v:R^\times$, we have that $Q(a,b) = Q(u^2a,v^2b)$.
\end{lemma}

\begin{proof}
We use the change of variable $i\mapsto ui$ and $j\mapsto vj$.
\end{proof}

\begin{lemma}
Given $a,b:R^\times$, we have that $Q(a,b)$ is an Azumaya algebra. 
\end{lemma}

\begin{proof}
We have that $Q(a,b)$ is finite free as a vector space so it is a sheaf. So we can assume $\sqrt{a}$ such that $\sqrt{a}^2 = a$ to prove that $Q(a,b) = M_2(R)$. Then by Lemma \ref{quaternion-change-variable} we have $Q(a,b) = Q(\sqrt{a}^2,b) = Q(1,b)$ and we conclude by Lemma \ref{quaternion-split}.
\end{proof}


\subsection{A remark on Azumaya algebras}

\begin{lemma}\label{MnR-endomorphism-multiplication}
For any $n:\N$, the map:
\[M_{n+1}(R)\otimes M_{n+1}(R)^{op}\to \mathrm{End}_R(M_{n+1}(R))\]
\[M\otimes N\mapsto (P\mapsto MPN)\]
is an equivalence.
\end{lemma}

\begin{proof}
Let us denote by $(E_{i,j})_{0\leq i,j\leq n}$ the canonical basis of $M_{n+1}(R)$. We consider the basis: 
\[(E_{i,j}\otimes E_{k,l})_{0\leq i,j,k,l\leq n}\] 
of $M_{n+1}(R)\otimes M_{n+1}(R)^{op}$, as well as the basis:
\[(C_{i,j,k,l})_{0\leq i,j,k,l\leq n}\] 
of $\mathrm{End}_R(M_{n+1}(R))$ where $C_{i,j,k,l}(E_{m,n}) = \delta_{j,m}\delta_{k,n} E_{i,l}$. It is clear that the morphism sends one basis to the other. We can then check that both algebras have the same multiplication table. %, indeed:
%\[(E_{i,j}\otimes E_{k,l}) (E_{i',j'}\otimes E_{k',l'}) = \delta_{j,i'}\delta_{k,l'}E_{i,j'}\otimes E_{k',l}\]
%\[C_{i,j,k,l}C_{i',j',k',l'} = \delta_{j,i'}\delta_{k,l'}C_{i,j',k',l}\]
\end{proof}

\begin{lemma}
Given $A:\AZ_n$, the map $A\otimes A^{op}\to \mathrm{End}_R(A)$ sending $a\otimes b$ to $c\mapsto acb$ is an equivalence.
\end{lemma}

\begin{proof}
We have that $A$ is a finite free module by Lemma \ref{azumayas-are-finite-free}. Then both $A\otimes A^{op}$ and $\mathrm{End}_R(A)$ are finite free modules and therefore they are sheaves, so that the map being an equivalence is itself a sheaf and we can assume $A=M_{n+1}(R)$. Then we conclude by Lemma \ref{MnR-endomorphism-multiplication}.
\end{proof}

\begin{remark}\label{azumaya-independent-modality}
A constructive proof of the converse when $T$-modal types are included in étale sheaves is given in \cite{coqazumaya}. This means that an $R$-algebra $A$ is an Azumaya algebra of rank $n$ for such a $T$ if and only if $A$ is free of rank $(n+1)^2$ as a module and the canonical map $A\otimes A^{op}\to \mathrm{End}_R(A)$ is an isomorphism. In particular this means that $\AZ_n$ do not depend on the choice of such a $T$.
\end{remark}

%% \begin{remark}\label{azumaya-independent-modality}
%% We expect a converse in the sense that if $A$ is an algebra which is finite free as a module, such that the map $A\otimes A^{op} \to \mathrm{End}_R(A)$ is an equivalence, we should get $\propTrunc{A=M_{n+1}(R)}_{\mathrm{Et}}$. This would mean that given any modality $T$ such that:
%% \begin{itemize}
%% \item Schemes are $T$-modal.
%% \item The type of finite free modules is $T$-modal.
%% \item $T$-modal types are étale sheaves.
%% \end{itemize}
%% We have that an algebra $A$ is an Azumaya algebra for $T$ if and only if $A$ finite free as a module and the map $A\otimes A^{op} \to \mathrm{End}_R(A)$ is an equivalence. In particular Azumaya algebras do not depend on the choice of such a $T$.
%% \end{remark}



\section{Constructing an equivalence from $\AZ_n$ to $\SB_n$}

By Remarks \ref{SB-is-delooping} and \ref{AZ-is-delooping}, we have that both $\SB_n$ and $\AZ_n$ are deloopings of $PGL_{n+1}(R)$ inside sheaves. Since deloopings are unique in homotopy type theory, and sheaves form a model of homotopy type theory, we can immediately conclude that $\SB_n = \AZ_n$. But this is not enough to conclude Ch\^atelet's result, as we need some control on how this equivalence looks like. In section we give an explicit description of such an equivalence.


\subsection{Constructing a Severi-Brauer variety from an Azumaya algebra}

First we define Grassmanians.

\begin{definition}
Given a module $V$ and $k:\N$, we define $\Gr_k(V)$ as the type of submodule of $V$ which are finite free of dimension $k$.
\end{definition}

\begin{lemma}\label{grassmanians-are-schemes}
Let $V$ be a finite free module, then $\Gr_k(V)$ is a scheme.
\end{lemma}

\begin{proof}
We can assume $V=R^n$. The type of $k$-dimensional subspaces of $R^n$ is the type of $n\times k$ matrices of rank $k$ quotiented by the natural action of $\GL_k$. For all $k\times k$ minor, we consider the open proposition stating that this minor is invertible, which is well defined as it is invariant under the $\GL_k$-action. This gives a finite open cover of $\Gr_k(R^n)$.

Let us show this any such open is affine. For example consider the subtype of matrices of the form:
\[\begin{pmatrix}
P & N
\end{pmatrix}\]
where $P$ is invertible of size $k\times k$. Any orbit in this subtype has a unique element of the form:
\[\begin{pmatrix}
I_k & N'
\end{pmatrix}\]
where $I_k$ is the identity matrix, so this subtype is equivalent to $R^{(n-k)k}$.
\end{proof}

\begin{lemma}\label{being-ideal-in-azumaya-closed}
For all $A:\AZ_n$ and $I:\Gr_{n+1}(A)$, we have that $I$ being a left ideal in $A$ is a closed proposition.
\end{lemma}

\begin{proof}
By Lemma \ref{azumayas-are-finite-free} we have that $A$ is finite free as a module. Consider $a_0,\hdots,a_n$ a basis of $I$ and extend it to a basis of $A$ by adding $b_1,\hdots,b_l$. We can do this as if $R$ was a field because a non-zero vector has an invertible coefficient \cite{draft}.

For any $a:A$, we have that $a\in I$ is a closed proposition as it says that the $b_1,\hdots,b_l$ coordinates of $a$ are $0$. Then $I$ is a left ideal if and only if for any $a$ in the chosen basis of $A$ and any $i$ in the chosen basis of $I$ we have that $ai\in I$, which is a closed proposition.
\end{proof}

We are now ready to give the definition of the map that will turn out to be an equivalence.

\begin{definition}
For all $A:\AZ_n$ we define:
\[\RI_n(A) = \{I:\Gr_{n+1}(A)\ |\ I\ \mathrm{is\ a\ left\ ideal}\}\]
\end{definition}

\begin{lemma}\label{severi-brauer-are-schemes}
For all $A:\AZ_n$ we have that $\RI_n(A)$ is a scheme.
\end{lemma}

\begin{proof}
By Lemma \ref{azumayas-are-finite-free} we have that $A$ is finite free as a module, so that by Lemma \ref{grassmanians-are-schemes} we have that $\Gr_{n+1}(A)$ is a scheme, and then by Lemma \ref{being-ideal-in-azumaya-closed} we have that $\RI_n(A)$ is a closed subtype of a scheme, so it is a scheme.
\end{proof}

\begin{lemma}\label{right-ideal-of-matrices-are-projective}
Consider the map:
\[\delta:\bP^n \to \RI_n(M_{n+1}(R))\]
defined by:
\[\delta([x_0:\cdots:x_n]) = \{M:M_n(R)\ |\ \forall i,j.\ x_i\cdot M_j = x_j\cdot M_i\}\]
where $M_i$ is the $i$-th column of $M$. Then $\delta$ is an equivalence.
\end{lemma}

\begin{proof}
Let us denote $[x_0:\cdots:x_n]:\bP^n$ by $X$. First we check $\delta$ is well defined. It is clear that for all $\lambda:R$ invertible we have that:
\[\delta(\lambda X) = \delta(X)\]
and that $\delta(X)$ is a left ideal. To check the dimension we can assume $x_k\not=0$. Then $M\in\delta(X)$ if and only if for all $i$ we have that $M_i = \frac{x_i}{x_k} M_k$, which means giving $M\in\delta(X)$ is equivalent to giving $M_k$ in $R^{n+1}$, so $\delta(X)$ is free of dimension $n+1$.

Next we check injectivity. Assume given $[x_0:\cdots:x_n]$ and $[y_0:\cdots:y_n]$ in $\bP^n$ such that for all $M:M_n(R)$ we have:
\[(\forall i,j.\ x_i\cdot M_j = x_j\cdot M_i) \leftrightarrow (\forall i,j.\ y_i \cdot M_j = y_j\cdot M_i)\]
In particular considering the matrix $N$ such that its $j$-th column $N_j$ is $(y_j,\hdots,y_j)$ we get that:
\[\forall i,j.\ x_iy_j=x_jy_i\] 
so that $[x_0:\cdots:x_n] = [y_0:\cdots:y_n]$.

Finally we check surjectivity. Assume $I:\RI_n(M_{n+1}(R))$, since $\propTrunc{I=R^{n+1}}$ there exists $M\in I$ such that $M\not=0$, for example assume $M_{0,0}\not=0$. Then for all $k$ we have $E_{k,0}M\in I$. Let us denote the matrix $E_{k,0}M$ by $N_k$, so that $N_k\in I$ and $N_k$ has the first line of $M$ as its $k$-th line, and $0$ elsewhere. Then since $M_{0,0}\not=0$ the matrices $(N_k)_{0\leq k\leq n}$ are linearly independent and since $I$ has dimension $n+1$, it is precisely the ideal spanned by  $(N_k)_{0\leq k\leq n}$. But this ideal is $\delta([M_{0,0}:\cdots:M_{n,0}])$.
\end{proof}

\begin{lemma}
We have that $\RI_n$ gives a pointed map from $\AZ_n$ to $\SB_n$.
\end{lemma}

\begin{proof}
By Lemma \ref{severi-brauer-are-schemes} and the assumption that schemes are sheaves, we have that $\RI_n(A)$ is a sheaf. Then to prove:
\[\propTrunc{A=M_{n+1}(R)}_T \to \propTrunc{\RI_n(A)=\bP^n}_T\]
it is enough to prove that the map is pointed, i.e. that $\RI_n(M_{n+1}(R)) = \bP^n$. This is Lemma \ref{right-ideal-of-matrices-are-projective}.
\end{proof}


\subsection{Generalities on deloopings in sheaves}

\begin{definition}
A type $A$ is called $T$-connected if we have that $\propTrunc{x=y}_T$ for all $x,y:A$.
\end{definition}

A key intuition for next lemma is that both $A$ and $B$ are deloopings of the same group inside sheaves.

\begin{lemma}\label{deloopings-equivalence}
Assume $A$ and $B$ pointed $T$-connected sheaves. Let $f:A\to B$ be a pointed map such that the induced map $\Omega f : \Omega A \to \Omega B$ is an equivalence. Then $f$ is an equivalence.
\end{lemma}

\begin{proof}
First we prove that $f$ is an embedding. We have to prove that for all $x,y:A$ the map:
\[\mathrm{ap}_f : x=y \to f(x)=f(y)\]
is an equivalence. Since $A$ and $B$ are sheaves so are their identity types, so $\mathrm{ap}_f$ being an equivalence is a sheaf, so by $T$-connectedness of $A$ and $B$ we can assume that $x$ and $y$ are the basepoints, and conclude from $\Omega f$ being an equivalence.

Now we prove $f$ is surjective. For any $x:B$, we have that $\fib_f(x)$ is a sheaf and a proposition so when proving it is inhabited we can assume $x$ is the basepoint of $B$. Then the basepoint of $A$ is in the corresponding fiber.
\end{proof}



\subsection{The Severi-Brauer construction is an equivalence}

Next lemma is where our choice of map $\PGL_{n+1}(R)\to \Aut(M_{n+1}(R))$ comes in handy.

\begin{proposition}\label{right-ideal-is-equivalence}
The pointed map $\RI_n:\AZ_n\to\SB_n$ is an equivalence.
\end{proposition}

\begin{proof}
By Lemma \ref{deloopings-equivalence} it is enough to prove that the top map in the triangle:
\begin{center}
\begin{tikzcd}
\Aut(M_{n+1}(R))\ar[rr,"\Omega\RI_n"] & & \Aut(\bP^n) \\
& \PGL_{n+1}(R)\ar[ru,swap,"\beta"]\ar[lu,"\alpha"] & 
\end{tikzcd}
\end{center}
is an equivalence. But since the two other maps in the triangle are equivalences by Propositions \ref{Aut-Pn-PGL} and \ref{Aut-MnR-PGL}, it is enough to prove that the triangle commutes. To do this we need to check that for all $P:\PGL_{n+1}(R)$ we have that:
\[\delta^{-1}\circ \mathrm{ap}_{\RI_n}(\alpha(P))\circ\delta = \beta(P)\]
in $\Aut(\bP^n)$, with $\delta$ defined in Lemma \ref{right-ideal-of-matrices-are-projective}. So we need to prove the following square commutes:
\begin{center}
\begin{tikzcd}
\RI_n(M_{n+1}(R))\ar[rr,"I\mapsto (P^t)^{-1}IP^t"]&& \RI_n(M_{n+1}(R)) \\
\bP^n\ar[u,"\delta"]\ar[rr,swap,"X\mapsto PX"]&& \bP^n\ar[u,swap,"\delta"] 
\end{tikzcd}
\end{center}
where $\mathrm{ap}_{\RI_n}$ was computed using path induction.

We see that:
\[\delta(X) = \{M:M_n(R)\ |\ \forall (A,B:R^{n+1}).\, X^tA\cdot MB = X^tB\cdot MA\}\]
To check that:
\[(P^t)^{-1}\delta(X)P^t = \delta(PX)\]
we just need to check an inclusion as both are finite free modules of the same dimension. Assume $M\in\delta(X)$, to check that $(P^t)^{-1}MP^t\in\delta(PX)$ we need to check that for all $A,B:R^{n+1}$ we have that:
\[(PX)^tA \cdot (P^t)^{-1}MP^tB = (PX)^tB \cdot (P^t)^{-1}MP^tA \]
but since $M\in\delta(X)$ we have that:
\[X^tP^tA \cdot MP^tB = X^tP^tB\cdot MP^tA \]
which gives us what we want.
\end{proof}

\begin{remark}
By Lemmas \ref{severi-brauer-are-schemes} and \ref{right-ideal-is-equivalence} we can conclude than any Severi-Brauer variety is a scheme. This was not clear a priori because for $X$ a sheaf, the proposition ``$X$ is a scheme'' is a not itself a sheaf.
\end{remark}

\begin{remark}\label{severi-brauer-independent-modality}
By Remark \ref{azumaya-independent-modality} and Lemma \ref{right-ideal-is-equivalence}, we can conclude that given $T$ such that $T$-modal types are included in étale sheaf, a type $X$ is a Severi-Brauer variety for a $T$ if and only if it is a Severi-Brauer variety for étale sheafification. So Severi-Brauer varieties do not depend on the choice of such a $T$.
\end{remark}


\section{Inhabited Severi-Brauer varieties are projective spaces}

Now that we have a pointed equivalence $\RI_n:\AZ_n \to \SB_n$, our plan is to assume $X:\SB_n$ with $x:X$ and try to conclude that $\AZ_n^{-1}(X) = M_{n+1}(R)$. Next lemma does just that.

\begin{lemma}\label{azumaya-with-right-ideal}
Assume $A:\AZ_n$ with $I:\RI_n(A)$, then $A = \mathrm{End}_R(I)$.
\end{lemma}

\begin{proof}
Since $I$ is a left ideal, there is a canonical map of algebra:
\[A \to\mathrm{End}_R(I)\]
Since both algebras are sheaves (indeed $\propTrunc{I=R^{n+1}}$ implies that $I$ is a sheaf), this map being an equivalence is a sheaf so we can assume $A=M_{n+1}(R)$.

By Lemma \ref{right-ideal-of-matrices-are-projective} we can assume $X=[x_0:\cdots:x_n]$ in $\bP^n$ such that $I=\delta(X)$. There is a $k$ such that $x_k\not=0$, then there exists an isomorphism:
\[\theta:R^{n+1}\to I\]
sending $Y:R^{n+1}$ to the matrix $N$ with its $i$-th column $N_i=x_iY$. Then for all $M:M_n(R)$ we have a commutative square:
\begin{center}
\begin{tikzcd}
I\ar[rr,"N\mapsto MN"] && I \\
R^{n+1}\ar[u,"\theta"]\ar[rr,swap,"Y\mapsto MY"] && R^{n+1}\ar[u,swap,"\theta"]\\
\end{tikzcd}
\end{center}
meaning that the natural map:
\[ M_{n+1}(R)\to \mathrm{End}_R(I)\]
sends $M$ to $\theta\circ M\circ\theta^{-1}$, so it is an equivalence.
\end{proof}

\begin{theorem}[Ch\^atelet's Theorem]\label{chatelet-theorem}
Assume $X:\SB_n$, then:
\[\propTrunc{X}\to\propTrunc{X=\bP^n}\]
\end{theorem}

\begin{proof}
By Proposition \ref{right-ideal-is-equivalence} we can assume $X=\RI_n(A)$ for some $A:\AZ_n$. Then we can assume $I:\RI_n(A)$, so that by Lemma \ref{azumaya-with-right-ideal} we have that:
\[A=\mathrm{End}_R(I)\]
Since we merely have that $I=R^{n+1}$, we merely have:
\[A = M_{n+1}(R)\]
Applying Lemma \ref{right-ideal-of-matrices-are-projective} we merely conclude:
\[X=\RI_n(A)=\RI_n(M_{n+1}(R)) = \bP^n\]
\end{proof}

\begin{thebibliography}{10}

\bibitem{azumaya51}
Gor\^o Azumaya.
\newblock On maximally central algebras.
\newblock {\em Nagoya Math. J.}, 2:119--150, 1951.

\bibitem{chatelet44}
Fran\c~cois Ch\^atelet.
\newblock {\em Variations sur un th\`eme de {H}. {P}oincar\'e}.
\newblock Annales scientifiques de l'\'Ecole Normale Sup\'erieure, S\'erie 3,
  Volume 61, 1944.

\bibitem{draft}
Felix {Cherubini}, Thierry {Coquand}, and Matthias {Hutzler}.
\newblock A foundation for synthetic algebraic geometry.
\newblock {\em Mathematical Structures in Computer Science}, 34(9):1008--1053,
  2024.

\bibitem{sag-projective}
Felix Cherubini, Thierry Coquand, Matthias Hutzler, and David Wärn.
\newblock Projective space in synthetic algebraic geometry.
\newblock 2024.

\bibitem{colliot88}
Jean-Louis Colliot-Th\'el\`ene.
\newblock Les grands th\`emes de {F}ran\c cois {C}h\^atelet.
\newblock {\em Enseign. Math. (2)}, 34(3-4):387--405, 1988.

\bibitem{coqazumaya}
Thierry Coquand, Henri Lombardi, and Stefan Neuwirth.
\newblock Constructive remarks on azumaya algebra, 2023.

\bibitem{gille2017central}
Philippe Gille and Tam{\'a}s Szamuely.
\newblock {\em Central simple algebras and Galois cohomology}, volume 165.
\newblock Cambridge University Press, 2017.

\bibitem{grothendieck68}
Alexander Grothendieck.
\newblock Le groupe de {B}rauer. {I}. {A}lg\`ebres d'{A}zumaya et
  interpr\'etations diverses.
\newblock In {\em Dix expos\'es sur la cohomologie des sch\'emas}, volume~3 of
  {\em Adv. Stud. Pure Math.}, pages 46--66. North-Holland, Amsterdam, 1968.

\bibitem{lombardi-quitte}
Henri Lombardi and Claude Quitt{\'{e}}.
\newblock {\em Commutative Algebra: Constructive Methods}.
\newblock Springer Netherlands, 2015.

\bibitem{hott}
The Univalent~Foundations Program.
\newblock {\em Homotopy type theory: Univalent foundations of mathematics}.
\newblock \url{https://homotopytypetheory.org/book}, 2013.

\bibitem{Quirin16}
Kevin Quirin.
\newblock {\em Lawvere-Tierney sheafification in Homotopy Type Theory.
  (Faisceautisation de Lawvere-Tierney en th{\'{e}}orie des types
  homotopiques)}.
\newblock PhD thesis, {\'{E}}cole des mines de Nantes, France, 2016.

\bibitem{modalities}
Egbert Rijke, Michael Shulman, and Bas Spitters.
\newblock {Modalities in homotopy type theory}.
\newblock {\em {Logical Methods in Computer Science}}, {Volume 16, Issue 1},
  January 2020.

\bibitem{serre62}
Jean-Pierre Serre.
\newblock Corps locaux.
\newblock Publications de l'{Institut} de {Math{\'e}matique} de
  l'{Universit{\'e}} de {Nancago}. 8; {Actualit{\'e}s} {Scientifiques} et
  {Industrielles}. 1296. {Paris}: {Hermann} \& {Cie}. 243 p., 1962.

\bibitem{wraith79}
G.~C. Wraith.
\newblock Generic {Galois} theory of local rings.
\newblock Applications of sheaves, {Proc}. {Res}. {Symp}., {Durham} 1977,
  {Lect}. {Notes} {Math}. 753, 739-767, 1979.

\end{thebibliography}



\end{document}











