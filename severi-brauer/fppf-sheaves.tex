\label{fppf-sheaves}

\subsection{Generalities}

\begin{definition}
A type $X$ is called an fppf sheaf if for all $g:R[X]$ monic we have that $X$ is $\propTrunc{\Spec(R[X]/g)}$-local.
\end{definition}

This means that being an fppf sheaf is a lex modality, as it is localisation at a family of propositions.



\subsection{Decent for affine schemes and finitely presented algebras}

\begin{lemma}\label{fppf-subcanonical}
The type $R$ is an fppf sheaf. In other words, the fppf topology is subcanonical.
\end{lemma}

\begin{proof}
Let $g:R[X]$ be monic and write $S=\Spec(R[X]/g)$. We have a coequaliser in sets:
\[S\times S\rightrightarrows S \to \propTrunc{S}\]
So since $R$ is a set we have an equaliser diagram:
\[R^\propTrunc{S} \to R^S\rightrightarrows R^{S\times S}\]
so that it is enough to prove that $R$ is the equaliser of:
\[R[X]/g \rightrightarrows R[X]/g \otimes R[X]/g\]
to conclude. But since $g$ is monic we merely have:
\[R[X]/g \simeq R^n\]
and it is clear that $R$ is the equaliser of:
\[R^n \rightrightarrows R^n\otimes R^n\]
\end{proof}

\begin{lemma}
TODO descent affine schemes
\end{lemma}


\subsection{Schemes are fppf sheaves}

\begin{proposition}
Any scheme is an fppf sheaf.
\end{proposition}

\begin{proof}
TODO
\end{proof}


\subsection{Descent for finite free modules}

\begin{lemma}\label{descent-fp-fcop}
Any finitely presented (resp. finitely copresented) module is an fppf sheaf.

The type of finitely presented (resp. finitely copresented) modules is an fppf sheaf.
\end{lemma}

\begin{proof}
For the first part we just use \cref{fppf-subcanonical} and the fact that $M=M^{\star\star}$ for $M$ finitely presented of finitely copresented.

TODO
\end{proof}

\begin{proposition}
Let $M$ be a module that is an fppf sheaf, then $M$ being finite free is an fppf-sheaf.
\end{proposition}

\begin{proof}
Since $R$ is local being finite free is equivalent to being finitely presented and finitely copresented \cite{TODO}. But both are fppf sheaves by \cref{descent-fp-fcop} so we can conclude.
\end{proof}