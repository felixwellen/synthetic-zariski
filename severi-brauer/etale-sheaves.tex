\label{etale-sheaves}

\subsection{Affine schemes are \'etale sheaves}

%\begin{definition}
%Unramifiable polynomial
%\end{definition}

\rednote{Should we recall the definition? I don't think so as it is never used. TODO}

Monic unramifiable polynomials are defined in \cite{wraith79}.

\begin{definition}
A type $X$ is called an \'etale sheaf if for all $g:R[X]$ monic unramifiable, we have that $X$ is $\propTrunc{\Spec(R[X]/g)}$-local.
\end{definition}

This means that being an \'etale sheaf is a lex modality, as it is localisation at a family of propositions. 

\begin{remark}
By \cite{wraith79} this should agree with the usual \'etale topology. It should also be noted that we will never use the unramifiability assumption, so we could just use non-constant monic polynomials instead.
\end{remark}

\begin{lemma}\label{etale-subcanonical}
The type $R$ is an \'etale sheaf.
\end{lemma}

\begin{proof}
Let $g:R[X]$ be monic and write $S=\Spec(R[X]/g)$. We have a coequaliser in sets:
\[S\times S\rightrightarrows S \to \propTrunc{S}\]
So since $R$ is a set we have an equaliser diagram:
\[R^{\propTrunc{S}} \to R^S\rightrightarrows R^{S\times S}\]
so that it is enough to prove that $R$ is the equaliser of:
\[R[X]/g \rightrightarrows R[X]/g \otimes R[X]/g\]
to conclude. But since $g$ is monic we merely have:
\[R[X]/g \simeq R^n\]
and it is clear that $R$ is the equaliser of:
\[R^n \rightrightarrows R^n\otimes R^n\]
\end{proof}

\begin{remark}\label{R-modal-subcanonical}
If $R$ is modal, then so is $\mathrm{Hom}(A,R)$ for any $R$-algebra $A$ by general reasoning on modalities, so that every affine scheme is modal. By duality this implies that every finitely presented algebra is modal.
\end{remark}


\subsection{Schemes are \'etale sheaves}

\begin{lemma}\label{scheme-are-sheaf-from-affine}
Assume given a proposition $P$ such that:
\begin{itemize}
\item The type $R$ is $P$-local.
\item Any open proposition is $P$-local.
\item The type of open propositions is $P$-local.
\end{itemize}
Then any scheme is $P$-local.
\end{lemma}

\begin{proof}
Since $R$ is $P$-local, all affine schemes are $P$-local as explained in \Cref{R-modal-subcanonical}.

We check that for all scheme $X$, any map:
\[f:P\to X\]
merely factors through $1$. Take $(U_i)_{i:I}$ a finite cover of $X$ by affine scheme. Then for any $i:I$ we have that $f^{-1}(U_i)$ is an in $P$, so since the type of open is $P$-local, we merely have an open proposition $V_i$ such that for all $x:P$, we have:
\[(x\in f^{-1}(U_i) )\leftrightarrow V_i\]
Since the $f^{-1}(U_i)$ cover $P$, we have that:
\[P\to \lor_{i:I} V_i\]
But open propositions are assumed to be $P$-local, so we have that:
\[ \lor_{i:I} V_i\]
Assume $k:I$ such that $V_k$ holds. Then $f^{-1}(U_k) = P$ and the map $f$ factors through the affine scheme $U_k$. Since affine schemes are $P$-local, we merely have a lift for $f$.

Now we conclude that any scheme is $P$-local by proving that its identity types are $P$-local. Indeed they are schemes, so the previous point implies they are $P$-local.
\end{proof}

\rednote{Next lemma surely has a more direct, algebraic proof. TODO}

\begin{lemma}\label{roots-monic-proper}
For any monic $g:R[X]$, we have that $\Spec(R[X]/g)$ is projective. In particular it is compact, meaning that for any open $U$ in $\Spec(R[X]/g)$ the proposition:
\[\prod_{x:\Spec(R[X]/g)}U(x)\]
is open.
\end{lemma}

\begin{proof}
Assume that:
\[g=X^n+a_{n-1}X^{n-1}+\cdots+a_0\]
Then we consider the homogeneous polynomial:
\[f(X,Y) = X^n + a_{n-1}X^{n-1}Y+\cdots+a_0Y^n\]
We prove that:
\[\sum_{[x,y]:\bP^1}f(x,y) = 0\]
is equivalent to $\Spec(R[X]/g)$. Indeed for any $x,y:R$ such that $f(x,y)=0$, we have that $x\not=0$ implies $y\not=0$, so that $(x,y)\not=0$ implies $y\not=0$. Then:
\[\sum_{[x,y]:\bP^1}f(x,y) = 0\]
is equivalent to:
\[\sum_{x:R} f(x,1)=0\]
which is the type of roots of $g$. 

Now we conclude using the fact that 
\[\sum_{[x,y]:\bP^1}f(x,y) = 0\]
is closed in the compact scheme $\bP^1$, so that it is compact.
\end{proof}

\begin{proposition}\label{scheme-is-etale-sheaf}
Any scheme is an \'etale sheaf.
\end{proposition}

\begin{proof}
Assume given $g:R[X]$ monic, we can apply \Cref{scheme-are-sheaf-from-affine} because:
\begin{itemize}
\item The type $R$ is an \'etale sheaf by \Cref{etale-subcanonical}.
\item Any open proposition $U$ is an \'etale sheaf because if:
\[\propTrunc{\Spec(R[X]/g)}\to U\]
then since $\neg\neg\Spec(R[X]/g)$ we have $\neg\neg U$, which implies $U$.
\item Since open propositions are \'etale sheaves, it is enough that any map:
\[\propTrunc{\Spec(R[X]/g)}\to \mathrm{Open}\]
merely factors through $1$. But given a constant open $U$ in $\Spec(R[X]/g)$, for any $x:\Spec(R[X]/g)$ we have that:
\[x\in U \leftrightarrow \prod_{y:\Spec(R[X]/g)} y\in U)\]
The right hand side is open by \Cref{roots-monic-proper}, giving the required lift. 
\end{itemize}
\end{proof}


\subsection{Descent for finite free modules}

\rednote{Maybe give a clearer proof, for finite free only? TODO}

\begin{lemma}\label{fp-equivalent-pointwise}
If we have $M_x$ a finitely presented $R$-module depending on $x:\Spec(A)$, then $\prod_{x:\Spec(A)}M_x$ is a finitely presented $A$-module.
\end{lemma}

\begin{proof}
See Theorem 7.2.3 in \cite{draft}.
\end{proof}

\begin{lemma}\label{descent-sqc-etale}
Let $M$ be a module that is an \'etale sheaf such that we have the \'etale sheafification of "$M$ is f.p.", then for any monic $g$ we have that:
\[R[X]/g\otimes M \simeq M^{\Spec(R[X]/g)}\]
\end{lemma}

\begin{proof}
We have that $R[X]/g\otimes M$ is merely equal to $M^n$ where $n$ is the degree of $g$, therefore it is an \'etale sheaf. As $M^{\Spec(R[X]/g)}$ is an \'etale sheaf as well, so when proving that:
\[R[X]/g\otimes M \to M^{\Spec(R[X]/g)}\]
is an equivalence, we can assume that $M$ is finitely presented. In this case we conclude by Theorem 7.2.3 in \cite{foundation-sag}.
\end{proof}

\begin{lemma}\label{fp-stable-etale-tensor}
Let $A$ be an fppf algebra and let $M$ be an $R$-module. Then if $A\otimes M$ is f.p. as an $A$-module if and only if $M$ is f.p. as an $R$-module.
\end{lemma}

\begin{proof}
See VIII.6.7 in \cite{lombardi-quitte}.
\end{proof}

\begin{lemma}\label{descent-fp-fcop}
Any finitely presented (resp. finitely copresented) module is an \'etale sheaf.

For $M$ a module that is an \'etale sheaf, the proposition "$M$ is an \'etale sheaf" is itself an \'etale sheaf. %The type of finitely presented (resp. finitely copresented) modules is an \'etale sheaf.
\end{lemma}

\begin{proof}
For the first part we just use \Cref{etale-subcanonical} and the fact that $M=M^{\star\star}$ for $M$ finitely presented or finitely copresented \cite{TODO-diffgeo}.

By duality it is enough to treat the finitely presented case. Assume $M$ a module that is an \'etale sheaf, and $g$ monic with $A=R[X]/g$ such that:
\[\Spec(A)\to M\ \mathrm{is\ f.p.}\]
Then by \Cref{fp-equivalent-pointwise} we have that $M^{\Spec(A)}$ is an f.p. $A$-module. By \Cref{descent-sqc-etale} we have that:
\[A\otimes M \simeq M^{\Spec(A)}\]
so that $A\otimes M$ is an f.p. $A$-module and we conclude using \Cref{fp-stable-etale-tensor}.
\end{proof}

\begin{proposition}\label{descent-finite-free}
For $M$ a module that is an \'etale sheaf, the proposition "$M$ is an finite free" is itself an \'etale sheaf.
\end{proposition}

\begin{proof}
Since $R$ is local, $M$ being finite free is equivalent to $M$ being finitely presented and finitely copresented \cite{TODO-diffgeo}. But both are \'etale sheaves by \Cref{descent-fp-fcop} so we can conclude.
\end{proof}

