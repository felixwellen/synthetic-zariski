% latexmk -pdflatex='xelatex %O %S' -pvc -pdf slides.tex
\documentclass{beamer}

% für xelatex:
\usepackage{xltxtra}
\usepackage{unicode-math}

% literatur
\usepackage[backend=biber,style=alphabetic]{biblatex}

\addbibresource{../util/literatur.bib}

\usepackage{csquotes}
\usepackage{hyperref}
\usepackage{tikz}
\usetikzlibrary{cd,arrows,shapes,calc,through,backgrounds,matrix,trees,decorations.pathmorphing,positioning,automata}
\usepackage{graphicx}
\usepackage{color}

\usepackage{mathpartir}
\newcommand{\yields}{\vdash}
\newcommand{\cbar}{\, | \,}


% für tabellen
\usepackage{booktabs}


% Modalities
\newcommand{\Red}{\Re}
\newcommand{\Cored}{\Im}
\newcommand{\Wat}{\&}
\newcommand{\shape}{\int} %{\ensuremath{\mathord{\raisebox{0.5pt}{\text{\rm\esh}}}}}
\newcommand{\submodality}{\rotatebox[origin=c]{90}{$\subset$}}

\newcommand{\inT}{\colon\hspace{-1.16mm}}
\newcommand{\NN}{\mathbb N}
\newcommand{\ZZ}{\mathbb Z}
\newcommand{\RR}{\mathbb R}
\newcommand{\CC}{\mathbb C}
\newcommand{\QQ}{\mathbb Q}

\newcommand{\PP}{\mathbb P}
\newcommand{\id}{\mathrm{id}}
\newcommand{\cat}[1]{\ensuremath{\text{#1}}}
\newcommand{\topos}[1]{\ensuremath{\mathcal{#1}}}
\newcommand{\functor}[1]{\ensuremath{\operatorname{\mathrm{#1}}}}
\newcommand{\Abb}[5]{\ensuremath{#1:\begin{array}{ccc} #2 & \longrightarrow & #3 \\ #4 & \longmapsto & #5 \end{array}}}
\newcommand{\Spec}{\ensuremath{\mathrm{Spec}}}
\newcommand{\red}[1]{\ensuremath{{#1}_{\mathrm{red}}}}
\newcommand{\Hom}{\ensuremath{\mathrm{Hom}}}
\newcommand{\Cos}{\ensuremath{\mathrm{cos}}}
\newcommand{\Sin}{\ensuremath{\mathrm{sin}}}
\newcommand{\Sh}{\ensuremath{\mathrm{Sh}}}
\newcommand{\Lim}{\ensuremath{\mathrm{lim~}}}
\newcommand{\op}{\ensuremath{\mathrm{op}}}
\newcommand{\bu}{\ensuremath{\mathsmaller{\mathsmaller{\,\bullet\,}}}}
\newcommand{\defequal}{\ensuremath{\colon\!\!\!\equiv}}
\newcommand{\equaldef}{\ensuremath{\equiv\!\!\colon}}
\newcommand{\pointbetween}[2]{($(#1)!0.5!(#2)$)} % use like this:  \node (middle) at \pointbetween{A}{B};
\newcommand{\pbsquare}[3]{
        \node (h-mid) at \pointbetween{#2}{#3} {};
        \node (v-mid) at \pointbetween{#2}{#1} {};
        \node (center) at (h-mid |- v-mid) {\small{(pb)}};
}
\newcommand{\oneImage}{\ensuremath{\text{1-image}}}
\newcommand{\ImCompute}{\ensuremath{\text{$\Im$-compute}}}
\newcommand{\ImInduction}{\ensuremath{\text{$\Im$-induction}}}
\newcommand{\ImRecursion}{\ensuremath{\text{$\Im$-recursion}}}
\newcommand{\ImNaturality}{\ensuremath{\text{$\Im$-naturality}}}
\newcommand{\epi}{\ensuremath{\twoheadrightarrow}}
\newcommand{\mono}{\ensuremath{\hookrightarrow}}
\newcommand{\et}{\ensuremath{-\text{ét}\to}}
\newcommand{\Aut}{\ensuremath{\mathrm{Aut}}}
\newcommand{\BAut}{\ensuremath{\mathrm{BAut}}}
\newcommand{\BGL}{\ensuremath{\mathrm{BGL}}}
\newcommand{\GL}{\ensuremath{\mathrm{GL}}}
\newcommand{\Sch}{\ensuremath{\mathfrak{Sch}}}
\newcommand{\Zar}{\ensuremath{\mathfrak{Zar}}}
\newcommand{\ba}{\ensuremath{\mathbb{A}}}
\newcommand{\DD}{\ensuremath{\mathbb{D}}}
\newcommand{\colim}{\ensuremath{\mathrm{colim}}}
\newcommand{\agda}[1]{\href{#1}{\color{darkgreen} \ensuremath{✓}}}
\newcommand{\ignore}[1]{}
\newcommand{\infclose}{\sim_{\text{\tiny{inf}}}}
% \newcommand{\submodality}{\rotatebox[origin=c]{90}{$\subset$}}


\title[ConCoh]
{Synthetic Algebraic Geometry}
\author[Author, Anders] 
{Ingo Blechschmidt, Felix Cherubini, Thierry Coquand, Matthias Hutzler, Hugo Moeneclaey, Josselin Poiret and David Wärn}
\institute
{Working on various sub-projects}

\begin{document}

\date{}
\begin{frame}
  \titlepage
\end{frame}

\begin{frame}
  \frametitle{Acknowledgements and Subprojects}
  The approach to cohomology we use was proposed by \textbf{Michael Shulman} in 2013 and was later worked out by \textbf{Floris van Doorn}.
  It is known to many people in the field. \\ ~\\
  \pause
  The approach to synthetic algebraic geometry is based on work by  \textbf{Ingo Blechschmidt}, \textbf{Anders Kock} and \textbf{David Jaz Myers}. \\  ~\\
  \pause
  Currently, there are the following projects:
  \begin{table}
    \centering
    \begin{tabular}{lp{7.5cm}}
      Foundations & Felix, Matthias, Thierry \\
      Proper Schemes & David, Felix, Matthias, Thierry \\
      Differential Geometry & David, Felix, Hugo, Matthias \\
      \v{C}ech Cohomology & David, Felix, Ingo \\
      Formalization & Felix, Joisselin, Matthias \\
    \end{tabular}
  \end{table}
  
\end{frame}

\begin{frame}
  \begin{center}
    \begin{tikzpicture} 
      [ 
      node distance=.8cm, 
      circled/.style={draw, ellipse, ultra thick, fill=blue!12},
      edge from parent/.style={very thick,draw=black,-latex},
      plaintext/.style={}
      ]

      \tikzstyle{level 1}=[sibling angle=81,level distance=4cm]

      \node[circled] (hott) {HoTT+Axioms} [counterclockwise from=207]
      child { 
        node[circled] (grp) {$\infty$-Groupoids} 
        edge from parent [-latex] node (grp) 
        {
          \begin{tikzpicture}[scale=0.8, rotate=25]
            \draw[-, red, ultra thick] (0,0) to (1,1);
            \draw[-, red, ultra thick] (1,0) to (0,1);
          \end{tikzpicture}
        } 
      }
      child { node[matrix, circled, inner sep=0pt] (smgrp) 
        {
          \node[circled, minimum height=0.7cm, minimum width=3cm] (mfd) {Schemes};
          \node[below of=mfd, text width=3cm, 
          node distance=0.9cm] 
          {``Cubical Zariski-sheaves''}; \\
        }
      }
      ;
    \end{tikzpicture}
  \end{center}
  ~\\
  ~\\
  {\footnotesize Schemes = quasi-compact, quasic-seperated schemes of finite type}
\end{frame}

\begin{frame}
  \frametitle{Synthetic algebraic geometry}
  There is a local, commutative ring $R$. \\
  \pause
  We define $\mathrm{Spec}(A)$ for a finitely presented $R$-algebra $A$ by
  \[ \mathrm{Spec}(A):\equiv \mathrm{Hom}_{R\mathrm{-algebra}}(A,R)\]
  \pause
  Then the \emph{synthetic quasi-coherence} (SQC) axiom states that for finitely presented $R$-algebras $A$, the evaluation map
  \[ r\mapsto (f\mapsto f(r)):A\to R^{\mathrm{Spec}(A)} \]
  is an equivalence.
\end{frame}

\begin{frame}
  \frametitle{Cohomology of sheaves}
  Let $X$ be a type and $\mathcal F:X\to \mathrm{Ab}$ a dependent abelian group on $X$. \\
  \pause
  The $n$-th cohomology group of $\mathcal F$ is
  \[ H^n(X,\mathcal F):\equiv\left\|\prod_{x:X}B^n\mathcal F_x\right\|_0\equiv \|(x:X)\to B^n\mathcal F_x\|_0 \]
  \pause
  Properties: \\
  \pause
  The $H^n(X,\mathcal F)$ are all abelian groups. \\
  \pause
  Functoriality, covariant in $\mathcal F$, contravariant in $X$. \\
  \pause
  Some long exact sequence for coefficients. \\
  \pause
  We have a Mayer-Vietoris-Lemma and more generally correspondence with \v{C}ech-Cohomology, for nice enough spaces. \\
\end{frame}

\begin{frame}
  \frametitle{Zariski-Choice and Cohomology}  
  Let $X=\mathrm{Spec}(A)$ and $M:X\to R\text{-Mod}$ such that $((x:D(f))\to M_x)=((x:X)\to M)_f$, then
  \[ H^1(X,M)=0 \]
  \pause
  \textbf{Proof:} TODO
\end{frame}

\end{document}
