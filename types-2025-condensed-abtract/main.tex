% latexmk -pdf -pvc main.tex
% latexmk -pdf -pvc -interaction=nonstopmode main.tex
\documentclass{../util/zariski}

\RequirePackage[safe]{tipa}
%
\title{Synthetic Stone Duality 
}

% Authors are joined by \and. 
% Their affiliations are given by \inst, which indexes
% into the list defined using \institute
%
\author{
Felix Cherubini %\inst{1}
% uncomment the following for multiple authors.
\and 
 Thierry Coquand% \inst{2}%
\and 
 Freek Geerligs% \inst{3}%
% \thanks{Speaker.}%
\and
 Hugo Moeneclaey %\inst{4}%
}

% Institutes for affiliations are also joined by \and,
% \institute{
%  University of Gothenburg and Chalmers University of Technology, Gothenburg, Sweden%\\
% }

% \authorrunning{Cherubini, Coquand, Geerligs and Moeneclaey}

\title{Cohomology in Synthetic Stone Duality}

\begin{document}
\maketitle
We use an axiomatization of (separable) Stone spaces within a homotopy type theory \cite{synthetic-stone-duality}, which is related to synthetic algebraic geometry \cite{draft}.
In \cite{synthetic-stone-duality} compact Hausdorff spaces are defined as certain quotients of stone spaces and LLPO and Markov's principle hold.
Using homotopy type theory, it is possible to define cohomology using higher types.

IMPORTANT POITNS:

We follow an idea of Wärn \cite{cech-draft}[Theorem 3.4] to get $H^n(S,A)=0$ for $n>1$, for Stone spaces $S$ and $A : S\to \mathrm{Ab}_\ODisc$.
This is an internal version of \cite{dyckhoff76}.


FROM LAST YEAR:

We conjecture that internal results on light condensed sets \cite{Dagur,Scholze,Condensed} can be shown using univalent
type theory extended by these axioms.
Furthermore, this work can be seen as a variation of the work in \cite{XuE13}. We also expect to build a constructive
sheaf model of these axioms, similar to the constructive model of synthetic algebraic geometry presented in \cite{draft}.

\medskip

We denote the type of countably presented Boolean algebras by $\Boole$.
Given a Boolean algebra $B$, we define $Sp(B)$, the spectrum of $B$ as the set of Boolean morphisms from $B$ to $2$.  
A type of the form $Sp(B)$ for $B:\Boole$ is called Stone.
%
Two motivating examples of elements of $\Boole$ are as follows:
 \begin{itemize}
   \item $C$ is the free Boolean algebra on countably many generators $(p_n)_{n\in\mathbb N}$. 
     The corresponding set $Sp(C)$ is Cantor space $2^\mathbb N$. 
   \item 
     The Boolean algebra $ B_\infty$ is %given by 
     the quotient of $C$ by the relations $p_n\wedge p_m = 0$ for $n\neq m$.  
     A term of $Sp(B_\infty)$ sends $p_n$ to $1$ for at most one $n$. 
     For this reason, $Sp(B_\infty)$ is denoted $\Noo$. 
  \end{itemize} 

\begin{axiom}[Stone duality]
  For any 
  $B:\Boole$, 
  the evaluation map $B \to 2^{Sp(B)}$ is an isomorphism. 
\end{axiom}
It follows from Stone duality that being Stone is a proposition and $Sp$ defines an embedding from $\Boole$ 
to any universe $\mathcal U$. We denote its image $\Stone$. 
Any $X:Stone$ has a topology where basic clopens are given by decidable subsets. 
Using Stone duality we can show that any map from a Stone space to $\mathbb N$ is uniformly continuous. 
Both $\Stone$ and $\Boole$ have a natural structure of a category, and 
Stone duality gives that $Sp$ induces a dual equivalence between them. 

\begin{axiom}[Surjections are Formal Surjections]
  A map $Sp(B')\to Sp(B)$ is surjective iff the corresponding Boolean map $B \to B'$ is injective.
\end{axiom} 
Using this axiom, we can show that if $B$ is nontrivial, $Sp(B)$ is merely inhabited.
%
%
Note that the sum of the maps $\Noo \to \Noo$ sending $n$ to $2n,2n+1$ respectively has no section. 
However, we can use the above axiom to show that it is surjective. 
This implies that $\Noo$ is not projective and that LLPO holds. 
However, we can also show the negation of WLPO from Axiom 1.  

\medskip

Analogously to synthetic algebraic geometry, we need an axiom of local choice. 
\begin{axiom}[Local choice]
  Given $X$ Stone, $E,F$ arbitrary types, a map $X \to F$ and $E\twoheadrightarrow F$ surjective, 
  there is some $Y$ Stone,
    a surjection $Y \twoheadrightarrow X$ and a map $Y\to E$ such that the following diagram commutes:
    \begin{equation*}\begin{tikzcd}
      Y \arrow [d, two heads,dashed] \arrow [r,dashed] & E \arrow[d,""',two heads]\\
      X \arrow[r] & F
    \end{tikzcd}\end{equation*}  
\end{axiom} 

We define a type to be compact Haussdorf if it is the quotient of a Stone type by a closed equivalence relation. 
We denote $\CHaus$ for the the type of compact Hausdorff types. 
A motivating example for compact Haussdorf types is the unit interval, which can be given as a quotient of Cantor space. 
To show that such an interval is isomorphic to the standard Cauchy interval, 
we need LLPO and an axiom of dependent choice. 
\begin{axiom}[Dependent Choice]
  Given a sequence of arbitrary types $(X_n)_{n:\mathbb N}$ and surjections $X_n \twoheadrightarrow X_{n-1}$, 
  all the limit projection maps $X \to X_n$ are surjective. 
\end{axiom}


It is important that these two last axioms are stated for {\em arbitrary} types and not types that are only
homotopy sets. One application should be a proof of $H^n(S,\ints) = 0$ for $n>0$, for $S$ Stone,
that we have checked for $n = 1$ (similar to the proof of $H^1(X,R) = 0$ for $X$ affine in the setting of \cite{draft}).
For this, we use the general definition of cohomology group in Homotopy Type Theory \cite{hott}, which refers to
types that are not necessarily homotopy sets.
We also expect to have for $X:\CHaus$ that $H^n(X,\ints)$ coincides with the singular cohomology of $X$.

 Finally, we checked that all axioms suggested recently by R. Barton and J. Commelin \cite{bc24} follow from our axioms.





\printbibliography

\end{document}
