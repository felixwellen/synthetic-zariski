% latexmk -pdf -pvc main.tex
% latexmk -pdf -pvc -interaction=nonstopmode main.tex
\documentclass{../util/zariski}

\RequirePackage[safe]{tipa}
%
\title{Synthetic Stone Duality 
}

% Authors are joined by \and. 
% Their affiliations are given by \inst, which indexes
% into the list defined using \institute
%
\author{
Felix Cherubini %\inst{1}
% uncomment the following for multiple authors.
\and 
 Thierry Coquand% \inst{2}%
\and 
 Freek Geerligs% \inst{3}%
% \thanks{Speaker.}%
\and
 Hugo Moeneclaey %\inst{4}%
}

% Institutes for affiliations are also joined by \and,
% \institute{
%  University of Gothenburg and Chalmers University of Technology, Gothenburg, Sweden%\\
% }

% \authorrunning{Cherubini, Coquand, Geerligs and Moeneclaey}

\title{Cohomology in Synthetic Stone Duality}

\begin{document}
\maketitle
Peter Scholze and Dustin Clausen \cite{Scholze} introduced light condensed sets which can be used as an alternative to topological spaces. Light condensed sets are defined as sheaves on profinite sets.
Inspired by the interpretation of homotopy type theory into the higher topos on this site, \cite{synthetic-stone-duality} introduced an extension of homotopy type theory by axioms which is called synthetic stone duality. In this theory some results where shown about the first cohomology group with integer coefficients. We extend these results to higher cohomology groups with non-constant countably presented abelian groups as coefficients.

In topology, a Stone space is a totally disconnected Hausdorff space and Stone duality is a one to one correspondence between Stone spaces and boolean algebras. In \textbf{synthetic stone duality} Stone duality for countably presented boolean algebras is one of the four axioms of the theory. Compact Hausdorff spaces are defined 

\textbf{Barton-Commelin}

\textbf{Vanishing of higher cohomology of stone spaces}

First we prove that $H^1(S,A) = 0$ with $S$ Stone and  $A : S\to \mathrm{Ab}_\ODisc$. To do that we assume $\alpha:\prod_{x:S}K(A_x,1)$, then we proceed in 4 setps steps:
\begin{enumerate}[(1)]
\item By local choice we get a surjection $p:T\twoheadrightarrow S$ with $T$ Stone which trivialise $\alpha$.
\item We get an approximation of $p$ as a sequential limit of surjective maps $p_k:T_k\to S_k$ between finite sets.
\item We show that the trivialisation of the induced $\alpha_k:S_k\to BA_k$ over $T_k$ induced by $p_k$ gives a trivialisation of $\alpha_k$, essentially using that $p_k$ merely has a section.
\item We conclude through a dependent version of Scott Continuity that $\alpha$ is trivial over $\mathrm{lim}_kS_k =S$.
\end{enumerate}

We follow an idea of Wärn \cite{cech-draft}[Theorem 3.4] to go from the $H^1(S,A)=0$ to $H^n(S,A)=0$ for $n>1$. The proof is subtle but one key idea is to generalize $H^n(S,A) = 0$ for all $S,A$ to: 
\begin{itemize}
\item $K(\prod_{x:S}A_x,n) \to \prod_{x:S}K(A_x,n)$ is an equivalence.
\item $K(\prod_{x:S}A_x,n+1) \to \prod_{x:S}K(A_x,n+1)$ is an embedding
\end{itemize}
Assume the hypothesis for $n-1\geq 1$, let's prove it for $n$.
\begin{enumerate}
\item The embedding follows immediately from induction hypothesis on the equivalence.
\item To prove the equivalence hypothesis we just need surjectivity, so it is enough to prove $\prod_{x:S}K(A_x,n)$ is connected, i.e. $H^n(S,A)=0$. We assume $\alpha:\prod_{x:S}K(A_x,n)$, by local choice we get a trivialisation $p:T\twoheadrightarrow S$ of $\alpha$ with $T$ Stone. Then writing $T_x$ the fiber of $p$ over $x$, we consider the exact sequence:
\[0\to A_x\to A_x^{T_x}\to L_x\to 0\]
giving a sequence:
\[H^{n-1}(S,L)\to H^n(S,A)\to H^n(S,\lambda x. A_x^{T_x})\]
exact at $H^n(S,A)$. By induction hypothesis $H^{n-1}(S,L) = 0$ so we have an injection:
\[H^n(S,A)\to H^n(S,\lambda x. A_x^{T_x})\]
By induction hypothesis again the map: 
\[\prod_{x:S}K(A_x^{T_x},n)\to \prod_{x:S}K(A_x,n)^{T_x}\]
is an embedding so the map:
\[H^n(S,\lambda x. A_x^{T_x}) \to H^n(\Sigma_{x:S}T_x,A_x)\]
is an injection and we get an injection:
\[H^n(S,A)\to H^n(\Sigma_{x:S}T_x,A_x)\] 
But $p$ trivialise $\alpha$ so it vanish in $H^n(\Sigma_{x:S}T_x,A_x)$, therefore $\alpha=0$.
\end{enumerate}

\textbf{Cech-cohomology}



\printbibliography

\end{document}
