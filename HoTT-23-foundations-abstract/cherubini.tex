% latexmk -pdfxe -pvc cherubini.tex
% This is a template for submissions to HoTT 2023.
% The file was created on the basis of easychair.tex,v 3.5 2017/03/15 
%
% Before submitting, rename the resulting pdf file to "yourname.pdf" 
%
\documentclass[letterpaper]{../util/easychair}
\usepackage{doc}
%
\newcommand{\easychair}{\textsf{easychair}}
\newcommand{\miktex}{MiK{\TeX}}
\newcommand{\texniccenter}{{\TeX}nicCenter}
\newcommand{\makefile}{\texttt{Makefile}}
\newcommand{\latexeditor}{LEd}

% some stuff from ../util/zarisky.cls:

\RequirePackage{amsmath,amssymb,mathtools}
% Referenzen
\RequirePackage[backend=biber,style=alphabetic, backref, backrefstyle=none]{biblatex}
\addbibresource{../util/literature.bib}


% content of ../util/zarisky.sty:
% names for types
\newcommand{\R}{\mathbb{R}}
\newcommand{\Z}{\mathbb{Z}}
\newcommand{\N}{\mathbb{N}}
\newcommand{\Bool}{\mathrm{Bool}}
\DeclareMathOperator{\Fin}{Fin}
\newcommand{\unit}{\mathbf{1}}
\newcommand{\two}{\mathbf{2}}
\newcommand{\isContr}{\mathrm{isContr}}
\newcommand{\isProp}{\mathrm{isProp}}
\newcommand{\isSet}{\mathrm{isSet}}
\newcommand{\isEquiv}{\mathrm{isEquiv}}
\newcommand{\qinv}{\mathrm{qinv}}
\newcommand{\mU}{\mathcal{U}}
\newcommand{\Eq}[1]{\mathrm{Eq}_{#1}}
\newcommand{\isNType}[1]{\mathrm{is}\mbox{-}{#1}\mbox{-}\mathrm{type}}
\newcommand{\nType}[1]{#1\mbox{-}\mathrm{Type}}
\newcommand{\Type}{\mathrm{Type}}
\newcommand{\Prop}{\mathrm{Prop}}
\newcommand{\Open}{\mathrm{Open}}
\newcommand{\propTrunc}[1]{\lVert #1 \rVert}

% names for terms
\newcommand{\id}{\mathrm{id}}
\newcommand{\refl}{\mathrm{refl}}
\newcommand{\pair}{\mathrm{pair}}
\newcommand{\FunExt}{\mathrm{FunExt}}
\newcommand{\transp}{\mathrm{tr}}
\newcommand{\transpconst}{\mathrm{tconst}}
\newcommand{\ua}{\mathrm{ua}}
\newcommand{\fib}{\mathrm{fib}}

% category theory
\newcommand{\Hom}{\mathrm{Hom}}
\newcommand{\Sh}{\mathrm{Sh}}
\newcommand{\yo}{\mathrm{y}}

% algebra
\newcommand{\inv}{\mathrm{inv}}
\newcommand{\divides}{\mathbin{|}}
\DeclareMathOperator{\AbGroup}{Ab}
\DeclareMathOperator{\im}{im}
\DeclareMathOperator{\coker}{coker}
\newcommand{\Tors}[1]{#1\text{-}\mathrm{Tors}}
\newcommand{\Mod}[1]{#1\text{-}\mathrm{Mod}}
\newcommand{\Vect}[2]{#1\text{-}\mathrm{Vect}_{#2}}
\newcommand{\fpMod}[1]{#1\text{-}\mathrm{Mod}_{\text{fp}}}
\newcommand{\Alg}[1]{#1\text{-}\mathrm{Alg}}

% algebraic geometry
\DeclareMathOperator{\Spec}{Spec}
\DeclareMathOperator{\Sch}{\mathrm{Sch}_{qc}}
\newcommand{\A}{\mathbb{A}}
\newcommand{\D}{\mathbb{D}}
\newcommand{\bP}{\mathbb{P}}


% misc
\newcommand{\notion}[1]{\emph{#1}\index{#1}}
\providecommand*\colonequiv{\vcentcolon\mspace{-1.2mu}\equiv}
\newcommand{\ignore}[1]{}
\newcommand{\rednote}[1]{{\color{red}(#1)}}


%
\title{A Foundation for Synthetic Algebraic Geometry
}

% Authors are joined by \and. 
% Their affiliations are given by \inst, which indexes
% into the list defined using \institute
%
\author{
Felix Cherubini \inst{1}
% uncomment the following for multiple authors.
\and 
 Thierry Coquand \inst{1}
\and 
 Matthias Hutzler \inst{1}
 \thanks{Speaker.}
}

% Institutes for affiliations are also joined by \and,
\institute{
  University of Gothenburg\\
  \email{felix.cherubini@posteo.de}
% uncomment the following for multiple authors.
\and
  University of Gothenburg\\
  \email{thierry.coquand@cse.gu.se}
\and
  University of Gothenburg\\
  \email{matthias-hutzler@posteo.net}
}

%  \authorrunning{} has to be set for the shorter version of the authors' names;
% otherwise a warning will be rendered in the running heads. When processed by
% EasyChair, this command is mandatory: a document without \authorrunning
% will be rejected by EasyChair
\authorrunning{Cherubini, Coquand and Hutzler}

% \titlerunning{} has to be set to either the main title or its shorter
% version for the running heads. When processed by
% EasyChair, this command is mandatory: a document without \titlerunning
% will be rejected by EasyChair
\titlerunning{Synthetic Algebraic Geometry}

\begin{document}
\maketitle

Algebraic geometry is the study of solutions of non-linear equations using methods from geometry.
Most prominently, algebraic geometry was essential in the proof of Fermat's last theorem by Wiles and Taylor.
The central geometric objects in algebraic geometry are called \emph{schemes}.
Their basic building blocks are called \emph{affine schemes},
where, informally, an affine scheme corresponds to a solution sets of polynomial equations.
While this correspondence is clearly visible in the functorial approach to algebraic geometry and our synthetic approach,
it is somewhat obfuscated in the most commonly used, topological appraoch.

In recent years,
formalization of the intricate notion of affine schemes
received some attention as a benchmark problem
-- this is, however, \emph{not} a problem addressed by this work.
Instead, we use a synthetic approach to algebraic geometry,
very much alike to that of synthetic differential geometry.
This means, while a scheme in classical algebraic geometry is a complicated compound datum,
we work in a setting where schemes are types,
with an additional property that can be defined within our synthetic theory.

In our work in progress \cite{draft},
following ideas of Ingo Blechschmidt and Anders Kock  (\cite{ingo-thesis}, \cite{kock-sdg}[I.12]),
we use a base ring $R$, which is local and satisfies an axiom reminiscent of the Kock-Lawvere axiom.
A more general axiom, is called \emph{synthetic quasi coherence (SQC)} by Blechschmidt and
a version quatifying over external algbras is called the \emph{comprehensive axiom}\footnote{
  In \cite{kock-sdg}[I.12], Kock's ``axiom $2_k$'' could equivalently be Theorem 12.2,
  which is exactly our synthetic quasi coherence axiom, except that it only quantifies over external algebras.
}
by Kock.
The exact concise form of the SQC axiom we use, was noted by David Jaz Myers (\cite{myers-talk1, myers-talk2}).

Before we state the SQC axiom, let us take a step back and look at the basic objects of study in algebraic geometry,
solutions of polynomial equations.
Given a system of polynomial equations
\begin{align*}
  p_1(X_1, \dots, X_n) = 0, \dots, p_m(X_1, \dots, X_n) = 0
\end{align*}
the solution set
$\{ x : R^n \mid \forall i.\; p_i(x_1, \dots, x_n) = 0 \}$
is in canonical bijection to the set of $R$-algebra homomorphisms
\[ \Hom_R(R[X_1, \dots, X_n]/(p_1, \dots, p_m), R) \]
by identifying a solution $(x_1,\dots,x_n)$ with the homomorphism that maps each $X_i$ to $x_i$.
Conversely, for any $R$-algebra $A$, which is merely of the form $R[X_1, \dots, X_n]/(p_1, \dots, p_m)$,
we define the \emph{spectrum} of $A$ to be
\[
  \Spec A \colonequiv \Hom_R(A, R)
  \rlap{.}
\]
In contrast to classical, non-synthetic algebraic geometry,
where this set needs to be equipped with additional structure,
we postulate axioms that will ensure that $\Spec A$ has the expected geometric properties.
Namely, SQC is the statement that, for all finitely presented $R$-algebras $A$, the canonical map
  \begin{align*}
    A&\xrightarrow{\sim} (\Spec A\to R) \\
    a&\mapsto (\varphi\mapsto \varphi(a))
  \end{align*}
is an equivalence.
A prime example of a spectrum is $\A^1\colonequiv \Spec R[X]$,
which turns out to be the underlying set of $R$.
With the SQC axiom,
\emph{any} function $f:\A^1\to \A^1$ is given as a polynomial with coefficients in $R$.
In fact, all functions between affine schemes are given by polynomials.
Furthermore, for any affine scheme $\Spec A$,
the axiom ensures that
the algebra $A$ can be reconstructed as the algebra of functions $\Spec A \to R$,
therefore establishing a duality between affine schemes and algebras.

The Kock-Lawvere axiom used in synthetic differential geometry,
might be stated as the SQC axiom restricted to (external) \emph{Weil-algebras},
whose spectra correspond to pointed infinitesimal spaces.
These spaces can be used in both, synthetic differential and algebraic geometry,
in very much the same way.

An important result
is the construction of cohomology groups.
This is where the \emph{homotopy} type theory really comes to bear --
instead of the hopeless adaption of classical, non-constructive definitions of cohomology,
we make use of higher types,
for example the $k$-th Eilenberg-MacLane space $K(R,k)$ of the group $(R,+)$.
As an analogue of classical cohomology with values in the structure sheaf,
we then define cohomology with coefficients in the base ring as:
\[
  H^k(X,R):\equiv \|X\to K(R,k)\|_0
  \rlap{.}
\]
This definition is very convenient for proving abstract properties of cohomology.
For concrete calculations we make use of another axiom,
which we call \emph{Zariski-local choice}.
While this axiom was conceived of for exactly these kind of calculations,
it turned out to settle numerous questions with no apparent connection to cohomology.
One example is the equivalence of two notions of \emph{open subspace}.
A pointwise definition of openness was suggested to us by Ingo Blechschmidt and
is very convenient to work with.
However, classically, basic open subsets of an affine scheme are given
by functions on the scheme and the corresponding open is morally the collection of points where the function does not vanish.
With Zariski-local choice, we were able to show that these notions of openness agree in our setup.

Apart from SQC, locality of the base ring $R$ and Zariski-local choice,
we only use homotopy type theory, including univalent universes, truncations and some very basic higher inductive types.
Roughly, Zariski-local choice states, that any surjection into an affine scheme merely has sections on a \emph{Zariski}-cover.
The latter, internal, notion of cover corresponds quite directly to the covers in the site of the \emph{Zariski topos},
which we use to construct a model of homotopy type theory with our axioms.

The scope of our theory so far, includes quasi-compact, quasi-separated schemes of finite type over an arbitrary ring.
These are all finiteness assumptions, that were chosen for convenience and include examples like closed subspaces of projective space,
which we want to study in future work, as example applications.
So far, we know that basic internal constructions, like affine schemes, correspond to the correct classical external constructions.
This can be expanded using our model, which is of course also important to ensure the consistency of our setup.

Since, there are many aspects of our work that can be selected as a focus for a talk,
we plan to present our work at HoTT 23 in a way, that will have novel content for listeners from other conferences.

\printbibliography

\end{document}
