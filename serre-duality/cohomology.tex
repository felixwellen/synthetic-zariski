
We use the synthetic algebro-geometric notions of \cite{draft}
and recollect some results of \cite{cech-draft}.
This entails that we use a pullback $f_\ast$ and push-forward $f^\ast$ which arguably deviates somewhat from the usual notion and is more of a pointwise nature.

\begin{definition}
  \begin{enumerate}[(a)]
  \item Let $A,B:X\to \Mod{R}$.
    For subsets $U\subseteq X$, we write $A(U)\colonequiv (x:U)\to A_x$.
    Similarly, if for each $x:X$ we have a morphisms $f_x:A_x\to B_x$,
    we write $f:A\to B$ and $m_U\colonequiv (s:A(U))\mapsto ((x:U)\mapsto f_x(s(x)))$.
  \item Let $A,B,C:X\to \Mod{R}$.
    If for all $x:X$ we have a sequence
    \begin{center}
      \begin{tikzcd}
        A_x\ar[r,"m_x"] & B_x\ar[r,"e_x"] & C_x
      \end{tikzcd}
    \end{center}
    we write $A\to B\to C$ and call this datum a sequence.
    The sequence is called \notion{(pointwise) exact}
    if for all $x:X$ the sequence above is exact.
    The sequence is called \notion{locally exact},
    if $e$ is surjective and there is an open affine cover such that $m_U$ is injective.
  \end{enumerate}
\end{definition}

Note that pointwise exactness is a strictly stronger requirement than local exactness.

\begin{theorem}
  \begin{enumerate}[(a)]
  \item Let $A,B,C:X\to \AbGroup$ and 
    \[
    0\to A\to B\to C\to 0
    \]
    be a (pointwise) exact sequence, then there is a long exact sequence of cohomology groups:
    \begin{center}
      \begin{tikzcd}
        & \dots\ar[r]& H^{n-1}(X,C)\ar[dll]\\
        H^n(X,A)\ar[r] & H^n(X,B)\ar[r] & H^n(X,C)\ar[lld] \\
        H^{n+1}(X,A)\ar[r] & H^{n+1}(X,B)\ar[r] & \dots        
      \end{tikzcd}
    \end{center}
  \item Let $A,B,C:X\to \Mod{R}_{\mathrm{wqc}}$ and
    \[
    0\to A\to B\to C\to 0
    \]
    be a locally exact sequence,
    then there is an induced long exact sequence like above.
  \end{enumerate}
\end{theorem}

\begin{theorem}
  Let $X$ be a separated scheme with open affine cover $\{U\}$ and $M:X\to \Mod{R}_{\mathrm{wqc}}$ a bundle of weakly quasi-coherent $R$-modules.
  The natural isomorphism $H^0(X,M) \cong \check{H}^0(\{U\},M)$ extends to an isomorphism of $\partial$-functors,
  i.e.\ it extends to a natural isomorphism
  \[ H^k(X,M) \cong \check{H}^k(\{U\},M)\text{, for all $k\geq 0$,}
  \]
  compatible with long exact cohomology sequences.
\end{theorem}

There is also the helpful lemma:


\begin{lemma}
  \label{cohomologically-trivial-fibers}
  Let $f:Y\to X$ and $A:Y\to\AbGroup$ be such that $H^l(\fib_f(x),\pi_1^\ast A)=0$ for all $0<l\leq n$,
  then
  \[
  H^n(Y,A)=H^n(X,f_\ast A)
  \rlap{.}
  \]
  In particular, if $i:C\to \bP^n$ is closed and $M:\bP^n\to\Mod{R}_{wqc}$, then $H^n(C,i_\ast M)=H^n(\bP^n,M)$.
\end{lemma}

With these results, the proof of the following carries over\footnote{A full proof in the synthetic setting is given in \cite{cech-draft}.}
from \cite[Chapter III]{Hartshorne}:

\begin{theorem}
  \begin{enumerate}[(i)]
  \item For all $n:\N$, $d:\Z$, there are isomorphisms $R[X_0,\dots,X_n]_d\to H^0(\bP^n,\mathcal O(d))$ of $R$-modules, inducing an isomorphism $R[X_0,\dots,X_n]\to \bigoplus_{d:\Z} H^0(\bP^n,\mathcal O(d))$ of graded $R[X_0,\dots,X_n]$-modules.
  \item $H^n(\bP^n,\mathcal O(-n-1))=R$ is free of rank 1 and $H^n(\bP^n,\mathcal O(d))=0$ for $d>-n-1$.
  \item The canonical map given by tensoring
    \[
      H^0(\bP^n,\mathcal O(d)) \times H^n(\bP^n,\mathcal O(-d-n-1))\to R
    \]
    is a perfect pairing of finite free $R$-modules for all $d:\Z$.
  \item $H^i(\bP^n,\mathcal O(d))=0$ for $i\in\{1,\dots,n-1\}$ and all $d:\Z$.
  \end{enumerate}
\end{theorem}


Following \cite[Chapter 19]{vakil}.
