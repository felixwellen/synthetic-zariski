
We use notation from \cite{draft} and \cite{sag-projective}.

\begin{definition}
  \label{Pn-definitions}
  $\bP^n$ may be defined as one of the following equivalent types:
  \begin{enumerate}[(i)]
  \item The type of lines\footnote{Submodules $M\subseteq R^{n+1}$ with $\| M=R^1 \|$} through the origin in $R^{n+1}$.
  \item The set-quotient of $R^{n+1}$ by the relation $x\sim y$ iff $x=\lambda y$ for some $\lambda :R^\times$.
  \item\label{Pn-as-homotopy-quotient} The homotopy quotient:
    \[
      \sum_{l : K(R^\times,1)} l^{n+1}\setminus\{0\}
    \]
  \end{enumerate}
\end{definition}

\begin{lemma}
  \label{D-for-Pn}
  Let $U\subseteq \bP^n$.
  The following are equivalent:
  \begin{enumerate}[(i)]
  \item $U$ is open.
  \item $U=D(P_0,\dots,P_l)$ for $P_i:R[X_0,\dots,X_n]_d$.
  \item There are $N:\N$ and $s_0,\dots,s_l:\prod \mathcal O(1)^{\otimes N}$ such that\index{$D(s_0,\dots,s_l)$}
    \[
      U=D(s_0,\dots,s_l)\colonequiv\sum_{x:X}s_0(x)\neq 0\vee \dots \vee s_l(x)\neq 0
      \rlap{.}
    \]
  \end{enumerate}
\end{lemma}

The equivalence between the last two statements follows from \cite{cech-draft}[Theorem 8.3].
The following is a direct consequence of \Cref{Pn-definitions} \ref{Pn-as-homotopy-quotient}:

\begin{remark}
  The type of maps $X\to \bP^n$ is equivalent to the type
  \[
  (\mathcal L:X\to K(R^\times,1))\times (s_0,\dots,s_n:(x:X)\to \mathcal L_x) \to (x:X) \to \exists_{i} s_i(x)\neq 0
  \]
  i.e.\ the type of line bundles on $X$ that come with $n+1$ sections that do not simultaneously vanish.
\end{remark}

\begin{definition}
  A line bundle $\mathcal L:X\to K(R^\times,1)$ is called \notion{very ample}
  if one of the following equivalent statement holds:
  \begin{enumerate}[(i)]
  \item There is a closed embedding $i:X\to \bP^n$ and $\mathcal L=i_\ast \mathcal O(1)$.
  \item There are non-simultaneously vanishing sections $s_0,\dots,s_n$ of $\mathcal L$
    such that $[s_0,\dots,s_n]:X\to \bP^n$ is a closed embedding.
  \end{enumerate}
\end{definition}

\begin{proof}
  All we have to do is to show that if (ii) holds, for $i\colonequiv [s_0,\dots,s_n]$,
  there is an equality $\mathcal L=i_\ast \mathcal O(1)$.
  Let $x:X$ and note first that $\mathcal O(-1)_x$ is generated by $(s_0,\dots,s_n)^T$.
  Let $\varphi:R^1\to \mathcal L_x$ be an isomorphism of $R$-modules.
  Then there are $\lambda_0,\dots,\lambda_n$ such that $\sum \lambda_i s_i(x) =\varphi(1)$,
  which lets us define:
  \begin{center}
    \begin{tikzcd}
      \mathcal L_x \ar[r] & \mathcal O(1)=\Hom_R(\mathcal O(-1),R) \\
      \lambda \cdot \varphi(1)\ar[mapsto,r] & \sum \lambda_i \varphi^{-1}(s_i(x))
    \end{tikzcd}
  \end{center}
  which is an isomorphism independent of the choice of $\varphi$.
\end{proof}

This is a variant of \cite{vakil}[Theorem 17.6.2] (which might hold):

\begin{definition}
  Let $X$ be a proper scheme.
  A line bundle $\mathcal L : X\to K(R^\times,1)$ is \notion{ample}
  if one of the following equivalent statements holds:
  \begin{enumerate}[(i)]
  \item There is $N>0$ such that $\mathcal L^{\otimes N}$ is very ample.
  \item Any open subset $U\subseteq X$ is of the form $U=D(s_0,\dots,s_n)$,
    for some $s_i:\mathcal L^{\otimes l_i}(X)$ with $l_i>0$.
  \item Any open subset $U\subseteq X$ is of the form $U=D(s_0,\dots,s_n)$,
    for some $N>0$ and $s_i:\mathcal L^{\otimes N}(X)$.
  \item For any finite type, weakly quasi-coherent module bundle $\mathcal F:X\to \Mod{R}_{\mathrm{wqc,fg}}$,
    there is $N>0$, $j:\N$ and a (pointwise) surjection
    $\mathcal O^{\oplus j}\to \mathcal F\otimes \mathcal L^{\otimes N}$.
  \end{enumerate}
\end{definition}
