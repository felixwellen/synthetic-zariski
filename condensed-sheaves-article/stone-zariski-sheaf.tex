We write $\Boole$ for the type of c.p. algebras, and $\Boole_{\mathbb{B}}$ the type of c.p. boolean $\mathbb{B}$-algebras. We work in the presheaf topos, so by \cref{TODO} we can work in HoTT where we assume a boolean algebra $\mathbb{B}$ such that:

\begin{enumerate}[(i)]
\item For all $C:\Boole_\mathbb{B}$, the map:
\[C \to \mathbb{B}^{\Spec_\mathbb{B}(C)}\]
is an equivalence.
\item For all $C:\Boole_\mathbb{B}$, the type $\Spec_\mathbb{B}(C)$ has choice.
\item Dependent choice.
\end{enumerate}

\begin{definition}\label{zariski-characterisation}
A type $X$ is a Zariski sheaf if and only if:
\begin{itemize}
\item If $0=_\mathbb{B}1$ then $X$ is contractible.
\item For all $b:\mathbb{B}$ we have that the map:
\[X\to X^{b=0}\times X^{b=1}\]
is an equivalence.
\end{itemize}
\end{definition}

We can check there is a corresponding lex modality called Zariski sheafification. We denote this modality by $L_{Zar}$, and we denote the corresponding translation by $[\_]_{Zar}$.

\begin{remark}
We have that $X$ is a Zariski sheaf if and only if for all $b_1,\cdots,b_n:\mathbb{B}$ such that $(b_1,\cdots,b_n)=1$ we have that the map:
\[X\to X^{b_1=1\lor\cdots\lor b_n=1}\]
is an equivalence, which proves that being a Zariski sheaf is a lex modality. This makes the name Zariski sheaf reasonable.
\end{remark}

%\begin{definition}
%A type $X$ is called a Zariski sheaf if and only for all  $f_1,\cdots,f_n:\mathbb{B}$ such that:
%\[(f_1,\cdots,f_n) = 1\]
%we have that $X$ is $f_1 \inv \lor \cdots \lor f_n \inv$-local.
%\end{definition}

\begin{lemma}\label{zariski-subcanonical}
The type $\mathbb{B}$ is a Zariski sheaf. Any $C:\Boole_\mathbb{B}$ is a Zariski sheaf.
\end{lemma}

\begin{proof}
It is clear that $0=_\mathbb{B}1$ implies $\mathbb{B}$ contractible.

Then we need to check that for all $b:\mathbb{B}$ we have that the map:
\[\mathbb{B} \to \mathbb{B}^{b=0}\times\mathbb{B}^{b=1}\]
is an equivalence, but by duality it is equivalent to the map:
\[\mathbb{B} \to \mathbb{B}_{1-b}\times\mathbb{B}_b\]
being an equivalence, which is clear.

We extend to $\Boole_{\mathbb{B}}$ by duality.
\end{proof}

\begin{lemma}\label{bot-zariski}
We have that $L_{Zar}(\bot)$ is $0=_\mathbb{B}1$.
\end{lemma}

\begin{proof}
By \cref{zariski-subcanonical} we know that $0=_\mathbb{B}1$ is a Zariski sheaf. It is clear that the fibers of $\bot\to 0=_\mathbb{B}1$ are Zariski contractible as assuming $0=_\mathbb{B}1$ any Zariski sheaf is contractible.
\end{proof}

\begin{lemma}\label{bool-zariski}
We have that $L_{Zar}({2})$ is $\mathbb{B}$.
\end{lemma}

\begin{proof}
By \cref{zariski-subcanonical} we know that $\mathbb{B}$ is a Zariski sheaf. The fibers of the map ${2}\to\mathbb{B}$ are Zariski contractible as they are of the form $b=0+b=1$ for some $b:\mathbb{B}$.
\end{proof}

\begin{lemma}\label{truncation-zariski}
Let $X$ be a Zariski sheaf, then $\propTrunc{X}$ is a Zariski sheaf.
\end{lemma}

\begin{proof}
If $0=_\mathbb{B}1$ then $X$ is contractible so $\propTrunc{X}$ is contractible as well. For all $b:\mathbb{B}$ we have an equivalence:
\[X\to X^{b=0}\times X^{b=1}\]
so that:
\[\propTrunc{X}\to \propTrunc{X^{b=0}\times X^{b=1}}\]
is an equivalence, but since $b=0$ and $b=1$ have choice this means that:
\[\propTrunc{X}\to \propTrunc{X}^{b=0}\times \propTrunc{X}^{b=1}\]
is an equivalence.
\end{proof}

\begin{lemma}\label{zariski-cp-iff-cp}
Assume $C$ be a boolean algebra that is a Zariski sheaf, if we have the interpretation of $C:\Boole$ in the Zariski topos then $C:\Boole_\mathbb{B}$.
\end{lemma}

\begin{proof}
By \cref{bool-zariski} we have that the interpretation of ${2}=\mathbb{B}$ holds in the Zariski topos, so we can assume that we have that the interpretation of $C:\Boole_\mathbb{B}$ holds in the Zariski topos. By \cref{zariski-subcanonical} and \cref{truncation-zariski} we see this means that we merely have a map:
\[P:L_{Zar}(\N)\to \mathbb{B}[\N]\]
with corresponding map:
\[\bar{P}:\N\to \mathbb{B}[\N]\]
such that $C =  \mathbb{B}[\N]/P$. Let us show we have an equivalence of $\mathbb{B}$-algebras between $ \mathbb{B}[\N]/P$ and $\mathbb{B}[\N]/\bar{P}$. The map $\mathbb{B}[\N]/\bar{P}\to \mathbb{B}[\N]/P$ is easy to define, to go the other way we need to prove that for all $y:L_{Zar}(\N)$ we have that $P(y) =_{\mathbb{B}[\N]/\bar{P}} 0$, but since $\mathbb{B}[\N]/\bar{P}:\Boole_\mathbb{B}$ is a sheaf, we can use the dependent elimination on $L_{Zar}$.
\end{proof}

\begin{theorem}\label{zariski-sheaf-axioms}
We have that the Zariski interpretation of the following hold:
\begin{enumerate}[(i)]
\item For all $C:\Boole$, the map $C\to {2}^{\Spec(C)}$ is an equivalence.
\item For all $C:\Boole$, the type $\Spec(C)$ has choice.
\item Dependent choice.
\end{enumerate}
\end{theorem}

\begin{proof}
By \cref{bool-zariski} we just need to prove the first two results replacing ${2}$ by $\mathbb{B}$. Using \cref{zariski-cp-iff-cp} the first point is clear, using \cref{truncation-zariski} we have the second one.

By \cref{truncation-zariski} being surjective is not changed by the interpretation, and a limit of sheaf is a sheaf, so that dependent choice is not changed by the interpretation.
\end{proof}


