\documentclass[proof]{beamer}
\usetheme{default}

\usecolortheme{rose}
\usepackage[english]{babel}
\usepackage[utf8]{inputenc}
%\usepackage[latin1]{inputenc}
 \usepackage{amssymb}
 \usepackage{latexsym}
 \usepackage{amsmath}
 \usepackage{amssymb}
 \usepackage{tikz}
 \usepackage{tikz-cd}
\usepackage{array}
\usepackage{rotating}
\usepackage{forest}
\usepackage{color}
\usepackage[all,cmtip]{xy}

\usepackage{pgfplots}
\usepackage{array}
    \newcolumntype{C}{>{\centering\arraybackslash}c}



\definecolor{qbblue}{RGB}{43,126,128}
%\definecolor{qbblue}{RGB}{43,126,204}
\definecolor{qborange}{RGB}{242,151,36}
\definecolor{qblight}{RGB}{210,210,210}
\definecolor{qbdark}{RGB}{180,180,180}
\definecolor{qbred}{RGB}{184,13,72}

\setbeamercolor{title}{fg=black}
\setbeamercolor{section in toc}{fg=black}
\setbeamercolor{block title}{bg=qblight,fg=black}
\setbeamercolor{frametitle}{bg=qblight,fg=black}
\setbeamercolor{section in head/foot}{bg=black}
\setbeamertemplate{itemize item}{\color{qbdark}$\blacktriangleright$}
\setbeamertemplate{itemize subitem}{\color{qbdark}$\blacktriangleright$}
%\setbeamercolor*{palette tertiary}{bg=black}
\setbeamercolor{local structure}{fg=black}

\usepackage[style=verbose,backend=biber]{biblatex}
\addbibresource{biblio.bib}

\renewcommand{\_}{\rule{.6em}{.5pt}\hspace{0.023cm}}

\newcommand{\bloc}[2]{\begin{block}{#1}\setlength\abovedisplayskip{0pt}#2\end{block}}

\setbeamertemplate{navigation symbols}{} 
\addtobeamertemplate{footline}{\hfill{\tiny \insertframenumber}\hspace{2em} \vspace{1em}}

\newcommand{\red}[1]{\textcolor{qbred}{#1}}
\newcommand{\blue}[1]{\textcolor{qbblue}{#1}}
\newcommand{\orange}[1]{\textcolor{qborange}{#1}}


\newcommand{\propTrunc}[1]{\lVert #1 \rVert}

\newcommand{\CHaus}{\mathsf{CHaus}}
\newcommand{\Stone}{\mathsf{Stone}}
\newcommand{\ODisc}{\mathsf{ODisc}}
\newcommand{\Type}{\mathsf{Type}}
\newcommand{\Ab}{\mathsf{Ab}}
\newcommand{\Open}{\mathsf{Open}}
\newcommand{\Closed}{\mathsf{Closed}}

\renewcommand{\lim}{\mathsf{lim}}
\newcommand{\colim}{\mathsf{colim}}

\newcommand{\Z}{\mathbb{Z}}
\newcommand{\N}{\mathbb{N}}
\newcommand{\I}{\mathbb{I}}
\newcommand{\R}{\mathbb{R}}

\def\mhyphen{{\hbox{-}}}

%\newcommand{\trunc}[1]{|| #1 ||}

\AtBeginSection[]
{
 \begin{frame}<beamer>
 \frametitle{Outline}
 \tableofcontents[currentsection]
 \end{frame}
}

\begin{document}


%\abovedisplayskip=0.0cm
%\abovedisplayshortskip=-0.3cm
%\belowdisplayskip=0.6cm


\title{\red{Cohomology in Synthetic Stone Duality}}
\author{Hugo Moeneclaey\\
\vspace{0.32cm}
j.w.w. Felix Cherubini, Thierry Coquand and Freek Geerligs}
\date{\blue{TYPES 2025}\\
Glasgow}



\frame{\titlepage}



\frame{\frametitle{Overview}

We work in Synthetic Stone Duality (SSD). 

SSD = HoTT + 4 axioms.

\bloc{Cohomology in HoTT}{
Given $n:\N,\ X:\Type,\ A:X\to \Ab$, we define a group $H^n(X,A)$.
}

$H^n(X,A)$ is the $n$-th cohomology group of $X$ with coefficient $A$.

\bloc{Our previous work [CCGM24]}{
\begin{itemize}
\item Showed SSD is suitable for synthetic topological study of Stone and compact Hausdorff spaces.
\item Proved $H^1(X,\Z)$ is well-behaved for $X:\CHaus$ .
\end{itemize}
}

}



\frame{\frametitle{Overview}

\bloc{Today}{
$H^n(X,A)$ is well-behaved for $n:\N,\ X:\CHaus,\ A:X\to \Ab_{cp}$.
}

\bloc{Plan}{ 
\begin{enumerate}
\item Introduce SSD, Stone spaces and compact Hausdorff spaces.
\item Introduce the cohomology groups $H^n(X,A)$.
\item Introduce overtly discrete types and Barton-Commelin axioms:
\begin{center}
$\Pi_{x:X}I(x)$ is well-behaved for $X:\CHaus,\ I:X\to \ODisc$. 
\end{center}
\item Explain our main result:
\begin{center}
$H^n(X,A)$ is well-behaved for $X:\CHaus,\ A:X\to \Ab_{\ODisc}$.
\end{center}
\end{enumerate}
}

An abelian group is overtly discrete iff it is countably presented.

}


\section{SSD, Stone spaces and compact Hausdorff spaces}


\frame{\frametitle{Stone spaces}

\bloc{Definition}{
A type $X$ is a Stone space if it is a sequential limit of finite types.
}

%\bloc{Example 0: Finite types}{}

\bloc{Example 1 (Cantor space)}{
The type $2^\N$ is a Stone space.
}

Indeed $2^\N = \lim_{i:\N}\, 2^i$.

\bloc{Example 2 (Compactification of $\N$)}{
The type:
\[\N_\infty = \{\alpha:2^\N\ |\ \alpha\ hits\ 1\ at\ most\ once\}\]
is a Stone space.
}

}

\frame{
\vspace{0.3cm}

Indeed $\N_\infty$ is the limit of:
\[\xymatrix{
Fin(1) & Fin(2)\ar[l]_{-1} & Fin(3)\ar[l]_{-1} &Fin(4)\ar[l]_{-1} & \cdots\ar[l]_{-1} \\
\{\} & \{0\}\ar[l] & \{00\}\ar[l] & \{000\}\ar[l] &\cdots\ar[l] \\
	& \{1\}\ar[ul] & \{01\}\ar[ul] & \{001\} \ar[lu]  & \cdots\ar[lu] \\
	& 	& \{10\}\ar[ul] & \{010\}\ar[lu]  & \cdots\ar[lu] \\
	&	&	& \{100\}\ar[lu]  & \cdots\ar[lu]\\
	&	&	&	&\cdots\ar[lu]
}\]

}

\frame{\frametitle{Synthetic Stone duality}

\bloc{Axiom 1: Scott continuity}{
If $(S_k)_{k:\N}$ is a sequence of finite types, then
\[\colim_k(S_k\to 2) \overset{\simeq}{\to} (\lim_k\, S_k \to 2)\]
}

\bloc{Axiom 2: Weak K\"onig's lemma}{
If $(S_k)_{k:\N}$ is a sequence of inhabited finite types, then $\propTrunc{\lim_k\, S_k}$.
}

\bloc{Axiom 3: Local choice}{
Given $S:\Stone$ and $Y:S\to \Type$ such that $\Pi_{s:S}\propTrunc{Y(s)}$, 

there exists $T:\Stone$ and $p:T\twoheadrightarrow S$ such that $\Pi_{t:T}Y(p(t))$.
}

\bloc{Axiom 4: Dependent choice}{
}

}

%\frame{\frametitle{Stability of Stone spaces}

%\bloc{Definition}{
%A proposition that is a Stone space is called closed.
%}

%$P$ is closed iff it is of the form $\forall_{n:\N}\, D_n$ with $D_n$ decidable.

%\bloc{Proposition}{
%$\Stone$ is stable under $+$, $\Sigma$, $=$ and sequential limits.
%}

%Stable under $\propTrunc{\_}$, but not under general quotients!

%}


\frame{\frametitle{Compact Hausdorff space}

Stone spaces are not stable under quotients.

\bloc{Definition}{
A set $X$ is a compact Hausdorff space if: 
\begin{itemize}
\item Its identity types are Stone spaces.
\item There exists $S:\Stone$ and $S\twoheadrightarrow X$.
\end{itemize}
}

\bloc{Examples}{
The unit interval $\I = [0,1]$ is a such a quotient of $2^\N$.
}

%\bloc{Proposition}{
%$\CHaus$ is stable under $+$, $\Sigma$, $=$, sequential limits and quotients.
%}

%\bloc{Remark [CCGM24]}{
%Closed subtypes of Stone spaces and of $\mathbb{I}$ are as expected.
%}

}


\section{Intoduction to cohomology in HoTT}

\frame{\frametitle{Delooping abelian groups}

Fix $A$ an abelian group. We define $K(A,0)=A$.

\bloc{Proposition}{
Given $n>0$, there is a unique pointed type $K(A,n)$ such that:
\begin{itemize}
\item $K(A,n)$ is $(n-1)$-connected and $n$-truncated.
\item $\Omega^n K(A,n) = A$.
\end{itemize}
}

$K(A,n)$ is called the $n$-th delooping of $A$.

}

\frame{\frametitle{Cohomology groups}

\bloc{Definition}{
Given $n:\N,\, X:\Type,\, A:X\to \Ab$, we define:

\[H^n(X,A) = \propTrunc{\Pi_{x:X}K(A_x,n)}_0\]
}

\bloc{Remark}{
If $X$ has choice then $H^n(X,A) = 0$ for any $A$ and $n>0$. Indeed:
\begin{itemize}
\item For all $x:X,\, y:K(A_x,n)$ we have that $\propTrunc{y=*}$.
\item So for all $f:\Pi_{x:X}K(A_x,n)$ we have that $\Pi_{x:X} \propTrunc{f(x)=*}$.
\item So by choice $\propTrunc{f=0}$.
\end{itemize}
}

}


\section{Overtly discrete types and Barton-Commelin axioms}

\frame{\frametitle{Overtly discrete types}

We want a class of $A$ such that $H^n(X,A)$ is well-behaved for $X:\CHaus$

\bloc{Definition}{
A type is overtly discrete if it is a sequential colimit of finite types.
}

%\bloc{Remark}{
%A type $X$ is overtly discrete if and only if:
%\begin{itemize}
%\item It has open identity types.
%\item There exists a decidable subset $I\subset \N$ with $I\twoheadrightarrow X$.
%\end{itemize}
%}

An abelian group is overtly discrete iff it is countably presented.

%\bloc{Definition}{
%A proposition that is overtly discrete is called open.
%}

%$U$ is open iff it is of the form $\exists_{n:\N}D_n$ with $D_n$ decidable.

%\bloc{Remark [CCGM24]}{
%\vspace{-0.3cm}
%\[\xymatrix{
%\Open\ar@/^1.0pc/[rr]^\neg & \simeq & \Closed\ar@/^1.0pc/[ll]^\neg
%}\]
%\] \
%\Open \Closed
%}

}

%\frame{\frametitle{Overtly discrete types as quotients}

%\bloc{Remark}{
%A type $X$ is overtly discrete if and only if:
%\begin{itemize}
%\item It has open identity types.
%\item There exists a decidable subset $I\subset \N$ with $I\twoheadrightarrow X$.
%\end{itemize}
%}

%An abelian group is overtly discrete iff it is countably presented.

%\bloc{Stability}{
%$\ODisc$ is stable under $+$, $\Sigma$, $=$ sequential colimits and quotients.
%}

%}

\frame{\frametitle{Tychonov and its dual}

We prove Barton-Commelin's condensed type theory axioms.

\bloc{Lemma (Tychonov)}{
If $I:\ODisc$ and $X:I\to \CHaus$, then $\Pi_{i:I}X_i$ is compact Hausdorff.
}

\bloc{Proposition (Tychonov dual)}{
If $X:\CHaus$ and $I:X\to \ODisc$, then $\Pi_{x:X}I_x$ is overtly discrete.
}

This is encouraging. We have better!

}

\frame{\frametitle{Scott continuity}

\bloc{Definition}{
We have a category $\mathcal C$ where:
\begin{eqnarray}
\mathsf{Ob}_{\mathcal C} & = & \Sigma(X:\CHaus).\, X\to\ODisc\nonumber\\
\mathsf{Hom}_{\mathcal C}((X,I),(Y,J)) & = & \Sigma(f:Y\to X).\, \Pi_{y:Y} I_{f(x)}\to J_x\nonumber
\end{eqnarray}
%\begin{itemize}
%\item $\mathsf{Ob}_{\mathcal C} = \Sigma(X:\CHaus).\, X\to\ODisc$.
%\item $\mathsf{Hom}_{\mathcal C}((X,I),(Y,J)) = \Sigma(f:Y\to X).\, \Pi_{y:Y} I_{f(x)}\to J_x$
%\end{itemize}
}

\bloc{Theorem (Generalized Scott continuity)}{
The covariant functor:
\[\Pi : {\mathcal C} \to \ODisc\]
commutes with sequential colimits.
}

}


\section{Cohomology of Stone and compact Hausdorff spaces}

\frame{\frametitle{\v{C}ech cohomology}

\bloc{Definition}{
A \v{C}ech cover consists of $X:\CHaus, S:\Stone$ with $p:S\twoheadrightarrow X$.
}

\bloc{Definition}{
Given a \v{C}ech cover $p:S\twoheadrightarrow X$ and $A:X\to \Ab$, 

its \v{C}ech cohomology $\check{H}^n(X,S,A)$ is the $n$-th cohomology group of:

\[\Pi_{x:X}A_x^{S_x}\to \Pi_{x:X}A_x^{S_x^2} \to \Pi_{x:X}A_x^{S_x^3}\to \cdots\]
}

%\bloc{Definition}{
%Given a \v{C}ech cover $p:S\twoheadrightarrow X$ and $A:X\to \Ab$, its \v{C}ech cohomology $\check{H}^n(X,S,A)$ is defined as the $n$-th cohomology of $\check{C}(X,S,A)$.
%}

}

\frame{\frametitle{$\check{H}^1(S,T,A)=0$}

\bloc{Lemma}{
Given $S,T:\Stone,\, T\twoheadrightarrow S,\, A:S\to \Ab_{cp}$, we have that 

\[\check{H}^1(S,T,A) = 0\]
}

\bloc{Proof}{
\begin{itemize}
\item Show $p$ is a sequential limits of maps between finite types.
\item Show that $\check{H}^1(S,T,A)=0$ for $S$ finite.
\item Conclude by Scott continuity.
\end{itemize}
}

}

\frame{\frametitle{$H^1(S,A) = 0$}

\bloc{Lemma}{
Given $S:\Stone,\, A:S\to \Ab_{cp}$, we have that $H^1(S,A) = 0$.
}

\bloc{Proof}{
Assume $\alpha:\Pi_{x:S} K(A_x,1)$.
\begin{itemize}
\item By local choice we get $T\twoheadrightarrow S$ trivializing $\alpha$.
\item Since $\check{H}^1(S,T,A) = 0$ we can push the trivializing forward.
\end{itemize}
}

}

\frame{\frametitle{Main results}

\bloc{Theorem}{
Given $n>0$, $S:\Stone$ and $A:S\to\Ab_{cp}$, we have that

\[H^n(S,A) = 0.\]
}

\bloc{Theorem}{
Given a \v{C}ech cover $p:S\twoheadrightarrow X$ and $A:X\to\Ab_{cp}$, we have that

\[H^n(X,A) = \check{H}^n(X,S,A).\]
}

}

\frame{\frametitle{Applications}

\bloc{Lemma}{
For all $n>0$, $A:\mathbb{I}\to \Ab_{cp}$, we have that $H^n(\mathbb{I},A) = 0$.
}

This means $\I$-localisation is well-behaved. For example:

\bloc{Lemma}{
For $\mathbb{S}^k = \{x_0,\hdots,x_k:\R\ |\ \sum_i x_i^2 = 1\}$ and $A:\Ab_{cp}$, we have

\[H^n(\mathbb{S}^k,A) = 
\begin{cases}
	A & $if$\ n=0\ $or$\ n=k \\
    	0 & $otherwise$
\end{cases}\]
}

}

%\frame{\frametitle{$H^n(S,A) = 0$}

%\bloc{Theorem}{
%Given $S:\Stone,\, A:S\to\Ab_\ODisc$, we have that $H^n(S,A) = 0$.
%}

%\bloc{Proof}{
%By induction on $n$. Assume given $\alpha:\Pi_{x:S} K(A_x,n)$.
%\begin{itemize}
%\item By local choice there is $p:T\twoheadrightarrow S$ trivialising $\alpha$.
%\item We define $L_x$ as $A_x^{T_x}/A_x$ giving the exact sequence:
%\[H^{n-1}(S,L)\to H^n(S,A)\to H^n(S,A^T)\]
%By induction the second map is injective.
%\item By induction $K(A_x^{T_x},n)\to K(A_x,n)^{T_x}$ is an embedding, so:
%\[H^n(S,A^T) \to H^n(T,A\circ p)\]
%is injective.
%\item So $p^*:H^{n}(S,A)\to H^n(T,A\circ p)$ is injective, and $p^*(\alpha)=0$.
%\end{itemize}
%}

%}

%\frame{\frametitle{$H^n(X,A) = \check{H}^n(X,S,A)$}

%\bloc{Theorem}{
%Given a \v{C}ech cover $p:S\twoheadrightarrow X$ and $A:X\to\Ab_\ODisc$, we have that:
%\[H^n(X,A) = \check{H}^n(X,S,A)\]
%}

%\bloc{Proof}{
%From $0\to A_x\to A_x^{S_x}\to L_x\to 0$, we get long exact sequences:
%\[H^{n-1}(X,\lambda x.A_x^{S_x})\to H^{n-1}(X,L)\to H^n(X,A)\to H^n(X,\lambda x.A_x^{S_x})\]
%\[\check{H}^{n-1}(X,S,A^S)\to H^{n-1}(X,S,L)\to \check{H}^n(X,S,A)\to \check{H}^n(X,S,A^S)\]
%\begin{itemize}
%\item By vanishing, $H^n(X,A^S) = H^n(S,A\circ p) = 0$.
%\item By direct computation, $\check{H}^n(X,S,A^S)=0$
%\item We conclude by induction.
%\end{itemize}
%}

%}


\end{document}