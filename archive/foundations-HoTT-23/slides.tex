% latexmk -pdflatex='xelatex %O %S' -pvc -pdf slides.tex
\documentclass{beamer}

\beamertemplatenavigationsymbolsempty

% für xelatex:
\usepackage{xltxtra}
\usepackage{unicode-math}

% literatur
\usepackage[backend=biber,style=alphabetic]{biblatex}

\addbibresource{../util/literature.bib}

\usepackage{../util/zariski}

\usepackage{csquotes}
\usepackage{cancel}
\usepackage{tabularx}
\usepackage{hyperref}
\usepackage{tikz}
\usetikzlibrary{cd,arrows,shapes,calc,through,backgrounds,matrix,trees,decorations.pathmorphing,positioning,automata}
\usepackage{graphicx}
\usepackage{color}

\usepackage{mathpartir}
\newcommand{\yields}{\vdash}
\newcommand{\cbar}{\, | \,}


% für tabellen
\usepackage{booktabs}

\title{A Foundation for Synthetic Algebraic Geometry}
\author[Author, Anders] 
{Felix Cherubini, Thierry Coquand, Matthias Hutzler}
\date{Homotopy Type Theory 2023}

\begin{document}

\begin{frame}
  \titlepage
\end{frame}

\begin{frame}
  \frametitle{Thinking about algebra in a geometric way}
  Let $R$ be a ring.

  \[
    \onslide<+(1)->{\Spec \left(}
    \begin{aligned}
      X^2 + Y = 0 \\
      Y^2 = 0
    \end{aligned}
    \onslide<.(1)->{\right)
    \coloneq
    \Big\{\, (x, y) \in R^2 \mid x^2 + y = 0, y^2 = 0 \,\Big\}}
  \]
  %
  \pause
  \[
    \Spec\Big(\underbrace{(X^2 + Y, Y^2)}_{\text{ideal in $R[X, Y]$}}\Big)
    \coloneq
    \Big\{\, (x, y) \in R^2 \mid x^2 + y = 0, y^2 = 0 \,\Big\}
  \]
  %
  \pause
  \[
    \Spec\Big(R[X, Y]/(X^2 + Y, Y^2)\Big)
    \coloneq
    \Big\{\, (x, y) \in R^2 \mid x^2 + y = 0, y^2 = 0 \,\Big\}
  \]
  \[
    \Spec(A) \coloneq \Hom_{R\text{-Alg}}(A, R)
  \]

  \vspace{5mm}
  \pause
  But does $\Spec(A)$ retain all information from $A$? No. :-(
\end{frame}

\begin{frame}
  \frametitle{Classical vs synthetic}

  How can we make the functor
  \[ A \mapsto \Spec(A) \]
  fully faithful?
  \vspace{5mm}

  \begin{tabularx}{\textwidth}{X|X}
    Classical algebraic geometry
    &
    Synthetic algebraic geometry
    \\ \midrule
    Endow $\Spec(A)$ with additional structure:
    \begin{itemize}
      \item
        Zariski topology
      \item
        structure sheaf $\mathcal{O}_{\Spec(A)}$
    \end{itemize}
    &
    Just postulate it! :-)

    \textbf{Axiom (SQC)\footnote{%
    \enquote{Synthetic Quasi-Coherence}, due to Ingo Blechschmidt}.}
    The map
    \begin{align*}
      A \to R^{\Spec A} \\
      a\mapsto (\varphi\mapsto \varphi(a))
    \end{align*}
    is an equivalence,
    for every finitely presented $R$-algebra $A$.
  \end{tabularx}
\end{frame}

\begin{frame}
  \frametitle{Basic consequences of SQC}

  \[ A \xrightarrow{\sim} R^{\Spec A}\]

  \vspace{5mm}
  \begin{itemize}
    \item
      $\Spec(R[X]) = R$.
      Thus:
      $R[X] \xrightarrow{\sim} R^R$
  \end{itemize}

  \pause
  \vspace{5mm}
  If $\Spec(A) = \varnothing$,
  then $A = R^\varnothing = 0$.

  \vspace{2.5mm}
  \begin{itemize}
    \item
      $\Spec(R/(r)) = (r = 0)$.
      Thus:
      if $r \neq 0$,
      then $r$ is invertible.
    \item
      $\Spec(R[r^{-1}]) = (\text{$r$ is invertible})$.
      Thus:
      if $r$ is not invertible,
      then $r$ is nilpotent.
  \end{itemize}

  \pause
  \vspace{5mm}
  \textbf{Axiom:} The ring $R$ is local.

  \vspace{2.5mm}
  \begin{itemize}
    \item
      If $r_1, \dots, r_n : R$ are not all zero,
      then some $r_i$ is invertible.
  \end{itemize}
\end{frame}

\begin{frame}
  \frametitle{Closed and open propositions}

  For $r_1, \dots, r_n : R$ we have the propositions
  \[ V(r_1, \dots, r_n) \coloneq (r_1 = \dots = r_n = 0) \rlap{,}\]
  \[ D(r_1, \dots, r_n) \coloneq (\text{$r_1$ inv.} \lor \dots \lor \text{$r_1$ inv.}) \rlap{.}\]

  Then define:
  \[ \mathrm{closedProp} \coloneq
     \sum_{p : \mathrm{hProp}} \exists r_1, \dots, r_n.\, (p = V(r_1, \dots, r_n))
  \]
  \[ \mathrm{openProp} \coloneq
     \sum_{p : \mathrm{hProp}} \exists r_1, \dots, r_n.\, (p = D(r_1, \dots, r_n))
  \]

  \vspace{5mm}
  A \emph{closed subtype} of $X$
  is a map $X \to \mathrm{closedProp}$.\\
  An \emph{open subtype} of $X$
  is a map $X \to \mathrm{openProp}$.
\end{frame}

\begin{frame}
  \frametitle{Schemes}

  A type $X$ is an \emph{affine scheme} if
  it is of the form $X = \Spec(A)$.

  \vspace{5mm}
  A type $X$ is a \emph{scheme} if
  there exist $U_1, \dots, U_n : X \to \mathrm{openProp}$
  such that $X = \bigcup_i U_i$
  and every $U_i$ is an affine scheme.

  \pause
  \vspace{5mm}
  \textbf{Example.}
  \emph{Projective $n$-space}:
  \begin{align*}
    \bP^n
    &\coloneq \{\,\text{lines through $0$ in $R^{n+1}$}\,\} \\
    &\coloneq \{\,\text{sub-$R$-modules $L \subseteq R^{n+1}$ such that $\lVert L = R^1 \rVert$}\,\}
  \end{align*}
  is a scheme:
  \[
    U_i(L) \coloneq \text{($b_i$ is invertible) \textit{(for any chosen base $\{b\}$ of $L$)}}
  \]
\end{frame}

\begin{frame}
  \frametitle{Line bundles}

  The type
  \[ \mathrm{Lines} \coloneq \sum_{L : R\text{-Mod}} \lVert L = R^1 \rVert \]
  has a wild group structure:
  \begin{itemize}
    \item
      $L \otimes L'$ is again a line
    \item
      $L^{\vee} \coloneq \Hom(L, R^1)$ is the inverse
  \end{itemize}

  \pause
  \vspace{5mm}
  A \emph{line bundle} on $X$ is a map $X \to \mathrm{Lines}$.

  \textbf{Example.}
  tautological line bundle on $\bP^n$

  \pause
  \vspace{5mm}
  The \emph{Picard group} of $X$ is
  \[ \mathrm{Pic}(X) \coloneq \lVert X \to \mathrm{Lines} \rVert_{\text{set}} \rlap{.}\]

  \vspace{5mm}
  (In fact, $\mathrm{Lines} = K(R^{\times}, 1)$
  and $\mathrm{Pic}(X) = H^1(X, R^{\times})$.)
\end{frame}

\begin{frame}
  \frametitle{Zariski-local choice}

  For $f : A$ define $D(f) \coloneq \{\,\varphi : \Spec(A) \mid \text{$\varphi(f)$ is invertible}\,\}$.

  \vspace{2.5mm}
  \textbf{Axiom (Zariski-local choice):}\\
  For every surjective $\pi$, there merely exist local sections $s_i$
  \[ \begin{tikzcd}[ampersand replacement=\&, column sep=small]
    \& E \ar[d, two heads, "\pi"] \\
    D(f_i) \ar[r, hook] \ar[ur, bend left, dashed, "s_i"] \& \Spec(A)
  \end{tikzcd} \]
  with $f_1, \dots, f_n : A$ coprime.

  \pause
  \vspace{5mm}
  Some consequences:
  \begin{itemize}
    \item
      Every line bundle (on a scheme) is locally trivial.
    \item
      $(\Spec A \to \mathrm{closedProp}) \cong \{\,\text{fin. gen. ideals in $A$}\,\}$
    \item
      $(\Spec A \to \mathrm{openProp}) \cong \{\,\text{fin. gen. radical ideals in $A$}\,\}$
    \item
      If $U \subseteq X$ open and $V \subseteq U$ open,
      then $V \subseteq X$ open.
  \end{itemize}
\end{frame}

\begin{frame}
  \frametitle{The scheme classifier}

  Let $\mathrm{Sch} \hookrightarrow \mathrm{Type}$
  be the type of schemes.

  \vspace{5mm}
  \textbf{Theorem.}
  Let $X$ be a scheme and $Y : X \to \mathrm{Sch}$ be given.
  Then $\sum_{x : X} Y(x)$ is a scheme.

  \vspace{2.5mm}
  \textbf{Corollary.}
  For $f : Y \to X$ a map between schemes
  and $x : X$,
  the fiber $\sum_{y : Y} \underbrace{f(y) = x}_{\text{a scheme}}$
  is a scheme.

  \pause
  \vspace{5mm}
  This means we have a scheme classifier:
  \[ \mathrm{Sch}/X = (X \to \mathrm{Sch}) \]

  \vspace{2.5mm}
  In particular, we have a subscheme-classifier:
  $\mathrm{Sch} \cap \mathrm{hProp}$.
\end{frame}

\begin{frame}
  \centering
  Thank you!
\end{frame}

\end{document}


\begin{frame}
  \frametitle{Cohomology of sheaves}
  Let $X$ be a type and $\mathcal F:X\to \mathrm{Ab}$ a dependent abelian group on $X$. \\
  \pause
  The $n$-th cohomology group of $\mathcal F$ is
  \[ H^n(X,\mathcal F):\equiv\left\|\prod_{x:X}K(\mathcal F_x,n)\right\|_0 \]
  \pause
  Properties: \\
  \pause
  The $H^n(X,\mathcal F)$ are all abelian groups. \\
  \pause
  Functoriality, covariant in $\mathcal F$, contravariant in $X$. \\
  \pause
  Some long exact sequence for coefficients. \\
  \pause
  We have a Mayer-Vietoris-Lemma and more generally correspondence with \v{C}ech-Cohomology, for nice enough spaces. \\
\end{frame}

\begin{frame}
  \frametitle{Zariski-Choice and Cohomology}  
  Let $X=\Spec(A)$ and $M:X\to R\text{-Mod}$ such that $((x:D(f))\to M_x)=((x:X)\to M)_f$, then
  \[ H^1(X,M)=0 \]
  \pause
  \textbf{Proof:} Let $|T|: H^1(X,M)\equiv \|(x:X)\to K(M_x,1)\|_0$ and from that $(x:X)\to \|T_x=M_x\|$.
  \pause
  Our third axiom, \textbf{Zariski-local choice}, merely gives us coprime $f_1,\dots,f_n:A$, such that for each $i$ we have
  \[
    s_i:(x:D(f_i)) \to T_x=M_x
    \rlap{.}
  \]
  \pause
  So for $t_{ij}(x):\equiv s_j(x)^{-1}\cdot s_i(x)$ we have $t_{ij}+t_{jk}=t_{ik}$.
  \pause
  By algebra, we get $u_i:(x:D(f_i))\to M_x$ with $t_{ij}=u_i-u_j$.
  Then the $\tilde{s}_i:\equiv s_i-u_i$ glues to a global trivialization.
\end{frame}

