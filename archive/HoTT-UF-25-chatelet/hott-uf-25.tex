\documentclass[proof]{beamer}
\usetheme{default}

\usecolortheme{rose}
\usepackage[english]{babel}
\usepackage[utf8]{inputenc}
%\usepackage[latin1]{inputenc}
 \usepackage{amssymb}
 \usepackage{latexsym}
 \usepackage{amsmath}
 \usepackage{amssymb}
 \usepackage{tikz}
 \usepackage{tikz-cd}
\usepackage{array}
\usepackage{rotating}
\usepackage{forest}
\usepackage{color}
\usepackage[all,cmtip]{xy}

\usepackage{pgfplots}
\usepackage{array}
    \newcolumntype{C}{>{\centering\arraybackslash}c}



\definecolor{qbblue}{RGB}{43,126,128}
%\definecolor{qbblue}{RGB}{43,126,204}
\definecolor{qborange}{RGB}{242,151,36}
\definecolor{qblight}{RGB}{210,210,210}
\definecolor{qbdark}{RGB}{180,180,180}
\definecolor{qbred}{RGB}{184,13,72}

\setbeamercolor{title}{fg=black}
\setbeamercolor{section in toc}{fg=black}
\setbeamercolor{block title}{bg=qblight,fg=black}
\setbeamercolor{frametitle}{bg=qblight,fg=black}
\setbeamercolor{section in head/foot}{bg=black}
\setbeamertemplate{itemize item}{\color{qbdark}$\blacktriangleright$}
\setbeamertemplate{itemize subitem}{\color{qbdark}$\blacktriangleright$}
%\setbeamercolor*{palette tertiary}{bg=black}
\setbeamercolor{local structure}{fg=black}

\usepackage[style=verbose,backend=biber]{biblatex}
\addbibresource{biblio.bib}

\renewcommand{\_}{\rule{.6em}{.5pt}\hspace{0.023cm}}

\newcommand{\bloc}[2]{\begin{block}{#1}\setlength\abovedisplayskip{0pt}#2\end{block}}

\setbeamertemplate{navigation symbols}{} 
\addtobeamertemplate{footline}{\hfill{\tiny \insertframenumber}\hspace{2em} \vspace{1em}}

\newcommand{\red}[1]{\textcolor{qbred}{#1}}
\newcommand{\blue}[1]{\textcolor{qbblue}{#1}}
\newcommand{\orange}[1]{\textcolor{qborange}{#1}}

\newcommand{\D}{\mathbb{D}}
\newcommand{\bP}{\mathbb{P}}
\newcommand{\propTrunc}[1]{\lVert #1 \rVert}
\newcommand{\eqspace}{\vspace{0.32cm}}

\def\mhyphen{{\hbox{-}}}

%\newcommand{\trunc}[1]{|| #1 ||}

\AtBeginSection[]
{
 \begin{frame}<beamer>
 \frametitle{Outline}
 \tableofcontents[currentsection]
 \end{frame}
}

\begin{document}


%\abovedisplayskip=0.0cm
%\abovedisplayshortskip=-0.3cm
%\belowdisplayskip=0.6cm


\title{\red{Ch\^atelet's Theorem in Synthetic Algebraic Geometry}}
\author{Hugo Moeneclaey\\
Gothenburg University and Chalmers University of Technology\\
\vspace{0.32cm}
j.w.w Thierry Coquand}
\date{\blue{HoTT/UF 2025}\\
Genoa}



\frame{\titlepage}




\frame{\frametitle{Overview}

Algebraic geometry is the study of \red{roots of systems of polynomials}.
\pause
\bloc{Theorem [Ch\^atelet 1944]}{
Any inhabited Severi-Brauer variety is a projective space.
}
\pause
\bloc{Goal}{
Give a \red{synthetic} proof of Ch\^atelet's theorem.
}

}


\frame{\frametitle{Overview}

\bloc{Today}{ 
\begin{enumerate}
\item Introduce synthetic algebraic geometry.
\item Define \red{\'etale sheaves}. Being an \'etale sheaf is a \red{lex modality}.
\item Use \red{\'etale sheafification} to define Severi-Brauer varieties.
\item Sketch the proof of Ch\^atelet's theorem. 

Essentially \red{lex modality reasoning} + some linear algebra.
\end{enumerate}
}
\pause
\bloc{Main point of this talk}{
Working with \red{\'etale sheaves} in synthetic algebraic geometry is \red{convenient} and \red{helpful}.
}

}


% \begin{frame}<beamer>
% \frametitle{Outline}
% \tableofcontents
% \end{frame}



\section{Introduction to synthetic algebraic geometry}

%\frame{\frametitle{Traditional algebraic geometry, in one slide}}

\frame{\frametitle{What is synthetic algebraic geometry?}

It consists of HoTT plus 3 axioms:

\bloc{Axiom 1}{
There is a \red{local ring $R$}.
}

$R$ is assumed to be a set.

}

\frame{\frametitle{Affine schemes}

For $A$ a finitely presented algebra, we define:
\[Spec(A) = Hom_{R\mhyphen Alg}(A,R)\]

\bloc{Example}{
If:
\[A = R[X]/P\]
then:
\[Spec(A) = \{x:R\ |\ P(x) = 0\}\]
}
\pause
\bloc{Definition}{
A type $X$ is an \red{affine scheme} if there is an f.p.\ algebra $A$ such that:
\vspace{-0.1cm}
\[X=Spec(A)\]
}

}

\frame{

\bloc{Axiom 2: \red{Duality}}{
For any f.p.\ algebra $A$ the map:

\[A \to R^{Spec(A)}\]
is an equivalence.
}
\pause
Then:
\begin{itemize}
\item $Spec : \{f.p.\ algebras\} \simeq \{Affine\ schemes\}$
\pause
\item All maps between affine schemes are polynomials.
\end{itemize}

}


\frame{

\bloc{Axiom 3: \red{Zariski local choice}}{
Affine schemes enjoys a weakening of the axiom of choice.
}

}


\frame{\frametitle{Schemes}

\bloc{Defintion}{
A type is a \red{scheme} if it has a finite open cover by affine schemes.
}
\pause
\bloc{Example}{
The \red{projective space}:
\vspace{0.42cm}
\[\bP^n = \{Lines\ in\ R^{n+1}\ going\ through\ the\ origin\}\] %= (R^{n+1}\backslash\{0\}) / R^\times\]
is a scheme.
}

}


\frame{\frametitle{Ch\^atelet's theorem in traditional algebraic geometry}

\bloc{Definition (Traditional algebraic geometry, $k$ a field)}{
A \red{Severi-Brauer variety over $k$} is a $k$-scheme $X$ such that $X$ is a projective space over the separable closure of $k$.
}

\bloc{Theorem [Ch\^atelet 1944]}{
Any inhabited Severi-Brauer variety is a projective space.
}
\pause
Synthetically, for $X$ a Severi-Brauer variety:
\vspace{-0.12cm}
\[\propTrunc{X}\to\exists (n:\mathbb{N}).\, \propTrunc{X=\bP^n}\]
\pause
\vspace{-0.64cm}
\bloc{Challenge}{
How to define synthetic Severi-Brauer varieties?
}

%We use \'etale sheaves.

}



\section{\'Etale sheaves}


\frame{\frametitle{Definition}

\bloc{Definition}{
A type $X$ is an \red{\'etale sheaf} if for any monic unramifiable $P:R[X]$, we have a unique filler:
\vspace{0.24cm}
\[\xymatrix{
\exists(x:R).\, P(x)=0\ar[r]\ar[d] & X\\
1\ar@{-->}[ru]_{\exists !} & 
}\]
}

\'Etale sheaves behave as if any monic unramifiable $P$ had a root.
\pause
\bloc{Remark}{
By [Wraith 79], this should correspond to \red{traditional \'etale sheaves}.
}

}


\frame{

Being an \'etale sheaf is a \red{lex modality}.

So we have an \red{\'etale sheafification} [Rijke, Shulman, Spitters 2017].

\bloc{Remark}{
HoTT can be interpreted \red{inside \'etale sheaves}:
\begin{itemize}
\item The universe $U$ is interpreted as the type $U_{Et}$ of \'etale sheaves. %is itself an \'etale sheaf .
\item $\Sigma$, $\Pi$ and identity types are interpreted as themselves.
\item Truncation $\propTrunc{\_}$ is interpreted as the \'etale sheafification of the truncation, denoted $\propTrunc{\_}_{Et}$.
\end{itemize}
}

}


%\frame{

%Why do this match up with sheaves in the usual sense?

%Sets, $1$-types

%A bit complicated that we use monic unramifiable...

%}


\frame{\frametitle{Required properties of \'etale sheaves}


%The ring $R$ is an \'etale sheaf, so any affine scheme is an \'etale sheaf.

%Moreover:


\bloc{Lemma}{
\red{Any scheme} is an \'etale sheaf.
}
\pause
\bloc{Lemma}{
Let $M$ be a module that is an \'etale sheaf. 

The proposition \red{``$M$ is finite free"} is an \'etale sheaf.
}

Equivalently, \red{the type of finite free module} is an \'etale sheaf.

\bloc{Remark}{
Traditionally phrased as \red{\'etale descent for finite free modules}.
}

}


\section{Severi-Brauer varieties and Ch\^atelet's theorem}

\frame{\frametitle{Severi-Brauer varieties}

We fix a natural number $n$.

\bloc{Definition}{
A \red{Severi-Brauer variety} is an \'etale sheaf $X$ such that $\propTrunc{X=\bP^n}_{Et}$.
}

We write:
\[SB_n = \{X:U_{Et}\ |\ \propTrunc{X=\bP^n}_{Et}\}\]

%So $X:SB_n$ is a projective space, assuming some roots in $R$.

}


\frame{\frametitle{Example: Conics}

Assume $2\not=0$.

\bloc{Example}{
For all $a,b:R$ invertible:
\eqspace
\[C(a,b) = \{[x:y:z]:\bP^2\ |\  ax^2 + by^2 = z^2\}\]
is a Severi-Brauer variety.
}

%\bloc{Proof}{
%\begin{itemize}
%\item $C(1,1)=\bP^1$ by direct computation.
%\item $C(c^2,d^2) = C(1,1)$ by the variable change $x\mapsto cx, y\mapsto dy$.
%\item $X^2-a$ and $X^2-b$ are unramifiable (as $2\not=0$).
%\end{itemize}
%Then assuming $\sqrt{a}$ and $\sqrt{b}$, we have: 
%\[C(a,b) = C(\sqrt{a}^2,\sqrt{b}^2) = C(1,1) = \bP^1\]
%}

}


\frame{\frametitle{Severi-Brauer varieties form a delooping}

\bloc{Theorem [Cherubini, Coquand, Hutzler, W\"arn 2024]}{
%The group of automorphism of $\bP^n$ is $PGL_{n+1}(R)$.
\vspace{-0.32cm}
\[Aut(\bP^n) = PGL_{n+1}(R)\]
}

\pause

Therefore:
\[SB_n = \{X:U_{Et}\ |\ \propTrunc{X=\bP^n}_{Et}\}\]
is the interpretation inside \'etale sheaves of: 
\vspace{-0.1cm}
\[ \{X:U\ |\ \propTrunc{X=\bP^n}\} = B(Aut(\bP^n)) = B(PGL_{n+1}(R))\] 

}



\frame{\frametitle{Azumaya algebras form a delooping}

\bloc{Lemma}{
%The group of automorphisms of the algebra $M_{n+1}(R)$ is $PGL_{n+1}(R)$.
\vspace{-0.32cm}
\[Aut_{R\mhyphen Alg}(M_{n+1}(R)) = PGL_{n+1}(R)\]
}

\pause

So we define the type of Azumaya algebras:
\[AZ_n = \{A:R\mhyphen Alg_{Et}\ |\ \propTrunc{A=M_{n+1}(R)}_{Et}\}\]
It is also the interpretation of $B(PGL_{n+1}(R))$ inside \'etale sheaves.
}


\frame{\frametitle{Severi-Brauer varieties from Azumaya algebras}

By \red{unicity of deloopings}, we know that $AZ_n \simeq SB_n$. 

\pause 

We can be more concrete:
\bloc{Lemma}{
The pointed map:
\vspace{0.12cm}
\begin{eqnarray}
LI &:& AZ_n \to SB_n\nonumber\\
LI(A) &=& \{I\ left\ ideals\ in\ A\ |\ I\ free\ of\ rank\ n+1\}\nonumber
\end{eqnarray}
\vspace{-0.76cm}

is an equivalence.
}

\pause

\bloc{Proof sketch}{
$LI$ induces a map: 
\vspace{0.1cm}
\[\alpha:PGL_{n+1}(R) = \Omega AZ_n \overset{\Omega LI}{\longrightarrow} \Omega SB_n = PGL_{n+1}(R)\]

\vspace{-0.22cm}

By \red{lex modality reasoning}, we just need that $\alpha$ is the identity.

Check this by direct computation.
}

%\bloc{Remark}{
%This means all Severi-Brauer varieties are schemes.
%}

}


\frame{

\bloc{Lemma}{
Given $A:AZ_n$, if we have $I:LI(A)$ then $A = End_R(I)$.
}

\pause

\bloc{Proof sketch}{
By \red{lex modality reasoning}, we can assume $A=M_{n+1}(R)$.

Then we do some linear algebra.
}

\pause

\bloc{Corollary}{
Given $A:AZ_n$, if we have $\propTrunc{LI(A)}$ then $\propTrunc{A = M_{n+1}(R)}$.
}

}

\frame{\frametitle{Ch\^atelet's theorem in synthetic algebraic geometry}

\bloc{Theorem}{
For all $X:SB_n$, we have that $\propTrunc{X}$ implies $\propTrunc{X=\bP^n}$.
}

\bloc{Proof}{
\vspace{-0.32cm}

%\begin{eqnarray}
% & & \propTrunc{X}\nonumber\\
%(LI\ is\ an\ equivalence)&\Rightarrow& \exists(I:LI(LI^{-1}(X)))\nonumber\\ 
%(previous\ Lemma) &\Rightarrow& \propTrunc{LI^{-1}(X) = End_R(I)}\nonumber\\
%(I\ is\ finite\ free\ of\ rank\ n+1) &\Rightarrow&  \propTrunc{LI^{-1}(X) = M_{n+1}(R)}\nonumber\\
%(LI\ is\ an\ equivalence) &\Rightarrow& \propTrunc{X=LI(M_{n+1}(R))}\nonumber\\
%(direct\ computation) &\Rightarrow& \propTrunc{X=\bP^n}\nonumber
%\end{eqnarray}

\begin{eqnarray}
  & \propTrunc{X}& \nonumber\\
\Rightarrow& \propTrunc{LI(A)}&(write\ A=LI^{-1}(X))\nonumber\\ 
%\Rightarrow& \propTrunc{A = End_R(I)}& (previous\ Lemma) \nonumber\\
\Rightarrow&  \propTrunc{A = M_{n+1}(R)}& (previous\ Corollary) \nonumber\\
\Rightarrow& \propTrunc{X=LI(M_{n+1}(R))}& (apply\ LI) \nonumber\\
\Rightarrow& \propTrunc{X=\bP^n} & (LI\ pointed) \nonumber
\end{eqnarray}

%$\propTrunc{X}\Rightarrow \exists(I:LI(LI^{-1}(X))) \Rightarrow \propTrunc{LI^{-1}(X) = End_R(I)} \Rightarrow  \propTrunc{LI^{-1}(X) = M_{n+1}(R)} \Rightarrow \propTrunc{X=LI(M_{n+1}(R))} \Rightarrow \propTrunc{X=\bP^n}$.
}

}







\end{document}