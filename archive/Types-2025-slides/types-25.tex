\documentclass[proof]{beamer}
\usetheme{default}

\usecolortheme{rose}
\usepackage[english]{babel}
\usepackage[utf8]{inputenc}
%\usepackage[latin1]{inputenc}
 \usepackage{amssymb}
 \usepackage{latexsym}
 \usepackage{amsmath}
 \usepackage{amssymb}
 \usepackage{tikz}
 \usepackage{tikz-cd}
\usepackage{array}
\usepackage{rotating}
\usepackage{forest}
\usepackage{color}
\usepackage[all,cmtip]{xy}

\usepackage{pgfplots}
\usepackage{array}
    \newcolumntype{C}{>{\centering\arraybackslash}c}



\definecolor{qbblue}{RGB}{43,126,128}
%\definecolor{qbblue}{RGB}{43,126,204}
\definecolor{qborange}{RGB}{242,151,36}
\definecolor{qblight}{RGB}{210,210,210}
\definecolor{qbdark}{RGB}{180,180,180}
\definecolor{qbred}{RGB}{184,13,72}

\setbeamercolor{title}{fg=black}
\setbeamercolor{section in toc}{fg=black}
\setbeamercolor{block title}{bg=qblight,fg=black}
\setbeamercolor{frametitle}{bg=qblight,fg=black}
\setbeamercolor{section in head/foot}{bg=black}
\setbeamertemplate{itemize item}{\color{qbdark}$\blacktriangleright$}
\setbeamertemplate{itemize subitem}{\color{qbdark}$\blacktriangleright$}
%\setbeamercolor*{palette tertiary}{bg=black}
\setbeamercolor{local structure}{fg=black}

\usepackage[style=verbose,backend=biber]{biblatex}
\addbibresource{biblio.bib}

\renewcommand{\_}{\rule{.6em}{.5pt}\hspace{0.023cm}}

\newcommand{\bloc}[2]{\begin{block}{#1}\setlength\abovedisplayskip{0pt}#2\end{block}}

\setbeamertemplate{navigation symbols}{} 
\addtobeamertemplate{footline}{\hfill{\tiny \insertframenumber}\hspace{2em} \vspace{1em}}

\newcommand{\red}[1]{\textcolor{qbred}{#1}}
\newcommand{\blue}[1]{\textcolor{qbblue}{#1}}
\newcommand{\orange}[1]{\textcolor{qborange}{#1}}


\newcommand{\propTrunc}[1]{\lVert #1 \rVert}
\newcommand{\CHaus}{CHaus}
\newcommand{\Stone}{Stone}
\newcommand{\ODisc}{ODisc}
\newcommand{\Z}{\mathbb{Z}}
\newcommand{\N}{\mathbb{N}}
\newcommand{\U}{Type}

\def\mhyphen{{\hbox{-}}}

%\newcommand{\trunc}[1]{|| #1 ||}

\AtBeginSection[]
{
 \begin{frame}<beamer>
 \frametitle{Outline}
 \tableofcontents[currentsection]
 \end{frame}
}

\begin{document}


%\abovedisplayskip=0.0cm
%\abovedisplayshortskip=-0.3cm
%\belowdisplayskip=0.6cm


\title{\red{Cohomology in Synthetic Stone Duality}}
\author{Hugo Moeneclaey\\
Gothenburg University and Chalmers University of Technology\\
\vspace{0.32cm}
j.w.w Thierry Coquand, Felix Cherubini and Freek Geerligs}
\date{\blue{TYPES 2025}\\
Glasgow}



\frame{\titlepage}



\frame{\frametitle{Overview}

We work in Synthetic Stone Duality (SSD). 

SSD = HoTT + 4 axioms.

\bloc{Cohomology in HoTT}{
Given $n:\N,\ X:\U,\ A:X\to Ab$, we define a group $H^n(X,A)$.
}

$H^n(X,A)$ is the $n$-th cohomology group of $X$ with coefficient $A$.

\bloc{Our previous work [CCGM24]}{
\begin{itemize}
\item Showed SSD is suitable for synthetic topological study of Stone and compact Hausdorff spaces.
\item Proved $H^1(X,\Z)$ is well-behaved for $X:\CHaus$ .
\end{itemize}
}

\bloc{Goal}{
$H^n(X,A)$ is well-behaved for $n:\N,\ X:\CHaus,\ A:X\to Ab_{cp}$.
}

}



\frame{\frametitle{Overview}

\bloc{Today}{ 
\begin{enumerate}
\item Introduce SSD, Stone spaces and compact Hausdorff spaces.
\item Introduce the cohomology groups $H^n(X,A)$.
\item Introduce overtly discrete types and Barton-Commelin axioms:
\begin{center}
$\prod_{x:X}I(x)$ is well-behaved for $X:\CHaus,\ I:X\to \ODisc$. 
\end{center}
\item Explain our main result:
\begin{center}
$H^n(X,A)$ is well-behaved for $X:\CHaus,\ A:X\to Ab_{\ODisc}$.
\end{center}
\end{enumerate}
}

An abelian group is overtly discrete iff it is countably presented.

}


\section{SSD, Stone spaces and compact Hausdorff spaces}


\frame{\frametitle{Stone spaces}

\bloc{Definition}{
A type $X$ is a Stone space if it merely is a sequential limit of finite types.
}

\bloc{Example 1: Cantor space}{
The type $2^\N$ is a Stone space. Indeed $2^\N = lim_i\, 2^i$.
}

\bloc{Example 2: The compactification of $\N$}{
The type:
\[\N_\infty = \{\alpha:2^\N\ |\ \alpha\ hits\ 1\ at\ most\ once\}\]
is a Stone space.
}

}

\frame{
\vspace{0.3cm}
%\bloc{Example 2: The compactification of $\N$}{
%The compactification of $\N$, defined as:
%\[\N_\infty = \{\alpha:2^\N\ |\ \alpha\ hits\ 1\ at\ most\ once\}\]
%is the limit of:

Indeed $\N_\infty$ is the limit of:
\[\xymatrix{
Fin(1) & Fin(2)\ar[l]_{-1} & Fin(3)\ar[l]_{-1} &Fin(4)\ar[l]_{-1} & \cdots\ar[l]_{-1} \\
\{\} & \{0\}\ar[l] & \{00\}\ar[l] & \{000\}\ar[l] &\cdots\ar[l] \\
	& \{1\}\ar[ul] & \{01\}\ar[ul] & \{001\} \ar[lu]  & \cdots\ar[lu] \\
	& 	& \{10\}\ar[ul] & \{010\}\ar[lu]  & \cdots\ar[lu] \\
	&	&	& \{100\}\ar[lu]  & \cdots\ar[lu]\\
	&	&	&	&\cdots\ar[lu]
}\]

%}

}


\frame{\frametitle{Synthetic Stone duality}

\bloc{Axiom 1: Continuity}{
If $(S_k)_{k:\N}$ is a sequence of finite type, then
\[colim_k(S_k\to 2) \overset{\simeq}{\to} (lim_k\, S_k \to 2)\]
}

\bloc{Axiom 2: Completeness (Equivalent to WKL)}{
If $(S_k)_{k:\N}$ is a sequence of inhabited finite type, then $\propTrunc{lim_k\, S_k}$.
}

\bloc{Axiom 3: Local choice}{
Given $S:\Stone$ and $Y:S\to Type$ such that $\prod_{s:S}\propTrunc{Y(s)}$, 

there exists $T:\Stone$ and $p:T\twoheadrightarrow S$ such that $\prod_{t:T}Y(p(t))$.
}

\bloc{Axiom 4: Dependent choice}{
}

}


\frame{\frametitle{Stability of Stone spaces}

We have ...

But no quotients!

}


\frame{\frametitle{Compact Hausdorff space}

\bloc{Definition}{}

Examples

Compact Hausdorff have the same stability as Stone Spaces + quotients

}



\section{Intoduction to cohomology}

\frame{\frametitle{Eilenberg Mac Lane spaces}}

\frame{\frametitle{Cohomology groups}

Main challenge: find a good class of coefficient

}


\section{Overtly discrete types and Barton-Commelin axioms}

\frame{\frametitle{Definition}}

\frame{\frametitle{Stability}}

\frame{\frametitle{Tychonov and its dual}}

\frame{\frametitle{Scott Continuity}}


\section{Main results}




\end{document}