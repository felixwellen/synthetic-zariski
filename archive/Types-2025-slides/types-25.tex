\documentclass[proof]{beamer}
\usetheme{default}

\usecolortheme{rose}
\usepackage[english]{babel}
\usepackage[utf8]{inputenc}
%\usepackage[latin1]{inputenc}
 \usepackage{amssymb}
 \usepackage{latexsym}
 \usepackage{amsmath}
 \usepackage{amssymb}
 \usepackage{tikz}
 \usepackage{tikz-cd}
\usepackage{array}
\usepackage{rotating}
\usepackage{forest}
\usepackage{color}
\usepackage[all,cmtip]{xy}

\usepackage{pgfplots}
\usepackage{array}
    \newcolumntype{C}{>{\centering\arraybackslash}c}



\definecolor{qbblue}{RGB}{43,126,128}
%\definecolor{qbblue}{RGB}{43,126,204}
\definecolor{qborange}{RGB}{242,151,36}
\definecolor{qblight}{RGB}{210,210,210}
\definecolor{qbdark}{RGB}{180,180,180}
\definecolor{qbred}{RGB}{184,13,72}

\setbeamercolor{title}{fg=black}
\setbeamercolor{section in toc}{fg=black}
\setbeamercolor{block title}{bg=qblight,fg=black}
\setbeamercolor{frametitle}{bg=qblight,fg=black}
\setbeamercolor{section in head/foot}{bg=black}
\setbeamertemplate{itemize item}{\color{qbdark}$\blacktriangleright$}
\setbeamertemplate{itemize subitem}{\color{qbdark}$\blacktriangleright$}
%\setbeamercolor*{palette tertiary}{bg=black}
\setbeamercolor{local structure}{fg=black}

\usepackage[style=verbose,backend=biber]{biblatex}
\addbibresource{biblio.bib}

\renewcommand{\_}{\rule{.6em}{.5pt}\hspace{0.023cm}}

\newcommand{\bloc}[2]{\begin{block}{#1}\setlength\abovedisplayskip{0pt}#2\end{block}}

\setbeamertemplate{navigation symbols}{} 
\addtobeamertemplate{footline}{\hfill{\tiny \insertframenumber}\hspace{2em} \vspace{1em}}

\newcommand{\red}[1]{\textcolor{qbred}{#1}}
\newcommand{\blue}[1]{\textcolor{qbblue}{#1}}
\newcommand{\orange}[1]{\textcolor{qborange}{#1}}


\newcommand{\propTrunc}[1]{\lVert #1 \rVert}
\newcommand{\CHaus}{\mathsf{CHaus}}
\newcommand{\Stone}{\mathsf{Stone}}
\newcommand{\ODisc}{\mathsf{ODisc}}
\newcommand{\Type}{\mathsf{Type}}
\newcommand{\Ab}{\mathsf{Ab}}
\newcommand{\Z}{\mathbb{Z}}
\newcommand{\N}{\mathbb{N}}
\newcommand{\I}{\mathbb{I}}

\def\mhyphen{{\hbox{-}}}

%\newcommand{\trunc}[1]{|| #1 ||}

\AtBeginSection[]
{
 \begin{frame}<beamer>
 \frametitle{Outline}
 \tableofcontents[currentsection]
 \end{frame}
}

\begin{document}


%\abovedisplayskip=0.0cm
%\abovedisplayshortskip=-0.3cm
%\belowdisplayskip=0.6cm


\title{\red{Cohomology in Synthetic Stone Duality}}
\author{Hugo Moeneclaey\\
\vspace{0.32cm}
j.w.w. Felix Cherubini, Thierry Coquand and Freek Geerligs}
\date{\blue{TYPES 2025}\\
Glasgow}



\frame{\titlepage}



\frame{\frametitle{Overview}

We work in Synthetic Stone Duality (SSD). 

SSD = HoTT + 4 axioms.

\bloc{Cohomology in HoTT}{
Given $n:\N,\ X:\Type,\ A:X\to \Ab$, we define a group $H^n(X,A)$.
}

$H^n(X,A)$ is the $n$-th cohomology group of $X$ with coefficient $A$.

\bloc{Our previous work [CCGM24]}{
\begin{itemize}
\item Showed SSD is suitable for synthetic topological study of Stone and compact Hausdorff spaces.
\item Proved $H^1(X,\Z)$ is well-behaved for $X:\CHaus$ .
\end{itemize}
}

\bloc{Goal}{
$H^n(X,A)$ is well-behaved for $n:\N,\ X:\CHaus,\ A:X\to \Ab_{cp}$.
}

}



\frame{\frametitle{Overview}

\bloc{Today}{ 
\begin{enumerate}
\item Introduce SSD, Stone spaces and compact Hausdorff spaces.
\item Introduce the cohomology groups $H^n(X,A)$.
\item Introduce overtly discrete types and Barton-Commelin axioms:
\begin{center}
$\prod_{x:X}I(x)$ is well-behaved for $X:\CHaus,\ I:X\to \ODisc$. 
\end{center}
\item Explain our main result:
\begin{center}
$H^n(X,A)$ is well-behaved for $X:\CHaus,\ A:X\to \Ab_{\ODisc}$.
\end{center}
\end{enumerate}
}

An abelian group is overtly discrete iff it is countably presented.

}


\section{SSD, Stone spaces and compact Hausdorff spaces}


\frame{\frametitle{Stone spaces}

\bloc{Definition}{
A type $X$ is a Stone space if it merely is a sequential limit of finite types.
}

\bloc{Example 0: Finite types}{}

\bloc{Example 1: Cantor space}{
The type $2^\N$ is a Stone space. Indeed $2^\N = lim_i\, 2^i$.
}

\bloc{Example 2: The compactification of $\N$}{
The type:
\[\N_\infty = \{\alpha:2^\N\ |\ \alpha\ hits\ 1\ at\ most\ once\}\]
is a Stone space.
}

}

\frame{
\vspace{0.3cm}

Indeed $\N_\infty$ is the limit of:
\[\xymatrix{
Fin(1) & Fin(2)\ar[l]_{-1} & Fin(3)\ar[l]_{-1} &Fin(4)\ar[l]_{-1} & \cdots\ar[l]_{-1} \\
\{\} & \{0\}\ar[l] & \{00\}\ar[l] & \{000\}\ar[l] &\cdots\ar[l] \\
	& \{1\}\ar[ul] & \{01\}\ar[ul] & \{001\} \ar[lu]  & \cdots\ar[lu] \\
	& 	& \{10\}\ar[ul] & \{010\}\ar[lu]  & \cdots\ar[lu] \\
	&	&	& \{100\}\ar[lu]  & \cdots\ar[lu]\\
	&	&	&	&\cdots\ar[lu]
}\]

}

\frame{\frametitle{Synthetic Stone duality}

\bloc{Axiom 1: Continuity}{
If $(S_k)_{k:\N}$ is a sequence of finite type, then
\[colim_k(S_k\to 2) \overset{\simeq}{\to} (lim_k\, S_k \to 2)\]
}

\bloc{Axiom 2: Weak K\"onig's Lemma)}{
If $(S_k)_{k:\N}$ is a sequence of inhabited finite type, then $\propTrunc{lim_k\, S_k}$.
}

\bloc{Axiom 3: Local choice}{
Given $S:\Stone$ and $Y:S\to \Type$ such that $\prod_{s:S}\propTrunc{Y(s)}$, 

there exists $T:\Stone$ and $p:T\twoheadrightarrow S$ such that $\prod_{t:T}Y(p(t))$.
}

\bloc{Axiom 4: Dependent choice}{
}

}

\frame{\frametitle{Stability of Stone spaces}

\bloc{Definition}{
A proposition that is a Stone space is called closed.
}

$P$ is closed iff it merely is of the form $\forall_{n:\N}D_n$ with $D_n$ decidable.

\bloc{Proposition}{
Stone spaces are stable under $+$, $\Sigma$-types (subtle), identity types and sequential limits.
}

But they are not stable under quotients!

}


\frame{\frametitle{Compact Hausdorff space}

\bloc{Definition}{
A set $X$ is called a compact Hausdorff space if: 
\begin{itemize}
\item It has closed identity types.
\item There exists $S:\Stone$ and $S\twoheadrightarrow X$.
\end{itemize}
}

\bloc{Examples}{
The unit interval $\I = [0,1]$ is a closed quotient of $2^\N$
}

Compact Hausdorff have the same stability as Stone Spaces + quotients.

\bloc{Remark [CCGM24]}{
Closed subtypes of Stone spaces and of $\mathbb{I}$ are as expected.
}

}


\section{Intoduction to cohomology in HoTT}

\frame{\frametitle{Delooping abelian groups}

We define $K(A,0)=A$.

\bloc{Proposition}{
Given $A:\Ab,\, n>0$, there exists a unique pointed $(n-1)$-connected $n$-truncated type $K(A,n)$ such that
\[\Omega^nK(A,n) = A\]
}

$K(A,n)$ is called the $n$-th delooping of $A$.

}

\frame{\frametitle{Cohomology groups}

\bloc{Definition}{
Given $X:\Type,\, A:X\to \Ab,\, n:\N$, we define:

\[H^n(X,A) = \propTrunc{\prod_{x:X}K(A_x,n)}_0\]
}

\bloc{Remark}{
If $X$ has choice then $H^n(X,A) = 0$ for any $A$ and $n>0$. Indeed:
\begin{itemize}
\item For all $y:K(A,n)$ we have that $\propTrunc{y=*}$.
\item So for all $f:\prod_{x:X}K(A_x,n)$ we have that $\prod_{x:X} \propTrunc{f(x)=*}$.
\item So by choice $\propTrunc{f=0}$.
\end{itemize}
}

}


\section{Overtly discrete types and Barton-Commelin axioms}

\frame{\frametitle{Overtly discrete types}

\bloc{Main challenge}{
Find a class of coefficients $A$ such that $H^n(X,A)$ is well-behaved for $X:\CHaus$.
}

\bloc{Definition}{
A type is overtly discrete if it merely is a sequential colimit of finite types.
}

\bloc{Definition}{
A proposition that is overtly discrete is called open.
}

$U$ open iff it merely is of the form $\exists_{n:\N}D_n$ with $D_n$ decidable.

\bloc{Remark [CCGM24]}{
Open propositions are precisely negation of closed propositions and vice versa.
}

}

\frame{\frametitle{Overtly discrete types as quotients}

\bloc{Remark}{
A type $X$ is overtly discrete if and only if:
\begin{itemize}
\item It has open identity types.
\item There exists a countable type $I$ (i.e. a decidable subtype of $\N$) with $I\twoheadrightarrow X$.
\end{itemize}
}

An abelian group is overtly discrete iff it is countably presented.

\bloc{Stability}{
Overtly discrete types are stable under $+$, $\Sigma$-types, identity types, sequential colimits and quotients by open equivalence relations.
}

}

\frame{\frametitle{Tychonov and its dual}

\bloc{Lemma (Tychonov)}{
For $I:\ODisc$ and $X:I\to \CHaus$, we have that $\prod_{i:I}X_i$ is compact Hausdorff.
}

\bloc{Proposition (Tychonov dual)}{
For $X:\CHaus$ and $I:X\to \ODisc$, we have that $\prod_{x:X}I_x$ is overtly discrete.
}

This is a first sign that taking overtly discrete (aka cp) abelian groups as coefficient is a good idea. We have better!

}

\frame{\frametitle{Scott Continuity revisited}

\bloc{Definition}{
We have a category $\mathcal C$ where:
\begin{itemize}
\item An object consists of $X:\CHaus$ and $I:X\to\ODisc$.
\item A morphism  from $(X,I)$ to $(Y,J)$ consists of $f:Y\to X$ with $\prod_{y:Y} I_{f(x)}\to J_x$.
\end{itemize}
}

\bloc{Theorem (Generalized Scott continuity)}{
The covariant functor:
\[\prod : {\mathcal C} \to \ODisc\]
commutes with sequential colimits.
}

The last three results are assumed as axioms in Barton-Commelin's condensed type theory.

}


\section{Cohomology results}

\frame{\frametitle{\v{C}ech cohomology}

\bloc{Definition}{
A \v{C} TODO
}

}

\frame{\frametitle{Vanishing of the first cohomology group}

\bloc{Lemma}{
Given $S:\Stone,\, A:S\to \Ab_\ODisc$, we have that $\check{H}^1(S,T,A) = 0$.
}

\bloc{Proof}{
\begin{itemize}
\item By finite approximation, 
\item $\check{H}^1(S,T,A)=0$ for $S$ finite 
\item conclude by Scott continuity.
\end{itemize}
}

\bloc{Lemma}{
Given $S:\Stone,\, A:S\to \Ab_\ODisc$, we have that $H^1(S,A) = 0$.
}

\bloc{Proof}{
Assume $\alpha:S\to K(A,1)$, 
\begin{itemize}
\item By local choice we get $T\to S$ trivializing $\alpha$.
\item Since $\check{H}^1(S,T,A) = 0$ we can glue the trivializing.
\end{itemize}
}

}


\frame{\frametitle{Going to $n$}}

\frame{\frametitle{\v{C}ech agree with usual}}






\end{document}