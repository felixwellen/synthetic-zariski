% latexmk -pvc -pdf slides.tex
\documentclass{beamer}

\beamertemplatenavigationsymbolsempty

% literatur
\usepackage[backend=biber,style=alphabetic]{biblatex}

\addbibresource{../util/literature.bib}

\usepackage{../util/zariski}

\usepackage{csquotes}
\usepackage{cancel}
\usepackage{tabularx}
\usepackage{hyperref}
\usepackage{tikz}
\usetikzlibrary{cd,arrows,shapes,calc,through,backgrounds,matrix,trees,decorations.pathmorphing,positioning,automata}
\usepackage{graphicx}
\usepackage{color}

\usepackage{mathpartir}
\newcommand{\yields}{\vdash}
\newcommand{\cbar}{\, | \,}

\newcommand{\nop}[1]{\textcolor{bg}{#1}}


% für tabellen
\usepackage{booktabs}

\title{Synthetic Algebraic Geometry}
\date{December 15, 2025}

\begin{document}

\begin{frame}
  \titlepage
\end{frame}

\begin{frame}
  This is the main part of a small field, containing work of the following people:
  \begin{center}
    \begin{tabularx}{\textwidth}{XXX}
      Peter Arndt & Reid Barton & Ulrik Buchholtz \\
      Ingo Blechschmidt & Felix Cherubini & Johan Commelin \\
      Thierry Coquand & Fabian Endres & Freek Geerligs \\
      Jonas Höfer & Tim Lichtnau & Hugo Moeneclaey \\
      David Jaz Myers & Marc N-W & Armaan Rashid \\ 
      Matthias Ritter & Christian Sattler & Lukas Stoll \\
      David Wärn & Mark Williams &
    \end{tabularx}
  \end{center}
\end{frame}

\begin{frame}
  The work I will describe now in more detail, was done by:
  \vspace{.5cm}
  \begin{center}
    \begin{tabularx}{\textwidth}{XXX}
      \nop{Peter Arndt} & \nop{Reid Barton} & \nop{Ulrik Buchholtz} \\
      \nop{Ingo Blechschmidt} & Felix Cherubini & \nop{Johan Commelin} \\
      Thierry Coquand & \nop{Fabian Endres} & \nop{Freek Geerligs} \\
      \nop{Jonas Höfer} & \nop{Tim Lichtnau} & Hugo Moeneclaey \\
      \nop{David Jaz Myers} & \nop{Marc N-W} & \nop{Armaan Rashid} \\ 
      Matthias Ritter & \nop{Christian Sattler} & \nop{Lukas Stoll} \\
      David Wärn & \nop{Mark Williams} &
    \end{tabularx}
  \end{center}
\end{frame}

\begin{frame}
  A more complete overview is at \url{https://github.com/felixwellen/synthetic-zariski}:
  \includegraphics[width=\textwidth]{repo.png}
\end{frame}

\begin{frame}
  Functor of points approach\dots \\
  \includegraphics[width=\textwidth]{demazure-fx.png}
\end{frame}

\begin{frame}
  \dots allows you to pretend you are working with sets! \\
  \includegraphics[width=\textwidth]{demazure-fx-circled.png}
\end{frame}

\begin{frame}
  Some constructions:\\
  \vspace{.3cm}
    \begin{tabular}{l|l}
      Concept & Functor of Points \\
      \midrule
      $\A^1$ & $A\mapsto A:\Alg{k}\to \mathrm{Set}$ \\
      $X\to Y$ & $\eta_A:X(A)\to Y(A)$ such that \dots \\
      $X\times_Z Y$ & $A\mapsto X(A)\times_{Z(A)} Y(A)$  \\
      $\bP^n$ & $A\mapsto \{M\subseteq A^{n+1}\mid \text{$M$ rank 1 submodule,} \text{ $M\oplus N\cong A^{n+1}$} \}$ \\
      $\Spec(B)$ & $A\mapsto \Hom_{\Alg{k}}(B,A)$ 
    \end{tabular}
    \pause
    \vspace{.7cm}\\
  Synthetically, we can work without the argument $A$:\\
  \vspace{.3cm}
    \begin{tabular}{l|l}
      Concept & Synthetic Algebraic Geometry \\
      \midrule
      $\A^1$ & axiomatically given ring $R$ \\ 
      $X\to Y$ & $f:X\to Y$ (function between sets) \\
      $X\times_Z Y$ & Just $X\times_Z Y$  \\
      $\bP^n$ & $\{L\subseteq R^{n+1}\mid \text{ $L$ rank 1 submodule}\}$ \\
      $\Spec(B)$ & $\Hom_{\Alg{R}}(B,R)$
    \end{tabular}
\end{frame}

\begin{frame}
  \frametitle{\textbf{Synthetic Algebraic Geometry}}
  \pause
  $R$ is a commutative ring. \\
  \vspace{0.4cm}
  \pause

  $V:= \{(x,y):R^2\mid x^2+y^2=1 \}$ \\

  \pause
  Adding equations like $0=0$, $x(x^2+y^2)=x$ does not change the type $V$. \\
  \pause
  $V$ is determined by the $R$-algebra $A:= R[X,Y]/(X^2+Y^2-1)$.
  \pause
  We can recover it: $V=\Hom_{\Alg{R}}(A,R)$. \\
  \vspace{0.4cm}
  \pause
  \begin{definition}
    \begin{enumerate}[(i)]
    \item An $R$-algebra is finitely presented (fp) if it is merely $R[X_1,\dots,X_n]/(P_1,\dots,P_l)$.
    \item $\Spec(A):= \Hom_{\Alg{R}}(A,R)$ is the \emph{spectrum} of an fp $R$-algebra $A$.
    \item Any $X$ such that there is an $A$ with $X=\Spec(A)$ is called \emph{affine scheme}.
    \end{enumerate}
  \end{definition}
\end{frame}

\begin{frame}
  \frametitle{The 3 Axioms of SAG}
    \textbf{Axiom (Duality):}
    The map
    \begin{align*}
      A \to R^{\Spec A} \\
      a\mapsto (\varphi\mapsto \varphi(a))
    \end{align*}
    is an equivalence
    for any finitely presented $R$-algebra $A$. \\
  \pause
  \vspace{5mm}
  \textbf{Axiom (Locality):} The ring $R$ is local:
  $1\neq 0$ and for all $x,y:R$ such that $x+y$ is invertible, $x$ is invertible or $y$ is invertible.
  
  \vspace{5mm}
  \textbf{Axiom (Zariski-local choice):}\\
  For every surjective $\pi$, there merely exist local sections $s_i$
  with $f_1, \dots, f_n : A$ such that $(f_1,\dots,f_n)=(1)$.
\end{frame}

\begin{frame}
  \frametitle{\textbf{Synthetic Algebraic Geometry}}
  A proposition $P$ is \textbf{open} if there merely are $r_1,\dots,r_n:R$
  such that
  \[ P \Leftrightarrow r_1\neq 0 \vee \dots \vee r_n\neq 0\]
  \pause
  ($r\neq 0 \Leftrightarrow r\text{ invertible}$) \\
  \pause
  $U\subseteq X$ is open if $x\in U$ is an open proposition for all $x:X$. \\
  \pause
  \textbf{Example:} $\{x:R\mid x\neq 0\}$ and $\{x:R\mid x\neq 1\}$ are open subsets. By locality, they cover $R$. \\
  \vspace{5mm}
  \pause
  By Zariski-local choice every open $U\subseteq \Spec(A)$ is a finite union of subsets of the form ($f:\Spec(A)\to R$)
  \[ D(f) := \{x:\Spec(A) \mid f(x)\neq 0 \} \] \\
  \pause
  A type $X$ is a \textbf{scheme} if
  there exist open $U_1, \dots, U_n \subseteq X$
  such that $X = \bigcup_i U_i$
  and every $U_i$ is an affine scheme. \\
  \vspace{5mm}
  \pause
  \textbf{Morphisms} of schemes are just maps between types.
  
\end{frame}

\begin{frame}
  \frametitle{Properties of Projective Space}
  A \textbf{line bundle} on a type $X$ is a map $\mathcal L:X\to \mathrm{Lines}$.
  Note that $\mathrm{Lines}=K(R^\times,1)$.\\
  \vspace{4mm}
  \pause
  \textbf{Theorem:} $(\bP^n\to \mathrm{Lines}) = \Z\times K(R^\times,1)$. \\
  \pause
  \textbf{Or:} Each line bundle on $\bP^n$ is merely one of the following:
  \begin{enumerate}[(i)]
  \item $\mathcal O(-1):=(([x] : \bP^n) \mapsto \text{ $R\langle x\rangle : \mathrm{Lines}$})$
  \item $\mathcal O(-d):=\mathcal O(-1)^{\otimes d}$ for $d:\N$.
  \item $\mathcal O(d):=(x\mapsto \Hom_{\Mod{R}}(\mathcal O(-d)_x,R))$.
  \end{enumerate}
  \vspace{4mm}
  \pause
  Any open $U\subseteq \bP^n$ is a finite union of more general
  \[ D(f):=\{x:\bP^n\mid f(x)\neq 0 \}\]
  with $f:(x:\bP^n)\to \mathcal O(d)_x$ for some $d:\N$.
\end{frame}

\begin{frame}
  \frametitle{Projective Schemes}
  \pause
  A scheme $X$ is \textbf{projective} if it merely is a closed subset of some $\bP^n$. \\
  \vspace{4mm}
  \textbf{Example and tentative definition:}
  An elliptic curve is a pointed projective scheme $C$
  which is smooth of dimension $1$, connected and $H^1(C,R)=1$.
  \pause
  Assume $2\neq 0,3\neq 0$ in $R$.
  For all $a,b:R$ such that $4a^3+27b^2\neq 0$ we have the elliptic curve
  \[ E_{a,b}:=\{ [x,y,z] :\bP^2\mid y^2z-x^3-axz^2-bz^3=0 \}.\] \\
  \pause
  \textbf{Classically}, over a field $k$, there is a converse - each elliptic curve is of the above form (if $2\neq 0, 3\neq 0$ in $k$).
  % Silverman, p. 59
  \pause
  There is a proof using the Riemann-Roch theorem, which can be derived from Serre-Duality.
  The latter relies on cohomology computations on $\bP^n$.
\end{frame}

\begin{frame}
  \frametitle{Cohomology of Line Bundles on Projective Space}
\begin{theorem}
  \label{calculate-cohomology-twisting-sheaves}
  \begin{enumerate}[(a)]
  \item \label{calculate-cohomology-twisting-sheaves-a}
  For all $n:\N$, $d:\Z$, there are isomorphisms $R[X_0,\dots,X_n]_d\to H^0(\bP^n,\mathcal O(d))$ of $R$-modules, inducing an isomorphism $R[X_0,\dots,X_n]\to \bigoplus_{d:\Z} H^0(\bP^n,\mathcal O(d))$ of graded $R[X_0,\dots,X_n]$-modules.
  \item \label{calculate-cohomology-twisting-sheaves-b}
        $H^n(\bP^n,\mathcal O(-n-1))=R$ is free of rank 1 and $H^n(\bP^n,\mathcal O(d))=0$ for $d>-n-1$.
  \item \label{calculate-cohomology-twisting-sheaves-c}
    The canonical map given by tensoring
    \[
      H^0(\bP^n,\mathcal O(d)) \times H^n(\bP^n,\mathcal O(-d-n-1))\to R
    \]
    is a perfect pairing of finite free $R$-modules for all $d:\Z$.
  \item $H^i(\bP^n,\mathcal O(d))=0$ for $i\in\{1,\dots,n-1\}$ and all $d:\Z$.
  \end{enumerate}
\end{theorem}
\end{frame}

\begin{frame}
  \begin{center}
    Thank you!
  \end{center}
\end{frame}
\end{document}
