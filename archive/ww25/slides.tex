% latexmk -pvc -pdf slides.tex
\documentclass{beamer}

\beamertemplatenavigationsymbolsempty

% literatur
\usepackage[backend=biber,style=alphabetic]{biblatex}

\addbibresource{../util/literature.bib}

\usepackage{../util/zariski}

\usepackage{csquotes}
\usepackage{cancel}
\usepackage{tabularx}
\usepackage{hyperref}
\usepackage{tikz}
\usetikzlibrary{cd,arrows,shapes,calc,through,backgrounds,matrix,trees,decorations.pathmorphing,positioning,automata}
\usepackage{graphicx}
\usepackage{color}

\usepackage{mathpartir}
\newcommand{\yields}{\vdash}
\newcommand{\cbar}{\, | \,}

\newcommand{\nop}[1]{\textcolor{bg}{#1}}


% für tabellen
\usepackage{booktabs}

\title{Synthetic Algebraic Geometry}
\date{December 15, 2025}

\begin{document}

\begin{frame}
  \titlepage
\end{frame}

\begin{frame}
  This is the main part of a small field, containing work of the following people:
  \begin{center}
    \begin{tabularx}{\textwidth}{XXX}
      Peter Arndt & Reid Barton & Ulrik Buchholtz \\
      Ingo Blechschmidt & Felix Cherubini & Dan Christensen \\
      Johan Commelin &Thierry Coquand & Fabian Endres \\
      Freek Geerligs & Jonas Höfer & Tim Lichtnau \\
      Hugo Moeneclaey & David Jaz Myers & Marc Nieper-Wiß\-kir\-chen \\ 
      Armaan Rashid & Matthias Ritter & Christian Sattler \\
      Lukas Stoll & Thomas Thorbjørnsen & David Wärn \\
      Mark Williams & &
    \end{tabularx}
  \end{center}
\end{frame}

\begin{frame}
  The work I will describe now in more detail, was done by:
  \vspace{.5cm}
  \begin{center}
    \begin{tabularx}{\textwidth}{XXX}
      \nop{Peter Arndt} & \nop{Reid Barton} & \nop{Ulrik Buchholtz} \\
      \nop{Ingo Blechschmidt} & Felix Cherubini & Dan Christensen \\
      \nop{Johan Commelin} & Thierry Coquand & \nop{Fabian Endres} \\
      \nop{Freek Geerligs} & \nop{Jonas Höfer} & \nop{Tim Lichtnau} \\
      Hugo Moeneclaey & \nop{David Jaz Myers} & \nop{Marc Nieper-Wißkirchen} \\ 
      \nop{Armaan Rashid} & Matthias Ritter & \nop{Christian Sattler} \\
      \nop{Lukas Stoll} & Thomas Thorbjørnsen & David Wärn \\
      \nop{Mark Williams} & &
    \end{tabularx}
  \end{center}
\end{frame}

\begin{frame}
  A more complete overview is at \url{https://github.com/felixwellen/synthetic-zariski}:
  \includegraphics[width=\textwidth]{repo.png}
\end{frame}

\begin{frame}
  Functor of points approach\dots \\
  \includegraphics[width=\textwidth]{demazure-fx.png}
\end{frame}

\begin{frame}
  \dots allows you to pretend you are working with sets! \\
  \includegraphics[width=\textwidth]{demazure-fx-circled.png}
\end{frame}

\begin{frame}
  Some constructions:\\
  \vspace{.3cm}
    \begin{tabular}{l|l}
      Concept & Functor of Points \\
      \midrule
      $\A^1$ & $A\mapsto A:\Alg{k}\to \mathrm{Set}$ \\
      $X\to Y$ & $\eta_A:X(A)\to Y(A)$ such that \dots \\
      $X\times_Z Y$ & $A\mapsto X(A)\times_{Z(A)} Y(A)$  \\
      $\bP^n$ & $A\mapsto \{M\subseteq A^{n+1}\mid \text{$M$ rank 1 submodule,} \text{ $M\oplus N\cong A^{n+1}$} \}$ \\
      $\Spec(B)$ & $A\mapsto \Hom_{\Alg{k}}(B,A)$ 
    \end{tabular}
    \pause
    \vspace{.7cm}\\
  Synthetically, we can work without the argument $A$:\\
  \vspace{.3cm}
    \begin{tabular}{l|l}
      Concept & Synthetic Algebraic Geometry \\
      \midrule
      $\A^1$ & axiomatically given ring $R$ \\ 
      $X\to Y$ & $f:X\to Y$ (function between sets) \\
      $X\times_Z Y$ & Just $X\times_Z Y$  \\
      $\bP^n$ & $\{L\subseteq R^{n+1}\mid \text{ $L$ rank 1 submodule}\}$ \\
      $\Spec(B)$ & $\Hom_{\Alg{R}}(B,R)$
    \end{tabular}
\end{frame}

\begin{frame}
  Already in the late 60s, Kock, Lawvere and others used that ``from the inside'',
  objects in a sheaf-topos look like sets:
  \vspace{.3cm}\\
  \includegraphics[width=\textwidth]{topos-theoretic-methods.png}
\end{frame}

\begin{frame}

\begin{itemize}
\item  Use of Kripke-Joyal semantics started -- elementary reasoning about sets can be translated to sheaves on a site.
  \pause
\item Anders Kock works synthetically on Grassmannians.
  \pause
\item 1972, Monique Hakim finishes her thesis with Grothendieck, which reasons in a similar spirit, using topos theory.
  \pause
\item 2017, Ingo Blechschmidt uses a more expressive language, ``stack semantics'' to find lot's of interesting properties of the Zariski-topos $\mathrm{Sh}(\Alg{k}_{\mathrm{fp}}^{\mathrm{op}})$.
  Language upgrades and Ingo's work, mentioning Hakim.
\pause
\item We use Homotopy Type Theory now. This means we use a higher sheaf toposes as a model.
\end{itemize}

\end{frame}
\end{document}
