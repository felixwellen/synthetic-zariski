\subsection{Stone spaces are acyclic}

We should prove for $S$ stone and $A$ an overtly discrete abelian group, for all $k>0$ we have:
\[H^n(S,A) = 0\]


\subsection{The unit interval is acyclic}

We should prove for $A$ overtly discrete abelian group, for all $k$ we have that:
\[H^k(\mathbb{I},A) = 0\]

\begin{lemma}\label{delooping-overtly-discrete-local}
For all $A$ overtly discrete abelian group and any $k:\N$, we have that:
\[B_kA\]
is $\shape$-local.
\end{lemma}

\begin{proof}
We proceed by induction on $n$. 
\begin{itemize}
\item For $k=0$ we use $H^0(\mathbb{I},A) = 0$.
\item For $k=n+1$, by induction hypothesis we know that $B_{n+1}A$ is $\shape$-separated, and we merely have a lift by the cohomology hypothesis $H^k(\mathbb{I},A) = 0$.
\end{itemize}
\end{proof}

\begin{corollary}
Let $f:X\to Y$ be a shape equivalence and $A$ be an overtly discrete abelian group. Then for all $k$ we have:
\[f^* : H^k(Y,A) \simeq H^k(X,A)\]
\end{corollary}