Here we start studying the shape modality.

\subsection{Deloopings of overtly discrete abelian groups are local}

\begin{lemma}\label{delooping-overtly-discrete-local}
For all $A$ overtly discrete abelian group and any $k:\N$, we have that:
\[B_kA\]
is $\shape$-local.
\end{lemma}

\begin{proof}
We proceed by induction on $n$. 
\begin{itemize}
\item For $k=0$ we use $H^0(\mathbb{I},A) = A$, for which is is enough to know that maps in $\mathbb{I}\to 2$ are constant. 
\item For $k=n+1$, by induction hypothesis we know that $B_{n+1}A$ is $\shape$-separated, and we merely have a lift by \cref{vanishing-cohomology-interval}.
\end{itemize}
\end{proof}

\begin{corollary}
Let $f:X\to Y$ be a shape equivalence and $A$ be an overtly discrete abelian group. Then for all $k$ we have:
\[f^* : H^k(Y,A) \simeq H^k(X,A)\]
\end{corollary}


\subsection{The shape of the circle is the circle}

\begin{proposition}
We have that:
\[\shape (\R/\Z) = B\Z\]
\end{proposition}

\begin{proof}
The fibers of the map:
\[\R\to \R/\Z\]
are $\Z$-torsors, as for any group quotient. This means we have a fiber sequence:
\[\R \to \R/\Z \to B\Z\]
We check the second map is $\shape$-localisation. We have that $B\Z$ is $\shape$-local by \cref{delooping-overtly-discrete-local}. Since $B\Z$ is connected we just need to prove that $\R$ is $\shape$-contractible to conclude. But $0:\R$ and for any $x:\R$ there is $f:\mathbb{I}\to\R$ such that $f(0)=0$ and $f(1)=x$ so we can conclude.
\end{proof}


\subsection{Cellular cohomology for topological CW complex}

TODO


\subsection{Finite homotopical CW complex are the shape of topological CW complexes}

WIP TODO???

\begin{lemma}\label{finite-homotopy-groups-countably-presented}
Let $X$ be a finite homotopical CW complex, then for any $x:X$ and any $n$ we have that $\pi_n(X,x)$ is a countably presented abelian group.
\end{lemma}

\begin{proof}
TODO maybe find a reference?
\end{proof}

\begin{proposition}
Let $X$ be a finite homotopical CW complex, then $X$ is $\shape$-local.
\end{proposition}

\begin{proof}
We decompose $X$ as its Postnikov tower:
\[\cdots\to \propTrunc{X}_{n+1}\to \propTrunc{X}_n\to\cdots \to \propTrunc{X}_0\]
First we show by induction on $n$ then $\propTrunc{X}_n$ is $\shape$-local:
\begin{itemize}
\item We have that $\propTrunc{X}_0$ is a finite set so it is $\shape$-local.
\item Assuming $\propTrunc{X}_n$ is $\shape$-local, it is enough to prove that the fibers of the map:
\[\propTrunc{X}_{n+1}\to \propTrunc{X}_n\]
are $\shape$-local. But they merely are of the form $B_n\pi_n(X,x)$ for some $x:X$, but $\pi_n(X,x)$ is overtly discrete by \cref{finite-homotopy-groups-countably-presented} so that $B_n\pi_n(X,x)$ is $\shape$-local by \cref{delooping-overtly-discrete-local}.
\end{itemize}
Therefore the limit of the Postnikov tower is $\shape$-local as a limit of $\shape$-local group, and we can conclude as a finite homotopical CW complex $X$ is the limit of its Postnikov tower (why TODO, maybe optimistic?).
\end{proof}

\begin{remark}
By Anel / Barton "Choice axioms and Postnikov completeness" we know that Postnikov completion and hypercompletion agree in our setting because we have countable choice. Do we have hypercompleteness?
\end{remark}

