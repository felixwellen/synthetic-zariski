% !TEX encoding = UTF-8 Unicode
\subsection{Subquotient systems}

\begin{definition}
A subquotient pre-system consists of $X$ a type and $U$ a class of propositions.
\end{definition}

\begin{definition}
For $(X,U)$ a subquotient pre-system, we define:
\[Sub_{X,U} = \sum_{Y:\Type} \exists (P : X\to U).\ Y = \Sigma_XP\]
\[SubQ_{X,U} = \sum_{Y:\Type} \exists (P : X\to U)(R: \Sigma_XP\to \Sigma_XP\to U\ \mathrm{equivalence\ relation}).\ Y = (\Sigma_XP)/R\]
\end{definition}

\begin{definition}
We say a class of types $T$ has local choice if for all $X\in T$ and $P:X\to\Type$ such that:
\[\prod_{x:X}\propTrunc{P(x)}\]
there merely exists $Y\in T$ and a surjection:
\[f:Y\to X\]
such that:
\[\prod_{y:Y}P(f(y))\]
\end{definition}

\begin{proposition}\label{lex-sub-pro}
Assume $(X,U)$ a Subquotient pre-system such that:
\begin{itemize}
\item Identity types in $X$ are in $U$.
\item $U$ is closed by $\sum$ and $\top$.
\item $\propTrunc{X}$ and $X\times X = X$.
\item $Sub_{X,U}$ has local choice.
\end{itemize}
Then we $SubQ_{X,U}$ has the following:
\begin{itemize}
\item Stability under finite limits.
\item Stability by quotient by equivalence relation with value in $U$.
\item Local choice.
\end{itemize}
\end{proposition}

\begin{proof}
\begin{itemize}
\item First we check that $SubQ_{X,U}$ has local choice. Since we assume that $Sub_{X.U}$ has local choice and that any type in $SubQ_{X,U}$ is covered by a type in $Sub_{X,U}$ by definition, it is enough to check that $Sub_{X,U}\subset SubQ_{X,U}$ to conclude. But given $S = \Sigma_XP$ in $Sub_{X,U}$ we have that:
\[\Sigma_XP = (\Sigma_XP)/L\]
where:
\[L((x,_),(y,_))= (x=_Xy)\]
and since $x=_Xy$ is assumed to be in $U$ we conclude.

\item Stability by quotient by equivalence relation with value in $U$ is clear.

\item Now we want to check stability under finite limits.

First we check that $U\subset SubQ_{X,U}$. Indeed assume $P\in U$, then with $L$ the trivial relation we have:
\[(X\times P) / L = \propTrunc{X\times P} = P\]
as $\propTrunc{X} = 1$ so that since $\top\in U$ we conclude $P\in SubQ_{X,U}$.

This means that $SubQ_{X,U}$ is stable by identity type, and that $1\in SubQ_{X,U}$.

All that is left is to check stability under $\Sigma$. Assume $S: SubQ_{X,U}$ and $T:S\to SubQ_{X,U}$. Through the fact that $S$ is covered by a type in $Sub_{X,U}$ and local choice for $Sub_{X,U}$ we merely get $S':Sub_{X,U}$, say $S'=\Sigma_XP$ and a surjective map:
\[f:S'\to S\]
such that for all $x:\sum_XP$ we have:
\[T(f(x)) = (\Sigma_XP_x)/R_x\]
so we get a surjective map:
\[\sum_{x:X}\sum_{P(x)}(\Sigma_X P_x)/R_x  \to \sum_ST\]
Then the identity types in $\sum_ST$ are in $U$ as $U$ is stable by $\Sigma$, so it is enough to check that:
\[\sum_{x:X}\sum_{P(x)}(\Sigma_X P_x)/R_x\]
is in $SubQ_{X,U}$ to conclude, as we can then apply the previous bullet-point. But this type is equivalent to:
\[\left(\sum_{(x,x'):X\times X}\Sigma_{P(x)}P_x(x')\right)/ L\]
where:
\[L((x,x'),(y,y')) =\sum_{x=y} R_y(x',y') \]
which is in $SubQ_{X,U}$ as $U$ is stable by $\Sigma$, $x=y$ in in $U$ and $X\times X = X$.
\end{itemize}
\end{proof}

\begin{proposition}\label{coproducts-sub-quo}
Assume $(X,U)$ a subquotient pre-system such that $\bot\in U$ and $X+X = X$. Then $SubQ_{X,U}$ is stable by finite coproducts.
\end{proposition}

\begin{proof}
We have that:
\[\bot = (X\times \bot) / L\]
where $L$ is the unique such equivalence relation. Since $\bot\in U$ we conclude that $\bot\in SubQ_{X,U}$.

Given $S$ and $S'$ in $SubQ_{X,U}$, say:
\[S = (\sigma_XP)/R\]
\[S' = (\sigma_XP')/R'\]
Then we have that:
\[S+S' = \left(\sum_{X+X}[P,P']\right) L\]
where:
\[[P,P'](l(x)) = P(x)\]
\[[P,P'](r(x)) = P'(x)\]
and:
\[L(l(x),l(y)) = R(x,y)\]
\[L(l(x),r(y)) = \bot\]
\[L(r(x),l(y)) = \bot\]
\[L(r(x),r(y)) = R'(x,y)\]
Since $\bot\in U$ and $X+X=X$ we conclude that $S+S'$ is in $SubQ_{X,U}$.
\end{proof}

\begin{proposition}\label{prop-sub-quo}
Assume $(X,U)$ a subquotient pre-system such that $\top\in U$ and for all $S\in Sub_{X,U}$ we have that $\propTrunc{S}\in U$. Then any proposition in $SubQ_{X,U}$ is in $U$. 
\end{proposition}

\begin{proof}
If we have a proposition $S$ in $SubQ_{X,U}$, say:
\[S = (\Sigma_XP)/R\]
then:
\[S = \propTrunc{S} = \propTrunc{\Sigma_XP}\]
and we can conclude.
\end{proof}

\begin{definition}
A subquotient system is a subquotient pre-system obeying the hypothesis of \cref{lex-sub-pro}, \cref{coproducts-sub-quo} and \cref{prop-sub-quo}.
\end{definition}

We just pack all this up in one theorem:

\begin{theorem}\label{stabitity-sub-quo}
Let $(X,U)$ be a subquotient system, then $SubQ_{X,U}$ has the following:
\begin{itemize}
\item Stability under finite limits.
\item Stability under finite coproducts.
\item Stability under quotient by equivalence relations.
\item Local choice.
\end{itemize}
\end{theorem}

We have two main examples in mind.

\begin{example}
The subquotient pre-system $St = (2^\N,\mathrm{Closed})$ is a quotient system. We have that $Sub_{St}$ is the type of stone spaces, and $CHaus = SubQ_{St}$ the type of compact Haussdorf spaces.

Closed propositions are stable by $\Sigma$. TODO 

We also need that for any stone space $S$ we have that $\propTrunc{S}$ is a closed proposition. TODO
\end{example}

\begin{example}
The subquotient pre-system $Od = (\N,\mathrm{Open})$ is a quotient system. We have that $ODisc = SubQ_{Od}$ the type of so-called overtly discrete types.

A key observation is that open propositions are stable by countable disjunctions.

This means open propositions are stable by $\sum$ because we can assume:
\[P = \Sigma_{n:\N} A(n)\]
with $A(n)$ decidable and:
\[Q:P \to \mathrm{Open}\]
Then we have that:
\[\Sigma_PQ = \exists(n:\N).\ \Sigma_{A(n)} Q(n)\]
which is open as $\Sigma_{A(n)} Q(n)$ is open for all $n$, as $A(n)$ is decidable.

Types in $Sub_{Od}$ even have full choice because both $\N$ and decidable propositions have full choice.
\end{example}

So both $ODisc$ and $CHaus$ enjoys the conclusion of \cref{stabitity-sub-quo}.


\subsection{Tychonov}

\begin{proposition}\label{tychonov}
Assume $(X,U)$ and $(Y,C)$ two subquotient system such that:
\begin{itemize}
\item $S\to Y$ is in $SubQ_{Y,C}$ for all $S:Sub_{X,U}$.
\item If $P\in U$ and $Q:P\to C$ then $\prod_{p:P}Q(p) \in C$.
\item If $Q:X\to C$ then $\prod_{x:X}Q(x) \in C$.
\end{itemize}
Then we have the following:
\begin{itemize}
\item If $S:SubQ_{X,U}$ and $T:S\to SubQ_{Y,C}$, then:
\[\prod_{s:S}T(s) \in SubQ_{Y,C}\]
\end{itemize}
\end{proposition}

\begin{proof}
Note that for $S':Sub_{X,U}$ and $Q:S'\to C$ we have that:
\[\prod_{S'}Q\]
is in $C$.

Now we use local choice to get $S':Sub_{X,U}$ with a surjective map:
\[f:S'\to S\]
such that for all $s:S'$ we have:
\[T(f(s)) = (\Sigma_YU_s)/R_s\]

Then the map:
\[\prod_ST \to \prod_{s:S'}(\Sigma_YU_s)/R_s\]
is an embedding, its fiber over $\alpha$ is:
\[\prod_{s,t:S'} \prod_{f(s) =_S f(t)} \alpha(s) = \alpha(t)\]
which is in $C$ by the hypothesis. Therefore it is enough to prove that:
\[\prod_{s:S'}(\Sigma_YP_s)/R_s\]
is in $SubQ_{Y,C}$. 

But this type is the quotient of:
\[\prod_{s:S'}(\Sigma_YP_s)\]
by:
\[L(\alpha,\beta) = \prod_{s:S'} R_s(\alpha(s),\beta(s))\]
which is in $C$, therefore it is enough to to prove that:
\[\prod_{s:S'}(\Sigma_YP_s)\]
is in $SubQ_{Y,C}$.

But this type is equivalent to:
\[\sum_{f:S'\to Y} \prod_{s:S'}P_s(f(s))\]
Since $\prod_{s:S'}P_s(f(s))$ is in $C$, it is enough to prove that:
\[S'\to Y\]
is in $SubQ_{Y,C}$. But this is one of the hypothesis.
\end{proof}

\begin{definition}
Two subquotient systems $A,B$ are called dual if both $(A,B)$ and $(B,A)$ satisfy the hypothesis of \cref{tychonov}.
\end{definition}

\begin{example}
We have that $St = (2^\N,\mathrm{Closed})$ and $Od = (\N,\mathrm{Open})$ are dual quotient systems.

\begin{itemize}
\item We need that if $P$ open and $Q:P\to \mathrm{Closed}$, then $\Sigma_PQ$ is closed. Assume $Q=\Sigma_{n:\N}A(n)$ with $A(n)$ decidable, since open propositions have choice we can assume for $n:\N$ such that $A(n)$ that $Q(n) = \forall_{k:\N} B_n(k)$ with $B_n(k)$ decidable. Then:
\[\Sigma_PQ = \prod_{n,k:\N} \prod_{A(n)} B_n(k) \]
which is indeed closed.

It is clear that closed propositions are closed by countable products.

$\Sigma_\N P\to 2^\N$ is compact Hausdorff? Yes indeed, it is even Stone because it is equivalent to:
\[\prod_{k,n:\N} 2^{P(n)}\]
and $2^{P(n)}$ is Stone as $P(n)$ is open, indeed:
\[(\Sigma_\N A)\to 2\] 
for $A$ decidable is a countable product of Stone space, as $2^A$ is Stone for $A$ decidable.

\item First we check that given $S$ Stone, we have that:
\[S\to \N\]
is overtly discrete. Indeed identity types in $S\to \N$ are closed and there is a surjection from fundamental systems of idempotent in $2^S$ to $S\to \N$, so it is enough to prove that the type fundamental systems of idempotent in a c.p. algebra is overtly discrete. To have this it is enough to prove that countably presented algebra are overtly discrete. TODO
\end{itemize}
\end{example}

When applying \cref{tychonov} to $ODisc$ and $CHaus$ we get Tychonov theorem and its dual.


\subsection{Factorisation}

\begin{proposition}\label{factorisation-subquotient}
Assume given dual subquotient systems $(X,C)$ and $(Y,U)$. such that
such that:
\begin{itemize}
\item Given $S:Sub_{X,U}$ any map:
\[S\to \N\]
merely factors through a finite type.
\item Given $P\in C$ and $Q\in U$ we have that:
\[(P\to Q) \to (\neg P \lor Q)\]
\end{itemize}
Then for $S:SubQ_{X,C}$ and $T:SubQ_{Y,U}$, any map:
\[S\to T\]
merely factors through a finite type.
\end{proposition}

\begin{proof}
We proceed in three steps:
\begin{enumerate}[(i)]
\item First we show the factorisation of any map:
\[S\to T\]
for $S:Sub_{X,C}$ and $T:Sub_{Y,U}$. Indeed we then we merely have $Q:Y\to U$ such that:
\[T = \Sigma_YQ\]
so the map:
\[S\to \Sigma_YQ\]
induces a map:
\[S\to Y\]
which we factor as:
\[S\to \mathrm{Fin}(n) \to Y\]
This gives a factorisation of the starting map as follows:
\[S\to \Sigma_{\mathrm{Fin}(n)} Q \to T\]
Then for any $k:\mathrm{Fin}(n)$ we define:
\[S_k \subset S\]
\[s\in S_k := (f(s) = k)\]
which is decidable and therefore in $C$, so that $\propTrunc{S_k}$ is in $C$. We have that:
\[\propTrunc{S_k} \to Q(k)\]
therefore we have
\[\neg S_k \lor Q(k)\]
Now we just need to remove the $k$ such that $\neg S_k$ to get a factorisation:
\[S\to \mathrm{Fin}(l) \to \Sigma_{\mathrm{Fin}(n)}Q\]
\item We show how to factor any map:
\[S\to \mathrm{Fin}(n)/R\]
where $S:SubQ_{X,U}$ and $R$ in an equivalence relation in $U$. We do this by induction on $n$, if $n=0,1$ it is clear. Using local choice we get:
\begin{center}
\begin{tikzcd}
S'\ar[d,"p"]\ar[r,"f"] & \mathrm{Fin}(n)\ar[d] \\
S\ar[r,"g"] & \mathrm{Fin}(n)/R
\end{tikzcd}
\end{center}
Then for all $k:\mathrm{Fin}(n)$ we define:
\[S'_k \subset S'\]
\[s'\in S'_k := (f(s) = k)\]
which is a decidable cover of $S'$. Then we use:
\[S_k = p(S'_k)\]
which is a cover of $S$ such that:
\[(S_k\cap S_l) \to R(k,l)\]
since $g$ restricted to an $S_i$ has value $[i]$. But since $\propTrunc{S_k\cap S_l}$ is in $C$, this means that:
\[\neg(S_k\cap S_l) \lor R(k,l)\]
If for all $k\not=l$ we have $\neg(S_k\cap S_l)$ then we can conclude by factoring through $\mathrm{Fin}(n)$, otherwise we have $R(k,l)$ for some $k\not=l$ and we have that $\mathrm{Fin}(n)/R$ is equivalent to $\mathrm{Fin}(n-1)/R$ where we have removed $l$. We conclude by induction on $n$.
\item  Now we show how to factor any map:
\[S\to T\]
for $S:SubQ_{X,C}$ and $T:SubQ_{Y,U}$. There is $Q:Y\to U$ and $R$ an $U$-valued equivalence relation on $\Sigma_YQ$ such that:
\[T = (\Sigma_YQ)/R\]
Using local choice we have that:
\begin{center}
\begin{tikzcd}
S'\ar[d]\ar[r] & \Sigma_YQ\ar[d] \\
S\ar[r] & (\Sigma_YQ)/R
\end{tikzcd}
\end{center}
and by factoring the top map using (i), we can get a factorisation:
\[S\to \mathrm{Fin}(n)/R\to (\Sigma_YQ)/R\]
and we conclude using (ii).
\end{enumerate}
\end{proof}

\begin{definition}
Dual subquotient systems $(X,C)$ and $(Y,U)$ obeying the hypothesis of \cref{factorisation-subquotient} are called factorial.
\end{definition}

\begin{remark}
We have that $St = (2^\N,\mathrm{Closed})$ and $Od = (\N,\mathrm{Open})$ are factorial. We need to check the following:
\begin{itemize}
\item Given a stone space $S$, any map $S\to\N$ merely factors through a finite type. This is known.
\item Given $P$ closed and $Q$ open, we have that:
\[(P\to Q) = \neg P \lor Q\]
We just note that as $\neg P \lor Q$ is open it is $\neg\neg$-stable, and the rest is just intuitionistic logic.
\end{itemize}
From this we know that for $S$ compact Hausdorff and $T$ overtly discrete, any map:
\[S\to T\]
merely factors through a finite type. 
\end{remark}

\subsection{Scott continuity}



