\rednote{TODO, is $\Boole$ closed under countable limits? 
  It has finite colimits, as it has pushouts and initial object.
  It should also have sequential colimits (TODO). 
  Is a countable coproduct the sequential colimit of it's initial finite coproducts? 
}
\begin{lemma}\label{BoolePushouts}
  Countably presented Boolean algebras are closed under pushout. 
\end{lemma} 
\begin{proof}
  Let $A,B,C:\Boole$, and suppose $f:A\to B, g:A \to C$ are Boolean morphisms. 
  Let $G_A, G_B,G_C$ be the underlying countable sets of generators for $B,C$ and 
  let $R_A,R_B,R_C$ be the underlying countable sets of relations. 
  Consider $P$ the Boolean algebra generated by $G_B\sqcup G_C$ under the relations 
  $R_B\cup R_C \cup F$ where $F$ is the set of expressions $f(a)-g(a), a\in G_A$.
  
  Note that as the generators of $B$ are included in those of $P$, 
  and all relations of $B$ are included in those of $P$, there is a map $h:B\to P$. 
  Similarly there is a map $i:C\to P$. 
  We now claim that the following is a pushout square:
  \begin{equation}\begin{tikzcd}
    A \arrow[r,"f"] \arrow[d,"g"] & B \arrow[d,"h"]\\
    C \arrow[r,"i"] & P
  \end{tikzcd}\end{equation}  
  Suppose $\beta:B \to X, \gamma:C\to X$ are such that $\beta\circ f = \gamma \circ h$. 
  $\beta,\gamma$ then induce maps on the generators of $P$. 
  These maps respect $F$ as $\beta\circ f=\gamma\circ h$, and they must respect $R_B,R_C$ as they are maps out of $B,C$. 
  Therefore, $\beta,\gamma$ induce a map $e:P\to X$, such that 
  $e(b) = \beta(b)$ for $b:G_B$ and $e(c)=\gamma(c)$ for $c:G_C$. 
  Furthermore, any map $P\to X$ with this property must agree with $e$ on all the generators of $P$, 
  and therefore equal $e$. Thus $e$ is the unique extension $P\to X$. 
  Thus $P$ the above square is indeed a pushout. 
\end{proof}

For some proofs in this paper, 
\rednote{(right now the counter's at two)}
we'd like a very concrete description of the fiber of a map of Stone spaces. 
The following construction turns out to be particularly useful. 
\begin{lemma}\label{FiberConstruction}
  Let $A,B:\Boole$, let $G$ be an explicit countable set of generators for $A$, and let 
  and maps $f:A \to B, x:A\to 2$. 
  Then we can construct a countable set $R\subseteq B$ such that 
  the pushout of $f$ and $x$ is given by $B/R$. 
\end{lemma}  
\begin{proof}
We consider the following pullback in the category of Stone spaces:
  \begin{equation}\begin{tikzcd}
    \sum\limits_{y:Sp(B)} y\circ f = x \arrow[d] \arrow[r] \arrow["\lrcorner"{pos=0.125}, phantom, dr] 
    & \top \arrow[d,"x"]\\
    Sp(B) \arrow[r,"(\cdot) \circ f"] & Sp(A)
  \end{tikzcd}  \end{equation}
Dual to this square, we have the following pushout in the category of Boolean algebras,
where $Sp(P) \simeq  (\sum\limits_{y:Sp(B)} y \circ f = x)$:
  \begin{equation}\begin{tikzcd}
    A \arrow[d,"x"'] \arrow[r,hook,"f"] \arrow[rd,phantom,"\ulcorner"{pos=0.125}] & B\arrow[d]\\
    2 \arrow[r] & P
  \end{tikzcd}\end{equation} 
  Following \Cref{BoolePushouts}, 
  the pushout $P$ is given by $B/R$ with $R$ the relations $f(a) -x(a)$ 
  where $a$ ranges over the generators of $A$.
  Note that $x(a) \in \{0,1\}$. 
  If $x(a)=0$, then $f(a)-x(a) = f(a)$, 
  and if $x(a) = 1$, then $f(a) -x(a) = \neg f(a) = f(\neg a)$. 
  So we can define the subset $G'\subseteq A$ given by 
  \begin{equation}
    G' = \{a | a\in G_A, x(a) = 0\} \cup \{\neg a | a \in G_A, x(a) = 1\}
  \end{equation} 
  $G'$ is in bijection with $G$, hence countable. 
  Furthermore, $x(g) = 0$ for all $g\in G_A'$. 
  And $R = f(G')$.
\end{proof}





%\begin{lemma}\label{BooleCoEqualizers}
%  Countably presented Boolean algebras are closed under coequalizers.
%\end{lemma}
%\begin{proof}
%  Let $f,g:A\to B$ be Boolean morphisms.
%  Define $C = B/R$, where $R$ is given by the relations $fa-ga,~a\in G_A$, for $G_A$ the set of generators of $A$.
%  Suppose that we have a map $x:B\to D$ with $xf = gf$. Then $x$ respects $R$, and thus defines a map $y:C \to D$. 
%  Furthermore, any map $C\to D$ extending $x$ agrees with $y$ on the generators of $C$, 
%  and is thus equal to $y$. Therefore $C$ is the coequalizer of $f,g$. 
%\end{proof}


%%
%%\begin{corollary}\label{CoCompletenessBoole}
%%  The category of countably presented Boolean algebras contains all finite colimits. 
%%\end{corollary}
%%\begin{proof}
%%  Recall that $\Boole$ has an initial object given by $2$. 
%%  By \Cref{BoolePushouts}, 
%%%  it is therefore closed under coproducts. 
%%%  By \Cref{BooleCoEqualizers}, 
%%  it follows that $\Boole$ contains all finite colimits. 
%%\end{proof}

