% latexmk -pdf -pvc main.tex
\documentclass{../util/zariski-small}


\title{Synthetic Stone Duality}

\begin{document}

\author{Felix Cherubini, Thierry Coquand and Freek Geerligs}

\maketitle

\begin{abstract}
In synthetic algebraic geometry (SAG) \cite{draft}, we study finitely presented algebras over a commutative ring. 
In this work, we study countably presented Boolean algebras instead. 
Where the finitely presented algebras over a commutative ring induce a Zariski topos, 
%the opposite category of these 
the countably presented Boolean algebras induce the topos of light condensed sets \cite{Scholze}. 
\cite{draft} proposes an axiomatization of the Zariski topos in univalent homotopy type theory \cite{hott}. 
In this work, we propose similar axioms, which we expect to be modelled by light condensed sets. 
% Furthermore, spectra of countably presented Boolean algebras correspond to quotients of Cantor space
% which is cool because reasons
\end{abstract} 

\rednote{The following is a collection of notes on work in progress.}
  

\section*{Introduction}
This draft is empty so far.
%

\section{Omniscience principles}
\begin{lemma}\label{LemDecidableSubsetsDeMorgan}
  For $(A_n)$ a family of decidable subsets, we have
    $
    (\bigcup_{n:\N} A_n)^C
    =
    \bigcap_{n:\N} (A_n^C)
    $ 
    and 
    $
    \bigcup_{n:\N} (A_n^C)
    =  
    (\bigcap_{n:\N} A_n)^C
    $
\end{lemma}

\begin{proof}
\begin{itemize}
  \item 
    Let 
    $x\notin \bigcup_{n:\N} A_n$. 
    Then for every $n:\N$, we cannot have $x\in A_n$
    and thus $x\in A_n^C$ by decidability of $A_n$. 
    Thus $x\in \bigcap_{n:\N} (A_n^C)$. 
    Therefore
    $$
    (\bigcup_{n:\N} A_n)^C
    \subseteq 
    \bigcap_{n:\N} (A_n^C).
    $$ 
  \item 
    Suppose that for every $n:\N$, we have $x\notin A_n$. 
    There does not exist an $n:\N$ with $x\in A_n$. Thus
    $$
    \bigcap_{n:\N} (A_n^C)
    \subseteq 
    (\bigcup_{n:\N} A_n)^C
    $$ 
  \item 
    Suppose there exists some $n$ with $x\in A_n^C$. Then 
    it cannot be the case that $x\in A_m$ for all $m:\N$.
    Thus
    $$
    \bigcup_{n:\N} (A_n^C)
    \subseteq 
    (\bigcap_{n:\N} A_n)^C
    $$ 
  \item 
    Suppose that $x\in (\bigcap_{n:\N} A_n)^C$. 
    Then define the binary sequence $\alpha$ by $\alpha(i) =1$ iff $i$ is the first index such that 
    $x\notin A_i$. This is well-defined as $A_n$ is decidable for all $n:\N$. 
    If $\alpha(i) = 0$ for all $i$, then $x\in A_i$ for all $i$. 
%    and it is the case that $x\in A_n$ for all $n:\N$. 
    Thus under our assumption $x\in (\bigcap_{n:\N} A_n)^C$, 
    we cannot have that $\alpha(i) = 0$ always. 
    By Markov, there then exists an $i$ such that $\alpha(i) = 1$. 
    Thus $x\notin A_i$ for some $i$. We conclude that. 
    $$
    (\bigcap_{n:\N} A_n)^C
    \subseteq
    \bigcup_{n:\N} (A_n^C)
    $$ 
\end{itemize}
\end{proof} 
Note that we only needed decidability for the first and last bullet point, 
and only the last bullet point used countability (and of course Markov's principle). 



\section{Topology}
Analogous to synthetic algebraic geometry,
we use pointwise and local definitions of open subsets,
which agree if we assume a corresponding choice axiom.

\subsection{Compact Hausdorff}
\subsection{Compact Hausdorff types}
\begin{definition}
  A type $X$ is called compact Hausdorff iff there exists some $S:\Stone$ and some 
  equivalence relation $\sim:S \times S \to \Closed$ such that $X \simeq S / \sim$. 
  We denote $\CHaus$ for the type of compact Hausdorff types. 
\end{definition} 

\begin{lemma}
Let $A\subseteq X$ be a subtype of a compact Hausdorff space. 
Let $S, \sim$ be any presentation of $X$. 
Then $A$ is closed iff it is the image of a closed subtype of $S$ under the quotient map. 
\end{lemma}
\begin{proof}
  If $A$ is closed, then it's pre-image under any map is also closed. 
  In particular for $q:S\to X$ the quotient map, $q^{-1}(A)$ is closed. 
  As $q$ is surjective, we have $q(q^{-1}(A)) = A$,
  hence $A$ is the image of a closed subtype of $S$. 
  Conversely, let $B\subseteq S$ be closed. 
%  Then for any $s:S$, the subtype $\{t:S| B(s) \wedge s \sim t\} \subseteq S$ is closed. 
%  Hence by 
  Define $A\subseteq S$ by 
  $$A(s) := \exists_{s:S} (B(t) \wedge s \sim t).$$
  As $B, \sim$ are closed, by \Cref{ClosedCountableConjunction} and \Cref{InhabitedClosedSubSpaceClosed}, 
  we have that $A$  is closed. 
  Also $A$ respects $\sim$, hence induces a map $A': X\to \Closed$.
  Furthermore, $A(q(s))$ iff $q(s)$ is in the image of $B$. 
  Therefore $A'(x)$ iff $x$ is in the image of $B$. 
\end{proof}
\begin{corollary}
  For $X:\CHaus$ a subtype $A\subseteq X$ is closed iff it is the image of 
  a map $T\to X$ for some $T:\Stone$. 
\end{corollary}
\begin{proof}
  Directly from the above and \Cref{StoneClosedSubsets}.
\end{proof}

\begin{corollary}\label{InhabitedClosedSubSpaceClosed}
  For $X:\CHaus$ and $A\subseteq X$ closed, we have 
  $\exists_{x:X} A(x)$ is closed. 
\end{corollary}
\begin{proof}
  Let $A$ be the image of a map map $T\to X$ for $T:\Stone$. 
  Then $\exists_{x:X} A(x) \leftrightarrow ||T||$, which is closed by \Cref{TruncationStoneClosed}
\end{proof}


\begin{corollary}\label{ClosedDependentSums}
  Closed propositions are closed under dependent sums. 
\end{corollary}
\begin{proof}
  Let $P:\Closed$ and $Q:P \to \Closed$. 
  Then $\Sigma_{p:P} Q(p) \leftrightarrow \exists_{p:P} Q(p)$.
  As $P$ is Stone by \Cref{PropositionsClosedIffStone}, it is also compact Hausdorff, thus
  \Cref{InhabitedClosedSubSpaceClosed} gives that $\Sigma_{p:P} Q(p)$ is closed. 
\end{proof}
\begin{remark}
  Analogously to \Cref{OpenTransitive} and \Cref{OpenDominance}, it follows that 
  closedness is transitive and $\Closed$ forms a dominance. 
\end{remark}
\begin{corollary}\label{AllOpenSubspaceOpen}
  For $U\subseteq X$ an open subset of a compact Hausdorff space, we have 
  $\forall_{x:X} U(x)$ open. 
\end{corollary}
\begin{proof}
  As $U$ is open, $\neg U$ is closed. 
  Hence $\exists_{x:X} \neg U(x)$ is closed. 
  Therefore, $\neg (\exists_{x:X} \neg U(x))$ is open. 
  Furthermore, it is equivalent to $\forall_{x:X} \neg \neg U(x)$, 
  which is equivalent to $\forall_{x:X} U(x)$ by \Cref{rmkOpenClosedNegation}.
\end{proof}

\begin{lemma}
  Let $X:\Chaus$ be presented by $S/\sim$. 
  Then $2^X$ is an open sub-Boolean algebra of $2^S$. 
\end{lemma}
Note that we do not claim $2^X$ is countable presented, 
but by \Cref{OpenSubsetEnumerableAreEnumerable}, it will be enumerable. 
\begin{proof}
  Denote $q:S \twoheadrightarrow X$ for the quotient map. 
  This induces an injection of Boolean algebras $2^X \hookrightarrow 2^S$.
  Note that $a:S\to 2$ lies in $2^X$ iff for all $s,t:S$, we have $a(s) = a(t)$ whenever $s\sim t$.
  Note that $a(s) = a(t)$ is decidable and $s\sim t$ is open, hence 
  $(s\sim t) \to (a(s) = a(t))$ is open (\Cref{ImplicationOpenClosed})
  By \Cref{AllOpenSubspaceOpen}, we conclude that 
  $\forall_{s:S} \forall_{t:S} ((s\sim t) \to (a(s) = a(t)))$ is open. 
  Hence $2^S$ is an open subobject of $2^X$. 
\end{proof}


\section{Analysis}

\subsection{Convergence}
\input{Convergence/convergenceClosed}
\paragraph{Extensional convergence }  
\begin{definition}
  Let $B_\infty$ be the Boolean algebra on countably many generators $(p_n)_{n:\N}$ 
  over the equivalence $p_n\wedge p_m = 0 $ whenever $n \neq m$. 
\end{definition} 
\begin{definition}
  We denote $\Noo$ be the spectrum of $B_\infty$. 
\end{definition} 
\begin{lemma}
  $B_\infty$ is isomorphic with the Boolean algebra of 
  finite/cofinite subsets of $\N$. 
\end{lemma}
\begin{proof}
  To go from $B_\infty$ to subsets of $\N$, we send
  the generators $p_n$ to the singleton $\{n\}$, which are clearly finite. 
  We call the induced Boolean operation $f$. 

  To go from finite/cofinite subsets of $\N$ to $B_\infty$,
  a finite subset $I$ of $\N$ is sent to the element 
  $\bigvee_{i \in I} p_i$, and a cofinite subset $J$ is sent to the element 
  $\bigwedge_{i \in J^C} \neg p_i$.  
  We call this function $g$ and we need to show that $g$ is a Boolean morphism. 
  \begin{itemize}
    \item By deMorgan's laws, $g$ preserves $\neg$. 
    \item To see that $g$ respects $\vee$, we need to check three cases
      \begin{itemize}
        \item If both $I,J$ are finite, then 
        \begin{equation} 
          g(I \cup J) = \bigvee_{i\in I \cup J} p_i= \bigvee_{i\in I} p_i \vee \bigvee_{j\in J} p_j 
        \end{equation}
      \item If both $I,J$ are cofinite, we have
        \begin{equation}
          g(I) \vee g(J) = 
          \bigwedge_{i \in I^C} \neg p_i \vee 
          \bigwedge_{j \in J^C} \neg p_j 
          = 
          \bigwedge_{i\in I^C} 
          \bigwedge_{j \in J^C}(\neg p_i \vee  \neg p_j) 
        \end{equation}
        Now note that $\neg p_i \vee \neg p_j = \neg ( p_i \wedge p_j)$, which 
        is $1$ if $i \neq j$ and $p_i$ if $i =j$. 
        We can leave $1$ out of the meet, and we are left with the intersection of $I^C$ and $J^C$, so
        \begin{equation}
          g(I) \vee g(J) = 
          \bigwedge_{i \in (I^C \cap J^C)} \neg p_i
          = 
          \bigwedge_{i \in (I \cup J)^C} \neg p_i 
        \end{equation} 
        as the union of $I$ and $J$ is also cofinite, this equals 
          $ g( I \cup J)$. 
        \item If $I$ is finite and $J$ cofinite, we have 
        \begin{equation}
        g(I) \vee g(J) = (\bigvee_{i\in I} p_i) \vee (\bigwedge_{j \in J^C} \neg p_j)
        = \bigwedge_{j \in J^C} (\bigvee_{i \in I}( p_i \vee \neg p_j))
        \end{equation}
        If $i\neq j$, then $p_i\wedge p_j = 0$, hence $\neg p_j \geq p_i$ and $p_i \vee \neg p_j  = \neg p_j$
        If $i = j$, then $p_i \vee \neg p_j = 1$.
%        \begin{equation}
%        g(I) \vee g(J) = 
%        = \bigwedge_{j \in J^C} (\bigvee_{i \in I-J}( p_i \vee \neg p_j))
%        \end{equation}
%
%        \item If $I$ is cofinite and $J$ is finite, we have that $I \cup J$ is cofinite.
%        Thus 
%        \begin{equation}
%          g(I \cup J) = \bigwedge_{i \in (I \cup J)^C} \neg p_i
%        \end{equation}
%
      \end{itemize}
    \item The case for $\wedge$ is completely dual to the case for $\vee$. 
  \end{itemize}
We conclude that $g$ is a Boolean morphism. 
Furthermore, $g$ and $f$ are clearly inverses, thus the Boolean algebras are isomorphic. 
\end{proof}

  \begin{lemma}\label{lemBinftyNormalForm}
  Any element of $B_\infty$ can be written as 
  either $\bigvee_{i\in I} p_i$  or
  as $\bigwedge_{j\in J} \neg p_j$ 
  for finite $I,J\subseteq \N$. 
\end{lemma}
\begin{proof}
  Remark that whenever $n \neq m$, we have that 
  $\neg p_n \geq p_m$ as $p_m \wedge p_n = 0$. 
\end{proof}
There is canonical embedding $\N \hookrightarrow \Noo$, 
wich sends $n$ to the unique function $\chi_{n}$ sending $p_n$ to $1$. 
We denote $\infty \in \Noo$ for the function which is constantly $0$. 
By \Cref{PropMarkov}, if an element is not $\infty$, 
it comes from the embedding $\N \hookrightarrow \Noo$. 
\begin{lemma}\label{LemmaOpensContainingInfty}
  Let $U$ be an open subset of $\Noo$ containing $\infty$.
  Then there merely exists an $N:\N$ such that whenever $n\geq N$, 
  $\chi_n\in U$ as well. 
\end{lemma}
\begin{proof}
  It is sufficient to prove the lemma for $U$ a basic open. 
  Assume $b : B_\infty $ is such that 
  $U = \{ \phi: B_\infty \to 2| \phi(b) = 1\}$.
  Assume furthermore that $\infty \in U$.
%  so $U$ contains the function sending every $p_i$ to $0$. 
  by \Cref{lemBinftyNormalForm}, $b$ can have two forms.
  If $b = \vee_{i\in I} p_i$, then as $\infty(b) = 0$, 
  we must have $I = \emptyset$, and thus $b = 0$, 
  which means $U$ is empty, contradicting $\infty\in U$. 
  Therefore, 
  $b$ must be of the form $\wedge_{j \in J} \neg p_j$. 
  Note that for $N = \max J + 1$, whenever $n>J$, 
  $\chi_n$  sends $b$ to $1$. 
  Thus $\chi_n \in U$ as well, and we are done. 
\end{proof}

\begin{definition}
  Let $\alpha$ be a sequence in $X$, we say that $\alpha$
  is convergent iff there exists an extension. 
  \begin{equation}\begin{tikzcd}
    \N \arrow[r, "\alpha"] \arrow[d,hook]  & X \\
    \Noo \arrow[ru,dashed]
  \end{tikzcd}\end{equation}  
\end{definition}  



\begin{proposition}
  A sequence is convergent iff it has a limit
\end{proposition}
\begin{proof}
  Let $\alpha$ be convergent, with extension $\overline \alpha$.
  we claim that $\overline \alpha(\infty)$ is a limit of $\alpha$.
  Let $U \subseteq X$ be an open containing $x$. 
  As $\overline\alpha^{-1}(U)$ is an open subset of $\Noo$ containing $\infty$,
  \Cref{LemmaOpensContainingInfty} tells us there exists some $N$ such that $[N,\infty]\subseteq \overline \alpha^{-1}(U)$. 
  Thus there exists an $N$ such that for $n\geq N$, we have $\alpha(n) \in U$, as required. 

  Conversely, suppose $\alpha$ has limit $x$. 
  Assume $X = Sp(B)$, and let $b\in B$. Then $b$ corresponds to a decidable subset $U\subseteq X$.
  For any decidable subset $U \subseteq X$, we have 
  $\alpha^{-1}(U)$ a decidable subset of $\N$. 
  We claim that $\alpha^{-1}(U)$ is either finite or cofinite. 
  As $U$ is decidable, we can decide wheter $x\in U$. If $x\in U$, $\alpha^{-1}(U)$ is cofinite, as 
  $\alpha(n) \in U$ for all $n \geq N$ for some $N$. 
  If $x\notin U$, we have $x\in U^C$, which is also decidable and therefore $\alpha^{-1}(U^C)$ is cofinite. 
  As $\alpha^{-1}(U) ^ C = \alpha^{-1}(U^C)$, it follows that $\alpha^{-1}(U)$ is finite. 
  Thus $\alpha^{-1}(U)$ is finite or cofinite for any decidable subset $U\subseteq X$. 
  Finite and cofinite subsets of $\N$ correspond to elements of $B_\infty$. 
  Therefore, $\alpha$ induces a map $B \to B_\infty$, which corresponds to a map 
  $\overline \alpha: \Noo \to X$. 

  We claim that $\overline \alpha$ extends $\alpha$. 
  Denote $\iota$ for the map $\N \to \Noo$. 
  We need to show that $\overline \alpha \circ \iota = \alpha$. 
  By definition, we have that $(\overline \alpha \circ \iota)^{-1}(U) = \alpha^{-1}(U)$ 
  for any decidable $U\subseteq X$. 
\end{proof}

\begin{lemma}
  Whenever $S = Sp(B)$ Stone, $f,g: A \to S$, and $f^{-1}(U) = g^{-1}(U)$ for any decidable 
  $U\subseteq S$, we have $f = g$. 
\end{lemma}
\begin{proof}
  By our assumption, we have for all $a:A$ that $f(a) \in U \iff g(a) \in U$ for 
  any decidable $U\subseteq X$. Such $U$ correspond to $b:B$.
  and $f(a) \in U \iff f(a)(b)  = 1$. 
  So the functions $f(a),g(a):B \to 2$ are such that 
  $f(a) (b) = g(a) (b)$ for all $b:B$. 
  This holds for all $a:A$ and by two uses of function extensionality we may conclude 
  $f=g$. 
\end{proof}




\subsection{The interval}
%The goal of this section is to define the interval $[-2,2]_\mathbb R$ as a scheme. 
We assume $\mathbb N, \mathbb Q$ have been defined in HoTT
with linear propostional order relations $<,\leq, > ,\geq$ playing nicely together 
and standard algebraic operations. 
From these, we can define the subtype $\mathbb Q_{>0}=\sum_{q : \mathbb Q} (q>0)$, 
and the absolute-value function $|\cdot|$ on $\mathbb Q$. 

\begin{definition}
  A pre-Cauchy sequence is a sequence of rational numbers $(q_n)_{n: \mathbb N}$ with $-2 \leq q_n \leq 2$ 
  for all $n:\mathbb N$
%  together with a term of type
  such that for every $\epsilon: \mathbb Q_{>0}$, we have an $N_\epsilon:\mathbb N$, 
  such that whenever $n,m \geq N_\epsilon$, we have 
\begin{equation}
%  \forall \epsilon : \mathbb Q_{>0} \Sigma N : \mathbb N \forall m,n : \mathbb N (m,n \geq N) \to 
  | q_n - q_m | \leq \epsilon
\end{equation} 
\end{definition}

\begin{definition}
Given two pre-Cauchy sequences $p = (p_n)_{n\in\mathbb N}, q=(q_n)_{n\in\mathbb N}$, 
we define the proposition $p \sim_C  q$ as 
%for all $\epsilon : \mathbb Q_{>0}$ there exists an $N :\mathbb N$ such that whenever $n \geq N$, we have
\begin{equation}
  p \sim_C q : = \forall (\epsilon : \mathbb Q_{>0} )\exists ( N :\mathbb N) \forall (n : \mathbb N) ((n \geq N) \to 
  (| p_n - q_n| \leq  \epsilon))
\end{equation}
\end{definition}
Note that $\sim_C$ defines an equivalence relation on pre-Cauchy sequences. 
\begin{definition}
We define the type of Cauchy sequences as the type of pre-Cauchy sequences quotiented by $\sim_C$. 
\end{definition}

%\begin{definition}
%  A binary sequence consists of an initial segment $I \subseteq \mathbb N$
%  and a function $x:I \to 2$. 
%If $I$ is (in)finite, we call the binary sequence (in)finite as well. 
%\end{definition} 
%
%For $x$ a finite binary sequence and $y$ any binary sequence, 
%we'll denote $(x,y)$ for their concatenation, 
%and $\overline x$ for the infinite sequence repeating $x$. 
%
Denote $T = \{-1,0,1\}$. 
\begin{lemma}
  $T^\mathbb N$ is a scheme. 
\end{lemma}
\begin{proof}
  Sketch: partition $2^\mathbb N$ as follows: 
  For $\alpha: 2^\mathbb N$, we'll make a sequence $\beta: T^\mathbb N$.
  consider for each $n$ the $n$'th block of 2 entries in $\alpha$
  if both are $0$, $\beta(n) = 0$. 
  If the first is $1$, $\beta(n) = -1$
  If first is $0$ and the second is $1$, then $\beta(n) = 1$. 
  This is a closed equivalence relation. 
\end{proof} 

Consider the relation $\sim_s$ on $T^{\mathbb N}$, 
such that for any finite binary sequence $x$, we have 
\begin{align}
  (x,1,\overline 0) &\sim_t (x ,0, \overline 1) \\
  (x,-1,\overline 0) &\sim_t (x ,0, \overline {-1})\\
  (x,1,\overline {-1}) &\sim_t (x , \overline 0) \\
  (x,-1,\overline {1}) &\sim_t (x , \overline 0) 
\end{align} 
\begin{lemma}
$\sim_t$ induces a closed equivalence relation on $2^\mathbb N$. 
\end{lemma}
\begin{proof}
  TODO
\end{proof} 

\begin{proposition}\label{propTernaryCauchy}
  $T^\mathbb N/ \sim_t$ is isomorphic to the type of Cauchy sequences. 
\end{proposition} 
\begin{definition}%Construction might be better than definition here, but WIP so who cares. 
  For $\alpha: T^\mathbb N$, define the rational sequence $tri(\alpha)$ by 
  \begin{equation} (tri(\alpha))_n :  = \sum\limits_{0 \leq i \leq n} \frac{\alpha(i)} { 2^{i}} \end{equation}  
  This sequence is pre-Cauchy with $N_\epsilon$ given by the first $n$ with $(\frac12)^n<\epsilon$. 
\end{definition}  
%
%  Also, whenever $\alpha\sim_t \beta$, we have 
%  $tri(\alpha) \sim_C tri(\beta)$. 
%  Therefore $tri$ induces a function from $T^\mathbb N / \sim_t$ to Cauchy sequences. 
\begin{definition}
  Given a pre-Cauchy sequence $p$, 
  we will define a $T$-sequence $\alpha  = c(p): T^\mathbb N$.
  Consider any $i:\mathbb N$, and suppose $\alpha(j)$ has been defined for $0 \leq j<i$. 
%
  Let $\epsilon_i = (\frac12)^{i+1}$. %Placeholder value.
  Define $N_i:= N_{\epsilon_i}$. %is such that for $n,m \geq N_i$, we have $|p_n - p_m| < \epsilon_i$. 
%
  Consider 
  \begin{equation}
    \widetilde p_i = p_N - \sum\limits_{0\leq j < i} \frac {\alpha(j)}{2^{j}}.
  \end{equation}
  As the order on $\mathbb Q$ is total, we can define 
  \begin{equation}
    \alpha(i) = \begin{cases}
    \phantom{-} 1  \text{ if } \widetilde p_i \geq    (\frac12)^{i} \\
    -1             \text{ if } \widetilde p_i \leq  - (\frac12)^{i} \\
    \phantom{-} 0 \text{ otherwise } 
    \end{cases} 
  \end{equation}  
\end{definition} 
We shall now prove the following four things: 
\begin{itemize}
  \item 
    $c(tri(\alpha)) \sim_t \alpha$ for any $\alpha: T^n$.
  \item 
    $tri(c(p)) \sim_C p$ for any pre-Caucy sequence $p$. 
  \item 
    Whenever $p \sim_C q$, we have $c(p)\sim_t c(q)$. 
  \item 
    Whenever $\alpha \sim_t \beta$, we have $tri(\alpha) \sim_C tri(\beta)$. 
\end{itemize}
It follows that $c$ and $tri$ are maps between Cauchy sequences and $T^\mathbb N /\sim_t$ 
which are each other's inverse, proving Proposition \ref{propTernaryCauchy}
\begin{lemma} $tri(c(p)) \sim_C p$ for any pre-Caucy sequence $p$. 
\end{lemma} 



\begin{proof}
  Let $\epsilon>0$ be given, consider $n:\mathbb N$ such that
  $(\frac12)^n < \epsilon$. 
  We claim that for $m\geq N_n$, we have that $|p_m- tri(c(p))_m| < \epsilon$. 

  By definition $p_{N_n} $  
\end{proof} 






\subsection{The Cauchy reals}
The goal of this section is to introduce the real numbers in a constructive setting, 
following the definition given in \cite{Bishop} with some small adaptations. 
We will later use this definition to show that the interval $[0,1]$ is compact Hausdorff in the sense 
of \Cref{dfnCompactHausdorff}. 

We will assume we are given natural and rational numbers, with decidable (in)equalities
working as expected. 

\begin{definition}
  A \textbf{Cauchy sequence} is a sequence $x : \mathbb N \to \mathbb Q$ such that
  for any $n,m:\mathbb N$, we have %$0\leq x_n \leq 1$ and 
$|x_n-x_m| \leq (\frac12)^n + (\frac12)^m$. 
\end{definition}
\begin{remark}
  If $x$ is a cauchy sequence and $q$ a rational number, the 
  sequence $(x-q)_n = (x_n-q)$ is also Cauchy.
\end{remark}

Following \cite{Bishop}, we define inequality relations between Cauchy sequences and
rational numbers. 
\begin{definition}
  For $x$ a Cauchy sequence and $q$ a rational number, we define 
  \begin{itemize}
%    \item $x> q = \Sigma(n:\mathbb N) x_n > q + {\frac12}^n$. %for some $n:\mathbb N$. 
%    \item $x< q = \Sigma(n:\mathbb N) x_n < q - {\frac12}^n$. %for some $n:\mathbb N$. 
    \item $x\leq  q = \Pi_{n:\mathbb N} x_n \leq q+(\frac12)^n$. 
    \item $x\geq  q = \Pi_{n:\mathbb N} x_n \geq q-(\frac12)^n$. 
  \end{itemize}
\end{definition}
%\begin{lemma}
%  For $x$ Cauchy and $q$ rational, we have that 
%  $x\leq q$ iff for each $n:\mathbb N$, we have a $N_n:\mathbb N$ with 
%  $x_m> q-(\frac12)^n$ for all $m \geq N_n$. 
%\end{lemma}
\begin{lemma}\label{ComparisonLemma}
  For $x$ a Cauchy sequence and $q$ a rational number, we have
  $x \leq q \vee x \geq q$. 
\end{lemma}
\begin{proof}
  For rational numbers, we have decidable inequalities, 
  therefore $\geq 0 \vee q \leq 0$. 
  It follows that 
  $ \forall (n:\mathbb N) \forall (m:\mathbb N) q \geq -(\frac12)^n \vee q \leq (\frac12)^m$. 
  Now by \Cref{TODO}, we may conclude 
  $ (\forall (n:\mathbb N) q \geq -(\frac12)^n ) \vee (\forall (m:\mathbb N) q \leq (\frac12)^m)$
  as required.
\end{proof}


%%%\begin{definition}
%%%  A Cauchy sequence $x$ is \textbf{nonnegative} if $x_n \geq -(\frac12)^n$
%%%  for every $n:\mathbb N$. 
%%%  $x$ is \textbf{nonpositive} if $x_n \leq (\frac12)^n$
%%%  for every $n:\mathbb N$. 
%%%\end{definition} 
%%%%\begin{lemma}
%%%%  A Cauchy sequence is nonnegative iff there exists an $N$ such that $x_n \geq -(\frac12)^N$
%%%%  for all $n\geq N$. 
%%%%  A Cauchy sequence is nonpositive iff there exists an $N$ such that $x_n \leq (\frac12)^N$
%%%%  for all $n\geq N$. 
%%%%\end{lemma}
%%%%\begin{proof}
%%%%  Assume $x$ is nonnegative. Thus for every $n:\mathbb N$, we have $x_n\geq -(\frac12)^n \geq -(\frac12)^0$. 
%%%%  Thus $N$ can taken to be $0$. 
%%%%%
%%%%  Conversely, as $x$ is Cauchy, we have
%%%%  for all $m :\mathbb N$ that  
%%%%%  \begin{equation}- (\frac12)^m -(\frac12)^n \leq    x_m-x_n \leq (\frac12)^m + (\frac12)^n \end{equation}
%%%%  \begin{equation}- (\frac12)^m -(\frac12)^n \leq    x_n-x_m \leq (\frac12)^m + (\frac12)^n \end{equation}
%%%%  If in addition there is an $N$ such that whenever $m\geq N$, we have 
%%%%  $x_m \geq -(\frac12)^N$, so $-x_m \leq (\frac12)^N$, 
%%%%  so $x_n -x_m \leq x_n - (\frac12)^N$. 
%%%%  Therefore, 
%%%%  \begin{equation}- (\frac12)^m -(\frac12)^n \leq    x_n-x_m \leq x_n-(\frac12)^N \end{equation}
%%%%  Thus 
%%%%  \begin{equation}- (\frac12)^m -(\frac12)^n  + (\frac12)^N \leq x_n \end{equation}
%%%%  As $N \geq N$, we have in particular 
%%%%  \begin{equation}- (\frac12)^N -(\frac12)^n  + (\frac12)^N \leq x_n \end{equation}
%%%%  \begin{equation} - (\frac12)^n  \leq x_n \end{equation}
%%%%  thus $x$ is nonnegative. 
%%%%
%%%%  The nonpositive case goes similar. 
%%%%\end{proof}   
%%% 
%%%
%%%\begin{lemma}
%%%  A Caucy sequence is nonnegative or nonpositive. 
%%%\end{lemma}

%\begin{lemma}
%  For any Cauchy sequence $p$, we have 
%  $(\forall (n:\mathbb N) p_n \leq (\frac12)^n) \vee (\forall (n:\mathbb N) p_n \geq -(\frac12)^n)$. 
%\end{lemma}
%\begin{proof}
%We 
%\end{proof}

\begin{definition}
Given two Cauchy sequences $p = (p_n)_{n\in\mathbb N}, q=(q_n)_{n\in\mathbb N}$, 
we define the proposition $p \sim_C  q$ as 
\begin{equation}
  p \sim_C q : = \forall (n,m : \mathbb N) ((| p_n - q_m| \leq  (\frac12)^n + (\frac12)^m))
\end{equation}
\end{definition}

%\begin{remark}
%  Note that $p\sim_C q$ is equivalent to 
%\begin{equation}
%  \forall (n : \mathbb N) | p_n - q_n| \leq  (\frac12)^{n-1}
%\end{equation}
%The equivalence doesn't hold, unless you cut off initial segments (which shouldn't matter). 
%\end{remark} 

\begin{definition}
  The type of \textbf{Cauchy reals} is given by 
  the type of Cauchy sequences modulo $\sim_C$.
\end{definition}

We claim that the inequality in \Cref{TODO} extends to a well-defined 
inequality between Cauchy reals and rational numbers. 

Furthermore, we claim that 
$\Pi_{x:\mathbb R} \Pi_{q:\mathbb Q} x \leq q \vee x \geq q$. 

%\begin{lemma}
%  For any Cauchy real $r$ any Cauchy sequence $p$ representing $r$, 
%  we have 
%  \begin{equation}
%    (\forall (n:\mathbb N) p_n \leq (\frac12)^n) \vee (\forall (n:\mathbb N) p_n \geq (\frac12)^n)
%  \end{equation}
%
%\end{lemma}

\begin{definition}
  A Cauchy sequence in the interval is a Cauchy sequence $x$ such that 
  for any $n:\mathbb N$, we have $0\leq x_n \leq 1$. 
 % 
  The interval of Cauchy reals is given by the type of Cauchy sequences in the interval 
  modulo $\sim_C$. We denote it by $[0,1]$. 
\end{definition}  


We want to show that the interval of Cauchy reals is Compact Hausdorff. 
Informally, to any binary sequence $\alpha : \mathbb N \to 2$, 
we can associate a Cauchy sequence 
\begin{equation}\label{eqnBinaryEncode}
  n\mapsto \sum\limits_{i = 0 }^n \frac {\alpha(i)}{2^{i+1}}
\end{equation}
and we are going to give a closed relation on Cantor space such that 
two binary sequences are equivalent iff they correspond to the same Cauchy reals. 
%
First, we'll need some notation.
\begin{definition}
Given a binary sequence $\alpha:\mathbb N \to 2$ and a natural number $n : \mathbb N$  
we denote $\alpha|_n: \mathbb N_{\leq n} \to 2$ for the 
restriction of $\alpha$ to a finite sequence of length $n$. 
We denote $\overline 0, \overline 1$ for the binary sequences which are constantly $0$ and $1$ respectively. 
We denote $0,1$ for the sequences of length $1$ hitting $0,1$ respectively. 
If $x$ is a finite sequence and $y$ is any sequence, denote $x\cdot y$ for their concatenation. 
\end{definition} 
Now we'll give a definition for when two finite binary sequences of length $n$ correspond 
to real numbers whose distance is $\leq (\frac12)^n$.
Basically, we want for every finite sequence $z$ that 
$(z \cdot 0 \cdot \overline 1)$ and  $(z \cdot 1 \cdot \overline 0)$ are equivalent. 

\begin{definition}
Now let $n:\mathbb N$ and $x,y:\mathbb N_{\leq n} \to 2$ be two sequences of length $n$. 
We say $x,y$ are near if we have an $m:\mathbb N$ with $m\leq n$
and some $a: \mathbb N_{\leq m} \to 2$, 
such that one of $(a \cdot 0 \cdot \overline 1)|_n,  ( a \cdot 1 \cdot \overline 0)|_n$
is equal to $x$ and the other is equal to $y$. 
We denote $\text{near}_n(x,y)$ if $x,y$ are near. 
%
To be precise, we define 
\begin{equation}
  \text{near}_n(x,y) = 
\Sigma(m:\mathbb N) m \leq n \wedge 
  \Sigma (a : Fin_m \to 2) 
\bigg( \big( (x,y) = 
((a \cdot 0 \cdot \overline 1)|_n,  ( a \cdot 1 \cdot \overline 0)|_n)
\big)
\bigvee 
\big(
  (y,x) = 
((a \cdot 0 \cdot \overline 1)|_n,  ( a \cdot 1 \cdot \overline 0)|_n)
\big)
\bigg)
\end{equation}
\end{definition}
\begin{remark}
Remark that when $x,y$ are near, $m$ and $a$ as above are unique. 
Thus $\text{near}_n(x,y)$ is a proposotion. 
%
Furthermore, to check whether $x,y$ are near, we need only make $n$ comparisons, 
thus $\text{near}_n(x,y)$ is decidable. 
%
Note that in the above definition, we allow $m = n$ and therefore $x$ is near to itself for any finite sequence $x$. 
Furthermore, we have defined nearness to be symmetric. 
However, it is not a transtive relation. 
After all, the sequence $010$ and $011$ are near and the sequence $011$ and $100$ are near, 
but $010$ is not near to $100$. This corresponds to the fact that $\frac14$ and $\frac38$ are distance $\leq (\frac12)^3$
apart, and so are $\frac38$ and $\frac12$, but $\frac14$ and $\frac12$ are not. 
\end{remark}
\begin{definition}
  We define the following relation on Cantor space for $\alpha, \beta: 2^\mathbb N$.
  \begin{equation}
    \alpha \sim_t \beta = \forall (n : \mathbb N) 
    \text{near}_n(\alpha|_n, \beta|_n)
  \end{equation}
\end{definition}
\begin{lemma}
  $\sim_t$ is a closed equivalence relation. 
\end{lemma}
\begin{proof}
   Let $\alpha, \beta, \gamma : 2^\mathbb N$. 
   As the dependent product of propositions is a proposition, $\alpha \sim_t\beta$ is a proposition. 
   %
   Furthermore, the closedness follows from decidability of $\text{near}_n(\alpha|_n, \beta|_n)$. 
   One could define $\gamma(n) = 1$ iff $\text{near}_n(\alpha|_n, \beta|_n)$
   
   As nearness is reflexive and symmetric, so is $\sim_t$. 

   Now suppose $\alpha \sim_t \beta$ and $\beta\sim_t \gamma$. 
   We claim that $\alpha \sim_t \gamma$. 

   Let $n:\mathbb N$, we need to show that 
   $\text{near}_n(\alpha|_n , \gamma|_n)$. 
   Let $(a,m)$ witness that $\text{near}_n(\alpha|_n, \beta|_n)$.
   and let $(b, k)$ witness that $\text{near}_n(\beta|_n, \gamma|_n)$
   We will make a case distinction on whether one of $m,k$ is equal to $n$, or
   both are strictly smaller than $n$. 
   \begin{itemize}
     \item 
       If $m=n$, we have that $\alpha|_n = \beta|_n$, and therefore 
       \begin{equation}
         \text{near}_n(\beta|_n, \gamma|_n) \leftrightarrow \text{near}_n(\alpha|_n, \gamma|_n)
       \end{equation} 
       The above also holds if $k = n$.
     \item 
       If $m< n$, we have that $\alpha(m+1) \neq \beta(m+1)$, thus 
       $\alpha|_l \neq \beta|_l$ for all $l>m$, 
       but we still have $\text{near}_l(\alpha|_l, \beta|_l)$ for these $l$. 
       Therefore $(\alpha, \beta)$ or $(\beta, \alpha)$ must be of the form
       $(a \cdot 0 \cdot \overline 1, a \cdot 1 \cdot \overline 0)$. 
       WLOG, we assume $\alpha = a \cdot 0 \cdot \overline 1$, and thus 
       $\beta = a \cdot 1 \cdot \overline 0$ (if not, we could do bitflips). 

       As $k<n$ also, by the same argument there is some $b$ such that one of 
       $(\beta,\gamma), (\gamma, \beta)$
       is equal to $(b\cdot 0 \cdot \overline 1, b \cdot 1 \cdot \overline 0)$. 
       However, $\beta$ is also of the form $a \cdot 1 \cdot \overline 0$, and 
       thus cannot also be of the form $b \cdot 0 \cdot \overline 1$. 
       Therefore we must have 
       $\beta = b\cdot 1 \cdot \overline 0$ and 
       $\gamma= b\cdot 0 \cdot \overline 1$. 

       But now $b \cdot 1 \cdot \overline 0 = a \cdot 1 \cdot \overline 0$, 
       The lengths of $a,b$ cannot be unequal, and by decidablity of natural numbers, 
       $a,b$ have the same length and it follows that $ a = b$. 
       Therefore $ \alpha = \gamma$, so $\alpha \sim_t\gamma$.
   \end{itemize}

   We conclude that $\sim_t$ is a closed equivalence relation. 
\end{proof}


\begin{lemma}
  $b$ sends $\sim_n$ equivalent binary sequences to $\sim_C$ equivalent Cauchy sequences. 
\end{lemma}
\begin{proof}
  Let $\alpha, \beta$ be binary sequences.
  We claim that $|b(\alpha)_n - b(\beta)_n| \leq (\frac12)^{n+1}$ 
  whenever $\text{near}_n(\alpha, \beta)$. 
  It will follow that if $\alpha\sim_n \beta$, then 
  $b(\alpha)\sim_C b(\beta)$. 

  Let $n:\mathbb N$ and assume $m:\mathbb N$ with $m\leq n$ and 
  let $z$ be a sequence of length $m$ such that 
  $\alpha|_n = z\cdot 1 \cdot \overline 0|_n$ and $\beta|_n = z \cdot 0 \cdot \overline q |_n$. 
  then $b(\alpha)_n = \sum_{i\leq m} \frac{z(i)}{2^{i+1}} + (\frac12)^{m+2}$ and 
  $b(\beta)_n = \sum_{i\leq m} \frac{z(i)}{2^{i+1}} + \sum\limits_{m+2 \leq i \leq n}(\frac12)^{i+1}$. 
  Thus 
  $b(\alpha)_n - b(\beta)_n = (\frac12)^{m+2} - \sum\limits_{m+2 \leq i \leq n}(\frac12)^{i+1} = 
  (\frac12)^{n+1}$, 
  which is smaller than required. 
\end{proof}  

\begin{lemma}
  Whenever $b(\alpha) \sim_C b(\beta)$, 
  we have $\alpha \sim_n \beta$. 
\end{lemma}
\begin{proof}
  Assume $b(\alpha) \sim_Cb (\beta)$. 
  Let $n:\mathbb N$. 
  We shall show that $\text{near}_n(\alpha , \beta)$. 

  As we're only checking finitely many entries, 
  we either have $\alpha|_n = \beta|_n$, 
  or there exists a smallest $m\leq n$ with 
  $\alpha(m) \neq \beta(m)$. 

  If $\alpha|_n = \beta|_n$, we have $\text{near}_n(\alpha,\beta)$ and are done. 
  WLOG assume $\alpha(m) = 1, \beta(m) = 0$ for $m$ minimal. 
  We claim that for any $k\geq m$, we have 
  $b(\alpha)_k - b(\beta)_k = (\frac12)^k$. 

  Now note that 
  \begin{equation} 
    b(\alpha)_{k+1} - b(\beta)_{k+1} = 
    b(\alpha)_{k} - b(\beta)_{k} + 
    \frac{\alpha(k+1) - \beta(k+1)}{2^{k+2}}.
  \end{equation}




  For $k>m$, we have that 
  \begin{equation}
  |b(\alpha)_k - b(\beta)_k |= 
  |(\frac12)^{m+1} + \sum\limits_{i=m+1}^k \frac{ \alpha(i) -\beta(i)}{2^{i+1}}|. 
  \end{equation}
  Note that the right summand is always $\leq (\frac12)^{m+1}$. 
  Therefore, we can leave out the absolute value function. 
  As $b(\alpha) \sim_Cb(\beta)$, we have that 
  \begin{equation}
  (\frac12)^{m+1} + \sum\limits_{i=m+1}^k \frac{ \alpha(i) -\beta(i)}{2^{i+1}} \leq (\frac12)^{k-1}
  \end{equation}
  Note that $\alpha(i) -\beta(i) \in \{-1,0,1\}$ always. 
  Also, 

  Denote $z = \alpha|_{m-1} = \beta_{m-1}$. 
  We will show that for $n\geq i>m$, we must have $\alpha(i) = 0, \beta(i) = 1$. 
  Suppose $\alpha(i) \neq 0$. 
\end{proof}

%
%
%\begin{theorem}
%  The interval of Cauchy reals is isomorphic to $2^\mathbb N / \sim_t$. 
%\end{theorem} 
%\begin{proof}
%  Define $b: 2^\mathbb N \to [0,1]$ by composing the map in \Cref{eqnBinaryEncode} with the projection map 
%  from Cauchy sequences in the interval to the interval in Cauchy reals. 
%  We will show $b$ is surjective, 
%  and such that $b(\alpha) = b(\beta)$ iff  $\alpha \sim_t \beta$. 
%  \begin{itemize}
%    \item For $b$ to be surjective, we need to show that for any Cauchy sequence $p$ in the interval, there 
%      merely exists some binary sequence $\alpha$ such that $b(\alpha)\sim_C p$. 
%
%      We will inductively define $\alpha$. 
%      Let $\alpha(i)$ be defined for $i<n$. 
%      Consider the sequence $p'(j) = p(j) - \sum\limits_{i<n} \frac{\alpha(i)}{2^{i+1}}$. 
%      As $p$ is a Cauchy sequence, so is $p'$. 
%  \end{itemize}
%\end{proof}
%

%\printindex

\printbibliography

\end{document}
