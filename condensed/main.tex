% latexmk -pdf -pvc main.tex
\documentclass{../util/zariski-small}


\title{Synthetic Stone Duality}

\begin{document}

\author{Felix Cherubini, Thierry Coquand, Freek Geerligs and Hugo Moeneclaey}

\maketitle

%\begin{abstract}
%In synthetic algebraic geometry (SAG) \cite{draft}, we study finitely presented algebras over a commutative ring. 
%In this work, we study countably presented Boolean algebras instead. 
%Where the finitely presented algebras over a commutative ring induce a Zariski topos, 
%%the opposite category of these 
%the countably presented Boolean algebras induce the topos of light condensed sets \cite{Scholze}. 
%\cite{draft} proposes an axiomatization of the Zariski topos in univalent homotopy type theory \cite{hott}. 
%In this work, we propose similar axioms, which we expect to be modelled by light condensed sets. 
%% Furthermore, spectra of countably presented Boolean algebras correspond to quotients of Cantor space
%% which is cool because reasons
%\end{abstract} 
%
\rednote{The following is a collection of notes on work in progress.}
\section{Countably presented Boolean algebras}
\subsection{Definitions}
\begin{definition}
  The language of Boolean algebras consists of
  \begin{itemize}
    \item Two symbols for constants $0,1$. 
    \item Two binary function symbols $\wedge, \vee$ called meet and join respectively. 
    \item One unary function symbol $\neg$ called negation. 
  \end{itemize}
\end{definition}
\begin{definition}
  The algebraic theory of Boolean algebras is as follows:
  \begin{itemize}
    \item We have axioms regarding order, sometimes called identity and absorption:
    \begin{itemize}
      \item $0 \wedge a = 0$. 
      \item $1 \wedge a = a$. 
      \item $0 \vee a = a$. 
      \item $1 \vee a = 1$. 
      \item $a \vee (a \wedge b) = a$
      \item $a \wedge (a \vee b) = a$. 
    \end{itemize}
    \item We have axioms regarding negations:
      \begin{itemize}
        \item $a \vee \neg a = 1$. 
        \item $a \wedge \neg a = 0$. 
        \item $\neg 1 = 0$. 
        \item $\neg \neg a = a$. 
      \end{itemize}
    \item We have axioms regarding commutativity and associativity:
      \begin{itemize}
        \item $a \vee b = b \vee a$. 
        \item $a \wedge b = b \wedge a$. 
        \item $(a \vee b) \vee c = a \vee (b \vee c)$. 
        \item $(a \wedge b) \wedge c = a \wedge (b \wedge c)$. 
      \end{itemize}
    \item We have axioms regarding distributivity:
      \begin{itemize}
        \item $a \vee (b \wedge c) = ( a \vee b) \wedge (a \vee c)$. 
        \item $a \wedge (b \vee c) = (a \wedge b) \vee ( a\wedge c)$. 
      \end{itemize}
    \item We have axioms called deMorgan's laws:
      \begin{itemize}
        \item $\neg (a \vee b ) = (\neg a) \wedge (\neg b)$. 
        \item $\neg (a \wedge b) = (\neg a) \vee (\neg b)$. 
      \end{itemize}
  \end{itemize}
\end{definition}
\begin{definition}
  Some language we will use is as follows:
  \begin{itemize}
    \item A Boolean algebra is a set-theoretic model of the theory of Boolean algebras. 
      It consists of a set $B$ and terms $0_B,1_B: B, \neg_B : B \to B, \vee_B, \wedge_B : B \times B \to B$ 
      satisfying the theory of Boolean algebras. 
      Sometimes we will leave out the subscript $(\cdot)_B$. 
    \item A morphism of Boolean algebra is a function between such sets preserving the operations above. 
      If $A,B$ are Boolean algebras, we write $A \to B$ for the type of such morphisms. 
%    \item We denote $id_B$ for the identity morphism $B \to B$. 
    \item If we have have two morphisms $f:A\to B , g : B \to A$ such that 
      $f\circ g (b) = b, g \circ f (a) = a$ for all $a:A, b:B$, 
      we say $A$ and $B$ are isomorphic and write $A \simeq B$. 
  \end{itemize}
\end{definition}
\begin{remark}
  We sometimes write $a \leq b$ and  $b \geq a$. Both of these mean that 
  $a\wedge b = a$, or equivalently $ a\vee b = b$. 
  Interpreting this notation as an order relation, we often think in the following terms:
  \begin{itemize}
    \item $a\wedge b$ is the biggest element $c$ with $c \leq a$ and $c\leq b$.
    \item $a\vee b$ is the smallest element $c$ with $c \geq a$ and $c \geq b$. 
    \item $0$ is the smallest element. 
    \item $1$ is the biggest element. 
  \end{itemize}
\end{remark}
\begin{remark}
  For a family of expressions $x_i$ over a 
  finite set of indices $I = \{i_0, \cdots, i_n\}$, we will sometimes denote 
  \begin{itemize}
    \item $\bigvee_{i:I} x_i = x_{i_0} \vee \cdots \vee x_{i_n}$,
      with the understanding that this equals $0$ if $I$ is empty. 
    \item $\bigwedge_{i:I} x_i = x_{i_0} \wedge \cdots \wedge x_{i_n}$,
      with the understanding that this equals $1$ if $I$ is empty. 
  \end{itemize}
\end{remark}
\begin{remark}
  Boolean algebras are in bijective correspondence with Boolean rings, which 
  are rings with the property that $x\cdot x = x$ for all $x$. 
  The underlying set remains the same, and the translation is as follows:
  \begin{itemize}
    \item 
      To go from such a ring to a Boolean algebra, we let 
      $0 = 0, 1 =1 ,  a \vee b = a + b + ( a \cdot b), a\wedge b = a \cdot b, \neg a = a + 1$. 
    \item 
      and to go from a Boolean algebra to a Boolean ring, we let 
      $0 = 0, 1 = 1, a + b = (a \wedge \neg  b) \vee ( \neg a \wedge b), a \cdot b = a\wedge b , -a = a$. 
\end{itemize}
\end{remark}

%\begin{definition}
%  Let $\mathcal L$ be a language, 
%  let $\phi,\psi$ be terms in any language,
%  and suppose that for some index set $I$ we have 
%  for each $i:I$ two terms $a_i,b_i$ of same sort in $\mathcal L$. 
%  Let $E= (a_i = b_i)_{i:I}$. 
%
%  We say that $E$ allows us to rewrite $\phi$ to $\psi$ if there is a finite list of expressions 
%  $\phi = \phi_0 , \phi_1, \phi_2, \cdots, \phi_{n-1}, \phi_n = \psi$, 
%  such that each $\phi_{n+1}$ is given by taking $\phi_n$ and either changing one occurence of $a_i$ to  $b_i$, 
%  or one occurence of $b_i$ to $a_i$ for some $i:I$. 
%\end{definition}


\begin{definition}
  Given any set of symbols $G$, the free Boolean algebra on $G$, denoted $\langle G \rangle$
  is constructed as follows:
  \begin{itemize}
    \item We add $G$ as set of constants to the language of Boolean algebras. 
    \item We consider the set of terms in this language. 
    \item If there merely exists some finite sequence of 
      equalities from the theory of Boolean algebras allowing us to rewrite one term to another, 
      we consider the two terms to be equivalent. 
    \item The free Boolean algebra is then the set of equivalence classes under this relation. 
  \end{itemize}
\end{definition}
\begin{remark}
  The free Boolean algebra is a Boolean algebra. 
\end{remark}
\begin{definition}
  Let $B$ be a Boolean algebra, a subset $I\subseteq B$ is an ideal iff 
  $0\in I$ and whenver $x,y \in I$ and $b\in B$, we have $x\vee y \in I$ and $x \wedge b \in B$. 

  If $I$ is an ideal of $B$, then for $a,b:B$, we write $x\sim_I y$ iff 
  $a - b  = 
  (a \wedge \neg b ) \vee ( \neg a \wedge b) 
  \in I$. 
\end{definition}
\begin{lemma}
  Let $B$ be a Boolean algebra, and let $I\subseteq B$ be an ideal. 
  Then the relation $\sim_I$ is an equivalence relation. 
\end{lemma}
\begin{proof}
  \begin{itemize}
    \item To see $\sim_I$ is reflexive, note that 
      $(x \wedge \neg x) \vee (\neg x \wedge x) = 0 \vee 0 = 0 \in I$. 
    \item To see $\sim_I$ is symmetric, note that 
      $ (a \wedge \neg b) \vee ( \neg a \wedge b) = 
        (b \wedge \neg a) \vee ( \neg b \wedge a)$. 
    \item To see $\sim_I$ is transitive, that 
      whenever $a,b \in I$, so is $ a + b$. 
      Thus if $a -b , b -c \in I$, so is 
      $(a -b) + (b-c) = a -c$.
  \end{itemize}
\end{proof}
\begin{lemma}
  Whenever $B$ is a Boolean algebra and $I\subseteq B$ an ideal, 
  $B/I$ is a Boolean algebra. 
\end{lemma}
\begin{proof}
  We need to show that the Boolean operations respect equivalence classes. 
  \begin{itemize}
    \item 
      Suppose $a \sim_I b$. We shall show that $\neg a \sim_I \neg b$. 
      Thus we need to show that 
      $((\neg a) \wedge (\neg \neg b)) \vee ((\neg \neg a) \wedge \neg b) \in I$, 
      this term is equal to 
      $( a \wedge \neg b) \vee (b \wedge \neg a)$, which is in $I$ as $ a\sim_I b$. 
    \item 
      Suppose $ a\sim _I b $. We shall show that 
      $a \wedge c \sim_I b \wedge c$. 
      Note that by distributivity of $+$ and $\cdot$, we have 
      $(a \cdot c) -( b \cdot c) = (a -b ) \cdot c = (a-b) \wedge c$. 
      By assumption $a -b \in I$, and thus $(a-b) \wedge c \in I$ as well. 
    \item Note that $a \vee b = \neg ( \neg a \wedge \neg b)$, by the above two properties, 
      joins therefore respect equivalence relations. 
  \end{itemize}
\end{proof}
\begin{definition}
  Let $B$ be a Boolean algebra, and let $R\subseteq B$. 
  We denote $\langle R\rangle $ for the ideal generated by $R$, 
  which is the set of expressions 
  $\{(\bigvee_{r \in R_0} r) \wedge b| R_0\subseteq R \text{ finite}, b \in B\}$
\end{definition} 
\begin{remark}
  For $R\subseteq B$ as above, $ \langle R \rangle$ is an ideal.
\end{remark}

\begin{remark}\label{rmkMorphismsOutOfQuotient}
  Let $B$ be a Boolean algebra. 
  
  Suppose $G$ is a set of symbols, to define a morphism 
  $\langle G \rangle \to B$ it is sufficient to define 
  a function from $G$ to the underlying set of $B$. 

  Let $C$ be a Boolean algebra, and let $I\subseteq C$ be an ideal. 
  To define a map $C / I \to B$, it is sufficient to define a map $f:C\to B$
  with $f(i) = 0$ for all $i\in I$. 
  If $I = \langle R\rangle $, it is sufficient that $f(r) = 0$ for all $r\in R$. 
\end{remark}
%\begin{remark}
%  For every element $x$ of $\langle G\rangle /R$ as above,
%  there merely exists a finite subset of generators $G_0\subseteq G$ such that $x$ is expressable 
%  as a Boolean combination of symbols from $G_0$. 
%\end{remark}
\begin{definition}
  Let $B$ be a Boolean algebra. 
  If there exist $G, R$ as above such that 
  $B \simeq \langle G \rangle /\langle R \rangle$, 
  we call $G,R$ respectively generators and relations for $B$. 
  %
  If $G,R$ can taken to be finite, we call $B$ finitely presented. 
  If $G,R$ can taken to be countable, we call $B$ countably presented. 
\end{definition}
\begin{remark}
  There is a category of Boolean algebras, which has a subcategory of countably presented Boolean algebras, 
  which has a subcategory of finitely presented Boolean algebras. 
\end{remark}
\begin{remark}\label{rmkBoolePushouts}
  All categories above have pushouts. 
\end{remark}
\begin{definition}
  Let $f:B\to C$ be a Boolean morphism. 
  The set $\{x \in B| fx = 0\}$ is called the kernel of $f$ and denoted 
  $Ker(f)$. 
\end{definition}
\begin{remark}
  For any map of Boolean algebras $f$, the kernel of $f$ is an ideal. 
  Furthermore, we have that $a\sim_{Ker(f)} b$ iff $f(a) = f(b)$. 
\end{remark}
\begin{remark}
  Let $f:B \to C$ be a Boolean morphism. 
  The epi-mono factorization of $f$ exists and is given by  
  $B \twoheadrightarrow B / Ker(f) \hookrightarrow C$.
\end{remark}

\subsection{Countably presented as colimit of finitely presented}
\subsection{Countably presented as colimit of finitely presented}
\begin{definition}
  A sequence in a category is a diagram of shape $\mathbb N$, 
  where $\mathbb N$ carries the natural structure of a poset. 
\end{definition}
\begin{lemma}\label{lemProFinitePresentation}
  For every countably presented Boolean algebra $B$
  there merely exists a sequence of finitely presented Boolean algebras 
  whose colimit in the category of Boolean algebras is $B$. 
\end{lemma}
\begin{proof}
  Consider $\langle G \rangle /R$ a countable presentation of a Boolean algebra $B$. 
%  We will show there exists a diagram of shape $\mathbb N$ taking values in Boolean algebras 
%  with $\langle G\rangle / R$ as the colimit.
  \paragraph{The diagram}
  Now let $R_n$ be the first $n$ terms in $R$. 
  Note that each of these finitely many terms uses only finitely many symbols from $G$.
  Let $G_n$ be the finite set of terms used in $R_n$, unioned with the finite set of the first $n$ elements of $G$. 
  Define for each $n\in\mathbb N$ the finitely presented Boolean algebra $B_n = G_n /R_n$. 
  If $n\leq m$, then \Cref{rmkMorphismsOutOfQuotient} gives us a map $B_n \to B_m$ 
  as $G_n \subseteq G_{n+1}$ and $R_n \subseteq R_{n+1}$. 
  Thus $(B_n)_{n\in \mathbb N}$ gives us a diagram of shape $\mathbb N$
  with values in finitely presented algebras. 

  \paragraph{The colimit}
  As $G_n\subseteq G$ and $R_n \subseteq R$, 
  \Cref{rmkMorphismsOutOfQuotient} also gives us a map $B_n\to \langle G \rangle /R$. 
  We claim the resulting cocone is a colimit. 

  Suppose we have a cocone $C$ on the diagram $(B_n)_{n\in\mathbb N}$. 
  We need to show that there exists a map $\langle G \rangle / R\to C$ and
  we need to show this map is unique as map between cocones. 
  \begin{itemize}
    \item To show there exists a map $\langle G \rangle / R \to C$, 
      we use remark \Cref{rmkMorphismsOutOfQuotient} again. 
      Let $g\in G$ be the $n$'th element of $G$, 
      note that $g\in G_n$, and consider the image of $g$ under the map $B_n \to C$. 
      This procedure defines a function from $G$ to the underlying set of $C$. 
      Let $\phi \in R$ be the $n$'th element of $R$, 
      note that $\phi \in R_n$, and the map $B_n \to C$ must send $\phi$ to $0$. 
      Thus the function from $G$ to the underlying set of $C$ also sends $\phi$ to $0$. 
      This thus defines a map $\langle G \rangle / R \to C$. 
    \item To show uniqueness, consider that any map of cocones $\langle G \rangle /R \to C$ 
      must take the same values on all $g\in G_n$ for all $n\in\mathbb N$. 
      Now all $g\in G$ occur in some $G_n$, so any map of cocones $\langle G \rangle /R \to C$ 
      takes the same values for all $g\in G$. 
      \Cref{rmkMorphismsOutOfQuotient} now tell us that these values uniquely determine the map. 
  \end{itemize}
\end{proof}
\begin{remark}
  Conversely, any colimit of a sequence of finite Boolean algebras 
  is a countably presented Boolean algebra with 
  as underlying sets of generators and relations the countable union of the finite sets of 
  generators and relations, which are both countable. 
\end{remark}
\begin{lemma}\label{lemFinitelyPresentedBACompact}
  For any finitely presented Boolean algebra $A$,
  and any sequence $(B_n)_{n:\mathbb N}$ of Boolean algebras with colimit $B$
  we have that the set $B^A$ is the colimit of the sequence of sets $(B_n^A)_{n:\mathbb N}$. 
\end{lemma}  
\begin{proof}
  First note that $B^A$ forms a cocone on $(B_n^A)_{n:\mathbb N}$ 
  because any map $A \to B_n$ induces a map $A \to B$. 
  Let $C$ be a cocone on $(B_n^A)_{n:\mathbb N}$. 
  We shall show there is an unique morphism of cocones $B^A \to C$. 
  \begin{itemize}
    \item For existence, let $f:B^A$. 
      As $A$ is finitely presented, we write $A = \langle G \rangle / R$ with $G$ finite.
      By \Cref{rmkMorphismsOutOfQuotient}, $f$ is uniquely determined by it's values on $g\in G$. 
      As $G$ is finite, so is it's image $f(G)\subseteq B$. 
      But any finite subset of $B$ already occurs in $B_n$ for some $n\in\mathbb N$. 
      Consequently, the image of $f$ is already contained in some $B_n$. 
      Thus there is some $f_n:(B_n^A)$ such that postcomposing 
      $f_n$ with the map $B_n \to B$ gives back $f$. 
      The image of $f_n$ under the map $(B_n^A) \to C$ is how we define the image of $f$. 
      This is well-defined by the cocone conditions on $C$. 
    \item 
      For uniqueness, by function extensionality maps $B^A \to C$ are uniquely determined by their values on 
      $f:B^A$. By the above, the value of $f$ is uniquely determined by it's value on $B_n$ for 
      any $n$ with the image of $f$ in $B_n$. Thus there is at most one morphism of cocones $B^A \to C$. 
  \end{itemize}
\end{proof}
\begin{remark}
  In the above proof, we used that any element $b\in B$ already occurs in some $B_n$. 
  However, please note that it is not necessarily the case that it occurs uniquely in $B_n$, 
  there might be multiple elements in $B_n$ which can all be sent to $b$ in the end. 

  In case our sequence comes from the construction in \Cref{lemProFinitePresentation}, 
  we can see that whenever there are a finite amount of elements in 
  $B_n$ corresponding to $b\in B$, they will become equal in $B_m$ for some $m\geq n$. 
  The reason is that if two elements are equal, we can rewrite them to each other using finitely many substitutions 
  from $R$, and these finitely many substititions will occur in some $B_m$, and thus also for some $B_m$ with $m\geq n$. 

  \rednote{Is Markov necessary for the following?}
  In general, if we assume $B$ is the colimit of an arbitrary sequence $(B_n)_{n:\mathbb N}$, 
  and there exist some $B_n$ with two elements corresponding to the same element in $B$, 
  Theorem 7.4 from \cite{SequentialColimitHoTT} then says that there merely exists some $m\geq n$
  such that they are already equal in $B_m$. 
\end{remark}

For our next lemma on this presentation of sequences we need the axiom of dependent choice. 
\begin{axiomNum}[Dependent choice]\label{axDependentChoice}
  Given a family of types $(E_n)_{n:\mathbb N}$ and 
  a relation 
  $R_n:E_n\rightarrow E_{n+1}\rightarrow {\mathcal U}$ such that
%  $\Pi_{n:\mathbb N} \Pi_{x:E_n} \exists_{y:E_{n+1}} R_n x y$.
  for all $n$ and $x:E_n$ there exists $y:E_{n+1}$ with $p:R_n~x~y$ 
  then given $x_0:E_0$ there exists
  $u:\Pi_{n:\N}E_n$ and $v:\Pi_{n:\N}R_n~(u~n)~(u~(n+1))$ and $u~0 = x_0$.
\end{axiomNum}
\begin{lemma}[Using dependent choice]\label{lemDecompositionOfColimitMorphisms}
  Let $B,C$ be countably presented Boolean algebras, 
  and suppose we have a morphism $f:B\to C$.
  There exists sequences of finitely presented Boolean algebras 
  $(B_n)_{n:\mathbb N}, (C_n)_{n:\mathbb N}$ with colimits $B,C$ respectively
  and compatible maps of Boolean algebras $f_n:B_n \to C_n$, 
  such that $f$ is the induced morphism $B\to C$.
\end{lemma}



\rednote{Active here}
\begin{proof}
  Let $(B_n)_{n:\mathbb N}, (C_n)_{n:\mathbb N}$ be 
  sequences of finitely presented Boolean algebras with colimits $B$ and $C$. 
  We will take a subsequence of $(C_n)_{n:\mathbb N}$, using the axiom of dependent choice above. 

  Our family of types $E_k$ as in \Cref{axDependentChoice} 
  will be strictly increasing sequences $(n_i)_{i\leq k}$ of natural numbers together with a finite family of maps 
  $(f_i: B_{i} \to C_{n_i})_{i\leq k}$ such that
  for all $0\leq i<k$ the following diagram commutes:
  \begin{equation}\label{eqnDecompositionOfColimitMorphisms}
    \begin{tikzcd}
      B_{i} \arrow[r] \arrow[d, "f_i"]& B_{{i+1}} \arrow[r] \arrow[d,"f_{i+1}"]& B \arrow[d,"f"] \\
      C_{n_i} \arrow[r] & C_{n_{i+1}} \arrow[r] & C 
    \end{tikzcd}
  \end{equation}
  Our relation $R_k$ will tell whether the second sequence extends the first one. 
%
  By \Cref{lemFinitelyPresentedBACompact} 
  there exists some $n_0:\mathbb N$ 
  such that $B_0 \to B \to C$ factors as 
  \begin{equation}
    \begin{tikzcd}
      B_{0} \arrow[r] \arrow[d, "f_0"]& B \arrow[d,"f"] \\
      C_{n_0} \arrow[r] & C 
    \end{tikzcd}
  \end{equation}
  Because our goal is a proposition, we can untracate this existence to data. 
  This data will form our $x_0:E_0$. %from \Cref{axDependentChoice}. 
%
  Now suppose we have $(f_i: B_{i} \to C_{n_i})_{i\leq k}$ for some $k\geq 0$ 
  such that
  for all $0\leq i<k$ the diagram of \Cref{eqnDecompositionOfColimitMorphisms} commutes.
  We shall show that in this case there exists an $n_{k+1}, f_{k+1}$ 
  making the same diagram commute for $i = k$. 
  Consider $B_{{k}+1}\to B \to C$. By the same argument as for $B_0$, we have a factorization 
  \begin{equation}
    \begin{tikzcd}
    B_{k+1} \arrow[r]  \arrow[d,"f'_{k+1}"]& B \arrow[d,"f"]\\
    C_{n'_{k+1}} \arrow[r] & C
    \end{tikzcd}
  \end{equation}
  Note that we may assume $n'_{k+1} > n_k$.
  Note that it is not necessarily the case that 
  $f'_{k+1}$ is compatibly with $f_k$, meaning the left square in the following diagram needn't commute 
  However, both $f'_{k+1}, f_k$ induce the same map $B_{n_k} \to C$, 
  so by \Cref{lemFinitelyPresentedBACompact} there exists an $n_{k+1}$ 
  such that the outer square and right square in 
  \rednote{missing square involving $C_{n_{k+1}}$, furthermore you need Markov as we mention above}
  \begin{equation}
    \begin{tikzcd}
      B_{k} \arrow[r] \arrow[d, "f_k"]& B_{{k+1}} \arrow[rr] \arrow[d,"f'_{k+1}"]& & B \arrow[d,"f"] \\
      C_{n_k} \arrow[r] & C_{n'_{k+1}} \arrow[r] & C_{n_{k+1}} \arrow[r] & C 
    \end{tikzcd}
  \end{equation}
  do commute. 
  Here we may also assume that $n_{k+1}>n_k$ 
  and we take $f_{k+1}$ to be the resulting morphism $B_{k+1} \to C_{n_{k+1}}$. 

  Now by dependent choice for this situation, we get a sequence $(f_i:B_i \to C_{n_i})$  for some 
  strinctly increasing sequence $n_i$ of natural numbers. 
  Note that any strictly increasing subsequence of a sequence still has the same colimit, 
  thus $(C_{n_i})_{i:\mathbb N}$ converges to $C$ and $(B_i)_{i:\mathbb N}$ still converges to $B$. 
  Thus our sequence $(f_i)_{i:\mathbb N}$ is as required. 
  \rednote{I'm missing some nuance regarding the second extension} 

\end{proof}
\begin{lemma}
  If $f:B \to C$ is a surjective/injective morphism of Boolean algebras, 
  the maps $f_n$ from \Cref{lemDecompositionOfColimitMorphisms} can taken to 
  be surjective/injective as well. 
\end{lemma}
\begin{proof}
\end{proof}

\rednote{End of activity}




\subsection{Examples of countably presented Boolean algebras}
\begin{example}
  $2$ is the Boolean algebra given by the empty set of generators and no relations. 
    It's underlying set is $\{0,1\}$. 
\end{example}
\begin{example}
  The trivial Boolean algebra given by the empty set of generators and the relation $\{1\}$, 
  It's underlying set contains only one element and we have $0=1$ in the trivial Boolean algebra. 
\end{example}

\begin{example}\label{ExampleBAunderCantor}
  $C = \langle N \rangle $ is the Boolean algebra given by $\mathbb N$ as set of generators and no relations. 
\end{example}
\begin{example}\label{ExampleBAunderNinfty}
  Denote $\mathbb N_{(co)fin}$ for the set of subsets of $\mathbb N$ which are finite or co-finite. 
  Under the interpretation of $\wedge = \cap , \vee = \cup, 0 = \emptyset, 1 = \mathbb N$, and $\neg$ 
  as the set-theoretic complemented (denoted $(\cdot)^C$). 

  These operations are well-defined on $\mathbb N_{(co)fin}$ 
  and they give the structure of a Boolean algebra.
  We claim it is countably presented. 

  Let $G = \{p_n| n:\mathbb N\} $ be a countably infinite set of generator symbols and let 
  $R = \{ p_n \wedge p_m | n\neq m :\mathbb N \}$. 
  We claim that $\langle G \rangle / R \simeq \mathbb N_{(co)fin}$. 

\begin{proof}
  We let $f:\langle G \rangle / R \to \mathbb N_{(co)fin}$, be the unique morphism of Boolean algebras 
  satisfying $f(p_n) = \{n\}$ for all $n:\mathbb N$. As $\{n\} \cap \{m\} = \emptyset$ whenever $n\neq m$, 
  \Cref{rmkMorphismsOutOfQuotient} indeed tells us this is sufficient to define $f$. 

  Now we define $g:\mathbb N_{(co)fin} \to \langle G \rangle / R$. 
  On a finite subset $I$, we define $g(I) = \bigvee_{i\in I} p_i$, 
  and on a cofinite subset $J$, we define $g(J) = \bigwedge _{i \in J^C} \neg p_i$. 
  Note that in these cases we indeed have $I,J^C$ are finite, so these are well-defined elements. 

  We need to show that $g$ is a Boolean morphism. 
  \begin{itemize}
    \item 
      By deMorgan's laws, $g$ preserves $\neg$:
      for $I$ finite we have
      \begin{equation}
      \neg g(I) = \neg (\bigvee_{i\in I} p_i) = \bigwedge_{i\in I} \neg p_i = g(I^C)
      \end{equation}
      And for $J$ cofinite, we apply similar reasoning. 
    \item To see that $g$ preserves $\vee$, we need to check three cases
      \begin{itemize}
        \item If both $I,J$ are finite, then 
        \begin{equation} 
          g(I \cup J) = \bigvee_{i\in I \cup J} p_i= \bigvee_{i\in I} p_i \vee \bigvee_{j\in J} p_j 
          = g(I) \vee g(J)
        \end{equation}
        and we're done. 
      \item If both $I,J$ are cofinite, we have
        \begin{equation}
          g(I) \vee g(J) = 
          \bigwedge_{i \in I^C} \neg p_i \vee 
          \bigwedge_{j \in J^C} \neg p_j 
          = 
          \bigwedge_{i\in I^C} 
          \bigwedge_{j \in J^C}(\neg p_i \vee  \neg p_j) 
        \end{equation}
        Now note that in $\langle G \rangle / R$, we have 
        \begin{equation}
          \neg p_i \vee \neg p_j = \neg ( p_i \wedge p_j) = 
          \begin{cases}
            p_i \text{ if } i = j\\
            1 \text{ if } i \neq j  
          \end{cases}
        \end{equation}
        Therefore, we can leave out the case that $i\neq j$ in the calculation of the above meet, and
        \begin{equation}
          \bigwedge_{i\in I^C} 
          \bigwedge_{j \in J^C}(\neg p_i \vee  \neg p_j)  
          = 
          \bigwedge_{i \in (I^C \cap J^C)} \neg p_i
          = 
          \bigwedge_{i \in (I \cup J)^C} \neg p_i 
        \end{equation}
        as $I\cup J$ must also be cofinite, this equals 
          $ g( I \cup J)$. 
        \item 
          If $I$ is finite and $J$ cofinite, we have 
          that $I\cup J$ is cofinite, hence 
          \begin{equation}
            g(I\cup J) = \bigwedge_{k\in (I \cup J)^C} \neg p_k
            = \bigwedge_{k \in (J^C -I)} \neg p_k
          \end{equation}
          Now note that 
          whenever $i\neq k$, we have 
          \begin{equation}
            p_i = (p_i \wedge \neg p_k) \vee (p_i \wedge p_k) = 
            (p_i \wedge \neg p_k) \vee 0 = p_i \wedge \neg p_k
          \end{equation}
          Hence by absorption
          \begin{equation} 
            (p_i \vee \neg p_k)  =
              \begin{cases}
                1 \text{ if } i = k \\
                \neg p_k \text{ if } i \neq k
              \end{cases}
          \end{equation}
          As for all $k\in J^C-I$ and all $i\in I$ we have $k\neq i$, we may thus write
          \begin{equation}
            \bigwedge_{k \in (J^C - I)} \neg p_k = 
            \bigwedge_{k \in (J^C - I)} (\neg p_k \vee (\bigvee_{i\in I} p_i))
          \end{equation}
          But now we can use that adding $1$ in a meet does not change the meet, and see that 
          \begin{equation}
            \bigwedge_{k \in (J^C - I)} (\neg p_k \vee (\bigvee_{i\in I} p_i))
            = 
            \bigwedge_{j \in J^C} (\neg p_j \vee (\bigvee_{i\in I} p_i))
          \end{equation}
          And using distributivity rules, we can see that 
          \begin{equation}
            \bigwedge_{j \in J^C} (\neg p_j \vee (\bigvee_{i\in I} p_i))
            = 
            (\bigwedge_{j \in J^C} \neg p_k) \vee (\bigvee_{i\in I} p_i)
          \end{equation}
          From which we may conclude that $g(I\cup J) = g(I) \cup g(J)$. 
      \end{itemize}
    \item The case for $\wedge$ is completely dual to the case for $\vee$. 
  \end{itemize}
We conclude that $g$ is a Boolean morphism. 
Furthermore, $g$ and $f$ are each other's inverse, thus the Boolean algebras are isomorphic. 
\end{proof}
\end{example}
\begin{remark}
  We will denote $B_\infty$ for $\langle G \rangle/ \langle R\rangle $ as above, 
  and by the above result, we have that any $b:B_\infty$ can be written 
  either as $\bigvee_{i\in I_0} p_i$ or as $\bigwedge_{i\in I_0} \neg p_i$ for some finite $I_0\subseteq \mathbb N$. 
\end{remark}
\section{Stone spaces}
\newcommand{\isSt}{\mathsf{isStone}}
\subsection{Definitions}
\begin{definition}
  For $B$ a countably presented Boolean algebra, we define $Sp(B)$ as the set of Boolean morphisms from $B$ to $2$. 
\end{definition}
\begin{definition}
  We define the predicate on types $\isSt$ by 
  \begin{equation}
    \isSt(X) := \sum\limits_{B : Boole} X = Sp(B)
  \end{equation} 
  A type $X$ is called \textit{Stone} if $\isSt(X)$ is inhabited.
\end{definition}
\subsection{Examples of Stone types}
\begin{example}\label{ExampleBAunderEmpty}
  The dual to the trivial Boolean algebra from $\Cref{ExampleBAunderEmpty}$ is the empty set $\emptyset$, 
  as $0\neq 1$ in $2$, but $0=1$ in the trivial Boolean algebra, there can be no functions preserving both $0$ and $1$ 
  from the trivial Boolean algebra to $2$. 
\end{example}
\begin{example}
  The dual to $C$ from \Cref{ExampleBAunderCantor} is called Cantor space 
  and denoted $2^\mathbb N$. 
  By \Cref{rmkMorphismsOutOfQuotient}, terms of $2^\mathbb N$ 
  correspond to set-theoretic functions $\mathbb N \to 2$. 
  We also call such functions binary sequences. 
\end{example}
\begin{example}
  The dual to $\mathbb N_{(co)fin}$ from \Cref{ExampleBAunderNinfty} is called 
  $\mathbb N_\infty$. By \Cref{rmkMorphismsOutOfQuotient}, terms of $\mathbb N_\infty$ 
  correspond to functions $\alpha: \mathbb N \to 2$ such that $\alpha(n) \wedge \alpha(m) = 0$ 
  whenever $n \neq m$. This means that $\alpha(n) = 1$ for at most one $n\in\mathbb N$. 
  There is an embedding $\mathbb N \to \mathbb N_\infty$ sending $n$ to the unique sequence $\chi_n$
  which sends $n$ to $1$. 
  There is furthermore a term $\infty:\mathbb N_\infty$ which is the sequence which is constantly $0$. 
\end{example}
\subsection{Axioms}
\begin{axiomNum}[Stone duality]
  For any countably presented Boolean algebra $B$, the evaluation map $B\rightarrow  2^{Sp(B)}$ is an isomorphism.
\end{axiomNum} 

\begin{remark}
Stone types will take over the role of affine scheme from \cite{draft}, 
and we repeat some results here. 
Analogously to Lemma 3.1.2 of \cite{draft}, 
for $X$ Stone, Stone duality tells us that $X = Sp(2^X)$. 
%
Proposition 2.2.1 of \cite{draft} now says that 
$Sp$ gives an equivalence 
\begin{equation}
   Hom_{\Boole} (A, B) = (Sp(B) \to Sp(A))
\end{equation}
Therefore $\isSt$ is a proposition.
Equivalently, 
%By Definition 4.6.1 %defines an embedding 
%of \cite{hott}, 
this means that 
$Sp$ is an embedding from $\Boole$ to any universe of types.
Its image, $\Stone$ also has a natural category structure.
By the above and Lemma 9.4.5 of \cite{hott}, 
the map $Sp$ defines a dual equivalence of categories between $\Boole$ and $\Stone$.
\end{remark}

\begin{axiomNum}[Surjections are formal Surjections]
  A map $f:Sp(B')\to Sp(B)$ is surjective iff the corresponding map $B \to B'$ is injective.
\end{axiomNum} 

\begin{lemma}\label{LemSurjectionsFormalToCompleteness}
 For $S:\Stone$, we have that $\neg \neg S \to || S ||$
\end{lemma}
\begin{proof}
  First, assume that surjections are formal surjections. 
  Let $B:\Boole$ and suppose $\neg \neg Sp(B)$. 
  %Note that if $0=1$ in $B$, then $Sp(B) =\emptyset$, meaning $\neg Sp(B)$. 
  %Therefore, we have $0\neq 1$ in $B$. 
  We will show that the map $f:2\to B$ is injective. 
  Let $f:2 \to B$, note that if $f(0) = f(1)$ then $0=1$ in $B$, 
  If $0=1$ in $B$, there are no maps $B\to 2$ preserving $0$ and $1$, thus $\neg Sp(B)$. 
  This is a contradiction with $\neg \neg Sp(B)$. Thus we may conclude that $f(0)\neq f(1)$. 
  Hence by case distinction on $2$ we can show $f$ we have that $f x = f y$ implies $ x= y$. Thus 
  $f$ is injective thus the map $Sp(B) \to Sp(2) = 1$ is surjective, thus $Sp(B)$ is merely inhabited. 
\end{proof} 
Actually, we will see in \Cref{CorDoubleNegToAx2} that the converse is also true. 

\begin{axiomNum}[Local choice]
  Whenever $X$ Stone and $E\twoheadrightarrow X$ surjective, then there is some $Y$ Stone,
    a surjection $Y \twoheadrightarrow X$ and a map $Y\to E$ such that the following diagram commutes:
    \begin{equation}\begin{tikzcd}
      E \arrow[d,""',two heads]\\
      X & \arrow[l, "", two heads] Y\arrow[lu, ""']
    \end{tikzcd}\end{equation}  
\end{axiomNum} 
\begin{axiomNum}[Dependent choice]\label{axDependentChoice}
  Given a family of types $(E_n)_{n:\mathbb N}$ and 
  a relation 
  $R_n:E_n\rightarrow E_{n+1}\rightarrow {\mathcal U}$ such that
  for all $n$ and $x:E_n$ there exists $y:E_{n+1}$ with $p:R_n~x~y$ 
  then given $x_0:E_0$ there exists
  $u:\Pi_{n:\N}E_n$ and $v:\Pi_{n:\N}R_n~(u~n)~(u~(n+1))$ and $u~0 = x_0$.
\end{axiomNum}
\subsection{Omniscience principles}
\begin{remark}\label{rmkTrivialBA}
  A Boolean algebra $B / R$ 
  is trivial iff there merely exists some $R_0\subseteq R$ finite with 
  $1 = \bigvee_{r \in R_0} r$ in $B$.
\end{remark}

\begin{theorem}[The negation of the weak limited principle of omniscience]
  It is not the case that for all $\alpha:\Noo$, we can decide whether $\alpha=\infty$.
\end{theorem}
\begin{proof}
  Suppose we could decide whether $\alpha = \infty$ for every $\alpha:\Noo$. We then have a term of type 
  $\Pi_{\alpha:\Noo} (\alpha \neq \infty) + (\alpha = \infty)$. 
  Using case distinction, we can define a map $f:\Noo \to 2$ such that 
  \begin{equation}
    f(\alpha) = \begin{cases} 0 \text{ if } \alpha \neq  \infty
    \\ 1 \text{ if } \alpha = \infty \end{cases} 
  \end{equation}
  By Stone duality, there must be some $b:B_\infty$ with 
  $f(\alpha) = 1 \iff \alpha(b) = 1$.
  $b$ is expressable using only finitely many generators $(p_n)_{n\leq N}$. 
  This means that $f$ can determine whether a sequence is constantly $0$ based on it's first $N$ entries. 
  In particular the value of $f$ on $\infty:\Noo$ and $\chi_{N+1}:\Noo$ must be equal, 
  as these sequences are equal on $(p_n)_{n\leq N}$ and thus on $b$. 
  However, $f(\infty) = 1$ and $f(\chi_N) = 0$. 
  We thus have a contradiction, and conclude we cannot decide whether $\alpha = \infty$ for general $\alpha:\Noo$. 
\end{proof}


The following result is due to David W\"arn.
\begin{theorem}[Markov's principle (uses Stone duality)]
  For $\alpha:\mathbb N_\infty$, we have that whenever $\alpha\neq \infty$, 
  there exists some $n:\mathbb N$ such that $\alpha = \chi_n$. 
\end{theorem}
\begin{proof}
  Suppose $\alpha \neq \infty$.%, then it is not the case that $\alpha(n) = 0$ for all $n:\mathbb N$. 
  We will show that $2/\{\alpha(n)|n\in\mathbb N\}$ is the trivial Boolean algebra. 
  Then by \Cref{rmkTrivialBA}, it follows there merely is a finite subset $N_0\subseteq \mathbb N$ 
  such that $\bigvee_{i:N_0}\alpha(i) =1$.
  For this finite set $N_0$, we can decide whether every $\alpha(i) =0$, 
  or there is some $i\in N_0$ such that $\alpha(i) =1$.
  As the join of these $\alpha(i)$ is $1$, we must have that there is some $n\in N_0$ with $\alpha(n) = 1$. 
  As $\alpha(j)=1$ for at most one $j$, we conclude that $\alpha = \chi_n$. 
  %

  To show that $2/\{\alpha(n)|n\in\mathbb N\}$ is trivial, we will show it has an empty spectrum. 
  Suppose $x: 2 \to 2$ is such that $x(\alpha(n)) = 0$ for every $n:\mathbb N$. 
  As $x(1) = 1\neq 0$, we must have for every $n:\mathbb N$ that $\alpha(n) \neq 1$. 
  Thus for every $n:\mathbb N$, we have $\alpha(n) = 0$, contradicting our assumption that $\alpha \neq \infty$. 
  Thus any such $x$ cannot exist and $Sp(2/\{\alpha(n)|n\in\mathbb N\}) = \emptyset$. 
  Thus $2/\{\alpha(n) | n \in \Noo \} = 2^\emptyset$, which is the trivial Boolean algebra. 
\end{proof}
\begin{corollary}\label{LemDecidableSubsetsDeMorgan}
  For $(A_n)$ a family of decidable subsets, we have
    $
    (\bigcup_{n:\mathbb N} A_n)^C
    =
    \bigcap_{n:\mathbb N} (A_n^C)
    $ 
    and 
    $
    \bigcup_{n:\mathbb N} (A_n^C)
    =  
    (\bigcap_{n:\mathbb N} A_n)^C
    $
\end{corollary}
\begin{proof}
\begin{itemize}
  \item 
    Let 
    $x\notin \bigcup_{n:\mathbb N} A_n$. 
    Then for every $n:\mathbb N$, we cannot have $x\in A_n$
    and thus $x\in A_n^C$ by decidability of $A_n$. 
    Thus $x\in \bigcap_{n:\mathbb N} (A_n^C)$. 
    Therefore
    $$
    (\bigcup_{n:\mathbb N} A_n)^C
    \subseteq 
    \bigcap_{n:\mathbb N} (A_n^C).
    $$ 
  \item 
    Suppose that for every $n:\mathbb N$, we have $x\notin A_n$. 
    There does not exist an $n:\mathbb N$ with $x\in A_n$. Thus
    $$
    \bigcap_{n:\mathbb N} (A_n^C)
    \subseteq 
    (\bigcup_{n:\mathbb N} A_n)^C
    $$ 
  \item 
    Suppose there exists some $n$ with $x\in A_n^C$. Then 
    it cannot be the case that $x\in A_m$ for all $m:\mathbb N$.
    Thus
    $$
    \bigcup_{n:\mathbb N} (A_n^C)
    \subseteq 
    (\bigcap_{n:\mathbb N} A_n)^C
    $$ 
  \item 
    Suppose that $x\in (\bigcap_{n:\mathbb N} A_n)^C$. 
    Then define the binary sequence $\alpha$ by $\alpha(i) =1$ iff $i$ is the first index such that 
    $x\notin A_i$. This is well-defined as $A_n$ is decidable for all $n:\mathbb N$. 
    If $\alpha(i) = 0$ for all $i$, then $x\in A_i$ for all $i$. 
%    and it is the case that $x\in A_n$ for all $n:\mathbb N$. 
    Thus under our assumption $x\in (\bigcap_{n:\mathbb N} A_n)^C$, 
    we cannot have that $\alpha(i) = 0$ always. 
    By Markov, there then exists an $i$ such that $\alpha(i) = 1$. 
    Thus $x\notin A_i$ for some $i$. We conclude that. 
    $$
    (\bigcap_{n:\mathbb N} A_n)^C
    \subseteq
    \bigcup_{n:\mathbb N} (A_n^C)
    $$ 
\end{itemize}
\end{proof} 
Note that we only needed decidability for the first and last bullet point, 
and only the last bullet point used countability (and of course Markov's principle). 

\begin{theorem}[The lesser limited principle of omniscience (LLPO)]
  For $\alpha:\mathbb N_\infty$, 
  we have that 
  \begin{equation}\label{eqnLLPO}
    \forall_{k:\mathbb N} \alpha(2k) = 0  \vee \forall_{k:\mathbb N} \alpha(2k+1) = 0
  \end{equation}
\end{theorem}
\begin{proof}
  We first will define a map $f:B_\infty \to B_\infty \times B_\infty$. 
  Because of \Cref{rmkMorphismsOutOfQuotient}, it is sufficient to define $f$ on $(p_n)_{n:\mathbb N}$ with 
  $f(p_n) \wedge f(p_m) = (0,0)$ for $n\neq m$. 
  To define $f(p_n)$, we use a case distinction on whether $n$ is odd or even. 
  \begin{equation}
    f(p_n) =\begin{cases}
      (p_k,0) \text{ if } n = 2k\\
      (0,p_k) \text{ if } n = 2k+1\\
    \end{cases}
  \end{equation}
  By making a case distinction on $n,m$ being odd or even, 
  we can indeed see that $f(p_n) \wedge f(p_m) = (0,0)$ when $n\neq m$. 
  We will show that $f$ is injective. 
  If it is, then as surjections are formal surjections, $f$ corresponds to a surjection $s:\Noo + \Noo \to \Noo$.
  This means that for every $\alpha:\Noo$ there merely is some $x:\Noo + \Noo$ such that $s(x) = \alpha$. 
  Now we want to show that \Cref{eqnLLPO} holds for $\alpha$. 
  As we need to show a propositional truncation, we can assume $x$ is given and make a case distinction whether 
  it is of the form $inl(\beta)$ or $inr(\beta)$ for some $\beta:\Noo$. 
  If $x = inl(\beta)$, then for any $k:\mathbb N$, we have that 
  $s(x) (p_{2k+1}) = x(f(p_{2k+1})) = inl(\beta) (0,p_k)  = \beta(0) = 0$. 
  And if $x=inr(\beta)$, by similar argument it follows that for all $k:\mathbb N$
  we must have $s(x)(p_{2k}) = 0$. 
  Thus if we can show that $f$ is injective, we have \Cref{eqnLLPO}.
  
  To see that $f$ is injective, it is sufficient to show that whenever $f(x) = 0$, we must have $x = 0$. 
  Suppose $f(x) = 0$. %We make a case distinction on whether $x$ is finite or cofinite. 
  We make a case distinction on whether $x$ corresponds
  to a finite or cofinite subsets of $\mathbb N$. 
  We'll denote $E,O\subseteq \mathbb N$ for the even and odd numbers.
  \begin{itemize}
    \item Let $x$ correspond to a finite subset of $\mathbb N$. Write 
      $x = \bigvee_{i\in I_0} p_i$ for $I_0\subseteq \mathbb N$ finite. 
      Then $f(x) = (\bigvee_{i\in I_0 \cap E } p_{\frac i2} , \bigvee_{i\in I_0 \cap O } p_{\frac {i-1}2} ) = (0,0)$
      But now as $p_j\neq 0$ for all $j\in\mathbb N$, we must have $I_0 \cap E = \emptyset = I_0 \cap O$. 
      Thus $I_0= \emptyset$, and $x = 0$. 
    \item Let $x$ correspond to a cofinite subset of $\mathbb N$. Write 
      $x = \bigwedge_{j\in J} \neg p_j$ for $J$ finite. 
      We will derive a contradiction, from which we can conclude that $x=0$ after all. 
      Then $f(x) = (\bigwedge_{j\in J \cap E } \neg p_j , \bigwedge_{j\in J \cap O } \neg p_j )$. 
      As $f(x) = (0,0)$, we have that 
      $\bigwedge_{j\in J \cap E } \neg p_j =0$ and
      $\bigwedge_{j\in J \cap O } \neg p_j  = 0$.
      However, any finite meet of negations will correspond to a cofinite set,
      in particular not to the empty set, giving a contradiction. 
      So $x = 0$. 
  \end{itemize}
  Therefore whenever $f(x) = 0$, we have $x = 0$, thus $f$ is injective, $s$ is surjective 
  and by the above reasoning for any $\alpha:\Noo$ we have \Cref{eqnLLPO}. 
\end{proof}
\begin{remark}
  The above function $f$ does not have a retraction. 
\begin{proof}
  Suppose it does, call the retraction $r:B_\infty \times B_\infty \to B_\infty$. 
  Consider $r(0,1)$. 
  As $r(0,1):B_\infty$, we merely have an representative in $\langle (p_k)_{k:\mathbb N}\rangle$ 
  expressing $r(0,1)$. This expression contains at most finitely many $p_k$. 
  Let $N$ be such that $r(0,1)$ does not contatin $p_k$ for $k \geq N$. 

  As maps of Boolean algebras are order-preserving, we have that 
  $r(0,1) \geq r(0,p_l) = p_{2l+1}$ for any $l:\mathbb N$. 
  So $r(0,1) \geq p_{k}$ for some $k \geq N$. 

  $r(0,1)$ cannot be finite, if it were, we'd have that  
  $r(0,1) = \bigvee_{i\in I_0} p_i$ with $k\notin I_0$, 
  but then $r(0,1) \wedge p_k = 0\neq p_k$ and $r(0,1) \not \geq p_k$.
  Thus $r(0,1)$ is cofinite. 
  By similar reasoning $r(1,0)$ is cofinite. 
  But now $r(0,1) \wedge r(1,0)$ is cofinite, but this should equal $r((1,0)\wedge (0,1)) = r(0,0) = 0$. 
  As $0$ is finite and $r(0,1)$ is cofinite, we have a contradiction. 

  Thus there does not exist a retraction of $f$. 
\end{proof}
\end{remark} 

\begin{corollary}\label{corAlternativeLLPO}
  LLPO is equivalent to the following statement:
  Let $(\phi_n)_{n:\mathbb N}, (\psi_m)_{m:\mathbb N}$ be families of decidable propositions indexed over $\mathbb N$.
  We then have 
  \begin{equation}
    (\forall_{n:\mathbb N} \forall_{m:\mathbb N} (\phi_n \vee \psi_m) )
    \leftrightarrow
    ((\forall_{n:\mathbb N} \phi_n) \vee (\forall_{m:\mathbb N} \psi_m) )
  \end{equation}
\end{corollary}
\begin{proof}
  Note that the implication from right to left in the above equation always holds
  

  \rednote{While \cite{ReverseMathsBishop} does mention LLPO, it doesn't mention this specific equivalence, 
  and there should be a reference for this result}
  Assume LLPO
  Assume that for every $n,m:\mathbb N$ we have $\phi_n \vee \psi_m$. 
  We will define a binary sequence $\beta$ and show that $\beta(n)$ is $1$ at most once. 
  First we define a binary sequence $\alpha$
  such that $\alpha(2n) = 0$ iff $\phi_n$ holds and $\alpha(2m+1) = 0$ iff $\psi_m$ holds. 
  We let $\beta(k) = 1$ iff $k$ is minimal with $\alpha(k) = 1$. 
  By this minimality, we clearly have that $\beta$ is $1$ at most once and thus defines a term of $\Noo$. 
  By LLPO, we have that 
  $\forall_{k:\mathbb N} \beta(2k+1) = 0\vee \forall_{k:\mathbb N} \beta(2k+1) = 0$. 
  As we're proving a proposition, we can unpack this truncation, and make a case distinction on
  $\forall_{k:\mathbb N} \beta(2k+1) = 0 + \forall_{k:\mathbb N} \beta(2k+1) = 0$. 

  Assume that $\forall_{k:\mathbb N} \beta(2k+1) = 0$. 
  Let now $m:\mathbb N$. We claim that $\psi_m$ will hold. 
  Suppose $\neg \psi_m$, then $\alpha(2m+1) = 1$. However, as $\beta(2m+1) = 0$, there 
  $k = 2m+1$ is not minimal such that $\alpha(k)  = 1$.
  There is thus some  some $l\leq 2k+1$ with $\alpha(l) = 1$. 
  As $\beta(2k+1) = 0$ for all $k$, we have that $l$ must be even. 
  Therefore $\alpha(2n) = 1$ for some $n:\mathbb N$, meaning that $\neg \phi_n$. 
  But now $\neg \phi_n \wedge \neg \psi_m$, which contradicts 
  $(\forall_{n:\mathbb N} \forall_{m:\mathbb N} (\phi_n \vee \psi_m) )$. 
  We thus have that $\neg \psi_m$ is false. As $\psi_m$ is decidable, we conclude $\psi_m$. 
  Thus for all $m:\mathbb N$, we have $\psi_m$. Thus 
  $((\forall_{n:\mathbb N} \phi_n) \vee (\forall_{m:\mathbb N} \psi_m) )$ 
  as required. 

  The other case is symmetric. We conclude that 
  \begin{equation}
    (\forall_{n:\mathbb N} \forall_{m:\mathbb N} (\phi_n \vee \psi_m) )
    \to
    ((\forall_{n:\mathbb N} \phi_n) \vee (\forall_{m:\mathbb N} \psi_m) )
  \end{equation}

  Conversely, assume the above equation holds. 
  Given any sequence $\alpha:\Noo$, because it is $1$ at most once, we have that 
  $\alpha(n) = 0 \vee \alpha(n+1) = 0$ for all $n:\mathbb N$. Applying the above equation then gives LLPO for $\alpha$. 
\end{proof}
\subsection{Topology}
\rednote{Active}
\begin{definition}
  Let $P$ be a proposition. 
  \begin{itemize}
    \item $P$ is decidable if $P + \neg P$
    \item $P$ is open if there merely exists some $\alpha:\Noo$ such that $P \leftrightarrow \alpha \neq \infty$. 
    \item $P$ is closed if there merely exists some $\alpha:\Noo$ such that $P \leftrightarrow \alpha = \infty$. 
  \end{itemize}
\end{definition}

\begin{remark}
  Let us state some alternative definitions of open and closed:
  \begin{itemize}
    \item 
      By Markov, $P$ being open iff there merely exists some $\alpha:\Noo$ such that 
      $P\leftrightarrow \exists_{n:\mathbb N} \alpha(n) = 1$. 
    \item 
      $P$ is open iff there merely exists some $\alpha:2^\mathbb N$ such that $P\leftrightarrow \exists n \alpha(n) = 1$.
    \item 
      $P$ is closed iff there merely exists some $\alpha:2^\mathbb N$ such that $P\leftrightarrow \forall n \alpha(n) = 0$.
\end{itemize}
\end{remark}

\begin{remark}\label{rmkOpenClosedNegation}
  The negation of an open proposition is a closed proposition.
  Furthermore, both open and closed propositions are double negation stable by Markov, 
  meaning that for $P$ open or closed, we have $\neg \neg P \to P$.
\end{remark}

\begin{lemma}\label{lemOpenClosedDisjunctionConjunction}
  Closed propositions are closed under countable $\Pi$-types and countable conjunctions.
  Open propositions are closed under countable $\Sigma$-types and finite conjunctions. 
\end{lemma}
\begin{proof}
  \begin{itemize}
    \item Let $P,Q$ be closed. We will show $P\vee Q$ closed as well. 
      Assume $P\leftrightarrow \alpha = \infty, Q \leftrightarrow \beta = \infty$. 
      Now $P\vee Q \leftrightarrow ((\forall_{n:\mathbb N} \alpha(n) = 0) \vee (\forall_{n:\mathbb N} \beta(n) = 0))$
      By \Cref{corAlternativeLLPO}, we have that 
      $P\vee Q \leftrightarrow \forall_{n:\mathbb N}  \forall_{m:\mathbb N}  (\alpha(n) = 0 \vee \beta(m) = 0)$
      For $s:\mathbb N \to \mathbb N \times \mathbb N$ surjective, this in turn is equivalent to 
      $\forall_{k:\mathbb N}  (\alpha(\pi_0(s(k))) = 0 \vee \beta(\pi_1(s_k)) = 0)$. 
      Define $\gamma(k) = 0$ iff $(\alpha(\pi_0(s(k))) = 0 \vee \beta(\pi_1(s_k)) = 0)$. 
      As this is decidable, $\gamma$ is well-defined as binary sequence, and 
      $P\vee Q \leftrightarrow \forall_{k:\mathbb N} \gamma(k) = 0$, thus $P\vee Q$ is closed. 
    \item Let $(P_n)_{n:\mathbb N}$ be a countable family of closed propositions. We will show that 
      $\forall_{n:\mathbb N} P_n$ is closed as well. 
      By \rednote{countable choice}, we have a function which for each $n:\mathbb N$ gives some $\alpha_n:\Noo$ 
      such that $P_n \leftrightarrow \alpha_n =\infty$. 
      Thus $(\forall_{n:\mathbb N} P_n )\leftrightarrow (\forall_{n:\mathbb N}(\alpha_n = \infty))$. 
      Let $s:\mathbb N \to \mathbb N \times \mathbb N$ surjective and let 
      $\beta(k) = \alpha_{\pi_0s(k)}(\pi_1 s(k))$. 
      Note that $\beta = \infty$ iff for every $n:\mathbb N$, and every $m:\mathbb N$, we have $\alpha_n(m) = 0$, 
      which happens iff $\alpha_n = \infty$. 
      We conclude that $\forall_{n:\mathbb N} P_n \leftrightarrow \beta = \infty$.
      Thus closed propositions are closed under countable conjunctions. 
    \item 
      Note that $\neg (P \vee Q) \leftrightarrow \neg P \wedge \neg Q$. 
      If $P,Q$ are open, then $\neg P, \neg Q$ are closed, and their conjunction is closed by the above. 
      Thus $\neg (P\vee Q)$ is closed, hence $P\vee Q$ is open. 
    \item Let $(P_n)_{n:\mathbb N}$ be a countable family of open propositions. 
      We will show that $\exists_{n:\mathbb N}$ is open as well. 
      By \rednote{countable choice}, we have a function which for each $n:\mathbb N$ gives some $\alpha_n:\Noo$ 
      such that $P_n \leftrightarrow \alpha_n \neq \infty$. 
      Thus $(\exists{n:\mathbb N} P_n) \leftrightarrow (\exists{n:\mathbb N}\alpha_n \neq \infty)$. 
      Let $s:\mathbb N \to \mathbb N \times \mathbb N$ surjective and let 
      $\beta(k) = \alpha_{\pi_0s(k)}(\pi_1 s(k))$. 
      Then let $\gamma(k) = 1$ iff $k$ is minimal with $\beta(k) = 1$. 
      Note that $\beta \neq \infty$ iff there is some $n:\mathbb N$, and some $m:\mathbb N$, with $\alpha_n(m) = 1$, 
      which happens iff $\alpha_n \neq \infty$. 
      We conclude that $\exists{n:\mathbb N} P_n \leftrightarrow \gamma \neq \infty$.
      Thus open propositions are closed under countable disjunctions. 
  \end{itemize}   
\end{proof}


\begin{lemma}
  For $P$ open and $Q$ closed or $P$ closed and $Q$ open, we have that $(P\to Q) \leftrightarrow (\neg P \vee Q)$. 
\end{lemma}
\begin{proof}
  Without losing generality, assume $P$ open and $Q$ closed. 
  The left direction is well-known. 
  What is also well-known is 
  $(P \to Q) \to \neg \neg (\neg P \vee Q)$.
  By \Cref{rmkOpenClosedNegation}, $\neg P$ is closed. 
  By \Cref{lemOpenClosedDisjunctionConjunction}, it follows that $\neg P \vee Q$ is closed. 
  By \Cref{rmkOpenClosedNegation} it follows that $\neg \neg (\neg P \vee Q) \to (\neg P \vee Q)$. 
  Thus $(P \to Q) \to (\neg P \vee Q)$ as required. 
\end{proof}
 


\begin{lemma}
  Equality in $B:\Boole$ is open and equality in $X:\Stone$ is closed. 
\end{lemma}
\begin{proof}
  Let $B:\Boole$, and let $a,b:B$. We will show that $a=b$ is closed. 
  Recall that $B = \langle G \rangle / \langle R \rangle$. 
  And that $(a=b):= (a-b \in \langle R \rangle)$, 
  where $\langle R \rangle$ is the set of expressions of the form
  $(\bigvee_{r\in R_0} r) \wedge b$ for $R_0\subseteq R$ finite and $b:\langle G \rangle$ 
  arbitrary. 
  In particular, note that $\langle R \rangle$ is countable. 
  Let $(r_n)_{n:\mathbb N}$ enumerate $ \langle R \rangle $
  Also note that equality in $\langle G \rangle$ is decidable. 
  Thus we can define a sequence $\alpha:\Noo$ with $\alpha(k) = 1$ iff $k$ is minimal with $a -b = r_n$. 
  Now $\alpha \neq \infty $ iff $a = b$. Thus $a=b$ is an open proposition. 

  Let $X:\Stone$ and let $x,y:X$. 
  Let $X= Sp(B)$ for $B = \langle G \rangle / \langle R \rangle $. 
  Note that $x=y$ iff $x(g) = y(g)$ for all generators $g:G$. 
  As $G$ is countable, we can enumerate it's elements as $(g_n)_{n:\mathbb N}$. 
  As equality in $B$ decidable, we can define a sequence $\alpha:\Noo$ 
  with $\alpha(k) = 1$ iff $k$ is minimal with $x(g_k) \neq y(g_k)$. 
  Now $\alpha(k) = 0$ for all $k$ iff for all $k$ we have $x(g_k) = y(g_k)$ and thus $x(g) = y(g)$ for all generators $g$ 
  and thus $ x= y$. 
  Thus $(x=y) \leftrightarrow \alpha= \infty$ and $ x=y$ is closed. 
\end{proof}


\begin{definition}
  For $S$ a set and $D\subseteq S$ an arbitrary subset, we call $D$ decidable/open/closed 
  iff for every $x:S$, $D(x)$ is decidable/open/closed. 
\end{definition}

\begin{lemma}
  Let $D\subseteq Sp(B)$. TFAE
  \begin{enumerate}[(i)]
    \item $D$ is decidable. 
    \item There merely exists some $b\in B$ such that $D = \{\alpha:Sp(B) | \alpha(b) = 1\}$. 
    \item There merely is a finite subset $G_0$ of the generators of $B$ such that 
      whenever $\alpha,\beta:Sp(B)$ are such that 
      $\alpha(g) = \beta(g)$ for all $g:G_0$ and $D(\alpha)$, we also have $D(\beta)$. 
    \item $D$ is both open and closed. 
  \end{enumerate}
\end{lemma}
\begin{proof}
  \begin{itemize}
    \item $(i) \to (ii)$.
      First, remark that decidable subsets $D$ give terms $f_D:2^{Sp(B)}$, by sending 
      $f(\alpha) = 1$ if $D(\alpha)$ and $f(\alpha) = 0$ if $\neg D(\alpha)$. 
      Stone duality then gives that this term corresponds to evaluation at some $b:B$. 
      Therefore $D(\alpha) \iff f(\alpha) = 1 \iff  \alpha(b) = 1$, as required. 
    \item $(ii) \to (1)$
      To decide $D(\alpha)$ is the same as to evaluate whether $\alpha(b) = 1$, which is decidable. 
    \item $(ii) \to (iii)$. If there exists a $b:B$ such that $D$ is the indicator of $b$ in the above way, 
      then we need only consider the generators occuring in a finite expression of $b$. 
    \item $(iii) \to (i)$. Because being decidable is a proposition, we can untruncate our assumption. 
      Suppose $D$ only depends on $n$ generators, then we need to only check the values 
  \end{itemize}
\end{proof}
\begin{lemma}
  Let $A\subseteq S$ be a subset of a Stone space. TFAE
  \begin{itemize}
    \item $A$ is closed. 
    \item There merely is a countable set set of decidable subsets $D_n\subseteq S$ with 
      $A = \bigcap_{n:\mathbb N} D_n$. 
    \item $\neg A$ is open. 
%    \item There merely is a countable set set of decidable subsets $E_n\subseteq S$ with 
%      $\neg A = \bigcup_{n:\mathbb N} D_n$. 
  \end{itemize}
\end{lemma}

\begin{corollary}\label{CorDoubleNegToAx2}
  If we have for $S:\Stone$ that $\neg \neg S \to ||S||$, 
  we can conclude that surjections are formal surjections. 
\end{corollary}
\begin{proof}
  \rednote{All the unkonw references are lemmas that still need to be done (or at least linked) without using ax2}
  Now assume that for $S:\Stone$, we have $\neg \neg S \to ||S||$. 
  Let $f:B \to B'$ be injective, and let 
  $g:S' \to S$ correspond to $f$. 
  To show that $g$ is surjective, we need to show that for any $x:S$, we have $||\sum\limits_{y:S'} g y = x||$. 
  By our assumption, we can prove this by showing that 
  \begin{equation}
    g^{-1}\{x\} := \sum\limits_{y:S'} g y = x
  \end{equation} is Stone and non-empty. 
    First note that $\sum\limits_{y:S'} g y = x$ is the pullback of $g$ and the map $1 \to S$ given by $x$.
      By duality and \Cref{rmkBoolePushouts} the pullback of Stone spaces is a Stone space. 
   

      By \Cref{lemPointsAreClosed} and \Cref{lemClosedIntersectionOfDecidable}, 
      $\{x\}$ is a countable intersection of decidable sets $\bigcap_{n:\mathbb N} D_n$. 
      We have that 
      \begin{equation}
        g^{-1}\{x\} = g^{-1}(\bigcap_{n:\mathbb N} D_n) = 
        \bigcap_{n:\mathbb N} g^{-1}(D_n)
      \end{equation} 
      Let $n:\mathbb N$. By Stone duality, the decidable subset $D_n$ corresponds to some $b_n:B$. 
      As $x\in D_n$, we have that $b_n \neq 0$. 
      As $f$ is injective, we have that $f(b_n) \neq 0$. 
      Therefore the corresponding decidable subset, which is $g^{-1}(D_n)$ is not empty. 

      Note that $D_n$ is a closed subset of a Stone space, hence Stone by \Cref{LemClosedSubsetOfStoneIsStone}, 
      and by the same argument as above $g^{-1}(D_n)$ is Stone. By our assumption $g^{-1}(D_n)$ is thus merely inhabited. 
      Now by \Cref{lemCompactnessCountableIntersection}, this gives that 
      $\bigcap_{n:\mathbb N} g^{-1}(D_n)$ is merely inhabited, as required. 
\end{proof}


\rednote{End of activity}

\subsection{Compact Hausdorff spaces}
\subsection{Analysis}

\section{Directed Univalence}
\subsection{Subquotient systems}

\begin{definition}
A subquotient pre-system consists of $X$ a type and $U$ a class of propositions.
\end{definition}

\begin{definition}
For $(X,U)$ a subquotient pre-system, we define:
\[Sub_{X,U} = \sum_{Y:\Type} \exists (P : X\to U).\ Y = \Sigma_XP\]
\[SubQ_{X,U} = \sum_{Y:\Type} \exists (P : X\to U)(R: \Sigma_XP\to \Sigma_XP\to U\ \mathrm{equivalence\ relation}).\ Y = (\Sigma_XP)/R\]
\end{definition}

\begin{definition}
We say a class of types $T$ has local choice if for all $X\in T$ and $P:X\to\Type$ such that:
\[\prod_{x:X}\propTrunc{P(x)}\]
there merely exists $Y\in T$ and a surjection:
\[f:Y\to X\]
such that:
\[\prod_{y:Y}P(f(y))\]
\end{definition}

\begin{proposition}\label{lex-sub-pro}
Assume $(X,U)$ a Subquotient pre-system such that:
\begin{itemize}
\item Identity types in $X$ are in $U$.
\item $U$ is closed by $\sum$ and $\top$.
\item $\propTrunc{X}$ and $X\times X = X$.
\item $Sub_{X,U}$ has local choice.
\end{itemize}
Then we $SubQ_{X,U}$ has the following:
\begin{itemize}
\item Stability under finite limits.
\item Stability by quotient by equivalence relation with value in $U$.
\item Local choice.
\end{itemize}
\end{proposition}

\begin{proof}
\begin{itemize}
\item First we check that $SubQ_{X,U}$ has local choice. Since we assume that $Sub_{X.U}$ has local choice and that any type in $SubQ_{X,U}$ is covered by a type in $Sub_{X,U}$ by definition, it is enough to check that $Sub_{X,U}\subset SubQ_{X,U}$ to conclude. But given $S = \Sigma_XP$ in $Sub_{X,U}$ we have that:
\[\Sigma_XP = (\Sigma_XP)/L\]
where:
\[L((x,_),(y,_))= (x=_Xy)\]
and since $x=_Xy$ is assumed to be in $U$ we conclude.

\item Stability by quotient by equivalence relation with value in $U$ is clear.

\item Now we want to check stability under finite limits.

First we check that $U\subset SubQ_{X,U}$. Indeed assume $P\in U$, then with $L$ the trivial relation we have:
\[(X\times P) / L = \propTrunc{X\times P} = P\]
as $\propTrunc{X} = 1$ so that since $\top\in U$ we conclude $P\in SubQ_{X,U}$.

This means that $SubQ_{X,U}$ is stable by identity type, and that $1\in SubQ_{X,U}$.

All that is left is to check stability under $\Sigma$. Assume $S: SubQ_{X,U}$ and $T:S\to SubQ_{X,U}$. Through the fact that $S$ is covered by a type in $Sub_{X,U}$ and local choice for $Sub_{X,U}$ we merely get $S':Sub_{X,U}$, say $S'=\Sigma_XP$ and a surjective map:
\[f:S'\to S\]
such that for all $x:\sum_XP$ we have:
\[T(f(x)) = (\Sigma_XP_x)/R_x\]
so we get a surjective map:
\[\sum_{x:X}\sum_{P(x)}(\Sigma_X P_x)/R_x  \to \sum_ST\]
Then the identity types in $\sum_ST$ are in $U$ as $U$ is stable by $\Sigma$, so it is enough to check that:
\[\sum_{x:X}\sum_{P(x)}(\Sigma_X P_x)/R_x\]
is in $SubQ_{X,U}$ to conclude, as we can then apply the previous bullet-point. But this type is equivalent to:
\[\left(\sum_{(x,x'):X\times X}\Sigma_{P(x)}P_x(x')\right)/ L\]
where:
\[L((x,x'),(y,y')) =\sum_{x=y} R_y(x',y') \]
which is in $SubQ_{X,U}$ as $U$ is stable by $\Sigma$, $x=y$ in in $U$ and $X\times X = X$.
\end{itemize}
\end{proof}

\begin{proposition}\label{coproducts-sub-quo}
Assume $(X,U)$ a subquotient pre-system such that $\bot\in U$ and $X+X = X$. Then $SubQ_{X,U}$ is stable by finite coproducts.
\end{proposition}

\begin{proof}
We have that:
\[\bot = (X\times \bot) / L\]
where $L$ is the unique such equivalence relation. Since $\bot\in U$ we conclude that $\bot\in SubQ_{X,U}$.

Given $S$ and $S'$ in $SubQ_{X,U}$, say:
\[S = (\sigma_XP)/R\]
\[S' = (\sigma_XP')/R'\]
Then we have that:
\[S+S' = \left(\sum_{X+X}[P,P']\right) L\]
where:
\[[P,P'](l(x)) = P(x)\]
\[[P,P'](r(x)) = P'(x)\]
and:
\[L(l(x),l(y)) = R(x,y)\]
\[L(l(x),r(y)) = \bot\]
\[L(r(x),l(y)) = \bot\]
\[L(r(x),r(y)) = R'(x,y)\]
Since $\bot\in U$ and $X+X=X$ we conclude that $S+S'$ is in $SubQ_{X,U}$.
\end{proof}

\begin{proposition}\label{prop-sub-quo}
Assume $(X,U)$ a subquotient pre-system such that $\top\in U$ and for all $S\in Sub_{X,U}$ we have that $\propTrunc{S}\in U$. Then any proposition in $SubQ_{X,U}$ is in $U$. 
\end{proposition}

\begin{proof}
If we have a proposition $S$ in $SubQ_{X,U}$, say:
\[S = (\Sigma_XP)/R\]
then:
\[S = \propTrunc{S} = \propTrunc{\Sigma_XP}\]
and we can conclude.
\end{proof}

\begin{definition}
A subquotient system is a subquotient pre-system obeying the hypothesis of \cref{lex-sub-pro}, \cref{coproducts-sub-quo} and \cref{prop-sub-quo}.
\end{definition}

We just pack all this up in one theorem:

\begin{theorem}\label{stabitity-sub-quo}
Let $(X,U)$ be a subquotient system, then $SubQ_{X,U}$ has the following:
\begin{itemize}
\item Stability under finite limits.
\item Stability under finite coproducts.
\item Stability under quotient by equivalence relations.
\item Local choice.
\end{itemize}
\end{theorem}

We have two main examples in mind.

\begin{example}
The subquotient pre-system $St = (2^\N,\mathrm{Closed})$ is a quotient system. We have that $Sub_{St}$ is the type of stone spaces, and $CHaus = SubQ_{St}$ the type of compact Haussdorf spaces.

Closed propositions are stable by $\Sigma$. TODO 

We also need that for any stone space $S$ we have that $\propTrunc{S}$ is a closed proposition. TODO
\end{example}

\begin{example}
The subquotient pre-system $Od = (\N,\mathrm{Open})$ is a quotient system. We have that $ODisc = SubQ_{Od}$ the type of so-called overtly discrete types.

A key observation is that open propositions are stable by countable disjunctions.

This means open propositions are stable by $\sum$ because we can assume:
\[P = \Sigma_{n:\N} A(n)\]
with $A(n)$ decidable and:
\[Q:P \to \mathrm{Open}\]
Then we have that:
\[\Sigma_PQ = \exists(n:\N).\ \Sigma_{A(n)} Q(n)\]
which is open as $\Sigma_{A(n)} Q(n)$ is open for all $n$, as $A(n)$ is decidable.

Types in $Sub_{Od}$ even have full choice because both $\N$ and decidable propositions have full choice.
\end{example}

So both $ODisc$ and $CHaus$ enjoys the conclusion of \cref{stabitity-sub-quo}.


\subsection{Tychonov}

\begin{proposition}\label{tychonov}
Assume $(X,U)$ and $(Y,C)$ two subquotient system such that:
\begin{itemize}
\item $S\to Y$ is in $SubQ_{Y,C}$ for all $S:Sub_{X,U}$.
\item If $P\in U$ and $Q:P\to C$ then $\prod_{p:P}Q(p) \in C$.
\item If $Q:X\to C$ then $\prod_{x:X}Q(x) \in C$.
\end{itemize}
Then we have the following:
\begin{itemize}
\item If $S:SubQ_{X,U}$ and $T:S\to SubQ_{Y,C}$, then:
\[\prod_{s:S}T(s) \in SubQ_{Y,C}\]
\end{itemize}
\end{proposition}

\begin{proof}
Note that for $S':Sub_{X,U}$ and $Q:S'\to C$ we have that:
\[\prod_{S'}Q\]
is in $C$.

Now we use local choice to get $S':Sub_{X,U}$ with a surjective map:
\[f:S'\to S\]
such that for all $s:S'$ we have:
\[T(f(s)) = (\Sigma_YU_s)/R_s\]

Then the map:
\[\prod_ST \to \prod_{s:S'}(\Sigma_YU_s)/R_s\]
is an embedding, its fiber over $\alpha$ is:
\[\prod_{s,t:S'} \prod_{f(s) =_S f(t)} \alpha(s) = \alpha(t)\]
which is in $C$ by the hypothesis. Therefore it is enough to prove that:
\[\prod_{s:S'}(\Sigma_YP_s)/R_s\]
is in $SubQ_{Y,C}$. 

But this type is the quotient of:
\[\prod_{s:S'}(\Sigma_YP_s)\]
by:
\[L(\alpha,\beta) = \prod_{s:S'} R_s(\alpha(s),\beta(s))\]
which is in $C$, therefore it is enough to to prove that:
\[\prod_{s:S'}(\Sigma_YP_s)\]
is in $SubQ_{Y,C}$.

But this type is equivalent to:
\[\sum_{f:S'\to Y} \prod_{s:S'}P_s(f(s))\]
Since $\prod_{s:S'}P_s(f(s))$ is in $C$, it is enough to prove that:
\[S'\to Y\]
is in $SubQ_{Y,C}$. But this is one of the hypothesis.
\end{proof}

\begin{definition}
Two subquotient systems $A,B$ are called dual if both $(A,B)$ and $(B,A)$ satisfy the hypothesis of \cref{tychonov}.
\end{definition}

\begin{example}
We have that $St = (2^\N,\mathrm{Closed})$ and $Od = (\N,\mathrm{Open})$ are dual quotient systems.

\begin{itemize}
\item We need that if $P$ open and $Q:P\to \mathrm{Closed}$, then $\Sigma_PQ$ is closed. Assume $Q=\Sigma_{n:\N}A(n)$ with $A(n)$ decidable, since open propositions have choice we can assume for $n:\N$ such that $A(n)$ that $Q(n) = \forall_{k:\N} B_n(k)$ with $B_n(k)$ decidable. Then:
\[\Sigma_PQ = \prod_{n,k:\N} \prod_{A(n)} B_n(k) \]
which is indeed closed.

It is clear that closed propositions are closed by countable products.

$\Sigma_\N P\to 2^\N$ is compact Hausdorff? Yes indeed, it is even Stone because it is equivalent to:
\[\prod_{k,n:\N} 2^{P(n)}\]
and $2^{P(n)}$ is Stone as $P(n)$ is open, indeed:
\[(\Sigma_\N A)\to 2\] 
for $A$ decidable is a countable product of Stone space, as $2^A$ is Stone for $A$ decidable.

\item First we check that given $S$ Stone, we have that:
\[S\to \N\]
is overtly discrete. Indeed identity types in $S\to \N$ are closed and there is a surjection from fundamental systems of idempotent in $2^S$ to $S\to \N$, so it is enough to prove that the type fundamental systems of idempotent in a c.p. algebra is overtly discrete. To have this it is enough to prove that countably presented algebra are overtly discrete. TODO
\end{itemize}
\end{example}

When applying \cref{tychonov} to $ODisc$ and $CHaus$ we get Tychonov theorem and its dual.


\subsection{Factorisation}



\subsection{Scott continuity}



%\chapter{Axioms}
%\chapter{Stone Spaces}
%\chapter{Open and closed propositions}
%



\appendix
\rednote{The following is not put in order}
\section*{Introduction}
This draft is empty so far.
%

\section{Omniscience principles}
\begin{lemma}\label{LemDecidableSubsetsDeMorgan}
  For $(A_n)$ a family of decidable subsets, we have
    $
    (\bigcup_{n:\N} A_n)^C
    =
    \bigcap_{n:\N} (A_n^C)
    $ 
    and 
    $
    \bigcup_{n:\N} (A_n^C)
    =  
    (\bigcap_{n:\N} A_n)^C
    $
\end{lemma}

\begin{proof}
\begin{itemize}
  \item 
    Let 
    $x\notin \bigcup_{n:\N} A_n$. 
    Then for every $n:\N$, we cannot have $x\in A_n$
    and thus $x\in A_n^C$ by decidability of $A_n$. 
    Thus $x\in \bigcap_{n:\N} (A_n^C)$. 
    Therefore
    $$
    (\bigcup_{n:\N} A_n)^C
    \subseteq 
    \bigcap_{n:\N} (A_n^C).
    $$ 
  \item 
    Suppose that for every $n:\N$, we have $x\notin A_n$. 
    There does not exist an $n:\N$ with $x\in A_n$. Thus
    $$
    \bigcap_{n:\N} (A_n^C)
    \subseteq 
    (\bigcup_{n:\N} A_n)^C
    $$ 
  \item 
    Suppose there exists some $n$ with $x\in A_n^C$. Then 
    it cannot be the case that $x\in A_m$ for all $m:\N$.
    Thus
    $$
    \bigcup_{n:\N} (A_n^C)
    \subseteq 
    (\bigcap_{n:\N} A_n)^C
    $$ 
  \item 
    Suppose that $x\in (\bigcap_{n:\N} A_n)^C$. 
    Then define the binary sequence $\alpha$ by $\alpha(i) =1$ iff $i$ is the first index such that 
    $x\notin A_i$. This is well-defined as $A_n$ is decidable for all $n:\N$. 
    If $\alpha(i) = 0$ for all $i$, then $x\in A_i$ for all $i$. 
%    and it is the case that $x\in A_n$ for all $n:\N$. 
    Thus under our assumption $x\in (\bigcap_{n:\N} A_n)^C$, 
    we cannot have that $\alpha(i) = 0$ always. 
    By Markov, there then exists an $i$ such that $\alpha(i) = 1$. 
    Thus $x\notin A_i$ for some $i$. We conclude that. 
    $$
    (\bigcap_{n:\N} A_n)^C
    \subseteq
    \bigcup_{n:\N} (A_n^C)
    $$ 
\end{itemize}
\end{proof} 
Note that we only needed decidability for the first and last bullet point, 
and only the last bullet point used countability (and of course Markov's principle). 


\section{Topology}
Analogous to synthetic algebraic geometry,
we use pointwise and local definitions of open subsets,
which agree if we assume a corresponding choice axiom.

\subsection{Compact Hausdorff}
\subsection{Compact Hausdorff types}
\begin{definition}
  A type $X$ is called compact Hausdorff iff there exists some $S:\Stone$ and some 
  equivalence relation $\sim:S \times S \to \Closed$ such that $X \simeq S / \sim$. 
  We denote $\CHaus$ for the type of compact Hausdorff types. 
\end{definition} 

\begin{lemma}
Let $A\subseteq X$ be a subtype of a compact Hausdorff space. 
Let $S, \sim$ be any presentation of $X$. 
Then $A$ is closed iff it is the image of a closed subtype of $S$ under the quotient map. 
\end{lemma}
\begin{proof}
  If $A$ is closed, then it's pre-image under any map is also closed. 
  In particular for $q:S\to X$ the quotient map, $q^{-1}(A)$ is closed. 
  As $q$ is surjective, we have $q(q^{-1}(A)) = A$,
  hence $A$ is the image of a closed subtype of $S$. 
  Conversely, let $B\subseteq S$ be closed. 
%  Then for any $s:S$, the subtype $\{t:S| B(s) \wedge s \sim t\} \subseteq S$ is closed. 
%  Hence by 
  Define $A\subseteq S$ by 
  $$A(s) := \exists_{s:S} (B(t) \wedge s \sim t).$$
  As $B, \sim$ are closed, by \Cref{ClosedCountableConjunction} and \Cref{InhabitedClosedSubSpaceClosed}, 
  we have that $A$  is closed. 
  Also $A$ respects $\sim$, hence induces a map $A': X\to \Closed$.
  Furthermore, $A(q(s))$ iff $q(s)$ is in the image of $B$. 
  Therefore $A'(x)$ iff $x$ is in the image of $B$. 
\end{proof}
\begin{corollary}
  For $X:\CHaus$ a subtype $A\subseteq X$ is closed iff it is the image of 
  a map $T\to X$ for some $T:\Stone$. 
\end{corollary}
\begin{proof}
  Directly from the above and \Cref{StoneClosedSubsets}.
\end{proof}

\begin{corollary}\label{InhabitedClosedSubSpaceClosed}
  For $X:\CHaus$ and $A\subseteq X$ closed, we have 
  $\exists_{x:X} A(x)$ is closed. 
\end{corollary}
\begin{proof}
  Let $A$ be the image of a map map $T\to X$ for $T:\Stone$. 
  Then $\exists_{x:X} A(x) \leftrightarrow ||T||$, which is closed by \Cref{TruncationStoneClosed}
\end{proof}


\begin{corollary}\label{ClosedDependentSums}
  Closed propositions are closed under dependent sums. 
\end{corollary}
\begin{proof}
  Let $P:\Closed$ and $Q:P \to \Closed$. 
  Then $\Sigma_{p:P} Q(p) \leftrightarrow \exists_{p:P} Q(p)$.
  As $P$ is Stone by \Cref{PropositionsClosedIffStone}, it is also compact Hausdorff, thus
  \Cref{InhabitedClosedSubSpaceClosed} gives that $\Sigma_{p:P} Q(p)$ is closed. 
\end{proof}
\begin{remark}
  Analogously to \Cref{OpenTransitive} and \Cref{OpenDominance}, it follows that 
  closedness is transitive and $\Closed$ forms a dominance. 
\end{remark}
\begin{corollary}\label{AllOpenSubspaceOpen}
  For $U\subseteq X$ an open subset of a compact Hausdorff space, we have 
  $\forall_{x:X} U(x)$ open. 
\end{corollary}
\begin{proof}
  As $U$ is open, $\neg U$ is closed. 
  Hence $\exists_{x:X} \neg U(x)$ is closed. 
  Therefore, $\neg (\exists_{x:X} \neg U(x))$ is open. 
  Furthermore, it is equivalent to $\forall_{x:X} \neg \neg U(x)$, 
  which is equivalent to $\forall_{x:X} U(x)$ by \Cref{rmkOpenClosedNegation}.
\end{proof}

\begin{lemma}
  Let $X:\Chaus$ be presented by $S/\sim$. 
  Then $2^X$ is an open sub-Boolean algebra of $2^S$. 
\end{lemma}
Note that we do not claim $2^X$ is countable presented, 
but by \Cref{OpenSubsetEnumerableAreEnumerable}, it will be enumerable. 
\begin{proof}
  Denote $q:S \twoheadrightarrow X$ for the quotient map. 
  This induces an injection of Boolean algebras $2^X \hookrightarrow 2^S$.
  Note that $a:S\to 2$ lies in $2^X$ iff for all $s,t:S$, we have $a(s) = a(t)$ whenever $s\sim t$.
  Note that $a(s) = a(t)$ is decidable and $s\sim t$ is open, hence 
  $(s\sim t) \to (a(s) = a(t))$ is open (\Cref{ImplicationOpenClosed})
  By \Cref{AllOpenSubspaceOpen}, we conclude that 
  $\forall_{s:S} \forall_{t:S} ((s\sim t) \to (a(s) = a(t)))$ is open. 
  Hence $2^S$ is an open subobject of $2^X$. 
\end{proof}

\subsection{Intersection of closed in compact Hausdorff}

\begin{lemma}
  In a compact Hausdorff, closed sets are closed under intersection. 
\end{lemma}
\begin{proof}
  
\end{proof}



\begin{lemma}
  Any Stone space merely is a closed subspace of Cantor space. 
\end{lemma}
\begin{proof}
  Let $S$ be a Stone space, and let it's underlying Boolean algebra $B$ be generated by 
  $(b_n)_{n:\N}$ under quotient of the relations ${\phi_i}_{i:\N}$. 
  Then $S = \{ x: 2^\N | \forall_{i:\mathbb n} x(\phi_i) = 0\}$, 
\end{proof}


\begin{lemma}
  For, $D\subseteq 2^\N$ decidable, $\sim$ a closed equivalence relation on $2^\N$,
  the set $$\{x:S | \exists y : D (x\sim y)\}$$ is closed. 
\end{lemma}
\begin{proof}
  Let $x:S$. We need to show that $\exists (y:S) D(y) \wedge x \sim y$ is a closed proposition. 

  Note first that as $\sim $ is closed, $x \sim \cdot $ is a closed subset of $S$. 
  Therefore, $x\sim \cdot = \bigcap_{n:\N} E_n$ for $(E_n)_{n:\N}$ a
  countable family of decidable subsets of $S$, without losing generality, 
  we may even assume that $E_n \subseteq E_m$ whenever $m\geq n$. 
  We thus need to show that 
  $$
  \exists (y:S) D(y) \wedge (\bigcap_{n:\N} E_n)(y) 
  = 
  \exists (y:S) (\bigcap_{n:\N} D \cap  E_n)(y) 
  $$
  is closed. 
%
  Now we claim that 
  $\exists (y:S) D(y) \wedge E_n(y)$ is closed for all $n:\N$. 
  There merely exists an $m:\N$ such that both $D$ and $E_n$ only consider 
  the first $m$ entries of a sequence. 
  
\end{proof}



\begin{lemma}
  For $S$ Stone, $D\subseteq S$ decidable, 
  $\sim$ a decidable equivalence relation on $S$,
  the set $\{x:S | \exists y : D (x\sim y)\}$ is closed. 
\end{lemma}
\begin{proof}
  Let $B = 2^S$, so $S = Sp(B)$. 
  As $D$ is decidable, 
%  there is some $b:B$ such that $D(y) := (y(b) = 1)$. 
  there is some $n:\N$ such that $D(y)$ only depends on $y|_n$. 

  As $\sim$ is decidable, there is a finite set $I_0\subseteq \N$,
  such that $x\sim y = \prod_{i:I_0} x(i) = y(i)$. 

  Thus 
  $$
   \exists (y : D) (x\sim y) = 
  || \Sigma(y:2^\N) y(b) = 1 \wedge \prod(i:I_0) x(i) = y(i)||
  $$
\end{proof}



\begin{lemma}
  Let $S$ Stone, then $D\subseteq S$ is closed iff 
  $D\subseteq S\subseteq 2^{\N}$ is closed. 
\end{lemma}
\begin{proof}
  Follows immediately from countable intersection of basic clopen. 
\end{proof}




%%  Let $A,B\subseteq X$ be two closed subsets of a compact Hausdorff space $X = S/ \sim$. 
%%  If we know that closed subsets contain are exactly those containing their limits this is very easy right? 
%%  Then any sequence has it's limit both in $A$ and $B$. 
%%\begin{lemma}
%%  Whenever $x_n$ is a convergent sequence, so is $f(x_n)$. 
%%\end{lemma}
%%\begin{proof}
%%  Follows immediately from \Cref{sequenceConvergentIffLimit}.
%%\end{proof}
%%
%%
%%\begin{lemma}
%%  In a compact Hausdorff, whenever a subset $A$ contains all of its limit points, it is closed. 
%%\end{lemma}
%%
%%\begin{proof}
%%  Suppose $A\subseteq X$ contains all of it's limit points. We will show that $f^{-1}(A)$ is closed. 
%%  Let $(x_n)_{n:\N}$ be a sequence in $f^{-1}(A)$ with limit $l$, 
%%  then 
%%  $(f(x_n))_{n:\N}$ is a sequence in $A$ with limit $f(l)$. 
%%  $A$ contains $f(l)$ by assumption. 
%%  Therefore $l\in f^{-1}(A)$. 
%%  Thus every sequence in $f^{-1}(A)$ with a limit has its limit in $f^{-1}(A)$. 
%%\end{proof}
%%
%%\begin{lemma}
%%  In a Stone space, whenever a subset $A$ contains all of its limit points, it is closed. 
%%\end{lemma}
%%\begin{proof}
%%  Let $A \subseteq S$ contain all of it's limit points. 
%%  We will show $A$ is a countable intersection of decidable subsets of $S$, hence closed. 
%%  As $S$ is a subset of Cantor space, we may assume it is Cantor space. 
%%  Thus $A$ is a set of binary sequences. 
%%
%%  We will denote $D_n$ be the set of initial segments of length $n$ occuring in $A$. 
%%  We claim this is well defined, that's not a problem, as it's the image of an operation. 
%%
%%  Counterexample : $A = \{ \overline 0 | p\}$ which contains all of it's limit points
%%  (any sequence in $A$ must be $\overline 0$ constantly, which has a limit if the sequence exists in $A$). 
%%  However, $D_n$ is not decidable. 
%%  Also $A$ is not the intersection of countably many decidable sets I believe. 
%%  Unless off course $p$ is of the form $\alpha=0$, but those are not the only propositions.
%%  For example, the proposition $\beta\neq 0$ cannot be written in that form for general $\beta$. 
%%\end{proof}


\subsection{Open propositions}
\input{OvertlyDiscrete/FactorizationFin}

\section{Analysis}

\subsection{Convergence}
\input{Convergence/convergenceClosed}
\paragraph{Extensional convergence }  
\begin{definition}
  Let $B_\infty$ be the Boolean algebra on countably many generators $(p_n)_{n:\N}$ 
  over the equivalence $p_n\wedge p_m = 0 $ whenever $n \neq m$. 
\end{definition} 
\begin{definition}
  We denote $\Noo$ be the spectrum of $B_\infty$. 
\end{definition} 
\begin{lemma}
  $B_\infty$ is isomorphic with the Boolean algebra of 
  finite/cofinite subsets of $\N$. 
\end{lemma}
\begin{proof}
  To go from $B_\infty$ to subsets of $\N$, we send
  the generators $p_n$ to the singleton $\{n\}$, which are clearly finite. 
  We call the induced Boolean operation $f$. 

  To go from finite/cofinite subsets of $\N$ to $B_\infty$,
  a finite subset $I$ of $\N$ is sent to the element 
  $\bigvee_{i \in I} p_i$, and a cofinite subset $J$ is sent to the element 
  $\bigwedge_{i \in J^C} \neg p_i$.  
  We call this function $g$ and we need to show that $g$ is a Boolean morphism. 
  \begin{itemize}
    \item By deMorgan's laws, $g$ preserves $\neg$. 
    \item To see that $g$ respects $\vee$, we need to check three cases
      \begin{itemize}
        \item If both $I,J$ are finite, then 
        \begin{equation} 
          g(I \cup J) = \bigvee_{i\in I \cup J} p_i= \bigvee_{i\in I} p_i \vee \bigvee_{j\in J} p_j 
        \end{equation}
      \item If both $I,J$ are cofinite, we have
        \begin{equation}
          g(I) \vee g(J) = 
          \bigwedge_{i \in I^C} \neg p_i \vee 
          \bigwedge_{j \in J^C} \neg p_j 
          = 
          \bigwedge_{i\in I^C} 
          \bigwedge_{j \in J^C}(\neg p_i \vee  \neg p_j) 
        \end{equation}
        Now note that $\neg p_i \vee \neg p_j = \neg ( p_i \wedge p_j)$, which 
        is $1$ if $i \neq j$ and $p_i$ if $i =j$. 
        We can leave $1$ out of the meet, and we are left with the intersection of $I^C$ and $J^C$, so
        \begin{equation}
          g(I) \vee g(J) = 
          \bigwedge_{i \in (I^C \cap J^C)} \neg p_i
          = 
          \bigwedge_{i \in (I \cup J)^C} \neg p_i 
        \end{equation} 
        as the union of $I$ and $J$ is also cofinite, this equals 
          $ g( I \cup J)$. 
        \item If $I$ is finite and $J$ cofinite, we have 
        \begin{equation}
        g(I) \vee g(J) = (\bigvee_{i\in I} p_i) \vee (\bigwedge_{j \in J^C} \neg p_j)
        = \bigwedge_{j \in J^C} (\bigvee_{i \in I}( p_i \vee \neg p_j))
        \end{equation}
        If $i\neq j$, then $p_i\wedge p_j = 0$, hence $\neg p_j \geq p_i$ and $p_i \vee \neg p_j  = \neg p_j$
        If $i = j$, then $p_i \vee \neg p_j = 1$.
%        \begin{equation}
%        g(I) \vee g(J) = 
%        = \bigwedge_{j \in J^C} (\bigvee_{i \in I-J}( p_i \vee \neg p_j))
%        \end{equation}
%
%        \item If $I$ is cofinite and $J$ is finite, we have that $I \cup J$ is cofinite.
%        Thus 
%        \begin{equation}
%          g(I \cup J) = \bigwedge_{i \in (I \cup J)^C} \neg p_i
%        \end{equation}
%
      \end{itemize}
    \item The case for $\wedge$ is completely dual to the case for $\vee$. 
  \end{itemize}
We conclude that $g$ is a Boolean morphism. 
Furthermore, $g$ and $f$ are clearly inverses, thus the Boolean algebras are isomorphic. 
\end{proof}

  \begin{lemma}\label{lemBinftyNormalForm}
  Any element of $B_\infty$ can be written as 
  either $\bigvee_{i\in I} p_i$  or
  as $\bigwedge_{j\in J} \neg p_j$ 
  for finite $I,J\subseteq \N$. 
\end{lemma}
\begin{proof}
  Remark that whenever $n \neq m$, we have that 
  $\neg p_n \geq p_m$ as $p_m \wedge p_n = 0$. 
\end{proof}
There is canonical embedding $\N \hookrightarrow \Noo$, 
wich sends $n$ to the unique function $\chi_{n}$ sending $p_n$ to $1$. 
We denote $\infty \in \Noo$ for the function which is constantly $0$. 
By \Cref{PropMarkov}, if an element is not $\infty$, 
it comes from the embedding $\N \hookrightarrow \Noo$. 
\begin{lemma}\label{LemmaOpensContainingInfty}
  Let $U$ be an open subset of $\Noo$ containing $\infty$.
  Then there merely exists an $N:\N$ such that whenever $n\geq N$, 
  $\chi_n\in U$ as well. 
\end{lemma}
\begin{proof}
  It is sufficient to prove the lemma for $U$ a basic open. 
  Assume $b : B_\infty $ is such that 
  $U = \{ \phi: B_\infty \to 2| \phi(b) = 1\}$.
  Assume furthermore that $\infty \in U$.
%  so $U$ contains the function sending every $p_i$ to $0$. 
  by \Cref{lemBinftyNormalForm}, $b$ can have two forms.
  If $b = \vee_{i\in I} p_i$, then as $\infty(b) = 0$, 
  we must have $I = \emptyset$, and thus $b = 0$, 
  which means $U$ is empty, contradicting $\infty\in U$. 
  Therefore, 
  $b$ must be of the form $\wedge_{j \in J} \neg p_j$. 
  Note that for $N = \max J + 1$, whenever $n>J$, 
  $\chi_n$  sends $b$ to $1$. 
  Thus $\chi_n \in U$ as well, and we are done. 
\end{proof}

\begin{definition}
  Let $\alpha$ be a sequence in $X$, we say that $\alpha$
  is convergent iff there exists an extension. 
  \begin{equation}\begin{tikzcd}
    \N \arrow[r, "\alpha"] \arrow[d,hook]  & X \\
    \Noo \arrow[ru,dashed]
  \end{tikzcd}\end{equation}  
\end{definition}  



\begin{proposition}
  A sequence is convergent iff it has a limit
\end{proposition}
\begin{proof}
  Let $\alpha$ be convergent, with extension $\overline \alpha$.
  we claim that $\overline \alpha(\infty)$ is a limit of $\alpha$.
  Let $U \subseteq X$ be an open containing $x$. 
  As $\overline\alpha^{-1}(U)$ is an open subset of $\Noo$ containing $\infty$,
  \Cref{LemmaOpensContainingInfty} tells us there exists some $N$ such that $[N,\infty]\subseteq \overline \alpha^{-1}(U)$. 
  Thus there exists an $N$ such that for $n\geq N$, we have $\alpha(n) \in U$, as required. 

  Conversely, suppose $\alpha$ has limit $x$. 
  Assume $X = Sp(B)$, and let $b\in B$. Then $b$ corresponds to a decidable subset $U\subseteq X$.
  For any decidable subset $U \subseteq X$, we have 
  $\alpha^{-1}(U)$ a decidable subset of $\N$. 
  We claim that $\alpha^{-1}(U)$ is either finite or cofinite. 
  As $U$ is decidable, we can decide wheter $x\in U$. If $x\in U$, $\alpha^{-1}(U)$ is cofinite, as 
  $\alpha(n) \in U$ for all $n \geq N$ for some $N$. 
  If $x\notin U$, we have $x\in U^C$, which is also decidable and therefore $\alpha^{-1}(U^C)$ is cofinite. 
  As $\alpha^{-1}(U) ^ C = \alpha^{-1}(U^C)$, it follows that $\alpha^{-1}(U)$ is finite. 
  Thus $\alpha^{-1}(U)$ is finite or cofinite for any decidable subset $U\subseteq X$. 
  Finite and cofinite subsets of $\N$ correspond to elements of $B_\infty$. 
  Therefore, $\alpha$ induces a map $B \to B_\infty$, which corresponds to a map 
  $\overline \alpha: \Noo \to X$. 

  We claim that $\overline \alpha$ extends $\alpha$. 
  Denote $\iota$ for the map $\N \to \Noo$. 
  We need to show that $\overline \alpha \circ \iota = \alpha$. 
  By definition, we have that $(\overline \alpha \circ \iota)^{-1}(U) = \alpha^{-1}(U)$ 
  for any decidable $U\subseteq X$. 
\end{proof}

\begin{lemma}
  Whenever $S = Sp(B)$ Stone, $f,g: A \to S$, and $f^{-1}(U) = g^{-1}(U)$ for any decidable 
  $U\subseteq S$, we have $f = g$. 
\end{lemma}
\begin{proof}
  By our assumption, we have for all $a:A$ that $f(a) \in U \iff g(a) \in U$ for 
  any decidable $U\subseteq X$. Such $U$ correspond to $b:B$.
  and $f(a) \in U \iff f(a)(b)  = 1$. 
  So the functions $f(a),g(a):B \to 2$ are such that 
  $f(a) (b) = g(a) (b)$ for all $b:B$. 
  This holds for all $a:A$ and by two uses of function extensionality we may conclude 
  $f=g$. 
\end{proof}




\subsection{The interval}
%The goal of this section is to define the interval $[-2,2]_\mathbb R$ as a scheme. 
We assume $\mathbb N, \mathbb Q$ have been defined in HoTT
with linear propostional order relations $<,\leq, > ,\geq$ playing nicely together 
and standard algebraic operations. 
From these, we can define the subtype $\mathbb Q_{>0}=\sum_{q : \mathbb Q} (q>0)$, 
and the absolute-value function $|\cdot|$ on $\mathbb Q$. 

\begin{definition}
  A pre-Cauchy sequence is a sequence of rational numbers $(q_n)_{n: \mathbb N}$ with $-2 \leq q_n \leq 2$ 
  for all $n:\mathbb N$
%  together with a term of type
  such that for every $\epsilon: \mathbb Q_{>0}$, we have an $N_\epsilon:\mathbb N$, 
  such that whenever $n,m \geq N_\epsilon$, we have 
\begin{equation}
%  \forall \epsilon : \mathbb Q_{>0} \Sigma N : \mathbb N \forall m,n : \mathbb N (m,n \geq N) \to 
  | q_n - q_m | \leq \epsilon
\end{equation} 
\end{definition}

\begin{definition}
Given two pre-Cauchy sequences $p = (p_n)_{n\in\mathbb N}, q=(q_n)_{n\in\mathbb N}$, 
we define the proposition $p \sim_C  q$ as 
%for all $\epsilon : \mathbb Q_{>0}$ there exists an $N :\mathbb N$ such that whenever $n \geq N$, we have
\begin{equation}
  p \sim_C q : = \forall (\epsilon : \mathbb Q_{>0} )\exists ( N :\mathbb N) \forall (n : \mathbb N) ((n \geq N) \to 
  (| p_n - q_n| \leq  \epsilon))
\end{equation}
\end{definition}
Note that $\sim_C$ defines an equivalence relation on pre-Cauchy sequences. 
\begin{definition}
We define the type of Cauchy sequences as the type of pre-Cauchy sequences quotiented by $\sim_C$. 
\end{definition}

%\begin{definition}
%  A binary sequence consists of an initial segment $I \subseteq \mathbb N$
%  and a function $x:I \to 2$. 
%If $I$ is (in)finite, we call the binary sequence (in)finite as well. 
%\end{definition} 
%
%For $x$ a finite binary sequence and $y$ any binary sequence, 
%we'll denote $(x,y)$ for their concatenation, 
%and $\overline x$ for the infinite sequence repeating $x$. 
%
Denote $T = \{-1,0,1\}$. 
\begin{lemma}
  $T^\mathbb N$ is a scheme. 
\end{lemma}
\begin{proof}
  Sketch: partition $2^\mathbb N$ as follows: 
  For $\alpha: 2^\mathbb N$, we'll make a sequence $\beta: T^\mathbb N$.
  consider for each $n$ the $n$'th block of 2 entries in $\alpha$
  if both are $0$, $\beta(n) = 0$. 
  If the first is $1$, $\beta(n) = -1$
  If first is $0$ and the second is $1$, then $\beta(n) = 1$. 
  This is a closed equivalence relation. 
\end{proof} 

Consider the relation $\sim_s$ on $T^{\mathbb N}$, 
such that for any finite binary sequence $x$, we have 
\begin{align}
  (x,1,\overline 0) &\sim_t (x ,0, \overline 1) \\
  (x,-1,\overline 0) &\sim_t (x ,0, \overline {-1})\\
  (x,1,\overline {-1}) &\sim_t (x , \overline 0) \\
  (x,-1,\overline {1}) &\sim_t (x , \overline 0) 
\end{align} 
\begin{lemma}
$\sim_t$ induces a closed equivalence relation on $2^\mathbb N$. 
\end{lemma}
\begin{proof}
  TODO
\end{proof} 

\begin{proposition}\label{propTernaryCauchy}
  $T^\mathbb N/ \sim_t$ is isomorphic to the type of Cauchy sequences. 
\end{proposition} 
\begin{definition}%Construction might be better than definition here, but WIP so who cares. 
  For $\alpha: T^\mathbb N$, define the rational sequence $tri(\alpha)$ by 
  \begin{equation} (tri(\alpha))_n :  = \sum\limits_{0 \leq i \leq n} \frac{\alpha(i)} { 2^{i}} \end{equation}  
  This sequence is pre-Cauchy with $N_\epsilon$ given by the first $n$ with $(\frac12)^n<\epsilon$. 
\end{definition}  
%
%  Also, whenever $\alpha\sim_t \beta$, we have 
%  $tri(\alpha) \sim_C tri(\beta)$. 
%  Therefore $tri$ induces a function from $T^\mathbb N / \sim_t$ to Cauchy sequences. 
\begin{definition}
  Given a pre-Cauchy sequence $p$, 
  we will define a $T$-sequence $\alpha  = c(p): T^\mathbb N$.
  Consider any $i:\mathbb N$, and suppose $\alpha(j)$ has been defined for $0 \leq j<i$. 
%
  Let $\epsilon_i = (\frac12)^{i+1}$. %Placeholder value.
  Define $N_i:= N_{\epsilon_i}$. %is such that for $n,m \geq N_i$, we have $|p_n - p_m| < \epsilon_i$. 
%
  Consider 
  \begin{equation}
    \widetilde p_i = p_N - \sum\limits_{0\leq j < i} \frac {\alpha(j)}{2^{j}}.
  \end{equation}
  As the order on $\mathbb Q$ is total, we can define 
  \begin{equation}
    \alpha(i) = \begin{cases}
    \phantom{-} 1  \text{ if } \widetilde p_i \geq    (\frac12)^{i} \\
    -1             \text{ if } \widetilde p_i \leq  - (\frac12)^{i} \\
    \phantom{-} 0 \text{ otherwise } 
    \end{cases} 
  \end{equation}  
\end{definition} 
We shall now prove the following four things: 
\begin{itemize}
  \item 
    $c(tri(\alpha)) \sim_t \alpha$ for any $\alpha: T^n$.
  \item 
    $tri(c(p)) \sim_C p$ for any pre-Caucy sequence $p$. 
  \item 
    Whenever $p \sim_C q$, we have $c(p)\sim_t c(q)$. 
  \item 
    Whenever $\alpha \sim_t \beta$, we have $tri(\alpha) \sim_C tri(\beta)$. 
\end{itemize}
It follows that $c$ and $tri$ are maps between Cauchy sequences and $T^\mathbb N /\sim_t$ 
which are each other's inverse, proving Proposition \ref{propTernaryCauchy}
\begin{lemma} $tri(c(p)) \sim_C p$ for any pre-Caucy sequence $p$. 
\end{lemma} 



\begin{proof}
  Let $\epsilon>0$ be given, consider $n:\mathbb N$ such that
  $(\frac12)^n < \epsilon$. 
  We claim that for $m\geq N_n$, we have that $|p_m- tri(c(p))_m| < \epsilon$. 

  By definition $p_{N_n} $  
\end{proof} 






\subsection{The Cauchy reals}
The goal of this section is to introduce the real numbers in a constructive setting, 
following the definition given in \cite{Bishop} with some small adaptations. 
We will later use this definition to show that the interval $[0,1]$ is compact Hausdorff in the sense 
of \Cref{dfnCompactHausdorff}. 

We will assume we are given natural and rational numbers, with decidable (in)equalities
working as expected. 

\begin{definition}
  A \textbf{Cauchy sequence} is a sequence $x : \mathbb N \to \mathbb Q$ such that
  for any $n,m:\mathbb N$, we have %$0\leq x_n \leq 1$ and 
$|x_n-x_m| \leq (\frac12)^n + (\frac12)^m$. 
\end{definition}
\begin{remark}
  If $x$ is a cauchy sequence and $q$ a rational number, the 
  sequence $(x-q)_n = (x_n-q)$ is also Cauchy.
\end{remark}

Following \cite{Bishop}, we define inequality relations between Cauchy sequences and
rational numbers. 
\begin{definition}
  For $x$ a Cauchy sequence and $q$ a rational number, we define 
  \begin{itemize}
%    \item $x> q = \Sigma(n:\mathbb N) x_n > q + {\frac12}^n$. %for some $n:\mathbb N$. 
%    \item $x< q = \Sigma(n:\mathbb N) x_n < q - {\frac12}^n$. %for some $n:\mathbb N$. 
    \item $x\leq  q = \Pi_{n:\mathbb N} x_n \leq q+(\frac12)^n$. 
    \item $x\geq  q = \Pi_{n:\mathbb N} x_n \geq q-(\frac12)^n$. 
  \end{itemize}
\end{definition}
%\begin{lemma}
%  For $x$ Cauchy and $q$ rational, we have that 
%  $x\leq q$ iff for each $n:\mathbb N$, we have a $N_n:\mathbb N$ with 
%  $x_m> q-(\frac12)^n$ for all $m \geq N_n$. 
%\end{lemma}
\begin{lemma}\label{ComparisonLemma}
  For $x$ a Cauchy sequence and $q$ a rational number, we have
  $x \leq q \vee x \geq q$. 
\end{lemma}
\begin{proof}
  For rational numbers, we have decidable inequalities, 
  therefore $\geq 0 \vee q \leq 0$. 
  It follows that 
  $ \forall (n:\mathbb N) \forall (m:\mathbb N) q \geq -(\frac12)^n \vee q \leq (\frac12)^m$. 
  Now by \Cref{TODO}, we may conclude 
  $ (\forall (n:\mathbb N) q \geq -(\frac12)^n ) \vee (\forall (m:\mathbb N) q \leq (\frac12)^m)$
  as required.
\end{proof}


%%%\begin{definition}
%%%  A Cauchy sequence $x$ is \textbf{nonnegative} if $x_n \geq -(\frac12)^n$
%%%  for every $n:\mathbb N$. 
%%%  $x$ is \textbf{nonpositive} if $x_n \leq (\frac12)^n$
%%%  for every $n:\mathbb N$. 
%%%\end{definition} 
%%%%\begin{lemma}
%%%%  A Cauchy sequence is nonnegative iff there exists an $N$ such that $x_n \geq -(\frac12)^N$
%%%%  for all $n\geq N$. 
%%%%  A Cauchy sequence is nonpositive iff there exists an $N$ such that $x_n \leq (\frac12)^N$
%%%%  for all $n\geq N$. 
%%%%\end{lemma}
%%%%\begin{proof}
%%%%  Assume $x$ is nonnegative. Thus for every $n:\mathbb N$, we have $x_n\geq -(\frac12)^n \geq -(\frac12)^0$. 
%%%%  Thus $N$ can taken to be $0$. 
%%%%%
%%%%  Conversely, as $x$ is Cauchy, we have
%%%%  for all $m :\mathbb N$ that  
%%%%%  \begin{equation}- (\frac12)^m -(\frac12)^n \leq    x_m-x_n \leq (\frac12)^m + (\frac12)^n \end{equation}
%%%%  \begin{equation}- (\frac12)^m -(\frac12)^n \leq    x_n-x_m \leq (\frac12)^m + (\frac12)^n \end{equation}
%%%%  If in addition there is an $N$ such that whenever $m\geq N$, we have 
%%%%  $x_m \geq -(\frac12)^N$, so $-x_m \leq (\frac12)^N$, 
%%%%  so $x_n -x_m \leq x_n - (\frac12)^N$. 
%%%%  Therefore, 
%%%%  \begin{equation}- (\frac12)^m -(\frac12)^n \leq    x_n-x_m \leq x_n-(\frac12)^N \end{equation}
%%%%  Thus 
%%%%  \begin{equation}- (\frac12)^m -(\frac12)^n  + (\frac12)^N \leq x_n \end{equation}
%%%%  As $N \geq N$, we have in particular 
%%%%  \begin{equation}- (\frac12)^N -(\frac12)^n  + (\frac12)^N \leq x_n \end{equation}
%%%%  \begin{equation} - (\frac12)^n  \leq x_n \end{equation}
%%%%  thus $x$ is nonnegative. 
%%%%
%%%%  The nonpositive case goes similar. 
%%%%\end{proof}   
%%% 
%%%
%%%\begin{lemma}
%%%  A Caucy sequence is nonnegative or nonpositive. 
%%%\end{lemma}

%\begin{lemma}
%  For any Cauchy sequence $p$, we have 
%  $(\forall (n:\mathbb N) p_n \leq (\frac12)^n) \vee (\forall (n:\mathbb N) p_n \geq -(\frac12)^n)$. 
%\end{lemma}
%\begin{proof}
%We 
%\end{proof}

\begin{definition}
Given two Cauchy sequences $p = (p_n)_{n\in\mathbb N}, q=(q_n)_{n\in\mathbb N}$, 
we define the proposition $p \sim_C  q$ as 
\begin{equation}
  p \sim_C q : = \forall (n,m : \mathbb N) ((| p_n - q_m| \leq  (\frac12)^n + (\frac12)^m))
\end{equation}
\end{definition}

%\begin{remark}
%  Note that $p\sim_C q$ is equivalent to 
%\begin{equation}
%  \forall (n : \mathbb N) | p_n - q_n| \leq  (\frac12)^{n-1}
%\end{equation}
%The equivalence doesn't hold, unless you cut off initial segments (which shouldn't matter). 
%\end{remark} 

\begin{definition}
  The type of \textbf{Cauchy reals} is given by 
  the type of Cauchy sequences modulo $\sim_C$.
\end{definition}

We claim that the inequality in \Cref{TODO} extends to a well-defined 
inequality between Cauchy reals and rational numbers. 

Furthermore, we claim that 
$\Pi_{x:\mathbb R} \Pi_{q:\mathbb Q} x \leq q \vee x \geq q$. 

%\begin{lemma}
%  For any Cauchy real $r$ any Cauchy sequence $p$ representing $r$, 
%  we have 
%  \begin{equation}
%    (\forall (n:\mathbb N) p_n \leq (\frac12)^n) \vee (\forall (n:\mathbb N) p_n \geq (\frac12)^n)
%  \end{equation}
%
%\end{lemma}

\begin{definition}
  A Cauchy sequence in the interval is a Cauchy sequence $x$ such that 
  for any $n:\mathbb N$, we have $0\leq x_n \leq 1$. 
 % 
  The interval of Cauchy reals is given by the type of Cauchy sequences in the interval 
  modulo $\sim_C$. We denote it by $[0,1]$. 
\end{definition}  


We want to show that the interval of Cauchy reals is Compact Hausdorff. 
Informally, to any binary sequence $\alpha : \mathbb N \to 2$, 
we can associate a Cauchy sequence 
\begin{equation}\label{eqnBinaryEncode}
  n\mapsto \sum\limits_{i = 0 }^n \frac {\alpha(i)}{2^{i+1}}
\end{equation}
and we are going to give a closed relation on Cantor space such that 
two binary sequences are equivalent iff they correspond to the same Cauchy reals. 
%
First, we'll need some notation.
\begin{definition}
Given a binary sequence $\alpha:\mathbb N \to 2$ and a natural number $n : \mathbb N$  
we denote $\alpha|_n: \mathbb N_{\leq n} \to 2$ for the 
restriction of $\alpha$ to a finite sequence of length $n$. 
We denote $\overline 0, \overline 1$ for the binary sequences which are constantly $0$ and $1$ respectively. 
We denote $0,1$ for the sequences of length $1$ hitting $0,1$ respectively. 
If $x$ is a finite sequence and $y$ is any sequence, denote $x\cdot y$ for their concatenation. 
\end{definition} 
Now we'll give a definition for when two finite binary sequences of length $n$ correspond 
to real numbers whose distance is $\leq (\frac12)^n$.
Basically, we want for every finite sequence $z$ that 
$(z \cdot 0 \cdot \overline 1)$ and  $(z \cdot 1 \cdot \overline 0)$ are equivalent. 

\begin{definition}
Now let $n:\mathbb N$ and $x,y:\mathbb N_{\leq n} \to 2$ be two sequences of length $n$. 
We say $x,y$ are near if we have an $m:\mathbb N$ with $m\leq n$
and some $a: \mathbb N_{\leq m} \to 2$, 
such that one of $(a \cdot 0 \cdot \overline 1)|_n,  ( a \cdot 1 \cdot \overline 0)|_n$
is equal to $x$ and the other is equal to $y$. 
We denote $\text{near}_n(x,y)$ if $x,y$ are near. 
%
To be precise, we define 
\begin{equation}
  \text{near}_n(x,y) = 
\Sigma(m:\mathbb N) m \leq n \wedge 
  \Sigma (a : Fin_m \to 2) 
\bigg( \big( (x,y) = 
((a \cdot 0 \cdot \overline 1)|_n,  ( a \cdot 1 \cdot \overline 0)|_n)
\big)
\bigvee 
\big(
  (y,x) = 
((a \cdot 0 \cdot \overline 1)|_n,  ( a \cdot 1 \cdot \overline 0)|_n)
\big)
\bigg)
\end{equation}
\end{definition}
\begin{remark}
Remark that when $x,y$ are near, $m$ and $a$ as above are unique. 
Thus $\text{near}_n(x,y)$ is a proposotion. 
%
Furthermore, to check whether $x,y$ are near, we need only make $n$ comparisons, 
thus $\text{near}_n(x,y)$ is decidable. 
%
Note that in the above definition, we allow $m = n$ and therefore $x$ is near to itself for any finite sequence $x$. 
Furthermore, we have defined nearness to be symmetric. 
However, it is not a transtive relation. 
After all, the sequence $010$ and $011$ are near and the sequence $011$ and $100$ are near, 
but $010$ is not near to $100$. This corresponds to the fact that $\frac14$ and $\frac38$ are distance $\leq (\frac12)^3$
apart, and so are $\frac38$ and $\frac12$, but $\frac14$ and $\frac12$ are not. 
\end{remark}
\begin{definition}
  We define the following relation on Cantor space for $\alpha, \beta: 2^\mathbb N$.
  \begin{equation}
    \alpha \sim_t \beta = \forall (n : \mathbb N) 
    \text{near}_n(\alpha|_n, \beta|_n)
  \end{equation}
\end{definition}
\begin{lemma}
  $\sim_t$ is a closed equivalence relation. 
\end{lemma}
\begin{proof}
   Let $\alpha, \beta, \gamma : 2^\mathbb N$. 
   As the dependent product of propositions is a proposition, $\alpha \sim_t\beta$ is a proposition. 
   %
   Furthermore, the closedness follows from decidability of $\text{near}_n(\alpha|_n, \beta|_n)$. 
   One could define $\gamma(n) = 1$ iff $\text{near}_n(\alpha|_n, \beta|_n)$
   
   As nearness is reflexive and symmetric, so is $\sim_t$. 

   Now suppose $\alpha \sim_t \beta$ and $\beta\sim_t \gamma$. 
   We claim that $\alpha \sim_t \gamma$. 

   Let $n:\mathbb N$, we need to show that 
   $\text{near}_n(\alpha|_n , \gamma|_n)$. 
   Let $(a,m)$ witness that $\text{near}_n(\alpha|_n, \beta|_n)$.
   and let $(b, k)$ witness that $\text{near}_n(\beta|_n, \gamma|_n)$
   We will make a case distinction on whether one of $m,k$ is equal to $n$, or
   both are strictly smaller than $n$. 
   \begin{itemize}
     \item 
       If $m=n$, we have that $\alpha|_n = \beta|_n$, and therefore 
       \begin{equation}
         \text{near}_n(\beta|_n, \gamma|_n) \leftrightarrow \text{near}_n(\alpha|_n, \gamma|_n)
       \end{equation} 
       The above also holds if $k = n$.
     \item 
       If $m< n$, we have that $\alpha(m+1) \neq \beta(m+1)$, thus 
       $\alpha|_l \neq \beta|_l$ for all $l>m$, 
       but we still have $\text{near}_l(\alpha|_l, \beta|_l)$ for these $l$. 
       Therefore $(\alpha, \beta)$ or $(\beta, \alpha)$ must be of the form
       $(a \cdot 0 \cdot \overline 1, a \cdot 1 \cdot \overline 0)$. 
       WLOG, we assume $\alpha = a \cdot 0 \cdot \overline 1$, and thus 
       $\beta = a \cdot 1 \cdot \overline 0$ (if not, we could do bitflips). 

       As $k<n$ also, by the same argument there is some $b$ such that one of 
       $(\beta,\gamma), (\gamma, \beta)$
       is equal to $(b\cdot 0 \cdot \overline 1, b \cdot 1 \cdot \overline 0)$. 
       However, $\beta$ is also of the form $a \cdot 1 \cdot \overline 0$, and 
       thus cannot also be of the form $b \cdot 0 \cdot \overline 1$. 
       Therefore we must have 
       $\beta = b\cdot 1 \cdot \overline 0$ and 
       $\gamma= b\cdot 0 \cdot \overline 1$. 

       But now $b \cdot 1 \cdot \overline 0 = a \cdot 1 \cdot \overline 0$, 
       The lengths of $a,b$ cannot be unequal, and by decidablity of natural numbers, 
       $a,b$ have the same length and it follows that $ a = b$. 
       Therefore $ \alpha = \gamma$, so $\alpha \sim_t\gamma$.
   \end{itemize}

   We conclude that $\sim_t$ is a closed equivalence relation. 
\end{proof}


\begin{lemma}
  $b$ sends $\sim_n$ equivalent binary sequences to $\sim_C$ equivalent Cauchy sequences. 
\end{lemma}
\begin{proof}
  Let $\alpha, \beta$ be binary sequences.
  We claim that $|b(\alpha)_n - b(\beta)_n| \leq (\frac12)^{n+1}$ 
  whenever $\text{near}_n(\alpha, \beta)$. 
  It will follow that if $\alpha\sim_n \beta$, then 
  $b(\alpha)\sim_C b(\beta)$. 

  Let $n:\mathbb N$ and assume $m:\mathbb N$ with $m\leq n$ and 
  let $z$ be a sequence of length $m$ such that 
  $\alpha|_n = z\cdot 1 \cdot \overline 0|_n$ and $\beta|_n = z \cdot 0 \cdot \overline q |_n$. 
  then $b(\alpha)_n = \sum_{i\leq m} \frac{z(i)}{2^{i+1}} + (\frac12)^{m+2}$ and 
  $b(\beta)_n = \sum_{i\leq m} \frac{z(i)}{2^{i+1}} + \sum\limits_{m+2 \leq i \leq n}(\frac12)^{i+1}$. 
  Thus 
  $b(\alpha)_n - b(\beta)_n = (\frac12)^{m+2} - \sum\limits_{m+2 \leq i \leq n}(\frac12)^{i+1} = 
  (\frac12)^{n+1}$, 
  which is smaller than required. 
\end{proof}  

\begin{lemma}
  Whenever $b(\alpha) \sim_C b(\beta)$, 
  we have $\alpha \sim_n \beta$. 
\end{lemma}
\begin{proof}
  Assume $b(\alpha) \sim_Cb (\beta)$. 
  Let $n:\mathbb N$. 
  We shall show that $\text{near}_n(\alpha , \beta)$. 

  As we're only checking finitely many entries, 
  we either have $\alpha|_n = \beta|_n$, 
  or there exists a smallest $m\leq n$ with 
  $\alpha(m) \neq \beta(m)$. 

  If $\alpha|_n = \beta|_n$, we have $\text{near}_n(\alpha,\beta)$ and are done. 
  WLOG assume $\alpha(m) = 1, \beta(m) = 0$ for $m$ minimal. 
  We claim that for any $k\geq m$, we have 
  $b(\alpha)_k - b(\beta)_k = (\frac12)^k$. 

  Now note that 
  \begin{equation} 
    b(\alpha)_{k+1} - b(\beta)_{k+1} = 
    b(\alpha)_{k} - b(\beta)_{k} + 
    \frac{\alpha(k+1) - \beta(k+1)}{2^{k+2}}.
  \end{equation}




  For $k>m$, we have that 
  \begin{equation}
  |b(\alpha)_k - b(\beta)_k |= 
  |(\frac12)^{m+1} + \sum\limits_{i=m+1}^k \frac{ \alpha(i) -\beta(i)}{2^{i+1}}|. 
  \end{equation}
  Note that the right summand is always $\leq (\frac12)^{m+1}$. 
  Therefore, we can leave out the absolute value function. 
  As $b(\alpha) \sim_Cb(\beta)$, we have that 
  \begin{equation}
  (\frac12)^{m+1} + \sum\limits_{i=m+1}^k \frac{ \alpha(i) -\beta(i)}{2^{i+1}} \leq (\frac12)^{k-1}
  \end{equation}
  Note that $\alpha(i) -\beta(i) \in \{-1,0,1\}$ always. 
  Also, 

  Denote $z = \alpha|_{m-1} = \beta_{m-1}$. 
  We will show that for $n\geq i>m$, we must have $\alpha(i) = 0, \beta(i) = 1$. 
  Suppose $\alpha(i) \neq 0$. 
\end{proof}

%
%
%\begin{theorem}
%  The interval of Cauchy reals is isomorphic to $2^\mathbb N / \sim_t$. 
%\end{theorem} 
%\begin{proof}
%  Define $b: 2^\mathbb N \to [0,1]$ by composing the map in \Cref{eqnBinaryEncode} with the projection map 
%  from Cauchy sequences in the interval to the interval in Cauchy reals. 
%  We will show $b$ is surjective, 
%  and such that $b(\alpha) = b(\beta)$ iff  $\alpha \sim_t \beta$. 
%  \begin{itemize}
%    \item For $b$ to be surjective, we need to show that for any Cauchy sequence $p$ in the interval, there 
%      merely exists some binary sequence $\alpha$ such that $b(\alpha)\sim_C p$. 
%
%      We will inductively define $\alpha$. 
%      Let $\alpha(i)$ be defined for $i<n$. 
%      Consider the sequence $p'(j) = p(j) - \sum\limits_{i<n} \frac{\alpha(i)}{2^{i+1}}$. 
%      As $p$ is a Cauchy sequence, so is $p'$. 
%  \end{itemize}
%\end{proof}
%

%\printindex

\printbibliography

\end{document}
