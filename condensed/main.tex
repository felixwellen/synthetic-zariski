% latexmk -pdf -pvc main.tex
\documentclass{../util/zariski}


\title{A Foundation for Synthetic Stone Duality}

\begin{document}

\author{Felix Cherubini, Thierry Coquand, Freek Geerligs and Hugo Moeneclaey}

\maketitle

%\begin{abstract}
%In synthetic algebraic geometry (SAG) \cite{draft}, we study finitely presented algebras over a commutative ring. 
%In this work, we study countably presented Boolean algebras instead. 
%Where the finitely presented algebras over a commutative ring induce a Zariski topos, 
%%the opposite category of these 
%the countably presented Boolean algebras induce the topos of light condensed sets \cite{Scholze}. 
%\cite{draft} proposes an axiomatization of the Zariski topos in univalent homotopy type theory \cite{hott}. 
%In this work, we propose similar axioms, which we expect to be modelled by light condensed sets. 
%% Furthermore, spectra of countably presented Boolean algebras correspond to quotients of Cantor space
%% which is cool because reasons
%\end{abstract} 
%

\begin{abstract}
  The language of homotopy type theory has proved to be appropriate to develop an internal language for higher toposes, for example with Synthetic Algebraic Geometry for the Zariski topos.
In this paper we use this for the higher topos corresponding to light condensed sets.
This seems to be an appropriate setting to develop synthetic topology (similar to Martin Escardo)
This consists of extending homotopy type theory with 4 axioms and use them to prove internal properties of light condensed sets.
We get an axiom system strong enough to prove Markov's principle, LLPO and the negation of WLPO. 
We also introduce types of open and closed propositions, inducing a topology on any type. 
This leads to a (synthetic) topological study of Stone and compact Hausdorff types. 
All functions are continuous, and the topology is as one would classically expect for compact separable Hausdorff space.
For example, any map from the unit interval to itself is continuous in the usual epsilon-delta sense.
We use the synthetic homotopy theory given by the higher types of homotopy type theory to make homotopical arguments.
As an application, we compute the cohomology of the interval and use this to prove Brouwer's fixed point theorem. 

\end{abstract}


%\tableofcontents

% Logic and Topology
% - Building blocks, explaining the types of Stone, Boole (and what we mean with countable)
% - Rules, stating the axioms and first consequences, including the omniscience principles
% - Topology on propositions, mentioning under what constructions open propositions are closed
% - Examples of closed/open propositions, explaining why Boole is discrete and Stone Hausdorff
% - Topology on Stone spaces, including the classification of closed spaces.
% - Compact Hausdorff spaces. 
% Directed Univalence
% - Tychonov
% - Directed univalence, link to Phoa
% Cohomology
% - Cohomology and the interval
% Appendix, 
%  - alternative formulations of axiom 2
%  - more details on technical constructions
%  - colimit presentation of countably presented Boolean algebras (I'm not sure where we actually use this)
%  - scott continuity instead of axiom 1 



%\section*{Introduction}
%This draft is empty so far.
\section*{Acknowledgements}
The idea to use the topological characterization of stone spaces as totally disconnected, compact Hausdorff spaces to prove \Cref{stone-sigma-closed} was explained to us by Martín Escardó.
We profited a lot from a discussion with Reid Barton and Johann Commelin. 
David Wärn noticed that Markov's principle (\Cref{MarkovPrinciple}) holds. 
At TYPES 2024, we had an interesting discussion with Bas Spitters on the topic of the article.
Work on this article was supported by the ForCUTT project, ERC advanced grant 101053291.

\rednote{TODO: 
  \begin{itemize}
  \item Write intro.
  \item Rereading.
  \item Interval as $\mathbb I$ everywhere. 
  \item Connected types definition.
  \item $\ODisc$ remark complete. 
\end{itemize}
}


\section{Stone duality}
\subsection{Preliminaries}
%
%In this section, we introduce the type of countably presented Boolean algebras $\Boole$ and of Stone spaces $\Stone$. 
%Both of these types carry a natural category structure. 
%In later sections, we will axiomatize an anti-equivalence between these categories, 
%which is classically valid and called Stone duality. 
\begin{definition}
  A type is countable if and only if it is 
  %By countable in the previous definition we mean sets that are 
  merely equal to some decidable subset of $\N$. 
\end{definition}

\begin{definition}
  A countably presented Boolean algebra $B$ is a Boolean algebra such that there merely are 
  countable sets $I,J$, 
  a set of generators $g_i,~{i\in I}$ and a set $f_j,~{j\in J}$ 
  of Boolean expressions over these generators 
  such that $B$ is equivalent to the quotient of the free Boolean 
  algebra over the generators by the relations
  $f_j=0$. We denote this algebra by $2[I]/(f_j)_{j:J}$.
\end{definition} 

\begin{remark}\label{BooleAsCQuotient}
Any countably presented algebra is merely of the form 
$2[\N]/ (r_n)_{n:\N}$.
%, if we add dummy variables that we equate to $0$, and dummy relations that equate $0$ to itself.
% 0 is not special here right? 
\end{remark}


%We will call the family $(f_j)_{j\in J}$ as above a set of relations. 
%If $I,J$ are finite, we call $B$ a finitely presented Boolean algebra. 
%Once we have postulated the axiom of dependent choice, 
%in \Cref{secBooleAsColimits}
%we will be able to show that every countably presented algebra 
%is actually a colimit of a sequence of finitely presented Boolean algebras.
%They are therefore dual to pro-finite objects, which are used 
%in the theory of light condensed sets \cite{Scholze,Dagur,TODO}.

\begin{remark}
  We denote the type of countably presented Boolean algebras by $\Boole$. 
  This type does not depend on a choice of universe. 
  Moreover $\Boole$ has a natural category structure. 
\end{remark}

\begin{example}
  If both the set of generators and relations are empty, we have the Boolean algebra $2$.
  Its underlying set is $\{0,1\}$ and $0\neq_2 1$.
  $2$ is initial in $\Boole$. 
\end{example}
%\begin{remark}
%Note that any Boolean algebra must contain the elements $0,1$. 
%Therefore, $2$ is initial in $\Boole$. 
%\end{remark} 
%We can therefore use it to define points of objects in the category dual to that of countably presented Boolean algebras. 
%
%\subsection{Stone spaces}
\begin{definition}
  For $B$ a countably presented Boolean algebra, 
  we define $Sp(B)$ as the set of Boolean morphisms from $B$ to $2$.
  Any type which is merely equivalent to a type of the form $Sp(B)$ is called a Stone space.
\end{definition}
%\begin{definition}
%  We define the predicate on types $\isSt$ by 
%  \begin{equation}
%    \isSt(X) := ||\sum\limits_{B : Boole} X = Sp(B)||
%  \end{equation} 
%  Given some universe $\mathcal U$, we denote 
%  $\Stone = \Sigma_{S:\mathcal U} \isSt(S)$. 
%%  A type $X$ is called \textit{Stone} if $\isSt(X)$ is inhabited.
%\end{definition}

%\subsection{Examples}
\begin{example}
  \label{boolean-algebra-examples}
  \item 
  \begin{enumerate}[(i)]
  \item There is only one Boolean morphism from $2$ to $2$, thus $Sp(2)$ is the singleton type $\top$. 
  \item   
    The tivial Boolean algebra is given by $2/(1)$. 
    We have $0=1$ in the trivial Boolean algebra. 
    As there cannot be a map from the trivial Boolean algebra into $2$ preserving both $0$ and $1$, 
    the corresponding Stone space is the empty type $\bot$.
  \item\label{ExampleBAunderCantor}   
    We denote by $C$ the Boolean algebra $2[\N]$.
    In this case $Sp(C)$ is the Cantor space, it is equivalent to $2^\N$ the set of binary sequences. 
    If $\alpha:Sp(C)$ and $n:\N$ we write $\alpha_n $ for $\alpha(g_n)$. 
%    %given by $\N$ as a set of generators and no relations. We write $p_n$ for the generator corresponding to $n$.
%    A morphism $C\to 2$ corresponds to a function $\mathbb N\to 2$, 
%    which is a binary sequence. 
%    So $Sp(C) = 2^\N$, 
%%    The Stone space $Sp(C)$ of these binary sequences is denoted 
%    $2^{\N}$ and called \notion{Cantor space}.
  \item\label{ExampleBAunderNinfty}
    We denote by $B_\infty$ the Boolean algebra generated by 
    $(g_n)_{n:\N}$ quotiented by the relations $g_m \wedge g_n = 0$ for ${n\neq m}$.
%    $C/(g_m\wedge g_n)_{m\neq n}$.
    A morphism $B_\infty\to 2$ corresponds to a function 
    $\mathbb N \to 2$ that hits $1$ at most once. 
    We denote $Sp(B_\infty)$ by $\Noo$. 
    For $\alpha:\Noo$ and $n:\N$ we write $\alpha_n$ for $\alpha(g_n)$. 
%    The corresponding Stone space is denoted by $\N_\infty$.
  By conjunctive normal form, 
  any element of $B_\infty$ can be written uniquely as 
  $\bigvee_{i\in I} g_n$ or as $\bigwedge_{i\in I} \neg g_n$ for some finite $I\subseteq \N$. 
  %If $I=\emptyset$, then $\vee_{i\in I} g_i = 0, \bigwedge_{i\in I} \neg g_i = 1$. 
  \end{enumerate}
\end{example}
%  In \Cref{N-co-fin-cp}, we will show 
%  It can be shown that $B_\infty$ is equivalent to the Boolean algebra on 
%  subsets of $\N$ which are finite or co-finite. 
%  Under this equivalence, the generator $g_n$ is sent to the singleton $\{n\}$. 
%  Because of this, we have that any $b:B_\infty$ can be written 

\begin{lemma}\label{ClosedPropAsSpectrum}
  For $\alpha:2^\N$, we have an equivalence of propositions: 
  $$
    (\forall_{n:\N} \alpha_n = 0 )\leftrightarrow Sp(2/(\alpha_n)_{n:\N}).
  $$
\end{lemma}
\begin{proof}
  There is only one boolean morphism $x:2\to 2$, and it satisfies 
  $x(\alpha_n) = 0$ for all $n:\N$ if and only if
  $\alpha_n = 0$ for all $n:\N$. 
  %As $2$ has underlying set $\{0,1\}$, 
 %we have $\alpha(n) =_2 0 $ for all $n:\N$. 
  %we have $(\alpha(n) \neq_2 1) \to (\alpha(n) =_2 0)$. 
\end{proof}


%\begin{remark}
%  As Boolean algebras are rings, any relation of the form $f=g$ with both $f,g$ Boolean expressions 
%  can be written as $h=0$ with $h=f-g$ a Boolean expression. 
%\end{remark} 




\input{Rules/Axioms}
\subsection{Anti-equivalence of $\Boole$ and $\Stone$}
By \Cref{AxStoneDuality}, the map $\Sp$ is an embedding of $\Boole$ into any universe of types. 
We denote its image by $\Stone$. 

\begin{remark}\label{SpIsAntiEquivalence}
Stone spaces will take over the role of the affine schemes from \cite{draft}, 
so let us repeat some results here. 
Analogously to Lemma 3.1.2 of \cite{draft}, 
for $X:\Stone$, \Cref{AxStoneDuality} tells us that $X = \Sp(2^X)$.
%
Proposition 2.2.1 of \cite{draft} now says that 
$\Sp$ gives a natural equivalence 
\[
   \Hom(A, B) = (\Sp(B) \to \Sp(A))
\]
%Therefore $Sp$ is an embedding from $\Boole$ to any universe of types, and $\isSt$ is a proposition.
%
%Its image, 
By the above and Lemma 9.4.5 of \cite{hott}, 
the map $\Sp$ defines a dual equivalence of categories between $\Boole$ and $\Stone$.
In particular the spectrum of any colimit in $\Boole$ is the limit of 
the spectrum of the opposite diagram. 
\end{remark}
\begin{remark}\label{LocalChoiceSurjectionForm}
  \Cref{AxLocalChoice} can also be formulated as follows:
  Given $S:\Stone$ with $E,F$ arbitrary types, a map $f:S \to F$ and a 
  surjection $e:E \twoheadrightarrow F$, 
  there exists a Stone space $T$, a surjective map 
  $T\twoheadrightarrow S$ and an arrow $T\to E$ making the following diagram commute:
    \[\begin{tikzcd}
      T \arrow[d,dashed, two heads ] \arrow[r,dashed]&  E \arrow[d,""',two heads, "e"]\\
      S  \arrow[r, swap,"f"] & F
    \end{tikzcd}\]  
\end{remark}

\begin{lemma}\label{SpectrumEmptyIff01Equal}
  For $B:\Boole$, we have $0=_B1$ if and only if $\neg \Sp(B)$.
\end{lemma}
\begin{proof}
  If $0=_B1$, there is no map in $B\to 2$ preserving both $0$ and $1$, thus $\neg \Sp(B)$. 
  Conversely, if $\neg \Sp(B)$ then 
  $\Sp(B)=\bot$. Since $\bot$ is the spectrum of the trivial Boolean algebra and $\Sp$ is an embedding, we conclude that $B$ is the trivial Boolean algebra, hence $0=_B1$. 
\end{proof}

\begin{corollary}\label{LemSurjectionsFormalToCompleteness}
 For $S:\Stone$, we have that $\neg \neg S \to  \propTrunc{S}$
\end{corollary}
\begin{proof}
  Let $B:\Boole$ and suppose $\neg \neg \Sp(B)$. By \Cref{SpectrumEmptyIff01Equal} we have that $0\not=_B1$, therefore the morphism $2\to B$ is injective. By \Cref{SurjectionsAreFormalSurjections} the map $\Sp(B) \to \Sp(2)$ is surjective, thus $\Sp(B)$ is merely inhabited. 
\end{proof} 

%\begin{corollary}\label{MoreConcreteCompleteness}
%  By the above and propositional completeness, we have that $||\Sp(B)||$ iff $0\neq_B1$. 
%\end{corollary}



%SurjectionsFormalSurjections%We conclude this section on the anti-equivalence of Stone and $\Boole$ by a relating surjections to injections. 
%SurjectionsFormalSurjections%This theorem is actually equivalent to completeness of propositional logic, which we'll discuss in 
%SurjectionsFormalSurjections%\Cref{NotesOnAxioms}. 
%SurjectionsFormalSurjections%
%SurjectionsFormalSurjections%\begin{theorem}\label{FormalSurjectionsAreSurjections}
%SurjectionsFormalSurjections%  Let $f:A\to B$ be a map of countably presented Boolean algebras. 
%SurjectionsFormalSurjections%  If $f$ is injective, then the corresponding map $(\cdot) \circ f : \Sp(B) \to \Sp(A)$ is surjective. 
%SurjectionsFormalSurjections%\end{theorem}
%SurjectionsFormalSurjections%
%SurjectionsFormalSurjections%\begin{proof}
%SurjectionsFormalSurjections%  Assume $f$ injective and let $x:\Sp(A)$.
%SurjectionsFormalSurjections%  By \Cref{FiberConstruction}, we have that $\left(\sum\limits_{y:\Sp(B)} y\circ f = x \right) = \Sp(B/R) $
%SurjectionsFormalSurjections%  for $R=f(G)$ for some countable $G\subseteq A$ with $x(g) = 0$ for all $g\in G$. 
%SurjectionsFormalSurjections%  By propositional completeness and \Cref{SpectrumEmptyIff01Equal}, 
%SurjectionsFormalSurjections%  it's sufficient to show that $0\neq_{B/R}1$. 
%SurjectionsFormalSurjections%  Note that $0=_{B/R} 1$ iff 
%SurjectionsFormalSurjections%  $1 =_B \bigvee R_0$ for some $R_0\subseteq R$ finite. 
%SurjectionsFormalSurjections%  But then $$1 = \bigvee f(G_0) = f(\bigvee  G_0)$$ for some $G_0\subseteq G$ finite. 
%SurjectionsFormalSurjections%  And as $f$ is injective, $\bigvee G_0 = 1$. 
%SurjectionsFormalSurjections%  However, 
%SurjectionsFormalSurjections%  $$
%SurjectionsFormalSurjections%  x(\bigvee G_0) = 
%SurjectionsFormalSurjections%  x(\bigvee_{g\in G_0} g ) = \bigvee_{g \in G_0} x(g) = \bigvee_{g\in G_0} 0 = 0$$
%SurjectionsFormalSurjections%  And as $x(1) = 1$, we get a contradiction. Therefore $0\neq_{B/R} 1$ as required. 
%SurjectionsFormalSurjections%\end{proof}  
%SurjectionsFormalSurjections%The converse to the above theorem is true as well, regardless of propositional completeness:
%SurjectionsFormalSurjections%\begin{lemma}\label{SurjectionsAreFormalSurjections}
%SurjectionsFormalSurjections%If $f:A\to B$ is a map in $\Boole$ and $(\cdot) \circ f :\Sp(B) \to \Sp(A)$ is surjective, 
%SurjectionsFormalSurjections%$f$ is injective. 
%SurjectionsFormalSurjections%\end{lemma}
%SurjectionsFormalSurjections%\begin{proof}
%SurjectionsFormalSurjections%  Suppose precomposition with $f$ is surjective. 
%SurjectionsFormalSurjections%  Let $a:A$ be such that $f(a)= 0$. 
%SurjectionsFormalSurjections%  By assumption, for every $x:A\to 2$, there is a $y:B\to 2$ with $y\circ f = x$. 
%SurjectionsFormalSurjections%  Consequentely $x(a) = y(f(a)) = y(0) = 0$. 
%SurjectionsFormalSurjections%  So $x(a) = 0$ for every $x:\Sp(A)$. 
%SurjectionsFormalSurjections%  Thus $\Sp(A) = \Sp(A/\{a\})$, and as $Sp$ is an embedding, 
%SurjectionsFormalSurjections%  $A \simeq A/\{a\}$, and $a = 0$ in $A$. 
%SurjectionsFormalSurjections%  So whenever $f(a) = 0$, we have $a=0$. Thus $f$ is injective. 
%SurjectionsFormalSurjections%\end{proof}

\subsection{Principles of omniscience}
In constructive mathematics, we do not assume the law of excluded middle (LEM).
There are some principles called principles of omniscience that are weaker than LEM, which can be used to describe 
how close a logical system is to satisfying LEM.
References on these principles include \cite{HannesDiener, ReverseMathsBishop}.
In this section, we will show that two of them (MP and LLPO) hold, 
and one (WLPO) fails in our system.

\begin{theorem}[The negation of the weak lesser principle of omniscience ($\neg$WLPO)]\label{NotWLPO}
  \begin{equation}
    \neg \forall_{\alpha:2^\N} 
    ((\forall_{n:\N} \alpha(n) = 0 ) \vee \neg (\forall_{n:\N} \alpha(n) = 0))
  \end{equation}
%  We cannot decide for general $\alpha:2^\N$, whether $\forall_{n:\mathbb N} \alpha(n) = 0$.
%  It is not the case that the statement %There is no method which given $\alpha:2^\mathbb N$ decides whether 
%  $\forall_{n:\mathbb N} \alpha(n) = 0$ is decidable for general $\alpha:2^\mathbb N$. 
\end{theorem}
\begin{proof}
%  Such a decision method is a function 
  Let $f:2^\mathbb N \to 2$ such that 
  $f(\alpha) = 0$ iff $\forall_{n:\mathbb N} \alpha (n)= 0$. 
  By \Cref{AxStoneDuality}, there is some $c:C$ with 
  $f(\alpha) = 0 \leftrightarrow \alpha(c) = 0$. 
  We can express $c$ using finitely many generators $(g_n)_{n\leq N}$. 
  Now consider $\beta,\gamma:2^\N$ given by 
  $\beta(g_n) = 0$ for all $n:\mathbb N$ and
  $\gamma(g_n) = 0$ iff $n\leq N$. 
  As $\beta, \gamma$ are equal on $(g_n)_{n\leq N}$, we have $\beta(c) = \gamma(c)$. 
  However, $f(\beta) = 0$ and $f(\gamma) = 1$, giving a contradiction as required. 
%  We thus have a contradiction, thus a decision method as required doesn't exist. 
\end{proof}

The following result is due to David W\"arn:
\begin{theorem}[Markov's principle (MP)]\label{MarkovPrinciple}
  For $\alpha:\Noo$, we have that 
  \begin{equation}
    (\neg (\forall_{n:\mathbb N} \alpha (n)= 0)) \to \Sigma_{n:\mathbb N} \alpha (n)= 1
  \end{equation}
\end{theorem}
\begin{proof}
  By \Cref{ClosedPropAsSpectrum}, we have that $\neg(\forall_{n:\N} \alpha(n) = 0)$ implies that 
  $Sp(2/(\alpha(n))_{n:\N}$ is empty. 
%  We will show that the spectrum of $2/(\alpha(n))_{n:\N}$ is empty. 
%  Suppose $x:2\to 2$, if  $x(\alpha(n)) = 0$, we get $\alpha(n) \neq 1$. 
%  Thus if $\neg (\forall_{n:\N} \alpha(n) = 0$, we have $\neg Sp(2/(\alpha(n))_{n:\N})$.
  Hence $2/(\alpha(n))_{n:\N}$ is trivial by \Cref{SpectrumEmptyIff01Equal}. 
  Then there is a finite subset $N_0\subseteq \N$ with $\bigvee_{i:N_0} \alpha(i) = 1$. 
  As $\alpha(i) \in \{0,1\}$ and $\alpha(i) = 1$ for at most one $i:\N$, 
  there exists an unique $n\in\mathbb N$ with $\alpha(n) = 1$. 
%  Assume $\neg (\forall_{n:\mathbb N} \alpha (n)= 0)$.
%  It is sufficient to show that $2/\{\alpha(n)|n\in\N\}$ is the trivial Boolean algebra. 
%  It will then follow that there is a finite subset $N_0\subseteq \N$ 
%  with $\bigvee_{i:N_0} \alpha(i) = 1$.
%  As $\alpha(i) \in \{0,1\}$ and $\alpha(i) = 1$ for at most one $i$, it then follows that 
%  there exists an unique $n\in\mathbb N$ with $\alpha(n) = 1$. 
%%
%  To show that $2/\{\alpha(n)|n\in\N\}$ is trivial, we will show it has an empty spectrum. 
%  Suppose $x: 2 \to 2$ is such that $x(\alpha(n)) = 0$ for every $n:\N$. 
%  As $x(1) = 1$, we must have for every $n:\N$ that $\alpha(n) \neq 1$. 
%  But then $\alpha(n) = 0$, contradicting our assumption. 
%  We get a contradicition and there thus there are no points in the spectrum of $2/\{\alpha(n)|n\in\N\}$ as required. 
\end{proof}

\begin{corollary}
  For $\alpha:2^\mathbb N$, we have that 
  \begin{equation}
    (\neg (\forall_{n:\mathbb N} \alpha (n)= 0)) \to \Sigma_{n:\mathbb N} \alpha (n)= 1
  \end{equation}
\end{corollary}
\begin{proof}
  Given $\alpha:2^\mathbb N$, consider the sequence $\alpha':\Noo$ satisfying $\alpha'(n) = 1$ iff 
  $n$ is minimal with $\alpha(n) = 1$. Then apply the above theorem.
\end{proof}

\begin{theorem}[The lesser limited principle of omniscience (LLPO)]\label{LLPO}
  For $\alpha:\N_\infty$, 
  we have that 
  \begin{equation}\label{eqnLLPO}
    \forall_{k:\N} \alpha(2k) = 0  \vee \forall_{k:\N} \alpha(2k+1) = 0
  \end{equation}
\end{theorem}
\begin{proof}
%
%  We first will define a map $f:B_\infty \to B_\infty \times B_\infty$. 
%  Because of \Cref{rmkMorphismsOutOfQuotient}, it is sufficient to define $f$ on $(p_n)_{n:\N}$ with 
%  $f(p_n) \wedge f(p_m) = (0,0)$ for $n\neq m$. 
%  To define $f(p_n)$, we use a case distinction on whether $n$ is odd or even. 
  Define $f:B_\infty \to B_\infty \times B_\infty$ as follows:
  \begin{equation}\label{eqnLLPOProofMap}
    f(p_n) =\begin{cases}
      (p_k,0) \text{ if } n = 2k\\
      (0,p_k) \text{ if } n = 2k+1\\
    \end{cases}
  \end{equation}
  Note that $f$ is well-defined as map in $\Boole$. 
 % , can make a case distinction on parity. 
%  By making a case distinction on $n,m$ being odd or even, 
%  we can see that 
%  $f(p_n) \wedge f(p_m) = (0,0)$ when $n\neq m$, thus $f$ is well-defined. 
  We claim $f$ is injective. Assume $f(x) = 0$, 
  to see that $x=0$, we make a case distinction on whether $x$ corresponds to a finite or a cofinite set as in \Cref{BinftyTermsWriting}.
%
%  We also claim it is injective.
%  Now let $x:B_\infty$ with $f(x) = 0$. 
  We denote $E,O\subseteq \N$ for the even and odd numbers respectively. 
%  and we make a case distincition based on \Cref{BinftyTermsWriting}.
  \begin{itemize}
    \item Suppose 
      $x = \bigvee_{i\in I_0} g_i$ with $I_0$ finite. 
      Then 
      $$f(x) = (\bigvee_{i\in I_0 \cap E } g_{\frac i2} , \bigvee_{i\in I_0 \cap O } g_{\frac {i-1}2} ) = (0,0)$$
      As $g_j\neq 0$ for all $j\in\N$, we must have $I_0 \cap E = \emptyset = I_0 \cap O$. 
      Thus $I_0= \emptyset$, and $x = 0$. 
    \item Suppose 
%      Let $x$ correspond to a cofinite subset of $\N$. Write 
      $x = \bigwedge_{j\in J} \neg g_j$ with $J$ finite. % for $J$ finite. 
      We will derive a contradiction. %, from which we can conclude that $x=0$ after all. 
      Note that   
      $$f(x) = (\bigwedge_{j\in J \cap E } \neg g_j , \bigwedge_{j\in J \cap O } \neg g_j ) = (0,0)$$
%      As $f(x) = (0,0)$, we have that 
%      $\bigwedge_{j\in J \cap E } \neg p_j =0$ and
%      $\bigwedge_{j\in J \cap O } \neg p_j  = 0$.
      However, any finite meet of negations corresponds to a cofinite set, hence is nonzero. 
      We get a contradiction and conclude $x=0$. 
%      However, any finite meet of negations will correspond to a cofinite set,
%      in particular it will not correspond to the empty set, and thus not be $0$.
%      Thus $f(x)\neq 0$, contradicting the assumption that $f(x) = 0$, hence $x=0$ ex falso. 
  \end{itemize}
%  In both cases, we conclude $x=0$, thus $f$ is injective. 
  By \Cref{SurjectionsAreFormalSurjections},
%  \Cref{FormalSurjectionsAreSurjections}, 
  $f$ corresponds to a surjection 
  $s:\Noo + \Noo \to \Noo$.
  Thus for $\alpha : \Noo$, 
  there exists some $x:\Noo + \Noo$ such that $s x = \alpha$. 
  If $x = inl(\beta)$, 
  for any $k:\N$, we have that 
  $$\alpha (g_{2k+1}) = s(x) (g_{2k+1}) = x(f(g_{2k+1})) = inl(\beta) (0,g_k)  = \beta(0) = 0.$$
  Similarly, if $x = inr(\beta)$, we have $\alpha(g_{2k}) = 0$ for all $k:\N$. 
  Thus \Cref{eqnLLPO} holds for $\alpha$ as required. 
\end{proof}
As the following shows, our use of \Cref{SurjectionsAreFormalSurjections} was non-trivial: 
%The use of \Cref{FormalSurjectionsAreSurjections}, and hence of propositional completeness, 
%was helpful in the above proof, as the following shows:
\begin{lemma}
  The function $f$  as in \Cref{eqnLLPOProofMap} does not have a retraction. 
\end{lemma}
\begin{proof}
  Suppose $r:B_\infty \times B_\infty \to B_\infty$ is a retraction of $f$. 
  Note that $r(0,1):B_\infty$ is expressable using only finitely many generators $(g_n)_{n\leq N}$
  Note that $r(0,1) \geq r(0,g_k) = g_{2k+1}$ for all $k:\N$. 
  As a consequence, $r(0,1)$ cannot be of the form $\bigvee_{i\in I_0} g_i$, and by \Cref{BinftyTermsWriting}, 
  $r(0,1)$ corresponds to a cofinite subset of $\N$. % = \bigwedge_{i:I_0} \neg p_i$, where $i\leq N$ for $i\in I_0$. 
  By similar reasoning so does $r(1,0)$.% corresponds to a cofinite subset of $\N$. 
  But the intersection of cofinite subsets is cofinite, while 
  $$r(0,1) \wedge r(1,0) = r( (1,0) \wedge (0,1)) = r(0,0) = 0$$
  which gives a contradiction. Thus no retraction exists. 
\end{proof}


%We finish with an equivalent formulation of LLPO:
%
%
%\begin{lemma}\label{corAlternativeLLPO}
%  Let $(\phi_n)_{n:\N}, (\psi_m)_{m:\N}$ be families of decidable propositions indexed over $\N$.
%  We then have 
%  \begin{equation}
%    (\forall_{n:\N} \forall_{m:\N} (\phi_n \vee \psi_m) )
%    \leftrightarrow
%    ((\forall_{n:\N} \phi_n) \vee (\forall_{m:\N} \psi_m) )
%  \end{equation}
%\end{lemma}
%\begin{proof}
%  See \cite{HannesDiener, ReverseMathsBishop}
%\end{proof}
%\begin{proof}
%  Note that the implication from right to left in the above equation always holds.
%  Assume that for all $m,n:\mathbb N$ we have $\phi_n\vee \psi_m$ 
%  Consider the sequence $\alpha:2^\mathbb N$ where $\alpha(2n) = 0$ iff $\phi_n$ and 
%  $\alpha(2m+1) = 0$ iff $\psi_m$. 
%  Let $\beta:\Noo$ be such that $\beta(i) = 1$ iff $i$ is minimal with $\alpha(i) = 1$
%  By LLPO, we have that 
%  $\beta$ is $0$ on all odd entries or on all even entries. 
%  Suppose that $\beta$ hits $0$ on all odd entries. 
%  We will show $\psi_m$ for all $m:\N$. 
%  As $\beta(2m+1) = 0$, there are two options:
%  \begin{itemize}
%    	\item If $\alpha(l)=0$ for all $l\leq 2m+1$. Then in particular $\alpha(2m+1)=0$ and $\psi_m$ holds.
%	\item Otherwise there is some $l<2m+1$ with $\beta(l) = 1$. 
%  As $\beta$ hits $0$ on odd entries, $l$ is even. 
%  So $\alpha(2n) = 1$ for $n = \frac{l}2$, meaning that $\neg \phi_n$. 
%  By assumption, $\phi_n \vee \psi_m$ holds, hence $\psi_m$ must hold. 
%  Thus for all $m:\N$, we have $\psi_m$ if $\beta$ hits $0$ on all odd entries. 
%  By a symmetric argument, if $\beta$ hits $0$ on all even entries, we have $\phi_n$ for all $n:\N$. 
%  We conclude that 
%  $((\forall_{n:\N} \phi_n) \vee (\forall_{m:\N} \psi_m) )$ 
%  as required. 
%  \end{itemize}
%\end{proof}
%
%\begin{remark}
%Note that the above statement implies LLPO as $\alpha(2n) =0 \vee \alpha(2m+1) =0$ for all $n,m:\mathbb N$ if $\alpha:\Noo$. 
%\end{remark}

%In this section, we will define the types of open and closed propositions. 
%These will allow us to define a (synthetic) topology  \cite{SyntheticTopologyLesnik} on any type.
%We will study this topology on Stone types in particular.
%
\subsection{Open and closed propositions}
In this section we will introduce a topology on the type of propositions, and 
study their logical properties.
We think of open and closed propositions respectively as countable disjunctions and conjunctions of decidable propositions.
Such a definition is universe-independent, and can be made internally.

\begin{definition}
A proposition $P$ is open (resp. closed) if there exists some $\alpha:2^\N$ such that $P \leftrightarrow \exists_{n:\mathbb N} \alpha_n = 0$ (resp. $P \leftrightarrow \forall_{n:\mathbb N} \alpha_n = 0$). We denote by $\Open$ and $\Closed$ the types of open and closed propositions.
\end{definition}

\begin{remark}\label{rmkOpenClosedNegation}
  The negation of an open proposition is closed, 
  and by MP (\Cref{MarkovPrinciple}), the negation of a closed proposition is open %. 
%  Also by MP, we have 
  and both open, closed propositions are $\neg\neg$-stable. 
%  and $\neg \neg P \to P$ whenever $P$ is open or closed. 
%  By the negation of WLPO (\Cref{NotWLPO}), 
  By $\neg$WLPO (\Cref{NotWLPO}), 
  not every closed proposition is decidable. 
  Therefore, not every open proposition is decidable. 
  % Both therefore and similarly can be used here, by a similar proof we can show it, or we can use that 
  % if $P$ is closed and $\neg P$ is decidable, so is $\neg \neg P = P$. 
  Every decidable proposition is both open and closed.
%  and in \Cref{ClopenDecidable} we shall see the converse. 
\end{remark}
\begin{lemma}
  We have the following results on open and closed propositions:
  \begin{itemize}
%    \item Closed propositions are closed under finite disjunctions. 
    \item Closed propositions are closed under countable conjunctions. 
    \item Open propositions are closed under finite conjunctions. 
    \item Open propositions are closed under countable disjunctions. 
  \end{itemize}
\end{lemma}
\begin{proof}
%  By Proposition 1.4.1 of \cite{HannesDiener}, LLPO (\Cref{LLPO}) is equivalent to the statement that 
%  the disjunction of two closed propositions are closed. 
  The statements have similar proofs, and we only present the proof that closed propositions are closed under 
  countable conjunctions. 
  Let $(P_n)_{n:\N}$ be a countable family of closed propositions. 
  By countable choice, for each 
  $n:\N$ we have an $\alpha_n:2^\N $ 
  such that $P_n \leftrightarrow \forall_{m:\N} \alpha_{n,m} =0$. 
  Consider a surjection $s:\N \twoheadrightarrow \N \times \N$, and let 
%  Let 
%  $$\beta_k = \alpha_{s(k)}.$$
  $\beta_k = \alpha_{s(k)}.$
  Note that $\forall_{k:\N} \beta_k = 0$ if and only if 
%  $\forall_{m,n:\N}\alpha_{m,n} = 0$, which happens if and only if 
  $\forall_{n:\N} P_n$. 
%  Hence the countable conjunction of closed propositions is closed. 
\end{proof}
\begin{remark}
  LLPO (\Cref{LLPO}) is equivalent to the statement that for $P,Q$ open, we have 
  $(\neg P \vee \neg Q) \leftrightarrow \neg (P\wedge Q)$. 
\end{remark}
%\begin{proof}
%  Assuming the above statement, let $\alpha:\Noo$, and consider 
%  the open propositions 
%  $\exists_{k:\mathbb N}\alpha_{2k+1} = 1, \exists_{k:\N} \alpha_{2k} = 1$. 
%  As $\alpha:\Noo$ there's at most one $n:\mathbb N$ with $\alpha_n =1$, so their conjunction is false. 
%  By the above statement, it follows one of them is false, leading to the statement of LLPO. 
%  
%  Now suppose LLPO holds, and assume $\neg (\exists_{n:\mathbb N} \alpha_n = 1\wedge \exists_{n:\mathbb N} \beta_n = 1$
%  for $\alpha,\beta :\Noo$. Then the sequence $\gamma:2^\N$ given by $\gamma_{2n} = \alpha_n, \gamma_{2n+1} = \beta_n$
%  can hit $1$ at most once and thus induces a sequence in $\Noo$. LLPO then shows that 
%  $\neg \forall_{n:\mathbb N} \alpha_n = 1\vee \neg \forall_{n:\mathbb N} \beta_n=1$ as required. 
%\end{proof}
\begin{corollary}
  Closed propositions are closed under finite disjunctions.
\end{corollary}
\begin{proof}
  Closed propositions are negations of open propositions. 
  As the conjunction of two open propositions is open, LLPO gives that 
  the disjunction of two closed propositions is closed. 
\end{proof}
We will use the above properties silently from now on. 
%OneBigLemma#
%OneBigLemma#\rednote{Phrase the following lemmas as one big lemma, 
%OneBigLemma#and use them silently without reference, also we should just state $\neg\neg$-stability instead of referring to the above all the time. }
%OneBigLemma#
%OneBigLemma#\begin{lemma}\label{ClosedCountableConjunction}
%OneBigLemma#  Closed propositions are closed under countable conjunctions.
%OneBigLemma#\end{lemma}
%OneBigLemma#\begin{proof}
%OneBigLemma#  Let $(P_n)_{n:\N}$ be a countable family of closed propositions. 
%OneBigLemma#  By countable choice, for each 
%OneBigLemma#  $n:\N$ we have an $\alpha_n:2^\N $ 
%OneBigLemma#  such that $P_n \leftrightarrow \forall_{m:\N} \alpha_{n,m} =0$. 
%OneBigLemma#  Consider a surjection $s:\N \twoheadrightarrow \N \times \N$, and let 
%OneBigLemma#%  Let 
%OneBigLemma#%  $$\beta_k = \alpha_{s(k)}.$$
%OneBigLemma#  $\beta_k = \alpha_{s(k)}.$
%OneBigLemma#  Note that $\forall_{k:\N} \beta_k = 0$ if and only if 
%OneBigLemma#%  $\forall_{m,n:\N}\alpha_{m,n} = 0$, which happens if and only if 
%OneBigLemma#  $\forall_{n:\N} P_n$. 
%OneBigLemma#  Hence the countable conjunction of closed propositions is closed. 
%OneBigLemma#\end{proof} 
%OneBigLemma#Using similar arguments, we can show the following two lemmas:
%OneBigLemma#\begin{lemma}\label{OpenCountableDisjunction}
%OneBigLemma#  Open propositions are closed under countable disjunctions. 
%OneBigLemma#\end{lemma}
%OneBigLemma#\begin{lemma}\label{OpenFiniteConjunction}
%OneBigLemma#Open propositions are closed under finite conjunctions. 
%OneBigLemma#\end{lemma}
%OneBigLemma#%\begin{proof}
%OneBigLemma#%We use \Cref{ClosedFiniteDisjunction} and the fact that $\neg(P\lor Q) \leftrightarrow \neg P \land \neg Q$.
%OneBigLemma#%\end{proof}
%OneBigLemma#%\begin{proof}
%OneBigLemma#%  Similar to the previous lemma. 
%OneBigLemma#%\end{proof}
%OneBigLemma#\begin{lemma}\label{ClosedFiniteDisjunction} 
%OneBigLemma#  Closed propositions are closed under finite disjunctions. 
%OneBigLemma#\end{lemma}
%OneBigLemma#\begin{proof}
%OneBigLemma#  This statement is equivalent to LLPO (\Cref{LLPO}) by  
%OneBigLemma#  Proposition 1.4.1 of \cite{HannesDiener}. 
%OneBigLemma#%  , LLPO is equivalent to the statement that 
%OneBigLemma#%  for $(\phi_n)_{n:\N}, (\psi_m)_{m:\N}$ families of decidable propositions indexed over $\N$, we have
%OneBigLemma#%  \begin{equation}
%OneBigLemma#%    (\forall_{n:\N} \forall_{m:\N} (\phi_n \vee \psi_m) )
%OneBigLemma#%    \leftrightarrow
%OneBigLemma#%    ((\forall_{n:\N} \phi_n) \vee (\forall_{m:\N} \psi_m) )
%OneBigLemma#%  \end{equation}
%OneBigLemma#%%  $(\forall_{n:\N} \alpha(n) = 0 )\vee (\forall_{n:\N} \beta(n) = 0 )$ is closed for any $\alpha,\beta:2^\N$.
%OneBigLemma#%%  By \Cref{corAlternativeLLPO}, the statement is equivalent to 
%OneBigLemma#%%  $ \forall_{n:\N}  \forall_{m:\N}  (\alpha(n) = 0 \vee \beta(m) = 0)$, 
%OneBigLemma#%  The latter which is a countable conjunction of decidable propositions, 
%OneBigLemma#%  hence closed by \Cref{ClosedCountableConjunction}.
%OneBigLemma#\end{proof}
%OneBigLemma#
\begin{corollary}\label{ClopenDecidable}
  If a proposition is both open and closed, it is decidable. 
\end{corollary}
\begin{proof}
  If $P$ is open and closed, %$\neg P$, and hence 
  $P\vee \neg P$ is open, 
  hence $\neg\neg$-stable and provable. 
%  and we conclude by $\neg\neg$-stability of open propositions. 
%  but open propositions are $\neg\neg$-stable by \Cref{rmkOpenClosedNegation} so we can conclude.
%  hence 
 % equivalent to $\neg \neg (P \vee \neg P)$ by \Cref{rmkOpenClosedNegation}.
 % As the latter proposition is provable, we may conclude $P$ is decidable. 
%  
%  If $P$ is open and closed, $P\vee \neg P$ is open, hence
%  equivalent to $\neg \neg (P \vee \neg P)$, which is provable. 
\end{proof}


%\begin{lemma}\label{OpenFiniteConjunction}
%  Open propositions are closed under finite conjunctions. 
%\end{lemma}
%\begin{proof}
%  We need to show that for any $\alpha,\beta:2^\N$, the following proposition is open:
%  \begin{equation}\label{eqnConjunctionOpen}
%    (\exists_{n:\N} \alpha(n) = 0 )\wedge(\exists_{n:\N} \beta(n) = 0 )
%  \end{equation}
%  Consider $\gamma:2^\N$ given by 
%  $\gamma(l) = 1$ iff there exist some $k,k'\leq l$ with 
%  $\alpha(k) = \beta(k') = 0$. 
%  As we only need to check finitely many combinations 
%  of $k,k'$, this is a decidable property for each $l:\N$ and $\gamma$ is well-defined. 
%  Then $\exists_{k:\N}\gamma(k)=0$ if and only if \Cref{eqnConjunctionOpen} holds.
%\end{proof}

\begin{lemma}\label{ClosedMarkov}
  For $(P_n)_{n:\N}$ a sequence of closed propositions, we have 
  $\neg \forall_{n:\N} P_n \leftrightarrow  \exists_{n:\N} \neg P_n$. 
\end{lemma}
\begin{proof}
  Both $\neg \forall_{n:\N} P_n$ and $\exists_{n:\N} \neg P_n$ are open, hence $\neg\neg$-stable.
  The equivalence follows. 
%  and the equivalence follows. 
%  from which the equivalence follows. 
%  We have that $\forall_{n:\N}P_n$ is closed and $\exists_{n:\N} \neg P_n$ is open by \Cref{OpenCountableDisjunction}, therefore both are $\neg\neg$-stable by \Cref{rmkOpenClosedNegation} and we can conclude.
%It is always the case that $\exists_{n:\N}\neg P_n \to \neg \forall_{n:\N} P_n$. 
  %For the converse direction,
  %note that $\neg \exists_{n:\N} \neg P_n(x) \to \forall_{n:\N} \neg \neg P_n(x).$
  %By \Cref{rmkOpenClosedNegation}, $\neg \neg  P_n(x)\leftrightarrow P_n(x)$ for all $n:\N$. 
  %It follows that 
  %$\neg \forall_{n:\N} P_n(x)\to 
  %\neg \neg \exists_{n:\N} \neg P_n(x).$
  %As $\exists_{n:\N}\neg P_n(x)$ is a countable disjunction of open propositions, 
  %it is open by \Cref{OpenCountableDisjunction} and thus equivalent to 
  %$\neg\neg\exists_{n:\N} \neg P_n(x)$ by \Cref{rmkOpenClosedNegation}.
  %We conclude that $\neg \forall_{n:\N} P_n \to \exists_{n:\N} \neg P_n$ as required. 
\end{proof} 

%\begin{lemma}\label{OpenDependentSums}
%  Open propositions are closed under dependent sums.
%\end{lemma}
%\begin{proof}
%  \rednote{If we show that Open propositions are exactly the overtly discrete ones, this is implied by $\Sigma$-closure}
%  First note that for $D$ a decidable proposition, and $X:D \to \Open$,
%  by case splitting on $D$, we can see 
%  $\Sigma_{d:D} X(d)$ is open.
%%
%  Then note that for $P$ an open proposition, 
%  there exists a sequence of decidable propositions $A_n$ with 
%  $P = \exists_{n:\N} A_n $.
%%
%  So for $Y : P \to Open $, the dependent sum $\Sigma_P Y$ is given by 
%  $\exists_{n:\N} (\Sigma_{a:A_n} Y(n,a))$,
%  which is a countable disjunction of open propositions, 
%  hence open by \Cref{OpenCountableDisjunction}.
%\end{proof}
%
%We will see the same holds for closed propositions in \Cref{ClosedDependentSums}.
%
%\begin{remark}\label{ImplicationOpenClosed}
%  If $P$ is open, $P \to \bot$ is only open if $P$ is decidable, which is not in general the case. 
%  Thus $\Open$ is not closed under dependent products. Neither is $\Closed$. 
%  However, as $(P\to Q)  \to \neg \neg (\neg P \vee Q)$,
%  we have that if $P$ is open and $Q$ is closed, then $P\to Q$ is closed, and similarly $Q\to P$ is open.
%\end{remark}
\begin{lemma}\label{ImplicationOpenClosed}
  If $P$ is open %(resp. closed) 
  and $Q$ is closed % (resp. open) 
  then $P\to Q$ is closed. % (resp. open). 
  If $P$ is closed and $Q$ open, then $P\to Q$ is open. 
\end{lemma}
\begin{proof}
  Note that $\neg P \vee Q$ is closed. Using $\neg\neg$-stability
  we can show $(P\to Q) \leftrightarrow (\neg P \vee Q)$. 
  The other proof is similar. 
%  and we conclude by 
%  Assume $P$ open and $Q$ closed, the other proof is similar. 
%  Note that $(\neg P \vee Q) \to (P \to Q)$ and 
%  $(P\to Q)\to \neg\neg(\neg P \vee Q)$. 
%  By \Cref{rmkOpenClosedNegation} it follows that 
%  $(\neg P \vee Q)\leftrightarrow (P \to Q)$, and using \Cref{ClosedFiniteDisjunction}, 
%  we can conclude that $P\to Q$ is closed. 
\end{proof}
%
%The following question was asked by Bas Spitters at TYPES 2024:


\subsection{Types as spaces}
The subset $\Open$ of the set of propositions induces a topology on every type. 
This is the viewpoint taken in synthetic topology, from which we borrow terminology \cite{SyntheticTopologyEscardo, SyntheticTopologyLesnik}. 
%other references include \cite{SyntheticTopologyEscardo}%, TODOSortOutTaylorsReferences}.
%Defining a topology in this way has some benefits, which we summarize in this section. 

\begin{definition}
  Let $T$ be a type, and let $A\subseteq T$ be a subtype. 
  We call $A\subseteq T$ open (resp. closed) if $A(t)$ is open (resp. closed) for all $t:T$.
\end{definition}

\begin{remark}
  It follows immediately that the pre-image of an open by any map is open, so that any map is continuous. 
%  This is only relevant for a space if the topology we defined above matches the topology one would expect. 
  In \Cref{StoneClosedSubsets}, we will see that the resulting topology is as expected for Stone spaces.
  In \Cref{IntervalTopologyStandard}, we will see that the same holds for the unit interval. 
\end{remark}



%\begin{remark}
%  Phao's principle is a special case of directed univalence. 
%\end{remark}
%\begin{proof}
%  \rednote{TODO}
%\end{proof}


\section{Overtly discrete spaces}
%\input{OvertlyDiscrete/ColimitRepresentation.tex}
\begin{definition}
  We call a type overtly discrete iff it can be described as 
  the colimit of an $(\N,\leq)$-indexed sequence of finite sets. 
\end{definition} 
\begin{remark}
  If we denote an overtly discrete type by $B$, we will denote the objects of the underlying sequence as 
  $B_n,~n:\N$, and for $n\leq m$, we denote the maps $B_n \to B_m$ with 
  Greek letters with lower index $n$ and upper index $m$. 
  So for example $\iota_n^m:B_n \to B_m$. 
  The maps $B_n \to B$ will in this case be denoted $\iota_n$. 
  If convenient, given a sequence $B_n$, we will denote $B_\infty$ for the colimit $B$.
\end{remark}
\begin{lemma}
  Every countably presented Boolean algebra is overtly discrete.
\end{lemma}
\begin{proof}
  Consider a countably presented Boolean algebra of the form $B = 2[\N]/(r_n)_{n:\N}$. 
%  We will show there exists a diagram of shape $\N$ taking values in Boolean algebras 
%  with $B$ as colimit.
%  \paragraph{The diagram}
  For each $n:\N$, let $G_n$ be the union of $\{g_i|{i\leq n}\}$ and 
  the finite set of terms occurring in $(r_i)_{i\leq n}$. 
  Denote $B_n = 2[G_n]/(r_i)_{i\leq n}$, and note that each $B_n$ is a finite Boolean algebra, 
  and we have maps $B_n \to B_{n+1}$.
%  The inclusion $G_n \hookrightarrow G_{n+1}$ induces maps $B_n \to B_{n+1}$.
  Hence $B_n,~n:\N$ is an $(\N,\leq)$-indexed sequence of finite sets. 
  We claim that $B$ is the colimit of this sequence. 
%
%  \paragraph{The colimit}
%  As $G_n\subseteq G$ and $R_n \subseteq R$, 
%  \Cref{rmkMorphismsOutOfQuotient} also gives us a map $B_n\to \langle G \rangle \langle R \rangle$. 
%  We claim the resulting cocone is a colimit. 
%
%  Suppose we have a cocone $C$ on the diagram $(B_n)_{n\in\N}$. 
%  We need to show that there exists a map $\langle G \rangle / R\to C$ and
%  we need to show this map is unique as map between cocones. 
%  \begin{itemize}
%    \item To show there exists a map $\langle G \rangle / R \to C$, 
%      we use remark \Cref{rmkMorphismsOutOfQuotient} again. 
%      Let $g\in G$ be the $n$'th element of $G$, 
%      note that $g\in G_n$, and consider the image of $g$ under the map $B_n \to C$. 
%      This procedure defines a function from $G$ to the underlying set of $C$. 
%      Let $\phi \in R$ be the $n$'th element of $R$, 
%      note that $\phi \in R_n$, and the map $B_n \to C$ must send $\phi$ to $0$. 
%      Thus the function from $G$ to the underlying set of $C$ also sends $\phi$ to $0$. 
%      This thus defines a map $\langle G \rangle / R \to C$. 
%    \item To show uniqueness, consider that any map of cocones $\langle G \rangle / \langle R \rangle \to C$ 
%      must take the same values on all $g\in G_n$ for all $n\in\N$. 
%      Now all $g\in G$ occur in some $G_n$, so any map of cocones $\langle G \rangle /  \langle R \rangle \to C$ 
%      takes the same values for all $g\in G$. 
%      \Cref{rmkMorphismsOutOfQuotient} now tell us that these values uniquely determine the map. 
%  \end{itemize}
\end{proof}
 
\begin{lemma}\label{colimitCompact}
  For any finite set $A$ and $(\N,\leq)$-indexed sequence of finite sets $B_n$ with colimit $B$, 
  the colimit of $B_n^A$ is $B^A$. 
%  and any overtly discrete type $B$ as colimit of the sequence $(B_n)_{n:\N}$, 
%  $(colim(B_n))^A \simeq colim (B_n^A)$
%  $B^A$ is the colimit of the sequence of sets $(B_n^A)_{n:\N}$. 
\end{lemma}  
\begin{proof}
%  First note that $B^A$ forms a cocone on $(B_n^A)_{n:\N}$ 
  Any map $A \to B_n$ induces a map $A \to B$, hence $B^A$ is a cocone on $(B_n^A)_{n:\N}$.
  Let $C$ form a cocone on $(B_n^A)_{n:\N}$. %with maps $F_n:B_n^A \to C$.
  For any $f:A \to B$, the finite image $f(A)$ must already occur in some $B_n$, 
  thus there is some $f':A\to B_n$ with $\iota_n\circ f' = f$.% occurs as some map $A\to B_n$, 
  As $C$ is a cocone, $f'$ corresponds to some $c$ in $C$, and this term does not depend on $n$. 
  Also any map $B^A \to C$ respecting the cocone conditions must send $f$ to $c$, 
  hence $B^A$ is indeed the colimit. 
%
%
%
%%  We shall show there is an unique morphism of cocones $B^A \to C$. 
%%  Denote $\iota_n:B_n \to B, F_n:B_n^A\to C$ for the cocone maps. 
%  \begin{itemize}
%    \item 
%      If $f:A\to B$, we have that $f(A)$ is a finite subset of $B$, and thus occurs already in some $B_n$. 
%      This induces a map $f'_n:A\to B_n$ with $\iota_n\circ f'_n = f$. 
%      As $C$ is a cocone, we have that $F_n(f'_n)$ does not depend on $n:\N$. 
%%      Thus $(F_n)_{n:\N}$ induces a map 
%      Thus we get a map $B^A\to C$. 
%    \item 
%      For uniqueness, by function extensionality maps $B^A \to C$ are uniquely determined by their values on 
%      $f:B^A$. By the above, the value of $f$ is uniquely determined by it's value on $B_n$ for 
%      any $n$ with the image of $f$ in $B_n$. Thus there is at most one morphism of cocones $B^A \to C$. 
%  \end{itemize}
\end{proof}
\begin{remark}\label{rmkEqualityColimit}
  In the above proof, we used that any element $b\in B$ already occurs in some $B_n$. 
  However, please note that it is not necessarily the case that it occurs uniquely in $B_n$.
%  there might be multiple elements in $B_n$ which can all be sent to $b$ in the end. 
%
  In general, if we assume $B$ is overtly discrete %is the colimit of any sequence $(B_n)_{n:\N}$, 
  and there exist some $B_n$ with two elements corresponding to the same element in $B$, 
  Theorem 7.4 from \cite{SequentialColimitHoTT} says that there merely exists some $m\geq n$
  such that these elements become equal in $B_m$. 
%  In general, if we have a sequence $(B_n)_{n:\N}$ with colimit $B$ and 
%  inclusion maps $\iota_n:B_n\to B$ and $\iota_n^m:B_n \to B_m$ for $n\leq m$, 
%  and there exist some $b,c:B_n$ with $\iota_n(b) = \iota_n(c)$, 
%  Theorem 7.4 from \cite{SequentialColimitHoTT} says that there merely exists some $m\geq n$
%  with $\iota_n^m(b) = \iota_n^m(c)$. 
\end{remark}
\begin{lemma}\label{lemDecompositionOfColimitMorphisms}
  Let $B,C$ be overtly discrete, 
  and let $f:B\to C$.
  There exists $(\N,\leq)$-indexed sequences of finite sets 
  $(B_n)_{n:\N}, (C_n)_{n:\N}$ with colimits $B,C$ respectively
  and compatible maps $f_n:B_n \to C_n$, 
  such that $f$ is the induced morphism $B\to C$.
\end{lemma}
\begin{proof}
  Let $(B_n)_{n:\N}, (C_n)_{n:\N}$ be 
  sequences of finite sets with colimits $B$ and $C$. 
  We will construct a subsequence of $(C_n)_{n:\N}$, using \Cref{axDependentChoice}.
%
  For $k:\N$, let $E_k$ consist of 
  strictly increasing sequences $(n_i)_{i<k}$ of natural numbers together with a finite family of maps 
  $(f_i: B_{i} \to C_{n_i})_{i<k}$ such that
  for all $0<i<k$ the following diagram commutes:
  \begin{equation}\label{eqnDecompositionOfColimitMorphisms}
    \begin{tikzcd}
      B_{i-1} \arrow[r,"\iota_{i-1}^i"] \arrow[d, "f_{i-1}"]& B_{{i}} \arrow[r, "\iota_i"] 
      \arrow[d,"f_{i}"]& B \arrow[d,"f"] \\
      C_{n_{i-1}} \arrow[r,"\kappa_{i-1}^i"] & C_{n_{i}} \arrow[r,"\kappa_i"] & C 
    \end{tikzcd}
  \end{equation}
  For $e:E_k, e':E_{k+1}$, we let 
  $R_k(e,e')$ denote  whether the underlying sequences of $e'$ extends that of $e$. 
  The empty sequence inhabits $E_0$. We will show that if $e:E_k$, there exists some $e':E_{k+1}$ with 
  $R_k(e,e')$. Then \Cref{axDependentChoice} will give the required sequence $(f_i:B_i\to C_{n_i})$.
%%
%  By \Cref{colimitCompact} 
%  there exists some $n_0:\N$ 
%  such that $B_0 \to B \to C$ factors as 
%  \begin{equation}
%    \begin{tikzcd}
%      B_{0} \arrow[r] \arrow[d, "f_0"]& B \arrow[d,"f"] \\
%      C_{n_0} \arrow[r] & C 
%    \end{tikzcd}
%  \end{equation}
%  Because our goal is a proposition, we can untruncate this existence to data. 
%  This data will form our $x_0:E_0$. %from \Cref{axDependentChoice}. 
%%

  Suppose we have $(f_i: B_{i} \to C_{n_i})_{i<k}$ for some $k\geq 0$ 
  such that for all $0<i<k$ the diagram of \Cref{eqnDecompositionOfColimitMorphisms} commutes.
  We shall show that in this case there exist $n_{k}:\N, f_{k}:B_k \to C_{n_k}$ 
  making the same diagram commute for $i = k$. 
  Consider the map $f\circ \iota_k: B_{k}\to C$. 
  As $B_k$ is finite, \Cref{colimitCompact} gives some $n_k':\N $ such that %, f_k':B_k \to C_{n_k'}$ such that 
  it factors over some $C_{n_k'}$.
%  \begin{equation}
%    \begin{tikzcd}
%    B_{k} \arrow[r,"\iota_k"]  \arrow[d,"f'_{k}"]& B \arrow[d,"f"]\\
%    C_{n'_{k}} \arrow[r, "\kappa_{n'_k}"'] & C
%    \end{tikzcd}
%  \end{equation}
  We may assume $n'_{k+1} > n_k$.
  Note that it is not necessarily the case that 
  $f'_{k} \circ \iota_{k-1}^k = \kappa_{n_{k-1}}^{n'_k}\circ f_{k-1}$. 
%  $f'_{k+1}$ is compatible with $f_k$, meaning the left square in the following diagram needn't commute:
%  \begin{equation}
%    \begin{tikzcd}
%      B_{k-1} \arrow[r] \arrow[d, "f_{k-1}"]& B_{{k}}  \arrow[r] \arrow[d,"f'_{k}"] & B \arrow[d,"f"] \\
%      C_{n_{k-1}} \arrow[r] & C_{n'_{k}} \arrow[r]  & C 
%    \end{tikzcd}
%  \end{equation}
  However, both $f'_{k}, f_{k-1}$ induce the same map $B_{k-1} \to C$. 
%  Recall by \Cref{rmkMorphismsOutOfQuotient} this map is induced by it's value on finitely many elements. 
  As $B_{k-1}$ is finite, from \Cref{rmkEqualityColimit} it follows there is some $n_{k} \geq {n'_{k}}$ 
  such that they become equal in $C_{n_k}$, and we have $f_k:B_k \to C_{n_k}$ such that the following does commute, 
  and we're done:
%  such that for $f_{k}$ the composition of $f'_{k+1}:B_{k+1} \to C_{n'_{k+1}}$ and 
%  the map $C_{n'_{k+1}} \to C_{n_{k+1}}$, the following diagram does commute:
  \begin{equation}
    \begin{tikzcd}
      B_{k-1} \arrow[d,"f_{k-1}"]\arrow[r] & B_{{k}} \arrow[rd, "f_{k}"] \arrow[rr] & & B \arrow[d,"f"] \\
      C_{n_{k-1}} \arrow[r] & C_{n'_{k}} \arrow[r] & C_{n_{k}} \arrow[r] & C 
    \end{tikzcd}
  \end{equation}
%  Now by dependent choice for the above $x_0, R_n, E_n$, we get a sequence $(f_i:B_i \to C_{n_i})$  for some 
%  strictly increasing sequence $n_i$ of natural numbers. 
%  Note that for such a sequence $(n_i)_{i:\N}$, 
%  $(C_{n_i})_{i:\N}$ converges to $C$. Also $(B_i)_{i:\N}$ still converges to $B$. 
%  Furthermore, by construction the map that sequence $f_i$ induces from $B \to C$ shares all values with $f$
%  and thus is equal to $f$. 
%  Thus our sequence $f_i$ is as required. 
\end{proof}

\begin{lemma}\label{lemDecompositionOfEpiMonoFactorization}
  Let $f:A_\infty\to B_\infty$ be a map between overtly discrete types, and suppose we have $f_n:A_n\to B_n$ such that 
  the following diagram commutes:
  \begin{equation}
    \begin{tikzcd}
      A_n \arrow[d,"f_n"]\arrow[r, "\iota_n^m"]  & A_m \arrow[d,"f_m"] \arrow[r,"\iota_m^\infty"]  & A_\infty \arrow[d,"f"] 
      \\
      B_n \arrow[r, "\kappa_n^m"'] & B_m \arrow[r,"\kappa_m^\infty"'] & B_\infty
    \end{tikzcd}
  \end{equation}
  Then $f(A)$ is the colimit of $f_n(A_n)$, 
  and the maps $A\twoheadrightarrow f(A)$ and $f(A) \hookrightarrow B$ 
  are induced by the maps $A_n\twoheadrightarrow f_n(A_n)$ and $f_n(A_n) \hookrightarrow B_n$ respectively. 
\end{lemma}
\begin{proof}
  For $n\leq m$, we have that $\kappa_n^m(f_n(A_n)) = f_m(\iota_n^m(A_n))\subseteq f_m(A_m)$, 
  hence we can take the corestriction of the map $f_n(A_n) \to B_m$ to $f_m(A_m)$ to get 
  maps $\lambda_n^m :f_n(A_n) \to f_m(A_m)$ making the following diagram commute:
  % https://q.uiver.app/#q=WzAsOSxbMSwwLCJBX20iXSxbMiwwLCJBX1xcaW5mdHkiXSxbMSwyLCJCX20iXSxbMiwyLCJCX1xcaW5mdHkiXSxbMiwxLCJmKEEpIl0sWzEsMSwiZl9tKEFfbSkiXSxbMCwwLCJBX24iXSxbMCwyLCJCX24iXSxbMCwxLCJmX24oQV9uKSJdLFswLDFdLFsyLDNdLFsxLDQsIiIsMCx7InN0eWxlIjp7ImhlYWQiOnsibmFtZSI6ImVwaSJ9fX1dLFs0LDMsIiIsMSx7InN0eWxlIjp7InRhaWwiOnsibmFtZSI6Imhvb2siLCJzaWRlIjoidG9wIn19fV0sWzAsNSwiIiwyLHsic3R5bGUiOnsiaGVhZCI6eyJuYW1lIjoiZXBpIn19fV0sWzUsMiwiIiwxLHsic3R5bGUiOnsidGFpbCI6eyJuYW1lIjoiaG9vayIsInNpZGUiOiJ0b3AifX19XSxbNywyLCJcXGthcHBhX25ebSIsMl0sWzYsMCwiXFxpb3RhX25ebSJdLFs2LDgsIiIsMix7InN0eWxlIjp7ImhlYWQiOnsibmFtZSI6ImVwaSJ9fX1dLFs4LDcsIiIsMSx7InN0eWxlIjp7InRhaWwiOnsibmFtZSI6Imhvb2siLCJzaWRlIjoidG9wIn19fV0sWzgsNSwiIiwxLHsic3R5bGUiOnsiYm9keSI6eyJuYW1lIjoiZGFzaGVkIn19fV0sWzUsNCwiIiwxLHsic3R5bGUiOnsiYm9keSI6eyJuYW1lIjoiZG90dGVkIn19fV1d
  \begin{equation}\label{eqnEpiMonoFactorizationDecomposition}
    \begin{tikzcd}
    {A_n} & {A_m} & {A_\infty} \\
    {f_n(A_n)} & {f_m(A_m)} & {f(A_\infty)} \\
    {B_n} & {B_m} & {B_\infty}
    \arrow["{\iota_n^m}", from=1-1, to=1-2]
    \arrow[two heads, from=1-1, to=2-1,"e_n"]
    \arrow[from=1-2, to=1-3,"\iota_m^\infty"]
    \arrow[two heads, from=1-2, to=2-2,"e_m"]
    \arrow[two heads, from=1-3, to=2-3,"e"]
    \arrow[dashed, from=2-1, to=2-2, "\lambda_n^m"]
    \arrow[hook, from=2-1, to=3-1, "i_n"]
    \arrow[dashed, from=2-2, to=2-3, "\lambda_m^\infty"]
    \arrow[hook, from=2-2, to=3-2,"i_m"]
    \arrow[hook, from=2-3, to=3-3,"i_\infty"]
    \arrow["{\kappa_n^m}"', from=3-1, to=3-2]
    \arrow[from=3-2, to=3-3,"\kappa_m^\infty"']
  \end{tikzcd}
\end{equation}
  To see that $f(A_\infty)$ is colimiting, note that whenever $b$ occurs in $f(A_\infty)$ as 
  $f(a)$ for some $a:A_\infty$, there is some $n:\N$ and $a':A_n$ with $\iota_n(a') = a$ and 
  $\kappa_n(e_n(a')) = \kappa_n(f_n(a')) = b$, so there is a unique extension from $f(A_\infty)$ into any other cocone. 
\end{proof}
\begin{corollary}
  In \Cref{lemDecompositionOfColimitMorphisms}, when $f$ is injective or surjective, 
  we can choose presentations such that each $f_n$ is also injective or surjective respectively. 
\end{corollary}
\begin{proof}
  Using \Cref{lemDecompositionOfColimitMorphisms} and \Cref{lemDecompositionOfEpiMonoFactorization}, 
  we get a factorization as in \Cref{eqnEpiMonoFactorizationDecomposition}. 
  If $f$ is injective, then $e$ is an isomorphism. 
  Hence $A$ is the colimit of $f_n(A_n)$, and we can take $f_n' = i_n$.
  Similarly, if $f$ is surjective $i$ is an isomorphism and we consider $B$ as colimit of $f_n(A_n)$ and 
  take $f_n' = e_n$.
\end{proof}



%
%\begin{lemma}
%  Let $A,B,C$ be overtly discrete, represented by sequences $(A_n)_{n:\N}, (B_n)_{n:\N},(C_n)_{n:\N}$. 
%  Let $u,v,(u_n)_{n:\N},(v_n)_{n:\N}$ be as in the following diagram of Boolean algebras:
%  \begin{equation}
%    \begin{tikzcd}
%      A_\infty \arrow[r,"u_\infty"] & B_\infty  \arrow[r,"v_\infty"] & C_\infty
%      \\
%      A_n \arrow[u,"i_n^\infty"] \arrow[r,"u_n"] & B_n \arrow[u,"j_n^\infty"] \arrow[r,"v_n"] & C_n \arrow[u,"k^\infty_n"]
%      \\
%      A_m \arrow[u,"i_m^n"] \arrow[r,"u_m"] & B_m \arrow[u,"j_m^n"] \arrow[r,"v_m"] & C_m \arrow[u,"k_m^n"]
%    \end{tikzcd} 
%  \end{equation} 
%  Furthermore, assume that $v\circ u = 0$ and $v_n \circ u_n = 0$ for all $n:\N$.
%  Then $Ker(v)/Im(u)$ is the colimit of the sequence $Ker(v_n)/Im(u_n)$. 
%\end{lemma}
%\begin{proof}
%  First, we will note what the maps in this sequence are, which by some abuse of notation also gives
%  the cocone maps. 
%\paragraph{If $n\leq m$, there are maps $Ker(v_n)/Im(u_n)\to Ker(v_m) / Im(u_m)$.}
%Let $x,y\in B_n, a \in A_n$ be such that $x - y  = u_n(a)$, 
%then $j_n^m(x) - j_n^m(y) = j_n^m(u_n(a)) = u_m(i_n^m(a))$.
%Thus whenever $x,y\in B_n$ are such that $x \sim_{Im(u_n)} y$, we have that 
%$j_n^m(x) \sim_{Im(u_m)} j_n^m(y)$. 
%
%%
%Furthermore, if $x\in Ker(v_n)$, then $v_n(x) = 0$, thus 
%\begin{equation}
%  v_m(j_n^m(x)) = k_n^m(v_n(x)) = k_n^m(0) = 0
%\end{equation} 
%and hence $j_n^m(x) \in Ker(v_m)$. 
%Thus $j_n^m$ induces a map $\iota_n^m:Ker(v_n)/Im(u_n) \to Ker(v_m)/Im(u_m)$, 
%with $\iota^n_m([x]) = [j_n^m(x)]$ for $x\in Ker(v_n)$. 
%
%\paragraph{These maps are compatible (in particular, the maps $j_n^\infty$ form a cocone).}. 
%If $k\leq n \leq m$, we have that $j_n^m \circ j_k^n = j_k^m$.
%We thus have that $\iota_n^m \circ \iota_k^n = \iota_k^m$.
%
%\paragraph{Given any cocone $\kappa_n : Ker(v_n)/Im(u_n)\to K$, 
%  there exists an extension $\kappa_\infty(v_\infty)/Im(u_\infty)\to K$}
%      If $\kappa_n$ forms a cocone, this means that for $n\leq m$ we have 
%      $\kappa_m = \kappa_n \circ \iota_m^n$.
%
%      We shall give a map $\kappa_\infty:Ker(v_\infty)/Im(u_\infty) \to K$ satisfying 
%      $\kappa_\infty \circ \iota_n^\infty= \kappa_n$ for all $n:\N$.
%      We're going to define a map $k:Ker(v_\infty) \to K$.
%%%
%      Let $x\in Ker(v_\infty)$. Then $x\in B_\infty$ and $v\infty(x) = 0$. 
%      As $B_\infty$ is the colimit of the sequence $B_n$, 
%      there is some $n:\N$ and some $x':B_n$ with $v_n(x') = 0$. 
%      We'd like to define $k(x) = \kappa_n([x'])$. We need to check this definition doesn't depend on $n$. 
%      \begin{itemize}
%        \item \textbf{$k$ is well-defined} 
%      Assume $n\leq m$ are such that we have $x':B_n, x'':B_m$ with $v_n(x') = 0, v_m(x'') = 0$ 
%      and $j_n(x') = j_m(x'') = x$. 
%      Then there exists some $l\geq m\geq n$ with 
%      $j_n^l (x') = j_m^l(x'')$, hence 
%      $$\iota_n^l[x'] = \iota_m^l[x'']$$ and thus 
%      $$\kappa_l(\iota_n^l[x']) = \kappa_l(\iota_m^l[x''])$$
%      But now $\kappa_l\circ \iota_n^l = \kappa_n$ and $\kappa_l\circ \iota_m^l = \kappa_m$, hence 
%      $$\kappa_n([x']) = \kappa_m([x''])$$
%      Thus $k$ is well-defined. 
%      \item \textbf{$k$ respects $Im(u)$}
%        Let $x,y:B$ and let $a:A$ be such that are such that $x-y = u(a)$ and $v(x) = v(y) = 0$.
%        Then there is some $n:\N$, and some $x',y':B_n, a':A_n$ with $x'-y'= u_n(a'), v_n(x') = v_n(y') = 0$ 
%        and $j_n(x') = x, j_n(y') = y, i_n(a') = a$. 
%        Hence $[x'] = [y']$, hence $\kappa_n([x']) = \kappa_n([y'])$.
%        Thus $k(x) = k(y)$. 
%      \end{itemize}
%      We conclude that $k$ induces a map $\kappa_\infty:Ker(v)/Im(u) \to K$. 
%    \paragraph{$\kappa$ is colimiting}
%    Let $\kappa_n$ be as above, and suppose that 
%    $\lambda \circ \iota_n^\infty = \kappa_n$. 
%    Then for all $n:\N$, $x':B_n$ such that $j_n(x) = x$, we have 
%    $$\lambda ([x]) = \lambda \circ \iota_n([x']) = \kappa_n([x']) = k(x)$$
%    As such $n,x'$ always exist, it follows that 
%    $\lambda([x]) = k(x)$, hence $\lambda[x] = \kappa[x]$, so $\lambda = \kappa$ as required. 
%\end{proof} 
%
%
%
%







%%\begin{remark}
%%  For any overtly discrete type $B$, we can choose a presentation of the underlying sequence 
%%  $B_n' = \iota_n(B_n)$ such that $B$ is the colimit of $B_n'$, and the inclusion maps $\iota'_n$ are injective. 
%%  Applying the above lemma to some $B$ of this form, we see that if $f$ is injective, so are all the $f_n$. 
%%%  
%%%
%%%  For the above lemma, if $f$ is injective, so are all the $f_n$. This isn't clear at all. Consider 
%%%  the sequence of $\{x,y\} \to \{x\} \to \{x,y\} \to \{x \} \to \cdots$ with colimit $\{x\}$. 
%%\end{remark}
%\begin{remark}
%  Let $B:\Boole$. 
%  Recall that $B$ can be seen as the colimit of $(B_n)_{n:\N}$ with $B_n$ finite Boolean algebras. 
%  Now in the category of Boolean algebras, we have $(B\to 2) \simeq lim(B_n\to 2)$.
%  As $B_n\to 2$ is finite whenever $B_n$ is, it follows that Stone spaces are limits of $\N$-indexed diagrams of finite sets. 
%\end{remark}
%
%
%\begin{remark}\label{rmkEpiMonoFactorizationCommutes}
%  For $f,(f_i)_{i:\N}$ as above, whenever $f_n(x) = 0$, we have $f_{n+1}(x \circ \iota_{n,n+1}) = 0$
%  for $\iota_{n,n+1}$ the map $A_n \to A_{n+1}$. 
%  By \Cref{rmkMorphismsOutOfQuotient}, $\iota_{n,n+1}$ induces a map $A_n/Ker(f_n)\to A_{n+1}/Ker(f_{n+1})$. 
%  This induced map is such that the following diagram commutes:
%  \begin{equation}\begin{tikzcd}
%    A_n \arrow[d, two heads] \arrow[r, "\iota_{n,n+1}"] & A_{n+1} \arrow[d,two heads]\\
%    A_n /Ker(f_n) \arrow[d,hook] \arrow[r] & A_{n+1} /Ker(f_{n+1}) \arrow[d,hook] \\
%    B_n \arrow[r] & B_{n+1}
%  \end{tikzcd}\end{equation}  
%  As the induced maps be epi's / mono's  is epi /mono, the colimit of the sequence 
%  $A_n / Ker(f_n)$ will fit into an epi-mono factorization of $f$ and thus be iso to $A/Ker(f)$. 
%  Thus the epi-mono factorization of the colimit is the colimit of the epi-mono factorizations. 
%\end{remark}
%\begin{remark}\label{rmkIsoEpiMonoMapColimit}
%  Whenever $f:B \to C$ is an iso, any sequence with $B$ as colimit, also has $C$ as colimit. 
%  Thus any iso can be represented this way as sequence of iso's. 
%  Conversely, any sequence of isomorphisms induces an isomorphism of their colimits. 
%
%  It follows from \Cref{rmkEpiMonoFactorizationCommutes} that when $f$ is epi/mono, 
%  we can say that $f$ can be induced by a sequence 
%  $(f_i)_{i\in \N}$ with all $f_i$ epi/mono. 
%\end{remark}
%
%
%

\input{Topology/OpenAreODisc}
\subsection{Relating $\ODisc$ and $\Boole$}
\begin{lemma}\label{BooleIsODisc}
  Every countably presented Boolean algebra is merely a sequential colimit of finite Boolean algebras. 
\end{lemma}
\begin{proof}
  Consider a countably presented Boolean algebra of the form $B = 2[\N]/(r_n)_{n:\N}$. 
%  We will show there exists a diagram of shape $\N$ taking values in Boolean algebras 
%  with $B$ as colimit.
%  \paragraph{The diagram}
  For each $n:\N$, let $G_n$ be the union of $\{g_i\ |\ {i\leq n}\}$ and 
  the finite set of terms occurring in $(r_i)_{i\leq n}$. 
  Denote $B_n = 2[G_n]/(r_i)_{i\leq n}$. 
  Each $B_n$ is a finite Boolean algebra, and there are canonical maps $B_n \to B_{n+1}$.
%  The inclusion $G_n \hookrightarrow G_{n+1}$ induces maps $B_n \to B_{n+1}$.
%  Hence $B_n,~n:\N$ is an $(\N,\leq)$-indexed sequence of finite sets. 
  We claim that $B$ is the colimit of this sequence. 
%
%  \paragraph{The colimit}
%  As $G_n\subseteq G$ and $R_n \subseteq R$, 
%  \Cref{rmkMorphismsOutOfQuotient} also gives us a map $B_n\to \langle G \rangle \langle R \rangle$. 
%  We claim the resulting cocone is a colimit. 
%
%  Suppose we have a cocone $C$ on the diagram $(B_n)_{n\in\N}$. 
%  We need to show that there exists a map $\langle G \rangle / R\to C$ and
%  we need to show this map is unique as map between cocones. 
%  \begin{itemize}
%    \item To show there exists a map $\langle G \rangle / R \to C$, 
%      we use remark \Cref{rmkMorphismsOutOfQuotient} again. 
%      Let $g\in G$ be the $n$'th element of $G$, 
%      note that $g\in G_n$, and consider the image of $g$ under the map $B_n \to C$. 
%      This procedure defines a function from $G$ to the underlying set of $C$. 
%      Let $\phi \in R$ be the $n$'th element of $R$, 
%      note that $\phi \in R_n$, and the map $B_n \to C$ must send $\phi$ to $0$. 
%      Thus the function from $G$ to the underlying set of $C$ also sends $\phi$ to $0$. 
%      This thus defines a map $\langle G \rangle / R \to C$. 
%    \item To show uniqueness, consider that any map of cocones $\langle G \rangle / \langle R \rangle \to C$ 
%      must take the same values on all $g\in G_n$ for all $n\in\N$. 
%      Now all $g\in G$ occur in some $G_n$, so any map of cocones $\langle G \rangle /  \langle R \rangle \to C$ 
%      takes the same values for all $g\in G$. 
%      \Cref{rmkMorphismsOutOfQuotient} now tell us that these values uniquely determine the map. 
%  \end{itemize}
\end{proof}




\begin{corollary}\label{ODiscBAareBoole}
  A Boolean algebra $B$ is overtly discrete if and only if it is countably presented. 
\end{corollary}
\begin{proof}
  Assume $B:\ODisc$. 
  By \Cref{OdiscQuotientCountableByOpen}, $B$ has open equality. 
  Also $F = 2[\Sigma_{n:\N}B_n]$ is countable and we have a canonical 
  Boolean morphism $f:F \to B$. 
  By countable choice, we get for each $a,b:F$
  a sequence $\alpha_{(a,b)}:2^\N$ such that 
  $(f(a) = f(b))\leftrightarrow \exists_{k:\N} (\alpha_{(a,b)}k =1)$. 
  Consider 
  $r:F \times F \times \N \to F$ 
  given by 
  $$r(a,b,k) =\begin{cases}
    a-b &\text{ if } \alpha_{(a,b)}(k) = 1\\
    0   &\text{ if } \alpha_{(a,b)}(k) = 0
  \end{cases}
  $$
  Then $B= F/(r (a,b,n))_{(a,b,n): F \times F \times \N}$. 
  \Cref{BooleIsODisc} gives the converse.
\end{proof}
\begin{remark}\label{BooleEpiMono}
%  In particular equality in overtly discrete types is open. 
  By \Cref{ODiscEqualityOpen}, \Cref{OdiscSigma} and \Cref{ODiscBAareBoole}, 
  it follows that any 
  $g:B\to C$ in $\Boole$ has an overtly discrete kernel.
  As a consequence, the kernel is countable and $B/Ker(g)$ is in $\Boole$. 
  By uniqueness of epi-mono factorizations and \Cref{SurjectionsAreFormalSurjections}, 
  the factorization 
  $B\twoheadrightarrow B/Ker(g) \hookrightarrow C$ corresponds to 
  $Sp(C) \twoheadrightarrow Sp(B/Ker(g)) \hookrightarrow Sp(B)$. 
\end{remark}

%\begin{lemma}\label{OpenOfOdiscIsOdisc}
  Any open subset of an overtly discrete type is overtly discrete. 
\end{lemma}

\begin{lemma}\label{OdiscEnumeration}
  Any overtly discrete type has an enumeration. 
\end{lemma}
\begin{proof}

\end{proof}

\begin{corollary}
  Whenever $f:B\to C$ a map in $\Boole$, we have that
  $Ker(f)$ is overtly discrete and $B/Ker(f):\Boole$.
\end{corollary}
\begin{proof}
  As equality in Boolean algebras is open, 
  $Ker(f)$ is an open subset of $B$, hence overtly discrete by \Cref{OpenOfOdiscIsOdisc}. 
  By \Cref{OdiscEnumeration}, we thus have an enumeration of $Ker(f)$, and $B/Ker(f):Boole$. 
\end{proof}

\begin{lemma}\label{imageAsQuotientKernel}
Let $f:B \to C$ be a map in $\Boole$. 
Then the image of $(-)\circ f : Sp(C) \to Sp(B)$ is given by 
$Sp(B/Ker(f))$. 
\end{lemma}
\begin{proof}
  By \Cref{SurjectionsAreFormalSurjections}, the epi-mono factorization of $f$ as 
  $B\twoheadrightarrow B/Ker(f) \hookrightarrow C$
  gives an epi-mono factorization of $(-) \circ f$ as $Sp(C) \twoheadrightarrow Sp(B/Ker(f)) \hookrightarrow B$, 
  and by unqiueness of epi-mono factorization, $Sp(B/Ker(f))$ is the image of $(-)\circ f$. 
\end{proof}

%\begin{lemma}
  Whenever $g:B\to C$ a map in $\Boole$, we have 
  $(-)\circ g : Sp(C) \to Sp(B)$ has image 
  $Sp(B/Ker(g))$. 
\end{lemma}
\begin{proof}
  Note that $Sp(B/Ker(g))\subseteq Sp(B)$  is given by those $x:Sp(B)$ such that whenever $g(b) =_C 0$, we have $x(b) =_2 0$. 
  Thus the image of $(-)\circ g$ is a subset of $Sp(B/Ker(g))$.
  Conversely, suppose that $x$ lies in $Sp(B/Ker(g))$. 
  Then $((-)\circ g)^{-1}(x)$ defines a closed subset of $C$. 
\end{proof}

\begin{lemma}
  Let $f:A \to B$ be a map in $\Boole$. 
  Then for any $x:Sp(A)$, we have that $\Sigma_{y:Sp(B)} y\circ f = x$ is equal to 
  $Sp(B/f(Ker(x)))$. 
\end{lemma}
\begin{proof}
  Any $y:Sp(B)$ lies in $Sp(B/f(Ker(x))$ iff whenever $a:A$ is such that $x(a) = 0$, then $y(f(a)) = 0$. 
  Clearly if $y\circ f = x$, then $y(f(a)) = 0$ whenever $x(a) = 0$. 
  Now suppose $y$ lies in $Sp(B/f(Ker(x)))$. 
  We claim that then $y\circ f = x$. 
\end{proof}


\begin{theorem}\label{FormalSurjectionsAreSurjections}
  Let $f:A\to B$ be a map of countably presented Boolean algebras. 
  If $f$ is injective, then the corresponding map $(\cdot) \circ f : Sp(B) \to Sp(A)$ is surjective. 
\end{theorem}

\begin{proof}
  Assume $f$ injective and let $x:Sp(A)$.
  By \Cref{FiberConstruction}, we have that $\left(\sum\limits_{y:Sp(B)} y\circ f = x \right) = Sp(B/R) $
  for $R=f(G)$ for some countable $G\subseteq A$ with $x(g) = 0$ for all $g\in G$. 
  By propositional completeness and \Cref{SpectrumEmptyIff01Equal}, 
  it's sufficient to show that $0\neq_{B/R}1$. 
  Note that $0=_{B/R} 1$ iff 
  $1 =_B \bigvee R_0$ for some $R_0\subseteq R$ finite. 
  But then $$1 = \bigvee f(G_0) = f(\bigvee  G_0)$$ for some $G_0\subseteq G$ finite. 
  And as $f$ is injective, $\bigvee G_0 = 1$. 
  However, 
  $$
  x(\bigvee G_0) = 
  x(\bigvee_{g\in G_0} g ) = \bigvee_{g \in G_0} x(g) = \bigvee_{g\in G_0} 0 = 0$$
  And as $x(1) = 1$, we get a contradiction. Therefore $0\neq_{B/R} 1$ as required. 
\end{proof}  
The converse to the above theorem is true as well, regardless of propositional completeness:
\begin{lemma}\label{SurjectionsAreFormalSurjections}
If $f:A\to B$ is a map in $\Boole$ and $(\cdot) \circ f :Sp(B) \to Sp(A)$ is surjective, 
$f$ is injective. 
\end{lemma}
\begin{proof}
  Suppose precomposition with $f$ is surjective. 
  Let $a:A$ be such that $f(a)= 0$. 
  By assumption, for every $x:A\to 2$, there is a $y:B\to 2$ with $y\circ f = x$. 
  Consequentely $x(a) = y(f(a)) = y(0) = 0$. 
  So $x(a) = 0$ for every $x:Sp(A)$. 
  Thus $Sp(A) = Sp(A/\{a\})$, and as $Sp$ is an embedding, 
  $A \simeq A/\{a\}$, and $a = 0$ in $A$. 
  So whenever $f(a) = 0$, we have $a=0$. Thus $f$ is injective. 
\end{proof}


\begin{lemma}\label{BoolePushouts}
  Countably presented Boolean algebras are closed under pushout. 
\end{lemma} 
\begin{proof}
  Let $A,B,C:\Boole$, and suppose $f:A\to B, g:A \to C$ are Boolean morphisms. 
  Let $G_A, G_B,G_C$ be the underlying countable sets of generators for $B,C$ and 
  let $R_A,R_B,R_C$ be the underlying countable sets of relations. 
  Consider $P$ the Boolean algebra generated by $G_B\sqcup G_C$ under the relations 
  $R_B\cup R_C \cup F$ where $F$ is the set of expressions $f(a)-g(a), a\in G_A$.
  
  Note that as the generators of $B$ are included in those of $P$, 
  and all relations of $B$ are included in those of $P$, there is a map $h:B\to P$. 
  Similarly there is a map $i:C\to P$. 
  We now claim that the following is a pushout square:
  \begin{equation}\begin{tikzcd}
    A \arrow[r,"f"] \arrow[d,"g"] & B \arrow[d,"h"]\\
    C \arrow[r,"i"] & P
  \end{tikzcd}\end{equation}  
  Suppose $\beta:B \to X, \gamma:C\to X$ are such that $\beta\circ f = \gamma \circ h$. 
  $\beta,\gamma$ then induce maps on the generators of $P$. 
  These maps respect $F$ as $\beta\circ f=\gamma\circ h$, and they must respect $R_B,R_C$ as they are maps out of $B,C$. 
  Therefore, $\beta,\gamma$ induce a map $e:P\to X$, such that 
  $e(b) = \beta(b)$ for $b:G_B$ and $e(c)=\gamma(c)$ for $c:G_C$. 
  Furthermore, any map $P\to X$ with this property must agree with $e$ on all the generators of $P$, 
  and therefore equal $e$. Thus $e$ is the unique extension $P\to X$. 
  Thus $P$ the above square is indeed a pushout. 
\end{proof}

For some proofs in this paper, 
\rednote{(right now the counter's at two)}
we'd like a very concrete description of the fiber of a map of Stone spaces. 
The following construction turns out to be particularly useful. 
\begin{lemma}\label{FiberConstruction}
  Let $A,B:\Boole$, let $G$ be an explicit countable set of generators for $A$, and let 
  $f:A \to B, x:A\to 2$. 
  Define the countable set 
  \begin{equation}
    G' = \{a | a\in G, x(a) = 0\} \cup \{\neg a | a \in G, x(a) = 1\}
  \end{equation} 
  For $R = f(G')$,
%  Then we can construct a countable set $R\subseteq B$ such that 
  the pushout of $f$ and $x$ is given by $B/R$. 
\end{lemma}  
\begin{proof}
We consider the following pullback in the category of Stone spaces:
  \begin{equation}\begin{tikzcd}
    \sum\limits_{y:Sp(B)} y\circ f = x \arrow[d] \arrow[r] \arrow["\lrcorner"{pos=0.125}, phantom, dr] 
    & \top \arrow[d,"x"]\\
    Sp(B) \arrow[r,"(\cdot) \circ f"] & Sp(A)
  \end{tikzcd}  \end{equation}
Dual to this square, we have the following pushout in the category of Boolean algebras,
where $Sp(P) \simeq  (\sum\limits_{y:Sp(B)} y \circ f = x)$:
  \begin{equation}\begin{tikzcd}
    A \arrow[d,"x"'] \arrow[r,hook,"f"] \arrow[rd,phantom,"\ulcorner"{pos=0.125}] & B\arrow[d]\\
    2 \arrow[r] & P
  \end{tikzcd}\end{equation} 
  Following \Cref{BoolePushouts}, 
  the pushout $P$ is given by $B/R$ with $R$ the relations $f(a) -x(a)$ 
  where $a$ ranges over the generators of $A$.
  Note that $x(a) \in \{0,1\}$. 
  If $x(a)=0$, then $f(a)-x(a) = f(a)$, 
  and if $x(a) = 1$, then $f(a) -x(a) = \neg f(a) = f(\neg a)$. 
  So we can define the subset $G'\subseteq A$ given by 
  \begin{equation}
    G' = \{a | a\in G, x(a) = 0\} \cup \{\neg a | a \in G, x(a) = 1\}
  \end{equation} 
  $G'$ is in bijection with $G$, hence countable. 
  Furthermore, $x(g) = 0$ for all $g\in G'$. 
  And $R = f(G')$.
\end{proof}

%\section{Topology}
\section{Stone spaces}

\subsection{Stone spaces as profinite sets}
%\begin{remark}\label{StoneClosedUnderPullback}
%  Note that Boolean algebras are closed under finite colimits. 
%  By \Cref{ODiscBAareBoole} and \Cref{ODiscFiniteColim}, $\Boole$ is closed under finite colimits.
%  By \Cref {SpIsAntiEquivalence}  it follows that the category of Stone spaces is closed under finite limits. 
%\end{remark}
Here we present Stone spaces as sequential limits of finite sets. 
This is the perspective taken in Condensed Mathematics \cite{Condensed,Dagur,Scholze}.
Some of the results in this section are versions of the axioms used in 
\cite{bc24}. A full proof of all these axioms is part of future work. 

\begin{lemma}
  Any $S:\Stone$ is a sequential limit of finite sets. 
\end{lemma}
\begin{proof}
  Assume $B:\Boole$. By \Cref{SpIsAntiEquivalence} and \Cref{BooleIsODisc}, %and \Cref{ODiscClosedUnderSequentialColimits}, 
 we have that $\Sp(B)$ is a sequential limit of spectrum of finite Boolean algebras, which are finite sets. 
\end{proof}

\begin{lemma}\label{StoneAreProfinite} 
  A sequential limit of finite sets is a Stone space. 
\end{lemma}
\begin{proof}
  %For finite sets, we have that $\Sp(2^{S_n}) = S_n$, hence each $S_n$ is Stone. 
  By \Cref{SpIsAntiEquivalence} and %\Cref{BooleIsODisc}
  \Cref{ODiscClosedUnderSequentialColimits}, 
  we have that $\Stone$ is closed under sequential limits, and finite sets are Stone.
%  As $\Boole$ is closed under sequential colimits, \Cref{SpIsAntiEquivalence} gives that the 
%  sequential limit of Stone space is Stone, hence $S$ is Stone. 
\end{proof}

%\begin{remark}
 % For all $S:\Stone$, we shall denote by $S_n$ any sequence of finite sets which limit is $S$. 
 % Whenever $n\leq m$, we denote $\pi_m^n$ for the maps $S_m \to S_n$, 
 % and $\pi_n:S\to S_n$. 
%\end{remark}
\begin{corollary}
Stone spaces are stable under finite limits.
\end{corollary}
\begin{remark}\label{StoneClosedUnderPullback}\label{ProFiniteMapsFactorization}
  %Dually to \Cref{ODiscFiniteColim} and \Cref{ODiscClosedUnderSequentialColimits}, 
  %Stone spaces are closed under $\Sigma$-types and sequential limits.
%  Dually to \Cref{lemDecompositionOfColimitMorphisms} 
%  maps of Stone spaces are sequential limits of maps of finite sets. 
  By \Cref{decompositionBooleMaps} and 
  \Cref{SurjectionsAreFormalSurjections}, maps (resp. surjections, injections) of Stone spaces
  are sequential limits of maps (resp. surjections, injections) of finite sets. 
%
%
%
%
%  when we have a map of Stone spaces $f:S\to T$, 
%  we have $(\N,\geq)$-indexed sequences of finite sets $S_n,T_n$ with limits $S$ and $T$ respectively
%  and maps $f_n:S_n\to T_n$ inducing $f$. Moreover if $f$ is surjective or injective, we 
%  can choose all $f_n$ to be surjective or injective respectively as well. 
\end{remark}

\begin{lemma}\label{ScottFiniteCodomain}
  For $(S_n)_{n:\N}$ a sequence of finite types with $S=\lim_nS_n$ and $k:\N$, we have that $\Fin(k)^{S}$ is the sequential colimit of $\Fin(k)^{S_n}$.
\end{lemma}
\begin{proof}
  By \Cref{SpIsAntiEquivalence} we have $\Fin(k)^S = \Hom(2^{k},2^S)$.
  Since $2^{k}$ is finite, we have that $\Hom(2^k,\_)$ commutes with sequential colimits, therefore $\Hom(2^{k},2^S)$ is the sequential colimit of $\Hom(2^{k},2^{S_n})$. 
  By applying \Cref{SpIsAntiEquivalence} again, %these types are 
  the latter type is $\Fin(k)^{S_n}$.% as required. 
\end{proof}

\begin{lemma}\label{MapsStoneToNareBounded}
  For $S:\Stone$ and $f:S \to \N$, there exists some $k:\N$ such that $f$ factors through $\Fin(k)$. 
\end{lemma}
\begin{proof}
  For each $n:\N$, the fiber of $f$ over $n$ is a decidable subset $f_n:S \to 2$. 
  We must have that $\Sp(2^S/(f_n)_{n:\N}) = \bot$, hence there exists some $k:\N$ with 
  $\bigvee_{n\leq k} f_n =_{2^S} 1 $. 
  It follows that $f(s)\leq k$ for all $s:S$ as required. 
\end{proof}

%\begin{lemma}\label{scott-continuity}
%  Let $E:\ODisc$ and $S:\Stone$, then 
%  Then $E^S$ is the colimit of the $(\N,\leq)$-indexed sequence $E^{S_n}$.
%%  $\mathrm{colim}_k(Z^{S_k}) \to \Z^S$
%%  is an equivalence.
%\end{lemma}
%\begin{proof}
%  Let $f:S \to E$. By \Cref{OdiscQuotientCountableByOpen}, 
%  we have an enumeration $\N\twoheadrightarrow 1 + E$. 
%  By \Cref{MapsStoneToNareBounded} and \Cref{AxLocalChoice}, there is some $N:\N$ such that 
%  $f(S)\subseteq e(\N_{\leq N})$. 
%\end{proof} 
\begin{corollary}\label{scott-continuity}
  For $(S_n)_{n:\N}$ a sequence of finite types with $S=\lim_nS_n$, we have that $\N^S$ is the sequential colimit of $\N^{S_n}$. 
\end{corollary}
\begin{proof}
  By \Cref{MapsStoneToNareBounded} we have that $\N^S$ is the sequential colimit of $\Fin(k)^S$. 
  By \Cref{ScottFiniteCodomain}, $\Fin(k)^S$ is the sequential colimit of the $\Fin(k)^{S_n}$ and we can swap the sequential colimits to conclude.
  \end{proof} 



\subsection{$\Closed$ and $\Stone$}
%\begin{lemma}\label{BooleEqualityOpen}
%  Whenever $B:\Boole$, $a,b:B$ the proposition $a=_Bb$ is open. 
%\end{lemma}
%\begin{proof}
%  Let $G,R$ be the generators and relations of $B$. 
%  Let $a,b$ be represented by $x,y$ in the free Boolean algebra on $G$. 
%  Now let $R_n$ denote the first $n$ elements of $R$. 
%  Note that $a=b$ iff there exists some $n:\N$ with $x-y \leq \bigvee_{r\in R_n} r$. 
%  Furthermore, inequality is decidable in the free Boolean algebra, hence
%  $a=b$ is a countable disjunction of decidable propositions, hence open. 
%\end{proof}

\begin{corollary}\label{TruncationStoneClosed}
  For all $S:\Stone$, the proposition $\propTrunc{S}$ is closed. 
\end{corollary}
\begin{proof}
  By \Cref{SpectrumEmptyIff01Equal}, $\neg S$ is equivalent to $0=_{2^S} 1$, which is open by \Cref{BooleIsODisc} and \Cref{OdiscQuotientCountableByOpen}. 
  Hence $\neg \neg S$ is a closed proposition, and by \Cref{LemSurjectionsFormalToCompleteness}, so is $\propTrunc{S}$. 
\end{proof}
%\begin{remark}\label{ExplicitTruncationStoneClosed}
%  \rednote{New check later}
%  The above lemma and corollary actually show that if we have an explicit 
%  presentation of a Stone space as $S = \Sp(2[G] / R)$, 
%  we can construct an explicit sequence $\alpha:2^\N$ such that $||S|| \leftrightarrow \forall_{n:\N} \alpha(n) = 0$. 
%\end{remark}


\begin{corollary}\label{PropositionsClosedIffStone}
  A proposition $P$ is closed if and only if it is a Stone space. 
\end{corollary}
\begin{proof}
  By the above, if $S$ is both a Stone space and a proposition, it is closed. 
  By \Cref{ClosedPropAsSpectrum}, any closed proposition is Stone. 
\end{proof}

\begin{lemma}\label{StoneEqualityClosed}
For all $S:\Stone$ and $s,t:S$, the proposition $s=t$ is closed. 
\end{lemma}
\begin{proof}
  Suppose $S= \Sp(B)$ and let $G$ be a countable set of generators for $B$. 
  Then $s=t$ if and only if $s(g) = t(g)$ for all $g:G$. 
  So $s=t$ is a countable conjunction of decidable propositions, hence 
  closed.
\end{proof}

\subsection{The topology on a Stone space}
\begin{theorem}\label{StrongVersionOfEquivalencesOfClosedSubsetsOfStone}
  Let $A\subseteq S$ be a subset of a Stone space. TFAE:
  \begin{enumerate}[(i)]
    \item There exists a map $\alpha_{(\cdot)}:S \to 2^\N$ such that 
      $A s \leftrightarrow \forall_{n:\N} \alpha_s(n) = 0$.
    \item There exists some countable family 
      $D_n,~{n:\N}$ 
      of decidable subsets of $S$ with $A = \bigcap_{n:\N} D_n$. 
    \item There exists a Stone space $T$ and some map $T\to S$ whose image is $A$. 
    \item $A$ is closed.
  \end{enumerate}
\end{theorem}
\begin{proof}
\item 
  \begin{itemize}
  \item[$(i)\leftrightarrow (ii)$.] 
    $D_n$ and $\alpha_{(\cdot)}$ can be defined from each other by 
%    Define the decidable subsets of $S$ 
     by $D_n(s) \leftrightarrow (\alpha_s(n) = 0)$. Then observe that %$A=\bigcap_{n:\N} D_n$ as 
     \begin{equation}
      (\bigcap_{n:\N} D_n) (s) \leftrightarrow 
      \forall_{n:\mathbb N} (\alpha_s(n) = 0) 
%      \leftrightarrow A s. 
     \end{equation}
   \item[$(ii) \to (iii)$.]
      Let $S=Sp(B)$. 
      By Stone duality, we have $d_n,~n:\N$ terms of $B$ such that $D_n = \{x:S| x(d_n) = 1\}$. 
      Let $C = B/\langle (\neg d_n)_{n:\N}\rangle$.
      By \Cref{FormalSurjectionsAreSurjections}, the quotient map $B \twoheadrightarrow C$
      corresponds to a injection $\iota:Sp(C) \hookrightarrow  S$. 
%      For $s:S$, $s$ lies in the image of this map iff $s(\neg d_n) = 0$, 
      and for any $n:\N$, we have 
      \begin{equation}
        x\in \iota(Sp(C)) \leftrightarrow x(\neg d_n) = 0 \leftrightarrow x(d_n) = 1 \leftrightarrow x\in D_n
      \end{equation}
      Thus the image of $\iota$ is given by $\bigcap_{n:\N} D_n$. 
%   \item[$(iii) \to (i)$.]
%     Let $\iota:T\hookrightarrow S$ be a injective map of Stone spaces with image $A$. 
%     Then $A(s) \leftrightarrow$
%
%  \item[$(iii) \leftrightarrow (iv)$.]
   \item [$(i) \to (iv)$.] By definition.
   \item[$(iv) \to (iii)$.]
     As $A$ is closed, it induces a map $a:S\to \Closed$. 
     We can cover the closed propositions with Cantor space
     by sending 
     $\alpha \mapsto \forall_{n:\mathbb N} \alpha n = 0.$
     Now local choice gives us that there merely exists $T, e, \beta_\cdot$ as follows:
     \begin{equation}
       \begin{tikzcd}
         T \arrow[r,"\beta_\cdot"] \arrow[d, two heads,"e"] & 2^\mathbb N 
         \arrow[d,two heads, "\forall_{n:\mathbb N} (\cdot)n = 0"] \\
         S \arrow[r,"a"] & \Closed
       \end{tikzcd} 
     \end{equation} 
     Define $B(t) \leftrightarrow \forall_{n:\mathbb N} \beta_t(n) = 0$. 
     As $(i) \to (iii)$ by the above, $B$ is the image of some Stone space. 
     Furthermore, note that $A$ is the image of $B$, thus $A$ is the image of some Stone space. 
   \item[$(iii) \to (i)$.] 
      Let $f:T\to S$ be a map between Stone spaces. 
      Let $A,B$ be the underlying Boolean algebras of $S,T$ respectively. 
      Let $G_A,G_B$ and $R_A,R_B$ be the countable sets of generators and relations of $A,B$.
      It is important to note that we can consider these countable sets before we consider any $s:S$. 
      \rednote{TODO can someone give feedback on this last sentence, I'm not sure it's clear}

      Following \Cref{FiberConstruction}, for each $s:S$, we can construct 
      a countable set $I\subseteq B$ such that $Sp(B/I) = \Sigma_{t:T} f(y) = t $.
      Now note that $s$ in the image of $f$ iff $0\neq_{B/I} 1$. 
      At this point, we have constructed the the generators and relations of $B/I$, 
      hence using the proof of \Cref{BooleEqualityOpen}, we can construct a sequence 
      $\alpha_s:2^\N$ such that $0 =_{B/I}1\leftrightarrow \exists_{n:\N} \alpha_s(n) = 0$. 
      And for $\beta_s(n) = 1-\alpha_s(n)$, we conclude that 
      \begin{equation}
        s\in f(T) \leftrightarrow \forall_{n:\N} \beta_s(n) = 0
      \end{equation}
\end{itemize} 
\end{proof} 


\section{Compact Hausdorff spaces}
\subsection{Compact Hausdorff types}
\begin{definition}
  A type $X$ is called compact Hausdorff iff there exists some $S:\Stone$ and some 
  equivalence relation $\sim:S \times S \to \Closed$ such that $X \simeq S / \sim$. 
  We denote $\CHaus$ for the type of compact Hausdorff types. 
\end{definition} 

\begin{lemma}
Let $A\subseteq X$ be a subtype of a compact Hausdorff space. 
Let $S, \sim$ be any presentation of $X$. 
Then $A$ is closed iff it is the image of a closed subtype of $S$ under the quotient map. 
\end{lemma}
\begin{proof}
  If $A$ is closed, then it's pre-image under any map is also closed. 
  In particular for $q:S\to X$ the quotient map, $q^{-1}(A)$ is closed. 
  As $q$ is surjective, we have $q(q^{-1}(A)) = A$,
  hence $A$ is the image of a closed subtype of $S$. 
  Conversely, let $B\subseteq S$ be closed. 
%  Then for any $s:S$, the subtype $\{t:S| B(s) \wedge s \sim t\} \subseteq S$ is closed. 
%  Hence by 
  Define $A\subseteq S$ by 
  $$A(s) := \exists_{s:S} (B(t) \wedge s \sim t).$$
  As $B, \sim$ are closed, by \Cref{ClosedCountableConjunction} and \Cref{InhabitedClosedSubSpaceClosed}, 
  we have that $A$  is closed. 
  Also $A$ respects $\sim$, hence induces a map $A': X\to \Closed$.
  Furthermore, $A(q(s))$ iff $q(s)$ is in the image of $B$. 
  Therefore $A'(x)$ iff $x$ is in the image of $B$. 
\end{proof}
\begin{corollary}
  For $X:\CHaus$ a subtype $A\subseteq X$ is closed iff it is the image of 
  a map $T\to X$ for some $T:\Stone$. 
\end{corollary}
\begin{proof}
  Directly from the above and \Cref{StoneClosedSubsets}.
\end{proof}

\begin{corollary}\label{InhabitedClosedSubSpaceClosed}
  For $X:\CHaus$ and $A\subseteq X$ closed, we have 
  $\exists_{x:X} A(x)$ is closed. 
\end{corollary}
\begin{proof}
  Let $A$ be the image of a map map $T\to X$ for $T:\Stone$. 
  Then $\exists_{x:X} A(x) \leftrightarrow ||T||$, which is closed by \Cref{TruncationStoneClosed}
\end{proof}


\begin{corollary}\label{ClosedDependentSums}
  Closed propositions are closed under dependent sums. 
\end{corollary}
\begin{proof}
  Let $P:\Closed$ and $Q:P \to \Closed$. 
  Then $\Sigma_{p:P} Q(p) \leftrightarrow \exists_{p:P} Q(p)$.
  As $P$ is Stone by \Cref{PropositionsClosedIffStone}, it is also compact Hausdorff, thus
  \Cref{InhabitedClosedSubSpaceClosed} gives that $\Sigma_{p:P} Q(p)$ is closed. 
\end{proof}
\begin{remark}
  Analogously to \Cref{OpenTransitive} and \Cref{OpenDominance}, it follows that 
  closedness is transitive and $\Closed$ forms a dominance. 
\end{remark}
\begin{corollary}\label{AllOpenSubspaceOpen}
  For $U\subseteq X$ an open subset of a compact Hausdorff space, we have 
  $\forall_{x:X} U(x)$ open. 
\end{corollary}
\begin{proof}
  As $U$ is open, $\neg U$ is closed. 
  Hence $\exists_{x:X} \neg U(x)$ is closed. 
  Therefore, $\neg (\exists_{x:X} \neg U(x))$ is open. 
  Furthermore, it is equivalent to $\forall_{x:X} \neg \neg U(x)$, 
  which is equivalent to $\forall_{x:X} U(x)$ by \Cref{rmkOpenClosedNegation}.
\end{proof}

\begin{lemma}
  Let $X:\Chaus$ be presented by $S/\sim$. 
  Then $2^X$ is an open sub-Boolean algebra of $2^S$. 
\end{lemma}
Note that we do not claim $2^X$ is countable presented, 
but by \Cref{OpenSubsetEnumerableAreEnumerable}, it will be enumerable. 
\begin{proof}
  Denote $q:S \twoheadrightarrow X$ for the quotient map. 
  This induces an injection of Boolean algebras $2^X \hookrightarrow 2^S$.
  Note that $a:S\to 2$ lies in $2^X$ iff for all $s,t:S$, we have $a(s) = a(t)$ whenever $s\sim t$.
  Note that $a(s) = a(t)$ is decidable and $s\sim t$ is open, hence 
  $(s\sim t) \to (a(s) = a(t))$ is open (\Cref{ImplicationOpenClosed})
  By \Cref{AllOpenSubspaceOpen}, we conclude that 
  $\forall_{s:S} \forall_{t:S} ((s\sim t) \to (a(s) = a(t)))$ is open. 
  Hence $2^S$ is an open subobject of $2^X$. 
\end{proof}

\subsection{Compact Hausdorff spaces are stable under dependent sums}

\begin{lemma}
A type $X$ is Stone iff it is merely a closed in $2^\N$.
\end{lemma}

\begin{proof}
Any countably presented boolean algebra $B$ is enumerable, which gives a surjective morphism:
$$ 2[\N]\to B$$
so that by \Cref{DualCompleteness} we merely have a closed embedding:
$$ Sp(B)\to 2^\N$$
\end{proof}

\begin{lemma}\label{SigmaStoneCompactHausdorff}
Assume given $S:\Stone$ and $T:X\to \Stone$. Then $\Sigma_{x:S}T(x)$ is Compact Hausdorff.
\end{lemma}

\begin{proof}
By \Cref{ClosedDependentSums} we have that identity type in $\Sigma_{x:S}T(x)$ are closed.

We know that for any $x:S$ we have that $\exists_{C:2^\N\to \Closed} T_x = \Sigma_{y:2^\N}C(y)$. Using local choice we get $S':\Stone$ with a surjective map:
$$q:S'\to S$$
and:
$$ C : S'\to 2^\N\to\Closed$$
such that for all $x:S'$ we have $T(q(x)) = \Sigma_{y:2^\N}C(x,y)$. This gives a surjective map:
$$ \Sigma_{x:S'}\Sigma_{y:2^\N} C(x,y)\to \Sigma_{x:S}T(x)$$
The source is Stone by \Cref{StoneClosedUnderPullback} and \Cref{ClosedInStoneIsStone} so we can conclude.
\end{proof}

\begin{lemma}
Assume given $C:\CHaus$ and $D:X\to \CHaus$. Then $\Sigma_{x:C}D(x)$ is Compact Hausdorff.
\end{lemma}

\begin{proof}
By \Cref{ClosedDependentSums} we have that identity type in $\Sigma_{x:C}D(x)$ are closed.

We know that for any $x:C$ we have that $\exists_{T:\Stone} T\twoheadrightarrow C(x)$. Using a surjective map:
$$ S\to C$$
with $S:\Stone$ and local choice we get $S':\Stone$ with a surjective map:
$$q:S'\to C$$
such that for all $x:S'$ we have $T(x):\Stone$ and a surjective map $T(x)\to D(q(x))$. This gives a surjective map:
$$ \Sigma_{x:S'}T(x)\to \Sigma_{x:C}D(x)$$
By \Cref{SigmaStoneCompactHausdorff} we have a surjective map from a Stone space to the source so we can conclude.
\end{proof}
\subsection{Stone spaces are stable under dependent sums}
We will show that Stone spaces are precisely totally disconnected compact Hausdorff spaces. 
We will use this to prove that a dependent sum of Stone spaces is Stone.

\begin{lemma}\label{AlgebraCompactHausdorffCountablyPresented}
Assume $X:\Chaus$, then $2^X$ is countably presented.
\end{lemma}

\begin{proof}
  Consider some quotient map $q:S\twoheadrightarrow X$ with $S:\Stone$. 
%First we choose $S\to X$ surjective with $S$ Stone and prove that $2^X$ is an open subalgebra of $2^S$.
%
  This induces an injection of Boolean algebras $2^X \hookrightarrow 2^S$.
  Note that $a:S\to 2$ lies in $2^X$ if and only if: %for all $s,t:S$, 
  $$\forall_{s,t:S}\ q(s) =_X q(t) \to a(s) =_2a(t).$$
  As equality in $X$ is closed and equality in $2$ is decidable, so \Cref{ImplicationOpenClosed}
  tells us that the implication is open for every $s,t:S$. 
  By \Cref{AllOpenSubspaceOpen}, we conclude that 
%  $\forall_{s:S} \forall_{t:S} ((q(s) =_X q(t)) \to (a(s) =_2 a(t)))$ is open. 
%  Hence 
  $2^X$ is an open subalgebra of $2^S$. 
  Therefore, it is in $\ODisc$ by
  \Cref{PropOpenIffOdisc} and \Cref{OdiscSigma} 
  and in $\Boole$ by \Cref{ODiscBAareBoole}.
%
%
%  \rednote{It might be nice to show that Boolean algebras are countably presented iff they are overtly discrete}
%Now we prove that open subalgebras of countably presented agebras are countably presented. Assume $U\subset 2[\N] / F$ such a subalgebra. We have that $U$ is equivalent to the algebra generated by the $s:2[\N]$ such that $[s]\in U$ quotiented by the relation $s=t$ for all $s,t:2[\N]$ such that $[s],[t]\in U$ and $[s]=[t]$.
%
%Using that $2[\N]$ is countable and that $[s]=[t]$ is open by \Cref{BooleEqualityOpen}, we see that $U$ is generated by variables and relations each indexed by an open in $\N$. But by \Cref{OpenInNAreDecidableInN} any open in $\N$ is countable, so $U$ is countably presented.
\end{proof}
\begin{definition}
For all $X:\Chaus$ and $x:X$,
  we define $Q_x$ the connected component of $x$
  as the intersection of all decidable subsets of $X$ containing $x$. 
\end{definition}

\begin{lemma}\label{ConnectedComponentClosedInCompactHausdorff}
For all $X:\CHaus$ with $x:X$, we have that $Q_x$ is a countable intersection of decidables in $X$.
\end{lemma}
\begin{proof}
%  By \Cref{AlgebraCompactHausdorffCountablyPresented} we have that $2^X$ is countably presented, 
%  therefore we can enumerate the elements of $2^X$, say as $(D_n)_{n:\N}$. 
  By \Cref{AlgebraCompactHausdorffCountablyPresented},
  we can enumerate the elements of $2^X$, say as $(D_n)_{n:\N}$. 
  Define $E_n$ for $n:\N$ as $D_n$ if $x\in D_n$ and $X$ otherwise. 
  Then $\cap_{n:\N}E_n = Q_x$.
\end{proof}
%

\begin{lemma}\label{ConnectedComponentSubOpenHasDecidableInbetween}
  Assume $X:\Chaus$ with $x:X$ and suppose $U\subseteq X$ open with $Q_x\subseteq U$. 
  Then we have some decidable $E\subseteq X$ with $x\in E$ and $E\subseteq U$. 
\end{lemma}
\begin{proof}
  By \Cref{ConnectedComponentClosedInCompactHausdorff}, 
  we have $Q_x = \bigcap_{n:\N}D_n$ with $D_n\subseteq X$ decidable. 
  If $Q_x \subseteq U$, we have that 
  $$Q_x\cap \neg U = \bigcap_{n:\N} (D_n \cap \neg U) = \emptyset.$$
  By \Cref{CHausFiniteIntersectionProperty} there is some $k:\N$ with 
  $$(\bigcap_{n\leq k} D_n )\cap \neg U  = \bigcap_{n\leq k} (D_n \cap \neg U) = \emptyset.$$
  Therefore $\bigcap_{n\leq k} D_n \subseteq \neg\neg U$, which equals $U$ by \Cref{rmkOpenClosedNegation}. So $\bigcap_{n\leq k} D_n$ gives us the desired decidable subset.
\end{proof}

\begin{lemma}\label{ConnectedComponentConnected}
Assume $X:\Chaus$ with $x:X$. Then any map in $Q_x\to 2$ is constant.
\end{lemma}
\begin{proof}
Assume given a separation $Q_x = A\cup B$ with $A,B$ disjoint and decidable in $Q_x$. Assume $x\in A$. 
By \Cref{ConnectedComponentClosedInCompactHausdorff}, $Q_x\subseteq X$ is closed. 
Using \Cref{ClosedTransitive}, it follows that $A,B\subseteq X$ are closed and disjoint.
By \Cref{CHausSeperationOfClosedByOpens} there exist $U,V$ disjoint open such that $A\subseteq U$ and $B\subseteq V$. 
By \Cref{ConnectedComponentSubOpenHasDecidableInbetween} we have a decidable $D$ such that $Q_x\subseteq D\subseteq U\cup V$. 
Note that $E := D\cap U = D \cap (\neg V)$ is clopen, hence decidable by \Cref{ClopenDecidable}.
But $x\in E$, hence $B\subseteq Q_x \subseteq E$ but $B \cap E = \emptyset$, hence $B=\emptyset$. 
\end{proof}

\begin{lemma}\label{StoneCompactHausdorffTotallyDisconnected}
Let $X:\CHaus$, then $X$ is Stone if and only $\forall_{x:X}\ Q_x=\{x\}$.
\end{lemma}

\begin{proof}
  By \Cref{AxStoneDuality}, it is clear that for all $x:S$ with $S:\Stone$ we have that $Q_x=\{x\}$.
%
  Conversely, assume $X:\CHaus$ such that $\forall_{x:X}\ Q_x = \{x\}$.
  We claim that the evaluation map $e:X \to Sp(2^X)$ is both injective and surjective, hence an equivalence. 
%  \item 
    Let $x,y:X$. If $f(x) = f(y)$ for all $f:2^X$, then $y \in Q_x$, hence $x=y$ by assumption. 
    Thus $e$ is injective. 
%  \item 
    Let $q:S\twoheadrightarrow X$ be a surjective map. 
    It induces an injection $2^X \hookrightarrow 2^S$, which by \Cref{SurjectionsAreFormalSurjections}
    induces a surjection $Sp(2^S) \twoheadrightarrow Sp(2^X)$. 
    Note that $e\circ q$ factors as $S\simeq Sp(2^S) \twoheadrightarrow Sp(2^X)$. 
    It follows that $e$ is surjective. 
%
%For the converse, we show that the map:
%\[X\to Sp(2^X)\]
%is an equivalence and conclude by \Cref{AlgebraCompactHausdorffCountablyPresented}. 
%
%Surjectivity always hold, indeed considering $q:S\to X$ surjective with $S$ Stone, we have that $2^X\subset 2^S$ as so that the by \Cref{FormalSurjectionsAreSurjections} the map:
%$$S = Sp(2^S)\to Sp(2^X)$$
%is surjective and it factors though $X$.
%
%Now let us prove injectivity. Assume $x,y:X$ having the same image in $Sp(2^X)$. This means that any map in $X\to 2$ has the same value on $x$ and $y$, so $x\in Q_y$ and by hypothesis $x=y$.
\end{proof}

\begin{theorem}
  \label{stone-sigma-closed}
Assume $S:\Stone$ and $T:S\to\Stone$. Then $\Sigma_{x:S}T(x)$ is Stone.
\end{theorem}

\begin{proof}
By \Cref{SigmaCompactHausdorff} we have that $\Sigma_{x:S}T(x)$ is compact Hausdorff. 
By \Cref{StoneCompactHausdorffTotallyDisconnected} 
it is enough to show that for all $x:S$ and $y:T(x)$ 
we have that $Q_{(x,y)}$ is a singleton.
%
Assume $(x',y')\in Q_{(x,y)}$, then for any map $f:S\to 2$ we have that:
$$ f(x) = f\circ \pi_1(x,y) = f\circ \pi_1(x',y') = f(x')$$
so that $x'\in Q_x$ and since $S$ is Stone by \Cref{StoneCompactHausdorffTotallyDisconnected} we have that $x=x'$.
%
Therefore we have $Q_{(x,y)}\subseteq \{x\}\times T(x)$. Assume $z,z':Q_{(x,y)}$, then for any map $g:T(x)\to 2$ we have that $g(z)=g(z')$ by \Cref{ConnectedComponentConnected}. Since $T(x):\Stone$, we conclude $z=z'$ by \Cref{StoneCompactHausdorffTotallyDisconnected}.
\end{proof}




\section{The Unit interval}
\subsection{The unit interval as Compact Hausdorff space}
\subsection{The Cauchy reals}
The goal of this section is to introduce the real numbers in a constructive setting, 
following the definition given in \cite{Bishop} with some small adaptations. 
We will later use this definition to show that the interval $[0,1]$ is compact Hausdorff in the sense 
of \Cref{dfnCompactHausdorff}. 

We will assume we are given natural and rational numbers, with decidable (in)equalities
working as expected. 

\begin{definition}
  A \textbf{Cauchy sequence} is a sequence $x : \mathbb N \to \mathbb Q$ such that
  for any $n,m:\mathbb N$, we have %$0\leq x_n \leq 1$ and 
$|x_n-x_m| \leq (\frac12)^n + (\frac12)^m$. 
\end{definition}
\begin{remark}
  If $x$ is a cauchy sequence and $q$ a rational number, the 
  sequence $(x-q)_n = (x_n-q)$ is also Cauchy.
\end{remark}

Following \cite{Bishop}, we define inequality relations between Cauchy sequences and
rational numbers. 
\begin{definition}
  For $x$ a Cauchy sequence and $q$ a rational number, we define 
  \begin{itemize}
%    \item $x> q = \Sigma(n:\mathbb N) x_n > q + {\frac12}^n$. %for some $n:\mathbb N$. 
%    \item $x< q = \Sigma(n:\mathbb N) x_n < q - {\frac12}^n$. %for some $n:\mathbb N$. 
    \item $x\leq  q = \Pi_{n:\mathbb N} x_n \leq q+(\frac12)^n$. 
    \item $x\geq  q = \Pi_{n:\mathbb N} x_n \geq q-(\frac12)^n$. 
  \end{itemize}
\end{definition}
%\begin{lemma}
%  For $x$ Cauchy and $q$ rational, we have that 
%  $x\leq q$ iff for each $n:\mathbb N$, we have a $N_n:\mathbb N$ with 
%  $x_m> q-(\frac12)^n$ for all $m \geq N_n$. 
%\end{lemma}
\begin{lemma}\label{ComparisonLemma}
  For $x$ a Cauchy sequence and $q$ a rational number, we have
  $x \leq q \vee x \geq q$. 
\end{lemma}
\begin{proof}
  For rational numbers, we have decidable inequalities, 
  therefore $\geq 0 \vee q \leq 0$. 
  It follows that 
  $ \forall (n:\mathbb N) \forall (m:\mathbb N) q \geq -(\frac12)^n \vee q \leq (\frac12)^m$. 
  Now by \Cref{TODO}, we may conclude 
  $ (\forall (n:\mathbb N) q \geq -(\frac12)^n ) \vee (\forall (m:\mathbb N) q \leq (\frac12)^m)$
  as required.
\end{proof}


%%%\begin{definition}
%%%  A Cauchy sequence $x$ is \textbf{nonnegative} if $x_n \geq -(\frac12)^n$
%%%  for every $n:\mathbb N$. 
%%%  $x$ is \textbf{nonpositive} if $x_n \leq (\frac12)^n$
%%%  for every $n:\mathbb N$. 
%%%\end{definition} 
%%%%\begin{lemma}
%%%%  A Cauchy sequence is nonnegative iff there exists an $N$ such that $x_n \geq -(\frac12)^N$
%%%%  for all $n\geq N$. 
%%%%  A Cauchy sequence is nonpositive iff there exists an $N$ such that $x_n \leq (\frac12)^N$
%%%%  for all $n\geq N$. 
%%%%\end{lemma}
%%%%\begin{proof}
%%%%  Assume $x$ is nonnegative. Thus for every $n:\mathbb N$, we have $x_n\geq -(\frac12)^n \geq -(\frac12)^0$. 
%%%%  Thus $N$ can taken to be $0$. 
%%%%%
%%%%  Conversely, as $x$ is Cauchy, we have
%%%%  for all $m :\mathbb N$ that  
%%%%%  \begin{equation}- (\frac12)^m -(\frac12)^n \leq    x_m-x_n \leq (\frac12)^m + (\frac12)^n \end{equation}
%%%%  \begin{equation}- (\frac12)^m -(\frac12)^n \leq    x_n-x_m \leq (\frac12)^m + (\frac12)^n \end{equation}
%%%%  If in addition there is an $N$ such that whenever $m\geq N$, we have 
%%%%  $x_m \geq -(\frac12)^N$, so $-x_m \leq (\frac12)^N$, 
%%%%  so $x_n -x_m \leq x_n - (\frac12)^N$. 
%%%%  Therefore, 
%%%%  \begin{equation}- (\frac12)^m -(\frac12)^n \leq    x_n-x_m \leq x_n-(\frac12)^N \end{equation}
%%%%  Thus 
%%%%  \begin{equation}- (\frac12)^m -(\frac12)^n  + (\frac12)^N \leq x_n \end{equation}
%%%%  As $N \geq N$, we have in particular 
%%%%  \begin{equation}- (\frac12)^N -(\frac12)^n  + (\frac12)^N \leq x_n \end{equation}
%%%%  \begin{equation} - (\frac12)^n  \leq x_n \end{equation}
%%%%  thus $x$ is nonnegative. 
%%%%
%%%%  The nonpositive case goes similar. 
%%%%\end{proof}   
%%% 
%%%
%%%\begin{lemma}
%%%  A Caucy sequence is nonnegative or nonpositive. 
%%%\end{lemma}

%\begin{lemma}
%  For any Cauchy sequence $p$, we have 
%  $(\forall (n:\mathbb N) p_n \leq (\frac12)^n) \vee (\forall (n:\mathbb N) p_n \geq -(\frac12)^n)$. 
%\end{lemma}
%\begin{proof}
%We 
%\end{proof}

\begin{definition}
Given two Cauchy sequences $p = (p_n)_{n\in\mathbb N}, q=(q_n)_{n\in\mathbb N}$, 
we define the proposition $p \sim_C  q$ as 
\begin{equation}
  p \sim_C q : = \forall (n,m : \mathbb N) ((| p_n - q_m| \leq  (\frac12)^n + (\frac12)^m))
\end{equation}
\end{definition}

%\begin{remark}
%  Note that $p\sim_C q$ is equivalent to 
%\begin{equation}
%  \forall (n : \mathbb N) | p_n - q_n| \leq  (\frac12)^{n-1}
%\end{equation}
%The equivalence doesn't hold, unless you cut off initial segments (which shouldn't matter). 
%\end{remark} 

\begin{definition}
  The type of \textbf{Cauchy reals} is given by 
  the type of Cauchy sequences modulo $\sim_C$.
\end{definition}

We claim that the inequality in \Cref{TODO} extends to a well-defined 
inequality between Cauchy reals and rational numbers. 

Furthermore, we claim that 
$\Pi_{x:\mathbb R} \Pi_{q:\mathbb Q} x \leq q \vee x \geq q$. 

%\begin{lemma}
%  For any Cauchy real $r$ any Cauchy sequence $p$ representing $r$, 
%  we have 
%  \begin{equation}
%    (\forall (n:\mathbb N) p_n \leq (\frac12)^n) \vee (\forall (n:\mathbb N) p_n \geq (\frac12)^n)
%  \end{equation}
%
%\end{lemma}

\begin{definition}
  A Cauchy sequence in the interval is a Cauchy sequence $x$ such that 
  for any $n:\mathbb N$, we have $0\leq x_n \leq 1$. 
 % 
  The interval of Cauchy reals is given by the type of Cauchy sequences in the interval 
  modulo $\sim_C$. We denote it by $[0,1]$. 
\end{definition}  


We want to show that the interval of Cauchy reals is Compact Hausdorff. 
Informally, to any binary sequence $\alpha : \mathbb N \to 2$, 
we can associate a Cauchy sequence 
\begin{equation}\label{eqnBinaryEncode}
  n\mapsto \sum\limits_{i = 0 }^n \frac {\alpha(i)}{2^{i+1}}
\end{equation}
and we are going to give a closed relation on Cantor space such that 
two binary sequences are equivalent iff they correspond to the same Cauchy reals. 
%
First, we'll need some notation.
\begin{definition}
Given a binary sequence $\alpha:\mathbb N \to 2$ and a natural number $n : \mathbb N$  
we denote $\alpha|_n: \mathbb N_{\leq n} \to 2$ for the 
restriction of $\alpha$ to a finite sequence of length $n$. 
We denote $\overline 0, \overline 1$ for the binary sequences which are constantly $0$ and $1$ respectively. 
We denote $0,1$ for the sequences of length $1$ hitting $0,1$ respectively. 
If $x$ is a finite sequence and $y$ is any sequence, denote $x\cdot y$ for their concatenation. 
\end{definition} 
Now we'll give a definition for when two finite binary sequences of length $n$ correspond 
to real numbers whose distance is $\leq (\frac12)^n$.
Basically, we want for every finite sequence $z$ that 
$(z \cdot 0 \cdot \overline 1)$ and  $(z \cdot 1 \cdot \overline 0)$ are equivalent. 

\begin{definition}
Now let $n:\mathbb N$ and $x,y:\mathbb N_{\leq n} \to 2$ be two sequences of length $n$. 
We say $x,y$ are near if we have an $m:\mathbb N$ with $m\leq n$
and some $a: \mathbb N_{\leq m} \to 2$, 
such that one of $(a \cdot 0 \cdot \overline 1)|_n,  ( a \cdot 1 \cdot \overline 0)|_n$
is equal to $x$ and the other is equal to $y$. 
We denote $\text{near}_n(x,y)$ if $x,y$ are near. 
%
To be precise, we define 
\begin{equation}
  \text{near}_n(x,y) = 
\Sigma(m:\mathbb N) m \leq n \wedge 
  \Sigma (a : Fin_m \to 2) 
\bigg( \big( (x,y) = 
((a \cdot 0 \cdot \overline 1)|_n,  ( a \cdot 1 \cdot \overline 0)|_n)
\big)
\bigvee 
\big(
  (y,x) = 
((a \cdot 0 \cdot \overline 1)|_n,  ( a \cdot 1 \cdot \overline 0)|_n)
\big)
\bigg)
\end{equation}
\end{definition}
\begin{remark}
Remark that when $x,y$ are near, $m$ and $a$ as above are unique. 
Thus $\text{near}_n(x,y)$ is a proposotion. 
%
Furthermore, to check whether $x,y$ are near, we need only make $n$ comparisons, 
thus $\text{near}_n(x,y)$ is decidable. 
%
Note that in the above definition, we allow $m = n$ and therefore $x$ is near to itself for any finite sequence $x$. 
Furthermore, we have defined nearness to be symmetric. 
However, it is not a transtive relation. 
After all, the sequence $010$ and $011$ are near and the sequence $011$ and $100$ are near, 
but $010$ is not near to $100$. This corresponds to the fact that $\frac14$ and $\frac38$ are distance $\leq (\frac12)^3$
apart, and so are $\frac38$ and $\frac12$, but $\frac14$ and $\frac12$ are not. 
\end{remark}
\begin{definition}
  We define the following relation on Cantor space for $\alpha, \beta: 2^\mathbb N$.
  \begin{equation}
    \alpha \sim_t \beta = \forall (n : \mathbb N) 
    \text{near}_n(\alpha|_n, \beta|_n)
  \end{equation}
\end{definition}
\begin{lemma}
  $\sim_t$ is a closed equivalence relation. 
\end{lemma}
\begin{proof}
   Let $\alpha, \beta, \gamma : 2^\mathbb N$. 
   As the dependent product of propositions is a proposition, $\alpha \sim_t\beta$ is a proposition. 
   %
   Furthermore, the closedness follows from decidability of $\text{near}_n(\alpha|_n, \beta|_n)$. 
   One could define $\gamma(n) = 1$ iff $\text{near}_n(\alpha|_n, \beta|_n)$
   
   As nearness is reflexive and symmetric, so is $\sim_t$. 

   Now suppose $\alpha \sim_t \beta$ and $\beta\sim_t \gamma$. 
   We claim that $\alpha \sim_t \gamma$. 

   Let $n:\mathbb N$, we need to show that 
   $\text{near}_n(\alpha|_n , \gamma|_n)$. 
   Let $(a,m)$ witness that $\text{near}_n(\alpha|_n, \beta|_n)$.
   and let $(b, k)$ witness that $\text{near}_n(\beta|_n, \gamma|_n)$
   We will make a case distinction on whether one of $m,k$ is equal to $n$, or
   both are strictly smaller than $n$. 
   \begin{itemize}
     \item 
       If $m=n$, we have that $\alpha|_n = \beta|_n$, and therefore 
       \begin{equation}
         \text{near}_n(\beta|_n, \gamma|_n) \leftrightarrow \text{near}_n(\alpha|_n, \gamma|_n)
       \end{equation} 
       The above also holds if $k = n$.
     \item 
       If $m< n$, we have that $\alpha(m+1) \neq \beta(m+1)$, thus 
       $\alpha|_l \neq \beta|_l$ for all $l>m$, 
       but we still have $\text{near}_l(\alpha|_l, \beta|_l)$ for these $l$. 
       Therefore $(\alpha, \beta)$ or $(\beta, \alpha)$ must be of the form
       $(a \cdot 0 \cdot \overline 1, a \cdot 1 \cdot \overline 0)$. 
       WLOG, we assume $\alpha = a \cdot 0 \cdot \overline 1$, and thus 
       $\beta = a \cdot 1 \cdot \overline 0$ (if not, we could do bitflips). 

       As $k<n$ also, by the same argument there is some $b$ such that one of 
       $(\beta,\gamma), (\gamma, \beta)$
       is equal to $(b\cdot 0 \cdot \overline 1, b \cdot 1 \cdot \overline 0)$. 
       However, $\beta$ is also of the form $a \cdot 1 \cdot \overline 0$, and 
       thus cannot also be of the form $b \cdot 0 \cdot \overline 1$. 
       Therefore we must have 
       $\beta = b\cdot 1 \cdot \overline 0$ and 
       $\gamma= b\cdot 0 \cdot \overline 1$. 

       But now $b \cdot 1 \cdot \overline 0 = a \cdot 1 \cdot \overline 0$, 
       The lengths of $a,b$ cannot be unequal, and by decidablity of natural numbers, 
       $a,b$ have the same length and it follows that $ a = b$. 
       Therefore $ \alpha = \gamma$, so $\alpha \sim_t\gamma$.
   \end{itemize}

   We conclude that $\sim_t$ is a closed equivalence relation. 
\end{proof}


\begin{lemma}
  $b$ sends $\sim_n$ equivalent binary sequences to $\sim_C$ equivalent Cauchy sequences. 
\end{lemma}
\begin{proof}
  Let $\alpha, \beta$ be binary sequences.
  We claim that $|b(\alpha)_n - b(\beta)_n| \leq (\frac12)^{n+1}$ 
  whenever $\text{near}_n(\alpha, \beta)$. 
  It will follow that if $\alpha\sim_n \beta$, then 
  $b(\alpha)\sim_C b(\beta)$. 

  Let $n:\mathbb N$ and assume $m:\mathbb N$ with $m\leq n$ and 
  let $z$ be a sequence of length $m$ such that 
  $\alpha|_n = z\cdot 1 \cdot \overline 0|_n$ and $\beta|_n = z \cdot 0 \cdot \overline q |_n$. 
  then $b(\alpha)_n = \sum_{i\leq m} \frac{z(i)}{2^{i+1}} + (\frac12)^{m+2}$ and 
  $b(\beta)_n = \sum_{i\leq m} \frac{z(i)}{2^{i+1}} + \sum\limits_{m+2 \leq i \leq n}(\frac12)^{i+1}$. 
  Thus 
  $b(\alpha)_n - b(\beta)_n = (\frac12)^{m+2} - \sum\limits_{m+2 \leq i \leq n}(\frac12)^{i+1} = 
  (\frac12)^{n+1}$, 
  which is smaller than required. 
\end{proof}  

\begin{lemma}
  Whenever $b(\alpha) \sim_C b(\beta)$, 
  we have $\alpha \sim_n \beta$. 
\end{lemma}
\begin{proof}
  Assume $b(\alpha) \sim_Cb (\beta)$. 
  Let $n:\mathbb N$. 
  We shall show that $\text{near}_n(\alpha , \beta)$. 

  As we're only checking finitely many entries, 
  we either have $\alpha|_n = \beta|_n$, 
  or there exists a smallest $m\leq n$ with 
  $\alpha(m) \neq \beta(m)$. 

  If $\alpha|_n = \beta|_n$, we have $\text{near}_n(\alpha,\beta)$ and are done. 
  WLOG assume $\alpha(m) = 1, \beta(m) = 0$ for $m$ minimal. 
  We claim that for any $k\geq m$, we have 
  $b(\alpha)_k - b(\beta)_k = (\frac12)^k$. 

  Now note that 
  \begin{equation} 
    b(\alpha)_{k+1} - b(\beta)_{k+1} = 
    b(\alpha)_{k} - b(\beta)_{k} + 
    \frac{\alpha(k+1) - \beta(k+1)}{2^{k+2}}.
  \end{equation}




  For $k>m$, we have that 
  \begin{equation}
  |b(\alpha)_k - b(\beta)_k |= 
  |(\frac12)^{m+1} + \sum\limits_{i=m+1}^k \frac{ \alpha(i) -\beta(i)}{2^{i+1}}|. 
  \end{equation}
  Note that the right summand is always $\leq (\frac12)^{m+1}$. 
  Therefore, we can leave out the absolute value function. 
  As $b(\alpha) \sim_Cb(\beta)$, we have that 
  \begin{equation}
  (\frac12)^{m+1} + \sum\limits_{i=m+1}^k \frac{ \alpha(i) -\beta(i)}{2^{i+1}} \leq (\frac12)^{k-1}
  \end{equation}
  Note that $\alpha(i) -\beta(i) \in \{-1,0,1\}$ always. 
  Also, 

  Denote $z = \alpha|_{m-1} = \beta_{m-1}$. 
  We will show that for $n\geq i>m$, we must have $\alpha(i) = 0, \beta(i) = 1$. 
  Suppose $\alpha(i) \neq 0$. 
\end{proof}

%
%
%\begin{theorem}
%  The interval of Cauchy reals is isomorphic to $2^\mathbb N / \sim_t$. 
%\end{theorem} 
%\begin{proof}
%  Define $b: 2^\mathbb N \to [0,1]$ by composing the map in \Cref{eqnBinaryEncode} with the projection map 
%  from Cauchy sequences in the interval to the interval in Cauchy reals. 
%  We will show $b$ is surjective, 
%  and such that $b(\alpha) = b(\beta)$ iff  $\alpha \sim_t \beta$. 
%  \begin{itemize}
%    \item For $b$ to be surjective, we need to show that for any Cauchy sequence $p$ in the interval, there 
%      merely exists some binary sequence $\alpha$ such that $b(\alpha)\sim_C p$. 
%
%      We will inductively define $\alpha$. 
%      Let $\alpha(i)$ be defined for $i<n$. 
%      Consider the sequence $p'(j) = p(j) - \sum\limits_{i<n} \frac{\alpha(i)}{2^{i+1}}$. 
%      As $p$ is a Cauchy sequence, so is $p'$. 
%  \end{itemize}
%\end{proof}
%

In this section, we will show that the topology on $I$ as one would expect. 

\begin{definition}
  For $n:\N$ we define 
  $cs_n:2^n \to \mathbb Q$ by 
  \begin{equation}
    cs_n(a) = \sum\limits_{i=0}^{n-1} \frac{a(i)} {2^{i+1}}
  \end{equation}
  And for $\alpha:2^\N$, we define the sequence $cs(\alpha) : \N \to \mathbb Q$ by 
  \begin{equation}
    cs(\alpha)_n = cs_n(\alpha|_n)
  \end{equation}
\end{definition}
\begin{remark}
  One can check that for each $\alpha:2^\N$, 
  $cs(\alpha)$ is a Cauchy sequence. 
  Thus $cs$ gives a map from Cantor space to the Cauchy reals. 
%  In \Cref{AppendixCauchyEquivalenceProof}, we show that 
  We claim without proof that
  this map induces an equivalence from $I$ to the interval of Cauchy reals. 
\end{remark}
\begin{lemma}
  For each $n:\N$, $cs_n$ is injective. 
\end{lemma}  
\begin{proof}
  \rednote{This is just copied from below, suggesting a more general lemma}
  Assume $cs_n(s) = cs_n(t)$. As equality of finite sequences is decidable, 
  we are satisfied if we assume $s\neq t$ and find a contradiction. 
  We may without loss of generality assume there is some $m<n$ and some $u:2^m$ such that 
  \begin{equation}
    (s|_{m+1} = u \cdot ) \wedge ( t|_{m+1} = u \cdot 1) . 
  \end{equation}
  Then 
  WLOG, we assume that $s(m) = 1, t(m) = 0$. 
  We thus have that 
  \begin{align}
    cs_n(s) &= 
    cs_m(u) + \frac1{2^{m+1}} + \sum\limits_{i = m+1}^{n-1} \frac{s(i)}{2^{i+1}}\\
    cs_n(t) &= 
    cs_m(u) + 0  + \sum\limits_{i = m+1}^{n-1} \frac{t(i)}{2^{i+1}}
  \end{align}
  And thus 
  \begin{align}
    cs_n(s)-cs_n(t) = \frac{1}{2^{m+1}} + \sum\limits_{i = m+1}^{n-1} \frac{s(i)-t(i)}{2^{i+1}}
  \end{align}
  Note that as $s(i),t(i) \in \{0,1\}$, we must have that $s(i) -t(i) \in \{-1,0,1\}$. 
  Therefore 
  $$\sum\limits_{i = m+1}^{n-1} \frac{s(i)-t(i)}{2^{i+1}}$$
  is minimal iff $s(i) -t(i) = -1$ for all $m<i<n$. 
  In that case, we have that 
  $$
  \frac{1}{2^{m+1}}-
  \sum\limits_{i = m+1}^{n-1} \frac{s(i)-t(i)}{2^{i+1}}= 
  \frac{1}{2^{m+1}}-
  \sum\limits_{i = m+1}^{n-1} \frac{1}{2^{i+1}}= 
  \frac{1}{2^n}
  $$
  which is non-zero, contradicting $cs_n(s) = cs_n(t)$. 
\end{proof}

\begin{lemma}\label{alternativeSimByCauchyDistance}
  Let $n:\N$ and let $s,t:2^n$. Then 
  \begin{equation}
    s\sim_n t \leftrightarrow |cs_n(s) - cs_n(t)| \leq \frac{1}{2^{n}}.
  \end{equation} 
\end{lemma}

\begin{proof}
  \item  
    Assume $ s \sim_n t$. If $s=t$, we have $cs_n(s) - cs_n(t) = 0$, 
    otherwise, we may without loss of generality assume there is some $m<n$ and some $u:2^m$ such that 
  \begin{equation}
    (s = u \cdot 0 \cdot \overline 1|_n) \wedge ( t = u \cdot 1 \cdot \overline 0 |_n) . 
  \end{equation}
  Then 
  \begin{align}
    cs_n(s) &= 
    cs_m(u) + 0 + \sum\limits_{i = m+1}^{n-1} \frac{1}{2^{i+1}}\\
    cs_n(t) &= 
    cs_m(u) + \frac{1}{2^{m+1}} + 0  
  \end{align}
  And hence 
  \begin{equation}
    cs_n(t) - cs_n(s) = \frac{1}{2^{m+1}} - \sum\limits_{i = m+1}^{n-1} \frac{1}{2^{i+1}} = \frac{1}{2^n}
  \end{equation}
  Thus in all cases, from $s\sim_n t$, we can conclude that 
  \begin{equation}
    |cs_n(s) -cs_n(t) |\leq \frac{1}{2^n}
  \end{equation}
  \item 
  Conversely, assume that $|cs_n(s) - cs_n(t)| \leq \frac{1}{2^n}$. 
  If $s = t$, it is clear that $s \sim_n t$.
  If $s\neq t$, there must be some smallest number $m<n$ such that 
  $s(m) \neq t(m)$. As $m$ is minimal, we have $s|_m = t|_m = : u$. 
  WLOG, we assume that $s(m) = 1, t(m) = 0$. 
  We thus have that 
  \begin{align}
    cs_n(s) &= 
    cs_m(u) + \frac1{2^{m+1}} + \sum\limits_{i = m+1}^{n-1} \frac{s(i)}{2^{i+1}}\\
    cs_n(t) &= 
    cs_m(u) + 0  + \sum\limits_{i = m+1}^{n-1} \frac{t(i)}{2^{i+1}}
  \end{align}
  And thus 
  \begin{align}
    cs_n(s)-cs_n(t) = \frac{1}{2^{m+1}} + \sum\limits_{i = m+1}^{n-1} \frac{s(i)-t(i)}{2^{i+1}}
  \end{align}
  Note that as $s(i),t(i) \in \{0,1\}$, we must have that $s(i) -t(i) \in \{-1,0,1\}$. 
  Therefore 
  $$\sum\limits_{i = m+1}^{n-1} \frac{s(i)-t(i)}{2^{i+1}}$$
  is minimal iff $s(i) -t(i) = -1$ for all $m<i<n$. 
  In that case, we have that 
  $$
  \frac{1}{2^{m+1}}-
  \sum\limits_{i = m+1}^{n-1} \frac{s(i)-t(i)}{2^{i+1}}= 
  \frac{1}{2^{m+1}}-
  \sum\limits_{i = m+1}^{n-1} \frac{1}{2^{i+1}}= 
  \frac{1}{2^n}
  $$
  Now if $s(i) -t(i) > -1$ for any $m<i<n$, we have that
    $$
    \frac{1}{2^{m+1}} + \sum\limits_{i = m+1}^{n-1} \frac{s(i)-t(i)}{2^{i+1}}> \frac{1}{2^n},$$
  contradicting our assumption that 
  $|cs_n(s) - cs_n(t)| \leq \frac{1}{2^n}$. 
  We conclude that $s(i) -t(i) = -1$ for all $m<i<n$, hence for those $i$, we have that 
  $s(i) = 0, t(i) = 1$. Hence 
  \begin{equation}
    s = (u \cdot 1\cdot \overline 0) |_n \wedge 
    t = (u \cdot 0\cdot \overline 1) |_n.
  \end{equation}
  and thus we can conclude $s\sim_n t$ as required. 
\end{proof}


Inspired by Definition 2.7 and 2.10 \Cite{Bishop}, 
we define inequality on $I$ as follows:
\begin{definition}
  Let $\alpha,\beta:2^\N$. 
  We define $\alpha\leq_I \beta$ and $\alpha<_I\beta$ as follows:
  \begin{align}
  \alpha\leq_I\beta : = \forall_{n:\N} ( cs(\alpha)_n \leq cs(\beta)_n + \frac {1} {2^n})\\ 
    \alpha   <_I \beta : = \exists{n:\N} ( cs(\alpha)_n < cs(\beta)_n - \frac {1} {2^n})
%    \\\rednote{Can become n\pm1, \leq ,<, +\frac1{2^n+2} }
\end{align}
\end{definition}

\begin{lemma}\label{SqueezeLemma}
  For any $n:\N$ and $a,b,c:2^n$, we have that whenever 
  $cs_n(a) \leq cs_n(b)\pm\frac{1}{2^n} \leq cs_n(c)$ and $a\sim_n c$, we have 
  $a = b \vee b = c$. 
\end{lemma}
\begin{proof}
  Assume 
\end{proof}
\newpage

\begin{lemma}
  Suppose $n:\N$ and $a,b:2^n$. 
  Then if $cs_n(a) > cs_n(b)$, 
  we have that $cs_n(a) \geq  cs_n(b) +\frac{1}{2^n}.$
\end{lemma}



\begin{lemma}
  $\leq_I$ respects $\sim_I$. 
\end{lemma}
\begin{proof}

  CLAIM 1: 
  Whenever $a\sim_n b$, we have that 
  \begin{equation} 
    cs_n(a) = cs_n(b) - \frac{1}{2^n}
    \vee
    cs_n(a) = cs_n(b) 
    \vee
    cs_n(a) = cs_n(b) + \frac{1}{2^n}
  \end{equation}
  
  CLAIM 2: 
  whenever $a,b:2^n$ and $cs_n(a) > cs_n(b)$, the minimum distance is $\frac{1}{2^n}$. 
  So $cs_n(a) \geq cs_n(b) + \frac{1}{2^n} $. 

  CLAIM 3: 
  $cs_n$ is injective for any $n:\N$. 

  Assume $\alpha\leq_I \gamma$ and $\alpha\sim_I\beta$. 
  So 
  $\forall_{n:\N} (cs(\alpha)_n \leq_\mathbb Q cs(\gamma)_n)+\frac{1}{2^n}$. 
  and $\forall_{n:\N}(\alpha|_n\sim_n\beta|_n)$. 

  We need to show that 
  $\forall_{n:\N} (cs(\beta)_n \leq_\mathbb Q cs(\gamma)_n+ \frac{1}{2^n} ) $. 

  This is closed, so we can show the double negation instead. 
  By Markov, the negation is that there is some 
  $N$ with 
  $$cs(\beta)_N >_\mathbb Q cs(\gamma)_N + \frac{1}{2^N}.$$
  
  We are given that 
  $cs(\gamma)_N + \frac{1}{2^n}\geq cs(\alpha)|_N $. 
  Thus $$cs(\beta)_N > cs(\gamma)_N + \frac{1}{2^N} \geq cs(\alpha)_N$$
  By Claim 1 and as $\alpha|_N\sim_N\beta|_N$, we must have that  
  $cs(\beta)_N = cs(\alpha)_N +\frac{1}{2^N} $. 
%  \begin{itemize}
%    \item 
%      If $cs(\beta)|_N = cs(\alpha)_N$, we have $\alpha|_N = \beta|_N$ by claim 3. 
%      Hence as $cs(\alpha)_N \leq_Q cs(\gamma)_N$, we also have 
%      $cs(\beta)_N \leq_Q cs(\gamma)_N$, contradicting our assumption and we're done. 
%    \item 
%      If $cs(\beta)|_N = cs(\alpha)_N + \frac{1}{2^N}$, we have that 
  Therefore 
      $$
      cs(\alpha)_N+\frac{1}{2^N} \geq cs(\gamma)_N + \frac{1}{2^N} \geq cs(\alpha)_N
      $$
      So $$cs(\alpha)_N \geq cs(\gamma)_N\geq cs(\alpha)_N -\frac{1}{2^N}.$$
      As a consequence of claim 2, we have that 
      $cs(\alpha)_N = cs(\gamma)_N \vee cs(\alpha)_N -\frac{1}{2^N} = \gamma_N$. 
      \begin{itemize}
        \item If $cs(\alpha)_N = cs(\gamma)_N$, we have that 
          $\alpha|_N = \gamma|_N$, and hence $cs(\beta)_N = cs(\gamma)_N + \frac{1}{2^N}$, 
          contradicting our assumption. 
        \item 
          If $cs(\alpha)_N -\frac{1}{2^N} = cs(\gamma)_N$, we still need that 
          $cs(\alpha)_n \leq cs(\gamma)_n + \frac{1}{2^n}$ for all $n\geq N$. 
          But I now claim this can only happen if
          \rednote{TODO, combine this with minimality thing in the other lemma}
          $cs(\gamma)_n = cs(\alpha)_n + \frac{1}{2^n}$ for all $n\geq N$. 
          But in this case, we can still deduce that 
          $\alpha\sim_I \gamma$, but then $\alpha = \gamma \vee \beta = \gamma \vee \alpha = \beta$, 
          contradicting our assumption. 
      \end{itemize}
%\end{itemize} 



  %  In this case, we have 
  %  $cs(\gamma)_N +\frac{1}{2^N} \geq cs(\alpha)_N > cs(\gamma)_N$. 
  %  In particular $\gamma|_N \sim_N \alpha|_N$. 
  %  $cs(\alpha)_N > cs(\gamma)_N$. 


    
  \newpage



















































%  Assume $\alpha\leq_I \beta$ and $\alpha\sim_I\gamma$. 
%  We need to show that $\gamma\leq_I\beta$ as well, so for each $n:\N$, we need to show that 
%  $cs(\gamma)_n \leq cs(\beta)_n + \frac{1}{2^n}$. 
%  As inequality in $\mathbb Q$ is decidable, we can proceed with a proof by contradiction. 
%  Suppose $cs(\gamma)_n > cs(\beta)_n + \frac{1}{2^n} \geq cs(\alpha)_n$. 
%  Then by the above Lemma, we get 
%  $\beta|_n =\gamma|_n \vee \beta|_n = \alpha|_n$. 
%  In both cases, we get $|\beta|_n - \gamma|_n|\leq \frac{1}{2^n}$, contradicting our assumption. 
%  Thus $cs(\gamma)_n \leq cs(\beta)_n + \frac1{2^n}$. 
%
%  Assume now $\alpha\leq_I\beta$ and $\beta\sim_I\gamma$. 
%  We need to show that $\alpha \leq_I\gamma$ as well. 
\end{proof}
\begin{lemma}
  $<_I$ respects $\sim_I$. 
\end{lemma} 
\begin{proof}
  TODO
\end{proof}
\begin{remark}
  By the above, $\leq_I, <_I$ induce relations $\leq,<$ on $I$.
  As inequality in $\mathbb Q$ is decidable, $\leq, <$ are closed and open respectively. 
%
  We can use these inequalities to define the standard open and closed intervals. 
  Let $a,b:I$. 
  Following standard notation, we denote
  \begin{equation}
    [a,b]:= \Sigma_{x:I} (a\leq x \wedge x \leq b),
  \end{equation}
  which is closed by \Cref{ClosedCountableConjunction}, 
  we call subsets of $I$ of this form closed intervals. 
%
  We also denote 
  \begin{equation}
    (a,b) := \Sigma_{x:I} (a < x \wedge x < b),
  \end{equation}
  which is open by \Cref{OpenFiniteConjunction}.
  we call subsets of $I$ of this form open intervals. 
\end{remark}

\begin{lemma}
  Let $D_n:2^\N \to 2$ be a sequence of decidable subsets with $D_{n+1}\subseteq D_n$.
  For $p$ the quotient map $2^\N \to I$, we have that 
  $p(\bigcap_{n:\N} D_n) = p(\bigcap_{n:\N} D_n)$
\end{lemma}
\begin{proof}
  It is always the case that $$p(\bigcap_{n:\N} D_n) \subseteq \bigcap_{n:\N} p(D_n).$$
  For the converse direction, let $(\bigcap_{n:\N} p(D_n))(x)$. 
  We will show that $ \neg \neg (p(\bigcap D_n)) (x)$, which is sufficient by \Cref{rmkOpenClosedNegation}. 
%
  As $(\bigcap_{n:\N} p(D_n))(x)$, there exists some $y\in D_0$ with $p(y) = x$. 
%
  If $x\notin p(\bigcap_{n:\N} D_n)$, we cannot have for all $n:\N$ that $y_0 \in  D_n$. 
  By Markov, there must exist some $k:\N$ with $\neg D_k(y_0)$. 
  As $D_{n+1}\subseteq D_n$ for all $n:\N$, it follows that $y_0\notin D_n$ for all $n\geq k$. 
%
  As $x\in \bigcap_{n:\N}p(D_n)$, there is however some $y_k\in D_k$ with $p(y_k) = x$. 
  By a similar argument, we have some $l>k$ with $y_k\notin D_l$, and some $y_l$ with $p(y_l) = x, y_l \in D_l$. 
  But now we have that $y_0, y_k, y_l:2^\N$ are all distinct, but $p(y_0) = p(y_k) = p(y_l) = x$. 
  This contradicts \Cref{IntervalFiberSizeAtMost2}, and we're done. 
\end{proof}


\begin{lemma}
  For $p:2^\N \to I$ the quotient map and $D\subseteq 2^\N$ decidable, we have $p(D)$ a finite union of closed intervals. 
\end{lemma}
\begin{proof}
  We will show the above if there exists some $n:\N, u:2^n$ such that $D(x) \leftrightarrow x|_n = a$.
  This is sufficient, as any decidable subset of $2^\N$ can be written as finite union of such decidable subsets. 
  We claim that $p(D) = [p(a\cdot \overline 0) , p(a \cdot \overline 1)$. 
\item 
  We will first show that $p(D) \subseteq [p(a\cdot \overline 0) , p(a \cdot \overline 1)$. 
  Let $x\in D$. 
  Then $x|_n = a$ and hence for $m\leq n$ we have 
  \begin{equation}
    cs(x)_m = cs_m(a|_m) = cs(a\cdot \overline 0)_m= cs(a\cdot \overline 1)_m
  \end{equation}
  For $m>n$, we have that 
  \begin{align}
    cs(a\cdot \overline 1)_m =
    cs_n(a) +\sum_{i = n} ^{m-1} \frac{1}{2^{i+1}}
    \\
    cs(x)_m =
    cs_n(a) +\sum_{i = n} ^{m-1} \frac{x(i)}{2^{i+1}}
    \\
    cs(a\cdot \overline 0)_m = 
    cs_n(a) +\sum_{i = n} ^{m-1} \frac{0}{2^{i+1}}
  \end{align} 
  Hence for all $m:\N$, we have 
  \begin{equation}
    cs(a\cdot \overline 1)_m \geq 
    cs(x)_m \geq 
    cs(a\cdot\overline 0)_m
  \end{equation}
 which implies that $p(a\cdot \overline 1) \geq_I p(x) \geq_I p(a\cdot\overline 0)$, as required. 
\item 
  To show that $[p(a\cdot \overline 0) , p(a \cdot \overline 1)\subseteq p(D)$, 
  Suppose
  $p(a\cdot \overline 0) \leq p(x) \leq p(a \cdot \overline 1)$. 
  We will show that 
  $$x|_n = a \vee x \sim_I a \cdot \overline 0 \vee x \sim_I a \cdot \overline 1.$$
  As this is a disjunction of closed propositions, by \Cref{ClosedFiniteDisjunction} it's closed, and by 
  \Cref{rmkOpenClosedNegation}, we can instead show the double negation. 
  So suppose that none of the disjoints hold. 
  As $x|_n \neq a$, there is some minimal $m$ with $x(m) \neq a(m)$. 
  We assume that $x(m) = 1, a(m) = 0$, the other case goes similarly. 
  Then for all $k:\N$, we have 
  $cs(x)_k \geq cs(a \cdot \overline 1)|_k$. 
  As also 
  $p(a\cdot \overline 1)\geq p(x)$, we have 
  $$cs(a \cdot \overline 1)|_k + \frac{1}{2^k} \geq cs(x)_k \geq cs(a\cdot \overline 1)_k,$$
  From which it follows that $|cs(a\cdot\overline 1)_k - cs(x)_k|\leq \frac{1}{2^k}$. 
  Hence $(a\cdot \overline 1)|_k \sim_k x|_k$ by \Cref{alternativeSimByCauchyDistance}. 
  Hence $x\sim_I (a\cdot\overline 1)$, contradicting our assumption as required. 
\end{proof}

\begin{lemma}
  Every open $U\subseteq I$ can be written as countable union of open intervals.
\end{lemma} 
\begin{proof}
  TODO
%  Let $U\subseteq I$ open, then $U^C\subseteq I$ is closed. 
%  By \Cref{StoneClosedSubsets} and \Cref{CompactHausdorffClosed}, we have that 
%  there is some sequence $D_n\subseteq 2^\N$ with $p^{-1}(U^C) = \bigcap_{n:\N} D_n$. 
%  As the quotient map $p$ is surjective, we have that $U^C = p(p^{-1}(U^C))$. 
%  By the above, it follows that $\neg U = \bigcap_{n:\N} p(D_n)$. 
%  By \Cref{TODO}, it follows that 
%  $\neg U$ is a countable intersection of finite unions of closed intervals. 
%  Thus $\neg\neg U$ is a countable union of finite intersections of complements of closed intervals. 
%  As complements of closed intervals are finite unions of open intervals (TODO), 
%  and finite intersections of such things are still finite unions of open intervals, 
%  it follows that $\neg\neg U$ is a countable union of open intervals. 
%  By \Cref{rmkOpenClosedNegation}, $\neg \neg U = U$ and we're done. 
%  \rednote{Lotta handwaving here, definitely not finished} 
\end{proof}


\begin{remark}
  It follows that the topology of $I$ is generated by open intervals, 
  which corresponds to the standard topology on $I$. 
  Hence our notion of continuity corresponds with the $\epsilon,\delta$-definition of continuity one would expect. 
  Thus every function $f:I\to I$ in the system we presented is continuous in the $\epsilon,\delta$-sense. 
\end{remark}

 

\section{Cohomology}
In non-synthetic algebraic geometry,
the structure sheaf~$\mathcal{O}_X$ is part of the data constituting a scheme~$X$.
In our internal setting,
the scheme $X$ is just a set without any additional data,
but when we want to consider the structure sheaf as an object in its own right,
then we can represent it by the trivial bundle
that assings to every point $x : X$ the set $R$.
Indeed, for an affine scheme $X = \Spec A$,
taking the sections of this bundle over a basic open $D(f) \subseteq X$
\[ (\prod_{x : D(f)} R) = (D(f) \to R) = A[f^{-1}] \]
yields the localizations of the ring $A$
expected from the structure sheaf $\mathcal{O}_X$.
More generally,
instead of sheaves of abelian groups, $\mathcal{O}_X$-modules, etc.,
we will consider bundels of abelian groups, $R$-modules, etc.,
in the form of maps from $X$ to the respective type of algebraic structures.

\subsection{Quasi-coherent bundles}

This subsection is still experimental.

Sometimes we want to ``apply'' a bundle to a subtype,
like sheaves can be evaluated on open subspaces
and introduce the common notation ``$M(U)$'' for that below.
It is, however, not justified to expect, that this application
and the corresponding theory of ``sheaves'' is ``the same'' as the external one,
since the definition below, uses the internal hom ``$\prod$''
-- where the corresponding external construction, would be the set of continuous sections of a bundle.

\begin{definition}
  \index{$M(U)$}
  Let $X$ be a type and $M:X\to \Mod{R}$ a dependent module.
  Let $U\subseteq X$ be any subtype.
  \begin{enumerate}[(a)]
  \item We write:
    \[
      M(U)\colonequiv \prod_{x:U}M_x
      \rlap{.}
    \]
  \item With pointwise structure, $U\to R$ is an $R$-algebra
    and $M(U)$ is a $(U\to R)$-module.
  \end{enumerate}
\end{definition}

Somewhat surprisingly, localization of modules $M(U)$
can be done pointwise:

\begin{lemma}[using \axiomref{loc}, \axiomref{sqc}, \axiomref{Z-choice}]%
  \label{module-bundle-localization-pointwise}
  Let $X$ be a scheme and $M:X\to \Mod{R}$ a dependent module.
  Let $U=\Spec A\subseteq X$ be open affine.
  Let $f:A$.
  \begin{enumerate}[(a)]
  \item There is a morphism
    \[
      M(U)_f\to \prod_{x:U}(M_x)_{f(x)}
      \rlap{.}
    \]
  \item Let $g,h:M(U)_f$. Then $g=h$ if and only if
    \[
      \prod_{x:U}g(x)=_{(M_x)_{f(x)}}h(x)
      \rlap{.}
    \]
  \item The morphism in (a) is an equivalence, i.e.
    \[
      M(U)_f=\prod_{x:U}(M_x)_{f(x)}
      \rlap{.}
    \]
  \end{enumerate}
\end{lemma}

\begin{proof}
  \begin{enumerate}[(a)]
  \item We have to show, that the map
    \[
      \frac{m}{f^k}\mapsto\left(x\mapsto \frac{m(x)}{f(x)^k}\right)
    \]
    is well-defined. So let $\frac{m}{f^k}=\frac{m'}{f^{k'}}$,
    i.e. let there be an $l:\N$ such that $f^l(mf^{k'}-m'f^k)=0$.
    But then we can choose the same $l:\N$ for each $x:U$
    and apply the equation to each $x:U$.
  \item The forward direction was treated in (a).
    So let $g,h:M(U)_f$ such that $p:\prod_{x:U}g(x)=_{(M_x)_{f(x)}}h(x)$.
    Let $m_g,m_h:\prod_{x:U} M_x$ and $k_g,k_h:\N$ such that
    \[
      g=\frac{m_g}{f^{k_g}} \quad\text{and}\quad h=\frac{m_h}{f^{k_h}}
      \rlap{.}
    \]
    From $p$ we know $\prod_{x:U}\exists_{k_x:\N}f(x)^{k_x}(m_g(x)f(x)^{k_h}-m_h(x)f(x)^{k_g})=0$.
    By \cref{strengthened-boundedness},
    we find one $k : \N$ with
    \[
      \prod_{x:U}f(x)^{k}(m_g(x)f(x)^{k_h}-m_h(x)f(x)^{k_g})=0
    \]
    --- which shows $g=h$.
  \item The map in (a) is injective by (b);
    it remains to show that it is surjective.
    So let $\varphi:\prod_{x:U}(M_x)_{f(x)}$ and
    note that
    \[
      \prod_{x:U}
      \exists_{k_x:\N,m_x:M_x}
      \varphi(x)=\frac{m_x}{f(x)^{k_x}}
      \rlap{.}
    \]
    By \cref{strengthened-boundedness} and \axiomref{Z-choice},
    we get $k:\N$, coprime $a_1,\dots,a_l:A$ and $m_i:(x : D(a_i))\to M_x$
    such that for each $i$ and $x:D(a_i)$ we have
    \[
      \varphi(x)=\frac{m_i(x)}{f(x)^{k}}
      \rlap{.}
    \]
    The problem is now to construct a global $m:(x:U)\to M_x$ from the $m_i$.
    We have
    \[
        \prod_{x:D(a_ia_j)}\frac{m_i(x)}{f(x)^k}=\varphi(x)=\frac{m_j(x)}{f(x)^k}
    \]
    meaning there is pointwise an exponent $t_x:\N$,
    such that $f(x)^{t_x}m_i(x)=f(x)^{t_x}m_j(x)$.
    By \cref{strengthened-boundedness},
    we can find a single $t:\N$ with this property and define
    \[
      \tilde{m}_i(x) \colonequiv f(x)^t m_i(x)
      \rlap{.}
    \]
    Then we have $\tilde{m}_i(x)=\tilde{m}_j(x)$ on all intersections $D(a_i)\cap D(a_j)$,
    which is what we need to get a global $m:(x:U)\to M_x$ from \cref{kraus-glueing}.
    Since $\varphi(x)=\frac{f(x)^t m_i(x)}{f(x)^{t+k}}=\frac{\tilde{m}_i(x)}{f(x)^{t+k}}$
    for all $i$ and $x : D(a_i)$,
    we have found a preimage of $\varphi$ in $M(U)_f$.
  \end{enumerate}
\end{proof}

We will need the following algebraic lemma:

\begin{lemma}%
  \label{localization-to-module-if-non-zero}
  Let $M$ be an $R$-module and $f:R$,
  then there is an $R$-linear map
  \[
    M_f\to M^{D(f)}
    \rlap{.}
  \]
\end{lemma}

\begin{proof}
  Let $x\equiv \frac{m}{f^k}:M_f$ and $p:D(f)$.
  Then $f$ is invertible, so we have
  \[
    x\equiv \frac{m}{f^k}=\frac{f^{-k}m}{1}
  \]
  and mapping $x$ to $f^{-k}m$ is an $R$-linear map.
  
\end{proof}

\begin{lemma}[using \axiomref{sqc}, \axiomref{loc}, \axiomref{Z-choice}]%
  \label{localization-to-restriction}                    
  Let $X$ be a scheme, $M:X\to\Mod{R}$, $U=\Spec A\subseteq X$ open and $f:A$.
  Then there is an $R$-linear map
  \[
    M(U)_f \to M(D(f)) 
    \rlap{.}
  \]
\end{lemma}

\begin{proof}
  Combining \cref{module-bundle-localization-pointwise}
  and pointwise application of \cref{localization-to-module-if-non-zero} we get
  \[
    M(U)_f=\left(\prod_{x:U}(M_x)_{f(x)}\right)\to \left(\prod_{x:U}(M_x)^{D(f(x))}\right)
    =\left(\prod_{x:D(f)}M_x\right)
    =M(D(f))
  \]
\end{proof}

The following is an experimental definition,
which might be suitable
to mimic the external notion of quasi-coherent $\mathcal O_X$-module sheaves.

\begin{definition}%
  \label{quasi-coherent-bundle}
  Let $X$ be a scheme.
  A dependent module $M:X\to \Mod{R}$ is \notion{quasi-coherent},
  if for all $x:X$ and $f:R$,
  the canonical map from \cref{localization-to-module-if-non-zero} is an equivalence:
  \[
    (M_x)_f\simeq M_x^{D(f)}
    \rlap{.}
  \]
\end{definition}

An immediate consequence is, that
quasi coherent dependent modules have
the property that ``restricting is the same as localizing'':

\begin{lemma}[using \axiomref{sqc}, \axiomref{loc}, \axiomref{Z-choice}]
  Let $X$ be a scheme and $M:X\to \Mod{R}$ quasi-coherent,
  then for all open affine $U=\Spec A\subseteq X$ and $f:A$
  the canonical morphism
  \[
    M(U)_f\to M(D(f))
  \]
  is an equivalence.
\end{lemma}

\begin{proof}
  By construction of the canonical map from \cref{localization-to-restriction}.
\end{proof}

Let us look at an example.

\begin{proposition}
  \label{fp-algebra-bundle-is-quasi-coherent}
  Let $X$ be a scheme and $C:X\to \Alg{R}_{fp}$.
  Then $C$, as a bundle of $R$-modules, is quasi coherent.
\end{proposition}

\begin{proof}
  Then for any $f:R$ and $x:X$, using \cref{algebra-valued-functions-on-affine}, we have
  \[
    (C_x)_f=C_x\otimes_R R_f=(\Spec R_f \to C_x)=(D(f)\to C_x)={C_x}^{D(f)}
    \rlap{.}
  \]
\end{proof}

\begin{proposition}[using \axiomref{loc}, \axiomref{sqc}, \axiomref{Z-choice}]
  Not every $R$-module is quasi-coherent
  in the sense of \cref{quasi-coherent-bundle}.
\end{proposition}

\begin{proof}
  We construct a family of $R$-modules,
  parametrized by the elements of $R$,
  and deduce a contradiction from the assumption that
  all modules of this family are quasi-coherent.

  Given an element $f : R$,
  the $R$-module we want to consider is
  the countable product
  \[ M(f) \colonequiv \prod_{n : \N} R/(f^n) \rlap{.} \]
  If $f \neq 0$ then $M(f) = 0$
  (using \cref{non-zero-invertible}).
  This implies that the $R$-module $M(f)^{f \neq 0}$
  is trivial:
  any function $f \neq 0 \to M(f)$ can only assign the value $0$
  to any of the at most one witnesses of $f \neq 0$.
  If $M(f)$ is quasi-coherent,
  then this means that $M(f)_f$ is also trivial.
  Noting that
  $M(f)$ is not only an $R$-module
  but even an $R$-algebra in a natural way,
  we have
  \begin{align*}
    M(f)_f = 0
    &\;\Leftrightarrow\;
    \exists k : \N.\; \text{$f^k = 0$ in $M(f)$} \\
    &\;\Leftrightarrow\;
    \exists k : \N.\; \forall n : \N.\; f^k \in (f^n) \subseteq R \\
    &\;\Leftrightarrow\;
    \exists k : \N.\; f^k \in (f^{k + 1}) \subseteq R
    \rlap{.}
  \end{align*}

  In summary,
  if the module $M(f)$ is quasi-coherent
  for every $f : R$,
  then the ring $R$ is zero-dimensional
  in the sense of \cref{zero-dimensional-ring}.
  But this is not the case,
  as we saw in \cref{R-not-zero-dimensional}.
\end{proof}

\begin{lemma}[using \axiomref{sqc}, \axiomref{loc}, \axiomref{Z-choice}]%
  \label{weakly-quasi-coherent-pi}
  Let $X$ be an affine scheme and $M_x$ a weakly quasi-coherent $R$-module for any $x:X$,
  then
  \[
    \prod_{x:X}M_x
  \]
  is weakly quasi-coherent.
\end{lemma}

\begin{proof}
  TODO
\end{proof}

Quasi-coherent dependent modules turn out to have very good properties,
which are to be expected from what is known about their external counterparts.
We will show below, that quasi coherence is preserved by the following constructions:

\begin{definition}
  \label{pullback-push-forward}
  Let $X,Y$ be types and $f:X\to Y$ be a map.
  \begin{enumerate}[(a)]
  \item \index{$f^*M$} For any dependent module $N:Y\to\Mod{R}$,
    the \notion{pullback} or \notion{inverse image} is the dependent module
    \[
      f^*N\colonequiv (x:X) \mapsto M_{f(x)}\rlap{.}
    \]
  \item \index{$f_*M$} For any dependent module $M:X\to\Mod{R}$,
    the \notion{push-forward} or \notion{direct image} is the dependent module
    \[
      f_*M\colonequiv (y:Y) \mapsto \prod_{x:\fib_f(y)}M_{\pi_1(x)}\rlap{.}
    \]
  \end{enumerate}
\end{definition}

\begin{theorem}[using \axiomref{sqc}, \axiomref{loc}, \axiomref{Z-choice}]%
  \label{pullback-push-forward-qcoh}
  Let $X,Y$ be schemes and $f:X\to Y$ be a map.
  \begin{enumerate}[(a)]
  \item For any quasi-coherent dependent module $N:Y\to\Mod{R}$,
    the inverse image $f^*N$ is quasi-coherent.
  \item For any dependent module $M:X\to\Mod{R}$,
    the direct image $f_*M$ is quasi-coherent.
  \end{enumerate}
\end{theorem}

\begin{proof}
  \begin{enumerate}[(a)]
  \item There is nothing to do, when we use the pointwise definition of quasi-coherence. 
  \item TODO, Ideas:

    Show that the dependent product of modules is a module.
    Show that this product preserves qcoh, if the index type is a scheme.
    Use that the fiber of a scheme morphism is a scheme.
  \end{enumerate}
\end{proof}

\subsection{Finitely presented bundles}

We now investigate the relationship between bundles of $R$-modules on $X = \Spec A$
and $A$-modules.

\begin{proposition}
  Let $A$ be a finitely presented $R$-algebra.
  There is an adjunction
  \[ \begin{tikzcd}[row sep=tiny]
    M \ar[r, mapsto] & {(M \otimes x)}_{x : \Spec A} \\
    \Mod{A} \ar[r, shift left=2] \ar[r, phantom, "\rotatebox{90}{$\vdash$}"] &
    \Mod{R}^{\Spec A} \ar[l, shift left=2] \\
    \prod_{x : \Spec A} N_x & N \ar[l, mapsto]
  \end{tikzcd} \]
  between the category of $A$-modules
  and the category of bundles of $R$-modules on $\Spec A$.
\end{proposition}

\begin{theorem}%
  \label{fp-module}
  Let $X=\Spec(A)$ be affine and
  let a bundle of finitely presented $R$-modules $M : X\to \fpMod{R}$ be given.
  Then the $A$-module
  \[ \tilde{M}\coloneqq\prod_{x:X}M_x \]
  is finitely presented and for any $x:X$ the $R$-module $\tilde{M}\otimes_A R$ is $M_x$.
  Under this correspondence, localizing $\tilde{M}$ at $f:A$ corresponds to restricting $M$ to $D(f)$.
\end{theorem}

\subsection{Cohomology on affine schemes}

\begin{definition}%
  \label{torsor}
  Let $X$ be a type and $A:X\to \AbGroup$ a map to the type of abelian groups.
  For $x:X$ let $T_x$ be a set with an $A_x$ action.
  \begin{enumerate}[(a)]
  \item $T$ is an \notion{$A$-pseudotorsor}, if the action is free and transitive for all $x:X$.
  \item $T$ is an \notion{$A$-torsor}, if it is an $A$-pseudotorsor and
    \[ \prod_{x:X} \| T_x \| \rlap{.}\]
  \item We write $\Tors{A}(X)$ for the type of $A$-torsors on $X$.
  \end{enumerate}
\end{definition}

Torsors on a point are a concrete implementaion of first deloopings:

\begin{definition}
  \label{delooping}
  Let $n:\N$.
  A $n$-th \notion{delooping}\index{$K(A,n)$} of an abelian group $A$,
  is a pointed, $(n-1)$-connected, $n$-truncated type $K(A,n)$,
  such that $\Omega^nK(A,n)=_{\AbGroup}A$.
\end{definition}

For any abelian group and any $n$, a delooping $K(A,n)$ exists by \cite{licata-finster}.
Deloopings can be used to represent cohomology groups by mapping spaces.
This is usually done in homotopy type theory to study higher inductive types, such as spheres and CW-complexes,
but the same approach works for internally representing sheaf cohomology,
which is the intent of the following definition:

\begin{definition}
  \label{cohomology}
  Let $X$ be a type and $\mathcal F:X\to\AbGroup$ a dependent abelian group.
  The $k$-th cohomology group of $X$ with coefficients in $\mathcal F$ is
  \[
    H^k(X,\mathcal F)\colonequiv \left\|\prod_{x:X}K(\mathcal F,k)\right\|_0\rlap{.}
  \]
\end{definition}


The following is an explicit formulation of the fact, that the Čech-Complex for an
$\mathcal{O}_X$-module sheaf on $X=\Spec(A)$ given by an $A$-module $M$ is exact in degree 1.
\begin{lemma}%
  \label{H1-algebra}
  Let $M$ be a module over a commutative ring $A$, $F_1,\dots,F_l$ a coprime system on $A$
  and for $i,j\in\{1,\dots,l\}$, let $s_{ij} : F_i^{-1} F_j^{-1} M$ such that:
  \[ s_{jk}-s_{ik}+s_{ij}=0 \rlap{.}\]
  Then there are $u_i:F_i^{-1}M$ such that $s_{ij}=u_j - u_i$.
\end{lemma}

\begin{proof}
  Let $s_{ij}=\frac{m_{ij}}{f_i f_j}$ with $m_{ij}:M$, $f_i:F_i$ and $f_j:F_j$ such that:
  \[ f_i\cdot m_{jk}-f_j\cdot m_{ik}+f_k\cdot m_{ij}=0 \rlap{.}\]
  Let $r_i$ such that $\sum r_i f_i =1$.
  Then for
  \[ u_i \coloneqq -\sum_{k=1}^l\frac{r_k}{f_i}m_{ik} \]
  we have:
  \begin{align*}
      u_j-u_i &= -\sum_{k=1}^l\frac{r_k}{f_j}m_{jk} + \sum_{k=1}^l\frac{r_k}{f_i}m_{ik} \\
              &= -\sum_{k=1}^l\frac{r_k}{f_j f_i}f_i m_{jk} + \sum_{k=1}^l\frac{r_k}{f_i f_j} f_j m_{ik} \\
              &= \sum_{k=1}^l\frac{r_k}{f_j f_i}(-f_i m_{jk} + f_j m_{ik}) \\
              &= \sum_{k=1}^l\frac{r_k}{f_j f_i}f_k m_{ij} \\
              &= \frac{m_{ij}}{f_i f_j}
  \end{align*}
  \ %
\end{proof}

\begin{theorem}[using \axiomref{Z-choice}]%
  \label{H1-fp-module-affine-trivial}
  For any affine scheme $X=\Spec(A)$ and coefficients $M: X\to \fpMod{R}$, we have
  \[ H^1(X,M)=0 \rlap{.} \]
\end{theorem}
\begin{proof}
  We need to show, that any $M$-torsor $T$ on $X$ is merely equal to the trivial torsor $M$,
  or equivalently show the existence of a section of $T$.
  We have
  \[ \prod_{x:X}\| T_x \|\]
  and therefore, by (\axiomref{Z-choice}),
  there merely are $f_1,\dots,f_l:A$,
  such that the $U_i\coloneqq \Spec(A_{f_i})$ cover $X$ and
  there are local sections
  \[ s_i:\prod_{x:U_i}T_x\]
  of $T$. Our goal is to construct a matching family from the $s_i$.
  On intersections, let $t_{ij}\coloneqq s_i-s_j$ be the difference, so $t_{ij}:(x : U_i\cap U_j) \to M_x$.
  By \cref{fp-module} equivalently, we have $t_{ij}:\tilde{M}_{f_i f_j}$.
  Since the $t_{ij}$ were defined as differences,
  the condition in \cref{H1-algebra} is satisfied and we get
  $u_i:\tilde{M}_{f_i}$, such that $t_{ij}=u_i-u_j$.
  So we merely have a matching family $\tilde{s}_i\coloneqq s_i-u_i$ and therefore, using Lemma \ref{kraus-glueing} merely a section of $T$.
\end{proof}

A similar result is provable for $H^2(X,M)$ and we expect that $H^n(X,M)$ holds, at least for any external $n$.

\subsection{Čech-Cohomology}

In this section, let $X$ be a type, $U_1,\dots,U_n\subseteq X$ open subtypes that cover $X$
and $\mathcal F:X\to \AbGroup$ a dependent abelian group on $X$.
We start by repeating the classical definition of Chech-Cohomology groups for a given cover.

\begin{definition}%
  \label{chech-complex}
  \begin{enumerate}[(a)]
  \item \index{$\mathcal F(U)$} For open $U\subseteq X$, we use the notation
    \[
      \mathcal F(U)\colonequiv \prod_{x:U}\mathcal F_x\rlap{.}
    \]
  \item For $s:\mathcal F(U)$ and open $V\subseteq U$ we use the notation $s\colonequiv s_{|V} \colonequiv (x:V)\mapsto s_x$.
  \item \index{$U_{i_1\dots i_l}$}For a selection of indices $i_1,...,i_l:\{1,\dots,n\}$, we use the notation
    \[
      U_{i_1\dots i_l}\colonequiv U_{i_1}\cap\dots\cap U_{i_l}\rlap{.}
    \]
  \item For a list of indices $i_1,\dots,i_l$, let $i_1,\dots,\hat{i_t},\dots,i_l$ be the same list with the $t$-th element removed.
  \item For $k:\Z$, the $k$-th \notion{Čech-boundary operator}\index{$\partial^k$} is the homomorphism
    \[
      \partial^k:\bigoplus_{i_0,\dots,i_k}\mathcal F(U_{i_0\dots i_k})\to \bigoplus_{i_0,\dots,i_{k+1}}\mathcal F(U_{i_0\dots i_{k+1}})
    \]
    given by $\partial^k(s)\colonequiv (l_0,\dots,l_{k+1}) \mapsto \sum_{j=0}^k (-1)^j s_{l_0,\dots,\hat{l_j},\dots,l_k|U_{l_0,\dots,l_{k+1}}}$.
  \item The $k$-th \notion{Čech-Cohomology group} for the cover $U_1,\dots,U_n$ with coefficients in $\mathcal F$ is
    \[
      \check{H}^k(\{U\},\mathcal F)\colonequiv \ker\partial^{k} / \im(\partial^{k-1})\rlap{.}
    \]
  \end{enumerate}
\end{definition}

\begin{definition}
  The cover $U_1,\dots,U_n$ is called \notion{acyclic} for $\mathcal F$,
  if for all $k:\N$ and $i_0,\dots,i_k$, we have that the higher (non Čech) cohomology groups are trivial:
  \[
    \forall l>0. H^l(U_{i_0,\dots,i_k},\mathcal F)=0\rlap{.}
  \]
\end{definition}

\begin{example}
  If $X$ is a scheme, $U_1,\dots,U_n$ a cover by affine open subtypes and $\mathcal F$ pointwise a finitely presented $R$-module,
  then $U_1,\dots,U_n$ is acyclic for $\mathcal F$ by \cref{H1-fp-module-affine-trivial}.
\end{example}

\begin{theorem}[using \axiomref{Z-choice}]%
  If $U_1,\dots,U_n$ is an acyclic cover for $\mathcal F$, then
  \[
    \check{H}^1(\{U\},\mathcal F)=H^1(X,\mathcal F)\rlap{.}
  \]
\end{theorem}

\begin{proof}
  Let $\pi$ be the projection map
  \[
    \pi :
    \left(
      \sum_{T:\Tors{\mathcal F}(X)}\prod_{i}\prod_{x:U_i}T_x
    \right)
    \to \Tors{\mathcal F}(X)\rlap{.}
  \]
  Let us abbreviate the left hand side with $T(\mathcal F,U)$.
  Since the cover is acyclic, $\pi$ is surjective.
  There is a map $\iota$ into the kernel of $\partial^1$ (\cref{chech-complex} (e)):
  \[
    \iota \colonequiv
    (T,t) \mapsto (i,j\mapsto t_i - t_j) :
    T(\mathcal F,U)
    \to
    \ker(\partial^1)
    \subseteq
    \bigoplus_{i,j}\mathcal F(U_{ij})\rlap{.}
  \]
  We will now show, that $\iota$ is an embedding and therefore also, that its domain is a set.
  Let $(T,t),(T',t'):T(\mathcal F,U)$ such that $\iota((T,t))=\iota((T',t'))$,
  i.e. for all $i,j$ we have $t_i-t_j=t'_i-t'_j$.
  The latter shows the well-definedness (needed to apply \cref{kraus-glueing})
  of a global map $T\simeq T'$, given by sending $t_i(x)$ to $t'_i(x)$
  for all $i$ and $x$.

  The map $\iota$ is also a surjection and therefore an isomorphism:
  Let $s:\ker(\partial^1)$.
  Then we can contruct a torsor,
  by starting with the trivial torsor on each $U_i$.
  We use \cref{kraus-glueing-1-type} to get a torsor
  with the identification given by the $s_{ij}$
  where the cocycle condition holds because $s$ is in the kernel.

  Realizing, that $\im(\partial^0)$ corresponds to the subtype of $T(\mathcal F,U)$ of trivial torsors,
  we arrive at the following diagram:
  \begin{center}
    \begin{tikzcd}
      & \Tors{\mathcal F}(X)\ar[r,->>] & H^1(X,\mathcal F) \\
      \sum_{T:T(\mathcal F,U)}\|\pi_1(T)=\mathcal F\|\ar[r,hook] & T(\mathcal F,U)\ar[u,->>]\ar[d,equal] & \\
      \im{\partial^0}\ar[r,hook]\ar[u,equal] & \ker{\partial^1}\ar[r,->>] & \check{H}^1(\{U\},\mathcal F)
    \end{tikzcd}
  \end{center}
  By \cref{MISSING},
  the composed map $T(\mathcal F,U)\to H^1(X,\mathcal F)$ is a homomorphism
  and therefore by \cref{surjective-abgroup-hom-is-cokernel} a cokernel.
  So the two cohomology groups are equal, since they are cokernels of the same diagram.
\end{proof}

%Appendix%\appendix
%Appendix%%\section{Technical details}
%Appendix%\section{Some notes on $\Noo$}
%Appendix%Recall that we defined $B_\infty$ as the quotient of the freely generated algebra 
over $p_n,~n\in\N$ by the relations $\{p_n \wedge p_m | n\neq m\}$. 

\begin{lemma}\label{N-co-fin-cp}
  The Boolean algebra of co-finite subsets of $\N$
  is equivalent to $B_\infty$. 
\end{lemma}
\begin{proof}
  Let $f:B_\infty \to \N_{(co)fin}$ be induced by sending $p_n$ to $\{n\}$. 
  Note that whenever $n\neq m$, we have 
  $f(p_n)\wedge f(p_m) = \{n\} \cap \{m\} = \emptyset$, 
  thus $f$ respects the relations of $B_\infty$ and is well-defined.

  Define $g:N_{(co)fin)} \to B_\infty$ as follows:
  \begin{itemize}
    \item On a finite subset $I$, we define $g(I) = \bigvee_{i\in I} p_i$, 
    \item On a cofinite subset $J$, we define $g(J) = \bigwedge _{i \in J^C} \neg p_i$. 
  \end{itemize}
  Note that in these cases we indeed have $I,J^C$ are finite, so these are well-defined elements. 
  We must show that $g$ is a Boolean morphism. 

  \begin{itemize}
    \item 
      By deMorgan's laws, $g$ preserves $\neg$:
      for $I$ finite we have
      \begin{equation}
      \neg g(I) = \neg (\bigvee_{i\in I} p_i) = \bigwedge_{i\in I} \neg p_i = g(I^C)
      \end{equation}
      And for $J$ cofinite, we apply similar reasoning. 
    \item To see that $g$ preserves $\vee$, we need to check three cases
      \begin{itemize}
        \item If both $I,J$ are finite, then 
        \begin{equation} 
          g(I \cup J) = \bigvee_{i\in I \cup J} p_i= \bigvee_{i\in I} p_i \vee \bigvee_{j\in J} p_j 
          = g(I) \vee g(J)
        \end{equation}
        and we're done. 
      \item If both $I,J$ are cofinite, we have
        \begin{equation}
          g(I) \vee g(J) = 
          \bigwedge_{i \in I^C} \neg p_i \vee 
          \bigwedge_{j \in J^C} \neg p_j 
          = 
          \bigwedge_{i\in I^C} 
          \bigwedge_{j \in J^C}(\neg p_i \vee  \neg p_j) 
        \end{equation}
        Now note that in $B_\infty$, we have 
        \begin{equation}
          \neg p_i \vee \neg p_j = \neg ( p_i \wedge p_j) = 
          \begin{cases}
            \neg p_i \text{ if } i = j\\
            1 \text{ if } i \neq j  
          \end{cases}
        \end{equation}
        Therefore, we can leave out the case that $i\neq j$ in the calculation of the above meet, and
        \begin{equation}
          \bigwedge_{i\in I^C} 
          \bigwedge_{j \in J^C}(\neg p_i \vee  \neg p_j)  
          = 
          \bigwedge_{i \in (I^C \cap J^C)} \neg p_i
          = 
          \bigwedge_{i \in (I \cup J)^C} \neg p_i 
        \end{equation}
        as $I\cup J$ must also be cofinite, this equals 
          $ g( I \cup J)$. 
        \item 
          If $I$ is finite and $J$ cofinite, we have 
          that $I\cup J$ is cofinite, hence 
          \begin{equation}
            g(I\cup J) = \bigwedge_{k\in (I \cup J)^C} \neg p_k
            = \bigwedge_{k \in (J^C -I)} \neg p_k
          \end{equation}
          Now note that 
          whenever $i\neq k$, we have 
          \begin{equation}
            p_i = (p_i \wedge \neg p_k) \vee (p_i \wedge p_k) = 
            (p_i \wedge \neg p_k) \vee 0 = p_i \wedge \neg p_k
          \end{equation}
          Hence by absorption
          \begin{equation} 
            (p_i \vee \neg p_k)  =
              \begin{cases}
                1 \text{ if } i = k \\
                \neg p_k \text{ if } i \neq k
              \end{cases}
          \end{equation}
          As for all $k\in J^C-I$ and all $i\in I$ we have $k\neq i$, we may thus write
          \begin{equation}\label{eqnCofiniteHelper1}
            \bigwedge_{k \in (J^C - I)} \neg p_k = 
            \bigwedge_{k \in (J^C - I)} (\neg p_k \vee (\bigvee_{i\in I} p_i))
          \end{equation}
          We now note that 
          \begin{equation}\label{eqnCofiniteHelper2}
            1=\bigwedge_{i\in I} 1 = \bigwedge_{i\in I} (\neg p_i \vee (\bigvee_{i\in I} p_i)).
          \end{equation}
          Taking the meet of the expressions in \Cref{eqnCofiniteHelper1} and \Cref{eqnCofiniteHelper2}, 
          we see that 
          \begin{equation}
            \bigwedge_{k \in (J^C - I)} \neg p_k = 
            \bigwedge_{j \in J^C} (\neg p_j \vee (\bigvee_{i\in I} p_i))
          \end{equation}
          And using distributivity rules, we can see that 
          \begin{equation}
            \bigwedge_{j \in J^C} (\neg p_j \vee (\bigvee_{i\in I} p_i))
            = 
            (\bigwedge_{j \in J^C} \neg p_k) \vee (\bigvee_{i\in I} p_i)
          \end{equation}
          From which we may conclude that $g(I\cup J) = g(I) \cup g(J)$. 
      \end{itemize}
    \item The case for $\wedge$ is completely dual to the case for $\vee$. 
  \end{itemize}
We conclude that $g$ is a Boolean morphism. 
Furthermore, it is easy to see that $g$ and $f$ are each other's inverse, 
thus the Boolean algebras are isomorphic. 
\end{proof}
\begin{remark}\label{AppendixCofiniteOrFinite}
  As a consequence of the above proof, any $b:B_\infty$ corresponds either to 
  \begin{itemize}
    \item a finite set $I$, in which case $b = \bigvee_{i\in I} p_i$. 
    \item a cofinite set $J$, in which case $b = \bigwedge_{j\in J^C} \neg p_j$. 
  \end{itemize}
  We will call $b$ finite/cofinite respectively. 
\end{remark}
\begin{remark}
Recall that $\Noo$ is defined as the spectrum of $B_\infty$. 
If $\alpha:\Noo$ satisfies $\alpha(p_n) = 1$, then $\alpha(p_m) = 0$ for all $n\neq m$. 
Therefore, for each $n:\N$, there is an unique map $\chi_n$ with $\chi_n(p_n) = 1$. 
There is also the point $\chi_\infty : \Noo$ which is unique 
with the property that $ \chi_\infty(p_n) = 0$ for all $n:\N$. 
We will call decidable subsets of $\Noo$ finite/cofinite iff their corresponding elements of $B_\infty$ are. 
\end{remark}
\rednote{Active WIP}
\begin{lemma}\label{FiniteDecidableSubsetsCharacterization}
  Finite decidable subsets of $\Noo$ are of the form 
%  If $d:B_\infty$ is of the form $\bigvee_{i\in I} p_i$,
%  it corresponds to the decidable set 
  $\{\chi_i | i \in I\}$ for some finite $I\subseteq \N$. 
\end{lemma}
\begin{proof}
  Let $d= \bigvee_{i\in I} p_i$. 
  Clearly whenever $i\in I$, we have $\chi_i(d) = 1$. 
%
  Now suppose $f:B_\infty \to 2$ is such that $f(d) = 1$. 
  Then $\bigvee_{i\in I}(f(p_i)) = 1$, hence it is not the case that $f(p_i) = 0$ for all $i\in I$. 
  Now as $I$ is finite and $f(p_i) = 0 \vee f(p_i) = 1$ for all $i\in I$, 
  there must exist some (necessarily unique) $i\in I$ with $f(p_i) = 1$. Hence $f = \chi_i$. 
%
  Thus $f(d) = 1$ iff there is some $i\in I$ with $f = \chi_i$. 
\end{proof}
\begin{corollary}\label{CoFiniteDecidableSubsetsCharacterization}
  Cofinite decidable subsets of $\Noo$ are of the form
%  If $d:B_\infty$ is of the form $\bigwedge_{j\in J^C} \neg p_j$,
%  it corresponds to the decidable set 
  $\neg \{\chi_i | i \in I\}$ for $J\subseteq\N$ finite. 
\end{corollary}
\begin{proof}
  Let $D$ be a cofinite decidable subset. Then $\neg D$ is a finite decidable subset, 
  By the above lemma it follows that $\neg D = \{\chi_i | i\in I\}$. 
  As $\neg \neg D = D$, the result follows. 
\end{proof}
\begin{corollary}
 Any a decidable subset $D\subseteq\Noo$ is cofinite iff $\chi_\infty\in D$. 
\end{corollary}
\begin{proof}
  This follows from the observation that $\chi_\infty \in \neg \{\chi_j | j \in J^i\}$, 
  the observation that all decidable subsets are either finite or cofinite, 
  and the characterization of finite a cofinite decidable subsets in 
  \Cref{FiniteDecidableSubsetsCharacterization} and 
  \Cref{CoFiniteDecidableSubsetsCharacterization}.
\end{proof}
\begin{corollary}
  If $U\subseteq \Noo$ is open and $\chi_\infty \in U$, there exists some $n\in \N$ such that 
  $\{\chi_k | k\geq n\} \subseteq U$. 
\end{corollary}
\begin{proof}
  If $U$ is open, by \Cref{StoneOpenSubsets}, it is a countable union of decidable subsets. 
  One of these must contain $\chi_\infty$, hence be cofinite and 
  of the form $\neg \{ \chi_i | i \in I\}$ for some finite $I\subseteq \N$.
  As $I$ is finite, there is some $n:\N $ with $n>i$ for all $i\in I$. 
  For all $k\geq n$, we have that $k\notin I$, hence $\chi_k \in \neg \{\chi_i | i \in I\}\subseteq U$ as required. 
\end{proof}



%
%\begin{lemma}
%  For all decidable subsets $D:\Noo\to 2$,
%  with $D$ non-empty, there exists some $n:\N$ with $\chi_n \in D$. 
%\end{lemma}
%\begin{proof}
%  We make a case distinction based on \Cref{AppendixCofiniteOrFinite}. 
%  \begin{itemize}
%    \item 
%      If $D$ corresponds to a finite $d:B_\infty$, but is non-empty, then 
%      $d=\bigvee_{i\in I} p_i$ for $I\subseteq \N$ finite and non-empty. 
%      If $I$ is finite (as in \Cref{dfnFinite}) and non-empty, 
%      $I\simeq Fin_k$ for some $k\neq 0$. 
%      In particular, there is a map $1 \to I$,
%      hence a term $i:I$. 
%      Then $\chi_i(d) = 1$, hence $\chi_i \in D$. 
%    \item 
%      If $D$ corresponds to some cofinite $d:B_\infty$, we have 
%      $d = \bigvee_{i\in I} \neg p_i$ for some $I\subseteq \N$ finite. 
%      Then there is some 
%\end{proof}
%



%Appendix%
%Appendix%\section{Cocompleteness of $\Boole$}
%Appendix%\rednote{TODO, is $\Boole$ closed under countable limits? 
  It has finite colimits, as it has pushouts and initial object.
  It should also have sequential colimits (TODO). 
  Is a countable coproduct the sequential colimit of it's initial finite coproducts? 
}
\begin{lemma}\label{BoolePushouts}
  Countably presented Boolean algebras are closed under pushout. 
\end{lemma} 
\begin{proof}
  Let $A,B,C:\Boole$, and suppose $f:A\to B, g:A \to C$ are Boolean morphisms. 
  Let $G_A, G_B,G_C$ be the underlying countable sets of generators for $B,C$ and 
  let $R_A,R_B,R_C$ be the underlying countable sets of relations. 
  Consider $P$ the Boolean algebra generated by $G_B\sqcup G_C$ under the relations 
  $R_B\cup R_C \cup F$ where $F$ is the set of expressions $f(a)-g(a), a\in G_A$.
  
  Note that as the generators of $B$ are included in those of $P$, 
  and all relations of $B$ are included in those of $P$, there is a map $h:B\to P$. 
  Similarly there is a map $i:C\to P$. 
  We now claim that the following is a pushout square:
  \begin{equation}\begin{tikzcd}
    A \arrow[r,"f"] \arrow[d,"g"] & B \arrow[d,"h"]\\
    C \arrow[r,"i"] & P
  \end{tikzcd}\end{equation}  
  Suppose $\beta:B \to X, \gamma:C\to X$ are such that $\beta\circ f = \gamma \circ h$. 
  $\beta,\gamma$ then induce maps on the generators of $P$. 
  These maps respect $F$ as $\beta\circ f=\gamma\circ h$, and they must respect $R_B,R_C$ as they are maps out of $B,C$. 
  Therefore, $\beta,\gamma$ induce a map $e:P\to X$, such that 
  $e(b) = \beta(b)$ for $b:G_B$ and $e(c)=\gamma(c)$ for $c:G_C$. 
  Furthermore, any map $P\to X$ with this property must agree with $e$ on all the generators of $P$, 
  and therefore equal $e$. Thus $e$ is the unique extension $P\to X$. 
  Thus $P$ the above square is indeed a pushout. 
\end{proof}
%\begin{lemma}\label{BooleCoEqualizers}
%  Countably presented Boolean algebras are closed under coequalizers.
%\end{lemma}
%\begin{proof}
%  Let $f,g:A\to B$ be Boolean morphisms.
%  Define $C = B/R$, where $R$ is given by the relations $fa-ga,~a\in G_A$, for $G_A$ the set of generators of $A$.
%  Suppose that we have a map $x:B\to D$ with $xf = gf$. Then $x$ respects $R$, and thus defines a map $y:C \to D$. 
%  Furthermore, any map $C\to D$ extending $x$ agrees with $y$ on the generators of $C$, 
%  and is thus equal to $y$. Therefore $C$ is the coequalizer of $f,g$. 
%\end{proof}


%%
%%\begin{corollary}\label{CoCompletenessBoole}
%%  The category of countably presented Boolean algebras contains all finite colimits. 
%%\end{corollary}
%%\begin{proof}
%%  Recall that $\Boole$ has an initial object given by $2$. 
%%  By \Cref{BoolePushouts}, 
%%%  it is therefore closed under coproducts. 
%%%  By \Cref{BooleCoEqualizers}, 
%%  it follows that $\Boole$ contains all finite colimits. 
%%\end{proof}

%Appendix%\section{Some notes on our axioms}
%Appendix%\label{NotesOnAxioms}
%\subsection{Alternatives to propositional completeness}
In \Cref{Axioms}, we have chosen to present propositional completeness as an axiom. 
However, assuming Stone duality, we could have made some other choices, 
and left propositional completeness as a theorem. 
What's more, assuming the axiom of Dependent choice,
the axiom is equivalent to LLPO. 
In this section, we will show these equivalences. 

\begin{theorem}\label{AlternativesToAxiom2}
  Assuming Stone duality, the following are equivalent:
  \begin{enumerate}[(i)]
    \item For $S$ Stone, we have $\neg \neg S \to ||S||$. 
    \item For $S$ Stone, we have that $||S||$ is closed. 
    \item A map $f:A \to B$ in $\Boole$ is injective iff the map $(\cdot) \circ f : Sp(B) \to Sp(A)$ is surjective. 
  \end{enumerate}
\end{theorem}
\begin{proof}
  We assume that $S= Sp(B)$. 
  Note the proof of \Cref{SpectrumEmptyIff01Equal} only uses Stone duality. 
  The proof of \Cref{BooleEqualityOpen} only relies on the definition of $\Boole$.
  Hence the argument in \Cref{TruncationStoneClosed}, which shows $(i)\to (ii)$ only relies on Stone duality. 
  Furthermore, the argument that closed propositions are double negation stable (\Cref{rmkOpenClosedNegation})
  only used \Cref{MarkovPrinciple}, which followed from Stone duality as well. 
  Hence if $||S||$ is closed, we have $\neg \neg ||S|| \leftrightarrow ||S||$, thus $(ii) \to (i)$. 
  $(i)\to (iii)$ is \Cref{FormalSurjectionsAreSurjections}. 
  By the above discussion, we also have that $\neg \neg S$ iff $0\neq_B 1$. 
  Note that $0\neq_B 1$ iff the map $2\to B$ is injective. 
  Furthermore, $||S||$ iff the map $S \to \top $ is surjective. 
  Hence $(iii) \to (i)$. 
\end{proof} 

\begin{lemma}\label{LLPOAndDCToCompleteness}
Assuming dependent choice, Stone duality, and that closed propositions are closed under disjunctions, 
we can show propositional completeness. 
\end{lemma}
\begin{proof}
  Let $B:\Boole$ satisfy $0\neq_B 1$. We will show there merely exists a map $B\to 2$. 
  Let $G$ be the set of generators of $B$. 
  We will use dependent choice on the the following $E_n,R_n$:
  \begin{itemize}
    \item 
  Let $E_n$ be the type consisting of 
  \begin{itemize}
    \item A map from the first $n$ generators of $B$ to $2$, denoted $x_n:G_n \to 2$. 
    \item A proposition denoting that $0\neq_{B_n} 1$ for $B_n$ given by:
      \begin{equation}
        B_n := B/\big( \{g|g\in G_n, x_n(g) = 0\} \cup \{ \neg g| g\in G_n, x_n(g) = 1\}\big).
      \end{equation}
  \end{itemize}
  \item 
    And let $R_n:E_n \to E_{n+1} \to \mathcal U$ denote the relation that $x_{n+1}$ extends $x_n$. 
  \end{itemize} 
  Note that $E_0$ is inhabited as $0\neq_B 1$. Assume $E_n$.
%  Now assume $x_n:G_n\to 2$ witnesses $E_n$. 
  As $0\neq_{B_n}1$, for all $g:B_n$, we can show 
%  we have $$\neg ((g =1)  \wedge ((\neg g) = 1)).$$
% % 
%%  Now suppose that $E_n$ is inhabited,  and let $x_n:G_n \to 2$. 
%%  Note that in $B_n$, we have $0\neq 1$ and thus $$\neg ((g =1)  \wedge ((\neg g) = 1))$$
%%  for all $g:B_n$.
%  Therefore, we have 
  $$\neg \neg (( g\neq 1) \vee ((\neg g) \neq 1)).$$
  By \Cref{BooleEqualityOpen}, and \Cref{rmkOpenClosedNegation}, 
  (which could be shown using Stone Duality)
  and the assumption that 
  closed statements are closed under disjunction, we have that the above statement is equivalent to 
  $(g \neq 1) \vee ((\neg g) \neq 1)$. 
  This holds in particular for $g$ the $n+1$'th generator of $B$. 
  Therefore, we have that $0\neq 1$ in $B_n/\{g\}$ or in $B_n/\{\neg g\}$. 
  Thus we can extend $x_n$ by letting $x_{n+1}(g) = 0$ or $x_{n+1}(g) = 1$ respectively. 
  
  By dependent choice, we get a map $x:G\to 2$. 
  We claim that for this map $x$, we have $0\neq 1$ in 
  \begin{equation}
    B' := B/\big( \{g|g\in G, x(g) = 0\} \cup \{ \neg g| g\in G, x(g) = 1\}\big).
  \end{equation}
  Note that $B'$ is the colimit of the sequence $B_n$ with projection maps $B_n \to B_{n+1}$. 
  Thus if $0=1$ in $B'$, $0=1$ in some $B_n$, which doesn't happen by assumption. 
  Therefore we have $0\neq 1$ in $B'$. 
  Furthermore, note that $B'$ is equivalent to a Boolean algebra with no generators, 
  as any generator in $B$ is sent to either $0$ or $1$ by the relations in $B'$. 
%
  But now any Boolean algebra with no generators and $0\neq 1$ is isomorphic to $2$. 
  Therefore $B'\simeq 2$, and the projection map $B\to B'$ gives a map $B \to 2$. 
  
\end{proof}

\begin{corollary}
Assuming dependent choice and Stone duality, TFAE:
\begin{enumerate}[(i)]
  \item For $S$ Stone, we have $\neg \neg S \to ||S||$. 
  \item LLPO.
  \item The disjunction of two closed propositions is closed. 
\end{enumerate}
\end{corollary}
\begin{proof}
  $(i) \to (ii)$ is \Cref{LLPO}, $(ii) \to (iii)$ is \Cref{ClosedFiniteDisjunction}, 
  and $(iii) \to (i)$ is \Cref{LLPOAndDCToCompleteness}
\end{proof}
\rednote{
  @Hugo, you mentioned that axiom 2 was independent from the other axioms. 
This might be a good place to reference to that proof}


\subsection{The formulation of local choice}

\begin{lemma}
  TFAE:
  \begin{itemize}
\item  Whenever $S$ Stone and $E\twoheadrightarrow S$ surjective, then there is some $T$ Stone,
    a surjection $T \twoheadrightarrow S$ and a map $T\to E$ 
    such that the following diagram commutes:
    \begin{equation}\begin{tikzcd}
      & E \arrow[d,""',two heads]\\
      T \arrow[ru,dashed]  \arrow[r,two heads, dashed ] &S %& \arrow[l, "", two heads, dashed] T\arrow[lu, ""',dashed ]
    \end{tikzcd}\end{equation}  
\item
  Whenever we have $S:\Stone$, $E,F$ arbitrary types, a map $f:S \to F$ and a 
  surjection $e:E \twoheadrightarrow F$, 
  there exists a Stone space $T$, a cover $T\twoheadrightarrow S$ and an arrow $T\to E$ making the following diagram commute:
    \begin{equation}\begin{tikzcd}
      T \arrow[d,dashed, two heads ] \arrow[r,dashed]&  E \arrow[d,""',two heads, "e"]\\
      S  \arrow[r, "f"] & F
    \end{tikzcd}\end{equation}  
\end{itemize} 
\end{lemma}
\begin{proof}
  By considering $f=id$ we can see that the second statement implies the first. 

  For the converse, let $S,E,F,e,f$ be as in the second statement. 
  As $e$ is surjective, whenever $s:S$, we there merely exists some $b:E$ with $e(b) = f(s)$. 
  This induces an element $(s,b):S\times_F E$. 
  Thus the projection $S\times_F E \to S$ is surjective, and 
  the first axiom provides us with a $T$ as required. 
    \begin{equation}\begin{tikzcd}
     & S \times_F E \arrow[r] \arrow[d,two heads] &  E \arrow[d,""',two heads, "e"]\\
       T \arrow[r, two heads ,dashed ] \arrow[ru,dashed]& 
       S  \arrow[r, "f"] & F
    \end{tikzcd}\end{equation}  
    
\end{proof}

\subsection{Scott continuity implies Stone duality}

%Appendix%\section{Countability}\label{CountabilityDiscussion}
In the system presented in this paper, 
one of the fundamental building blocks are countably presented Boolean algebras. 
There are several definitions of countable, which are not necissarily constructively equivalent. 

\begin{definition}
  A type $T$ is enumerable iff there exists a surjection $\N \to 1 + T$. 
\end{definition}
\begin{definition}\label{dfnFinite}
  A type $T$ is finite if there exists some $k:\N$ with $T\simeq Fin_k$. 
\end{definition}
\begin{definition}
  A type $T$ is strongly countable if
  $T$ is finite or merely isomorphic to $\N$.
\end{definition}
\begin{definition}
  A type $T$ is subcountable iff it is merely isomorphic to a decidable subset of $\N$. 
\end{definition}


\begin{lemma}
  Every strongly countable type is subcountable. 
\end{lemma}
\begin{proof}
  Note that $Fin_k$ and $\N$ are both isomorphic to a decidable subset of $\N$. 
\end{proof}
\begin{lemma}
  Every subcountable type is enumerable. 
\end{lemma}
\begin{proof}
  For $A\subseteq \N$ decidable, define $f:\N \to 1 + \Sigma_{n:\N} A(n)$ by 
  $$
  f(n) = 
  \begin{cases}
    inl(*) \text{ if } \neg A(n)\\
    inr(n) \text{ if } A(n)
  \end{cases}
  $$
\end{proof} 


\begin{lemma}\label{OpenSubsetNAreSubCountable}
  Any open subset of $\N$ is subcountable. 
\end{lemma} 
\begin{proof}
  Let $A:\N \to \Open$. 
  By countable choice, there exists a map $\alpha_{(\cdot)}:\N \to \Noo$ such that 
  $\exists_{m:\N} \alpha_n(m) = 0 \leftrightarrow A (n)$. 
  Define $B\subseteq \N \times \N$ by 
  $B(m,n) = (\alpha_{n}(m) = 0)$. 
  Note that $A(n) \leftrightarrow || \Sigma_{m:\N } B(m,n) ||$
\end{proof}

\begin{lemma}\label{OpenSubsetEnumerableAreEnumerable}
  Any open subset of an enumerable type is enumerable. 
\end{lemma}
\begin{proof}
  Let $A$ be enumerable and let $P:A \to Open$.
  We will show that $\Sigma_{a:A} P a$ is enumerable. 
  Let $s:\N \to 1 + A$ surjective. 
  Define $s':\N \to Open$ by 
  $$
  s'(n) = 
  \begin{cases}
    \bot \text { if } s(n) = inl(*)\\
    P(a) \text { if } s(n) = inr(a)
  \end{cases}
  $$ 
  By countable choice, we get a map 
  $\alpha_{(\cdot)}: \N \to 2^\N$  such that 
  $(\exists_{m:\N} \alpha_n(m) = 0) \leftrightarrow s'(n)$. 
  Note that $\alpha_n(m) = 1$ iff 
  $s'(n)$ which happens iff $s(n) = inr(a_n)$ for some $a_n:A$ with $P(a_n)$. 
  Therefore, we can define 
  $z:\N \times \N \to 1 + \Sigma_{a:A} P a$ by 
  \begin{equation}
    z(m,n) = 
    \begin{cases}
      inl(*) \text{ if } \alpha_{n}(m) = 0 \\
      a_n  \text{ if } \alpha_{n}(m) = 1 \text{ and $a_n$ as above}
    \end{cases}
  \end{equation}
  Note that if $a:A$ satisfies $P(a)$, there is some $n:\N$ with $s(n) = inr(a)$. 
  And as $P(a)$, there exists some $m:\N$ with $\alpha_n(m) = 1$. 
  Hence $z(m,n) = a$. 
  Thus $z$ is surjective. 
  As $\N \times \N \simeq \N$, we conclude that 
  $\sum_{a:A} P a$ is enumerable. 
\end{proof}

\begin{lemma}
  Any open subset of $\N$ is subcountable. 
\end{lemma}
\begin{proof}
  Let $A:\N \to \Open$. 
  By countable choice, we get a map $\alpha_{\cdot}: \N \to \Noo$ such that 
  $A(n) \leftrightarrow \Sigma_{m:\N} \alpha_n (m) = 1$. 
  Define $B:\N \times \N \to 2$ by $B(m,n) = \alpha_n(m)$. 
  We then have a bijection $\Sigma_{n:\N} A(n) \to \Sigma_{(n,m) : \N \times \N} B(m,n)$ sending 
  $(n,(m,p))$ to $(n,m,p)$.
\end{proof}



\begin{lemma}\label{OpenSubsetOfNNotDecidable}
  It is not the case that for every $P:\N \to \Open$, 
  $||\Sigma_{n:\N} P(n)||$ is decidable.
  % The subset being decidable could be interpreted as 
  %    that P(n) is decidable for all $n:\N$,
  % or that \Sigma_{n:\N} P(n) + \neg \Sigma_{n:\N} P(n) 
  % or that || \Sigma_{n:\N} P(n) || + \neg ||\Sigma_{n:\N} P(n) ||
\end{lemma}
\begin{proof}
  For $p$ any open proposition and $P(n) = p$ constantly, we have 
  $||\Sigma_{n:\N} P(n)||\leftrightarrow p$. 
  As not every open proposition is decidable (\Cref{rmkOpenClosedNegation}), 
  not every $||\Sigma_{n:\N} P(n)||$ is decidable. 
\end{proof}

\begin{lemma}\label{StronglyCountableTruncationDecidable}
  For every strongly countable type $A$, $||A||$ is decidable. 
\end{lemma}
\begin{proof}
  For a proposition, being decidable is a proposition. 
  Hence we may untruncate the definition of strongly open. 
  If $A \simeq \N$ or $A\simeq Fin_k$ for $k\neq 0$, we have $||A||$. 
  If $A \simeq Fin_0$, then $\neg ||A||$. 
\end{proof}

\begin{corollary}
  Not every enumerable type is strongly countable.
\end{corollary}
\begin{proof}
  If every enumerable type is strongly countable, 
  by \Cref{OpenSubsetEnumerableAreEnumerable}, every open subset of open subsets of $\N$ is strongly countable. 
  By \Cref{StronglyCountableTruncationDecidable}, the truncation of the corresponding type is decidable, which 
  contradicts\Cref{OpenSubsetOfNNotDecidable}.
\end{proof}

\begin{remark}
  Every enumerably represented Boolean algebra has an enumerable underlying set. 
  and every enumerable Boolean algebra is enumerably represented. 
\end{remark}

%Appendix%%\section{Representing Stone spaces as limits}
%Appendix%%\input{ColimitRepresentation.tex}
%Appendix%
%Appendix%\rednote{WIP}
\begin{lemma}
  Let $S:\Stone$ be represented as limit of a sequence of finite Stone spaces $S_n$. 
  Let $P:S \to \Closed$. 
  Then there merely exists a function $P_{\cdot} : \Pi_{n:\N} S_n \to 2$
  such that $P(s) \leftrightarrow \forall_{n:\N} P_n(s|_n)$. 
\end{lemma}  
\begin{proof}
  Assume $S=Sp(B)$, and $S_n = Sp(B_n)$. 
  Note that by Stone duality, $B$ is the colimit of the sequence $B_n$
  we denote $\iota_n:B_n\to B$ for the colimit inclusion maps.
  
  By \Cref{StoneClosedSubsets}, there merely exists some sequence $D_n:S\to 2$
  such that $P(s) \leftrightarrow \forall_{n:\N} D_n(s)$. 
  WLOG, we may assume $D_0 = \top$ and $D_{n+1}\subseteq D_n$ for all $n:\N$. 
  
  By Stone duality, each $D_n$ corresponds to some $b_n:B$. 
  Now each $b_n:B$ already occurs in some $B_{m}$.
  By countable choice, we can find a sequence of natural numbers $(m_n)_{n:\N}$, 
  and elements $b_{m_n}:B_{m_n}$ such that $\iota_{m_n}(b_{m_n}) = b_n$. 
  Then each $b_{m_n}:B_{m_n}$ corresponds to a decidable set 
  $D_{m_n}:S_{m_n} \to 2$. 
  WLOG we may assume that $(m_n)_{n:\N}$ is a strictly increasing sequence with $m_0 = 0$. 
  Then we can define a function $f:\N \to \N$ such that 
  for all $k:\N$, we have $m_{f(k)} \leq k \leq m_{f(k)+1}$. 
  % f 0       = 0, 
  % f (n + 1) | k  < f(n) + 1 = f(n)
  %           | k >= f(n) + 1 = f(n) + 1 

  Now define $P_k(s) = D_{m_{f(k)}}$.
  \rednote{Needs to be rewritten clearly}

  
\end{proof}

%Appendix%
%Appendix%\begin{lemma}
  \rednote{Countability and Boolean doesn't matter once you have the representations, 
  to get the representations with $v_n\circ u_n = 0$, you might need countable choice.}
  Let $A,B,C$ be countably presented Boolean algebras, represented by sequences $(A_n)_{n:\N}, (B_n)_{n:\N},(C_n)_{n:\N}$. 
  Let $u,v,(u_n)_{n:\N},(v_n)_{n:\N}$ be as in the following diagram of Boolean algebras:
  \begin{equation}
    \begin{tikzcd}
      A_\infty \arrow[r,"u_\infty"] & B_\infty  \arrow[r,"v_\infty"] & C_\infty
      \\
      A_n \arrow[u,"i_n^\infty"] \arrow[r,"u_n"] & B_n \arrow[u,"j_n^\infty"] \arrow[r,"v_n"] & C_n \arrow[u,"k^\infty_n"]
      \\
      A_m \arrow[u,"i_m^n"] \arrow[r,"u_m"] & B_m \arrow[u,"j_m^n"] \arrow[r,"v_m"] & C_m \arrow[u,"k_m^n"]
    \end{tikzcd} 
  \end{equation} 
  Furthermore, assume that $v\circ u = 0$ and $v_n \circ u_n = 0$ for all $n:\N$.
  Then $Ker(v)/Im(u)$ is the colimit of the sequence $Ker(v_n)/Im(u_n)$. 
\end{lemma}
\begin{proof}
  First, we will note what the maps in this sequence are, which by some abuse of notation also gives
  the cocone maps. 
\paragraph{If $n\leq m$, there are maps $Ker(v_n)/Im(u_n)\to Ker(v_m) / Im(u_m)$.}
Let $x,y\in B_n, a \in A_n$ be such that $x - y  = u_n(a)$, 
then $j_n^m(x) - j_n^m(y) = j_n^m(u_n(a)) = u_m(i_n^m(a))$.
Thus whenever $x,y\in B_n$ are such that $x \sim_{Im(u_n)} y$, we have that 
$j_n^m(x) \sim_{Im(u_m)} j_n^m(y)$. 
%
%
Furthermore, if $x\in Ker(v_n)$, then $v_n(x) = 0$, thus 
\begin{equation}
  v_m(j_n^m(x)) = k_n^m(v_n(x)) = k_n^m(0) = 0
\end{equation} 
and hence $j_n^m(x) \in Ker(v_m)$. 
Thus $j_n^m$ induces a map $\iota_n^m:Ker(v_n)/Im(u_n) \to Ker(v_m)/Im(u_m)$, 
with $\iota^n_m([x]) = [j_n^m(x)]$ for $x\in Ker(v_n)$. 

\paragraph{These maps are compatible (in particular, the maps $j_n^\infty$ form a cocone).}. 
If $k\leq n \leq m$, we have that $j_n^m \circ j_k^n = j_k^m$.
We thus have that $\iota_n^m \circ \iota_k^n = \iota_k^m$.

\paragraph{Given any cocone $\kappa_n : Ker(v_n)/Im(u_n)\to K$, 
  there exists an extension $\kappa_\infty(v_\infty)/Im(u_\infty)\to K$}
      If $\kappa_n$ forms a cocone, this means that for $n\leq m$ we have 
      $\kappa_m = \kappa_n \circ \iota_m^n$.

      We shall give a map $\kappa_\infty:Ker(v_\infty)/Im(u_\infty) \to K$ satisfying 
      $\kappa_\infty \circ \iota_n^\infty= \kappa_n$ for all $n:\N$.
      We're going to define a map $k:Ker(v_\infty) \to K$.
%%
      Let $x\in Ker(v_\infty)$. Then $x\in B_\infty$ and $v\infty(x) = 0$. 
      As $B_\infty$ is the colimit of the sequence $B_n$, 
      there is some $n:\N$ and some $x':B_n$ with $v_n(x') = 0$. 
      We'd like to define $k(x) = \kappa_n([x'])$. We need to check this definition doesn't depend on $n$. 
      \begin{itemize}
        \item \textbf{$k$ is well-defined} 
      Assume $n\leq m$ are such that we have $x':B_n, x'':B_m$ with $v_n(x') = 0, v_m(x'') = 0$ 
      and $j_n(x') = j_m(x'') = x$. 
      Then there exists some $l\geq m\geq n$ with 
      $j_n^l (x') = j_m^l(x'')$, hence 
      $$\iota_n^l[x'] = \iota_m^l[x'']$$ and thus 
      $$\kappa_l(\iota_n^l[x']) = \kappa_l(\iota_m^l[x''])$$
      But now $\kappa_l\circ \iota_n^l = \kappa_n$ and $\kappa_l\circ \iota_m^l = \kappa_m$, hence 
      $$\kappa_n([x']) = \kappa_m([x''])$$
      Thus $k$ is well-defined. 
      \item \textbf{$k$ respects $Im(u)$}
        Let $x,y:B$ and let $a:A$ be such that are such that $x-y = u(a)$ and $v(x) = v(y) = 0$.
        Then there is some $n:\N$, and some $x',y':B_n, a':A_n$ with $x'-y'= u_n(a'), v_n(x') = v_n(y') = 0$ 
        and $j_n(x') = x, j_n(y') = y, i_n(a') = a$. 
        Hence $[x'] = [y']$, hence $\kappa_n([x']) = \kappa_n([y'])$.
        Thus $k(x) = k(y)$. 
      \end{itemize}
      We conclude that $k$ induces a map $\kappa_\infty:Ker(v)/Im(u) \to K$. 
    \paragraph{$\kappa$ is colimiting}
    Let $\kappa_n$ be as above, and suppose that 
    $\lambda \circ \iota_n^\infty = \kappa_n$. 
    Then for all $n:\N$, $x':B_n$ such that $j_n(x) = x$, we have 
    $$\lambda ([x]) = \lambda \circ \iota_n([x']) = \kappa_n([x']) = k(x)$$
    As such $n,x'$ always exist, it follows that 
    $\lambda([x]) = k(x)$, hence $\lambda[x] = \kappa[x]$, so $\lambda = \kappa$ as required. 
\end{proof} 





\printbibliography

\end{document}
