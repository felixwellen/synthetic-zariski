% latexmk -pdf -pvc main.tex
\documentclass{../util/zariski}


\title{Synthetic Stone Duality}

\begin{document}

\author{Felix Cherubini, Thierry Coquand, Freek Geerligs and Hugo Moeneclaey}

\maketitle

%\begin{abstract}
%In synthetic algebraic geometry (SAG) \cite{draft}, we study finitely presented algebras over a commutative ring. 
%In this work, we study countably presented Boolean algebras instead. 
%Where the finitely presented algebras over a commutative ring induce a Zariski topos, 
%%the opposite category of these 
%the countably presented Boolean algebras induce the topos of light condensed sets \cite{Scholze}. 
%\cite{draft} proposes an axiomatization of the Zariski topos in univalent homotopy type theory \cite{hott}. 
%In this work, we propose similar axioms, which we expect to be modelled by light condensed sets. 
%% Furthermore, spectra of countably presented Boolean algebras correspond to quotients of Cantor space
%% which is cool because reasons
%\end{abstract} 
%
\rednote{The following is a collection of notes on work in progress.}

\rednote{I'm cleaning up, not all results that were in older versions have a place yet.}
\tableofcontents

% Logic and Topology
% - Building blocks, explaining the types of Stone, Boole (and what we mean with countable)
% - Rules, stating the axioms and first consequences, including the omniscience principles
% - Topology on propositions, mentioning under what constructions open propositions are closed
% - Examples of closed/open propositions, explaining why Boole is discrete and Stone Hausdorff
% - Topology on Stone spaces, including the classification of closed spaces.
% - Compact Hausdorff spaces. 
% Directed Univalence
% - Tychonov
% - Directed univalence, link to Phoa
% Cohomology
% - Cohomology and the interval
% Appendix, 
%  - alternative formulations of axiom 2
%  - more details on technical constructions
%  - colimit presentation of countably presented Boolean algebras (I'm not sure where we actually use this)
%  - scott continuity instead of axiom 1 



%\section*{Introduction}
%
Algebraic geometry is the study of solutions of polynomial equations using methods from geometry.
The central geometric objects in algebraic geometry are called \emph{schemes}.
Their basic building blocks are called \emph{affine schemes},
where, informally, an affine scheme corresponds to a solution sets of polynomial equations.
While this correspondence is clearly visible in the functorial approach to algebraic geometry and our synthetic approach,
it is somewhat obfuscated in the most commonly used, topological approach.

In recent years,
computer formalization of the intricate notion of affine schemes
received some attention as a benchmark problem
-- this is, however, \emph{not} a problem addressed by this work.
Instead, we use a synthetic approach to algebraic geometry,
very much alike to that of synthetic differential geometry.
This means, while a scheme in classical algebraic geometry is a complicated compound datum,
we work in a setting, based on homotopy type theory, where schemes are types,
with an additional property that can be defined within our synthetic theory.

Following ideas of Ingo Blechschmidt and Anders Kock  (\cite{ingo-thesis}, \cite{kock-sdg}[I.12]),
we use a base ring $\A$ which is local and satisfies an axiom reminiscent of the Kock-Lawvere axiom.
This more general axiom is called \emph{synthetic quasi coherence (SQC)} by Blechschmidt and
a version quantifying over external algebras is called the \emph{comprehensive axiom}\footnote{
  In \cite{kock-sdg}[I.12], Kock's ``axiom $2_k$'' could equivalently be Theorem 12.2,
  which is exactly our synthetic quasi coherence axiom, except that it only quantifies over external algebras.
}
by Kock.
The exact concise form of the SQC axiom we use, was noted by David Jaz Myers in 2018 and communicated to the first author.

Before we state the SQC axiom, let us take a step back and look at the basic objects of study in algebraic geometry,
solutions of polynomial equations.
Given a system of polynomial equations
\begin{align*}
  p_1(X_1, \dots, X_n) &= 0\rlap{,} \\
  \vdots\quad\quad\;\;   \\
  p_m(X_1, \dots, X_n) &= 0\rlap{,}
\end{align*}
the solution set
$\{\, x : \A^n \mid \forall i.\; p_i(x_1, \dots, x_n) = 0 \,\}$
is in canonical bijection to the set of $\A$-algebra homomorphisms
\[ \Hom_{\AAlg}(\A[X_1, \dots, X_n]/(p_1, \dots, p_m), \A) \]
by identifying a solution $(x_1,\dots,x_n)$ with the homomorphism that maps each $X_i$ to $x_i$.
Conversely, for any $\A$-algebra $A$ which is merely of the form $\A[X_1, \dots, X_n]/(p_1, \dots, p_m)$,
we define the \emph{spectrum} of $A$ to be
\[
  \Spec A \colonequiv \Hom_{\AAlg}(A, \A)
  \rlap{.}
\]
In contrast to classical, non-synthetic algebraic geometry,
where this set needs to be equipped with additional structure,
we postulate axioms that will ensure that $\Spec A$ has the expected geometric properties.
Namely, SQC is the statement that, for all finitely presented $\A$-algebras $A$, the canonical map
  \begin{align*}
    A&\xrightarrow{\sim} (\Spec A\to \A) \\
    a&\mapsto (\varphi\mapsto \varphi(a))
  \end{align*}
is an equivalence.
A prime example of a spectrum is $\A^1\colonequiv \Spec \A[X]$,
which turns out to be the underlying set of $\A$.
With the SQC axiom,
\emph{any} function $f:\A^1\to \A^1$ is given as a polynomial with coefficients in $\A$.
In fact, all functions between affine schemes are given by polynomials.
Furthermore, for any affine scheme $\Spec A$,
the axiom ensures that
the algebra $A$ can be reconstructed as the algebra of functions $\Spec A \to \A$,
therefore establishing a duality between affine schemes and algebras.

The Kock-Lawvere axiom used in synthetic differential geometry
might be stated as the SQC axiom restricted to (external) \emph{Weil-algebras},
whose spectra correspond to pointed infinitesimal spaces.
These spaces can be used in both synthetic differential and algebraic geometry
in very much the same way.

In the accompanying formalization \cite{formalization} of some basic results,
we use a setup which was already proposed by David Jaz Myers
in a conference talk (\cite{myers-talk1, myers-talk2}).
On top of Myers' ideas,
we were able to define schemes, develop some topological properties of schemes,
and construct projective space.

An important, not yet formalized result
is the construction of cohomology groups.
This is where the \emph{homotopy} type theory really comes to bear --
instead of the hopeless adaption of classical, non-constructive definitions of cohomology,
we make use of higher types,
for example the $n$-th Eilenberg-MacLane space $K(\A,n)$ of the group $(\A,+)$.
As an analogue of classical cohomology with values in the structure sheaf,
we then define cohomology with coefficients in the base ring as:
\[
  H^n(X,\A):\equiv \propTrunc{X\to K(\A,n)}_0
  \rlap{.}
\]
This definition is very convenient for proving abstract properties of cohomology.
For concrete calculations we make use of another axiom,
which we call \emph{Zariski-local choice}.
While this axiom was conceived of for exactly these kind of calculations,
it turned out to settle numerous questions with no apparent connection to cohomology.
One example is the equivalence of two notions of \emph{open subspace}.
A pointwise definition of openness was suggested to us by Ingo Blechschmidt and
is very convenient to work with.
However, classically, basic open subsets of an affine scheme are given
by functions on the scheme and the corresponding open is morally the collection of points where the function does not vanish.
With Zariski-local choice, we were able to show that these notions of openness agree in our setup.

Apart from SQC, locality of the base ring $\A$ and Zariski-local choice,
we only use homotopy type theory, including univalent universes, truncations and some very basic higher inductive types.
Roughly, Zariski-local choice states that any surjection into an affine scheme merely has sections on a \emph{Zariski}-cover.%
\footnote{It is related to the set-theoretic axiom called
\emph{axiom of multiple choice} (AMC) \cite{vandenberg-moerdijk-amc} or \emph{weakly initial set of covers axiom} (WISC):
the set of all Zariski-covers of an affine scheme is weakly initial among all covers.
However, our axiom only applies to (affine) schemes, not all types or sets.}
The latter, internal, notion of cover corresponds quite directly to the covers in the site of the \emph{Zariski topos},
which we use to construct a model of homotopy type theory with our axioms.

More precisely, we can use the \emph{Zariski topos} over any base ring.
Toposes built using other Grothendieck topologies, like for example the étale topology, are not compatible with Zariski-local choice.
We did not explore whether an analogous setup can be used for derived algebraic geometry%
\footnote{Here, the word ``derived'' refers to the rings the algebraic geometry is built up from --
  instead of the 0-truncated rings we use, ``derived'' algebraic geometry would use simplicial or spectral rings.
  Sometimes, ``derived'' refers to homotopy types appearing in ``the other direction'', namely as the values of the sheaves that are used.
  In that direction, our theory is already derived, since we use homotopy type theory.
  Practically that means that we expect no problems when expanding our theory of synthetic schemes to what classic algebraic geometers
  call ``stacks''.
}
-- meaning that the 0-truncated rings we used are replaced by higher rings.
This is only because for a derived approach, we would have to work with higher monoids, which is currently infeasible
-- we are not aware of any obstructions for, say, an SQC axiom holding in derived algebraic geometry.

In total, the scope of our theory so far includes quasi-compact, quasi-separated schemes of finite presentation over an arbitrary ring.
These are all finiteness assumptions, that were chosen for convenience and include examples like closed subspaces of projective space,
which we want to study in future work, as example applications.
So far, we know that basic internal constructions, like affine schemes, correspond to the correct classical external constructions.
This can be expanded using our model, which is of course also important to ensure the consistency of our setup.



\section{Stone duality}
\subsection{Preliminaries}

In this section, we introduce the type of countably presented Boolean algebras $\Boole$ and of Stone spaces $\Stone$. 
Both of these types carry a natural category structure. 
In later sections, we will axiomatize an anti-equivalence between these categories, 
which is classically valid and called Stone duality. 

%\subsection{Countably Presented Boolean Algebras}
%We will use the type of countably presented (c.p.) boolean algebras $\Boole$,
%for more definitions and notation see \Cref{A-cp-boolean-algebras}.

%\subsection{Countably presented Boolean algebras}
\begin{definition}
  A countably presented Boolean algebra $B$ is a Boolean algebra such that there merely are 
  countable sets $I,J$, 
  a set of generators $g_i,~{i\in I}$ and a set $f_j,~{j\in J}$ of Boolean expressions over these generators 
  such that $B$ is equivalent to the quotient of the free Boolean algebra over the generators by the relations
  $f_j=0$. We denote this algebra by $2[I]/(f_j)_{j:J}$.
\end{definition} 

\begin{remark}
By countable in the previous definition we mean sets that are merely equal to a decidable in $\N$. Note that any countably presented algebra is also merely of the form $2[\N]\rangle / (r_n)_{n:\N}$, if we add dummy variables that we equate to $0$, and dummy relations that equate $0$ to itself.
\end{remark}

We will call the family $(f_j)_{j\in J}$ as above a set of relations. 
If $I,J$ are finite, we call $B$ a finitely presented Boolean algebra. 
Once we have postulated the axiom of dependent choice, 
in \Cref{secBooleAsColimits}
we will be able to show that every countably presented algebra 
is actually a colimit of a sequence of finitely presented Boolean algebras.
They are therefore dual to pro-finite objects, which are used 
in the theory of light condensed sets \cite{Scholze,Dagur,TODO}.

\begin{remark}
  We denote the type of countably presented Boolean algebras $\Boole$. 
  Note that this type does not depend on a choice of universe. 
\end{remark}

\begin{example}
  If both the set of generators and relations are empty, we have the Boolean algebra $2$.
  We have $0\neq_2 1$, and the underlying set of $2$ is given by $\{0,1\}$.
\end{example}
Note that any Boolean algebra must contain the elements $0,1$. 
Therefore, $2$ is the initial Boolean algebra. 
We can therefore use it to define points of objects in the category dual to that of countably presented Boolean algebras. 

%\subsection{Stone spaces}
\begin{definition}
  For $B$ a countably presented Boolean algebra, we define $Sp(B)$ as the set of Boolean morphisms from $B$ to $2$. 
\end{definition}
\begin{definition}
  We define the predicate on types $\isSt$ by 
  \begin{equation}
    \isSt(X) := \sum\limits_{B : Boole} X = Sp(B)
  \end{equation} 
  A type $X$ is called \textit{Stone} if $\isSt(X)$ is inhabited.
\end{definition}


%\subsection{Examples}
\begin{example}
  \label{boolean-algebra-examples}
  \begin{enumerate}[(i)]
  \item There is only one Boolean map $2\to 2$, thus $Sp(2)$ is the singleton type $\top$. 
  \item   The trivial Boolean algebra is given by the empty set of generators and the relation $\{1\}$.
    We have $0=1$ in the trivial Boolean algebra. 
    As there cannot be a map from the trivial Boolean algebra into $2$ preserving both $0,1$, 
    the corresponding Stone space is the empty type $\bot$, 
  \item\label{ExampleBAunderCantor}   
    We denote by $C$ the Boolean algebra $2[\N]$ given by $\N$ as a set of generators and no relations. We write $p_n$ for the generator corresponding to $n$.
    A morphism $C\to 2$ corresponds to a function $\mathbb N\to 2$, which is a binary sequence. 
    The Stone space $Sp(C)$ of these binary sequences is denoted $2^{\N}$ and called \notion{Cantor space}.
  \item\label{ExampleBAunderNinfty}
    We denote by $B_\infty$ the quotient of $C$ by the relations $p_m\wedge p_n$ for $m\neq n$. 
    A morphism $B_\infty\to 2$ corresponds to a function $\mathbb N \to 2$ that hits $1$ at most once. 
    The corresponding Stone space is denoted by $\N_\infty$. 
  \end{enumerate}
\end{example}

\begin{remark}\label{BinftyTermsWriting}
  In \Cref{N-co-fin-cp}, we will show that $B_\infty$ is equivalent to the Boolean algebra on 
  subsets of $\N$ which are finite or co-finite. 
  Under this equivalence, the generator $p_n$ is sent to the singleton $\{n\}$. 
  Because of this, we have that any $b:B_\infty$ can be written 
  either as $\bigvee_{i\in I_0} p_i$ or as $\bigwedge_{i\in I_0} \neg p_i$ for some finite $I_0\subseteq \N$. 
\end{remark}


%\begin{remark}
%  As Boolean algebras are rings, any relation of the form $f=g$ with both $f,g$ Boolean expressions 
%  can be written as $h=0$ with $h=f-g$ a Boolean expression. 
%\end{remark} 




\input{Rules/Axioms.tex}
\subsection{Anti-equivalence of $\Boole$ and $\Stone$}

\begin{remark}\label{SpIsAntiEquivalence}
Stone types will take over the role of affine scheme from \cite{draft}, 
and we repeat some results here. 
Analogously to Lemma 3.1.2 of \cite{draft}, 
for $X$ Stone, Stone duality tells us that $X = Sp(2^X)$. 
%
Proposition 2.2.1 of \cite{draft} now says that 
$Sp$ gives a natural equivalence 
\begin{equation}
   Hom_{\Boole} (A, B) = (Sp(B) \to Sp(A))
\end{equation}
Therefore $Sp$ is an embedding from $\Boole$ to any universe of types, and $\isSt$ is a proposition.

Its image, $\Stone$ also has a natural category structure.
By the above and Lemma 9.4.5 of \cite{hott}, the map $Sp$ defines a dual equivalence of categories between $\Boole$ and $\Stone$.
\end{remark}

\begin{lemma}\label{SpectrumEmptyIff01Equal}
  For $B:\Boole$, we have $0=_B1$ iff $\neg Sp(B)$.
\end{lemma}
\begin{proof}
  Note that whenever $0=1$ in $B$, there is no map $B\to 2$ respecting both $0$ and $1$ as $0\neq 1$ in $2$. 
  Thus $\neg Sp(B)$ whenever $0=1$ in $B$. 
  % 
  Conversely, if $\neg Sp(B)$, then $Sp(B) = \emptyset$, which is also the spectrum of the trivial Boolean algebra. 
  As $Sp$ is an embedding, $B$ is equivalent to the trivial Boolean algebra, and $0=_B1$. 
\end{proof}

%\begin{corollary}\label{MoreConcreteCompleteness}
%  By the above and propositional completeness, we have that $||Sp(B)||$ iff $0\neq_B1$. 
%\end{corollary}


\begin{remark}\label{StoneClosedUnderPullback}
%  By \Cref {SpIsAntiEquivalence} and the fact that that countably presented Boolean algebras form a 
%  finitely cocomplete category (\Cref{CoCompletenessBoole}), the category of Stone spaces is complete. 
  By \Cref {SpIsAntiEquivalence} and the fact that that countably presented Boolean algebras are closed under pushouts, 
  the category of Stone spaces is closed under pullbacks. 
\end{remark}

We conclude this section on the anti-equivalence of Stone and $\Boole$ by a relating surjections to injections. 
This theorem is actually equivalent to completeness of propositional logic, which we'll discuss in 
\Cref{NotesOnAxioms}. 

\begin{theorem}\label{FormalSurjectionsAreSurjections}
  Let $f:A\to B$ be a map of countably presented Boolean algebras. 
  If $f$ is injective, then the corresponding map $(\cdot) \circ f : Sp(B) \to Sp(A)$ is surjective. 
\end{theorem}

\begin{proof}
  Assume $f$ injective and let $x:Sp(A)$.
  By \Cref{FiberConstruction}, we have that $\left(\sum\limits_{y:Sp(B)} y\circ f = x \right) = Sp(B/R) $
  for $R=f(G)$ for some countable $G\subseteq A$ with $x(g) = 0$ for all $g\in G$. 
  By propositional completeness and \Cref{SpectrumEmptyIff01Equal}, 
  it's sufficient to show that $0\neq_{B/R}1$. 
  Note that $0=_{B/R} 1$ iff 
  $1 =_B \bigvee R_0$ for some $R_0\subseteq R$ finite. 
  But then $$1 = \bigvee f(G_0) = f(\bigvee  G_0)$$ for some $G_0\subseteq G$ finite. 
  And as $f$ is injective, $\bigvee G_0 = 1$. 
  However, 
  $$
  x(\bigvee G_0) = 
  x(\bigvee_{g\in G_0} g ) = \bigvee_{g \in G_0} x(g) = \bigvee_{g\in G_0} 0 = 0$$
  And as $x(1) = 1$, we get a contradiction. Therefore $0\neq_{B/R} 1$ as required. 
\end{proof}  
The converse to the above theorem is true as well, regardless of propositional completeness:
\begin{lemma}\label{SurjectionsAreFormalSurjections}
If $f:A\to B$ is a map in $\Boole$ and $(\cdot) \circ f :Sp(B) \to Sp(A)$ is surjective, 
$f$ is injective. 
\end{lemma}
\begin{proof}
  Suppose precomposition with $f$ is surjective. 
  Let $a:A$ be such that $f(a)= 0$. 
  By assumption, for every $x:A\to 2$, there is a $y:B\to 2$ with $y\circ f = x$. 
  Consequentely $x(a) = y(f(a)) = y(0) = 0$. 
  So $x(a) = 0$ for every $x:Sp(A)$. 
  Thus $Sp(A) = Sp(A/\{a\})$, and as $Sp$ is an embedding, 
  $A \simeq A/\{a\}$, and $a = 0$ in $A$. 
  So whenever $f(a) = 0$, we have $a=0$. Thus $f$ is injective. 
\end{proof}

\subsection{Principles of omniscience}
In constructive mathematics, we do not assume the law of excluded middle (LEM).
There are some principles called principles of omniscience that are weaker than LEM, which can be used to describe 
how close a logical system is to satisfying LEM.
In this section, we will show that two of them (Markov's principle and LLPO) hold, 
and one (WLPO) fails in our system.

\begin{theorem}[The negation of the weak lesser principle of omniscience ($\neg$WLPO)]\label{NotWLPO}
  It is not the case that the statement %There is no method which given $\alpha:2^\mathbb N$ decides whether 
  $\forall_{n:\mathbb N} \alpha(n) = 0$ is decidable for general $\alpha:2^\mathbb N$. 
\end{theorem}
\begin{proof}
  Such a decision method would give a function $f:2^\mathbb N \to 2$ such that 
  $f(\alpha) = 0$ iff $\forall_{n:\mathbb N} \alpha (n)= 0$. 
  By Stone duality, there must be some $c:C$ with 
  $f(\alpha) = 0 \iff \alpha(c) = 0$. 
  $c$ is expressable using only finitely many generators $(p_n)_{n\leq N}$. 
  Now consider $\beta,\gamma:C \to 2$ given by $\beta(p_n) = 0$ for all $n:\mathbb N$ and
  $\gamma(p_n) = 0$ iff $n\leq N$. 
  Note that these functions are equal on $(p_n)_{n\leq N}$, therefore, $\beta(c) = \gamma(c)$. 
  However, $f(\beta) = 0$ and $f(\gamma) = 1$.
  We thus have a contradiction, thus a decision method as required doesn't exist. 
\end{proof}

The following result is due to David W\"arn:
\begin{theorem}[Markov's principle (MP)]\label{MarkovPrinciple}
  For $\alpha:\Noo$, we have that 
  \begin{equation}
    (\neg (\forall_{n:\mathbb N} \alpha (n)= 0)) \to \Sigma_{n:\mathbb N} \alpha (n)= 1
  \end{equation}
\end{theorem}
\begin{proof}
  Assume $\neg (\forall_{n:\mathbb N} \alpha (n)= 0)$.
%  Suppose $\alpha \neq \infty$.%, then it is not the case that $\alpha(n) = 0$ for all $n:\N$. 
  It is sufficient to show that $2/\{\alpha(n)|n\in\N\}$ is the trivial Boolean algebra. 
  It will then follow that there is a finite subset $N_0\subseteq \N$ 
  with $\bigvee_{i:N_0} \alpha(i) = 1$.
  As $\alpha(i) \in \{0,1\}$ and $\alpha(i) = 1$ for at most one $i$, it then follows that 
  there exists an unique $n\in\mathbb N$ with $\alpha(n) = 1$. 

  To show that $2/\{\alpha(n)|n\in\N\}$ is trivial, we will show it has an empty spectrum. 
  Suppose $x: 2 \to 2$ is such that $x(\alpha(n)) = 0$ for every $n:\N$. 
  As $x(1) = 1$, we must have for every $n:\N$ that $\alpha(n) \neq 1$. 
  But then $\alpha(n) = 0$, contradicting our assumption. 
  We get a contradicition and there thus there are no points in the spectrum of $2/\{\alpha(n)|n\in\N\}$ as required. 
\end{proof}

\begin{corollary}
  For $\alpha:2^\mathbb N$, we have that 
  \begin{equation}
    (\neg (\forall_{n:\mathbb N} \alpha (n)= 0)) \to \Sigma_{n:\mathbb N} \alpha (n)= 1
  \end{equation}
\end{corollary}
\begin{proof}
  Given $\alpha:2^\mathbb N$, consider the sequence $\alpha':\Noo$ satisfying $\alpha'(n) = 1$ iff 
  $n$ is minimal with $\alpha(n) = 1$. Then apply the above theorem.
\end{proof}

\begin{theorem}[The lesser limited principle of omniscience (LLPO)]\label{LLPO}
  For $\alpha:\N_\infty$, 
  we have that 
  \begin{equation}\label{eqnLLPO}
    \forall_{k:\N} \alpha(2k) = 0  \vee \forall_{k:\N} \alpha(2k+1) = 0
  \end{equation}
\end{theorem}
\begin{proof}
%
%  We first will define a map $f:B_\infty \to B_\infty \times B_\infty$. 
%  Because of \Cref{rmkMorphismsOutOfQuotient}, it is sufficient to define $f$ on $(p_n)_{n:\N}$ with 
%  $f(p_n) \wedge f(p_m) = (0,0)$ for $n\neq m$. 
%  To define $f(p_n)$, we use a case distinction on whether $n$ is odd or even. 
  Define $f:B_\infty \to B_\infty \times B_\infty$ as follows:
  \begin{equation}
    f(p_n) =\begin{cases}
      (p_k,0) \text{ if } n = 2k\\
      (0,p_k) \text{ if } n = 2k+1\\
    \end{cases}
  \end{equation}
  To see that $f$ is well-defined, we can make a case distinction on parity. 
%  By making a case distinction on $n,m$ being odd or even, 
%  we can see that 
%  $f(p_n) \wedge f(p_m) = (0,0)$ when $n\neq m$, thus $f$ is well-defined. 
  We claim $f$ is injective. Assume $f(x) = 0$, 
  to see that $x=0$, we make a case distinction on whether $x$ corresponds to a finite or a cofinite set 
  (\Cref{BinftyTermsWriting}).
%
%  We also claim it is injective.
%  Now let $x:B_\infty$ with $f(x) = 0$. 
  We denote $E,O\subseteq \N$ for the even and odd numbers respectively. 
%  and we make a case distincition based on \Cref{BinftyTermsWriting}.
  \begin{itemize}
    \item Suppose 
      $x = \bigvee_{i\in I_0} p_i$ with $I_0$ finite. 
      Then 
      $$f(x) = (\bigvee_{i\in I_0 \cap E } p_{\frac i2} , \bigvee_{i\in I_0 \cap O } p_{\frac {i-1}2} ) = (0,0)$$
      As $p_j\neq 0$ for all $j\in\N$, we must have $I_0 \cap E = \emptyset = I_0 \cap O$. 
      Thus $I_0= \emptyset$, and $x = 0$. 
    \item Suppose 
%      Let $x$ correspond to a cofinite subset of $\N$. Write 
      $x = \bigwedge_{j\in J} \neg p_j$ with $J$ finite. % for $J$ finite. 
      We will derive a contradiction. %, from which we can conclude that $x=0$ after all. 
      Note that   
      $$f(x) = (\bigwedge_{j\in J \cap E } \neg p_j , \bigwedge_{j\in J \cap O } \neg p_j ) = (0,0)$$
%      As $f(x) = (0,0)$, we have that 
%      $\bigwedge_{j\in J \cap E } \neg p_j =0$ and
%      $\bigwedge_{j\in J \cap O } \neg p_j  = 0$.
      However, any finite meet of negations corresponds to a cofinite set, hence is nonzero. 
      We get a contradiction and conclude $x=0$. 
%      However, any finite meet of negations will correspond to a cofinite set,
%      in particular it will not correspond to the empty set, and thus not be $0$.
%      Thus $f(x)\neq 0$, contradicting the assumption that $f(x) = 0$, hence $x=0$ ex falso. 
  \end{itemize}
%  In both cases, we conclude $x=0$, thus $f$ is injective. 
  By \Cref{FormalSurjectionsAreSurjections}, $f$ corresponds to a surjection 
  $s:\Noo + \Noo \to \Noo$.
  Thus for $\alpha : \Noo$, 
  there exists some $x:\Noo + \Noo$ such that $s x = \alpha$. 
  If $x = inl(\beta)$, 
  for any $k:\N$, we have that 
  $$\alpha (p_{2k+1}) = s(x) (p_{2k+1}) = x(f(p_{2k+1})) = inl(\beta) (0,p_k)  = \beta(0) = 0.$$
  Similarly, if $x = inr(\beta)$, we have $\alpha(p_{2k}) = 0$ for all $k:\N$. 
  Thus \Cref{eqnLLPO} holds for $\alpha$ as required. 
\end{proof}

The use of \Cref{FormalSurjectionsAreSurjections}, and hence of propositional completeness, 
was helpful in the above proof, as the following shows:
\begin{lemma}
  The above function $f$ does not have a retraction. 
\end{lemma}
\begin{proof}
  Suppose $r:B_\infty \times B_\infty \to B_\infty$ is a retraction of $f$. 
  Note that $r(0,1):B_\infty$ is expressable using only finitely many generators $(p_n)_{n\leq N}$
  Note that $r(0,1) \geq r(0,p_k) = p_{2k+1}$ for all $k:\N$. 
  As a consequence, $r(0,1)$ cannot be of the form $\bigvee_{i\in I_0} p_i$, and by \Cref{BinftyTermsWriting}, 
  $r(0,1)$ corresponds to a cofinite subset of $\N$. % = \bigwedge_{i:I_0} \neg p_i$, where $i\leq N$ for $i\in I_0$. 
  By similar reasoning so does $r(1,0)$.% corresponds to a cofinite subset of $\N$. 
  But the intersection of cofinite subsets is cofinite, while 
  $$r(0,1) \wedge r(1,0) = r( (1,0) \wedge (0,1)) = r(0,0) = 0$$
  which gives a contradiction. Thus no retraction exists. 
\end{proof}

We finish with an equivalent formulation of LLPO:

\begin{lemma}\label{corAlternativeLLPO}
  Let $(\phi_n)_{n:\N}, (\psi_m)_{m:\N}$ be families of decidable propositions indexed over $\N$.
  We then have 
  \begin{equation}
    (\forall_{n:\N} \forall_{m:\N} (\phi_n \vee \psi_m) )
    \leftrightarrow
    ((\forall_{n:\N} \phi_n) \vee (\forall_{m:\N} \psi_m) )
  \end{equation}
\end{lemma}

\begin{proof}
  Note that the implication from right to left in the above equation always holds.
  Assume that for all $m,n:\mathbb N$ we have $\phi_n\vee \psi_m$ 
  Consider the sequence $\alpha:2^\mathbb N$ where $\alpha(2n) = 0$ iff $\phi_n$ and 
  $\alpha(2m+1) = 0$ iff $\psi_m$. 
  Let $\beta:\Noo$ be such that $\beta(i) = 1$ iff $i$ is minimal with $\alpha(i) = 1$
  By LLPO, we have that 
  $\beta$ is $0$ on all odd entries or on all even entries. 
  Suppose that $\beta$ hits $0$ on all odd entries. 
  We will show $\psi_m$ for all $m:\N$. 
  As $\beta(2m+1) = 0$, there are two options:
  \begin{itemize}
    	\item If $\alpha(l)=0$ for all $l\leq 2m+1$. Then in particular $\alpha(2m+1)=0$ and $\psi_m$ holds.
	\item Otherwise there is some $l<2m+1$ with $\beta(l) = 1$. 
  As $\beta$ hits $0$ on odd entries, $l$ is even. 
  So $\alpha(2n) = 1$ for $n = \frac{l}2$, meaning that $\neg \phi_n$. 
  By assumption, $\phi_n \vee \psi_m$ holds, hence $\psi_m$ must hold. 
  Thus for all $m:\N$, we have $\psi_m$ if $\beta$ hits $0$ on all odd entries. 
  By a symmetric argument, if $\beta$ hits $0$ on all even entries, we have $\phi_n$ for all $n:\N$. 
  We conclude that 
  $((\forall_{n:\N} \phi_n) \vee (\forall_{m:\N} \psi_m) )$ 
  as required. 
  \end{itemize}
\end{proof}

\begin{remark}
Note that the above statement implies LLPO as $\alpha(2n) =0 \vee \alpha(2m+1) =0$ for all $n,m:\mathbb N$ if $\alpha:\Noo$. 
\end{remark}


\section{Topology of Stone spaces}
In this section, we will define the types of open and closed propositions. 
These will allow us to define a (synthetic) topology  \cite{SyntheticTopologyLesnik} on any type.
We will study this topology on Stone types in particular.

\subsection{Open and closed propositions}
In this section we will introduce a topology on the type of propositions, and 
study their logical properties.
We think of open and closed propositions respectively as countable disjunctions and conjunctions of decidable propositions.
Such a definition is universe-independent, and can be made internally.

\begin{definition}
  We define the types $\Open, \Closed$ of open and closed propositions as follows:
  \begin{itemize}
    \item 
    A proposition $P$ is open iff there merely exists some $\alpha:2^\N$ such that 
      $P \leftrightarrow \exists_{n:\mathbb N} \alpha(n) = 0$. 
    \item 
    A proposition $P$ is closed iff there merely exists some $\alpha:2^\N$ such that 
      $P \leftrightarrow \forall_{n:\mathbb N} \alpha(n) = 0$. 
  \end{itemize}
\end{definition}

\begin{remark}\label{rmkOpenClosedNegation}
  The negation of an open proposition is closed, 
  and by Markov's principle (\Cref{MarkovPrinciple}), the negation of a closed proposition is open. 
  Also by Markov's principle, we have $\neg \neg P \to P$ whenever $P$ is open or closed. 
  By the negation of WLPO (\Cref{NotWLPO}), 
  not every closed proposition is decidable. 
  Therefore, not every open proposition is decidable. 
  % Both therefore and similarly can be used here, by a similar proof we can show it, or we can use that 
  % if $P$ is closed and $\neg P$ is decidable, so is $\neg \neg P = P$. 
  Every decidable proposition is both open and closed, 
  and in \Cref{ClopenDecidable} we shall see the converse. 
\end{remark}

\begin{lemma}\label{ClosedCountableConjunction}
  Closed propositions are closed under countable conjunctions.
\end{lemma}
\begin{proof}
  Let $(P_n)_{n:\N}$ be a countable family of closed propositions. 
  By countable choice, for each 
  $n:\N$ we have an $\alpha_n:2^\N $ 
  such that $P_n \leftrightarrow \forall_{m:\N} \alpha_n(m)  =0$. 
  Consider a surjection $s:\N \to \N \times \N$.
  Let 
  $$\beta(k) = \alpha_{\pi_0(s(k))}(\pi_1 (s(k))).$$
  Note that $\forall_{k:\N} \beta(k) = 0$ iff 
  $\forall_{m,n:\N}\alpha_m(n) = 0$, which happens iff $\forall_{n:\N} P_n$. 
  Hence the countable conjunction of closed propositions is closed. 
\end{proof} 

\begin{lemma}\label{OpenCountableDisjunction}
  Open propositions are closed under countable disjunctions. 
\end{lemma}
\begin{proof}
  Similar to the previous lemma. 
\end{proof}

\begin{corollary}\label{ClopenDecidable}
  If a proposition is both open and closed, it is decidable. 
\end{corollary}
\begin{proof}
  If $P$ is open and closed, $P\vee \neg P$ is open, hence
  equivalent to $\neg \neg (P \vee \neg P)$, which is provable. 
\end{proof}

\begin{lemma}\label{ClosedFiniteDisjunction} 
  Closed propositions are closed under finite disjunctions. 
\end{lemma}
\begin{proof}
  We shall show that 
  $(\forall_{n:\N} \alpha(n) = 0 )\vee (\forall_{n:\N} \beta(n) = 0 )$ is closed for any $\alpha,\beta:2^\N$.
  By \Cref{corAlternativeLLPO}, the statement is equivalent to 
  $ \forall_{n:\N}  \forall_{m:\N}  (\alpha(n) = 0 \vee \beta(m) = 0)$, 
  which is a countable conjunction of decidable propositions, 
  hence closed by \Cref{ClosedCountableConjunction}.
\end{proof}
\begin{lemma}\label{OpenFiniteConjunction}
  Open propositions are closed under finite conjunctions. 
\end{lemma}
\begin{proof}
  We need to show that for any $\alpha,\beta:2^\N$, the following proposition is open:
  \begin{equation}\label{eqnConjunctionOpen}
    (\exists_{n:\N} \alpha(n) = 0 )\wedge(\exists_{n:\N} \beta(n) = 0 )
  \end{equation}
  Consider $\gamma:2^\N$ given by 
  $\gamma(l) = 1$ iff there exist some $k,k'\leq l$ with 
  $\alpha(k) = \beta(k') = 0$. 
  As we only need to check finitely many combinations 
  of $k,k'$, this is a decidable property for each $l:\N$ and $\gamma$ is well-defined. 
  Then $\exists_{k:\N}\gamma(k)=0$ if and only if \Cref{eqnConjunctionOpen} holds.
\end{proof}

\begin{lemma}\label{OpenDependentSums}
  Open propositions are closed under dependent sums.
\end{lemma}
\begin{proof}
  First note that for $D$ a decidable proposition, and $X:D \to \Open$,
  by case splitting on $D$, we can see 
  $\Sigma_{d:D} X(d)$ is open.
%
  Then note that for $P$ an open proposition, 
  there exists a sequence of decidable propositions $A_n$ with 
  $P = \exists_{n:\N} A_n $.
%
  So for $Y : P \to Open $, the dependent sum $\Sigma_P Y$ is given by 
  $\exists_{n:\N} (\Sigma_{a:A_n} Y(n,a))$. 
  which is a countable a countable disjunction of open propositions, 
  hence open by \Cref{OpenCountableDisjunction}.
\end{proof}

We will see the same holds for closed propositions in \Cref{ClosedDependentSums}.

\begin{remark}\label{ImplicationOpenClosed}
  If $P$ is open, $P \to \bot$ is only open if $P$ is decidable, which is not in general the case. 
  Thus $\Open$ is not closed under dependent products. Neither is $\Closed$. 
  However, as $(P\to Q)  \to \neg \neg (\neg P \vee Q)$,
  we have that if $P$ is open and $Q$ is closed, then $P\to Q$ is closed, and similarly $Q\to P$ is open.
\end{remark}

\subsection{Equality in $\Boole$ and $\Stone$}
\begin{lemma}\label{BooleEqualityOpen}
  Whenever $B:\Boole$, $a,b:B$ the proposition $a=_Bb$ is open. 
\end{lemma}
\begin{proof}
  Let $G,R$ be the generators and relations of $B$. 
  Let $a,b$ be represented by $x,y$ in the free Boolean algebra on $G$. 
  Now let $R_n$ denote the first $n$ elements of $R$. 
  Note that $a=b$ iff there exists some $n:\N$ with $x-y \leq \bigvee_{r\in R_n} r$. 
  Furthermore, inequality is decidable in the free Boolean algebra, hence
  $a=b$ is a countable disjunction of decidable propositions, hence open. 
\end{proof}


\begin{corollary}\label{TruncationStoneClosed}
  Whenever $S:\Stone$, $||S||$ is closed. 
\end{corollary}
\begin{proof}
  By \Cref{SpectrumEmptyIff01Equal}, $\neg S$ is equivalent to $0=_B 1$, which is open by the above. 
  Hence $\neg \neg S$ is a closed proposition, and by propositional completeness, so is $||S||$. 
\end{proof}

\begin{remark}\label{ExplicitTruncationStoneClosed}
  \rednote{New check later}
  The above lemma and corollary actually show that if we have an explicit 
  presentation of a Stone space as $S = Sp(2[G] / R)$, 
  we can construct an explicit sequence $\alpha:2^\N$ such that $||S|| \leftrightarrow \forall_{n:\N} \alpha(n) = 0$. 
\end{remark}


\begin{corollary}\label{PropositionsClosedIffStone}
  A proposition $P$ is closed iff it is a Stone space. 
\end{corollary}
\begin{proof}
  By the above, if $S$ is both a Stone space and a proposition, it is closed. 
  Conversely, note that 
  $$
  (\forall_{n:\N} \alpha(n) = 0 )\leftrightarrow Sp(2/\{\alpha(n)| n:\N\}).
  $$
  The latter is a proposition, as there is at most one Boolean map $2/\{\alpha(n)|n:\N\} \to 2$.
\end{proof}

\begin{lemma}\label{StoneEqualityClosed}
  Whenever $S:\Stone$, and $s,t:S$, the proposition $s=t$ is closed. 
\end{lemma}
\begin{proof}
  Suppose $S= Sp(B)$ and let $G$ be the generators of $B$. 
  Note that $s=t$ iff $s(g) =_2 t(g)$ for all $g:G$. 
  As $G$ is countable, and equality in $2$ is decidable, 
  $s=t$ is a countable conjunction of decidable propositions, hence 
  closed. 
\end{proof}
%
The following question was asked by Bas Spitters at TYPES 2024:
\begin{corollary}
  For $S:\Stone$ and $x,y,z:S$ 
  \begin{equation}\label{Apartness}
  x \neq y \to (x\neq z \vee y \neq z)
  \end{equation}
\end{corollary}
\begin{proof}
  As $x\neq y$, we can show that $\neg ( x = z \wedge y = z)$. 
  This in turn implies $\neg \neg ( x \neq  z \vee y \neq  z)$. 
  As, $x\neq z$ and $y \neq z$ are both open propositions, by \Cref{OpenCountableDisjunction} so is their disjunction. 
  By \Cref{rmkOpenClosedNegation}, that disjunction is double negation stable and \Cref{Apartness} follows. 
\end{proof}
\begin{remark}
  If \Cref{Apartness} holds in a type, we say that it's inequality is an apartness relation. 
  By a similar proof as above, it can be shown that in our setting inequality is an apartness relation 
  as soon as equality is open or closed. 
\end{remark}

\subsection{Types as spaces}
The subobject $\Open$ of the type of propositions induces a topology on every type. 
This is the viewpoint taken in synthetic topology. 
We will follow the terminology of \cite{SyntheticTopologyLesnik}, 
other references include \cite{SyntheticTopologyEscardo, TODOSortOutTaylorsReferences}.
%Defining a topology in this way has some benefits, which we summarize in this section. 

\begin{definition}
  Let $T$ be a type, and let $A\subseteq T$ be a subtype. 
  We call $A\subseteq T$ open or closed iff $A(t)$ is open or closed respectively for all $t:T$.
\end{definition}

\begin{remark}
  It follows immediately that the pre-image of an open by any map of types sends is open, so that any map is continuous. 
  This is only relevant for a space if the topology we defined above matches the topology one would expect. 
  In \Cref{StoneClosedSubsets}, we shall see that it resembles the standard topology of Stone spaces.
  In \Cref{IntervalClosedSubsets}, we shall see that it is the standard topology for the unit interval. 
\end{remark}

\begin{lemma}[transitivity of openness]\label{OpenTransitive}
  Let $T$ be a type, let $V\subseteq T$ open and let $W\subseteq V$ open. 
  Then the composite $W\subseteq V\subseteq T$ is open as well. 
\end{lemma}
\begin{proof}
  Denote $W'\subseteq T$ for the composite. 
  Note that $W'(t) = \Sigma_{v:V(t)} W(t,v)$. 
  As open propositions are closed under dependent sums (\Cref{OpenDependentSums}), 
  $W'(t)$ is an open proposition, as required. 
\end{proof}

\begin{remark}\label{OpenDominance}
  As the true proposition is open and openness is transitive, 
  $\Open$ can be called a dominance according to Proposition 2.25 of \cite{SyntheticTopologyLesnik}
\end{remark}



%\begin{remark}
%  Phao's principle is a special case of directed univalence. 
%\end{remark}
%\begin{proof}
%  \rednote{TODO}
%\end{proof}

\subsection{The topology on Stone spaces}
\begin{theorem}\label{StoneClosedSubsets}
  Let $A\subseteq S$ be a subset of a Stone space. TFAE:
  \begin{enumerate}[(i)]
    \item There exists a map $\alpha_{(\cdot)}:S \to 2^\N$ such that 
      $A (x) \leftrightarrow \forall_{n:\N} \alpha_x(n) = 0$ for any $x:S$. 
    \item There exists some countable family 
      $D_n,~{n:\N}$ 
      of decidable subsets of $S$ with $A = \bigcap_{n:\N} D_n$. 
    \item There exists a Stone space $T$ and some embedding $T\to S$ which image is $A$
    \item There exists a Stone space $T$ and some map $T\to S$ which image is $A$. 
    \item $A$ is closed.
  \end{enumerate}
\end{theorem}
\begin{proof}
\item 
  \begin{itemize}
  \item[$(i)\leftrightarrow (ii)$.] 
    $D_n$ and $\alpha_{(\cdot)}$ can be defined from each other by 
%    Define the decidable subsets of $S$ 
     $D_n(x) \leftrightarrow (\alpha_x(n) = 0)$. Then observe that %$A=\bigcap_{n:\N} D_n$ as 
     \begin{equation}
      (\bigcap_{n:\N} D_n) (x) \leftrightarrow 
      \forall_{n:\mathbb N} (\alpha_x(n) = 0) 
%      \leftrightarrow A s. 
     \end{equation}
   \item[$(ii) \to (iii)$.]
      Let $S=Sp(B)$. 
      By Stone duality, we have $d_n,~n:\N$ terms of $B$ such that $D_n = \{x:S| x(d_n) = 1\}$. 
      Let $C = B/(\neg d_n)_{n:\N}$.
      Then the map $Sp(C) \to S$ is as desired because
      $$Sp(C) = \{x:S| \forall_{n:\N} x(\neg d_n) =0\}  = \bigcap_{n:\N} D_n.$$
%      By \Cref{SurjectionsAreFormalSurjections}, t
%      The quotient map $B \twoheadrightarrow C$
%      corresponds to a map $\iota:Sp(C) \hookrightarrow  S$. 
%      For $s:S$, $s$ lies in the image of this map iff 
%      for all $n:\N$ we have  $s(\neg d_n) = 0$, 
%      \begin{equation}
%        x\in \iota(Sp(C)) \leftrightarrow x(\neg d_n) = 0 \leftrightarrow x(d_n) = 1 \leftrightarrow x\in D_n
%      \end{equation}
%      Thus the image of $\iota$ is given by $\bigcap_{n:\N} D_n$. 
   \item[$(iii) \to (iv)$] Immediate.
   \item[$(iv) \to (i)$.] 
     \rednote{TODO 
       The order of untracating is important in this proof, 
     and I struggle a bit with stressing this in a way this is clear (and concise). 
    Check with fresh eyes later. }
      Let $f:T\to S$ be a map between Stone spaces. 
      Assume $S = Sp(A), T = Sp (B)$. 
%      For this proof, we work with explicit presentations for $A,B$. 
%
      Let $G$ be a countable set of generators of $A$. 
      Assume also we have countable sets of generators and relations for $B$. 
%
      Following \Cref{FiberConstruction}, using $G$, for each $x:S$, we can construct 
      a countable set $I_x\subseteq B$ such that $$Sp(B/I_x) = (\Sigma_{y:T} f(y) = x) .$$
      By \Cref{ExplicitTruncationStoneClosed}, we can construct a sequence 
      $\alpha_x$ such that this type is inhabited iff $\forall_{n:\N} \alpha_x(n) = 0$,
      as required. 
%
%      Recall that the propositional truncation of a Stone is closed, as it is the negation of $0=1$ in the underlying 
%      Boolean algebra, which is open as it f
%
%
%      the core idea of the proof was that the closed proposition corresponds to checking equality in the underlying BA, 
%      which was closed as 
%
%
%
%
%      Note that $x$ in the image of $f$ iff $0\neq_{B/I_x} 1$. 
%      At this point, we have generators and relations of $B/I_x$ as data.
%      Hence using the proof of \Cref{BooleEqualityOpen}, we can construct a sequence 
%      $\alpha_x:2^\N$ such that $0 =_{B/I_x}1\leftrightarrow \exists_{n:\N} \alpha_x(n) = 0$. 
%      And for $\beta_x(n) = 1-\alpha_x(n)$, we conclude that 
%      \begin{equation}
%        x\in f(T) \leftrightarrow \forall_{n:\N} \beta_x(n) = 0
%      \end{equation}
%      Note that we did not use any choice axioms in the proof of this implication,
%      as we untruncated our assumptions before we specified $x$. 
   \item [$(i) \to (v)$.] By definition.
   \item[$(v) \to (iv)$.]
     As $A$ is closed, it induces a map $a:S\to \Closed$. 
     We can cover the closed propositions with Cantor space
     by sending 
     $\alpha \mapsto \forall_{n:\mathbb N} \alpha n = 0.$
     Now local choice gives us that there merely exists $T, e, \beta_\cdot$ as follows:
     \begin{equation}
       \begin{tikzcd}
         T \arrow[r,"\beta_\cdot"] \arrow[d, two heads,"e"] & 2^\mathbb N 
         \arrow[d,two heads, "\forall_{n:\mathbb N} (\cdot)n = 0"] \\
         S \arrow[r,"a"] & \Closed
       \end{tikzcd} 
     \end{equation} 
     Define $B(x) \leftrightarrow \forall_{n:\mathbb N} \beta_x(n) = 0$. 
     As $(i) \to (iii)$ by the above, $B$ is the image of some Stone space. 
     Furthermore, note that $A$ is the image of $B$, thus $A$ is the image of some Stone space. 
\end{itemize} 
\end{proof} 
\rednote{Ordering from here on out is WIP}

\begin{remark}\label{ClosedInStoneIsStone}
Using condition $(iii)$, the previous result implies that closed subtype of Stone spaces are Stone.
\end{remark}

\begin{corollary}\label{InhabitedClosedSubSpaceClosed}
  For $S:\Stone$ and $A\subseteq S$ closed, we have 
  $\exists_{x:S} A(x)$ is closed. 
\end{corollary}
\begin{proof}
  By \Cref{ClosedInStoneIsStone} we have that $\Sigma_{x:S}A(x)$ is Stone, so its truncation is closed by \Cref{TruncationStoneClosed}.
\end{proof}

\begin{corollary}\label{ClosedDependentSums}
  Closed propositions are closed under dependent sums. 
\end{corollary}
\begin{proof}
  Let $P:\Closed$ and $Q:P \to \Closed$. 
  Then $\Sigma_{p:P} Q(p) \leftrightarrow \exists_{p:P} Q(p)$.
  As $P$ is Stone by \Cref{PropositionsClosedIffStone}, 
%  As $P$ is Stone by \Cref{PropositionsClosedIffStone}, it is also compact Hausdorff, thus
  \Cref{InhabitedClosedSubSpaceClosed} gives that $\Sigma_{p:P} Q(p)$ is closed. 
\end{proof}
\begin{remark}
  Analogously to \Cref{OpenTransitive} and \Cref{OpenDominance}, it follows that 
  closedness is transitive and $\Closed$ forms a dominance. 
\end{remark}

We can get a dual to completeness.

\begin{lemma}\label{DualCompleteness}
Let $A$ and $B$ be c.p. boolean algebra with a map:
\[i:Sp(A)\to Sp(B)\] 
The following are equivalent:
\begin{enumerate}[(i)]
\item The induced map $B\to A$ is surjective.
\item The map $i$ is an embedding.
\item The map $i$ is a closed embedding.
\end{enumerate}
\end{lemma}

\begin{proof}
\begin{itemize}
\item[$(i)\to (ii)$] Immediate.
\item[$(ii)\to (iii)$] By \Cref{StoneEqualityClosed} the fibers of $i$ are closed in $Sp(A)$, so by \Cref{ClosedInStoneIsStone} they are Stone, so they are closed by \Cref{PropositionsClosedIffStone}.
\item[$(iii)\to (i)$] By we have that $Sp(A) = \cap_{n:\N}D_n$ for $D_n$ decidable in $Sp(B)$. Assuming $D_n$ correponds to $b_n:B$ though duality, we then have that $A=B/ (b_n)_{n:\N}$ and the induced map is the quotient map:
$$B\to B/(b_n)_{n:\N}$$
which is surjective.
\end{itemize}
\end{proof}

\begin{lemma}\label{StoneSeperated}
  If $S:\Stone $, and $F,G:S \to \Closed$ be such that $F\cap G = \emptyset$. 
  Then there exists a decidable subset $D:S \to 2$ such $F\subseteq D, G \subseteq \neg D$. 
\end{lemma}
\begin{proof}
  Assume $S = Sp(B)$. 
  By the above theorem, there exists sequences $f_n,g_n:B,~n:\N$ such that 
  $x\in F$ iff $x(f_n) = 1$ for all $n:\N$ and 
  $y\in G$ iff $y(g_m) = 1$ for all $m:\N$. 
%
  Denote $R\subseteq B$ for $\{\neg f_n|n:\N\} \cup\{\neg g_n|n:\N\}$. 
  Note that any inhabitant $Sp(B/R)$ gives a map $x:B\to 2$ such that
  $x(g_n)= x(f_n) = 1$ for all $n:\N$, hence $x\in F \cap G$. 
  As $F\cap G = \emptyset $, it follows that $Sp(B/R)$ is empty.
%
  Thus there exists finite sets $I,J\subseteq \N $ such that 
  $$1 =_B ((\bigvee_{i\in I} \neg f_i) \vee (\bigvee_{j\in J} \neg g_j)).$$
%
  Let $y\in G$. Then $y(\neg g_j) = 0$ for all $j \in J$. 
  Hence 
  $$
  1 = y(1) 
  = 
  y(\bigvee_{i\in I} \neg f_i) = y (\neg (\bigwedge_{i\in I} f_i))
  $$
  Thus $y(\bigwedge_{i\in I} f_i) = 0$. 
  Note that if $x\in F$, we have $x(f_i) = 1$ for all $i\in I$, hence 
  $x(\bigwedge_{i\in I} f_i) = 1$. 
  Thus for $D$ corresponding to $\bigwedge_{i\in I} f_i$, we have that 
  $F\subseteq D, G\subseteq \neg D$ as required. 
\end{proof} 

\begin{corollary}\label{StoneOpenSubsets}
  Let $A\subseteq S$ be a subset of a Stone space, then 
  $A$ is open iff there exists some countable family $D_n,~n:\N$ of decidable subsets of $S$ with 
  $A = \bigcup_{n:\N} D_n$. 
\end{corollary}
\begin{proof}
  By \Cref{rmkOpenClosedNegation}, $A$ is open iff $\neg A$ is closed and $A = \neg \neg A$. 
  By \Cref{StoneClosedSubsets}, $\neg A$ is closed iff 
  $\neg A = \bigcap_{n\in \N} E_n$ for some countably family $E_n,~n:\N$. 
  Thus $\neg \neg A = \neg (\bigcap_{n\in \N} E_n)$. 
  As a concequence of Markov's principle (\Cref{MarkovPrinciple}), we have that 
  $\neg (\bigcap_{n\in \N} E_n)= \bigcup_{n\in \N} \neg E_n$. 
  Thus $D_n := \neg E_n$ is as required. 
\end{proof}


\section{Compact Hausdorff spaces}
\begin{definition}
  A type $X$ is called a compact Hausdorff space if its identity types are closed propositions and there exists some $S:\Stone$ with a surjection $S\twoheadrightarrow X$. We write $\CHaus$ for the type of compact Hausdorff spaces.
\end{definition}

%This means that compact Hausdorff spaces are precisely quotients of Stone spaces by closed equivalence relations.

\subsection{Topology on compact Hausdorff spaces}

\begin{lemma}\label{CompactHausdorffClosed}
  Let $X:\CHaus$ with $S:\Stone$ and a surjective map $q:S\twoheadrightarrow X$.
  Then $A\subseteq X$ is closed if and only if it is the image of a closed subset of $S$ by $q$. 
\end{lemma}
\begin{proof}
%  If $A$ is closed, then it's pre-image under any map is also closed. 
%  In particular for $q:S\to X$ the quotient map, $q^{-1}(A)$ is closed. 
  As $q$ is surjective, we have $q(q^{-1}(A)) = A$.
  If $A$ is closed, so is $q^{-1}(A)$ and 
  hence $A$ is the image of a closed subset of $S$. 
  Conversely, let $B\subseteq S$ be closed. Then $x\in q(B)$ if and only if
   \[\exists_{t:S} (B(t) \wedge q(s) = x).\]
   Hence by \Cref{InhabitedClosedSubSpaceClosed}, $q(B)$ is closed. 
  % Define $A'\subseteq S$ by 
  %\[A'(s) = \exists_{t:S} (B(t) \wedge q(s) = q(t)).\]
  %Note that $B(t)$ and $q(s) = q(t)$ are closed. 
  %Hence by \Cref{InhabitedClosedSubSpaceClosed}, $A'$ is closed. 
  %Also $A'$ factors through $q$ as a map $A: X\to \Closed$.
  %Furthermore, $A'(s) \leftrightarrow (q(s)\in q(B))$. 
  %Hence $A=q(B)$. 
\end{proof}

The next two corollaries mean that compact Hausdorff spaces behave as finite sets for the purposes of unions/intersections of open/closed sets.

\begin{corollary}\label{InhabitedClosedSubSpaceClosedCHaus}
Assume given $X:\Chaus$ with $A\subseteq X$ closed. Then $\exists_{x:X} A(x)$ is closed, and equivalent to $A \neq \emptyset$. 
\end{corollary}

\begin{proof}
From \Cref{CompactHausdorffClosed} and \Cref{StoneClosedSubsets}, it follows that $A\subseteq X$ is closed if and only if it is the image of a map $T\to X$ for some $T:\Stone$. Then $\exists_{x:X} A(x)$ if and only $\propTrunc{T}$, which is closed by \Cref{TruncationStoneClosed}. Therefore $\exists_{x:X} A(x)$ is $\neg\neg$-stable and equivalent to $A \neq \emptyset$. 
  %If $A$ is closed, it follows from \Cref{InhabitedClosedSubSpaceClosed} that $\exists_{x:X} A(x)$ is closed as well, 
 % hence $\neg\neg$-stable, and equivalent to $A \neq \emptyset$. 
\end{proof}

%\begin{remark}\label{InhabitedClosedSubSpaceClosedCHaus}
%  Let $X:\Chaus$.
%  From \Cref{StoneClosedSubsets}, it follows that $A\subseteq X$ is closed if and only if it is the image of a map 
%  $T\to X$ for some $T:\Stone$. 
%  If $A$ is closed, it follows from \Cref{InhabitedClosedSubSpaceClosed} that $\exists_{x:X} A(x)$ is closed as well, 
%  hence $\neg\neg$-stable, and equivalent to $A \neq \emptyset$. 
%\end{remark}
%\begin{corollary}
%  For $X:\CHaus$ a subtype $A\subseteq X$ is closed iff it is the image of 
%  a map $T\to X$ for some $T:\Stone$. 
%\end{corollary}
%\begin{proof}
%  Directly from the above and \Cref{StoneClosedSubsets}.
%\end{proof}
%WhyDidWeNeedThis%\begin{remark}
%WhyDidWeNeedThis%  It is not the case that every closed subset of a compact Hausdorff space can be written 
%WhyDidWeNeedThis%  as countable intersection of decidable subsets. 
%WhyDidWeNeedThis%  In \Cref{UnitInterval}, we shall introduce the unit interval $[0,1]$ as a compact Hausdorff space with many closed 
%WhyDidWeNeedThis%  subsets, but only two decidable subsets. 
%WhyDidWeNeedThis%  In \Cref{ConnectedComponent}, we shall actually see that whenever every singleton of a compact Hausdorff space $X$
%WhyDidWeNeedThis%  can be written as countable intersection of decidable subsets, $X$ is Stone. 
%WhyDidWeNeedThis%  \rednote{Actually, we'll see that $\Sp(2^X)$ and $X$ are bijective sets, 
%WhyDidWeNeedThis%    which only implies that $X$ is Stone if $2^X:\Boole$, but this depends on our definition of countable, 
%WhyDidWeNeedThis%see \Cref{CountabilityDiscussion}}
%WhyDidWeNeedThis%\end{remark}


\begin{corollary}\label{AllOpenSubspaceOpen}
  Assume given $X:\Chaus$ with $U\subseteq X$ open. Then $\forall_{x:X} U(x)$ is open. 
\end{corollary}
%\begin{proof}
%  As $U$ is open, $\neg U$ is closed. 
%  So $\exists_{x:X} \neg U(x)$ is closed by \Cref{InhabitedClosedSubSpaceClosedCHaus}. 
%  Using \Cref{rmkOpenClosedNegation}, it follows that 
%  $\neg (\exists_{x:X} \neg U(x))$ is open. 
%  Furthermore, it is equivalent to $\forall_{x:X} \neg \neg U(x)$, 
%  which is equivalent to $\forall_{x:X} U(x)$ by \Cref{rmkOpenClosedNegation}.
%\end{proof}

Next lemma means that compact Hausdorff space are not too far from being compact in the classical sense.

\begin{lemma}\label{CHausFiniteIntersectionProperty}
  Given $X:\Chaus$ and $C_n:X\to \Closed$ closed subsets such that $\bigcap_{n:\N} C_n =\emptyset$, there is some $k:\N$ 
  with $\bigcap_{n\leq k} C_n  = \emptyset$. 
\end{lemma}
\begin{proof}
  By \Cref{CompactHausdorffClosed} it is enough to prove the result when $X$ is Stone, and by \Cref{StoneClosedSubsets} we can assume $C_n$ decidable.
  So assume 
  $X=\Sp(B)$ and $c_n:B$ such that
  \[C_n = \{x:B\to 2\ |\ x(c_n) = 0\}.\]
  Then we have that
  \[\Sp(B/(c_n)_{n:\N})%\ |\ n:\N\}) 
  \simeq \bigcap_{n:\N} C_n = \emptyset .\]
  Hence 
%  $0=_{B/(\neg c_n)_{n:\N}}1$ 
  $0=1$ in $B/(c_n)_{n:\N}$ %\ |\ n:\N\}$, 
  and there is some $k:\N$ with 
  $\bigvee_{n\leq k} c_n = 1$, which means that
  \[\emptyset = \Sp(B/(c_n)_{n\leq k}) %\ |\ n\leq k\})  
  \simeq \bigcap_{n\leq k} C_n \]
  as required.
\end{proof}

\begin{corollary}\label{ChausMapsPreserveIntersectionOfClosed}
  Let $X,Y:\CHaus$ and $f:X \to Y$. 
  Suppose $(G_n)_{n:\N}$ is a decreasing sequence of closed subsets of $X$. 
  Then $f(\bigcap_{n:\N} G_n) = \bigcap_{n:\N}f(G_n)$. 
\end{corollary}
\begin{proof}
  It is always the case that $f(\bigcap_{n:\N} G_n) \subseteq \bigcap_{n:\N} f(G_n)$. 
  For the converse direction, suppose that $y \in f(G_n)$ for all $n:\N$. 
  We define $F\subseteq X$ closed by $F=f^{-1}(y)$. 
  Then for all $n:\N$ we have that $F\cap G_n$ is %merely inhabited and therefore 
  non-empty. 
  By \Cref{CHausFiniteIntersectionProperty} this implies that $\bigcap_{n:\N}(F\cap G_n) \neq \emptyset$. 
  By \Cref{InhabitedClosedSubSpaceClosedCHaus},  we have that %and \Cref{rmkOpenClosedNegation}, 
  $\bigcap_{n:\N} (F\cap G_n)$ is merely inhabited. Thus $y\in f(\bigcap_{n:\N} G_n)$ as required. 
\end{proof}

\begin{corollary}\label{CompactHausdorffTopology}
Let $A\subseteq X$ be a subset of a compact Hausdorff space and $p:S\twoheadrightarrow X$ be a surjective map with $S:\Stone$. Then $A$ is closed (resp. open) if and only if there exists a sequence $(D_n)_{n:\N}$ of decidable subsets of $S$ such that $A = \bigcap_{n:\N} p(D_n)$ (resp. $A = \bigcup_{n:\N} \neg p(D_n)$).
%\begin{itemize}
%  \item $A$ is closed iff %if and only if 
%    it can be written as $\bigcap_{n:\N} p(D_n)$
%for some $D_n\subseteq S$ decidable. 
%  \item $A$ is open iff %if and only if 
%    it can be written as $\bigcup_{n:\N} \neg p(D_n)$
%for some $D_n\subseteq S$ decidable.
%\end{itemize}  
\end{corollary}
\begin{proof}
  The characterization of closed subsets follows from characterization (ii) in \Cref{StoneClosedSubsets}, 
  \Cref{CompactHausdorffClosed} 
  and \Cref{ChausMapsPreserveIntersectionOfClosed}. 
%  The characterization of open sets 
  To deduce the characterization of open subsets we use \Cref{rmkOpenClosedNegation} and
  \Cref{ClosedMarkov}.
\end{proof}
%
\begin{remark}
  For $S:\Stone$, there is a surjection $\N\twoheadrightarrow 2^S$. 
  It follows that for any $X:\CHaus$ there is a surjection from $\N$ to a basis of $X$. 
  Classically this means that $X$ is second countable. 
\end{remark}
%It follows that compact Hausdorff spaces are second countable:
%\begin{corollary}
%  Any $X:\Chaus$ is has a topological basis which is countable.
%\end{corollary}
%\begin{proof}
%  By \Cref{CompactHausdorffTopology}, 
%  a basis is given by the images of the decidable subsets of some $S:\Stone$. 
%  By \cref{ODiscBAareBoole}, $2^S$ is 
%  overtly discrete so we have a surjection $\N\to 2^S$.
%  \end{proof}
%

Next lemma means that compact Hausdorff spaces are normal.

\begin{lemma}\label{CHausSeperationOfClosedByOpens}
 Assume $X:\CHaus$ and $A,B\subseteq X$ closed such that $A\cap B=\emptyset$. 
  Then there exist $U,V\subseteq X$ open such that $A\subseteq U$, $B\subseteq V$ and $U\cap V=\emptyset$. 
\end{lemma}
\begin{proof}
  Let $q:S\to X$ be a surjective map with $S:\Stone$.
  As $q^{-1}(A)$ and $q^{-1}(B)$ are closed, 
  by \Cref{StoneSeperated}, there is some $D:S \to 2$ such that
  $q^{-1}(A) \subseteq D$ and $q^{-1}(B) \subseteq \neg D$. 
  Note that $q(D)$ and $q(\neg D)$ are closed by \Cref{CompactHausdorffClosed}. 
%  We define $U = \neg q(\neg D) $%\cap \neg B$ 
%  and $V=\neg  q(D) $.%\cap \neg A$. 
  As $q^{-1}(A) \cap \neg D  =\emptyset$, we have that 
  $A\subseteq \neg q(\neg D):=U$. 
%  As $A\cap B = \emptyset$, we have that $A\subseteq \neg B$ so $A\subseteq U$.
%  Similarly $B\subseteq V$. 
  Similarly $B\subseteq \neg q(D):=V$. 
  Then $U$ and $V$ are disjoint because $\neg q(D)\cap \neg q(\neg D) = \neg (q(D)\cup q(\neg D)) = \neg X = \emptyset$.
\end{proof}


\subsection{Compact Hausdorff spaces are stable under dependent sums}

\begin{lemma}
A type $X$ is Stone iff it is merely a closed in $2^\N$.
\end{lemma}

\begin{proof}
Any countably presented boolean algebra $B$ is enumerable, which gives a surjective morphism:
$$ 2[\N]\to B$$
so that by \Cref{DualCompleteness} we merely have a closed embedding:
$$ Sp(B)\to 2^\N$$
\end{proof}

\begin{lemma}\label{SigmaStoneCompactHausdorff}
Assume given $S:\Stone$ and $T:X\to \Stone$. Then $\Sigma_{x:S}T(x)$ is Compact Hausdorff.
\end{lemma}

\begin{proof}
By \Cref{ClosedDependentSums} we have that identity type in $\Sigma_{x:S}T(x)$ are closed.

We know that for any $x:S$ we have that $\exists_{C:2^\N\to \Closed} T_x = \Sigma_{y:2^\N}C(y)$. Using local choice we get $S':\Stone$ with a surjective map:
$$q:S'\to S$$
and:
$$ C : S'\to 2^\N\to\Closed$$
such that for all $x:S'$ we have $T(q(x)) = \Sigma_{y:2^\N}C(x,y)$. This gives a surjective map:
$$ \Sigma_{x:S'}\Sigma_{y:2^\N} C(x,y)\to \Sigma_{x:S}T(x)$$
The source is Stone by \Cref{StoneClosedUnderPullback} and \Cref{ClosedInStoneIsStone} so we can conclude.
\end{proof}

\begin{lemma}
Assume given $C:\CHaus$ and $D:X\to \CHaus$. Then $\Sigma_{x:C}D(x)$ is Compact Hausdorff.
\end{lemma}

\begin{proof}
By \Cref{ClosedDependentSums} we have that identity type in $\Sigma_{x:C}D(x)$ are closed.

We know that for any $x:C$ we have that $\exists_{T:\Stone} T\twoheadrightarrow C(x)$. Using a surjective map:
$$ S\to C$$
with $S:\Stone$ and local choice we get $S':\Stone$ with a surjective map:
$$q:S'\to C$$
such that for all $x:S'$ we have $T(x):\Stone$ and a surjective map $T(x)\to D(q(x))$. This gives a surjective map:
$$ \Sigma_{x:S'}T(x)\to \Sigma_{x:C}D(x)$$
By \Cref{SigmaStoneCompactHausdorff} we have a surjective map from a Stone space to the source so we can conclude.
\end{proof}
\subsection{Stone spaces are stable under dependent sums}

We will show that Stone spaces are precisely totally disconnected compact Hausdorff spaces. We will use this to prove that a dependent sum of Stone space is Stone.

\begin{lemma}\label{OpenInNAreDecidableInN}
For any open $U$ in $\N$, there merely exists a decidable $D$ in $\N$ such that $D=U$.
\end{lemma}

\begin{proof}
For any open proposition $U(n)$ we know that there merely exists $\alpha:\N_\infty$ such that:
$$ U(n) = \Sigma_{k:\N}\alpha_n(k)=0$$
Using countable choice we have that:
$$ U = \Sigma_{n,k:\N}\alpha_n(k)=0$$
and we conclude using $\N=\N\times\N$.
\end{proof}

\begin{lemma}\label{AlgebraCompactHausdorffCountablyPresented}
Assume $X$ compact Hausdorff, then $2^X$ is countably presented.
\end{lemma}

\begin{proof}
First we choose $S\to X$ surjective with $S$ Stone and prove that $2^X$ is an open subalgebra of $2^S$.

 The map $S\to X$ induces an injection of Boolean algebras $2^X \hookrightarrow 2^S$.
  Note that $a:S\to 2$ lies in $2^X$ iff for all $s,t:S$, we have $a(s) = a(t)$ whenever $s\sim t$.
  Note that $a(s) = a(t)$ is decidable and $s\sim t$ is closed, hence 
  $(s\sim t) \to (a(s) = a(t))$ is open (\Cref{ImplicationOpenClosed})
  By \Cref{AllOpenSubspaceOpen}, we conclude that 
  $\forall_{s:S} \forall_{t:S} ((s\sim t) \to (a(s) = a(t)))$ is open. 
  Hence $2^X$ is an open subalgebra of $2^S$. 

Now we prove that open subalgebras of countably presented agebras are countably presented. Assume $U\subset 2[\N] / F$ such a subalgebra. We have that $U$ is equivalent to the algebra generated by the $s:2[\N]$ such that $[s]\in U$ quotiented by the relation $s=t$ for all $s,t:2[\N]$ such that $[s],[t]\in U$ and $[s]=[t]$.

Using that $2[\N]$ is countable and that $[s]=[t]$ is open by \Cref{BooleEqualityOpen}, we see that $U$ is generated by variables and relations each indexed by an open in $\N$. But by \Cref{OpenInNAreDecidableInN} any open in $\N$ is countable, so $U$ is countably presented.
\end{proof}

\begin{lemma}\label{ConnectedComponentClosedInCompactHausdorff}
For all $X:\CHaus$ with $x:X$, then we have that $Q_x$ is a countable intersection of decidable in $X$.
\end{lemma}

\begin{proof}
By \Cref{AlgebraCompactHausdorffCountablyPresented} we have that $2^X$ is countably presented, therefore we can enumerate the elements of $2^X$, say as $(D_n)_{n:\N}$. if we define $E_n$ for $n:\N$ as $D_n$ if $x\in D_n$ and $X$ otherwise, we have that:
$$\cap_{n:\N}E_n = Q_x$$
\end{proof}

\begin{definition}
  Let $X:\Chaus$ and $x:X$. 
  We define the connected component of $x$ (denoted $Q_x$)
  as the intersection of all decidable subsets of $X$ containing $x$. 
\end{definition}

\begin{lemma}\label{ConnectedComponentSubOpenHasDecidableInbetween}
  Let $X:\Chaus, x:X$ and suppose $U\subseteq X$ is open with $Q_x\subseteq U$. 
  Then we have some decidable $E\subseteq X$ with $E(x)$ and $E\subseteq U$. 
\end{lemma}
\begin{proof}
  By \Cref{ConnectedComponentClosedInCompactHausdorff}, we have $Q_x = \bigcap_{n:\N}B_n$ with $B_n$ decidable. 
  If $Q_x \subseteq U$, we have that 
  $$Q_x\cap \neg U = \bigcap_{n:\N} (B_n \cap \neg U)$$ is empty. 
  By \Cref{CHausFiniteIntersectionProperty} there is some $N:\N$ with 
  $$(\bigcap_{n\leq N} B_n )\cap \neg U  = \bigcap_{n\leq N} (B_n \cap \neg U) = \emptyset.$$
  Therefore $\bigcap_{n\leq N} B_n \subseteq U$, furthermore a finite intersection of decidable subsets is decidable. 
  As $x\in B_n$ for all $n:\N$, $x\in \bigcap_{n\leq N} B_n$ as well and we're done. 
\end{proof}

\begin{lemma}\label{ConnectedComponentConnected}
Let $X$ be Compact Hausdorff with $x:X$. Then $Q_x$ is connected.
\end{lemma}
\begin{proof}
Assume given a separation $Q_x = A\cap B$ with $A,B$ disjoint and decidable in $Q_x$, and let us assume that $x\in A$. We want to prove that $B=\empty$. 

Since $Q_x$ is closed in $X$ by \Cref{ConnectedComponentClosedInCompactHausdorff}, we have that $A$ and $B$ are closed disjoint in $X$, so that by \Cref{CHausSeperationOfClosedByOpens} we have $U,V$ disjoint open such that $A\subset U$ and $B\subset V$. 

By \Cref{ConnectedComponentSubOpenHasDecidableInbetween} we have a decidable $E$ such that $Q_x\subset E\subset U\cup V$. Then we define $F = E\cap U$. We have that $F$ is open, it is closed as $F=E\cap \neg V$, therefore it is decidable by \Cref{ClopenDecidable}.

Then $Q_x\subset F$ with $F$ decidable and $B\cap F = \empty$ so that $Q_x\cap B = \empty$ and $B=\empty$.
\end{proof}

\begin{lemma}\label{StoneCompactHausdorffTotallyDisconnected}
Let $X:\CHaus$, then $X$ is Stone iff $\forall_{x:X}Q_x=\{x\}$.
\end{lemma}

\begin{proof}
By duality, it is clear that for all $x:S$ with $S$ Stone we have that $Q_x=\{x\}$.

For the converse, we show that the map:
\[X\to Sp(2^X)\]
is an equivalence and conclude by \Cref{AlgebraCompactHausdorffCountablyPresented}. 

Surjectivity always hold, indeed considering $q:S\to X$ surjective with $S$ Stone, we have that $2^X\subset 2^S$ as so that the by \Cref{FormalSurjectionsAreSurjections} the map:
$$S = Sp(2^S)\to Sp(2^X)$$
is surjective and it factors though $X$.

Now let us prove injectivity. Assume $x,y:X$ having the same image in $Sp(2^X)$. This means that any map in $X\to 2$ has the same value on $x$ and $y$, so $x\in Q_y$ and by hypothesis $x=y$.
\end{proof}

\begin{theorem}
Assume given $S:\Stone$ with $T:S\to\Stone$. Then $\Sigma_{x:S}T(x)$ is Stone.
\end{theorem}

\begin{proof}
By \Cref{SigmaStoneCompactHausdorff} we have that $\Sigma_{x:S}T(x)$ is compact Hausdorff. By \Cref{StoneCompactHausdorffTotallyDisconnected} it is enough to show that for all $x:S$ and $y:T(x)$ we have that $Q_{(x,y)}$ is a singleton.

Assume $(x',y')\in Q_{(x,y)}$, then for any map $q:S\to 2$ we have that:
$$ q(x) = q\circ \pi_1(x,y) = q\circ \pi_1(x',y') = q(x')$$
so that $x'\in Q_x$ and since $S$ is Stone by \Cref{StoneCompactHausdorffTotallyDisconnected} we have that $x=x'$.

Therefore we have $Q_{(x,y)}\subset \{x\}\times T_x$ so that by \Cref{ConnectedComponentConnected} we have an inhabited  connected subtype of a Stone space. Then any map $T_x\to 2$ is constant on $Q_{(x,y)}$ and by \Cref{StoneCompactHausdorffTotallyDisconnected} we conclude that it is a singleton.
\end{proof}




%\section{The Unit interval}
%In this section we will introduce the unit interval $I$ as compact Hausdorff space. 
The definition is based on \cite{Bishop}. 
We will then calculate the cohomology of $I$. 
For a proof that the unit interval corresponds to the definition using Cauchy sequences, 
we refer to the appendix. 


\input{Interval/CauchySequences}
\input{Interval/BinaryClosedEquivalence}
\input{Interval/EquivalenceOfSims}
\input{Interval/Surjective}





\begin{theorem}
  The interval of Cauchy reals is isomorphic to $2^\N / \sim_t$. 
\end{theorem} 
\begin{proof}
  This follows from the fact that $b:2^\N$ is such that $\alpha\sim_n \beta$ iff $b(\alpha)\sim_t b(\beta)$. 
  and for every Cauchy real, there is a binary sequence being sent to it, so the composition of $b$ and the 
  quotient from Caucy sequences to Cauchy real is a surjection. 
\end{proof}

\begin{corollary}
  The interval is compact Hausdorff. 
\end{corollary}





\appendix
%\section{Technical details}
\section{Some notes on $\Noo$}
Recall that we defined $B_\infty$ as the quotient of the freely generated algebra 
over $p_n,~n\in\N$ by the relations $\{p_n \wedge p_m | n\neq m\}$. 

\begin{lemma}\label{N-co-fin-cp}
  The Boolean algebra of co-finite subsets of $\N$
  is equivalent to $B_\infty$. 
\end{lemma}
\begin{proof}
  Let $f:B_\infty \to \N_{(co)fin}$ be induced by sending $p_n$ to $\{n\}$. 
  Note that whenever $n\neq m$, we have 
  $f(p_n)\wedge f(p_m) = \{n\} \cap \{m\} = \emptyset$, 
  thus $f$ respects the relations of $B_\infty$ and is well-defined.

  Define $g:N_{(co)fin)} \to B_\infty$ as follows:
  \begin{itemize}
    \item On a finite subset $I$, we define $g(I) = \bigvee_{i\in I} p_i$, 
    \item On a cofinite subset $J$, we define $g(J) = \bigwedge _{i \in J^C} \neg p_i$. 
  \end{itemize}
  Note that in these cases we indeed have $I,J^C$ are finite, so these are well-defined elements. 
  We must show that $g$ is a Boolean morphism. 

  \begin{itemize}
    \item 
      By deMorgan's laws, $g$ preserves $\neg$:
      for $I$ finite we have
      \begin{equation}
      \neg g(I) = \neg (\bigvee_{i\in I} p_i) = \bigwedge_{i\in I} \neg p_i = g(I^C)
      \end{equation}
      And for $J$ cofinite, we apply similar reasoning. 
    \item To see that $g$ preserves $\vee$, we need to check three cases
      \begin{itemize}
        \item If both $I,J$ are finite, then 
        \begin{equation} 
          g(I \cup J) = \bigvee_{i\in I \cup J} p_i= \bigvee_{i\in I} p_i \vee \bigvee_{j\in J} p_j 
          = g(I) \vee g(J)
        \end{equation}
        and we're done. 
      \item If both $I,J$ are cofinite, we have
        \begin{equation}
          g(I) \vee g(J) = 
          \bigwedge_{i \in I^C} \neg p_i \vee 
          \bigwedge_{j \in J^C} \neg p_j 
          = 
          \bigwedge_{i\in I^C} 
          \bigwedge_{j \in J^C}(\neg p_i \vee  \neg p_j) 
        \end{equation}
        Now note that in $B_\infty$, we have 
        \begin{equation}
          \neg p_i \vee \neg p_j = \neg ( p_i \wedge p_j) = 
          \begin{cases}
            \neg p_i \text{ if } i = j\\
            1 \text{ if } i \neq j  
          \end{cases}
        \end{equation}
        Therefore, we can leave out the case that $i\neq j$ in the calculation of the above meet, and
        \begin{equation}
          \bigwedge_{i\in I^C} 
          \bigwedge_{j \in J^C}(\neg p_i \vee  \neg p_j)  
          = 
          \bigwedge_{i \in (I^C \cap J^C)} \neg p_i
          = 
          \bigwedge_{i \in (I \cup J)^C} \neg p_i 
        \end{equation}
        as $I\cup J$ must also be cofinite, this equals 
          $ g( I \cup J)$. 
        \item 
          If $I$ is finite and $J$ cofinite, we have 
          that $I\cup J$ is cofinite, hence 
          \begin{equation}
            g(I\cup J) = \bigwedge_{k\in (I \cup J)^C} \neg p_k
            = \bigwedge_{k \in (J^C -I)} \neg p_k
          \end{equation}
          Now note that 
          whenever $i\neq k$, we have 
          \begin{equation}
            p_i = (p_i \wedge \neg p_k) \vee (p_i \wedge p_k) = 
            (p_i \wedge \neg p_k) \vee 0 = p_i \wedge \neg p_k
          \end{equation}
          Hence by absorption
          \begin{equation} 
            (p_i \vee \neg p_k)  =
              \begin{cases}
                1 \text{ if } i = k \\
                \neg p_k \text{ if } i \neq k
              \end{cases}
          \end{equation}
          As for all $k\in J^C-I$ and all $i\in I$ we have $k\neq i$, we may thus write
          \begin{equation}\label{eqnCofiniteHelper1}
            \bigwedge_{k \in (J^C - I)} \neg p_k = 
            \bigwedge_{k \in (J^C - I)} (\neg p_k \vee (\bigvee_{i\in I} p_i))
          \end{equation}
          We now note that 
          \begin{equation}\label{eqnCofiniteHelper2}
            1=\bigwedge_{i\in I} 1 = \bigwedge_{i\in I} (\neg p_i \vee (\bigvee_{i\in I} p_i)).
          \end{equation}
          Taking the meet of the expressions in \Cref{eqnCofiniteHelper1} and \Cref{eqnCofiniteHelper2}, 
          we see that 
          \begin{equation}
            \bigwedge_{k \in (J^C - I)} \neg p_k = 
            \bigwedge_{j \in J^C} (\neg p_j \vee (\bigvee_{i\in I} p_i))
          \end{equation}
          And using distributivity rules, we can see that 
          \begin{equation}
            \bigwedge_{j \in J^C} (\neg p_j \vee (\bigvee_{i\in I} p_i))
            = 
            (\bigwedge_{j \in J^C} \neg p_k) \vee (\bigvee_{i\in I} p_i)
          \end{equation}
          From which we may conclude that $g(I\cup J) = g(I) \cup g(J)$. 
      \end{itemize}
    \item The case for $\wedge$ is completely dual to the case for $\vee$. 
  \end{itemize}
We conclude that $g$ is a Boolean morphism. 
Furthermore, it is easy to see that $g$ and $f$ are each other's inverse, 
thus the Boolean algebras are isomorphic. 
\end{proof}
\begin{remark}\label{AppendixCofiniteOrFinite}
  As a consequence of the above proof, any $b:B_\infty$ corresponds either to 
  \begin{itemize}
    \item a finite set $I$, in which case $b = \bigvee_{i\in I} p_i$. 
    \item a cofinite set $J$, in which case $b = \bigwedge_{j\in J^C} \neg p_j$. 
  \end{itemize}
  We will call $b$ finite/cofinite respectively. 
\end{remark}
\begin{remark}
Recall that $\Noo$ is defined as the spectrum of $B_\infty$. 
If $\alpha:\Noo$ satisfies $\alpha(p_n) = 1$, then $\alpha(p_m) = 0$ for all $n\neq m$. 
Therefore, for each $n:\N$, there is an unique map $\chi_n$ with $\chi_n(p_n) = 1$. 
There is also the point $\chi_\infty : \Noo$ which is unique 
with the property that $ \chi_\infty(p_n) = 0$ for all $n:\N$. 
We will call decidable subsets of $\Noo$ finite/cofinite iff their corresponding elements of $B_\infty$ are. 
\end{remark}
\begin{lemma}\label{FiniteDecidableSubsetsCharacterization}
  Finite decidable subsets of $\Noo$ are of the form 
  $\{\chi_i | i \in I\}$ for some finite $I\subseteq \N$. 
\end{lemma}
\begin{proof}
  Let $d= \bigvee_{i\in I} p_i$. 
  Clearly whenever $i\in I$, we have $\chi_i(d) = 1$. 
%
  Now suppose $f:B_\infty \to 2$ is such that $f(d) = 1$. 
  Then $\bigvee_{i\in I}(f(p_i)) = 1$, hence it is not the case that $f(p_i) = 0$ for all $i\in I$. 
  Now as $I$ is finite and $f(p_i) = 0 \vee f(p_i) = 1$ for all $i\in I$, 
  there must exist some (necessarily unique) $i\in I$ with $f(p_i) = 1$. Hence $f = \chi_i$. 
%
  Thus $f(d) = 1$ iff there is some $i\in I$ with $f = \chi_i$. 
\end{proof}
\begin{corollary}\label{CoFiniteDecidableSubsetsCharacterization}
  Cofinite decidable subsets of $\Noo$ are of the form
  $\neg \{\chi_i | i \in I\}$ for $I\subseteq\N$ finite. 
\end{corollary}
\begin{proof}
  Let $D$ be a cofinite decidable subset. Then $\neg D$ is a finite decidable subset, 
  By the above lemma it follows that $\neg D = \{\chi_i | i\in I\}$. 
  As $\neg \neg D = D$, the result follows. 
\end{proof}
\begin{corollary}
 Any a decidable subset $D\subseteq\Noo$ is cofinite iff $\chi_\infty\in D$. 
\end{corollary}
\begin{proof}
  This follows from the observation that $\chi_\infty \in \neg \{\chi_i | i \in I\}$ for $I\subseteq \N$, 
  the observation that all decidable subsets are either finite or cofinite, 
  and the characterization of finite a cofinite decidable subsets in 
  \Cref{FiniteDecidableSubsetsCharacterization} and 
  \Cref{CoFiniteDecidableSubsetsCharacterization}.
\end{proof}
\begin{corollary}
  If $U\subseteq \Noo$ is open and $\chi_\infty \in U$, there exists some $n\in \N$ such that 
  $\{\chi_k | k\geq n\} \subseteq U$. 
\end{corollary}
\begin{proof}
  If $U$ is open, by \Cref{StoneOpenSubsets}, it is a countable union of decidable subsets. 
  One of these must contain $\chi_\infty$, hence be cofinite and 
  of the form $\neg \{ \chi_i | i \in I\}$ for some finite $I\subseteq \N$.
  As $I$ is finite, there is some $n:\N $ with $n>i$ for all $i\in I$. 
  For all $k\geq n$, we have that $k\notin I$, hence $\chi_k \in \neg \{\chi_i | i \in I\}\subseteq U$ as required. 
\end{proof}



%
%\begin{lemma}
%  For all decidable subsets $D:\Noo\to 2$,
%  with $D$ non-empty, there exists some $n:\N$ with $\chi_n \in D$. 
%\end{lemma}
%\begin{proof}
%  We make a case distinction based on \Cref{AppendixCofiniteOrFinite}. 
%  \begin{itemize}
%    \item 
%      If $D$ corresponds to a finite $d:B_\infty$, but is non-empty, then 
%      $d=\bigvee_{i\in I} p_i$ for $I\subseteq \N$ finite and non-empty. 
%      If $I$ is finite (as in \Cref{dfnFinite}) and non-empty, 
%      $I\simeq Fin_k$ for some $k\neq 0$. 
%      In particular, there is a map $1 \to I$,
%      hence a term $i:I$. 
%      Then $\chi_i(d) = 1$, hence $\chi_i \in D$. 
%    \item 
%      If $D$ corresponds to some cofinite $d:B_\infty$, we have 
%      $d = \bigvee_{i\in I} \neg p_i$ for some $I\subseteq \N$ finite. 
%      Then there is some 
%\end{proof}
%




\section{Cocompleteness of $\Boole$}
\rednote{TODO, is $\Boole$ closed under countable limits? 
  It has finite colimits, as it has pushouts and initial object.
  It should also have sequential colimits (TODO). 
  Is a countable coproduct the sequential colimit of it's initial finite coproducts? 
}
\begin{lemma}\label{BoolePushouts}
  Countably presented Boolean algebras are closed under pushout. 
\end{lemma} 
\begin{proof}
  Let $A,B,C:\Boole$, and suppose $f:A\to B, g:A \to C$ are Boolean morphisms. 
  Let $G_A, G_B,G_C$ be the underlying countable sets of generators for $B,C$ and 
  let $R_A,R_B,R_C$ be the underlying countable sets of relations. 
  Consider $P$ the Boolean algebra generated by $G_B\sqcup G_C$ under the relations 
  $R_B\cup R_C \cup F$ where $F$ is the set of expressions $f(a)-g(a), a\in G_A$.
  
  Note that as the generators of $B$ are included in those of $P$, 
  and all relations of $B$ are included in those of $P$, there is a map $h:B\to P$. 
  Similarly there is a map $i:C\to P$. 
  We now claim that the following is a pushout square:
  \begin{equation}\begin{tikzcd}
    A \arrow[r,"f"] \arrow[d,"g"] & B \arrow[d,"h"]\\
    C \arrow[r,"i"] & P
  \end{tikzcd}\end{equation}  
  Suppose $\beta:B \to X, \gamma:C\to X$ are such that $\beta\circ f = \gamma \circ h$. 
  $\beta,\gamma$ then induce maps on the generators of $P$. 
  These maps respect $F$ as $\beta\circ f=\gamma\circ h$, and they must respect $R_B,R_C$ as they are maps out of $B,C$. 
  Therefore, $\beta,\gamma$ induce a map $e:P\to X$, such that 
  $e(b) = \beta(b)$ for $b:G_B$ and $e(c)=\gamma(c)$ for $c:G_C$. 
  Furthermore, any map $P\to X$ with this property must agree with $e$ on all the generators of $P$, 
  and therefore equal $e$. Thus $e$ is the unique extension $P\to X$. 
  Thus $P$ the above square is indeed a pushout. 
\end{proof}

For some proofs in this paper, 
\rednote{(right now the counter's at two)}
we'd like a very concrete description of the fiber of a map of Stone spaces. 
The following construction turns out to be particularly useful. 
\begin{lemma}\label{FiberConstruction}
  Let $A,B:\Boole$, let $G$ be an explicit countable set of generators for $A$, and let 
  $f:A \to B, x:A\to 2$. 
  Define the countable set 
  \begin{equation}
    G' = \{a | a\in G, x(a) = 0\} \cup \{\neg a | a \in G, x(a) = 1\}
  \end{equation} 
  For $R = f(G')$,
%  Then we can construct a countable set $R\subseteq B$ such that 
  the pushout of $f$ and $x$ is given by $B/R$. 
\end{lemma}  
\begin{proof}
We consider the following pullback in the category of Stone spaces:
  \begin{equation}\begin{tikzcd}
    \sum\limits_{y:Sp(B)} y\circ f = x \arrow[d] \arrow[r] \arrow["\lrcorner"{pos=0.125}, phantom, dr] 
    & \top \arrow[d,"x"]\\
    Sp(B) \arrow[r,"(\cdot) \circ f"] & Sp(A)
  \end{tikzcd}  \end{equation}
Dual to this square, we have the following pushout in the category of Boolean algebras,
where $Sp(P) \simeq  (\sum\limits_{y:Sp(B)} y \circ f = x)$:
  \begin{equation}\begin{tikzcd}
    A \arrow[d,"x"'] \arrow[r,hook,"f"] \arrow[rd,phantom,"\ulcorner"{pos=0.125}] & B\arrow[d]\\
    2 \arrow[r] & P
  \end{tikzcd}\end{equation} 
  Following \Cref{BoolePushouts}, 
  the pushout $P$ is given by $B/R$ with $R$ the relations $f(a) -x(a)$ 
  where $a$ ranges over the generators of $A$.
  Note that $x(a) \in \{0,1\}$. 
  If $x(a)=0$, then $f(a)-x(a) = f(a)$, 
  and if $x(a) = 1$, then $f(a) -x(a) = \neg f(a) = f(\neg a)$. 
  So we can define the subset $G'\subseteq A$ given by 
  \begin{equation}
    G' = \{a | a\in G, x(a) = 0\} \cup \{\neg a | a \in G, x(a) = 1\}
  \end{equation} 
  $G'$ is in bijection with $G$, hence countable. 
  Furthermore, $x(g) = 0$ for all $g\in G'$. 
  And $R = f(G')$.
\end{proof}





%\begin{lemma}\label{BooleCoEqualizers}
%  Countably presented Boolean algebras are closed under coequalizers.
%\end{lemma}
%\begin{proof}
%  Let $f,g:A\to B$ be Boolean morphisms.
%  Define $C = B/R$, where $R$ is given by the relations $fa-ga,~a\in G_A$, for $G_A$ the set of generators of $A$.
%  Suppose that we have a map $x:B\to D$ with $xf = gf$. Then $x$ respects $R$, and thus defines a map $y:C \to D$. 
%  Furthermore, any map $C\to D$ extending $x$ agrees with $y$ on the generators of $C$, 
%  and is thus equal to $y$. Therefore $C$ is the coequalizer of $f,g$. 
%\end{proof}


%%
%%\begin{corollary}\label{CoCompletenessBoole}
%%  The category of countably presented Boolean algebras contains all finite colimits. 
%%\end{corollary}
%%\begin{proof}
%%  Recall that $\Boole$ has an initial object given by $2$. 
%%  By \Cref{BoolePushouts}, 
%%%  it is therefore closed under coproducts. 
%%%  By \Cref{BooleCoEqualizers}, 
%%  it follows that $\Boole$ contains all finite colimits. 
%%\end{proof}


\section{Some notes on our axioms}
\label{NotesOnAxioms}
\input{Appendix/NotesOnAxioms}
\section{Countability}\label{CountabilityDiscussion}
In the system presented in this paper, 
one of the fundamental building blocks are countably presented Boolean algebras. 
There are several definitions of countable, which are not necissarily constructively equivalent. 

\begin{definition}
  A type $T$ is enumerable iff there exists a surjection $\N \to 1 + T$. 
\end{definition}
\begin{definition}\label{dfnFinite}
  A type $T$ is finite if there exists some $k:\N$ with $T\simeq Fin_k$. 
\end{definition}
\begin{definition}
  A type $T$ is strongly countable if
  $T$ is finite or merely isomorphic to $\N$.
\end{definition}
\begin{definition}
  A type $T$ is subcountable iff it is merely isomorphic to a decidable subset of $\N$. 
\end{definition}


\begin{lemma}
  Every strongly countable type is subcountable. 
\end{lemma}
\begin{proof}
  Note that $Fin_k$ and $\N$ are both isomorphic to a decidable subset of $\N$. 
\end{proof}
\begin{lemma}
  Every subcountable type is enumerable. 
\end{lemma}
\begin{proof}
  For $A\subseteq \N$ decidable, define $f:\N \to 1 + \Sigma_{n:\N} A(n)$ by 
  $$
  f(n) = 
  \begin{cases}
    inl(*) \text{ if } \neg A(n)\\
    inr(n) \text{ if } A(n)
  \end{cases}
  $$
\end{proof} 


\begin{lemma}\label{OpenSubsetNAreSubCountable}
  Any open subset of $\N$ is subcountable. 
\end{lemma} 
\begin{proof}
  Let $A:\N \to \Open$. 
  By countable choice, there exists a map $\alpha_{(\cdot)}:\N \to \Noo$ such that 
  $\exists_{m:\N} \alpha_n(m) = 0 \leftrightarrow A (n)$. 
  Define $B\subseteq \N \times \N$ by 
  $B(m,n) = (\alpha_{n}(m) = 0)$. 
  Note that $A(n) \leftrightarrow || \Sigma_{m:\N } B(m,n) ||$
\end{proof}

\begin{lemma}\label{OpenSubsetEnumerableAreEnumerable}
  Any open subset of an enumerable type is enumerable. 
\end{lemma}
\begin{proof}
  Let $A$ be enumerable and let $P:A \to Open$.
  We will show that $\Sigma_{a:A} P a$ is enumerable. 
  Let $s:\N \to 1 + A$ surjective. 
  Define $s':\N \to Open$ by 
  $$
  s'(n) = 
  \begin{cases}
    \bot \text { if } s(n) = inl(*)\\
    P(a) \text { if } s(n) = inr(a)
  \end{cases}
  $$ 
  By countable choice, we get a map 
  $\alpha_{(\cdot)}: \N \to 2^\N$  such that 
  $(\exists_{m:\N} \alpha_n(m) = 0) \leftrightarrow s'(n)$. 
  Note that $\alpha_n(m) = 1$ iff 
  $s'(n)$ which happens iff $s(n) = inr(a_n)$ for some $a_n:A$ with $P(a_n)$. 
  Therefore, we can define 
  $z:\N \times \N \to 1 + \Sigma_{a:A} P a$ by 
  \begin{equation}
    z(m,n) = 
    \begin{cases}
      inl(*) \text{ if } \alpha_{n}(m) = 0 \\
      a_n  \text{ if } \alpha_{n}(m) = 1 \text{ and $a_n$ as above}
    \end{cases}
  \end{equation}
  Note that if $a:A$ satisfies $P(a)$, there is some $n:\N$ with $s(n) = inr(a)$. 
  And as $P(a)$, there exists some $m:\N$ with $\alpha_n(m) = 1$. 
  Hence $z(m,n) = a$. 
  Thus $z$ is surjective. 
  As $\N \times \N \simeq \N$, we conclude that 
  $\sum_{a:A} P a$ is enumerable. 
\end{proof}

\begin{lemma}
  Any open subset of $\N$ is subcountable. 
\end{lemma}
\begin{proof}
  Let $A:\N \to \Open$. 
  By countable choice, we get a map $\alpha_{\cdot}: \N \to \Noo$ such that 
  $A(n) \leftrightarrow \Sigma_{m:\N} \alpha_n (m) = 1$. 
  Define $B:\N \times \N \to 2$ by $B(m,n) = \alpha_n(m)$. 
  We then have a bijection $\Sigma_{n:\N} A(n) \to \Sigma_{(n,m) : \N \times \N} B(m,n)$ sending 
  $(n,(m,p))$ to $(n,m,p)$.
\end{proof}



\begin{lemma}\label{OpenSubsetOfNNotDecidable}
  It is not the case that for every $P:\N \to \Open$, 
  $||\Sigma_{n:\N} P(n)||$ is decidable.
  % The subset being decidable could be interpreted as 
  %    that P(n) is decidable for all $n:\N$,
  % or that \Sigma_{n:\N} P(n) + \neg \Sigma_{n:\N} P(n) 
  % or that || \Sigma_{n:\N} P(n) || + \neg ||\Sigma_{n:\N} P(n) ||
\end{lemma}
\begin{proof}
  For $p$ any open proposition and $P(n) = p$ constantly, we have 
  $||\Sigma_{n:\N} P(n)||\leftrightarrow p$. 
  As not every open proposition is decidable (\Cref{rmkOpenClosedNegation}), 
  not every $||\Sigma_{n:\N} P(n)||$ is decidable. 
\end{proof}

\begin{lemma}\label{StronglyCountableTruncationDecidable}
  For every strongly countable type $A$, $||A||$ is decidable. 
\end{lemma}
\begin{proof}
  For a proposition, being decidable is a proposition. 
  Hence we may untruncate the definition of strongly open. 
  If $A \simeq \N$ or $A\simeq Fin_k$ for $k\neq 0$, we have $||A||$. 
  If $A \simeq Fin_0$, then $\neg ||A||$. 
\end{proof}

\begin{corollary}
  Not every enumerable type is strongly countable.
\end{corollary}
\begin{proof}
  If every enumerable type is strongly countable, 
  by \Cref{OpenSubsetEnumerableAreEnumerable}, every open subset of open subsets of $\N$ is strongly countable. 
  By \Cref{StronglyCountableTruncationDecidable}, the truncation of the corresponding type is decidable, which 
  contradicts\Cref{OpenSubsetOfNNotDecidable}.
\end{proof}

\begin{remark}
  Every enumerably represented Boolean algebra has an enumerable underlying set. 
  and every enumerable Boolean algebra is enumerably represented. 
\end{remark}

%\input{ColimitRepresentation.tex}


%
%\section{Topology}
%
\subsection{Closed subtypes}

\begin{definition}%
  \label{closed-proposition}\label{closed-subtype}
  \begin{enumerate}[(a)]
  \item
    A \notion{closed proposition} is a proposition
    which is merely of the form $x_1 = 0 \land \dots \land x_n = 0$
    for some elements $x_1, \dots, x_n \in \A$.
  \item
    Let $X$ be a type.
    A subtype $U : X \to \Prop$ is \notion{closed}
    if for all $x : X$, the proposition $U(x)$ is closed.
  \item
    For $A$ a finitely presented $\A$-algebra
    and $f_1, \dots, f_n : A$,
    we set
    $V(f_1, \dots, f_n) \colonequiv
    \{\, x : \Spec A \mid f_1(x) = \dots = f_n(x) = 0 \,\}$.
  \end{enumerate}
\end{definition}

Note that $V(f_1, \dots, f_n) \subseteq \Spec A$ is a closed subtype
and we have $V(f_1, \dots, f_n) = \Spec (A/(f_1, \dots, f_n))$.

\begin{proposition}[using \axiomref{sqc}]%
  There is an order-reversing isomorphism of partial orders
  \begin{align*}
    \text{f.g.-ideals}(\A) &\xrightarrow{{\sim}} \Omega_{cl} \\
    I &\mapsto (I = (0))
  \end{align*}
  between the partial order of finitely generated ideals of $\A$
  and the partial order of closed propositions.
\end{proposition}

\begin{proof}
  For a finitely generated ideal $I = (x_1, \dots, x_n)$,
  the proposition $I = (0)$ is indeed a closed proposition,
  since it is equivalent to $x_1 = 0 \land \dots \land x_n = 0$.
  It is also evident that we get all closed propositions in this way.
  What remains to show is that
  \[ I = (0) \Rightarrow J = (0)
     \qquad\text{iff}\qquad
     J \subseteq I
     \rlap{\text{.}}
  \]
  For this we use synthetic quasicoherence.
  Note that the set $\Spec \A/I = \Hom_{\AAlg}(\A/I, \A)$ is a proposition
  (has at most one element),
  namely it is equivalent to the proposition $I = (0)$.
  Similarly, $\Hom_{\AAlg}(\A/J, \A/I)$ is a proposition
  and equivalent to $J \subseteq I$.
  But then our claim is just the equation
  \[ \Hom(\Spec \A/I, \Spec \A/J) = \Hom_{\AAlg}(\A/J, \A/I) \]
  which holds by \Cref{spec-embedding},
  since $\A/I$ and $\A/J$ are finitely presented $\A$-algebras
  if $I$ and $J$ are finitely generated ideals.
\end{proof}

\begin{lemma}[using \axiomref{sqc}]%
  \label{ideals-embed-into-closed-subsets}
  We have $V(f_1, \dots, f_n) \subseteq V(g_1, \dots, g_m)$
  as subsets of $\Spec A$
  if and only if
  $(g_1, \dots, g_m) \subseteq (f_1, \dots, f_n)$
  as ideals of $A$.
\end{lemma}

\begin{proof}
  The inclusion $V(f_1, \dots, f_n) \subseteq V(g_1, \dots, g_m)$
  means a map $\Spec (A/(f_1, \dots, f_n)) \to \Spec (A/(g_1, \dots, g_m))$
  over $\Spec A$.
  By \Cref{spec-embedding}, this is equivalent to
  a homomorphism $A/(g_1, \dots, g_m) \to A/(f_1, \dots, f_n)$,
  which in turn means the stated inclusion of ideals.
\end{proof}

\begin{lemma}[using \axiomref{loc}, \axiomref{sqc}, \axiomref{Z-choice}]%
  \label{closed-subtype-affine}
  A closed subtype $C$ of an affine scheme $X=\Spec A$ is an affine scheme
  with $C=\Spec (A/I)$ for a finitely generated ideal $I\subseteq A$.
\end{lemma}

\begin{proof}
  By \axiomref{Z-choice} and boundedness,
  there is a cover $D(f_1),\dots,D(f_l)$, such that
  on each $D(f_i)$, $C$ is the vanishing set of functions
  \[ g_1,\dots,g_n:D(f_i)\to \A\rlap{.} \]
  By \Cref{ideals-embed-into-closed-subsets},
  the ideals generated by these functions
  agree in $A_{f_i f_j}$,
  so by \Cref{fg-ideal-local-global},
  there is a finitely generated ideal $I\subseteq A$,
  such that $A_{f_i}\cdot I$ is $(g_1,\dots,g_n)$
  and $C=\Spec A/I$.
\end{proof}

\subsection{Open subtypes}

While we usually drop the prefix ``qc'' in the definition below,
one should keep in mind, that we only use a definition of quasi compact open subsets.
The difference to general opens does not play a role so far,
since we also only consider quasi compact schemes later.

\begin{definition}%
  \label{qc-open}
  \begin{enumerate}[(a)]
  \item A proposition $P$ is \notion{(qc-)open}, if there merely are $f_1,\dots,f_n:\A$,
    such that $P$ is equivalent to one of the $f_i$ being invertible.
  \item Let $X$ be a type.
    A subtype $U:X\to\Prop$ is \notion{(qc-)open}, if $U(x)$ is an open proposition for all $x:X$.
  \end{enumerate}
\end{definition}

\begin{proposition}[using \axiomref{loc}, \axiomref{sqc}]%
  \label{open-iff-negation-of-closed}
  A proposition $P$ is open
  if and only if
  it is the negation of some closed proposition
  (\Cref{closed-proposition}).
\end{proposition}

\begin{proof}
  Indeed, by \Cref{generalized-field-property},
  the proposition $\inv(f_1) \lor \dots \lor \inv(f_n)$
  is the negation of ${f_1 = 0} \land \dots \land {f_n = 0}$.
\end{proof}

\begin{proposition}[using \axiomref{loc}, \axiomref{sqc}]%
  \label{open-union-intersection}
  Let $X$ be a type.
  \begin{enumerate}[(a)]
  \item The empty subtype is open in $X$.
  \item $X$ is open in $X$.
  \item Finite intersections of open subtypes of $X$ are open subtypes of $X$.
  \item Finite unions of open subtypes of $X$ are open subtypes of $X$.
  \item Open subtypes are invariant under pointwise double-negation.
  \end{enumerate}
  Axioms are only needed for the last statement.
\end{proposition}

In \Cref{open-subscheme} we will see that open subtypes of open subtypes of a scheme are open in that scheme.
Which is equivalent to open propositions being closed under dependent sums.

\begin{proof}[of \Cref{open-union-intersection}]
  For unions, we can just append lists.
  For intersections, we note that invertibility of a product
  is equivalent to invertibility of both factors.
  Double-negation stability
  follows from \Cref{open-iff-negation-of-closed}.
\end{proof}

\begin{lemma}%
  \label{preimage-open}
  Let $f:X\to Y$ and $U:Y\to\Prop$ open,
  then the \notion{preimage} $U\circ f:X\to\Prop$ is open.
\end{lemma}

\begin{proof}
  If $U(y)$ is an open proposition for all $y : Y$,
  then $U(f(x))$ is an open proposition for all $x : X$.
\end{proof}

\begin{lemma}[using \axiomref{loc}, \axiomref{sqc}]%
  \label{open-inequality-subtype}
  Let $X$ be affine and $x:X$, then the proposition
  \[ x\neq y \]
  is open for all $y:X$.
\end{lemma}

\begin{proof}
  We show a proposition, so we can assume $\iota: X\to \A^n$ is a subtype.
  Then for $x,y:X$, $x\neq y$ is equivalent to $\iota(x)\neq\iota(y)$.
  But for $x,y:\A^n$, $x\neq y$ is the open proposition that $x-y\neq 0$.
\end{proof}

The intersection of all open neighborhoods of a point in an affine scheme,
is the formal neighborhood of the point.
We will see in \Cref{intersection-of-all-opens}, that this also holds for schemes.

\begin{lemma}[using \axiomref{loc}, \axiomref{sqc}]%
  \label{affine-intersection-of-all-opens}
  Let $X$ be affine and $x:X$, then the proposition
  \[ \prod_{U:X\to \Open}U(x)\to U(y) \]
  is equivalent to $\neg\neg (x=y)$.
\end{lemma}

\begin{proof}
  By \Cref{open-union-intersection}, $\neg\neg (x=y)$ implies $\prod_{U:X\to \Open}U(x)\to U(y)$.
  For the other implication,
  $\neg (x=y)$ is open by \Cref{open-inequality-subtype}, so we get a contradiction.
\end{proof}

We now show that our two definitions (\Cref{affine-open}, \Cref{qc-open})
of open subtypes of an affine scheme are equivalent.

\begin{theorem}[using \axiomref{loc}, \axiomref{sqc}, \axiomref{Z-choice}]%
  \label{qc-open-affine-open}
  Let $X=\Spec A$ and $U:X\to\Prop$ be an open subtype,
  then $U$ is affine open, i.e. there merely are $h_1,\dots,h_n:X\to \A$ such that
  $U=D(h_1,\dots,h_n)$.
\end{theorem}

\begin{proof}
  Let $L(x)$ be the type of finite lists of elements of $\A$,
  such that one of them being invertible is equivalent to $U(x)$.
  By assumption, we know
  \[\prod_{x:X}\propTrunc{L(x)}\rlap{.}\]
  So by \axiomref{Z-choice}, we have $s_i:\prod_{x:D(f_i)}L(x)$.
  We compose with the length function for lists to get functions $l_i:D(f_i)\to\N$.
  By \Cref{boundedness}, the $l_i$ are bounded.
  Since we are proving a proposition, we can assume we have actual bounds $b_i:\N$.
  So we get functions $\tilde{s_i}:D(f_i)\to \A^{b_i}$,
  by append zeros to lists which are too short,
  i.e. $\widetilde{s}_i(x)$ is $s_i(x)$ with $b_i-l_i(x)$ zeros appended.

  Then one of the entries of $\widetilde{s}_i(x)$ being invertible,
  is still equivalent to $U(x)$.
  So if we define $g_{ij}(x)\colonequiv \pi_j(\widetilde{s}_i(x))$,
  we have functions on $D(f_i)$, such that
  \[
    D(g_{i1},\dots,g_{ib_i})=U\cap D(f_i)
    \rlap{.}
  \]
  By \Cref{affine-open-trans}, this is enough to solve the problem on all of $X$.
\end{proof}

This allows us to transfer one important lemma from affine-opens to qc-opens.
The subtlety of the following is that while it is clear that the intersection of two
qc-opens on a type, which are \emph{globally} defined is open again, it is not clear,
that the same holds, if one qc-open is only defined on the other.

\begin{lemma}[using \axiomref{loc}, \axiomref{sqc}, \axiomref{Z-choice}]%
  \label{qc-open-trans}
  Let $X$ be a scheme, $U\subseteq X$ qc-open in $X$ and $V\subseteq U$ qc-open in $U$,
  then $V$ is qc-open in $X$.
\end{lemma}

\begin{proof}
  Let $X_i=\Spec A_i$ be a finite affine cover of $X$.
  It is enough to show, that the restriction $V_i$ of $V$ to $X_i$ is qc-open.
  $U_i\colonequiv X_i\cap U$ is qc-open in $X_i$, since $X_i$ is qc-open.
  By \Cref{qc-open-affine-open}, $U_i$ is affine-open in $X_i$,
  so $U_i=D(f_1,\dots,f_n)$.
  $V_i\cap D(f_j)$ is affine-open in $D(f_j)$, so by \Cref{affine-open-trans},
  $V_i\cap D(f_j)$ is affine-open in $X_i$.
  This implies $V_i\cap D(f_j)$ is qc-open in $X_i$ and so is $V_i=\bigcup_{j}V_i\cap D(f_j)$.
\end{proof}

\begin{lemma}[using \axiomref{loc}, \axiomref{sqc}, \axiomref{Z-choice}]%
  \label{qc-open-sigma-closed}
  \begin{enumerate}[(a)]
  \item qc-open propositions are closed under dependent sums:
    if $P : \Open$ and $U : P \to \Open$,
    then the proposition $\sum_{x : P} U(x)$ is also open.
  \item Let $X$ be a type. Any open subtype of an open subtype of $X$ is an open subtype of $X$.
  \end{enumerate}
\end{lemma}

\begin{proof}
  \begin{enumerate}[(a)]
  \item Apply \Cref{qc-open-trans} to the point $\Spec \A$.
  \item Apply the above pointwise.
  \end{enumerate}
\end{proof}

\begin{remark}
  \Cref{qc-open-sigma-closed} means that
  the (qc-) open propositions constitute a \notion{dominance}
  in the sense of~\cite{rosolini-phd-thesis}.
\end{remark}

The following fact about the interaction of closed and open propositions
is due to David Wärn.

\begin{lemma}%
  \label{implication-from-closed-to-open}
  Let $P$ and $Q$ be propositions
  with $P$ closed and $Q$ open.
  Then $P \to Q$ is equivalent to $\lnot P \lor Q$.
\end{lemma}

\begin{proof}
  We can assume $P = (f_1 = \dots = f_n = 0)$
  and $Q = (\inv(g_1) \lor \dots \lor \inv(g_m))$.
  Then we have:
  \begin{align*}
    (P \to Q) &= \qquad
    \text{\Cref{generalized-field-property} for $g_1, \dots, g_m$}\\
    (P \to \lnot (g_1 = \dots = g_m = 0)) &= \\
    \lnot (f_1 = \dots = f_n = g_1 = \dots = g_m = 0) &= \qquad
    \text{\Cref{generalized-field-property} for $f_1, \dots, f_n, g_1, \dots, g_m$}\\
    (\inv(f_1) \lor \dots \lor \inv(f_n) \lor \inv(g_1) \lor \dots \lor \inv(g_m) &= \qquad
    \text{\Cref{generalized-field-property} for $f_1, \dots, f_n$}\\
    \lnot P \lor Q &
  \end{align*}
\end{proof}


%
%\subsection{Compact Hausdorff}
%\begin{definition}
  A type $X$ is called a compact Hausdorff space if its identity types are closed propositions and there exists some $S:\Stone$ with a surjection $S\twoheadrightarrow X$. We write $\CHaus$ for the type of compact Hausdorff spaces.
\end{definition}

%This means that compact Hausdorff spaces are precisely quotients of Stone spaces by closed equivalence relations.

\subsection{Topology on compact Hausdorff spaces}

\begin{lemma}\label{CompactHausdorffClosed}
  Let $X:\CHaus$ with $S:\Stone$ and a surjective map $q:S\twoheadrightarrow X$.
  Then $A\subseteq X$ is closed if and only if it is the image of a closed subset of $S$ by $q$. 
\end{lemma}
\begin{proof}
%  If $A$ is closed, then it's pre-image under any map is also closed. 
%  In particular for $q:S\to X$ the quotient map, $q^{-1}(A)$ is closed. 
  As $q$ is surjective, we have $q(q^{-1}(A)) = A$.
  If $A$ is closed, so is $q^{-1}(A)$ and 
  hence $A$ is the image of a closed subset of $S$. 
  Conversely, let $B\subseteq S$ be closed. Then $x\in q(B)$ if and only if
   \[\exists_{t:S} (B(t) \wedge q(s) = x).\]
   Hence by \Cref{InhabitedClosedSubSpaceClosed}, $q(B)$ is closed. 
  % Define $A'\subseteq S$ by 
  %\[A'(s) = \exists_{t:S} (B(t) \wedge q(s) = q(t)).\]
  %Note that $B(t)$ and $q(s) = q(t)$ are closed. 
  %Hence by \Cref{InhabitedClosedSubSpaceClosed}, $A'$ is closed. 
  %Also $A'$ factors through $q$ as a map $A: X\to \Closed$.
  %Furthermore, $A'(s) \leftrightarrow (q(s)\in q(B))$. 
  %Hence $A=q(B)$. 
\end{proof}

The next two corollaries mean that compact Hausdorff spaces behave as finite sets for the purposes of unions/intersections of open/closed sets.

\begin{corollary}\label{InhabitedClosedSubSpaceClosedCHaus}
Assume given $X:\Chaus$ with $A\subseteq X$ closed. Then $\exists_{x:X} A(x)$ is closed, and equivalent to $A \neq \emptyset$. 
\end{corollary}

\begin{proof}
From \Cref{CompactHausdorffClosed} and \Cref{StoneClosedSubsets}, it follows that $A\subseteq X$ is closed if and only if it is the image of a map $T\to X$ for some $T:\Stone$. Then $\exists_{x:X} A(x)$ if and only $\propTrunc{T}$, which is closed by \Cref{TruncationStoneClosed}. Therefore $\exists_{x:X} A(x)$ is $\neg\neg$-stable and equivalent to $A \neq \emptyset$. 
  %If $A$ is closed, it follows from \Cref{InhabitedClosedSubSpaceClosed} that $\exists_{x:X} A(x)$ is closed as well, 
 % hence $\neg\neg$-stable, and equivalent to $A \neq \emptyset$. 
\end{proof}

%\begin{remark}\label{InhabitedClosedSubSpaceClosedCHaus}
%  Let $X:\Chaus$.
%  From \Cref{StoneClosedSubsets}, it follows that $A\subseteq X$ is closed if and only if it is the image of a map 
%  $T\to X$ for some $T:\Stone$. 
%  If $A$ is closed, it follows from \Cref{InhabitedClosedSubSpaceClosed} that $\exists_{x:X} A(x)$ is closed as well, 
%  hence $\neg\neg$-stable, and equivalent to $A \neq \emptyset$. 
%\end{remark}
%\begin{corollary}
%  For $X:\CHaus$ a subtype $A\subseteq X$ is closed iff it is the image of 
%  a map $T\to X$ for some $T:\Stone$. 
%\end{corollary}
%\begin{proof}
%  Directly from the above and \Cref{StoneClosedSubsets}.
%\end{proof}
%WhyDidWeNeedThis%\begin{remark}
%WhyDidWeNeedThis%  It is not the case that every closed subset of a compact Hausdorff space can be written 
%WhyDidWeNeedThis%  as countable intersection of decidable subsets. 
%WhyDidWeNeedThis%  In \Cref{UnitInterval}, we shall introduce the unit interval $[0,1]$ as a compact Hausdorff space with many closed 
%WhyDidWeNeedThis%  subsets, but only two decidable subsets. 
%WhyDidWeNeedThis%  In \Cref{ConnectedComponent}, we shall actually see that whenever every singleton of a compact Hausdorff space $X$
%WhyDidWeNeedThis%  can be written as countable intersection of decidable subsets, $X$ is Stone. 
%WhyDidWeNeedThis%  \rednote{Actually, we'll see that $\Sp(2^X)$ and $X$ are bijective sets, 
%WhyDidWeNeedThis%    which only implies that $X$ is Stone if $2^X:\Boole$, but this depends on our definition of countable, 
%WhyDidWeNeedThis%see \Cref{CountabilityDiscussion}}
%WhyDidWeNeedThis%\end{remark}


\begin{corollary}\label{AllOpenSubspaceOpen}
  Assume given $X:\Chaus$ with $U\subseteq X$ open. Then $\forall_{x:X} U(x)$ is open. 
\end{corollary}
%\begin{proof}
%  As $U$ is open, $\neg U$ is closed. 
%  So $\exists_{x:X} \neg U(x)$ is closed by \Cref{InhabitedClosedSubSpaceClosedCHaus}. 
%  Using \Cref{rmkOpenClosedNegation}, it follows that 
%  $\neg (\exists_{x:X} \neg U(x))$ is open. 
%  Furthermore, it is equivalent to $\forall_{x:X} \neg \neg U(x)$, 
%  which is equivalent to $\forall_{x:X} U(x)$ by \Cref{rmkOpenClosedNegation}.
%\end{proof}

Next lemma means that compact Hausdorff space are not too far from being compact in the classical sense.

\begin{lemma}\label{CHausFiniteIntersectionProperty}
  Given $X:\Chaus$ and $C_n:X\to \Closed$ closed subsets such that $\bigcap_{n:\N} C_n =\emptyset$, there is some $k:\N$ 
  with $\bigcap_{n\leq k} C_n  = \emptyset$. 
\end{lemma}
\begin{proof}
  By \Cref{CompactHausdorffClosed} it is enough to prove the result when $X$ is Stone, and by \Cref{StoneClosedSubsets} we can assume $C_n$ decidable.
  So assume 
  $X=\Sp(B)$ and $c_n:B$ such that
  \[C_n = \{x:B\to 2\ |\ x(c_n) = 0\}.\]
  Then we have that
  \[\Sp(B/(c_n)_{n:\N})%\ |\ n:\N\}) 
  \simeq \bigcap_{n:\N} C_n = \emptyset .\]
  Hence 
%  $0=_{B/(\neg c_n)_{n:\N}}1$ 
  $0=1$ in $B/(c_n)_{n:\N}$ %\ |\ n:\N\}$, 
  and there is some $k:\N$ with 
  $\bigvee_{n\leq k} c_n = 1$, which means that
  \[\emptyset = \Sp(B/(c_n)_{n\leq k}) %\ |\ n\leq k\})  
  \simeq \bigcap_{n\leq k} C_n \]
  as required.
\end{proof}

\begin{corollary}\label{ChausMapsPreserveIntersectionOfClosed}
  Let $X,Y:\CHaus$ and $f:X \to Y$. 
  Suppose $(G_n)_{n:\N}$ is a decreasing sequence of closed subsets of $X$. 
  Then $f(\bigcap_{n:\N} G_n) = \bigcap_{n:\N}f(G_n)$. 
\end{corollary}
\begin{proof}
  It is always the case that $f(\bigcap_{n:\N} G_n) \subseteq \bigcap_{n:\N} f(G_n)$. 
  For the converse direction, suppose that $y \in f(G_n)$ for all $n:\N$. 
  We define $F\subseteq X$ closed by $F=f^{-1}(y)$. 
  Then for all $n:\N$ we have that $F\cap G_n$ is %merely inhabited and therefore 
  non-empty. 
  By \Cref{CHausFiniteIntersectionProperty} this implies that $\bigcap_{n:\N}(F\cap G_n) \neq \emptyset$. 
  By \Cref{InhabitedClosedSubSpaceClosedCHaus},  we have that %and \Cref{rmkOpenClosedNegation}, 
  $\bigcap_{n:\N} (F\cap G_n)$ is merely inhabited. Thus $y\in f(\bigcap_{n:\N} G_n)$ as required. 
\end{proof}

\begin{corollary}\label{CompactHausdorffTopology}
Let $A\subseteq X$ be a subset of a compact Hausdorff space and $p:S\twoheadrightarrow X$ be a surjective map with $S:\Stone$. Then $A$ is closed (resp. open) if and only if there exists a sequence $(D_n)_{n:\N}$ of decidable subsets of $S$ such that $A = \bigcap_{n:\N} p(D_n)$ (resp. $A = \bigcup_{n:\N} \neg p(D_n)$).
%\begin{itemize}
%  \item $A$ is closed iff %if and only if 
%    it can be written as $\bigcap_{n:\N} p(D_n)$
%for some $D_n\subseteq S$ decidable. 
%  \item $A$ is open iff %if and only if 
%    it can be written as $\bigcup_{n:\N} \neg p(D_n)$
%for some $D_n\subseteq S$ decidable.
%\end{itemize}  
\end{corollary}
\begin{proof}
  The characterization of closed subsets follows from characterization (ii) in \Cref{StoneClosedSubsets}, 
  \Cref{CompactHausdorffClosed} 
  and \Cref{ChausMapsPreserveIntersectionOfClosed}. 
%  The characterization of open sets 
  To deduce the characterization of open subsets we use \Cref{rmkOpenClosedNegation} and
  \Cref{ClosedMarkov}.
\end{proof}
%
\begin{remark}
  For $S:\Stone$, there is a surjection $\N\twoheadrightarrow 2^S$. 
  It follows that for any $X:\CHaus$ there is a surjection from $\N$ to a basis of $X$. 
  Classically this means that $X$ is second countable. 
\end{remark}
%It follows that compact Hausdorff spaces are second countable:
%\begin{corollary}
%  Any $X:\Chaus$ is has a topological basis which is countable.
%\end{corollary}
%\begin{proof}
%  By \Cref{CompactHausdorffTopology}, 
%  a basis is given by the images of the decidable subsets of some $S:\Stone$. 
%  By \cref{ODiscBAareBoole}, $2^S$ is 
%  overtly discrete so we have a surjection $\N\to 2^S$.
%  \end{proof}
%

Next lemma means that compact Hausdorff spaces are normal.

\begin{lemma}\label{CHausSeperationOfClosedByOpens}
 Assume $X:\CHaus$ and $A,B\subseteq X$ closed such that $A\cap B=\emptyset$. 
  Then there exist $U,V\subseteq X$ open such that $A\subseteq U$, $B\subseteq V$ and $U\cap V=\emptyset$. 
\end{lemma}
\begin{proof}
  Let $q:S\to X$ be a surjective map with $S:\Stone$.
  As $q^{-1}(A)$ and $q^{-1}(B)$ are closed, 
  by \Cref{StoneSeperated}, there is some $D:S \to 2$ such that
  $q^{-1}(A) \subseteq D$ and $q^{-1}(B) \subseteq \neg D$. 
  Note that $q(D)$ and $q(\neg D)$ are closed by \Cref{CompactHausdorffClosed}. 
%  We define $U = \neg q(\neg D) $%\cap \neg B$ 
%  and $V=\neg  q(D) $.%\cap \neg A$. 
  As $q^{-1}(A) \cap \neg D  =\emptyset$, we have that 
  $A\subseteq \neg q(\neg D):=U$. 
%  As $A\cap B = \emptyset$, we have that $A\subseteq \neg B$ so $A\subseteq U$.
%  Similarly $B\subseteq V$. 
  Similarly $B\subseteq \neg q(D):=V$. 
  Then $U$ and $V$ are disjoint because $\neg q(D)\cap \neg q(\neg D) = \neg (q(D)\cup q(\neg D)) = \neg X = \emptyset$.
\end{proof}


%
%
%%\begin{definition}
  A type $X$ is called a compact Hausdorff space if its identity types are closed propositions and there exists some $S:\Stone$ with a surjection $S\twoheadrightarrow X$. We write $\CHaus$ for the type of compact Hausdorff spaces.
\end{definition}

%This means that compact Hausdorff spaces are precisely quotients of Stone spaces by closed equivalence relations.

\subsection{Topology on compact Hausdorff spaces}

\begin{lemma}\label{CompactHausdorffClosed}
  Let $X:\CHaus$ with $S:\Stone$ and a surjective map $q:S\twoheadrightarrow X$.
  Then $A\subseteq X$ is closed if and only if it is the image of a closed subset of $S$ by $q$. 
\end{lemma}
\begin{proof}
%  If $A$ is closed, then it's pre-image under any map is also closed. 
%  In particular for $q:S\to X$ the quotient map, $q^{-1}(A)$ is closed. 
  As $q$ is surjective, we have $q(q^{-1}(A)) = A$.
  If $A$ is closed, so is $q^{-1}(A)$ and 
  hence $A$ is the image of a closed subset of $S$. 
  Conversely, let $B\subseteq S$ be closed. Then $x\in q(B)$ if and only if
   \[\exists_{t:S} (B(t) \wedge q(s) = x).\]
   Hence by \Cref{InhabitedClosedSubSpaceClosed}, $q(B)$ is closed. 
  % Define $A'\subseteq S$ by 
  %\[A'(s) = \exists_{t:S} (B(t) \wedge q(s) = q(t)).\]
  %Note that $B(t)$ and $q(s) = q(t)$ are closed. 
  %Hence by \Cref{InhabitedClosedSubSpaceClosed}, $A'$ is closed. 
  %Also $A'$ factors through $q$ as a map $A: X\to \Closed$.
  %Furthermore, $A'(s) \leftrightarrow (q(s)\in q(B))$. 
  %Hence $A=q(B)$. 
\end{proof}

The next two corollaries mean that compact Hausdorff spaces behave as finite sets for the purposes of unions/intersections of open/closed sets.

\begin{corollary}\label{InhabitedClosedSubSpaceClosedCHaus}
Assume given $X:\Chaus$ with $A\subseteq X$ closed. Then $\exists_{x:X} A(x)$ is closed, and equivalent to $A \neq \emptyset$. 
\end{corollary}

\begin{proof}
From \Cref{CompactHausdorffClosed} and \Cref{StoneClosedSubsets}, it follows that $A\subseteq X$ is closed if and only if it is the image of a map $T\to X$ for some $T:\Stone$. Then $\exists_{x:X} A(x)$ if and only $\propTrunc{T}$, which is closed by \Cref{TruncationStoneClosed}. Therefore $\exists_{x:X} A(x)$ is $\neg\neg$-stable and equivalent to $A \neq \emptyset$. 
  %If $A$ is closed, it follows from \Cref{InhabitedClosedSubSpaceClosed} that $\exists_{x:X} A(x)$ is closed as well, 
 % hence $\neg\neg$-stable, and equivalent to $A \neq \emptyset$. 
\end{proof}

%\begin{remark}\label{InhabitedClosedSubSpaceClosedCHaus}
%  Let $X:\Chaus$.
%  From \Cref{StoneClosedSubsets}, it follows that $A\subseteq X$ is closed if and only if it is the image of a map 
%  $T\to X$ for some $T:\Stone$. 
%  If $A$ is closed, it follows from \Cref{InhabitedClosedSubSpaceClosed} that $\exists_{x:X} A(x)$ is closed as well, 
%  hence $\neg\neg$-stable, and equivalent to $A \neq \emptyset$. 
%\end{remark}
%\begin{corollary}
%  For $X:\CHaus$ a subtype $A\subseteq X$ is closed iff it is the image of 
%  a map $T\to X$ for some $T:\Stone$. 
%\end{corollary}
%\begin{proof}
%  Directly from the above and \Cref{StoneClosedSubsets}.
%\end{proof}
%WhyDidWeNeedThis%\begin{remark}
%WhyDidWeNeedThis%  It is not the case that every closed subset of a compact Hausdorff space can be written 
%WhyDidWeNeedThis%  as countable intersection of decidable subsets. 
%WhyDidWeNeedThis%  In \Cref{UnitInterval}, we shall introduce the unit interval $[0,1]$ as a compact Hausdorff space with many closed 
%WhyDidWeNeedThis%  subsets, but only two decidable subsets. 
%WhyDidWeNeedThis%  In \Cref{ConnectedComponent}, we shall actually see that whenever every singleton of a compact Hausdorff space $X$
%WhyDidWeNeedThis%  can be written as countable intersection of decidable subsets, $X$ is Stone. 
%WhyDidWeNeedThis%  \rednote{Actually, we'll see that $\Sp(2^X)$ and $X$ are bijective sets, 
%WhyDidWeNeedThis%    which only implies that $X$ is Stone if $2^X:\Boole$, but this depends on our definition of countable, 
%WhyDidWeNeedThis%see \Cref{CountabilityDiscussion}}
%WhyDidWeNeedThis%\end{remark}


\begin{corollary}\label{AllOpenSubspaceOpen}
  Assume given $X:\Chaus$ with $U\subseteq X$ open. Then $\forall_{x:X} U(x)$ is open. 
\end{corollary}
%\begin{proof}
%  As $U$ is open, $\neg U$ is closed. 
%  So $\exists_{x:X} \neg U(x)$ is closed by \Cref{InhabitedClosedSubSpaceClosedCHaus}. 
%  Using \Cref{rmkOpenClosedNegation}, it follows that 
%  $\neg (\exists_{x:X} \neg U(x))$ is open. 
%  Furthermore, it is equivalent to $\forall_{x:X} \neg \neg U(x)$, 
%  which is equivalent to $\forall_{x:X} U(x)$ by \Cref{rmkOpenClosedNegation}.
%\end{proof}

Next lemma means that compact Hausdorff space are not too far from being compact in the classical sense.

\begin{lemma}\label{CHausFiniteIntersectionProperty}
  Given $X:\Chaus$ and $C_n:X\to \Closed$ closed subsets such that $\bigcap_{n:\N} C_n =\emptyset$, there is some $k:\N$ 
  with $\bigcap_{n\leq k} C_n  = \emptyset$. 
\end{lemma}
\begin{proof}
  By \Cref{CompactHausdorffClosed} it is enough to prove the result when $X$ is Stone, and by \Cref{StoneClosedSubsets} we can assume $C_n$ decidable.
  So assume 
  $X=\Sp(B)$ and $c_n:B$ such that
  \[C_n = \{x:B\to 2\ |\ x(c_n) = 0\}.\]
  Then we have that
  \[\Sp(B/(c_n)_{n:\N})%\ |\ n:\N\}) 
  \simeq \bigcap_{n:\N} C_n = \emptyset .\]
  Hence 
%  $0=_{B/(\neg c_n)_{n:\N}}1$ 
  $0=1$ in $B/(c_n)_{n:\N}$ %\ |\ n:\N\}$, 
  and there is some $k:\N$ with 
  $\bigvee_{n\leq k} c_n = 1$, which means that
  \[\emptyset = \Sp(B/(c_n)_{n\leq k}) %\ |\ n\leq k\})  
  \simeq \bigcap_{n\leq k} C_n \]
  as required.
\end{proof}

\begin{corollary}\label{ChausMapsPreserveIntersectionOfClosed}
  Let $X,Y:\CHaus$ and $f:X \to Y$. 
  Suppose $(G_n)_{n:\N}$ is a decreasing sequence of closed subsets of $X$. 
  Then $f(\bigcap_{n:\N} G_n) = \bigcap_{n:\N}f(G_n)$. 
\end{corollary}
\begin{proof}
  It is always the case that $f(\bigcap_{n:\N} G_n) \subseteq \bigcap_{n:\N} f(G_n)$. 
  For the converse direction, suppose that $y \in f(G_n)$ for all $n:\N$. 
  We define $F\subseteq X$ closed by $F=f^{-1}(y)$. 
  Then for all $n:\N$ we have that $F\cap G_n$ is %merely inhabited and therefore 
  non-empty. 
  By \Cref{CHausFiniteIntersectionProperty} this implies that $\bigcap_{n:\N}(F\cap G_n) \neq \emptyset$. 
  By \Cref{InhabitedClosedSubSpaceClosedCHaus},  we have that %and \Cref{rmkOpenClosedNegation}, 
  $\bigcap_{n:\N} (F\cap G_n)$ is merely inhabited. Thus $y\in f(\bigcap_{n:\N} G_n)$ as required. 
\end{proof}

\begin{corollary}\label{CompactHausdorffTopology}
Let $A\subseteq X$ be a subset of a compact Hausdorff space and $p:S\twoheadrightarrow X$ be a surjective map with $S:\Stone$. Then $A$ is closed (resp. open) if and only if there exists a sequence $(D_n)_{n:\N}$ of decidable subsets of $S$ such that $A = \bigcap_{n:\N} p(D_n)$ (resp. $A = \bigcup_{n:\N} \neg p(D_n)$).
%\begin{itemize}
%  \item $A$ is closed iff %if and only if 
%    it can be written as $\bigcap_{n:\N} p(D_n)$
%for some $D_n\subseteq S$ decidable. 
%  \item $A$ is open iff %if and only if 
%    it can be written as $\bigcup_{n:\N} \neg p(D_n)$
%for some $D_n\subseteq S$ decidable.
%\end{itemize}  
\end{corollary}
\begin{proof}
  The characterization of closed subsets follows from characterization (ii) in \Cref{StoneClosedSubsets}, 
  \Cref{CompactHausdorffClosed} 
  and \Cref{ChausMapsPreserveIntersectionOfClosed}. 
%  The characterization of open sets 
  To deduce the characterization of open subsets we use \Cref{rmkOpenClosedNegation} and
  \Cref{ClosedMarkov}.
\end{proof}
%
\begin{remark}
  For $S:\Stone$, there is a surjection $\N\twoheadrightarrow 2^S$. 
  It follows that for any $X:\CHaus$ there is a surjection from $\N$ to a basis of $X$. 
  Classically this means that $X$ is second countable. 
\end{remark}
%It follows that compact Hausdorff spaces are second countable:
%\begin{corollary}
%  Any $X:\Chaus$ is has a topological basis which is countable.
%\end{corollary}
%\begin{proof}
%  By \Cref{CompactHausdorffTopology}, 
%  a basis is given by the images of the decidable subsets of some $S:\Stone$. 
%  By \cref{ODiscBAareBoole}, $2^S$ is 
%  overtly discrete so we have a surjection $\N\to 2^S$.
%  \end{proof}
%

Next lemma means that compact Hausdorff spaces are normal.

\begin{lemma}\label{CHausSeperationOfClosedByOpens}
 Assume $X:\CHaus$ and $A,B\subseteq X$ closed such that $A\cap B=\emptyset$. 
  Then there exist $U,V\subseteq X$ open such that $A\subseteq U$, $B\subseteq V$ and $U\cap V=\emptyset$. 
\end{lemma}
\begin{proof}
  Let $q:S\to X$ be a surjective map with $S:\Stone$.
  As $q^{-1}(A)$ and $q^{-1}(B)$ are closed, 
  by \Cref{StoneSeperated}, there is some $D:S \to 2$ such that
  $q^{-1}(A) \subseteq D$ and $q^{-1}(B) \subseteq \neg D$. 
  Note that $q(D)$ and $q(\neg D)$ are closed by \Cref{CompactHausdorffClosed}. 
%  We define $U = \neg q(\neg D) $%\cap \neg B$ 
%  and $V=\neg  q(D) $.%\cap \neg A$. 
  As $q^{-1}(A) \cap \neg D  =\emptyset$, we have that 
  $A\subseteq \neg q(\neg D):=U$. 
%  As $A\cap B = \emptyset$, we have that $A\subseteq \neg B$ so $A\subseteq U$.
%  Similarly $B\subseteq V$. 
  Similarly $B\subseteq \neg q(D):=V$. 
  Then $U$ and $V$ are disjoint because $\neg q(D)\cap \neg q(\neg D) = \neg (q(D)\cup q(\neg D)) = \neg X = \emptyset$.
\end{proof}


%%\input{IntersectionOfClosedIsClosedCmptHausdorff}
%
%\subsection{Open propositions}
%\input{OvertlyDiscrete/FactorizationFin}
%
%\section{Analysis}
%
%\subsection{Convergence}
%\input{Convergence/convergenceClosed}
%\input{Convergence/ConvergenceExtension}
%
%\subsection{The interval}
%%The goal of this section is to define the interval $[-2,2]_\mathbb R$ as a scheme. 
We assume $\N, \mathbb Q$ have been defined in HoTT
with linear propostional order relations $<,\leq, > ,\geq$ playing nicely together 
and standard algebraic operations. 
From these, we can define the subtype $\mathbb Q_{>0}=\sum_{q : \mathbb Q} (q>0)$, 
and the absolute-value function $|\cdot|$ on $\mathbb Q$. 

\begin{definition}
  A pre-Cauchy sequence is a sequence of rational numbers $(q_n)_{n: \N}$ with $-2 \leq q_n \leq 2$ 
  for all $n:\N$
%  together with a term of type
  such that for every $\epsilon: \mathbb Q_{>0}$, we have an $N_\epsilon:\N$, 
  such that whenever $n,m \geq N_\epsilon$, we have 
\begin{equation}
%  \forall \epsilon : \mathbb Q_{>0} \Sigma N : \N \forall m,n : \N (m,n \geq N) \to 
  | q_n - q_m | \leq \epsilon
\end{equation} 
\end{definition}

\begin{definition}
Given two pre-Cauchy sequences $p = (p_n)_{n\in\N}, q=(q_n)_{n\in\N}$, 
we define the proposition $p \sim_C  q$ as 
%for all $\epsilon : \mathbb Q_{>0}$ there exists an $N :\N$ such that whenever $n \geq N$, we have
\begin{equation}
  p \sim_C q : = \forall (\epsilon : \mathbb Q_{>0} )\exists ( N :\N) \forall (n : \N) ((n \geq N) \to 
  (| p_n - q_n| \leq  \epsilon))
\end{equation}
\end{definition}
Note that $\sim_C$ defines an equivalence relation on pre-Cauchy sequences. 
\begin{definition}
We define the type of Cauchy sequences as the type of pre-Cauchy sequences quotiented by $\sim_C$. 
\end{definition}

%\begin{definition}
%  A binary sequence consists of an initial segment $I \subseteq \N$
%  and a function $x:I \to 2$. 
%If $I$ is (in)finite, we call the binary sequence (in)finite as well. 
%\end{definition} 
%
%For $x$ a finite binary sequence and $y$ any binary sequence, 
%we'll denote $(x,y)$ for their concatenation, 
%and $\overline x$ for the infinite sequence repeating $x$. 
%
Denote $T = \{-1,0,1\}$. 
\begin{lemma}
  $T^\N$ is a scheme. 
\end{lemma}
\begin{proof}
  Sketch: partition $2^\N$ as follows: 
  For $\alpha: 2^\N$, we'll make a sequence $\beta: T^\N$.
  consider for each $n$ the $n$'th block of 2 entries in $\alpha$
  if both are $0$, $\beta(n) = 0$. 
  If the first is $1$, $\beta(n) = -1$
  If first is $0$ and the second is $1$, then $\beta(n) = 1$. 
  This is a closed equivalence relation. 
\end{proof} 

Consider the relation $\sim_s$ on $T^{\N}$, 
such that for any finite binary sequence $x$, we have 
\begin{align}
  (x,1,\overline 0) &\sim_t (x ,0, \overline 1) \\
  (x,-1,\overline 0) &\sim_t (x ,0, \overline {-1})\\
  (x,1,\overline {-1}) &\sim_t (x , \overline 0) \\
  (x,-1,\overline {1}) &\sim_t (x , \overline 0) 
\end{align} 
\begin{lemma}
$\sim_t$ induces a closed equivalence relation on $2^\N$. 
\end{lemma}
\begin{proof}
  TODO
\end{proof} 

\begin{proposition}\label{propTernaryCauchy}
  $T^\N/ \sim_t$ is isomorphic to the type of Cauchy sequences. 
\end{proposition} 
\begin{definition}%Construction might be better than definition here, but WIP so who cares. 
  For $\alpha: T^\N$, define the rational sequence $tri(\alpha)$ by 
  \begin{equation} (tri(\alpha))_n :  = \sum\limits_{0 \leq i \leq n} \frac{\alpha(i)} { 2^{i}} \end{equation}  
  This sequence is pre-Cauchy with $N_\epsilon$ given by the first $n$ with $(\frac12)^n<\epsilon$. 
\end{definition}  
%
%  Also, whenever $\alpha\sim_t \beta$, we have 
%  $tri(\alpha) \sim_C tri(\beta)$. 
%  Therefore $tri$ induces a function from $T^\N / \sim_t$ to Cauchy sequences. 
\begin{definition}
  Given a pre-Cauchy sequence $p$, 
  we will define a $T$-sequence $\alpha  = c(p): T^\N$.
  Consider any $i:\N$, and suppose $\alpha(j)$ has been defined for $0 \leq j<i$. 
%
  Let $\epsilon_i = (\frac12)^{i+1}$. %Placeholder value.
  Define $N_i:= N_{\epsilon_i}$. %is such that for $n,m \geq N_i$, we have $|p_n - p_m| < \epsilon_i$. 
%
  Consider 
  \begin{equation}
    \widetilde p_i = p_N - \sum\limits_{0\leq j < i} \frac {\alpha(j)}{2^{j}}.
  \end{equation}
  As the order on $\mathbb Q$ is total, we can define 
  \begin{equation}
    \alpha(i) = \begin{cases}
    \phantom{-} 1  \text{ if } \widetilde p_i \geq    (\frac12)^{i} \\
    -1             \text{ if } \widetilde p_i \leq  - (\frac12)^{i} \\
    \phantom{-} 0 \text{ otherwise } 
    \end{cases} 
  \end{equation}  
\end{definition} 
We shall now prove the following four things: 
\begin{itemize}
  \item 
    $c(tri(\alpha)) \sim_t \alpha$ for any $\alpha: T^n$.
  \item 
    $tri(c(p)) \sim_C p$ for any pre-Caucy sequence $p$. 
  \item 
    Whenever $p \sim_C q$, we have $c(p)\sim_t c(q)$. 
  \item 
    Whenever $\alpha \sim_t \beta$, we have $tri(\alpha) \sim_C tri(\beta)$. 
\end{itemize}
It follows that $c$ and $tri$ are maps between Cauchy sequences and $T^\N /\sim_t$ 
which are each other's inverse, proving Proposition \ref{propTernaryCauchy}
\begin{lemma} $tri(c(p)) \sim_C p$ for any pre-Caucy sequence $p$. 
\end{lemma} 



\begin{proof}
  Let $\epsilon>0$ be given, consider $n:\N$ such that
  $(\frac12)^n < \epsilon$. 
  We claim that for $m\geq N_n$, we have that $|p_m- tri(c(p))_m| < \epsilon$. 

  By definition $p_{N_n} $  
\end{proof} 






%In this section we will introduce the unit interval $I$ as compact Hausdorff space. 
The definition is based on \cite{Bishop}. 
We will then calculate the cohomology of $I$. 
For a proof that the unit interval corresponds to the definition using Cauchy sequences, 
we refer to the appendix. 


\input{Interval/CauchySequences}
\input{Interval/BinaryClosedEquivalence}
\input{Interval/EquivalenceOfSims}
\input{Interval/Surjective}





\begin{theorem}
  The interval of Cauchy reals is isomorphic to $2^\N / \sim_t$. 
\end{theorem} 
\begin{proof}
  This follows from the fact that $b:2^\N$ is such that $\alpha\sim_n \beta$ iff $b(\alpha)\sim_t b(\beta)$. 
  and for every Cauchy real, there is a binary sequence being sent to it, so the composition of $b$ and the 
  quotient from Caucy sequences to Cauchy real is a surjection. 
\end{proof}

\begin{corollary}
  The interval is compact Hausdorff. 
\end{corollary}

%%\printindex
%
%\section{Directed Univalence}
%\input{DirectedUnivalence}
%
%\appendix
%\section{Appendix}
%\subsection{Rank of matrices}

\begin{definition}
A matrix is said to have rank $\leq n$ if all its $n+1$-minors are zero. It is said to have rank $n$ if it has rank $\leq n$ and does not have rank $\leq n-1$.
\end{definition}

Having a rank is a property of matrices, as a rank function defined on all matrices would allow to e.g. decide if an $r:R$ is invertible.

\begin{lemma}\label{rank-bloc-matrix}
Assume given a matrix $M$ of rank $n$ decomposed into blocks:
\[M = \begin{pmatrix}
P & Q  \\
R & S \\
\end{pmatrix}\]
Such that $P$ is square of size $n$ and invertible. Then we have:
\[S = RP^{-1}Q\]
\end{lemma}

\begin{proof}
By columns manipulation the matrix is equivalent to:
\[M = \begin{pmatrix}
P & Q  \\
0 & S - RP^{-1}Q \\
\end{pmatrix}\]
but equivalent matrices have the same rank so $S=RP^{-1}Q$.
\end{proof}

\begin{lemma}\label{rank-equivalent-definitions}
If a linear map $R^m \to R^n$ given by multiplication with $M$
has finite free kernel of rank $k$, then $M$ has rank $m-k$.
\end{lemma}

\begin{proof}
  Let $a_1,\dots,a_{k}$ be a basis for the kernel of $M$ in $R^m$, which we complete into a basis of $R^m$ via $b_{k+1},\dots,b_m$.
  By completing $Mb_{k+1},\dots, Mb_m$ to a basis of $R^n$, we get a basis where $M$ is written as:
\[\begin{pmatrix}
I_{m-k} & 0  \\
0 & 0 \\
\end{pmatrix}\]
so that $M$ has rank $m-k$.
\end{proof}

%\begin{definition}
%Two matrices $M,N$ are said equivalent if there are invertible matrices $P,Q$ such that $M = PNQ$.
%\end{definition}

%It is clear that equivalent matrices have the same rank.

%\begin{lemma}\label{rank-equivalent-definitions}
%Assume given a matrix:
%\[M : R^m\to R^k\]
%Then the following are equivalent:
%\begin{enumerate}[(i)]
%\item $M$ has rank $n$.
%\item The kernel of $M$ is equivalent to $R^{m-n}$.
%\item The image of $M$ is equivalent to $R^n$.
%\item $M$ is equivalent to the bloc matrix:
%\[\begin{pmatrix}
%I_n & (0)  \\
%(0) & (0) \\
%\end{pmatrix}\]
%\end{enumerate}
%\end{lemma}

%\begin{proof}
%\end{proof}

%
\printbibliography
%
\end{document}
