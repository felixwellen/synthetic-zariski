% latexmk -pdf -pvc main.tex
\documentclass{../util/zariski}


\title{Synthetic Stone Duality}

\begin{document}

\author{Felix Cherubini, Thierry Coquand, Freek Geerligs and Hugo Moeneclaey}

\maketitle

%\begin{abstract}
%In synthetic algebraic geometry (SAG) \cite{draft}, we study finitely presented algebras over a commutative ring. 
%In this work, we study countably presented Boolean algebras instead. 
%Where the finitely presented algebras over a commutative ring induce a Zariski topos, 
%%the opposite category of these 
%the countably presented Boolean algebras induce the topos of light condensed sets \cite{Scholze}. 
%\cite{draft} proposes an axiomatization of the Zariski topos in univalent homotopy type theory \cite{hott}. 
%In this work, we propose similar axioms, which we expect to be modelled by light condensed sets. 
%% Furthermore, spectra of countably presented Boolean algebras correspond to quotients of Cantor space
%% which is cool because reasons
%\end{abstract} 
%
\rednote{The following is a collection of notes on work in progress.}

\rednote{I'm cleaning up, not all results that were in older versions have a place yet.}
\tableofcontents

% Logic and Topology
% - Building blocks, explaining the types of Stone, Boole (and what we mean with countable)
% - Rules, stating the axioms and first consequences, including the omniscience principles
% - Topology on propositions, mentioning under what constructions open propositions are closed
% - Examples of closed/open propositions, explaining why Boole is discrete and Stone Hausdorff
% - Topology on Stone spaces, including the classification of closed spaces.
% - Compact Hausdorff spaces. 
% Directed Univalence
% - Tychonov
% - Directed univalence, link to Phoa
% Cohomology
% - Cohomology and the interval
% Appendix, 
%  - alternative formulations of axiom 2
%  - more details on technical constructions
%  - colimit presentation of countably presented Boolean algebras (I'm not sure where we actually use this)
%  - scott continuity instead of axiom 1 



%\section*{Introduction}
%This draft is empty so far.

\section{Stone types}

We will use \cite{draft}[Lemma 4.2.11]:

\begin{lemma}
  \label{closed-implies-open-to-or}
  Let $C$ be a closed proposition and $U$ be an open proposition,
  then $C\to U$ is equivalent to $\neg C \vee U$.
\end{lemma}

We will also need:

\begin{lemma}
  \label{commute-open-in-closed}
  Let $X$ be a scheme, $C\subseteq X$ a closed subtype and $U\subseteq C$ open.
  Then there is an open $\tilde{U}\subseteq X$ such that $\tilde{U}\cap C = U$.
\end{lemma}



\section{Stone duality}
\subsection{Statement of the axioms}
We always assume there is a commutative ring $R$.
Sometimes we will assume $R$ has additional properties, or, more generally,
axioms hold that involve $R$.
We will always mention which of these axioms are needed to prove each statement,
by listing the shorthands introduced in the axioms below.

\begin{axiom}[Loc]%
  \label{loc}\index{Loc}
  $R$ is a local ring.
\end{axiom}

\begin{axiom}[SQC]%
  \label{sqc}\index{sqc}
  For any finitely presented $R$-algebra $A$, the homomorphism
  \[ a \mapsto (\varphi\mapsto \varphi(a)) : A \to (\Spec A \to R)\]
  is an isomorphism of $R$-algebras.
\end{axiom}

\begin{axiom}[Z-choice]%
  \label{Z-choice}\index{Z-choice}
  Let $A$ be a finitely presented $R$-algebra
  and let $B : \Spec A \to \mU$ be a family of inhabited types.
  Then there merely exists
  a finite list of coprime elements $f_1, \dots, f_n \in A$
  together with dependent functions $s_i : \Pi_{x : D(f_i)} B(x)$.
  As a formula:
  \[ (\Pi_{x : \Spec A} \propTrunc{B(x)}) \to
     \propTrunc{ \Sigma_{n : \N} \Sigma_{f_1, \dots, f_n : A}
      ((f_1, \dots, f_n) = (1)) \times
      \Pi_i \Pi_{x : D(f_i)} B(x) }
     \rlap{.}
  \]
\end{axiom}

\subsection{First consequences}

\begin{proposition}[using \axiomref{sqc}]%
  \label{spec-embedding}
  For all finitely presented $R$-algebras $A$ and $B$ we have
  \[ (\Spec B \to \Spec A) =\Hom_{\Alg{R}}(A,B)\]
  -- where the equality is induced by exponentiation with $R$.
\end{proposition}

\begin{proof}
  TODO
\end{proof}

An important consequence, which may be called \notion{weak nullstellensatz}:

\begin{proposition}[using \axiomref{loc}, \axiomref{sqc}]%
  \label{weak-nullstellensatz}
  If $A$ is a finitely presented $R$-algebra,
  then we have $\Spec A=\emptyset$ if and only if $A=0$.
\end{proposition}

\begin{proof}
  If $\Spec A = \emptyset$
  then $A = R^{\Spec A} = R^\emptyset = 0$
  by (\axiomref{sqc}).
  If $A = 0$
  then there are no homomorphisms $A \to R$
  since $1 \neq 0$ in $R$ by (\axiomref{loc}).
\end{proof}

The following are originally proven in 
\cite{ingo-thesis}[Section 18]:

\begin{proposition}[using \axiomref{sqc}, \axiomref{loc}]%
  \label{nilpotence-double-negation}\label{non-zero-invertible}\label{generalized-field-property}
  
  \begin{enumerate}[(a)]
  \item An element $x:R$ is nilpotent,
    if and only if $\neg \neg (x=0)$.
  \item An element $x:R$ is invertible,
    if and only if $x\neq 0$.
  \item A vector $x:R^n$ is non-zero,
    if and only if one of its entries is invertible.
  \end{enumerate}
\end{proposition}

\subsection{Anti-equivalence of $\Boole$ and $\Stone$}
By \Cref{AxStoneDuality}, the map $\Sp$ is an embedding of $\Boole$ into any universe of types. 
We denote its image by $\Stone$. 

\begin{remark}\label{SpIsAntiEquivalence}
Stone spaces will take over the role of the affine schemes from \cite{draft}, 
so let us repeat some results here. 
Analogously to Lemma 3.1.2 of \cite{draft}, 
for $X:\Stone$, \Cref{AxStoneDuality} tells us that $X = \Sp(2^X)$.
%
Proposition 2.2.1 of \cite{draft} now says that 
$\Sp$ gives a natural equivalence 
\[
   \Hom(A, B) = (\Sp(B) \to \Sp(A))
\]
%Therefore $Sp$ is an embedding from $\Boole$ to any universe of types, and $\isSt$ is a proposition.
%
%Its image, 
By the above and Lemma 9.4.5 of \cite{hott}, 
the map $\Sp$ defines a dual equivalence of categories between $\Boole$ and $\Stone$.
In particular the spectrum of any colimit in $\Boole$ is the limit of 
the spectrum of the opposite diagram. 
\end{remark}
\begin{remark}\label{LocalChoiceSurjectionForm}
  \Cref{AxLocalChoice} can also be formulated as follows:
  Given $S:\Stone$ with $E,F$ arbitrary types, a map $f:S \to F$ and a 
  surjection $e:E \twoheadrightarrow F$, 
  there exists a Stone space $T$, a surjective map 
  $T\twoheadrightarrow S$ and an arrow $T\to E$ making the following diagram commute:
    \[\begin{tikzcd}
      T \arrow[d,dashed, two heads ] \arrow[r,dashed]&  E \arrow[d,""',two heads, "e"]\\
      S  \arrow[r, swap,"f"] & F
    \end{tikzcd}\]  
\end{remark}

\begin{lemma}\label{SpectrumEmptyIff01Equal}
  For $B:\Boole$, we have $0=_B1$ if and only if $\neg \Sp(B)$.
\end{lemma}
\begin{proof}
  If $0=_B1$, there is no map in $B\to 2$ preserving both $0$ and $1$, thus $\neg \Sp(B)$. 
  Conversely, if $\neg \Sp(B)$ then 
  $\Sp(B)=\bot$. Since $\bot$ is the spectrum of the trivial Boolean algebra and $\Sp$ is an embedding, we conclude that $B$ is the trivial Boolean algebra, hence $0=_B1$. 
\end{proof}

\begin{corollary}\label{LemSurjectionsFormalToCompleteness}
 For $S:\Stone$, we have that $\neg \neg S \to  \propTrunc{S}$
\end{corollary}
\begin{proof}
  Let $B:\Boole$ and suppose $\neg \neg \Sp(B)$. By \Cref{SpectrumEmptyIff01Equal} we have that $0\not=_B1$, therefore the morphism $2\to B$ is injective. By \Cref{SurjectionsAreFormalSurjections} the map $\Sp(B) \to \Sp(2)$ is surjective, thus $\Sp(B)$ is merely inhabited. 
\end{proof} 

%\begin{corollary}\label{MoreConcreteCompleteness}
%  By the above and propositional completeness, we have that $||\Sp(B)||$ iff $0\neq_B1$. 
%\end{corollary}



%SurjectionsFormalSurjections%We conclude this section on the anti-equivalence of Stone and $\Boole$ by a relating surjections to injections. 
%SurjectionsFormalSurjections%This theorem is actually equivalent to completeness of propositional logic, which we'll discuss in 
%SurjectionsFormalSurjections%\Cref{NotesOnAxioms}. 
%SurjectionsFormalSurjections%
%SurjectionsFormalSurjections%\begin{theorem}\label{FormalSurjectionsAreSurjections}
%SurjectionsFormalSurjections%  Let $f:A\to B$ be a map of countably presented Boolean algebras. 
%SurjectionsFormalSurjections%  If $f$ is injective, then the corresponding map $(\cdot) \circ f : \Sp(B) \to \Sp(A)$ is surjective. 
%SurjectionsFormalSurjections%\end{theorem}
%SurjectionsFormalSurjections%
%SurjectionsFormalSurjections%\begin{proof}
%SurjectionsFormalSurjections%  Assume $f$ injective and let $x:\Sp(A)$.
%SurjectionsFormalSurjections%  By \Cref{FiberConstruction}, we have that $\left(\sum\limits_{y:\Sp(B)} y\circ f = x \right) = \Sp(B/R) $
%SurjectionsFormalSurjections%  for $R=f(G)$ for some countable $G\subseteq A$ with $x(g) = 0$ for all $g\in G$. 
%SurjectionsFormalSurjections%  By propositional completeness and \Cref{SpectrumEmptyIff01Equal}, 
%SurjectionsFormalSurjections%  it's sufficient to show that $0\neq_{B/R}1$. 
%SurjectionsFormalSurjections%  Note that $0=_{B/R} 1$ iff 
%SurjectionsFormalSurjections%  $1 =_B \bigvee R_0$ for some $R_0\subseteq R$ finite. 
%SurjectionsFormalSurjections%  But then $$1 = \bigvee f(G_0) = f(\bigvee  G_0)$$ for some $G_0\subseteq G$ finite. 
%SurjectionsFormalSurjections%  And as $f$ is injective, $\bigvee G_0 = 1$. 
%SurjectionsFormalSurjections%  However, 
%SurjectionsFormalSurjections%  $$
%SurjectionsFormalSurjections%  x(\bigvee G_0) = 
%SurjectionsFormalSurjections%  x(\bigvee_{g\in G_0} g ) = \bigvee_{g \in G_0} x(g) = \bigvee_{g\in G_0} 0 = 0$$
%SurjectionsFormalSurjections%  And as $x(1) = 1$, we get a contradiction. Therefore $0\neq_{B/R} 1$ as required. 
%SurjectionsFormalSurjections%\end{proof}  
%SurjectionsFormalSurjections%The converse to the above theorem is true as well, regardless of propositional completeness:
%SurjectionsFormalSurjections%\begin{lemma}\label{SurjectionsAreFormalSurjections}
%SurjectionsFormalSurjections%If $f:A\to B$ is a map in $\Boole$ and $(\cdot) \circ f :\Sp(B) \to \Sp(A)$ is surjective, 
%SurjectionsFormalSurjections%$f$ is injective. 
%SurjectionsFormalSurjections%\end{lemma}
%SurjectionsFormalSurjections%\begin{proof}
%SurjectionsFormalSurjections%  Suppose precomposition with $f$ is surjective. 
%SurjectionsFormalSurjections%  Let $a:A$ be such that $f(a)= 0$. 
%SurjectionsFormalSurjections%  By assumption, for every $x:A\to 2$, there is a $y:B\to 2$ with $y\circ f = x$. 
%SurjectionsFormalSurjections%  Consequentely $x(a) = y(f(a)) = y(0) = 0$. 
%SurjectionsFormalSurjections%  So $x(a) = 0$ for every $x:\Sp(A)$. 
%SurjectionsFormalSurjections%  Thus $\Sp(A) = \Sp(A/\{a\})$, and as $Sp$ is an embedding, 
%SurjectionsFormalSurjections%  $A \simeq A/\{a\}$, and $a = 0$ in $A$. 
%SurjectionsFormalSurjections%  So whenever $f(a) = 0$, we have $a=0$. Thus $f$ is injective. 
%SurjectionsFormalSurjections%\end{proof}

\subsection{Principles of omniscience}
In constructive mathematics, we do not assume the law of excluded middle (LEM).
There are some principles called principles of omniscience that are weaker than LEM, which can be used to describe 
how close a logical system is to satisfying LEM.
References on these principles include \cite{HannesDiener, ReverseMathsBishop}.
In this section, we will show that two of them (MP and LLPO) hold, 
and one (WLPO) fails in our system.

\begin{theorem}[The negation of the weak lesser principle of omniscience ($\neg$WLPO)]\label{NotWLPO}
  \begin{equation}
    \neg \forall_{\alpha:2^\N} 
    ((\forall_{n:\N} \alpha(n) = 0 ) \vee \neg (\forall_{n:\N} \alpha(n) = 0))
  \end{equation}
%  We cannot decide for general $\alpha:2^\N$, whether $\forall_{n:\mathbb N} \alpha(n) = 0$.
%  It is not the case that the statement %There is no method which given $\alpha:2^\mathbb N$ decides whether 
%  $\forall_{n:\mathbb N} \alpha(n) = 0$ is decidable for general $\alpha:2^\mathbb N$. 
\end{theorem}
\begin{proof}
%  Such a decision method is a function 
  Let $f:2^\mathbb N \to 2$ such that 
  $f(\alpha) = 0$ iff $\forall_{n:\mathbb N} \alpha (n)= 0$. 
  By \Cref{AxStoneDuality}, there is some $c:C$ with 
  $f(\alpha) = 0 \leftrightarrow \alpha(c) = 0$. 
  We can express $c$ using finitely many generators $(g_n)_{n\leq N}$. 
  Now consider $\beta,\gamma:2^\N$ given by 
  $\beta(g_n) = 0$ for all $n:\mathbb N$ and
  $\gamma(g_n) = 0$ iff $n\leq N$. 
  As $\beta, \gamma$ are equal on $(g_n)_{n\leq N}$, we have $\beta(c) = \gamma(c)$. 
  However, $f(\beta) = 0$ and $f(\gamma) = 1$, giving a contradiction as required. 
%  We thus have a contradiction, thus a decision method as required doesn't exist. 
\end{proof}

The following result is due to David W\"arn:
\begin{theorem}[Markov's principle (MP)]\label{MarkovPrinciple}
  For $\alpha:\Noo$, we have that 
  \begin{equation}
    (\neg (\forall_{n:\mathbb N} \alpha (n)= 0)) \to \Sigma_{n:\mathbb N} \alpha (n)= 1
  \end{equation}
\end{theorem}
\begin{proof}
  By \Cref{ClosedPropAsSpectrum}, we have that $\neg(\forall_{n:\N} \alpha(n) = 0)$ implies that 
  $Sp(2/(\alpha(n))_{n:\N}$ is empty. 
%  We will show that the spectrum of $2/(\alpha(n))_{n:\N}$ is empty. 
%  Suppose $x:2\to 2$, if  $x(\alpha(n)) = 0$, we get $\alpha(n) \neq 1$. 
%  Thus if $\neg (\forall_{n:\N} \alpha(n) = 0$, we have $\neg Sp(2/(\alpha(n))_{n:\N})$.
  Hence $2/(\alpha(n))_{n:\N}$ is trivial by \Cref{SpectrumEmptyIff01Equal}. 
  Then there is a finite subset $N_0\subseteq \N$ with $\bigvee_{i:N_0} \alpha(i) = 1$. 
  As $\alpha(i) \in \{0,1\}$ and $\alpha(i) = 1$ for at most one $i:\N$, 
  there exists an unique $n\in\mathbb N$ with $\alpha(n) = 1$. 
%  Assume $\neg (\forall_{n:\mathbb N} \alpha (n)= 0)$.
%  It is sufficient to show that $2/\{\alpha(n)|n\in\N\}$ is the trivial Boolean algebra. 
%  It will then follow that there is a finite subset $N_0\subseteq \N$ 
%  with $\bigvee_{i:N_0} \alpha(i) = 1$.
%  As $\alpha(i) \in \{0,1\}$ and $\alpha(i) = 1$ for at most one $i$, it then follows that 
%  there exists an unique $n\in\mathbb N$ with $\alpha(n) = 1$. 
%%
%  To show that $2/\{\alpha(n)|n\in\N\}$ is trivial, we will show it has an empty spectrum. 
%  Suppose $x: 2 \to 2$ is such that $x(\alpha(n)) = 0$ for every $n:\N$. 
%  As $x(1) = 1$, we must have for every $n:\N$ that $\alpha(n) \neq 1$. 
%  But then $\alpha(n) = 0$, contradicting our assumption. 
%  We get a contradicition and there thus there are no points in the spectrum of $2/\{\alpha(n)|n\in\N\}$ as required. 
\end{proof}

\begin{corollary}
  For $\alpha:2^\mathbb N$, we have that 
  \begin{equation}
    (\neg (\forall_{n:\mathbb N} \alpha (n)= 0)) \to \Sigma_{n:\mathbb N} \alpha (n)= 1
  \end{equation}
\end{corollary}
\begin{proof}
  Given $\alpha:2^\mathbb N$, consider the sequence $\alpha':\Noo$ satisfying $\alpha'(n) = 1$ iff 
  $n$ is minimal with $\alpha(n) = 1$. Then apply the above theorem.
\end{proof}

\begin{theorem}[The lesser limited principle of omniscience (LLPO)]\label{LLPO}
  For $\alpha:\N_\infty$, 
  we have that 
  \begin{equation}\label{eqnLLPO}
    \forall_{k:\N} \alpha(2k) = 0  \vee \forall_{k:\N} \alpha(2k+1) = 0
  \end{equation}
\end{theorem}
\begin{proof}
%
%  We first will define a map $f:B_\infty \to B_\infty \times B_\infty$. 
%  Because of \Cref{rmkMorphismsOutOfQuotient}, it is sufficient to define $f$ on $(p_n)_{n:\N}$ with 
%  $f(p_n) \wedge f(p_m) = (0,0)$ for $n\neq m$. 
%  To define $f(p_n)$, we use a case distinction on whether $n$ is odd or even. 
  Define $f:B_\infty \to B_\infty \times B_\infty$ as follows:
  \begin{equation}\label{eqnLLPOProofMap}
    f(p_n) =\begin{cases}
      (p_k,0) \text{ if } n = 2k\\
      (0,p_k) \text{ if } n = 2k+1\\
    \end{cases}
  \end{equation}
  Note that $f$ is well-defined as map in $\Boole$. 
 % , can make a case distinction on parity. 
%  By making a case distinction on $n,m$ being odd or even, 
%  we can see that 
%  $f(p_n) \wedge f(p_m) = (0,0)$ when $n\neq m$, thus $f$ is well-defined. 
  We claim $f$ is injective. Assume $f(x) = 0$, 
  to see that $x=0$, we make a case distinction on whether $x$ corresponds to a finite or a cofinite set as in \Cref{BinftyTermsWriting}.
%
%  We also claim it is injective.
%  Now let $x:B_\infty$ with $f(x) = 0$. 
  We denote $E,O\subseteq \N$ for the even and odd numbers respectively. 
%  and we make a case distincition based on \Cref{BinftyTermsWriting}.
  \begin{itemize}
    \item Suppose 
      $x = \bigvee_{i\in I_0} g_i$ with $I_0$ finite. 
      Then 
      $$f(x) = (\bigvee_{i\in I_0 \cap E } g_{\frac i2} , \bigvee_{i\in I_0 \cap O } g_{\frac {i-1}2} ) = (0,0)$$
      As $g_j\neq 0$ for all $j\in\N$, we must have $I_0 \cap E = \emptyset = I_0 \cap O$. 
      Thus $I_0= \emptyset$, and $x = 0$. 
    \item Suppose 
%      Let $x$ correspond to a cofinite subset of $\N$. Write 
      $x = \bigwedge_{j\in J} \neg g_j$ with $J$ finite. % for $J$ finite. 
      We will derive a contradiction. %, from which we can conclude that $x=0$ after all. 
      Note that   
      $$f(x) = (\bigwedge_{j\in J \cap E } \neg g_j , \bigwedge_{j\in J \cap O } \neg g_j ) = (0,0)$$
%      As $f(x) = (0,0)$, we have that 
%      $\bigwedge_{j\in J \cap E } \neg p_j =0$ and
%      $\bigwedge_{j\in J \cap O } \neg p_j  = 0$.
      However, any finite meet of negations corresponds to a cofinite set, hence is nonzero. 
      We get a contradiction and conclude $x=0$. 
%      However, any finite meet of negations will correspond to a cofinite set,
%      in particular it will not correspond to the empty set, and thus not be $0$.
%      Thus $f(x)\neq 0$, contradicting the assumption that $f(x) = 0$, hence $x=0$ ex falso. 
  \end{itemize}
%  In both cases, we conclude $x=0$, thus $f$ is injective. 
  By \Cref{SurjectionsAreFormalSurjections},
%  \Cref{FormalSurjectionsAreSurjections}, 
  $f$ corresponds to a surjection 
  $s:\Noo + \Noo \to \Noo$.
  Thus for $\alpha : \Noo$, 
  there exists some $x:\Noo + \Noo$ such that $s x = \alpha$. 
  If $x = inl(\beta)$, 
  for any $k:\N$, we have that 
  $$\alpha (g_{2k+1}) = s(x) (g_{2k+1}) = x(f(g_{2k+1})) = inl(\beta) (0,g_k)  = \beta(0) = 0.$$
  Similarly, if $x = inr(\beta)$, we have $\alpha(g_{2k}) = 0$ for all $k:\N$. 
  Thus \Cref{eqnLLPO} holds for $\alpha$ as required. 
\end{proof}
As the following shows, our use of \Cref{SurjectionsAreFormalSurjections} was non-trivial: 
%The use of \Cref{FormalSurjectionsAreSurjections}, and hence of propositional completeness, 
%was helpful in the above proof, as the following shows:
\begin{lemma}
  The function $f$  as in \Cref{eqnLLPOProofMap} does not have a retraction. 
\end{lemma}
\begin{proof}
  Suppose $r:B_\infty \times B_\infty \to B_\infty$ is a retraction of $f$. 
  Note that $r(0,1):B_\infty$ is expressable using only finitely many generators $(g_n)_{n\leq N}$
  Note that $r(0,1) \geq r(0,g_k) = g_{2k+1}$ for all $k:\N$. 
  As a consequence, $r(0,1)$ cannot be of the form $\bigvee_{i\in I_0} g_i$, and by \Cref{BinftyTermsWriting}, 
  $r(0,1)$ corresponds to a cofinite subset of $\N$. % = \bigwedge_{i:I_0} \neg p_i$, where $i\leq N$ for $i\in I_0$. 
  By similar reasoning so does $r(1,0)$.% corresponds to a cofinite subset of $\N$. 
  But the intersection of cofinite subsets is cofinite, while 
  $$r(0,1) \wedge r(1,0) = r( (1,0) \wedge (0,1)) = r(0,0) = 0$$
  which gives a contradiction. Thus no retraction exists. 
\end{proof}


%We finish with an equivalent formulation of LLPO:
%
%
%\begin{lemma}\label{corAlternativeLLPO}
%  Let $(\phi_n)_{n:\N}, (\psi_m)_{m:\N}$ be families of decidable propositions indexed over $\N$.
%  We then have 
%  \begin{equation}
%    (\forall_{n:\N} \forall_{m:\N} (\phi_n \vee \psi_m) )
%    \leftrightarrow
%    ((\forall_{n:\N} \phi_n) \vee (\forall_{m:\N} \psi_m) )
%  \end{equation}
%\end{lemma}
%\begin{proof}
%  See \cite{HannesDiener, ReverseMathsBishop}
%\end{proof}
%\begin{proof}
%  Note that the implication from right to left in the above equation always holds.
%  Assume that for all $m,n:\mathbb N$ we have $\phi_n\vee \psi_m$ 
%  Consider the sequence $\alpha:2^\mathbb N$ where $\alpha(2n) = 0$ iff $\phi_n$ and 
%  $\alpha(2m+1) = 0$ iff $\psi_m$. 
%  Let $\beta:\Noo$ be such that $\beta(i) = 1$ iff $i$ is minimal with $\alpha(i) = 1$
%  By LLPO, we have that 
%  $\beta$ is $0$ on all odd entries or on all even entries. 
%  Suppose that $\beta$ hits $0$ on all odd entries. 
%  We will show $\psi_m$ for all $m:\N$. 
%  As $\beta(2m+1) = 0$, there are two options:
%  \begin{itemize}
%    	\item If $\alpha(l)=0$ for all $l\leq 2m+1$. Then in particular $\alpha(2m+1)=0$ and $\psi_m$ holds.
%	\item Otherwise there is some $l<2m+1$ with $\beta(l) = 1$. 
%  As $\beta$ hits $0$ on odd entries, $l$ is even. 
%  So $\alpha(2n) = 1$ for $n = \frac{l}2$, meaning that $\neg \phi_n$. 
%  By assumption, $\phi_n \vee \psi_m$ holds, hence $\psi_m$ must hold. 
%  Thus for all $m:\N$, we have $\psi_m$ if $\beta$ hits $0$ on all odd entries. 
%  By a symmetric argument, if $\beta$ hits $0$ on all even entries, we have $\phi_n$ for all $n:\N$. 
%  We conclude that 
%  $((\forall_{n:\N} \phi_n) \vee (\forall_{m:\N} \psi_m) )$ 
%  as required. 
%  \end{itemize}
%\end{proof}
%
%\begin{remark}
%Note that the above statement implies LLPO as $\alpha(2n) =0 \vee \alpha(2m+1) =0$ for all $n,m:\mathbb N$ if $\alpha:\Noo$. 
%\end{remark}


\section{Topology}
%In this section, we will define the types of open and closed propositions. 
%These will allow us to define a (synthetic) topology  \cite{SyntheticTopologyLesnik} on any type.
%We will study this topology on Stone types in particular.
%
\subsection{Open and closed propositions}
In this section we will introduce a topology on the type of propositions, and 
study their logical properties.
We think of open and closed propositions respectively as countable disjunctions and conjunctions of decidable propositions.
Such a definition is universe-independent, and can be made internally.

\begin{definition}
A proposition $P$ is open (resp. closed) if there exists some $\alpha:2^\N$ such that $P \leftrightarrow \exists_{n:\mathbb N} \alpha_n = 0$ (resp. $P \leftrightarrow \forall_{n:\mathbb N} \alpha_n = 0$). We denote by $\Open$ and $\Closed$ the types of open and closed propositions.
\end{definition}

\begin{remark}\label{rmkOpenClosedNegation}
  The negation of an open proposition is closed, 
  and by MP (\Cref{MarkovPrinciple}), the negation of a closed proposition is open %. 
%  Also by MP, we have 
  and both open, closed propositions are $\neg\neg$-stable. 
%  and $\neg \neg P \to P$ whenever $P$ is open or closed. 
%  By the negation of WLPO (\Cref{NotWLPO}), 
  By $\neg$WLPO (\Cref{NotWLPO}), 
  not every closed proposition is decidable. 
  Therefore, not every open proposition is decidable. 
  % Both therefore and similarly can be used here, by a similar proof we can show it, or we can use that 
  % if $P$ is closed and $\neg P$ is decidable, so is $\neg \neg P = P$. 
  Every decidable proposition is both open and closed.
%  and in \Cref{ClopenDecidable} we shall see the converse. 
\end{remark}
\begin{lemma}
  We have the following results on open and closed propositions:
  \begin{itemize}
%    \item Closed propositions are closed under finite disjunctions. 
    \item Closed propositions are closed under countable conjunctions. 
    \item Open propositions are closed under finite conjunctions. 
    \item Open propositions are closed under countable disjunctions. 
  \end{itemize}
\end{lemma}
\begin{proof}
%  By Proposition 1.4.1 of \cite{HannesDiener}, LLPO (\Cref{LLPO}) is equivalent to the statement that 
%  the disjunction of two closed propositions are closed. 
  The statements have similar proofs, and we only present the proof that closed propositions are closed under 
  countable conjunctions. 
  Let $(P_n)_{n:\N}$ be a countable family of closed propositions. 
  By countable choice, for each 
  $n:\N$ we have an $\alpha_n:2^\N $ 
  such that $P_n \leftrightarrow \forall_{m:\N} \alpha_{n,m} =0$. 
  Consider a surjection $s:\N \twoheadrightarrow \N \times \N$, and let 
%  Let 
%  $$\beta_k = \alpha_{s(k)}.$$
  $\beta_k = \alpha_{s(k)}.$
  Note that $\forall_{k:\N} \beta_k = 0$ if and only if 
%  $\forall_{m,n:\N}\alpha_{m,n} = 0$, which happens if and only if 
  $\forall_{n:\N} P_n$. 
%  Hence the countable conjunction of closed propositions is closed. 
\end{proof}
\begin{remark}
  LLPO (\Cref{LLPO}) is equivalent to the statement that for $P,Q$ open, we have 
  $(\neg P \vee \neg Q) \leftrightarrow \neg (P\wedge Q)$. 
\end{remark}
%\begin{proof}
%  Assuming the above statement, let $\alpha:\Noo$, and consider 
%  the open propositions 
%  $\exists_{k:\mathbb N}\alpha_{2k+1} = 1, \exists_{k:\N} \alpha_{2k} = 1$. 
%  As $\alpha:\Noo$ there's at most one $n:\mathbb N$ with $\alpha_n =1$, so their conjunction is false. 
%  By the above statement, it follows one of them is false, leading to the statement of LLPO. 
%  
%  Now suppose LLPO holds, and assume $\neg (\exists_{n:\mathbb N} \alpha_n = 1\wedge \exists_{n:\mathbb N} \beta_n = 1$
%  for $\alpha,\beta :\Noo$. Then the sequence $\gamma:2^\N$ given by $\gamma_{2n} = \alpha_n, \gamma_{2n+1} = \beta_n$
%  can hit $1$ at most once and thus induces a sequence in $\Noo$. LLPO then shows that 
%  $\neg \forall_{n:\mathbb N} \alpha_n = 1\vee \neg \forall_{n:\mathbb N} \beta_n=1$ as required. 
%\end{proof}
\begin{corollary}
  Closed propositions are closed under finite disjunctions.
\end{corollary}
\begin{proof}
  Closed propositions are negations of open propositions. 
  As the conjunction of two open propositions is open, LLPO gives that 
  the disjunction of two closed propositions is closed. 
\end{proof}
We will use the above properties silently from now on. 
%OneBigLemma#
%OneBigLemma#\rednote{Phrase the following lemmas as one big lemma, 
%OneBigLemma#and use them silently without reference, also we should just state $\neg\neg$-stability instead of referring to the above all the time. }
%OneBigLemma#
%OneBigLemma#\begin{lemma}\label{ClosedCountableConjunction}
%OneBigLemma#  Closed propositions are closed under countable conjunctions.
%OneBigLemma#\end{lemma}
%OneBigLemma#\begin{proof}
%OneBigLemma#  Let $(P_n)_{n:\N}$ be a countable family of closed propositions. 
%OneBigLemma#  By countable choice, for each 
%OneBigLemma#  $n:\N$ we have an $\alpha_n:2^\N $ 
%OneBigLemma#  such that $P_n \leftrightarrow \forall_{m:\N} \alpha_{n,m} =0$. 
%OneBigLemma#  Consider a surjection $s:\N \twoheadrightarrow \N \times \N$, and let 
%OneBigLemma#%  Let 
%OneBigLemma#%  $$\beta_k = \alpha_{s(k)}.$$
%OneBigLemma#  $\beta_k = \alpha_{s(k)}.$
%OneBigLemma#  Note that $\forall_{k:\N} \beta_k = 0$ if and only if 
%OneBigLemma#%  $\forall_{m,n:\N}\alpha_{m,n} = 0$, which happens if and only if 
%OneBigLemma#  $\forall_{n:\N} P_n$. 
%OneBigLemma#  Hence the countable conjunction of closed propositions is closed. 
%OneBigLemma#\end{proof} 
%OneBigLemma#Using similar arguments, we can show the following two lemmas:
%OneBigLemma#\begin{lemma}\label{OpenCountableDisjunction}
%OneBigLemma#  Open propositions are closed under countable disjunctions. 
%OneBigLemma#\end{lemma}
%OneBigLemma#\begin{lemma}\label{OpenFiniteConjunction}
%OneBigLemma#Open propositions are closed under finite conjunctions. 
%OneBigLemma#\end{lemma}
%OneBigLemma#%\begin{proof}
%OneBigLemma#%We use \Cref{ClosedFiniteDisjunction} and the fact that $\neg(P\lor Q) \leftrightarrow \neg P \land \neg Q$.
%OneBigLemma#%\end{proof}
%OneBigLemma#%\begin{proof}
%OneBigLemma#%  Similar to the previous lemma. 
%OneBigLemma#%\end{proof}
%OneBigLemma#\begin{lemma}\label{ClosedFiniteDisjunction} 
%OneBigLemma#  Closed propositions are closed under finite disjunctions. 
%OneBigLemma#\end{lemma}
%OneBigLemma#\begin{proof}
%OneBigLemma#  This statement is equivalent to LLPO (\Cref{LLPO}) by  
%OneBigLemma#  Proposition 1.4.1 of \cite{HannesDiener}. 
%OneBigLemma#%  , LLPO is equivalent to the statement that 
%OneBigLemma#%  for $(\phi_n)_{n:\N}, (\psi_m)_{m:\N}$ families of decidable propositions indexed over $\N$, we have
%OneBigLemma#%  \begin{equation}
%OneBigLemma#%    (\forall_{n:\N} \forall_{m:\N} (\phi_n \vee \psi_m) )
%OneBigLemma#%    \leftrightarrow
%OneBigLemma#%    ((\forall_{n:\N} \phi_n) \vee (\forall_{m:\N} \psi_m) )
%OneBigLemma#%  \end{equation}
%OneBigLemma#%%  $(\forall_{n:\N} \alpha(n) = 0 )\vee (\forall_{n:\N} \beta(n) = 0 )$ is closed for any $\alpha,\beta:2^\N$.
%OneBigLemma#%%  By \Cref{corAlternativeLLPO}, the statement is equivalent to 
%OneBigLemma#%%  $ \forall_{n:\N}  \forall_{m:\N}  (\alpha(n) = 0 \vee \beta(m) = 0)$, 
%OneBigLemma#%  The latter which is a countable conjunction of decidable propositions, 
%OneBigLemma#%  hence closed by \Cref{ClosedCountableConjunction}.
%OneBigLemma#\end{proof}
%OneBigLemma#
\begin{corollary}\label{ClopenDecidable}
  If a proposition is both open and closed, it is decidable. 
\end{corollary}
\begin{proof}
  If $P$ is open and closed, %$\neg P$, and hence 
  $P\vee \neg P$ is open, 
  hence $\neg\neg$-stable and provable. 
%  and we conclude by $\neg\neg$-stability of open propositions. 
%  but open propositions are $\neg\neg$-stable by \Cref{rmkOpenClosedNegation} so we can conclude.
%  hence 
 % equivalent to $\neg \neg (P \vee \neg P)$ by \Cref{rmkOpenClosedNegation}.
 % As the latter proposition is provable, we may conclude $P$ is decidable. 
%  
%  If $P$ is open and closed, $P\vee \neg P$ is open, hence
%  equivalent to $\neg \neg (P \vee \neg P)$, which is provable. 
\end{proof}


%\begin{lemma}\label{OpenFiniteConjunction}
%  Open propositions are closed under finite conjunctions. 
%\end{lemma}
%\begin{proof}
%  We need to show that for any $\alpha,\beta:2^\N$, the following proposition is open:
%  \begin{equation}\label{eqnConjunctionOpen}
%    (\exists_{n:\N} \alpha(n) = 0 )\wedge(\exists_{n:\N} \beta(n) = 0 )
%  \end{equation}
%  Consider $\gamma:2^\N$ given by 
%  $\gamma(l) = 1$ iff there exist some $k,k'\leq l$ with 
%  $\alpha(k) = \beta(k') = 0$. 
%  As we only need to check finitely many combinations 
%  of $k,k'$, this is a decidable property for each $l:\N$ and $\gamma$ is well-defined. 
%  Then $\exists_{k:\N}\gamma(k)=0$ if and only if \Cref{eqnConjunctionOpen} holds.
%\end{proof}

\begin{lemma}\label{ClosedMarkov}
  For $(P_n)_{n:\N}$ a sequence of closed propositions, we have 
  $\neg \forall_{n:\N} P_n \leftrightarrow  \exists_{n:\N} \neg P_n$. 
\end{lemma}
\begin{proof}
  Both $\neg \forall_{n:\N} P_n$ and $\exists_{n:\N} \neg P_n$ are open, hence $\neg\neg$-stable.
  The equivalence follows. 
%  and the equivalence follows. 
%  from which the equivalence follows. 
%  We have that $\forall_{n:\N}P_n$ is closed and $\exists_{n:\N} \neg P_n$ is open by \Cref{OpenCountableDisjunction}, therefore both are $\neg\neg$-stable by \Cref{rmkOpenClosedNegation} and we can conclude.
%It is always the case that $\exists_{n:\N}\neg P_n \to \neg \forall_{n:\N} P_n$. 
  %For the converse direction,
  %note that $\neg \exists_{n:\N} \neg P_n(x) \to \forall_{n:\N} \neg \neg P_n(x).$
  %By \Cref{rmkOpenClosedNegation}, $\neg \neg  P_n(x)\leftrightarrow P_n(x)$ for all $n:\N$. 
  %It follows that 
  %$\neg \forall_{n:\N} P_n(x)\to 
  %\neg \neg \exists_{n:\N} \neg P_n(x).$
  %As $\exists_{n:\N}\neg P_n(x)$ is a countable disjunction of open propositions, 
  %it is open by \Cref{OpenCountableDisjunction} and thus equivalent to 
  %$\neg\neg\exists_{n:\N} \neg P_n(x)$ by \Cref{rmkOpenClosedNegation}.
  %We conclude that $\neg \forall_{n:\N} P_n \to \exists_{n:\N} \neg P_n$ as required. 
\end{proof} 

%\begin{lemma}\label{OpenDependentSums}
%  Open propositions are closed under dependent sums.
%\end{lemma}
%\begin{proof}
%  \rednote{If we show that Open propositions are exactly the overtly discrete ones, this is implied by $\Sigma$-closure}
%  First note that for $D$ a decidable proposition, and $X:D \to \Open$,
%  by case splitting on $D$, we can see 
%  $\Sigma_{d:D} X(d)$ is open.
%%
%  Then note that for $P$ an open proposition, 
%  there exists a sequence of decidable propositions $A_n$ with 
%  $P = \exists_{n:\N} A_n $.
%%
%  So for $Y : P \to Open $, the dependent sum $\Sigma_P Y$ is given by 
%  $\exists_{n:\N} (\Sigma_{a:A_n} Y(n,a))$,
%  which is a countable disjunction of open propositions, 
%  hence open by \Cref{OpenCountableDisjunction}.
%\end{proof}
%
%We will see the same holds for closed propositions in \Cref{ClosedDependentSums}.
%
%\begin{remark}\label{ImplicationOpenClosed}
%  If $P$ is open, $P \to \bot$ is only open if $P$ is decidable, which is not in general the case. 
%  Thus $\Open$ is not closed under dependent products. Neither is $\Closed$. 
%  However, as $(P\to Q)  \to \neg \neg (\neg P \vee Q)$,
%  we have that if $P$ is open and $Q$ is closed, then $P\to Q$ is closed, and similarly $Q\to P$ is open.
%\end{remark}
\begin{lemma}\label{ImplicationOpenClosed}
  If $P$ is open %(resp. closed) 
  and $Q$ is closed % (resp. open) 
  then $P\to Q$ is closed. % (resp. open). 
  If $P$ is closed and $Q$ open, then $P\to Q$ is open. 
\end{lemma}
\begin{proof}
  Note that $\neg P \vee Q$ is closed. Using $\neg\neg$-stability
  we can show $(P\to Q) \leftrightarrow (\neg P \vee Q)$. 
  The other proof is similar. 
%  and we conclude by 
%  Assume $P$ open and $Q$ closed, the other proof is similar. 
%  Note that $(\neg P \vee Q) \to (P \to Q)$ and 
%  $(P\to Q)\to \neg\neg(\neg P \vee Q)$. 
%  By \Cref{rmkOpenClosedNegation} it follows that 
%  $(\neg P \vee Q)\leftrightarrow (P \to Q)$, and using \Cref{ClosedFiniteDisjunction}, 
%  we can conclude that $P\to Q$ is closed. 
\end{proof}
%
%The following question was asked by Bas Spitters at TYPES 2024:


\subsection{Equality in $\Boole$ and $\Stone$}
\begin{lemma}
  Whenever $P$ is a proposition and overtly discrete, $P$ is open. 
\end{lemma}
\begin{proof}
  If $P$ is overtly discrete, then $P\leftrightarrow \exists_{n:\N} P_n$. 
  As every $P_n$ is finite, there is some $k_n:\N$ with $P_n \simeq Fin(k_n)$. 
  Hence we have $P_n \leftrightarrow (k_n \neq 0)$, and as equality of natural numbers is decidable, so is $P_n$. 
  Hence $P$ is a countable disjunction of decidable propositions, hence open. 
\end{proof}
\begin{lemma}
  Whenever $P$ is a an open proposition, it is overtly discrete.
\end{lemma}
\begin{proof}
  Suppose $P\leftrightarrow \exists_{n:\N} \alpha_n = 1$. 
  Let $P_n = \exists_{k\leq n} (\alpha_k = 1)$, which is a decidable proposition, hence a finite set. 
  Then the colimit of $P_n$ is $P$. 
\end{proof} 
\begin{corollary}
  A proposition is open iff it is overtly discrete. 
\end{corollary}
\begin{proof}
  Immediate by the above two lemmas. 
\end{proof}

\begin{lemma}
  Whenever $B$ is overtly discrete and $a,b:B$, the proposition $a=_B b$ is open. 
\end{lemma}
\begin{proof}
  For $a,b:B$ there is some $n:\N$ with $a',b':B_n$ and $\iota_n(a') = a,\iota_n(b') = b$.
  By \Cref{rmkEqualityColimit}, we have that $a=_B b$ iff 
  there is some $m\geq n$ with $\iota_n^m (a') = \iota_n^m(b')$. 
  As equality in finite sets is decidable, this is a countable disjunction of decidable propositions, hence open. 
\end{proof}

%\begin{lemma}\label{BooleEqualityOpen}
%  Whenever $B:\Boole$, $a,b:B$ the proposition $a=_Bb$ is open. 
%\end{lemma}
%\begin{proof}
%  Let $G,R$ be the generators and relations of $B$. 
%  Let $a,b$ be represented by $x,y$ in the free Boolean algebra on $G$. 
%  Now let $R_n$ denote the first $n$ elements of $R$. 
%  Note that $a=b$ iff there exists some $n:\N$ with $x-y \leq \bigvee_{r\in R_n} r$. 
%  Furthermore, inequality is decidable in the free Boolean algebra, hence
%  $a=b$ is a countable disjunction of decidable propositions, hence open. 
%\end{proof}

\begin{corollary}\label{TruncationStoneClosed}
  Whenever $S:\Stone$, $||S||$ is closed. 
\end{corollary}
\begin{proof}
  By \Cref{SpectrumEmptyIff01Equal}, $\neg S$ is equivalent to $0=_B 1$, which is open by the above. 
  Hence $\neg \neg S$ is a closed proposition, and by propositional completeness, so is $||S||$. 
\end{proof}

\begin{remark}\label{ExplicitTruncationStoneClosed}
  \rednote{New check later}
  The above lemma and corollary actually show that if we have an explicit 
  presentation of a Stone space as $S = Sp(2[G] / R)$, 
  we can construct an explicit sequence $\alpha:2^\N$ such that $||S|| \leftrightarrow \forall_{n:\N} \alpha(n) = 0$. 
\end{remark}


\begin{corollary}\label{PropositionsClosedIffStone}
  A proposition $P$ is closed iff it is a Stone space. 
\end{corollary}
\begin{proof}
  By the above, if $S$ is both a Stone space and a proposition, it is closed. 
  Conversely, note that 
  $$
  (\forall_{n:\N} \alpha(n) = 0 )\leftrightarrow Sp(2/\{\alpha(n)| n:\N\}).
  $$
  The latter is a proposition, as there is at most one Boolean map $2/\{\alpha(n)|n:\N\} \to 2$.
\end{proof}

\begin{lemma}\label{StoneEqualityClosed}
  Whenever $S:\Stone$, and $s,t:S$, the proposition $s=t$ is closed. 
\end{lemma}
\begin{proof}
  Suppose $S= Sp(B)$ and let $G$ be the generators of $B$. 
  Note that $s=t$ iff $s(g) =_2 t(g)$ for all $g:G$. 
  As $G$ is countable, and equality in $2$ is decidable, 
  $s=t$ is a countable conjunction of decidable propositions, hence 
  closed. 
\end{proof}
%
The following question was asked by Bas Spitters at TYPES 2024:
\begin{corollary}
  For $S:\Stone$ and $x,y,z:S$ 
  \begin{equation}\label{Apartness}
  x \neq y \to (x\neq z \vee y \neq z)
  \end{equation}
\end{corollary}
\begin{proof}
  As $x\neq y$, we can show that $\neg ( x = z \wedge y = z)$. 
  This in turn implies $\neg \neg ( x \neq  z \vee y \neq  z)$. 
  As, $x\neq z$ and $y \neq z$ are both open propositions, by \Cref{OpenCountableDisjunction} so is their disjunction. 
  By \Cref{rmkOpenClosedNegation}, that disjunction is double negation stable and \Cref{Apartness} follows. 
\end{proof}
\begin{remark}
  If \Cref{Apartness} holds in a type, we say that it's inequality is an apartness relation. 
  By a similar proof as above, it can be shown that in our setting inequality is an apartness relation 
  as soon as equality is open or closed. 
\end{remark}

\subsection{Types as spaces}
The subset $\Open$ of the set of propositions induces a topology on every type. 
This is the viewpoint taken in synthetic topology, from which we borrow terminology \cite{SyntheticTopologyEscardo, SyntheticTopologyLesnik}. 
%other references include \cite{SyntheticTopologyEscardo}%, TODOSortOutTaylorsReferences}.
%Defining a topology in this way has some benefits, which we summarize in this section. 

\begin{definition}
  Let $T$ be a type, and let $A\subseteq T$ be a subtype. 
  We call $A\subseteq T$ open (resp. closed) if $A(t)$ is open (resp. closed) for all $t:T$.
\end{definition}

\begin{remark}
  It follows immediately that the pre-image of an open by any map is open, so that any map is continuous. 
%  This is only relevant for a space if the topology we defined above matches the topology one would expect. 
  In \Cref{StoneClosedSubsets}, we will see that the resulting topology is as expected for Stone spaces.
  In \Cref{IntervalTopologyStandard}, we will see that the same holds for the unit interval. 
\end{remark}



%\begin{remark}
%  Phao's principle is a special case of directed univalence. 
%\end{remark}
%\begin{proof}
%  \rednote{TODO}
%\end{proof}

\subsection{The topology on a Stone space}
\begin{theorem}\label{StrongVersionOfEquivalencesOfClosedSubsetsOfStone}
  Let $A\subseteq S$ be a subset of a Stone space. TFAE:
  \begin{enumerate}[(i)]
    \item There exists a map $\alpha_{(\cdot)}:S \to 2^\N$ such that 
      $A s \leftrightarrow \forall_{n:\N} \alpha_s(n) = 0$.
    \item There exists some countable family 
      $D_n,~{n:\N}$ 
      of decidable subsets of $S$ with $A = \bigcap_{n:\N} D_n$. 
    \item There exists a Stone space $T$ and some map $T\to S$ whose image is $A$. 
    \item $A$ is closed.
  \end{enumerate}
\end{theorem}
\begin{proof}
\item 
  \begin{itemize}
  \item[$(i)\leftrightarrow (ii)$.] 
    $D_n$ and $\alpha_{(\cdot)}$ can be defined from each other by 
%    Define the decidable subsets of $S$ 
     by $D_n(s) \leftrightarrow (\alpha_s(n) = 0)$. Then observe that %$A=\bigcap_{n:\N} D_n$ as 
     \begin{equation}
      (\bigcap_{n:\N} D_n) (s) \leftrightarrow 
      \forall_{n:\mathbb N} (\alpha_s(n) = 0) 
%      \leftrightarrow A s. 
     \end{equation}
   \item[$(ii) \to (iii)$.]
      Let $S=Sp(B)$. 
      By Stone duality, we have $d_n,~n:\N$ terms of $B$ such that $D_n = \{x:S| x(d_n) = 1\}$. 
      Let $C = B/\langle (\neg d_n)_{n:\N}\rangle$.
      By \Cref{FormalSurjectionsAreSurjections}, the quotient map $B \twoheadrightarrow C$
      corresponds to a injection $\iota:Sp(C) \hookrightarrow  S$. 
%      For $s:S$, $s$ lies in the image of this map iff $s(\neg d_n) = 0$, 
      and for any $n:\N$, we have 
      \begin{equation}
        x\in \iota(Sp(C)) \leftrightarrow x(\neg d_n) = 0 \leftrightarrow x(d_n) = 1 \leftrightarrow x\in D_n
      \end{equation}
      Thus the image of $\iota$ is given by $\bigcap_{n:\N} D_n$. 
%   \item[$(iii) \to (i)$.]
%     Let $\iota:T\hookrightarrow S$ be a injective map of Stone spaces with image $A$. 
%     Then $A(s) \leftrightarrow$
%
%  \item[$(iii) \leftrightarrow (iv)$.]
   \item [$(i) \to (iv)$.] By definition.
   \item[$(iv) \to (iii)$.]
     As $A$ is closed, it induces a map $a:S\to \Closed$. 
     We can cover the closed propositions with Cantor space
     by sending 
     $\alpha \mapsto \forall_{n:\mathbb N} \alpha n = 0.$
     Now local choice gives us that there merely exists $T, e, \beta_\cdot$ as follows:
     \begin{equation}
       \begin{tikzcd}
         T \arrow[r,"\beta_\cdot"] \arrow[d, two heads,"e"] & 2^\mathbb N 
         \arrow[d,two heads, "\forall_{n:\mathbb N} (\cdot)n = 0"] \\
         S \arrow[r,"a"] & \Closed
       \end{tikzcd} 
     \end{equation} 
     Define $B(t) \leftrightarrow \forall_{n:\mathbb N} \beta_t(n) = 0$. 
     As $(i) \to (iii)$ by the above, $B$ is the image of some Stone space. 
     Furthermore, note that $A$ is the image of $B$, thus $A$ is the image of some Stone space. 
   \item[$(iii) \to (i)$.] 
      Let $f:T\to S$ be a map between Stone spaces. 
      Let $A,B$ be the underlying Boolean algebras of $S,T$ respectively. 
      Let $G_A,G_B$ and $R_A,R_B$ be the countable sets of generators and relations of $A,B$.
      It is important to note that we can consider these countable sets before we consider any $s:S$. 
      \rednote{TODO can someone give feedback on this last sentence, I'm not sure it's clear}

      Following \Cref{FiberConstruction}, for each $s:S$, we can construct 
      a countable set $I\subseteq B$ such that $Sp(B/I) = \Sigma_{t:T} f(y) = t $.
      Now note that $s$ in the image of $f$ iff $0\neq_{B/I} 1$. 
      At this point, we have constructed the the generators and relations of $B/I$, 
      hence using the proof of \Cref{BooleEqualityOpen}, we can construct a sequence 
      $\alpha_s:2^\N$ such that $0 =_{B/I}1\leftrightarrow \exists_{n:\N} \alpha_s(n) = 0$. 
      And for $\beta_s(n) = 1-\alpha_s(n)$, we conclude that 
      \begin{equation}
        s\in f(T) \leftrightarrow \forall_{n:\N} \beta_s(n) = 0
      \end{equation}
\end{itemize} 
\end{proof} 




\appendix
%\section{Technical details}
\section{The equivalence of $B_\infty$ and $\N_{(co)fin}$}
Recall that we defined $B_\infty$ as the quotient of the freely generated algebra 
over $p_n,~n\in\N$ by the relations $\{p_n \wedge p_m | n\neq m\}$. 

\begin{lemma}\label{N-co-fin-cp}
  The Boolean algebra of co-finite subsets of $\N$
  is equivalent to $B_\infty$. 
\end{lemma}
\begin{proof}
  Let $f:B_\infty \to \N_{(co)fin}$ be induced by sending $p_n$ to $\{n\}$. 
  Note that whenever $n\neq m$, we have 
  $f(p_n)\wedge f(p_m) = \{n\} \cap \{m\} = \emptyset$, 
  thus $f$ respects the relations of $B_\infty$ and is well-defined.

  Define $g:N_{(co)fin)} \to B_\infty$ as follows:
  \begin{itemize}
    \item On a finite subset $I$, we define $g(I) = \bigvee_{i\in I} p_i$, 
    \item On a cofinite subset $J$, we define $g(J) = \bigwedge _{i \in J^C} \neg p_i$. 
  \end{itemize}
  Note that in these cases we indeed have $I,J^C$ are finite, so these are well-defined elements. 
  We must show that $g$ is a Boolean morphism. 

  \begin{itemize}
    \item 
      By deMorgan's laws, $g$ preserves $\neg$:
      for $I$ finite we have
      \begin{equation}
      \neg g(I) = \neg (\bigvee_{i\in I} p_i) = \bigwedge_{i\in I} \neg p_i = g(I^C)
      \end{equation}
      And for $J$ cofinite, we apply similar reasoning. 
    \item To see that $g$ preserves $\vee$, we need to check three cases
      \begin{itemize}
        \item If both $I,J$ are finite, then 
        \begin{equation} 
          g(I \cup J) = \bigvee_{i\in I \cup J} p_i= \bigvee_{i\in I} p_i \vee \bigvee_{j\in J} p_j 
          = g(I) \vee g(J)
        \end{equation}
        and we're done. 
      \item If both $I,J$ are cofinite, we have
        \begin{equation}
          g(I) \vee g(J) = 
          \bigwedge_{i \in I^C} \neg p_i \vee 
          \bigwedge_{j \in J^C} \neg p_j 
          = 
          \bigwedge_{i\in I^C} 
          \bigwedge_{j \in J^C}(\neg p_i \vee  \neg p_j) 
        \end{equation}
        Now note that in $B_\infty$, we have 
        \begin{equation}
          \neg p_i \vee \neg p_j = \neg ( p_i \wedge p_j) = 
          \begin{cases}
            \neg p_i \text{ if } i = j\\
            1 \text{ if } i \neq j  
          \end{cases}
        \end{equation}
        Therefore, we can leave out the case that $i\neq j$ in the calculation of the above meet, and
        \begin{equation}
          \bigwedge_{i\in I^C} 
          \bigwedge_{j \in J^C}(\neg p_i \vee  \neg p_j)  
          = 
          \bigwedge_{i \in (I^C \cap J^C)} \neg p_i
          = 
          \bigwedge_{i \in (I \cup J)^C} \neg p_i 
        \end{equation}
        as $I\cup J$ must also be cofinite, this equals 
          $ g( I \cup J)$. 
        \item 
          If $I$ is finite and $J$ cofinite, we have 
          that $I\cup J$ is cofinite, hence 
          \begin{equation}
            g(I\cup J) = \bigwedge_{k\in (I \cup J)^C} \neg p_k
            = \bigwedge_{k \in (J^C -I)} \neg p_k
          \end{equation}
          Now note that 
          whenever $i\neq k$, we have 
          \begin{equation}
            p_i = (p_i \wedge \neg p_k) \vee (p_i \wedge p_k) = 
            (p_i \wedge \neg p_k) \vee 0 = p_i \wedge \neg p_k
          \end{equation}
          Hence by absorption
          \begin{equation} 
            (p_i \vee \neg p_k)  =
              \begin{cases}
                1 \text{ if } i = k \\
                \neg p_k \text{ if } i \neq k
              \end{cases}
          \end{equation}
          As for all $k\in J^C-I$ and all $i\in I$ we have $k\neq i$, we may thus write
          \begin{equation}\label{eqnCofiniteHelper1}
            \bigwedge_{k \in (J^C - I)} \neg p_k = 
            \bigwedge_{k \in (J^C - I)} (\neg p_k \vee (\bigvee_{i\in I} p_i))
          \end{equation}
          We now note that 
          \begin{equation}\label{eqnCofiniteHelper2}
            1=\bigwedge_{i\in I} 1 = \bigwedge_{i\in I} (\neg p_i \vee (\bigvee_{i\in I} p_i)).
          \end{equation}
          Taking the meet of the expressions in \Cref{eqnCofiniteHelper1} and \Cref{eqnCofiniteHelper2}, 
          we see that 
          \begin{equation}
            \bigwedge_{k \in (J^C - I)} \neg p_k = 
            \bigwedge_{j \in J^C} (\neg p_j \vee (\bigvee_{i\in I} p_i))
          \end{equation}
          And using distributivity rules, we can see that 
          \begin{equation}
            \bigwedge_{j \in J^C} (\neg p_j \vee (\bigvee_{i\in I} p_i))
            = 
            (\bigwedge_{j \in J^C} \neg p_k) \vee (\bigvee_{i\in I} p_i)
          \end{equation}
          From which we may conclude that $g(I\cup J) = g(I) \cup g(J)$. 
      \end{itemize}
    \item The case for $\wedge$ is completely dual to the case for $\vee$. 
  \end{itemize}
We conclude that $g$ is a Boolean morphism. 
Furthermore, it is easy to see that $g$ and $f$ are each other's inverse, 
thus the Boolean algebras are isomorphic. 
\end{proof}
\begin{remark}\label{AppendixCofiniteOrFinite}
  As a consequence of the above proof, any $b:B_\infty$ corresponds either to 
  \begin{itemize}
    \item a finite set $I$, in which case $b = \bigvee_{i\in I} p_i$. 
    \item a cofinite set $J$, in which case $b = \bigwedge_{j\in J^C} \neg p_j$. 
  \end{itemize}
  We will call $b$ finite/cofinite respectively. 
\end{remark}
\begin{remark}
Recall that $\Noo$ is defined as the spectrum of $B_\infty$. 
If $\alpha:\Noo$ satisfies $\alpha(p_n) = 1$, then $\alpha(p_m) = 0$ for all $n\neq m$. 
Therefore, for each $n:\N$, there is an unique map $\chi_n$ with $\chi_n(p_n) = 1$. 
There is also the point $\chi_\infty : \Noo$ which is unique 
with the property that $ \chi_\infty(p_n) = 0$ for all $n:\N$. 
We will call decidable subsets of $\Noo$ finite/cofinite iff their corresponding elements of $B_\infty$ are. 
\end{remark}
\rednote{Active WIP}
\begin{lemma}\label{FiniteDecidableSubsetsCharacterization}
  Finite decidable subsets of $\Noo$ are of the form 
%  If $d:B_\infty$ is of the form $\bigvee_{i\in I} p_i$,
%  it corresponds to the decidable set 
  $\{\chi_i | i \in I\}$ for some finite $I\subseteq \N$. 
\end{lemma}
\begin{proof}
  Let $d= \bigvee_{i\in I} p_i$. 
  Clearly whenever $i\in I$, we have $\chi_i(d) = 1$. 
%
  Now suppose $f:B_\infty \to 2$ is such that $f(d) = 1$. 
  Then $\bigvee_{i\in I}(f(p_i)) = 1$, hence it is not the case that $f(p_i) = 0$ for all $i\in I$. 
  Now as $I$ is finite and $f(p_i) = 0 \vee f(p_i) = 1$ for all $i\in I$, 
  there must exist some (necessarily unique) $i\in I$ with $f(p_i) = 1$. Hence $f = \chi_i$. 
%
  Thus $f(d) = 1$ iff there is some $i\in I$ with $f = \chi_i$. 
\end{proof}
\begin{corollary}\label{CoFiniteDecidableSubsetsCharacterization}
  Cofinite decidable subsets of $\Noo$ are of the form
%  If $d:B_\infty$ is of the form $\bigwedge_{j\in J^C} \neg p_j$,
%  it corresponds to the decidable set 
  $\neg \{\chi_i | i \in I\}$ for $J\subseteq\N$ finite. 
\end{corollary}
\begin{proof}
  Let $D$ be a cofinite decidable subset. Then $\neg D$ is a finite decidable subset, 
  By the above lemma it follows that $\neg D = \{\chi_i | i\in I\}$. 
  As $\neg \neg D = D$, the result follows. 
\end{proof}
\begin{corollary}
 Any a decidable subset $D\subseteq\Noo$ is cofinite iff $\chi_\infty\in D$. 
\end{corollary}
\begin{proof}
  This follows from the observation that $\chi_\infty \in \neg \{\chi_j | j \in J^i\}$, 
  the observation that all decidable subsets are either finite or cofinite, 
  and the characterization of finite a cofinite decidable subsets in 
  \Cref{FiniteDecidableSubsetsCharacterization} and 
  \Cref{CoFiniteDecidableSubsetsCharacterization}.
\end{proof}
\begin{corollary}
  If $U\subseteq \Noo$ is open and $\chi_\infty \in U$, there exists some $n\in \N$ such that 
  $\{\chi_k | k\geq n\} \subseteq U$. 
\end{corollary}
\begin{proof}
  If $U$ is open, by \Cref{StoneOpenSubsets}, it is a countable union of decidable subsets. 
  One of these must contain $\chi_\infty$, hence be cofinite and 
  of the form $\neg \{ \chi_i | i \in I\}$ for some finite $I\subseteq \N$.
  As $I$ is finite, there is some $n:\N $ with $n>i$ for all $i\in I$. 
  For all $k\geq n$, we have that $k\notin I$, hence $\chi_k \in \neg \{\chi_i | i \in I\}\subseteq U$ as required. 
\end{proof}



%
%\begin{lemma}
%  For all decidable subsets $D:\Noo\to 2$,
%  with $D$ non-empty, there exists some $n:\N$ with $\chi_n \in D$. 
%\end{lemma}
%\begin{proof}
%  We make a case distinction based on \Cref{AppendixCofiniteOrFinite}. 
%  \begin{itemize}
%    \item 
%      If $D$ corresponds to a finite $d:B_\infty$, but is non-empty, then 
%      $d=\bigvee_{i\in I} p_i$ for $I\subseteq \N$ finite and non-empty. 
%      If $I$ is finite (as in \Cref{dfnFinite}) and non-empty, 
%      $I\simeq Fin_k$ for some $k\neq 0$. 
%      In particular, there is a map $1 \to I$,
%      hence a term $i:I$. 
%      Then $\chi_i(d) = 1$, hence $\chi_i \in D$. 
%    \item 
%      If $D$ corresponds to some cofinite $d:B_\infty$, we have 
%      $d = \bigvee_{i\in I} \neg p_i$ for some $I\subseteq \N$ finite. 
%      Then there is some 
%\end{proof}
%




\section{Cocompleteness of $\Boole$}
\rednote{TODO, is $\Boole$ closed under countable limits? 
  It has finite colimits, as it has pushouts and initial object.
  It should also have sequential colimits (TODO). 
  Is a countable coproduct the sequential colimit of it's initial finite coproducts? 
}
\begin{lemma}\label{BoolePushouts}
  Countably presented Boolean algebras are closed under pushout. 
\end{lemma} 
\begin{proof}
  Let $A,B,C:\Boole$, and suppose $f:A\to B, g:A \to C$ are Boolean morphisms. 
  Let $G_A, G_B,G_C$ be the underlying countable sets of generators for $B,C$ and 
  let $R_A,R_B,R_C$ be the underlying countable sets of relations. 
  Consider $P$ the Boolean algebra generated by $G_B\sqcup G_C$ under the relations 
  $R_B\cup R_C \cup F$ where $F$ is the set of expressions $f(a)-g(a), a\in G_A$.
  
  Note that as the generators of $B$ are included in those of $P$, 
  and all relations of $B$ are included in those of $P$, there is a map $h:B\to P$. 
  Similarly there is a map $i:C\to P$. 
  We now claim that the following is a pushout square:
  \begin{equation}\begin{tikzcd}
    A \arrow[r,"f"] \arrow[d,"g"] & B \arrow[d,"h"]\\
    C \arrow[r,"i"] & P
  \end{tikzcd}\end{equation}  
  Suppose $\beta:B \to X, \gamma:C\to X$ are such that $\beta\circ f = \gamma \circ h$. 
  $\beta,\gamma$ then induce maps on the generators of $P$. 
  These maps respect $F$ as $\beta\circ f=\gamma\circ h$, and they must respect $R_B,R_C$ as they are maps out of $B,C$. 
  Therefore, $\beta,\gamma$ induce a map $e:P\to X$, such that 
  $e(b) = \beta(b)$ for $b:G_B$ and $e(c)=\gamma(c)$ for $c:G_C$. 
  Furthermore, any map $P\to X$ with this property must agree with $e$ on all the generators of $P$, 
  and therefore equal $e$. Thus $e$ is the unique extension $P\to X$. 
  Thus $P$ the above square is indeed a pushout. 
\end{proof}
%\begin{lemma}\label{BooleCoEqualizers}
%  Countably presented Boolean algebras are closed under coequalizers.
%\end{lemma}
%\begin{proof}
%  Let $f,g:A\to B$ be Boolean morphisms.
%  Define $C = B/R$, where $R$ is given by the relations $fa-ga,~a\in G_A$, for $G_A$ the set of generators of $A$.
%  Suppose that we have a map $x:B\to D$ with $xf = gf$. Then $x$ respects $R$, and thus defines a map $y:C \to D$. 
%  Furthermore, any map $C\to D$ extending $x$ agrees with $y$ on the generators of $C$, 
%  and is thus equal to $y$. Therefore $C$ is the coequalizer of $f,g$. 
%\end{proof}


%%
%%\begin{corollary}\label{CoCompletenessBoole}
%%  The category of countably presented Boolean algebras contains all finite colimits. 
%%\end{corollary}
%%\begin{proof}
%%  Recall that $\Boole$ has an initial object given by $2$. 
%%  By \Cref{BoolePushouts}, 
%%%  it is therefore closed under coproducts. 
%%%  By \Cref{BooleCoEqualizers}, 
%%  it follows that $\Boole$ contains all finite colimits. 
%%\end{proof}

\section{Some notes on our axioms}
\label{NotesOnAxioms}
\subsection{Alternatives to propositional completeness}
In \Cref{Axioms}, we have chosen to present propositional completeness as an axiom. 
However, assuming Stone duality, we could have made some other choices, 
and left propositional completeness as a theorem. 
What's more, assuming the axiom of Dependent choice,
the axiom is equivalent to LLPO. 
In this section, we will show these equivalences. 

\begin{theorem}\label{AlternativesToAxiom2}
  Assuming Stone duality, the following are equivalent:
  \begin{enumerate}[(i)]
    \item For $S$ Stone, we have $\neg \neg S \to ||S||$. 
    \item For $S$ Stone, we have that $||S||$ is closed. 
    \item A map $f:A \to B$ in $\Boole$ is injective iff the map $(\cdot) \circ f : Sp(B) \to Sp(A)$ is surjective. 
  \end{enumerate}
\end{theorem}
\begin{proof}
  We assume that $S= Sp(B)$. 
  Note the proof of \Cref{SpectrumEmptyIff01Equal} only uses Stone duality. 
  The proof of \Cref{BooleEqualityOpen} only relies on the definition of $\Boole$.
  Hence the argument in \Cref{TruncationStoneClosed}, which shows $(i)\to (ii)$ only relies on Stone duality. 
  Furthermore, the argument that closed propositions are double negation stable (\Cref{rmkOpenClosedNegation})
  only used \Cref{MarkovPrinciple}, which followed from Stone duality as well. 
  Hence if $||S||$ is closed, we have $\neg \neg ||S|| \leftrightarrow ||S||$, thus $(ii) \to (i)$. 
  $(i)\to (iii)$ is \Cref{FormalSurjectionsAreSurjections}. 
  By the above discussion, we also have that $\neg \neg S$ iff $0\neq_B 1$. 
  Note that $0\neq_B 1$ iff the map $2\to B$ is injective. 
  Furthermore, $||S||$ iff the map $S \to \top $ is surjective. 
  Hence $(iii) \to (i)$. 
\end{proof} 

\begin{lemma}\label{LLPOAndDCToCompleteness}
Assuming dependent choice, Stone duality, and that closed propositions are closed under disjunctions, 
we can show propositional completeness. 
\end{lemma}
\begin{proof}
  Let $B:\Boole$ satisfy $0\neq_B 1$. We will show there merely exists a map $B\to 2$. 
  Let $G$ be the set of generators of $B$. 
  We will use dependent choice on the the following $E_n,R_n$:
  \begin{itemize}
    \item 
  Let $E_n$ be the type consisting of 
  \begin{itemize}
    \item A map from the first $n$ generators of $B$ to $2$, denoted $x_n:G_n \to 2$. 
    \item A proposition denoting that $0\neq_{B_n} 1$ for $B_n$ given by:
      \begin{equation}
        B_n := B/\big( \{g|g\in G_n, x_n(g) = 0\} \cup \{ \neg g| g\in G_n, x_n(g) = 1\}\big).
      \end{equation}
  \end{itemize}
  \item 
    And let $R_n:E_n \to E_{n+1} \to \mathcal U$ denote the relation that $x_{n+1}$ extends $x_n$. 
  \end{itemize} 
  Note that $E_0$ is inhabited as $0\neq_B 1$. Assume $E_n$.
%  Now assume $x_n:G_n\to 2$ witnesses $E_n$. 
  As $0\neq_{B_n}1$, for all $g:B_n$, we can show 
%  we have $$\neg ((g =1)  \wedge ((\neg g) = 1)).$$
% % 
%%  Now suppose that $E_n$ is inhabited,  and let $x_n:G_n \to 2$. 
%%  Note that in $B_n$, we have $0\neq 1$ and thus $$\neg ((g =1)  \wedge ((\neg g) = 1))$$
%%  for all $g:B_n$.
%  Therefore, we have 
  $$\neg \neg (( g\neq 1) \vee ((\neg g) \neq 1)).$$
  By \Cref{BooleEqualityOpen}, and \Cref{rmkOpenClosedNegation}, 
  (which could be shown using Stone Duality)
  and the assumption that 
  closed statements are closed under disjunction, we have that the above statement is equivalent to 
  $(g \neq 1) \vee ((\neg g) \neq 1)$. 
  This holds in particular for $g$ the $n+1$'th generator of $B$. 
  Therefore, we have that $0\neq 1$ in $B_n/\{g\}$ or in $B_n/\{\neg g\}$. 
  Thus we can extend $x_n$ by letting $x_{n+1}(g) = 0$ or $x_{n+1}(g) = 1$ respectively. 
  
  By dependent choice, we get a map $x:G\to 2$. 
  We claim that for this map $x$, we have $0\neq 1$ in 
  \begin{equation}
    B' := B/\big( \{g|g\in G, x(g) = 0\} \cup \{ \neg g| g\in G, x(g) = 1\}\big).
  \end{equation}
  Note that $B'$ is the colimit of the sequence $B_n$ with projection maps $B_n \to B_{n+1}$. 
  Thus if $0=1$ in $B'$, $0=1$ in some $B_n$, which doesn't happen by assumption. 
  Therefore we have $0\neq 1$ in $B'$. 
  Furthermore, note that $B'$ is equivalent to a Boolean algebra with no generators, 
  as any generator in $B$ is sent to either $0$ or $1$ by the relations in $B'$. 
%
  But now any Boolean algebra with no generators and $0\neq 1$ is isomorphic to $2$. 
  Therefore $B'\simeq 2$, and the projection map $B\to B'$ gives a map $B \to 2$. 
  
\end{proof}

\begin{corollary}
Assuming dependent choice and Stone duality, TFAE:
\begin{enumerate}[(i)]
  \item For $S$ Stone, we have $\neg \neg S \to ||S||$. 
  \item LLPO.
  \item The disjunction of two closed propositions is closed. 
\end{enumerate}
\end{corollary}
\begin{proof}
  $(i) \to (ii)$ is \Cref{LLPO}, $(ii) \to (iii)$ is \Cref{ClosedFiniteDisjunction}, 
  and $(iii) \to (i)$ is \Cref{LLPOAndDCToCompleteness}
\end{proof}
\rednote{
  @Hugo, you mentioned that axiom 2 was independent from the other axioms. 
This might be a good place to reference to that proof}


\subsection{The formulation of local choice}

\begin{lemma}
  TFAE:
  \begin{itemize}
\item  Whenever $S$ Stone and $E\twoheadrightarrow S$ surjective, then there is some $T$ Stone,
    a surjection $T \twoheadrightarrow S$ and a map $T\to E$ 
    such that the following diagram commutes:
    \begin{equation}\begin{tikzcd}
      & E \arrow[d,""',two heads]\\
      T \arrow[ru,dashed]  \arrow[r,two heads, dashed ] &S %& \arrow[l, "", two heads, dashed] T\arrow[lu, ""',dashed ]
    \end{tikzcd}\end{equation}  
\item
  Whenever we have $S:\Stone$, $E,F$ arbitrary types, a map $f:S \to F$ and a 
  surjection $e:E \twoheadrightarrow F$, 
  there exists a Stone space $T$, a cover $T\twoheadrightarrow S$ and an arrow $T\to E$ making the following diagram commute:
    \begin{equation}\begin{tikzcd}
      T \arrow[d,dashed, two heads ] \arrow[r,dashed]&  E \arrow[d,""',two heads, "e"]\\
      S  \arrow[r, "f"] & F
    \end{tikzcd}\end{equation}  
\end{itemize} 
\end{lemma}
\begin{proof}
  By considering $f=id$ we can see that the second statement implies the first. 

  For the converse, let $S,E,F,e,f$ be as in the second statement. 
  As $e$ is surjective, whenever $s:S$, we there merely exists some $b:E$ with $e(b) = f(s)$. 
  This induces an element $(s,b):S\times_F E$. 
  Thus the projection $S\times_F E \to S$ is surjective, and 
  the first axiom provides us with a $T$ as required. 
    \begin{equation}\begin{tikzcd}
     & S \times_F E \arrow[r] \arrow[d,two heads] &  E \arrow[d,""',two heads, "e"]\\
       T \arrow[r, two heads ,dashed ] \arrow[ru,dashed]& 
       S  \arrow[r, "f"] & F
    \end{tikzcd}\end{equation}  
    
\end{proof}

\subsection{Scott continuity implies Stone duality}

%\input{ColimitRepresentation.tex}


%
%\section{Topology}
%Analogous to synthetic algebraic geometry,
we use pointwise and local definitions of open subsets,
which agree if we assume a corresponding choice axiom.

%
%\subsection{Compact Hausdorff}
%\subsection{Compact Hausdorff types}
\begin{definition}
  A type $X$ is called compact Hausdorff iff there exists some $S:\Stone$ and some 
  equivalence relation $\sim:S \times S \to \Closed$ such that $X \simeq S / \sim$. 
  We denote $\CHaus$ for the type of compact Hausdorff types. 
\end{definition} 

\begin{lemma}
Let $A\subseteq X$ be a subtype of a compact Hausdorff space. 
Let $S, \sim$ be any presentation of $X$. 
Then $A$ is closed iff it is the image of a closed subtype of $S$ under the quotient map. 
\end{lemma}
\begin{proof}
  If $A$ is closed, then it's pre-image under any map is also closed. 
  In particular for $q:S\to X$ the quotient map, $q^{-1}(A)$ is closed. 
  As $q$ is surjective, we have $q(q^{-1}(A)) = A$,
  hence $A$ is the image of a closed subtype of $S$. 
  Conversely, let $B\subseteq S$ be closed. 
%  Then for any $s:S$, the subtype $\{t:S| B(s) \wedge s \sim t\} \subseteq S$ is closed. 
%  Hence by 
  Define $A\subseteq S$ by 
  $$A(s) := \exists_{s:S} (B(t) \wedge s \sim t).$$
  As $B, \sim$ are closed, by \Cref{ClosedCountableConjunction} and \Cref{InhabitedClosedSubSpaceClosed}, 
  we have that $A$  is closed. 
  Also $A$ respects $\sim$, hence induces a map $A': X\to \Closed$.
  Furthermore, $A(q(s))$ iff $q(s)$ is in the image of $B$. 
  Therefore $A'(x)$ iff $x$ is in the image of $B$. 
\end{proof}
\begin{corollary}
  For $X:\CHaus$ a subtype $A\subseteq X$ is closed iff it is the image of 
  a map $T\to X$ for some $T:\Stone$. 
\end{corollary}
\begin{proof}
  Directly from the above and \Cref{StoneClosedSubsets}.
\end{proof}

\begin{corollary}\label{InhabitedClosedSubSpaceClosed}
  For $X:\CHaus$ and $A\subseteq X$ closed, we have 
  $\exists_{x:X} A(x)$ is closed. 
\end{corollary}
\begin{proof}
  Let $A$ be the image of a map map $T\to X$ for $T:\Stone$. 
  Then $\exists_{x:X} A(x) \leftrightarrow ||T||$, which is closed by \Cref{TruncationStoneClosed}
\end{proof}


\begin{corollary}\label{ClosedDependentSums}
  Closed propositions are closed under dependent sums. 
\end{corollary}
\begin{proof}
  Let $P:\Closed$ and $Q:P \to \Closed$. 
  Then $\Sigma_{p:P} Q(p) \leftrightarrow \exists_{p:P} Q(p)$.
  As $P$ is Stone by \Cref{PropositionsClosedIffStone}, it is also compact Hausdorff, thus
  \Cref{InhabitedClosedSubSpaceClosed} gives that $\Sigma_{p:P} Q(p)$ is closed. 
\end{proof}
\begin{remark}
  Analogously to \Cref{OpenTransitive} and \Cref{OpenDominance}, it follows that 
  closedness is transitive and $\Closed$ forms a dominance. 
\end{remark}
\begin{corollary}\label{AllOpenSubspaceOpen}
  For $U\subseteq X$ an open subset of a compact Hausdorff space, we have 
  $\forall_{x:X} U(x)$ open. 
\end{corollary}
\begin{proof}
  As $U$ is open, $\neg U$ is closed. 
  Hence $\exists_{x:X} \neg U(x)$ is closed. 
  Therefore, $\neg (\exists_{x:X} \neg U(x))$ is open. 
  Furthermore, it is equivalent to $\forall_{x:X} \neg \neg U(x)$, 
  which is equivalent to $\forall_{x:X} U(x)$ by \Cref{rmkOpenClosedNegation}.
\end{proof}

\begin{lemma}
  Let $X:\Chaus$ be presented by $S/\sim$. 
  Then $2^X$ is an open sub-Boolean algebra of $2^S$. 
\end{lemma}
Note that we do not claim $2^X$ is countable presented, 
but by \Cref{OpenSubsetEnumerableAreEnumerable}, it will be enumerable. 
\begin{proof}
  Denote $q:S \twoheadrightarrow X$ for the quotient map. 
  This induces an injection of Boolean algebras $2^X \hookrightarrow 2^S$.
  Note that $a:S\to 2$ lies in $2^X$ iff for all $s,t:S$, we have $a(s) = a(t)$ whenever $s\sim t$.
  Note that $a(s) = a(t)$ is decidable and $s\sim t$ is open, hence 
  $(s\sim t) \to (a(s) = a(t))$ is open (\Cref{ImplicationOpenClosed})
  By \Cref{AllOpenSubspaceOpen}, we conclude that 
  $\forall_{s:S} \forall_{t:S} ((s\sim t) \to (a(s) = a(t)))$ is open. 
  Hence $2^S$ is an open subobject of $2^X$. 
\end{proof}

%
%
%%\subsection{Compact Hausdorff types}
\begin{definition}
  A type $X$ is called compact Hausdorff iff there exists some $S:\Stone$ and some 
  equivalence relation $\sim:S \times S \to \Closed$ such that $X \simeq S / \sim$. 
  We denote $\CHaus$ for the type of compact Hausdorff types. 
\end{definition} 

\begin{lemma}
Let $A\subseteq X$ be a subtype of a compact Hausdorff space. 
Let $S, \sim$ be any presentation of $X$. 
Then $A$ is closed iff it is the image of a closed subtype of $S$ under the quotient map. 
\end{lemma}
\begin{proof}
  If $A$ is closed, then it's pre-image under any map is also closed. 
  In particular for $q:S\to X$ the quotient map, $q^{-1}(A)$ is closed. 
  As $q$ is surjective, we have $q(q^{-1}(A)) = A$,
  hence $A$ is the image of a closed subtype of $S$. 
  Conversely, let $B\subseteq S$ be closed. 
%  Then for any $s:S$, the subtype $\{t:S| B(s) \wedge s \sim t\} \subseteq S$ is closed. 
%  Hence by 
  Define $A\subseteq S$ by 
  $$A(s) := \exists_{s:S} (B(t) \wedge s \sim t).$$
  As $B, \sim$ are closed, by \Cref{ClosedCountableConjunction} and \Cref{InhabitedClosedSubSpaceClosed}, 
  we have that $A$  is closed. 
  Also $A$ respects $\sim$, hence induces a map $A': X\to \Closed$.
  Furthermore, $A(q(s))$ iff $q(s)$ is in the image of $B$. 
  Therefore $A'(x)$ iff $x$ is in the image of $B$. 
\end{proof}
\begin{corollary}
  For $X:\CHaus$ a subtype $A\subseteq X$ is closed iff it is the image of 
  a map $T\to X$ for some $T:\Stone$. 
\end{corollary}
\begin{proof}
  Directly from the above and \Cref{StoneClosedSubsets}.
\end{proof}

\begin{corollary}\label{InhabitedClosedSubSpaceClosed}
  For $X:\CHaus$ and $A\subseteq X$ closed, we have 
  $\exists_{x:X} A(x)$ is closed. 
\end{corollary}
\begin{proof}
  Let $A$ be the image of a map map $T\to X$ for $T:\Stone$. 
  Then $\exists_{x:X} A(x) \leftrightarrow ||T||$, which is closed by \Cref{TruncationStoneClosed}
\end{proof}


\begin{corollary}\label{ClosedDependentSums}
  Closed propositions are closed under dependent sums. 
\end{corollary}
\begin{proof}
  Let $P:\Closed$ and $Q:P \to \Closed$. 
  Then $\Sigma_{p:P} Q(p) \leftrightarrow \exists_{p:P} Q(p)$.
  As $P$ is Stone by \Cref{PropositionsClosedIffStone}, it is also compact Hausdorff, thus
  \Cref{InhabitedClosedSubSpaceClosed} gives that $\Sigma_{p:P} Q(p)$ is closed. 
\end{proof}
\begin{remark}
  Analogously to \Cref{OpenTransitive} and \Cref{OpenDominance}, it follows that 
  closedness is transitive and $\Closed$ forms a dominance. 
\end{remark}
\begin{corollary}\label{AllOpenSubspaceOpen}
  For $U\subseteq X$ an open subset of a compact Hausdorff space, we have 
  $\forall_{x:X} U(x)$ open. 
\end{corollary}
\begin{proof}
  As $U$ is open, $\neg U$ is closed. 
  Hence $\exists_{x:X} \neg U(x)$ is closed. 
  Therefore, $\neg (\exists_{x:X} \neg U(x))$ is open. 
  Furthermore, it is equivalent to $\forall_{x:X} \neg \neg U(x)$, 
  which is equivalent to $\forall_{x:X} U(x)$ by \Cref{rmkOpenClosedNegation}.
\end{proof}

\begin{lemma}
  Let $X:\Chaus$ be presented by $S/\sim$. 
  Then $2^X$ is an open sub-Boolean algebra of $2^S$. 
\end{lemma}
Note that we do not claim $2^X$ is countable presented, 
but by \Cref{OpenSubsetEnumerableAreEnumerable}, it will be enumerable. 
\begin{proof}
  Denote $q:S \twoheadrightarrow X$ for the quotient map. 
  This induces an injection of Boolean algebras $2^X \hookrightarrow 2^S$.
  Note that $a:S\to 2$ lies in $2^X$ iff for all $s,t:S$, we have $a(s) = a(t)$ whenever $s\sim t$.
  Note that $a(s) = a(t)$ is decidable and $s\sim t$ is open, hence 
  $(s\sim t) \to (a(s) = a(t))$ is open (\Cref{ImplicationOpenClosed})
  By \Cref{AllOpenSubspaceOpen}, we conclude that 
  $\forall_{s:S} \forall_{t:S} ((s\sim t) \to (a(s) = a(t)))$ is open. 
  Hence $2^S$ is an open subobject of $2^X$. 
\end{proof}

%%\subsection{Intersection of closed in compact Hausdorff}

\begin{lemma}
  In a compact Hausdorff, closed sets are closed under intersection. 
\end{lemma}
\begin{proof}
  
\end{proof}



\begin{lemma}
  Any Stone space merely is a closed subspace of Cantor space. 
\end{lemma}
\begin{proof}
  Let $S$ be a Stone space, and let it's underlying Boolean algebra $B$ be generated by 
  $(b_n)_{n:\N}$ under quotient of the relations ${\phi_i}_{i:\N}$. 
  Then $S = \{ x: 2^\N | \forall_{i:\mathbb n} x(\phi_i) = 0\}$, 
\end{proof}


\begin{lemma}
  For, $D\subseteq 2^\N$ decidable, $\sim$ a closed equivalence relation on $2^\N$,
  the set $$\{x:S | \exists y : D (x\sim y)\}$$ is closed. 
\end{lemma}
\begin{proof}
  Let $x:S$. We need to show that $\exists (y:S) D(y) \wedge x \sim y$ is a closed proposition. 

  Note first that as $\sim $ is closed, $x \sim \cdot $ is a closed subset of $S$. 
  Therefore, $x\sim \cdot = \bigcap_{n:\N} E_n$ for $(E_n)_{n:\N}$ a
  countable family of decidable subsets of $S$, without losing generality, 
  we may even assume that $E_n \subseteq E_m$ whenever $m\geq n$. 
  We thus need to show that 
  $$
  \exists (y:S) D(y) \wedge (\bigcap_{n:\N} E_n)(y) 
  = 
  \exists (y:S) (\bigcap_{n:\N} D \cap  E_n)(y) 
  $$
  is closed. 
%
  Now we claim that 
  $\exists (y:S) D(y) \wedge E_n(y)$ is closed for all $n:\N$. 
  There merely exists an $m:\N$ such that both $D$ and $E_n$ only consider 
  the first $m$ entries of a sequence. 
  
\end{proof}



\begin{lemma}
  For $S$ Stone, $D\subseteq S$ decidable, 
  $\sim$ a decidable equivalence relation on $S$,
  the set $\{x:S | \exists y : D (x\sim y)\}$ is closed. 
\end{lemma}
\begin{proof}
  Let $B = 2^S$, so $S = Sp(B)$. 
  As $D$ is decidable, 
%  there is some $b:B$ such that $D(y) := (y(b) = 1)$. 
  there is some $n:\N$ such that $D(y)$ only depends on $y|_n$. 

  As $\sim$ is decidable, there is a finite set $I_0\subseteq \N$,
  such that $x\sim y = \prod_{i:I_0} x(i) = y(i)$. 

  Thus 
  $$
   \exists (y : D) (x\sim y) = 
  || \Sigma(y:2^\N) y(b) = 1 \wedge \prod(i:I_0) x(i) = y(i)||
  $$
\end{proof}



\begin{lemma}
  Let $S$ Stone, then $D\subseteq S$ is closed iff 
  $D\subseteq S\subseteq 2^{\N}$ is closed. 
\end{lemma}
\begin{proof}
  Follows immediately from countable intersection of basic clopen. 
\end{proof}




%%  Let $A,B\subseteq X$ be two closed subsets of a compact Hausdorff space $X = S/ \sim$. 
%%  If we know that closed subsets contain are exactly those containing their limits this is very easy right? 
%%  Then any sequence has it's limit both in $A$ and $B$. 
%%\begin{lemma}
%%  Whenever $x_n$ is a convergent sequence, so is $f(x_n)$. 
%%\end{lemma}
%%\begin{proof}
%%  Follows immediately from \Cref{sequenceConvergentIffLimit}.
%%\end{proof}
%%
%%
%%\begin{lemma}
%%  In a compact Hausdorff, whenever a subset $A$ contains all of its limit points, it is closed. 
%%\end{lemma}
%%
%%\begin{proof}
%%  Suppose $A\subseteq X$ contains all of it's limit points. We will show that $f^{-1}(A)$ is closed. 
%%  Let $(x_n)_{n:\N}$ be a sequence in $f^{-1}(A)$ with limit $l$, 
%%  then 
%%  $(f(x_n))_{n:\N}$ is a sequence in $A$ with limit $f(l)$. 
%%  $A$ contains $f(l)$ by assumption. 
%%  Therefore $l\in f^{-1}(A)$. 
%%  Thus every sequence in $f^{-1}(A)$ with a limit has its limit in $f^{-1}(A)$. 
%%\end{proof}
%%
%%\begin{lemma}
%%  In a Stone space, whenever a subset $A$ contains all of its limit points, it is closed. 
%%\end{lemma}
%%\begin{proof}
%%  Let $A \subseteq S$ contain all of it's limit points. 
%%  We will show $A$ is a countable intersection of decidable subsets of $S$, hence closed. 
%%  As $S$ is a subset of Cantor space, we may assume it is Cantor space. 
%%  Thus $A$ is a set of binary sequences. 
%%
%%  We will denote $D_n$ be the set of initial segments of length $n$ occuring in $A$. 
%%  We claim this is well defined, that's not a problem, as it's the image of an operation. 
%%
%%  Counterexample : $A = \{ \overline 0 | p\}$ which contains all of it's limit points
%%  (any sequence in $A$ must be $\overline 0$ constantly, which has a limit if the sequence exists in $A$). 
%%  However, $D_n$ is not decidable. 
%%  Also $A$ is not the intersection of countably many decidable sets I believe. 
%%  Unless off course $p$ is of the form $\alpha=0$, but those are not the only propositions.
%%  For example, the proposition $\beta\neq 0$ cannot be written in that form for general $\beta$. 
%%\end{proof}

%
%\subsection{Open propositions}
%\input{OvertlyDiscrete/FactorizationFin}
%
%\section{Analysis}
%
%\subsection{Convergence}
%\input{Convergence/convergenceClosed}
%\paragraph{Extensional convergence }  
\begin{definition}
  Let $B_\infty$ be the Boolean algebra on countably many generators $(p_n)_{n:\N}$ 
  over the equivalence $p_n\wedge p_m = 0 $ whenever $n \neq m$. 
\end{definition} 
\begin{definition}
  We denote $\Noo$ be the spectrum of $B_\infty$. 
\end{definition} 
\begin{lemma}
  $B_\infty$ is isomorphic with the Boolean algebra of 
  finite/cofinite subsets of $\N$. 
\end{lemma}
\begin{proof}
  To go from $B_\infty$ to subsets of $\N$, we send
  the generators $p_n$ to the singleton $\{n\}$, which are clearly finite. 
  We call the induced Boolean operation $f$. 

  To go from finite/cofinite subsets of $\N$ to $B_\infty$,
  a finite subset $I$ of $\N$ is sent to the element 
  $\bigvee_{i \in I} p_i$, and a cofinite subset $J$ is sent to the element 
  $\bigwedge_{i \in J^C} \neg p_i$.  
  We call this function $g$ and we need to show that $g$ is a Boolean morphism. 
  \begin{itemize}
    \item By deMorgan's laws, $g$ preserves $\neg$. 
    \item To see that $g$ respects $\vee$, we need to check three cases
      \begin{itemize}
        \item If both $I,J$ are finite, then 
        \begin{equation} 
          g(I \cup J) = \bigvee_{i\in I \cup J} p_i= \bigvee_{i\in I} p_i \vee \bigvee_{j\in J} p_j 
        \end{equation}
      \item If both $I,J$ are cofinite, we have
        \begin{equation}
          g(I) \vee g(J) = 
          \bigwedge_{i \in I^C} \neg p_i \vee 
          \bigwedge_{j \in J^C} \neg p_j 
          = 
          \bigwedge_{i\in I^C} 
          \bigwedge_{j \in J^C}(\neg p_i \vee  \neg p_j) 
        \end{equation}
        Now note that $\neg p_i \vee \neg p_j = \neg ( p_i \wedge p_j)$, which 
        is $1$ if $i \neq j$ and $p_i$ if $i =j$. 
        We can leave $1$ out of the meet, and we are left with the intersection of $I^C$ and $J^C$, so
        \begin{equation}
          g(I) \vee g(J) = 
          \bigwedge_{i \in (I^C \cap J^C)} \neg p_i
          = 
          \bigwedge_{i \in (I \cup J)^C} \neg p_i 
        \end{equation} 
        as the union of $I$ and $J$ is also cofinite, this equals 
          $ g( I \cup J)$. 
        \item If $I$ is finite and $J$ cofinite, we have 
        \begin{equation}
        g(I) \vee g(J) = (\bigvee_{i\in I} p_i) \vee (\bigwedge_{j \in J^C} \neg p_j)
        = \bigwedge_{j \in J^C} (\bigvee_{i \in I}( p_i \vee \neg p_j))
        \end{equation}
        If $i\neq j$, then $p_i\wedge p_j = 0$, hence $\neg p_j \geq p_i$ and $p_i \vee \neg p_j  = \neg p_j$
        If $i = j$, then $p_i \vee \neg p_j = 1$.
%        \begin{equation}
%        g(I) \vee g(J) = 
%        = \bigwedge_{j \in J^C} (\bigvee_{i \in I-J}( p_i \vee \neg p_j))
%        \end{equation}
%
%        \item If $I$ is cofinite and $J$ is finite, we have that $I \cup J$ is cofinite.
%        Thus 
%        \begin{equation}
%          g(I \cup J) = \bigwedge_{i \in (I \cup J)^C} \neg p_i
%        \end{equation}
%
      \end{itemize}
    \item The case for $\wedge$ is completely dual to the case for $\vee$. 
  \end{itemize}
We conclude that $g$ is a Boolean morphism. 
Furthermore, $g$ and $f$ are clearly inverses, thus the Boolean algebras are isomorphic. 
\end{proof}

  \begin{lemma}\label{lemBinftyNormalForm}
  Any element of $B_\infty$ can be written as 
  either $\bigvee_{i\in I} p_i$  or
  as $\bigwedge_{j\in J} \neg p_j$ 
  for finite $I,J\subseteq \N$. 
\end{lemma}
\begin{proof}
  Remark that whenever $n \neq m$, we have that 
  $\neg p_n \geq p_m$ as $p_m \wedge p_n = 0$. 
\end{proof}
There is canonical embedding $\N \hookrightarrow \Noo$, 
wich sends $n$ to the unique function $\chi_{n}$ sending $p_n$ to $1$. 
We denote $\infty \in \Noo$ for the function which is constantly $0$. 
By \Cref{PropMarkov}, if an element is not $\infty$, 
it comes from the embedding $\N \hookrightarrow \Noo$. 
\begin{lemma}\label{LemmaOpensContainingInfty}
  Let $U$ be an open subset of $\Noo$ containing $\infty$.
  Then there merely exists an $N:\N$ such that whenever $n\geq N$, 
  $\chi_n\in U$ as well. 
\end{lemma}
\begin{proof}
  It is sufficient to prove the lemma for $U$ a basic open. 
  Assume $b : B_\infty $ is such that 
  $U = \{ \phi: B_\infty \to 2| \phi(b) = 1\}$.
  Assume furthermore that $\infty \in U$.
%  so $U$ contains the function sending every $p_i$ to $0$. 
  by \Cref{lemBinftyNormalForm}, $b$ can have two forms.
  If $b = \vee_{i\in I} p_i$, then as $\infty(b) = 0$, 
  we must have $I = \emptyset$, and thus $b = 0$, 
  which means $U$ is empty, contradicting $\infty\in U$. 
  Therefore, 
  $b$ must be of the form $\wedge_{j \in J} \neg p_j$. 
  Note that for $N = \max J + 1$, whenever $n>J$, 
  $\chi_n$  sends $b$ to $1$. 
  Thus $\chi_n \in U$ as well, and we are done. 
\end{proof}

\begin{definition}
  Let $\alpha$ be a sequence in $X$, we say that $\alpha$
  is convergent iff there exists an extension. 
  \begin{equation}\begin{tikzcd}
    \N \arrow[r, "\alpha"] \arrow[d,hook]  & X \\
    \Noo \arrow[ru,dashed]
  \end{tikzcd}\end{equation}  
\end{definition}  



\begin{proposition}
  A sequence is convergent iff it has a limit
\end{proposition}
\begin{proof}
  Let $\alpha$ be convergent, with extension $\overline \alpha$.
  we claim that $\overline \alpha(\infty)$ is a limit of $\alpha$.
  Let $U \subseteq X$ be an open containing $x$. 
  As $\overline\alpha^{-1}(U)$ is an open subset of $\Noo$ containing $\infty$,
  \Cref{LemmaOpensContainingInfty} tells us there exists some $N$ such that $[N,\infty]\subseteq \overline \alpha^{-1}(U)$. 
  Thus there exists an $N$ such that for $n\geq N$, we have $\alpha(n) \in U$, as required. 

  Conversely, suppose $\alpha$ has limit $x$. 
  Assume $X = Sp(B)$, and let $b\in B$. Then $b$ corresponds to a decidable subset $U\subseteq X$.
  For any decidable subset $U \subseteq X$, we have 
  $\alpha^{-1}(U)$ a decidable subset of $\N$. 
  We claim that $\alpha^{-1}(U)$ is either finite or cofinite. 
  As $U$ is decidable, we can decide wheter $x\in U$. If $x\in U$, $\alpha^{-1}(U)$ is cofinite, as 
  $\alpha(n) \in U$ for all $n \geq N$ for some $N$. 
  If $x\notin U$, we have $x\in U^C$, which is also decidable and therefore $\alpha^{-1}(U^C)$ is cofinite. 
  As $\alpha^{-1}(U) ^ C = \alpha^{-1}(U^C)$, it follows that $\alpha^{-1}(U)$ is finite. 
  Thus $\alpha^{-1}(U)$ is finite or cofinite for any decidable subset $U\subseteq X$. 
  Finite and cofinite subsets of $\N$ correspond to elements of $B_\infty$. 
  Therefore, $\alpha$ induces a map $B \to B_\infty$, which corresponds to a map 
  $\overline \alpha: \Noo \to X$. 

  We claim that $\overline \alpha$ extends $\alpha$. 
  Denote $\iota$ for the map $\N \to \Noo$. 
  We need to show that $\overline \alpha \circ \iota = \alpha$. 
  By definition, we have that $(\overline \alpha \circ \iota)^{-1}(U) = \alpha^{-1}(U)$ 
  for any decidable $U\subseteq X$. 
\end{proof}

\begin{lemma}
  Whenever $S = Sp(B)$ Stone, $f,g: A \to S$, and $f^{-1}(U) = g^{-1}(U)$ for any decidable 
  $U\subseteq S$, we have $f = g$. 
\end{lemma}
\begin{proof}
  By our assumption, we have for all $a:A$ that $f(a) \in U \iff g(a) \in U$ for 
  any decidable $U\subseteq X$. Such $U$ correspond to $b:B$.
  and $f(a) \in U \iff f(a)(b)  = 1$. 
  So the functions $f(a),g(a):B \to 2$ are such that 
  $f(a) (b) = g(a) (b)$ for all $b:B$. 
  This holds for all $a:A$ and by two uses of function extensionality we may conclude 
  $f=g$. 
\end{proof}



%
%\subsection{The interval}
%%The goal of this section is to define the interval $[-2,2]_\mathbb R$ as a scheme. 
We assume $\mathbb N, \mathbb Q$ have been defined in HoTT
with linear propostional order relations $<,\leq, > ,\geq$ playing nicely together 
and standard algebraic operations. 
From these, we can define the subtype $\mathbb Q_{>0}=\sum_{q : \mathbb Q} (q>0)$, 
and the absolute-value function $|\cdot|$ on $\mathbb Q$. 

\begin{definition}
  A pre-Cauchy sequence is a sequence of rational numbers $(q_n)_{n: \mathbb N}$ with $-2 \leq q_n \leq 2$ 
  for all $n:\mathbb N$
%  together with a term of type
  such that for every $\epsilon: \mathbb Q_{>0}$, we have an $N_\epsilon:\mathbb N$, 
  such that whenever $n,m \geq N_\epsilon$, we have 
\begin{equation}
%  \forall \epsilon : \mathbb Q_{>0} \Sigma N : \mathbb N \forall m,n : \mathbb N (m,n \geq N) \to 
  | q_n - q_m | \leq \epsilon
\end{equation} 
\end{definition}

\begin{definition}
Given two pre-Cauchy sequences $p = (p_n)_{n\in\mathbb N}, q=(q_n)_{n\in\mathbb N}$, 
we define the proposition $p \sim_C  q$ as 
%for all $\epsilon : \mathbb Q_{>0}$ there exists an $N :\mathbb N$ such that whenever $n \geq N$, we have
\begin{equation}
  p \sim_C q : = \forall (\epsilon : \mathbb Q_{>0} )\exists ( N :\mathbb N) \forall (n : \mathbb N) ((n \geq N) \to 
  (| p_n - q_n| \leq  \epsilon))
\end{equation}
\end{definition}
Note that $\sim_C$ defines an equivalence relation on pre-Cauchy sequences. 
\begin{definition}
We define the type of Cauchy sequences as the type of pre-Cauchy sequences quotiented by $\sim_C$. 
\end{definition}

%\begin{definition}
%  A binary sequence consists of an initial segment $I \subseteq \mathbb N$
%  and a function $x:I \to 2$. 
%If $I$ is (in)finite, we call the binary sequence (in)finite as well. 
%\end{definition} 
%
%For $x$ a finite binary sequence and $y$ any binary sequence, 
%we'll denote $(x,y)$ for their concatenation, 
%and $\overline x$ for the infinite sequence repeating $x$. 
%
Denote $T = \{-1,0,1\}$. 
\begin{lemma}
  $T^\mathbb N$ is a scheme. 
\end{lemma}
\begin{proof}
  Sketch: partition $2^\mathbb N$ as follows: 
  For $\alpha: 2^\mathbb N$, we'll make a sequence $\beta: T^\mathbb N$.
  consider for each $n$ the $n$'th block of 2 entries in $\alpha$
  if both are $0$, $\beta(n) = 0$. 
  If the first is $1$, $\beta(n) = -1$
  If first is $0$ and the second is $1$, then $\beta(n) = 1$. 
  This is a closed equivalence relation. 
\end{proof} 

Consider the relation $\sim_s$ on $T^{\mathbb N}$, 
such that for any finite binary sequence $x$, we have 
\begin{align}
  (x,1,\overline 0) &\sim_t (x ,0, \overline 1) \\
  (x,-1,\overline 0) &\sim_t (x ,0, \overline {-1})\\
  (x,1,\overline {-1}) &\sim_t (x , \overline 0) \\
  (x,-1,\overline {1}) &\sim_t (x , \overline 0) 
\end{align} 
\begin{lemma}
$\sim_t$ induces a closed equivalence relation on $2^\mathbb N$. 
\end{lemma}
\begin{proof}
  TODO
\end{proof} 

\begin{proposition}\label{propTernaryCauchy}
  $T^\mathbb N/ \sim_t$ is isomorphic to the type of Cauchy sequences. 
\end{proposition} 
\begin{definition}%Construction might be better than definition here, but WIP so who cares. 
  For $\alpha: T^\mathbb N$, define the rational sequence $tri(\alpha)$ by 
  \begin{equation} (tri(\alpha))_n :  = \sum\limits_{0 \leq i \leq n} \frac{\alpha(i)} { 2^{i}} \end{equation}  
  This sequence is pre-Cauchy with $N_\epsilon$ given by the first $n$ with $(\frac12)^n<\epsilon$. 
\end{definition}  
%
%  Also, whenever $\alpha\sim_t \beta$, we have 
%  $tri(\alpha) \sim_C tri(\beta)$. 
%  Therefore $tri$ induces a function from $T^\mathbb N / \sim_t$ to Cauchy sequences. 
\begin{definition}
  Given a pre-Cauchy sequence $p$, 
  we will define a $T$-sequence $\alpha  = c(p): T^\mathbb N$.
  Consider any $i:\mathbb N$, and suppose $\alpha(j)$ has been defined for $0 \leq j<i$. 
%
  Let $\epsilon_i = (\frac12)^{i+1}$. %Placeholder value.
  Define $N_i:= N_{\epsilon_i}$. %is such that for $n,m \geq N_i$, we have $|p_n - p_m| < \epsilon_i$. 
%
  Consider 
  \begin{equation}
    \widetilde p_i = p_N - \sum\limits_{0\leq j < i} \frac {\alpha(j)}{2^{j}}.
  \end{equation}
  As the order on $\mathbb Q$ is total, we can define 
  \begin{equation}
    \alpha(i) = \begin{cases}
    \phantom{-} 1  \text{ if } \widetilde p_i \geq    (\frac12)^{i} \\
    -1             \text{ if } \widetilde p_i \leq  - (\frac12)^{i} \\
    \phantom{-} 0 \text{ otherwise } 
    \end{cases} 
  \end{equation}  
\end{definition} 
We shall now prove the following four things: 
\begin{itemize}
  \item 
    $c(tri(\alpha)) \sim_t \alpha$ for any $\alpha: T^n$.
  \item 
    $tri(c(p)) \sim_C p$ for any pre-Caucy sequence $p$. 
  \item 
    Whenever $p \sim_C q$, we have $c(p)\sim_t c(q)$. 
  \item 
    Whenever $\alpha \sim_t \beta$, we have $tri(\alpha) \sim_C tri(\beta)$. 
\end{itemize}
It follows that $c$ and $tri$ are maps between Cauchy sequences and $T^\mathbb N /\sim_t$ 
which are each other's inverse, proving Proposition \ref{propTernaryCauchy}
\begin{lemma} $tri(c(p)) \sim_C p$ for any pre-Caucy sequence $p$. 
\end{lemma} 



\begin{proof}
  Let $\epsilon>0$ be given, consider $n:\mathbb N$ such that
  $(\frac12)^n < \epsilon$. 
  We claim that for $m\geq N_n$, we have that $|p_m- tri(c(p))_m| < \epsilon$. 

  By definition $p_{N_n} $  
\end{proof} 






%\subsection{The Cauchy reals}
The goal of this section is to introduce the real numbers in a constructive setting, 
following the definition given in \cite{Bishop} with some small adaptations. 
We will later use this definition to show that the interval $[0,1]$ is compact Hausdorff in the sense 
of \Cref{dfnCompactHausdorff}. 

We will assume we are given natural and rational numbers, with decidable (in)equalities
working as expected. 

\begin{definition}
  A \textbf{Cauchy sequence} is a sequence $x : \mathbb N \to \mathbb Q$ such that
  for any $n,m:\mathbb N$, we have %$0\leq x_n \leq 1$ and 
$|x_n-x_m| \leq (\frac12)^n + (\frac12)^m$. 
\end{definition}
\begin{remark}
  If $x$ is a cauchy sequence and $q$ a rational number, the 
  sequence $(x-q)_n = (x_n-q)$ is also Cauchy.
\end{remark}

Following \cite{Bishop}, we define inequality relations between Cauchy sequences and
rational numbers. 
\begin{definition}
  For $x$ a Cauchy sequence and $q$ a rational number, we define 
  \begin{itemize}
%    \item $x> q = \Sigma(n:\mathbb N) x_n > q + {\frac12}^n$. %for some $n:\mathbb N$. 
%    \item $x< q = \Sigma(n:\mathbb N) x_n < q - {\frac12}^n$. %for some $n:\mathbb N$. 
    \item $x\leq  q = \Pi_{n:\mathbb N} x_n \leq q+(\frac12)^n$. 
    \item $x\geq  q = \Pi_{n:\mathbb N} x_n \geq q-(\frac12)^n$. 
  \end{itemize}
\end{definition}
%\begin{lemma}
%  For $x$ Cauchy and $q$ rational, we have that 
%  $x\leq q$ iff for each $n:\mathbb N$, we have a $N_n:\mathbb N$ with 
%  $x_m> q-(\frac12)^n$ for all $m \geq N_n$. 
%\end{lemma}
\begin{lemma}\label{ComparisonLemma}
  For $x$ a Cauchy sequence and $q$ a rational number, we have
  $x \leq q \vee x \geq q$. 
\end{lemma}
\begin{proof}
  For rational numbers, we have decidable inequalities, 
  therefore $\geq 0 \vee q \leq 0$. 
  It follows that 
  $ \forall (n:\mathbb N) \forall (m:\mathbb N) q \geq -(\frac12)^n \vee q \leq (\frac12)^m$. 
  Now by \Cref{TODO}, we may conclude 
  $ (\forall (n:\mathbb N) q \geq -(\frac12)^n ) \vee (\forall (m:\mathbb N) q \leq (\frac12)^m)$
  as required.
\end{proof}


%%%\begin{definition}
%%%  A Cauchy sequence $x$ is \textbf{nonnegative} if $x_n \geq -(\frac12)^n$
%%%  for every $n:\mathbb N$. 
%%%  $x$ is \textbf{nonpositive} if $x_n \leq (\frac12)^n$
%%%  for every $n:\mathbb N$. 
%%%\end{definition} 
%%%%\begin{lemma}
%%%%  A Cauchy sequence is nonnegative iff there exists an $N$ such that $x_n \geq -(\frac12)^N$
%%%%  for all $n\geq N$. 
%%%%  A Cauchy sequence is nonpositive iff there exists an $N$ such that $x_n \leq (\frac12)^N$
%%%%  for all $n\geq N$. 
%%%%\end{lemma}
%%%%\begin{proof}
%%%%  Assume $x$ is nonnegative. Thus for every $n:\mathbb N$, we have $x_n\geq -(\frac12)^n \geq -(\frac12)^0$. 
%%%%  Thus $N$ can taken to be $0$. 
%%%%%
%%%%  Conversely, as $x$ is Cauchy, we have
%%%%  for all $m :\mathbb N$ that  
%%%%%  \begin{equation}- (\frac12)^m -(\frac12)^n \leq    x_m-x_n \leq (\frac12)^m + (\frac12)^n \end{equation}
%%%%  \begin{equation}- (\frac12)^m -(\frac12)^n \leq    x_n-x_m \leq (\frac12)^m + (\frac12)^n \end{equation}
%%%%  If in addition there is an $N$ such that whenever $m\geq N$, we have 
%%%%  $x_m \geq -(\frac12)^N$, so $-x_m \leq (\frac12)^N$, 
%%%%  so $x_n -x_m \leq x_n - (\frac12)^N$. 
%%%%  Therefore, 
%%%%  \begin{equation}- (\frac12)^m -(\frac12)^n \leq    x_n-x_m \leq x_n-(\frac12)^N \end{equation}
%%%%  Thus 
%%%%  \begin{equation}- (\frac12)^m -(\frac12)^n  + (\frac12)^N \leq x_n \end{equation}
%%%%  As $N \geq N$, we have in particular 
%%%%  \begin{equation}- (\frac12)^N -(\frac12)^n  + (\frac12)^N \leq x_n \end{equation}
%%%%  \begin{equation} - (\frac12)^n  \leq x_n \end{equation}
%%%%  thus $x$ is nonnegative. 
%%%%
%%%%  The nonpositive case goes similar. 
%%%%\end{proof}   
%%% 
%%%
%%%\begin{lemma}
%%%  A Caucy sequence is nonnegative or nonpositive. 
%%%\end{lemma}

%\begin{lemma}
%  For any Cauchy sequence $p$, we have 
%  $(\forall (n:\mathbb N) p_n \leq (\frac12)^n) \vee (\forall (n:\mathbb N) p_n \geq -(\frac12)^n)$. 
%\end{lemma}
%\begin{proof}
%We 
%\end{proof}

\begin{definition}
Given two Cauchy sequences $p = (p_n)_{n\in\mathbb N}, q=(q_n)_{n\in\mathbb N}$, 
we define the proposition $p \sim_C  q$ as 
\begin{equation}
  p \sim_C q : = \forall (n,m : \mathbb N) ((| p_n - q_m| \leq  (\frac12)^n + (\frac12)^m))
\end{equation}
\end{definition}

%\begin{remark}
%  Note that $p\sim_C q$ is equivalent to 
%\begin{equation}
%  \forall (n : \mathbb N) | p_n - q_n| \leq  (\frac12)^{n-1}
%\end{equation}
%The equivalence doesn't hold, unless you cut off initial segments (which shouldn't matter). 
%\end{remark} 

\begin{definition}
  The type of \textbf{Cauchy reals} is given by 
  the type of Cauchy sequences modulo $\sim_C$.
\end{definition}

We claim that the inequality in \Cref{TODO} extends to a well-defined 
inequality between Cauchy reals and rational numbers. 

Furthermore, we claim that 
$\Pi_{x:\mathbb R} \Pi_{q:\mathbb Q} x \leq q \vee x \geq q$. 

%\begin{lemma}
%  For any Cauchy real $r$ any Cauchy sequence $p$ representing $r$, 
%  we have 
%  \begin{equation}
%    (\forall (n:\mathbb N) p_n \leq (\frac12)^n) \vee (\forall (n:\mathbb N) p_n \geq (\frac12)^n)
%  \end{equation}
%
%\end{lemma}

\begin{definition}
  A Cauchy sequence in the interval is a Cauchy sequence $x$ such that 
  for any $n:\mathbb N$, we have $0\leq x_n \leq 1$. 
 % 
  The interval of Cauchy reals is given by the type of Cauchy sequences in the interval 
  modulo $\sim_C$. We denote it by $[0,1]$. 
\end{definition}  


We want to show that the interval of Cauchy reals is Compact Hausdorff. 
Informally, to any binary sequence $\alpha : \mathbb N \to 2$, 
we can associate a Cauchy sequence 
\begin{equation}\label{eqnBinaryEncode}
  n\mapsto \sum\limits_{i = 0 }^n \frac {\alpha(i)}{2^{i+1}}
\end{equation}
and we are going to give a closed relation on Cantor space such that 
two binary sequences are equivalent iff they correspond to the same Cauchy reals. 
%
First, we'll need some notation.
\begin{definition}
Given a binary sequence $\alpha:\mathbb N \to 2$ and a natural number $n : \mathbb N$  
we denote $\alpha|_n: \mathbb N_{\leq n} \to 2$ for the 
restriction of $\alpha$ to a finite sequence of length $n$. 
We denote $\overline 0, \overline 1$ for the binary sequences which are constantly $0$ and $1$ respectively. 
We denote $0,1$ for the sequences of length $1$ hitting $0,1$ respectively. 
If $x$ is a finite sequence and $y$ is any sequence, denote $x\cdot y$ for their concatenation. 
\end{definition} 
Now we'll give a definition for when two finite binary sequences of length $n$ correspond 
to real numbers whose distance is $\leq (\frac12)^n$.
Basically, we want for every finite sequence $z$ that 
$(z \cdot 0 \cdot \overline 1)$ and  $(z \cdot 1 \cdot \overline 0)$ are equivalent. 

\begin{definition}
Now let $n:\mathbb N$ and $x,y:\mathbb N_{\leq n} \to 2$ be two sequences of length $n$. 
We say $x,y$ are near if we have an $m:\mathbb N$ with $m\leq n$
and some $a: \mathbb N_{\leq m} \to 2$, 
such that one of $(a \cdot 0 \cdot \overline 1)|_n,  ( a \cdot 1 \cdot \overline 0)|_n$
is equal to $x$ and the other is equal to $y$. 
We denote $\text{near}_n(x,y)$ if $x,y$ are near. 
%
To be precise, we define 
\begin{equation}
  \text{near}_n(x,y) = 
\Sigma(m:\mathbb N) m \leq n \wedge 
  \Sigma (a : Fin_m \to 2) 
\bigg( \big( (x,y) = 
((a \cdot 0 \cdot \overline 1)|_n,  ( a \cdot 1 \cdot \overline 0)|_n)
\big)
\bigvee 
\big(
  (y,x) = 
((a \cdot 0 \cdot \overline 1)|_n,  ( a \cdot 1 \cdot \overline 0)|_n)
\big)
\bigg)
\end{equation}
\end{definition}
\begin{remark}
Remark that when $x,y$ are near, $m$ and $a$ as above are unique. 
Thus $\text{near}_n(x,y)$ is a proposotion. 
%
Furthermore, to check whether $x,y$ are near, we need only make $n$ comparisons, 
thus $\text{near}_n(x,y)$ is decidable. 
%
Note that in the above definition, we allow $m = n$ and therefore $x$ is near to itself for any finite sequence $x$. 
Furthermore, we have defined nearness to be symmetric. 
However, it is not a transtive relation. 
After all, the sequence $010$ and $011$ are near and the sequence $011$ and $100$ are near, 
but $010$ is not near to $100$. This corresponds to the fact that $\frac14$ and $\frac38$ are distance $\leq (\frac12)^3$
apart, and so are $\frac38$ and $\frac12$, but $\frac14$ and $\frac12$ are not. 
\end{remark}
\begin{definition}
  We define the following relation on Cantor space for $\alpha, \beta: 2^\mathbb N$.
  \begin{equation}
    \alpha \sim_t \beta = \forall (n : \mathbb N) 
    \text{near}_n(\alpha|_n, \beta|_n)
  \end{equation}
\end{definition}
\begin{lemma}
  $\sim_t$ is a closed equivalence relation. 
\end{lemma}
\begin{proof}
   Let $\alpha, \beta, \gamma : 2^\mathbb N$. 
   As the dependent product of propositions is a proposition, $\alpha \sim_t\beta$ is a proposition. 
   %
   Furthermore, the closedness follows from decidability of $\text{near}_n(\alpha|_n, \beta|_n)$. 
   One could define $\gamma(n) = 1$ iff $\text{near}_n(\alpha|_n, \beta|_n)$
   
   As nearness is reflexive and symmetric, so is $\sim_t$. 

   Now suppose $\alpha \sim_t \beta$ and $\beta\sim_t \gamma$. 
   We claim that $\alpha \sim_t \gamma$. 

   Let $n:\mathbb N$, we need to show that 
   $\text{near}_n(\alpha|_n , \gamma|_n)$. 
   Let $(a,m)$ witness that $\text{near}_n(\alpha|_n, \beta|_n)$.
   and let $(b, k)$ witness that $\text{near}_n(\beta|_n, \gamma|_n)$
   We will make a case distinction on whether one of $m,k$ is equal to $n$, or
   both are strictly smaller than $n$. 
   \begin{itemize}
     \item 
       If $m=n$, we have that $\alpha|_n = \beta|_n$, and therefore 
       \begin{equation}
         \text{near}_n(\beta|_n, \gamma|_n) \leftrightarrow \text{near}_n(\alpha|_n, \gamma|_n)
       \end{equation} 
       The above also holds if $k = n$.
     \item 
       If $m< n$, we have that $\alpha(m+1) \neq \beta(m+1)$, thus 
       $\alpha|_l \neq \beta|_l$ for all $l>m$, 
       but we still have $\text{near}_l(\alpha|_l, \beta|_l)$ for these $l$. 
       Therefore $(\alpha, \beta)$ or $(\beta, \alpha)$ must be of the form
       $(a \cdot 0 \cdot \overline 1, a \cdot 1 \cdot \overline 0)$. 
       WLOG, we assume $\alpha = a \cdot 0 \cdot \overline 1$, and thus 
       $\beta = a \cdot 1 \cdot \overline 0$ (if not, we could do bitflips). 

       As $k<n$ also, by the same argument there is some $b$ such that one of 
       $(\beta,\gamma), (\gamma, \beta)$
       is equal to $(b\cdot 0 \cdot \overline 1, b \cdot 1 \cdot \overline 0)$. 
       However, $\beta$ is also of the form $a \cdot 1 \cdot \overline 0$, and 
       thus cannot also be of the form $b \cdot 0 \cdot \overline 1$. 
       Therefore we must have 
       $\beta = b\cdot 1 \cdot \overline 0$ and 
       $\gamma= b\cdot 0 \cdot \overline 1$. 

       But now $b \cdot 1 \cdot \overline 0 = a \cdot 1 \cdot \overline 0$, 
       The lengths of $a,b$ cannot be unequal, and by decidablity of natural numbers, 
       $a,b$ have the same length and it follows that $ a = b$. 
       Therefore $ \alpha = \gamma$, so $\alpha \sim_t\gamma$.
   \end{itemize}

   We conclude that $\sim_t$ is a closed equivalence relation. 
\end{proof}


\begin{lemma}
  $b$ sends $\sim_n$ equivalent binary sequences to $\sim_C$ equivalent Cauchy sequences. 
\end{lemma}
\begin{proof}
  Let $\alpha, \beta$ be binary sequences.
  We claim that $|b(\alpha)_n - b(\beta)_n| \leq (\frac12)^{n+1}$ 
  whenever $\text{near}_n(\alpha, \beta)$. 
  It will follow that if $\alpha\sim_n \beta$, then 
  $b(\alpha)\sim_C b(\beta)$. 

  Let $n:\mathbb N$ and assume $m:\mathbb N$ with $m\leq n$ and 
  let $z$ be a sequence of length $m$ such that 
  $\alpha|_n = z\cdot 1 \cdot \overline 0|_n$ and $\beta|_n = z \cdot 0 \cdot \overline q |_n$. 
  then $b(\alpha)_n = \sum_{i\leq m} \frac{z(i)}{2^{i+1}} + (\frac12)^{m+2}$ and 
  $b(\beta)_n = \sum_{i\leq m} \frac{z(i)}{2^{i+1}} + \sum\limits_{m+2 \leq i \leq n}(\frac12)^{i+1}$. 
  Thus 
  $b(\alpha)_n - b(\beta)_n = (\frac12)^{m+2} - \sum\limits_{m+2 \leq i \leq n}(\frac12)^{i+1} = 
  (\frac12)^{n+1}$, 
  which is smaller than required. 
\end{proof}  

\begin{lemma}
  Whenever $b(\alpha) \sim_C b(\beta)$, 
  we have $\alpha \sim_n \beta$. 
\end{lemma}
\begin{proof}
  Assume $b(\alpha) \sim_Cb (\beta)$. 
  Let $n:\mathbb N$. 
  We shall show that $\text{near}_n(\alpha , \beta)$. 

  As we're only checking finitely many entries, 
  we either have $\alpha|_n = \beta|_n$, 
  or there exists a smallest $m\leq n$ with 
  $\alpha(m) \neq \beta(m)$. 

  If $\alpha|_n = \beta|_n$, we have $\text{near}_n(\alpha,\beta)$ and are done. 
  WLOG assume $\alpha(m) = 1, \beta(m) = 0$ for $m$ minimal. 
  We claim that for any $k\geq m$, we have 
  $b(\alpha)_k - b(\beta)_k = (\frac12)^k$. 

  Now note that 
  \begin{equation} 
    b(\alpha)_{k+1} - b(\beta)_{k+1} = 
    b(\alpha)_{k} - b(\beta)_{k} + 
    \frac{\alpha(k+1) - \beta(k+1)}{2^{k+2}}.
  \end{equation}




  For $k>m$, we have that 
  \begin{equation}
  |b(\alpha)_k - b(\beta)_k |= 
  |(\frac12)^{m+1} + \sum\limits_{i=m+1}^k \frac{ \alpha(i) -\beta(i)}{2^{i+1}}|. 
  \end{equation}
  Note that the right summand is always $\leq (\frac12)^{m+1}$. 
  Therefore, we can leave out the absolute value function. 
  As $b(\alpha) \sim_Cb(\beta)$, we have that 
  \begin{equation}
  (\frac12)^{m+1} + \sum\limits_{i=m+1}^k \frac{ \alpha(i) -\beta(i)}{2^{i+1}} \leq (\frac12)^{k-1}
  \end{equation}
  Note that $\alpha(i) -\beta(i) \in \{-1,0,1\}$ always. 
  Also, 

  Denote $z = \alpha|_{m-1} = \beta_{m-1}$. 
  We will show that for $n\geq i>m$, we must have $\alpha(i) = 0, \beta(i) = 1$. 
  Suppose $\alpha(i) \neq 0$. 
\end{proof}

%
%
%\begin{theorem}
%  The interval of Cauchy reals is isomorphic to $2^\mathbb N / \sim_t$. 
%\end{theorem} 
%\begin{proof}
%  Define $b: 2^\mathbb N \to [0,1]$ by composing the map in \Cref{eqnBinaryEncode} with the projection map 
%  from Cauchy sequences in the interval to the interval in Cauchy reals. 
%  We will show $b$ is surjective, 
%  and such that $b(\alpha) = b(\beta)$ iff  $\alpha \sim_t \beta$. 
%  \begin{itemize}
%    \item For $b$ to be surjective, we need to show that for any Cauchy sequence $p$ in the interval, there 
%      merely exists some binary sequence $\alpha$ such that $b(\alpha)\sim_C p$. 
%
%      We will inductively define $\alpha$. 
%      Let $\alpha(i)$ be defined for $i<n$. 
%      Consider the sequence $p'(j) = p(j) - \sum\limits_{i<n} \frac{\alpha(i)}{2^{i+1}}$. 
%      As $p$ is a Cauchy sequence, so is $p'$. 
%  \end{itemize}
%\end{proof}
%

%%\printindex
%
%\section{Directed Univalence}
%\subsection{Subquotient systems}

\begin{definition}
A subquotient pre-system consists of $X$ a type and $U$ a class of propositions.
\end{definition}

\begin{definition}
For $(X,U)$ a subquotient pre-system, we define:
\[Sub_{X,U} = \sum_{Y:\Type} \exists (P : X\to U).\ Y = \Sigma_XP\]
\[SubQ_{X,U} = \sum_{Y:\Type} \exists (P : X\to U)(R: \Sigma_XP\to \Sigma_XP\to U\ \mathrm{equivalence\ relation}).\ Y = (\Sigma_XP)/R\]
\end{definition}

\begin{definition}
We say a class of types $T$ has local choice if for all $X\in T$ and $P:X\to\Type$ such that:
\[\prod_{x:X}\propTrunc{P(x)}\]
there merely exists $Y\in T$ and a surjection:
\[f:Y\to X\]
such that:
\[\prod_{y:Y}P(f(y))\]
\end{definition}

\begin{proposition}\label{lex-sub-pro}
Assume $(X,U)$ a Subquotient pre-system such that:
\begin{itemize}
\item Identity types in $X$ are in $U$.
\item $U$ is closed by $\sum$ and $\top$.
\item $\propTrunc{X}$ and $X\times X = X$.
\item $Sub_{X,U}$ has local choice.
\end{itemize}
Then we $SubQ_{X,U}$ has the following:
\begin{itemize}
\item Stability under finite limits.
\item Stability by quotient by equivalence relation with value in $U$.
\item Local choice.
\end{itemize}
\end{proposition}

\begin{proof}
\begin{itemize}
\item First we check that $SubQ_{X,U}$ has local choice. Since we assume that $Sub_{X.U}$ has local choice and that any type in $SubQ_{X,U}$ is covered by a type in $Sub_{X,U}$ by definition, it is enough to check that $Sub_{X,U}\subset SubQ_{X,U}$ to conclude. But given $S = \Sigma_XP$ in $Sub_{X,U}$ we have that:
\[\Sigma_XP = (\Sigma_XP)/L\]
where:
\[L((x,_),(y,_))= (x=_Xy)\]
and since $x=_Xy$ is assumed to be in $U$ we conclude.

\item Stability by quotient by equivalence relation with value in $U$ is clear.

\item Now we want to check stability under finite limits.

First we check that $U\subset SubQ_{X,U}$. Indeed assume $P\in U$, then with $L$ the trivial relation we have:
\[(X\times P) / L = \propTrunc{X\times P} = P\]
as $\propTrunc{X} = 1$ so that since $\top\in U$ we conclude $P\in SubQ_{X,U}$.

This means that $SubQ_{X,U}$ is stable by identity type, and that $1\in SubQ_{X,U}$.

All that is left is to check stability under $\Sigma$. Assume $S: SubQ_{X,U}$ and $T:S\to SubQ_{X,U}$. Through the fact that $S$ is covered by a type in $Sub_{X,U}$ and local choice for $Sub_{X,U}$ we merely get $S':Sub_{X,U}$, say $S'=\Sigma_XP$ and a surjective map:
\[f:S'\to S\]
such that for all $x:\sum_XP$ we have:
\[T(f(x)) = (\Sigma_XP_x)/R_x\]
so we get a surjective map:
\[\sum_{x:X}\sum_{P(x)}(\Sigma_X P_x)/R_x  \to \sum_ST\]
Then the identity types in $\sum_ST$ are in $U$ as $U$ is stable by $\Sigma$, so it is enough to check that:
\[\sum_{x:X}\sum_{P(x)}(\Sigma_X P_x)/R_x\]
is in $SubQ_{X,U}$ to conclude, as we can then apply the previous bullet-point. But this type is equivalent to:
\[\left(\sum_{(x,x'):X\times X}\Sigma_{P(x)}P_x(x')\right)/ L\]
where:
\[L((x,x'),(y,y')) =\sum_{x=y} R_y(x',y') \]
which is in $SubQ_{X,U}$ as $U$ is stable by $\Sigma$, $x=y$ in in $U$ and $X\times X = X$.
\end{itemize}
\end{proof}

\begin{proposition}\label{coproducts-sub-quo}
Assume $(X,U)$ a subquotient pre-system such that $\bot\in U$ and $X+X = X$. Then $SubQ_{X,U}$ is stable by finite coproducts.
\end{proposition}

\begin{proof}
We have that:
\[\bot = (X\times \bot) / L\]
where $L$ is the unique such equivalence relation. Since $\bot\in U$ we conclude that $\bot\in SubQ_{X,U}$.

Given $S$ and $S'$ in $SubQ_{X,U}$, say:
\[S = (\sigma_XP)/R\]
\[S' = (\sigma_XP')/R'\]
Then we have that:
\[S+S' = \left(\sum_{X+X}[P,P']\right) L\]
where:
\[[P,P'](l(x)) = P(x)\]
\[[P,P'](r(x)) = P'(x)\]
and:
\[L(l(x),l(y)) = R(x,y)\]
\[L(l(x),r(y)) = \bot\]
\[L(r(x),l(y)) = \bot\]
\[L(r(x),r(y)) = R'(x,y)\]
Since $\bot\in U$ and $X+X=X$ we conclude that $S+S'$ is in $SubQ_{X,U}$.
\end{proof}

\begin{proposition}\label{prop-sub-quo}
Assume $(X,U)$ a subquotient pre-system such that $\top\in U$ and for all $S\in Sub_{X,U}$ we have that $\propTrunc{S}\in U$. Then any proposition in $SubQ_{X,U}$ is in $U$. 
\end{proposition}

\begin{proof}
If we have a proposition $S$ in $SubQ_{X,U}$, say:
\[S = (\Sigma_XP)/R\]
then:
\[S = \propTrunc{S} = \propTrunc{\Sigma_XP}\]
and we can conclude.
\end{proof}

\begin{definition}
A subquotient system is a subquotient pre-system obeying the hypothesis of \cref{lex-sub-pro}, \cref{coproducts-sub-quo} and \cref{prop-sub-quo}.
\end{definition}

We just pack all this up in one theorem:

\begin{theorem}\label{stabitity-sub-quo}
Let $(X,U)$ be a subquotient system, then $SubQ_{X,U}$ has the following:
\begin{itemize}
\item Stability under finite limits.
\item Stability under finite coproducts.
\item Stability under quotient by equivalence relations.
\item Local choice.
\end{itemize}
\end{theorem}

We have two main examples in mind.

\begin{example}
The subquotient pre-system $St = (2^\N,\mathrm{Closed})$ is a quotient system. We have that $Sub_{St}$ is the type of stone spaces, and $CHaus = SubQ_{St}$ the type of compact Haussdorf spaces.

Closed propositions are stable by $\Sigma$. TODO 

We also need that for any stone space $S$ we have that $\propTrunc{S}$ is a closed proposition. TODO
\end{example}

\begin{example}
The subquotient pre-system $Od = (\N,\mathrm{Open})$ is a quotient system. We have that $ODisc = SubQ_{Od}$ the type of so-called overtly discrete types.

A key observation is that open propositions are stable by countable disjunctions.

This means open propositions are stable by $\sum$ because we can assume:
\[P = \Sigma_{n:\N} A(n)\]
with $A(n)$ decidable and:
\[Q:P \to \mathrm{Open}\]
Then we have that:
\[\Sigma_PQ = \exists(n:\N).\ \Sigma_{A(n)} Q(n)\]
which is open as $\Sigma_{A(n)} Q(n)$ is open for all $n$, as $A(n)$ is decidable.

Types in $Sub_{Od}$ even have full choice because both $\N$ and decidable propositions have full choice.
\end{example}

So both $ODisc$ and $CHaus$ enjoys the conclusion of \cref{stabitity-sub-quo}.


\subsection{Tychonov}

\begin{proposition}\label{tychonov}
Assume $(X,U)$ and $(Y,C)$ two subquotient system such that:
\begin{itemize}
\item $S\to Y$ is in $SubQ_{Y,C}$ for all $S:Sub_{X,U}$.
\item If $P\in U$ and $Q:P\to C$ then $\prod_{p:P}Q(p) \in C$.
\item If $Q:X\to C$ then $\prod_{x:X}Q(x) \in C$.
\end{itemize}
Then we have the following:
\begin{itemize}
\item If $S:SubQ_{X,U}$ and $T:S\to SubQ_{Y,C}$, then:
\[\prod_{s:S}T(s) \in SubQ_{Y,C}\]
\end{itemize}
\end{proposition}

\begin{proof}
Note that for $S':Sub_{X,U}$ and $Q:S'\to C$ we have that:
\[\prod_{S'}Q\]
is in $C$.

Now we use local choice to get $S':Sub_{X,U}$ with a surjective map:
\[f:S'\to S\]
such that for all $s:S'$ we have:
\[T(f(s)) = (\Sigma_YU_s)/R_s\]

Then the map:
\[\prod_ST \to \prod_{s:S'}(\Sigma_YU_s)/R_s\]
is an embedding, its fiber over $\alpha$ is:
\[\prod_{s,t:S'} \prod_{f(s) =_S f(t)} \alpha(s) = \alpha(t)\]
which is in $C$ by the hypothesis. Therefore it is enough to prove that:
\[\prod_{s:S'}(\Sigma_YP_s)/R_s\]
is in $SubQ_{Y,C}$. 

But this type is the quotient of:
\[\prod_{s:S'}(\Sigma_YP_s)\]
by:
\[L(\alpha,\beta) = \prod_{s:S'} R_s(\alpha(s),\beta(s))\]
which is in $C$, therefore it is enough to to prove that:
\[\prod_{s:S'}(\Sigma_YP_s)\]
is in $SubQ_{Y,C}$.

But this type is equivalent to:
\[\sum_{f:S'\to Y} \prod_{s:S'}P_s(f(s))\]
Since $\prod_{s:S'}P_s(f(s))$ is in $C$, it is enough to prove that:
\[S'\to Y\]
is in $SubQ_{Y,C}$. But this is one of the hypothesis.
\end{proof}

\begin{definition}
Two subquotient systems $A,B$ are called dual if both $(A,B)$ and $(B,A)$ satisfy the hypothesis of \cref{tychonov}.
\end{definition}

\begin{example}
We have that $St = (2^\N,\mathrm{Closed})$ and $Od = (\N,\mathrm{Open})$ are dual quotient systems.

\begin{itemize}
\item We need that if $P$ open and $Q:P\to \mathrm{Closed}$, then $\Sigma_PQ$ is closed. Assume $Q=\Sigma_{n:\N}A(n)$ with $A(n)$ decidable, since open propositions have choice we can assume for $n:\N$ such that $A(n)$ that $Q(n) = \forall_{k:\N} B_n(k)$ with $B_n(k)$ decidable. Then:
\[\Sigma_PQ = \prod_{n,k:\N} \prod_{A(n)} B_n(k) \]
which is indeed closed.

It is clear that closed propositions are closed by countable products.

$\Sigma_\N P\to 2^\N$ is compact Hausdorff? Yes indeed, it is even Stone because it is equivalent to:
\[\prod_{k,n:\N} 2^{P(n)}\]
and $2^{P(n)}$ is Stone as $P(n)$ is open, indeed:
\[(\Sigma_\N A)\to 2\] 
for $A$ decidable is a countable product of Stone space, as $2^A$ is Stone for $A$ decidable.

\item First we check that given $S$ Stone, we have that:
\[S\to \N\]
is overtly discrete. Indeed identity types in $S\to \N$ are closed and there is a surjection from fundamental systems of idempotent in $2^S$ to $S\to \N$, so it is enough to prove that the type fundamental systems of idempotent in a c.p. algebra is overtly discrete. To have this it is enough to prove that countably presented algebra are overtly discrete. TODO
\end{itemize}
\end{example}

When applying \cref{tychonov} to $ODisc$ and $CHaus$ we get Tychonov theorem and its dual.


\subsection{Factorisation}



\subsection{Scott continuity}


%
%\appendix
%\section{Appendix}
%\subsection{Rank of matrices}

\begin{definition}
A matrix is said to have rank $\leq n$ if all its $n+1$-minors are zero. It is said to have rank $n$ if it has rank $\leq n$ and does not have rank $\leq n-1$.
\end{definition}

Having a rank is a property of matrices, as a rank function defined on all matrices would allow to e.g. decide if an $r:R$ is invertible.

\begin{lemma}\label{rank-bloc-matrix}
Assume given a matrix $M$ of rank $n$ decomposed into blocks:
\[M = \begin{pmatrix}
P & Q  \\
R & S \\
\end{pmatrix}\]
Such that $P$ is square of size $n$ and invertible. Then we have:
\[S = RP^{-1}Q\]
\end{lemma}

\begin{proof}
By columns manipulation the matrix is equivalent to:
\[M = \begin{pmatrix}
P & Q  \\
0 & S - RP^{-1}Q \\
\end{pmatrix}\]
but equivalent matrices have the same rank so $S=RP^{-1}Q$.
\end{proof}

\begin{lemma}\label{rank-equivalent-definitions}
If a linear map $R^m \to R^n$ given by multiplication with $M$
has finite free kernel of rank $k$, then $M$ has rank $m-k$.
\end{lemma}

\begin{proof}
  Let $a_1,\dots,a_{k}$ be a basis for the kernel of $M$ in $R^m$, which we complete into a basis of $R^m$ via $b_{k+1},\dots,b_m$.
  By completing $Mb_{k+1},\dots, Mb_m$ to a basis of $R^n$, we get a basis where $M$ is written as:
\[\begin{pmatrix}
I_{m-k} & 0  \\
0 & 0 \\
\end{pmatrix}\]
so that $M$ has rank $m-k$.
\end{proof}

%\begin{definition}
%Two matrices $M,N$ are said equivalent if there are invertible matrices $P,Q$ such that $M = PNQ$.
%\end{definition}

%It is clear that equivalent matrices have the same rank.

%\begin{lemma}\label{rank-equivalent-definitions}
%Assume given a matrix:
%\[M : R^m\to R^k\]
%Then the following are equivalent:
%\begin{enumerate}[(i)]
%\item $M$ has rank $n$.
%\item The kernel of $M$ is equivalent to $R^{m-n}$.
%\item The image of $M$ is equivalent to $R^n$.
%\item $M$ is equivalent to the bloc matrix:
%\[\begin{pmatrix}
%I_n & (0)  \\
%(0) & (0) \\
%\end{pmatrix}\]
%\end{enumerate}
%\end{lemma}

%\begin{proof}
%\end{proof}

%
\printbibliography
%
\end{document}
