In this section, we will show that the topology on $I$ as one would expect. 

Inspired by \Cite{Bishop}, we define inequality on $I$ as follows:
\begin{definition}
  Let $\alpha,\beta:2^\N$. 
  We define $\alpha\leq_I \beta$ and $\alpha<_I\beta$ as follows:
  \begin{align}
  \alpha\leq_I\beta : = \forall_{n:\N} ( cs(\alpha)_n \leq cs(\beta)_n + \frac {1} {2^n})\\ 
    \alpha   <_I \beta : = \exists{n:\N} ( cs(\alpha)_n < cs(\beta)_n - \frac {1} {2^n})
\end{align}
\end{definition}

\begin{lemma}
  $\leq_I$ respects $\sim_I$. 
\end{lemma}
\begin{proof}
  Let $\alpha,\beta,\gamma:2^\N$ and suppose $\alpha\sim_I\beta$ and $\alpha \leq_I\gamma$.
  We will show that $\beta\leq_I\gamma$. 
  
  Let $n:\N$, we will show that $cs(\beta)_n \leq cs(\gamma)_n + \frac{1}{2^n}$. 
  If $\alpha|_n = \beta|_n$, this is trivial.
  Note that $cs (u 0 \overline 1)_n \leq  cs( u 1 \overline 0)_n$ for all $n:\N$, and all finite sequences $u$. 
  Thus we need only consider the case that there is some $m<n$ and finite sequence $u:2^m$ with 
  $\alpha|_n = (u 0 \overline 1)|_n, \beta|_n = (u 1 \overline 0)|_n$.
  In this case, 
  $cs(\beta)_n = cs(u)_m + \frac 1 {2^m}$, 
  and 
  $$cs(\alpha)_n = 
                   cs (u)_m + \sum_{i = m}^{n-1} \frac 1 {2^{i+1}} 
                   = cs(u)_m + \frac 1 {2^m} - \frac{1}{2^n}
                 = cs(\beta) - \frac{1} {2^n}
  $$.

  
  


\end{proof}
\begin{lemma}
  $<_I$ respects $\sim_I$. 
\end{lemma} 

\begin{remark}
  By the above, $\leq_I, <_I$ induce relations $\leq,<$ on $I$.
  As inequality in $\mathbb Q$ is decidable, $\leq, <$ are closed and open respectively. 

  Let $a,b:I$. Following standard notation, we denote
  $[a,b]$ for $\Sigma_{x:I} (a\leq x \wedge x \leq b)$, which is closed by \Cref{ClosedCountableConjunction}, 
  we call subsets of $I$ of this form closed intervals. 


  We also denote $(a,b)$ for $\Sigma_{x:I} (a < x \wedge x < b)$, which is open by \Cref{OpenFiniteConjunction}.
  we call subsets of $I$ of this form open intervals. 
\end{remark}

\begin{lemma}
  Every closed $U\subseteq I$ can be written as finite union of closed intervals.
\end{lemma}
\begin{proof}
  From \Cref{CompactHausdorffClosed} and \Cref{StoneClosedSubsets}, it follows that for every closed $U\subseteq I$, 
  there are some decidable subsets $(D_n)_{n:\N}$ such that $U = p(\bigcap_{n:\N} D_n)$ for $p:2^\N\to I$ the quotient map. 
  We may assume that $D_{n+1} \subseteq D_n$ for $n:\N$. 
  By Stone duality, each $D_n$ corresponds to some element of $C$. 

  We will use induction on the structure of the elements of $C$ to show that $p(D_n)$ is a finite union of closed intervals. 
  If $p$ is a generator $g_n$, it corresponds to the set of sequences with a $1$ at $n$


\end{proof}


\begin{corollary}
  Every open $U\subseteq I$ can be written as countable union of open intervals.
\end{corollary} 

\begin{remark}
  It follows that the topology of $I$ is generated by open intervals, 
  which corresponds to the standard topology on $I$. 
  Hence our notion of continuity corresponds with the $\epsilon,\delta$-definition of continuity one would expect. 
  Thus every function $f:I\to I$ in the system we presented is continuous in the $\epsilon,\delta$-sense. 
\end{remark}
