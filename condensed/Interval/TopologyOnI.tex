In this section, we will show that the topology on $I$ as one would expect. 

\begin{definition}
  For $n:\N$ we define 
  $cs_n:2^n \to \mathbb Q$ by 
  \begin{equation}
    cs_n(a) = \sum\limits_{i=0}^{n-1} \frac{a(i)} {2^{i+1}}
  \end{equation}
  And for $\alpha:2^\N$, we define the sequence $cs(\alpha) : \N \to \mathbb Q$ by 
  \begin{equation}
    cs(\alpha)_n = cs_n(\alpha|_n)
  \end{equation}
\end{definition}
\begin{remark}
  One can check that for each $\alpha:2^\N$, 
  $cs(\alpha)$ is a Cauchy sequence. 
  Thus $cs$ gives a map from Cantor space to the Cauchy reals. 
%  In \Cref{AppendixCauchyEquivalenceProof}, we show that 
  We claim without proof that
  this map induces an equivalence from $I$ to the interval of Cauchy reals. 
\end{remark}
\begin{lemma}
  For each $n:\N$, $cs_n$ is injective. 
\end{lemma}  
\begin{proof}
  \rednote{This is just copied from below, suggesting a more general lemma}
  Assume $cs_n(s) = cs_n(t)$. As equality of finite sequences is decidable, 
  we are satisfied if we assume $s\neq t$ and find a contradiction. 
  We may without loss of generality assume there is some $m<n$ and some $u:2^m$ such that 
  \begin{equation}
    (s|_{m+1} = u \cdot ) \wedge ( t|_{m+1} = u \cdot 1) . 
  \end{equation}
  Then 
  WLOG, we assume that $s(m) = 1, t(m) = 0$. 
  We thus have that 
  \begin{align}
    cs_n(s) &= 
    cs_m(u) + \frac1{2^{m+1}} + \sum\limits_{i = m+1}^{n-1} \frac{s(i)}{2^{i+1}}\\
    cs_n(t) &= 
    cs_m(u) + 0  + \sum\limits_{i = m+1}^{n-1} \frac{t(i)}{2^{i+1}}
  \end{align}
  And thus 
  \begin{align}
    cs_n(s)-cs_n(t) = \frac{1}{2^{m+1}} + \sum\limits_{i = m+1}^{n-1} \frac{s(i)-t(i)}{2^{i+1}}
  \end{align}
  Note that as $s(i),t(i) \in \{0,1\}$, we must have that $s(i) -t(i) \in \{-1,0,1\}$. 
  Therefore 
  $$\sum\limits_{i = m+1}^{n-1} \frac{s(i)-t(i)}{2^{i+1}}$$
  is minimal iff $s(i) -t(i) = -1$ for all $m<i<n$. 
  In that case, we have that 
  $$
  \frac{1}{2^{m+1}}-
  \sum\limits_{i = m+1}^{n-1} \frac{s(i)-t(i)}{2^{i+1}}= 
  \frac{1}{2^{m+1}}-
  \sum\limits_{i = m+1}^{n-1} \frac{1}{2^{i+1}}= 
  \frac{1}{2^n}
  $$
  which is non-zero, contradicting $cs_n(s) = cs_n(t)$. 
\end{proof}

\begin{lemma}\label{alternativeSimByCauchyDistance}
  Let $n:\N$ and let $s,t:2^n$. Then 
  \begin{equation}
    s\sim_n t \leftrightarrow |cs_n(s) - cs_n(t)| \leq \frac{1}{2^{n}}.
  \end{equation} 
\end{lemma}

\begin{proof}
  \item  
    Assume $ s \sim_n t$. If $s=t$, we have $cs_n(s) - cs_n(t) = 0$, 
    otherwise, we may without loss of generality assume there is some $m<n$ and some $u:2^m$ such that 
  \begin{equation}
    (s = u \cdot 0 \cdot \overline 1|_n) \wedge ( t = u \cdot 1 \cdot \overline 0 |_n) . 
  \end{equation}
  Then 
  \begin{align}
    cs_n(s) &= 
    cs_m(u) + 0 + \sum\limits_{i = m+1}^{n-1} \frac{1}{2^{i+1}}\\
    cs_n(t) &= 
    cs_m(u) + \frac{1}{2^{m+1}} + 0  
  \end{align}
  And hence 
  \begin{equation}
    cs_n(t) - cs_n(s) = \frac{1}{2^{m+1}} - \sum\limits_{i = m+1}^{n-1} \frac{1}{2^{i+1}} = \frac{1}{2^n}
  \end{equation}
  Thus in all cases, from $s\sim_n t$, we can conclude that 
  \begin{equation}
    |cs_n(s) -cs_n(t) |\leq \frac{1}{2^n}
  \end{equation}
  \item 
  Conversely, assume that $|cs_n(s) - cs_n(t)| \leq \frac{1}{2^n}$. 
  If $s = t$, it is clear that $s \sim_n t$.
  If $s\neq t$, there must be some smallest number $m<n$ such that 
  $s(m) \neq t(m)$. As $m$ is minimal, we have $s|_m = t|_m = : u$. 
  WLOG, we assume that $s(m) = 1, t(m) = 0$. 
  We thus have that 
  \begin{align}
    cs_n(s) &= 
    cs_m(u) + \frac1{2^{m+1}} + \sum\limits_{i = m+1}^{n-1} \frac{s(i)}{2^{i+1}}\\
    cs_n(t) &= 
    cs_m(u) + 0  + \sum\limits_{i = m+1}^{n-1} \frac{t(i)}{2^{i+1}}
  \end{align}
  And thus 
  \begin{align}
    cs_n(s)-cs_n(t) = \frac{1}{2^{m+1}} + \sum\limits_{i = m+1}^{n-1} \frac{s(i)-t(i)}{2^{i+1}}
  \end{align}
  Note that as $s(i),t(i) \in \{0,1\}$, we must have that $s(i) -t(i) \in \{-1,0,1\}$. 
  Therefore 
  $$\sum\limits_{i = m+1}^{n-1} \frac{s(i)-t(i)}{2^{i+1}}$$
  is minimal iff $s(i) -t(i) = -1$ for all $m<i<n$. 
  In that case, we have that 
  $$
  \frac{1}{2^{m+1}}-
  \sum\limits_{i = m+1}^{n-1} \frac{s(i)-t(i)}{2^{i+1}}= 
  \frac{1}{2^{m+1}}-
  \sum\limits_{i = m+1}^{n-1} \frac{1}{2^{i+1}}= 
  \frac{1}{2^n}
  $$
  Now if $s(i) -t(i) > -1$ for any $m<i<n$, we have that
    $$
    \frac{1}{2^{m+1}} + \sum\limits_{i = m+1}^{n-1} \frac{s(i)-t(i)}{2^{i+1}}> \frac{1}{2^n},$$
  contradicting our assumption that 
  $|cs_n(s) - cs_n(t)| \leq \frac{1}{2^n}$. 
  We conclude that $s(i) -t(i) = -1$ for all $m<i<n$, hence for those $i$, we have that 
  $s(i) = 0, t(i) = 1$. Hence 
  \begin{equation}
    s = (u \cdot 1\cdot \overline 0) |_n \wedge 
    t = (u \cdot 0\cdot \overline 1) |_n.
  \end{equation}
  and thus we can conclude $s\sim_n t$ as required. 
\end{proof}


Inspired by Definition 2.7 and 2.10 \Cite{Bishop}, 
we define inequality on $I$ as follows:
\begin{definition}
  Let $\alpha,\beta:2^\N$. 
  We define $\alpha\leq_I \beta$ and $\alpha<_I\beta$ as follows:
  \begin{align}
  \alpha\leq_I\beta : = \forall_{n:\N} ( cs(\alpha)_n \leq cs(\beta)_n + \frac {1} {2^n})\\ 
    \alpha   <_I \beta : = \exists{n:\N} ( cs(\alpha)_n < cs(\beta)_n - \frac {1} {2^n})
%    \\\rednote{Can become n\pm1, \leq ,<, +\frac1{2^n+2} }
\end{align}
\end{definition}

\begin{lemma}\label{SqueezeLemma}
  For any $n:\N$ and $a,b,c:2^n$, we have that whenever 
  $cs_n(a) \leq cs_n(b)\pm\frac{1}{2^n} \leq cs_n(c)$ and $a\sim_n c$, we have 
  $a = b \vee b = c$. 
\end{lemma}
\begin{proof}
  Assume 
\end{proof}
\newpage

\begin{lemma}
  Suppose $n:\N$ and $a,b:2^n$. 
  Then if $cs_n(a) > cs_n(b)$, 
  we have that $cs_n(a) \geq  cs_n(b) +\frac{1}{2^n}.$
\end{lemma}



\begin{lemma}
  $\leq_I$ respects $\sim_I$. 
\end{lemma}
\begin{proof}

  CLAIM 1: 
  Whenever $a\sim_n b$, we have that 
  \begin{equation} 
    cs_n(a) = cs_n(b) - \frac{1}{2^n}
    \vee
    cs_n(a) = cs_n(b) 
    \vee
    cs_n(a) = cs_n(b) + \frac{1}{2^n}
  \end{equation}
  
  CLAIM 2: 
  whenever $a,b:2^n$ and $cs_n(a) > cs_n(b)$, the minimum distance is $\frac{1}{2^n}$. 
  So $cs_n(a) \geq cs_n(b) + \frac{1}{2^n} $. 

  CLAIM 3: 
  $cs_n$ is injective for any $n:\N$. 

  Assume $\alpha\leq_I \gamma$ and $\alpha\sim_I\beta$. 
  So 
  $\forall_{n:\N} (cs(\alpha)_n \leq_\mathbb Q cs(\gamma)_n)+\frac{1}{2^n}$. 
  and $\forall_{n:\N}(\alpha|_n\sim_n\beta|_n)$. 

  We need to show that 
  $\forall_{n:\N} (cs(\beta)_n \leq_\mathbb Q cs(\gamma)_n+ \frac{1}{2^n} ) $. 

  This is closed, so we can show the double negation instead. 
  By Markov, the negation is that there is some 
  $N$ with 
  $$cs(\beta)_N >_\mathbb Q cs(\gamma)_N + \frac{1}{2^N}.$$
  
  We are given that 
  $cs(\gamma)_N + \frac{1}{2^n}\geq cs(\alpha)|_N $. 
  Thus $$cs(\beta)_N > cs(\gamma)_N + \frac{1}{2^N} \geq cs(\alpha)_N$$
  By Claim 1 and as $\alpha|_N\sim_N\beta|_N$, we must have that  
  $cs(\beta)_N = cs(\alpha)_N +\frac{1}{2^N} $. 
%  \begin{itemize}
%    \item 
%      If $cs(\beta)|_N = cs(\alpha)_N$, we have $\alpha|_N = \beta|_N$ by claim 3. 
%      Hence as $cs(\alpha)_N \leq_Q cs(\gamma)_N$, we also have 
%      $cs(\beta)_N \leq_Q cs(\gamma)_N$, contradicting our assumption and we're done. 
%    \item 
%      If $cs(\beta)|_N = cs(\alpha)_N + \frac{1}{2^N}$, we have that 
  Therefore 
      $$
      cs(\alpha)_N+\frac{1}{2^N} \geq cs(\gamma)_N + \frac{1}{2^N} \geq cs(\alpha)_N
      $$
      So $$cs(\alpha)_N \geq cs(\gamma)_N\geq cs(\alpha)_N -\frac{1}{2^N}.$$
      As a consequence of claim 2, we have that 
      $cs(\alpha)_N = cs(\gamma)_N \vee cs(\alpha)_N -\frac{1}{2^N} = \gamma_N$. 
      \begin{itemize}
        \item If $cs(\alpha)_N = cs(\gamma)_N$, we have that 
          $\alpha|_N = \gamma|_N$, and hence $cs(\beta)_N = cs(\gamma)_N + \frac{1}{2^N}$, 
          contradicting our assumption. 
        \item 
          If $cs(\alpha)_N -\frac{1}{2^N} = cs(\gamma)_N$, we still need that 
          $cs(\alpha)_n \leq cs(\gamma)_n + \frac{1}{2^n}$ for all $n\geq N$. 
          But I now claim this can only happen if
          \rednote{TODO, combine this with minimality thing in the other lemma}
          $cs(\gamma)_n = cs(\alpha)_n + \frac{1}{2^n}$ for all $n\geq N$. 
          But in this case, we can still deduce that 
          $\alpha\sim_I \gamma$, but then $\alpha = \gamma \vee \beta = \gamma \vee \alpha = \beta$, 
          contradicting our assumption. 
      \end{itemize}
%\end{itemize} 



  %  In this case, we have 
  %  $cs(\gamma)_N +\frac{1}{2^N} \geq cs(\alpha)_N > cs(\gamma)_N$. 
  %  In particular $\gamma|_N \sim_N \alpha|_N$. 
  %  $cs(\alpha)_N > cs(\gamma)_N$. 


    
  \newpage



















































%  Assume $\alpha\leq_I \beta$ and $\alpha\sim_I\gamma$. 
%  We need to show that $\gamma\leq_I\beta$ as well, so for each $n:\N$, we need to show that 
%  $cs(\gamma)_n \leq cs(\beta)_n + \frac{1}{2^n}$. 
%  As inequality in $\mathbb Q$ is decidable, we can proceed with a proof by contradiction. 
%  Suppose $cs(\gamma)_n > cs(\beta)_n + \frac{1}{2^n} \geq cs(\alpha)_n$. 
%  Then by the above Lemma, we get 
%  $\beta|_n =\gamma|_n \vee \beta|_n = \alpha|_n$. 
%  In both cases, we get $|\beta|_n - \gamma|_n|\leq \frac{1}{2^n}$, contradicting our assumption. 
%  Thus $cs(\gamma)_n \leq cs(\beta)_n + \frac1{2^n}$. 
%
%  Assume now $\alpha\leq_I\beta$ and $\beta\sim_I\gamma$. 
%  We need to show that $\alpha \leq_I\gamma$ as well. 
\end{proof}
\begin{lemma}
  $<_I$ respects $\sim_I$. 
\end{lemma} 
\begin{proof}
  TODO
\end{proof}
\begin{remark}
  By the above, $\leq_I, <_I$ induce relations $\leq,<$ on $I$.
  As inequality in $\mathbb Q$ is decidable, $\leq, <$ are closed and open respectively. 
%
  We can use these inequalities to define the standard open and closed intervals. 
  Let $a,b:I$. 
  Following standard notation, we denote
  \begin{equation}
    [a,b]:= \Sigma_{x:I} (a\leq x \wedge x \leq b),
  \end{equation}
  which is closed by \Cref{ClosedCountableConjunction}, 
  we call subsets of $I$ of this form closed intervals. 
%
  We also denote 
  \begin{equation}
    (a,b) := \Sigma_{x:I} (a < x \wedge x < b),
  \end{equation}
  which is open by \Cref{OpenFiniteConjunction}.
  we call subsets of $I$ of this form open intervals. 
\end{remark}

\begin{lemma}
  Let $D_n:2^\N \to 2$ be a sequence of decidable subsets with $D_{n+1}\subseteq D_n$.
  For $p$ the quotient map $2^\N \to I$, we have that 
  $p(\bigcap_{n:\N} D_n) = p(\bigcap_{n:\N} D_n)$
\end{lemma}
\begin{proof}
  It is always the case that $$p(\bigcap_{n:\N} D_n) \subseteq \bigcap_{n:\N} p(D_n).$$
  For the converse direction, let $(\bigcap_{n:\N} p(D_n))(x)$. 
  We will show that $ \neg \neg (p(\bigcap D_n)) (x)$, which is sufficient by \Cref{rmkOpenClosedNegation}. 
%
  As $(\bigcap_{n:\N} p(D_n))(x)$, there exists some $y\in D_0$ with $p(y) = x$. 
%
  If $x\notin p(\bigcap_{n:\N} D_n)$, we cannot have for all $n:\N$ that $y_0 \in  D_n$. 
  By Markov, there must exist some $k:\N$ with $\neg D_k(y_0)$. 
  As $D_{n+1}\subseteq D_n$ for all $n:\N$, it follows that $y_0\notin D_n$ for all $n\geq k$. 
%
  As $x\in \bigcap_{n:\N}p(D_n)$, there is however some $y_k\in D_k$ with $p(y_k) = x$. 
  By a similar argument, we have some $l>k$ with $y_k\notin D_l$, and some $y_l$ with $p(y_l) = x, y_l \in D_l$. 
  But now we have that $y_0, y_k, y_l:2^\N$ are all distinct, but $p(y_0) = p(y_k) = p(y_l) = x$. 
  This contradicts \Cref{IntervalFiberSizeAtMost2}, and we're done. 
\end{proof}


\begin{lemma}
  For $p:2^\N \to I$ the quotient map and $D\subseteq 2^\N$ decidable, we have $p(D)$ a finite union of closed intervals. 
\end{lemma}
\begin{proof}
  We will show the above if there exists some $n:\N, u:2^n$ such that $D(x) \leftrightarrow x|_n = a$.
  This is sufficient, as any decidable subset of $2^\N$ can be written as finite union of such decidable subsets. 
  We claim that $p(D) = [p(a\cdot \overline 0) , p(a \cdot \overline 1)$. 
\item 
  We will first show that $p(D) \subseteq [p(a\cdot \overline 0) , p(a \cdot \overline 1)$. 
  Let $x\in D$. 
  Then $x|_n = a$ and hence for $m\leq n$ we have 
  \begin{equation}
    cs(x)_m = cs_m(a|_m) = cs(a\cdot \overline 0)_m= cs(a\cdot \overline 1)_m
  \end{equation}
  For $m>n$, we have that 
  \begin{align}
    cs(a\cdot \overline 1)_m =
    cs_n(a) +\sum_{i = n} ^{m-1} \frac{1}{2^{i+1}}
    \\
    cs(x)_m =
    cs_n(a) +\sum_{i = n} ^{m-1} \frac{x(i)}{2^{i+1}}
    \\
    cs(a\cdot \overline 0)_m = 
    cs_n(a) +\sum_{i = n} ^{m-1} \frac{0}{2^{i+1}}
  \end{align} 
  Hence for all $m:\N$, we have 
  \begin{equation}
    cs(a\cdot \overline 1)_m \geq 
    cs(x)_m \geq 
    cs(a\cdot\overline 0)_m
  \end{equation}
 which implies that $p(a\cdot \overline 1) \geq_I p(x) \geq_I p(a\cdot\overline 0)$, as required. 
\item 
  To show that $[p(a\cdot \overline 0) , p(a \cdot \overline 1)\subseteq p(D)$, 
  Suppose
  $p(a\cdot \overline 0) \leq p(x) \leq p(a \cdot \overline 1)$. 
  We will show that 
  $$x|_n = a \vee x \sim_I a \cdot \overline 0 \vee x \sim_I a \cdot \overline 1.$$
  As this is a disjunction of closed propositions, by \Cref{ClosedFiniteDisjunction} it's closed, and by 
  \Cref{rmkOpenClosedNegation}, we can instead show the double negation. 
  So suppose that none of the disjoints hold. 
  As $x|_n \neq a$, there is some minimal $m$ with $x(m) \neq a(m)$. 
  We assume that $x(m) = 1, a(m) = 0$, the other case goes similarly. 
  Then for all $k:\N$, we have 
  $cs(x)_k \geq cs(a \cdot \overline 1)|_k$. 
  As also 
  $p(a\cdot \overline 1)\geq p(x)$, we have 
  $$cs(a \cdot \overline 1)|_k + \frac{1}{2^k} \geq cs(x)_k \geq cs(a\cdot \overline 1)_k,$$
  From which it follows that $|cs(a\cdot\overline 1)_k - cs(x)_k|\leq \frac{1}{2^k}$. 
  Hence $(a\cdot \overline 1)|_k \sim_k x|_k$ by \Cref{alternativeSimByCauchyDistance}. 
  Hence $x\sim_I (a\cdot\overline 1)$, contradicting our assumption as required. 
\end{proof}

\begin{lemma}
  Every open $U\subseteq I$ can be written as countable union of open intervals.
\end{lemma} 
\begin{proof}
  TODO
%  Let $U\subseteq I$ open, then $U^C\subseteq I$ is closed. 
%  By \Cref{StoneClosedSubsets} and \Cref{CompactHausdorffClosed}, we have that 
%  there is some sequence $D_n\subseteq 2^\N$ with $p^{-1}(U^C) = \bigcap_{n:\N} D_n$. 
%  As the quotient map $p$ is surjective, we have that $U^C = p(p^{-1}(U^C))$. 
%  By the above, it follows that $\neg U = \bigcap_{n:\N} p(D_n)$. 
%  By \Cref{TODO}, it follows that 
%  $\neg U$ is a countable intersection of finite unions of closed intervals. 
%  Thus $\neg\neg U$ is a countable union of finite intersections of complements of closed intervals. 
%  As complements of closed intervals are finite unions of open intervals (TODO), 
%  and finite intersections of such things are still finite unions of open intervals, 
%  it follows that $\neg\neg U$ is a countable union of open intervals. 
%  By \Cref{rmkOpenClosedNegation}, $\neg \neg U = U$ and we're done. 
%  \rednote{Lotta handwaving here, definitely not finished} 
\end{proof}


\begin{remark}
  It follows that the topology of $I$ is generated by open intervals, 
  which corresponds to the standard topology on $I$. 
  Hence our notion of continuity corresponds with the $\epsilon,\delta$-definition of continuity one would expect. 
  Thus every function $f:I\to I$ in the system we presented is continuous in the $\epsilon,\delta$-sense. 
\end{remark}
