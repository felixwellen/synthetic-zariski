In this section, we will show that the topology on $I$ as one would expect. 

\begin{definition}
  For $n:\N$ we define 
  $cs_n:2^n \to \mathbb Q$ by 
  \begin{equation}
    cs_n(a) = \sum\limits_{i=0}^{n-1} \frac{a(i)} {2^{i+1}}
  \end{equation}
  And for $\alpha:2^\N$, we define the sequence $cs(\alpha) : \N \to \mathbb Q$ by 
  \begin{equation}
    cs(\alpha)_n = cs_n(\alpha|_n)
  \end{equation}
\end{definition}
\begin{remark}
  One can check that for each $\alpha:2^\N$, 
  $cs(\alpha)$ is a Cauchy sequence. 
  Thus $cs$ gives a map from Cantor space to the Cauchy reals. 
  In \Cref{AppendixCauchyEquivalenceProof}, 
  we show that this map induces an equivalence from $I$ to the interval of Cauchy reals. 
\end{remark}

\begin{lemma}
  Let $n:\N$ and let $s,t:2^n$. Then 
  \begin{equation}
    s\sim_n t \leftrightarrow |cs_n(s) - cs_n(t)| \leq \frac{1}{2^{n}}.
  \end{equation} 
\end{lemma}

\begin{proof}
  \item  
    Assume $ s \sim_n t$. If $s=t$, we have $cs_n(s) - cs_n(t) = 0$, 
    otherwise, we may without loss of generality assume there is some $m<n$ and some $u:2^m$ such that 
  \begin{equation}
    (s = u \cdot 0 \cdot \overline 1|_n) \wedge ( t = u \cdot 1 \cdot \overline 0 |_n) . 
  \end{equation}
  Then 
  \begin{align}
    cs_n(s) &= 
    cs_m(u) + 0 + \sum\limits_{i = m+1}^{n-1} \frac{1}{2^{i+1}}\\
    cs_n(t) &= 
    cs_m(u) + \frac{1}{2^{m+1}} + 0  
  \end{align}
  And hence 
  \begin{equation}
    cs_n(t) - cs_n(s) = \frac{1}{2^{m+1}} - \sum\limits_{i = m+1}^{n-1} \frac{1}{2^{i+1}} = \frac{1}{2^n}
  \end{equation}
  
  \item 
  Conversely, assume that $|cs_n(s) - cs_n(t)| \leq \frac{1}{2^n}$. 
  If $s = t$, it is clear that $s \sim_n t$.
  If $s\neq t$, there must be some smallest number $m<n$ such that 
  $s(m) \neq t(m)$. As $m$ is minimal, we have $s|_m = t|_m = : u$. 
  WLOG, we assume that $s(m) = 1, t(m) = 0$. 
  We thus have that 
  \begin{align}
    cs_n(s) &= 
    cs_m(u) + \frac1{2^{m+1}} + \sum\limits_{i = m+1}^{n-1} \frac{s(i)}{2^{i+1}}\\
    cs_n(t) &= 
    cs_m(u) + 0  + \sum\limits_{i = m+1}^{n-1} \frac{t(i)}{2^{i+1}}
  \end{align}
  And thus 
  \begin{align}
    cs_n(s)-cs_n(t) = \frac{1}{2^{m+1}} + \sum\limits_{i = m+1}^{n-1} \frac{s(i)-t(i)}{2^{i+1}}
  \end{align}
  Note that as $s(i),t(i) \in \{0,1\}$, we must have that $s(i) -t(i) \in \{-1,0,1\}$. 
  Therefore 
  $\sum\limits_{i = m+1}^{n-1} \frac{s(i)-t(i)}{2^{i+1}}$
  is minimal iff $s(i) -t(i) = -1$ for all $m<i<n$. 
  In that case, we have that 
  $$
  \frac{1}{2^{m+1}}-
  \sum\limits_{i = m+1}^{n-1} \frac{s(i)-t(i)}{2^{i+1}}= 
  \frac{1}{2^{m+1}}-
  \sum\limits_{i = m+1}^{n-1} \frac{1}{2^{i+1}}= 
  \frac{1}{2^n}
  $$
  and if $s(i) -t(i) > -1$ for some $m<i<n$, we have that 
    $$
    \frac{1}{2^{m+1}} + \sum\limits_{i = m+1}^{n-1} \frac{s(i)-t(i)}{2^{i+1}}> \frac{1}{2^n},$$
  contradicting our assumption that 
  $|cs_n(s) - cs_n(t)| \leq \frac{1}{2^n}$. 
  We conclude that $s(i) -t(i) = -1$ for all $m<i<n$, hence for those $i$, we have that 
  $s(i) = 0, t(i) = 1$. Hence 
  \begin{equation}s = (s|_m \cdot 1\overline 0) |_n \wedge 
  t = (s|_m \cdot 0 \overline 1) |_n.
  \end{equation}
  and $s\sim_n t$ as required. 
\end{proof}


Inspired by \Cite{Bishop}, we define inequality on $I$ as follows:
\begin{definition}
  Let $\alpha,\beta:2^\N$. 
  We define $\alpha\leq_I \beta$ and $\alpha<_I\beta$ as follows:
  \begin{align}
  \alpha\leq_I\beta : = \forall_{n:\N} ( cs(\alpha)_n \leq cs(\beta)_n + \frac {1} {2^n})\\ 
    \alpha   <_I \beta : = \exists{n:\N} ( cs(\alpha)_n < cs(\beta)_n - \frac {1} {2^n})
    \\\rednote{Can become n\pm1, \leq ,< }
\end{align}
\end{definition}


\begin{lemma}
  $\leq_I$ respects $\sim_I$. 
\end{lemma}
\begin{proof}
  Assume $\alpha\leq_I \beta$ and $\alpha\sim_I\gamma$. 
  By the above, we have that for all $n:\N$, 
  $|cs(\alpha)_n - cs(\gamma)_n |\leq \frac{1}{2^n}$. 


%  Let $\alpha,\beta,\gamma:2^\N$ and suppose $\alpha\sim_I\beta$ and $\alpha \leq_I\gamma$.
%  We will show that $\beta\leq_I\gamma$. 
%  
%  Let $n:\N$, we will show that $cs(\beta)_n \leq cs(\gamma)_n + \frac{1}{2^n}$. 
%  If $\alpha|_n = \beta|_n$, this is trivial.
%  Note that $cs (u 0 \overline 1)_n \leq  cs( u 1 \overline 0)_n$ for all $n:\N$, and all finite sequences $u$. 
%  Thus we need only consider the case that there is some $m<n$ and finite sequence $u:2^m$ with 
%  $\alpha|_n = (u 0 \overline 1)|_n, \beta|_n = (u 1 \overline 0)|_n$.
%  In this case, 
%  $cs(\beta)_n = cs(u)_m + \frac 1 {2^m}$, 
%  and 
%  $$cs(\alpha)_n = 
%                   cs (u)_m + \sum_{i = m}^{n-1} \frac 1 {2^{i+1}} 
%                   = cs(u)_m + \frac 1 {2^m} - \frac{1}{2^n}
%                 = cs(\beta) - \frac{1} {2^n}
%  $$.
\end{proof}
\begin{lemma}
  $<_I$ respects $\sim_I$. 
\end{lemma} 
\begin{proof}
  TODO
\end{proof}
\begin{remark}
  By the above, $\leq_I, <_I$ induce relations $\leq,<$ on $I$.
  As inequality in $\mathbb Q$ is decidable, $\leq, <$ are closed and open respectively. 

  Let $a,b:I$. Following standard notation, we denote
  $[a,b]$ for $\Sigma_{x:I} (a\leq x \wedge x \leq b)$, which is closed by \Cref{ClosedCountableConjunction}, 
  we call subsets of $I$ of this form closed intervals. 


  We also denote $(a,b)$ for $\Sigma_{x:I} (a < x \wedge x < b)$, which is open by \Cref{OpenFiniteConjunction}.
  we call subsets of $I$ of this form open intervals. 
\end{remark}


%\begin{lemma}
%  Every closed $U\subseteq I$ can be written as finite union of closed intervals.
%\end{lemma}
%\begin{proof}
%  From \Cref{CompactHausdorffClosed} and \Cref{StoneClosedSubsets}, it follows that for every closed $U\subseteq I$, 
%  there are some decidable subsets $(D_n)_{n:\N}$ such that $U = p(\bigcap_{n:\N} D_n)$ for $p:2^\N\to I$ the quotient map. 
%  We may assume that $D_{n+1} \subseteq D_n$ for $n:\N$. 
%
%
%
%
%
%
%%
%%  Let $n:\N$ and let $a:2^n$. Suppose that $a(n-1) \neq  a(n-2) $.  
%%  \rednote{CHANGE END CONDITIONS of $a$ LATER}
%%
%%  We claim that 
%%  $$p(\{\alpha:2^\N | \alpha|_n = a\}) = \big[ [a|_{n-1}\overline 0] , [a|_{n-1} \overline 1] \big].$$
%%  Suppose that $\beta:2^\N$ is such that $p(\beta) \in p(\{\alpha:2^\N | \alpha|_n = a\})$.
%%  We will show that $p(a|_{n-1}\overline 0) \leq_I p(\beta) \leq_I p(a|_{n-1}1\overline 0$. 
%%
%%
%%  Then there exists some  $\alpha:2^N$ with $\alpha|_n = a$ and $\alpha \sim_m \beta$ for all $m:\N$. 
%%  If 
%%
%%
%%
%%  As $a(n-1) \neq a(n-2)
%
%\end{proof}
%


\begin{lemma}
  Let $D_n:2^\N \to 2$ be a sequence of decidable subsets with $D_{n+1}\subseteq D_n$.
  For $p$ the quotient map $2^\N \to I$, we have that 
  $p(\bigcap_{n:\N} D_n) = p(\bigcap_{n:\N} D_n)$
\end{lemma}
\begin{proof}
  It is always the case that $$p(\bigcap_{n:\N} D_n) \subseteq \bigcap_{n:\N} p(D_n).$$
  For the converse direction, let $(\bigcap_{n:\N} p(D_n))(x)$. 
  We will show that $ \neg \neg (p(\bigcap D_n)) (x)$, which is sufficient by \Cref{rmkOpenClosedNegation}. 
%
  As $(\bigcap_{n:\N} p(D_n))(x)$, there exists some $y\in D_0$ with $p(y) = x$. 
%
  If $x\notin p(\bigcap_{n:\N} D_n)$, we cannot have for all $n:\N$ that $y_0 \in  D_n$. 
  By Markov, there must exist some $k:\N$ with $\neg D_k(y_0)$. 
  As $D_{n+1}\subseteq D_n$ for all $n:\N$, it follows that $y_0\notin D_n$ for all $n\geq k$. 
%
  As $x\in \bigcap_{n:\N}p(D_n)$, there is however some $y_k\in D_k$ with $p(y_k) = x$. 
  By a similar argument, we have some $l>k$ with $y_k\notin D_l$, and some $y_l$ with $p(y_l) = x, y_l \in D_l$. 
  But now we have that $y_0, y_k, y_l:2^\N$ are all distinct, but $p(y_0) = p(y_k) = p(y_l) = x$. 
  This contradicts \Cref{IntervalFiberSizeAtMost2}, and we're done. 
\end{proof}


\begin{corollary}
  For $p:2^\N \to I$ the quotient map and $D\subseteq 2^\N$ decidable, we have $p(D)$ a finite union of closed intervals. 
\end{corollary}
\begin{proof}
  TODO
\end{proof}

\begin{lemma}
  Every open $U\subseteq I$ can be written as countable union of open intervals.
\end{lemma} 
\begin{proof}
%  We denote $p:2^\N \to I$ for the quotient map. 
%  Let $U\subseteq I$ open, then $p^{-1}(U)\subseteq 2^N$ is open in $2^\N$. 
%  As a corollary of \Cref{StoneClosedSubsets}, any open subset of a Stone space is a countable union of decidable subsets. 
%  Thus we may assume $p^{-1}(U) = \bigcup_{n:\N} D_n$. 
%  As the quotient map $p$ is surjective, we have that $U = p(p^{-1}(U)) = p(\bigcup_{n:\N} D_n)$. 
%  Note that $p(\bigcup_{n:\N} D_n) = \bigcup_{n:\N} p(D_n)$. 
%
%
%
%  It is thus sufficient to show that the image of a decidable subset is a countable union of open intervals. 
%
%  However the image of a decidable subset is a closed interval, so we cannot do this. 
%
  Let $U\subseteq I$ open, then $U^C\subseteq I$ is closed. 
  By \Cref{StoneClosedSubsets} and \Cref{CompactHausdorffClosed}, we have that 
  there is some sequence $D_n\subseteq 2^\N$ with $p^{-1}(U^C) = \bigcap_{n:\N} D_n$. 
  As the quotient map $p$ is surjective, we have that $U^C = p(p^{-1}(U^C))$. 
  By the above, it follows that $\neg U = \bigcap_{n:\N} p(D_n)$. 
  By \Cref{TODO}, it follows that 
  $\neg U$ is a countable intersection of finite unions of closed intervals. 
  Thus $\neg\neg U$ is a countable union of finite intersections of complements of closed intervals. 
  As complements of closed intervals are finite unions of open intervals (TODO), 
  and finite intersections of such things are still finite unions of open intervals, 
  it follows that $\neg\neg U$ is a countable union of open intervals. 
  By \Cref{rmkOpenClosedNegation}, $\neg \neg U = U$ and we're done. 
  \rednote{Lotta handwaving here, definitely not finished} 

\end{proof}




\begin{remark}
  It follows that the topology of $I$ is generated by open intervals, 
  which corresponds to the standard topology on $I$. 
  Hence our notion of continuity corresponds with the $\epsilon,\delta$-definition of continuity one would expect. 
  Thus every function $f:I\to I$ in the system we presented is continuous in the $\epsilon,\delta$-sense. 
\end{remark}
