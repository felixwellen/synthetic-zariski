In this section we will introduce the unit interval $I$ as compact Hausdorff space. 
The definition is based on \cite{Bishop}. 
We will then calculate the cohomology of $I$. 
For a proof that the unit interval corresponds to the definition using Cauchy sequences, 
we refer to the appendix. 


%\subsection{The Cauchy reals}
The goal of this section is to introduce the real numbers in a constructive setting, 
following the definition given in \cite{Bishop} with some small adaptations. 
We will later use this definition to show that the interval $[0,1]$ is compact Hausdorff in the sense 
of \Cref{dfnCompactHausdorff}. 

We will assume we are given natural and rational numbers, with decidable (in)equalities
working as expected. 

\begin{definition}
  A \textbf{Cauchy sequence} is a sequence $x : \mathbb N \to \mathbb Q$ such that
  for any $n,m:\mathbb N$, we have %$0\leq x_n \leq 1$ and 
$|x_n-x_m| \leq (\frac12)^n + (\frac12)^m$. 
\end{definition}
\begin{remark}
  If $x$ is a cauchy sequence and $q$ a rational number, the 
  sequence $(x-q)_n = (x_n-q)$ is also Cauchy.
\end{remark}

Following \cite{Bishop}, we define inequality relations between Cauchy sequences and
rational numbers. 
\begin{definition}
  For $x$ a Cauchy sequence and $q$ a rational number, we define 
  \begin{itemize}
%    \item $x> q = \Sigma(n:\mathbb N) x_n > q + {\frac12}^n$. %for some $n:\mathbb N$. 
%    \item $x< q = \Sigma(n:\mathbb N) x_n < q - {\frac12}^n$. %for some $n:\mathbb N$. 
    \item $x\leq  q = \Pi_{n:\mathbb N} x_n \leq q+(\frac12)^n$. 
    \item $x\geq  q = \Pi_{n:\mathbb N} x_n \geq q-(\frac12)^n$. 
  \end{itemize}
\end{definition}
%\begin{lemma}
%  For $x$ Cauchy and $q$ rational, we have that 
%  $x\leq q$ iff for each $n:\mathbb N$, we have a $N_n:\mathbb N$ with 
%  $x_m> q-(\frac12)^n$ for all $m \geq N_n$. 
%\end{lemma}
\begin{lemma}\label{ComparisonLemma}
  For $x$ a Cauchy sequence and $q$ a rational number, we have
  $x \leq q \vee x \geq q$. 
\end{lemma}
\begin{proof}
  For rational numbers, we have decidable inequalities, 
  therefore $\geq 0 \vee q \leq 0$. 
  It follows that 
  $ \forall (n:\mathbb N) \forall (m:\mathbb N) q \geq -(\frac12)^n \vee q \leq (\frac12)^m$. 
  Now by \Cref{TODO}, we may conclude 
  $ (\forall (n:\mathbb N) q \geq -(\frac12)^n ) \vee (\forall (m:\mathbb N) q \leq (\frac12)^m)$
  as required.
\end{proof}


%%%\begin{definition}
%%%  A Cauchy sequence $x$ is \textbf{nonnegative} if $x_n \geq -(\frac12)^n$
%%%  for every $n:\mathbb N$. 
%%%  $x$ is \textbf{nonpositive} if $x_n \leq (\frac12)^n$
%%%  for every $n:\mathbb N$. 
%%%\end{definition} 
%%%%\begin{lemma}
%%%%  A Cauchy sequence is nonnegative iff there exists an $N$ such that $x_n \geq -(\frac12)^N$
%%%%  for all $n\geq N$. 
%%%%  A Cauchy sequence is nonpositive iff there exists an $N$ such that $x_n \leq (\frac12)^N$
%%%%  for all $n\geq N$. 
%%%%\end{lemma}
%%%%\begin{proof}
%%%%  Assume $x$ is nonnegative. Thus for every $n:\mathbb N$, we have $x_n\geq -(\frac12)^n \geq -(\frac12)^0$. 
%%%%  Thus $N$ can taken to be $0$. 
%%%%%
%%%%  Conversely, as $x$ is Cauchy, we have
%%%%  for all $m :\mathbb N$ that  
%%%%%  \begin{equation}- (\frac12)^m -(\frac12)^n \leq    x_m-x_n \leq (\frac12)^m + (\frac12)^n \end{equation}
%%%%  \begin{equation}- (\frac12)^m -(\frac12)^n \leq    x_n-x_m \leq (\frac12)^m + (\frac12)^n \end{equation}
%%%%  If in addition there is an $N$ such that whenever $m\geq N$, we have 
%%%%  $x_m \geq -(\frac12)^N$, so $-x_m \leq (\frac12)^N$, 
%%%%  so $x_n -x_m \leq x_n - (\frac12)^N$. 
%%%%  Therefore, 
%%%%  \begin{equation}- (\frac12)^m -(\frac12)^n \leq    x_n-x_m \leq x_n-(\frac12)^N \end{equation}
%%%%  Thus 
%%%%  \begin{equation}- (\frac12)^m -(\frac12)^n  + (\frac12)^N \leq x_n \end{equation}
%%%%  As $N \geq N$, we have in particular 
%%%%  \begin{equation}- (\frac12)^N -(\frac12)^n  + (\frac12)^N \leq x_n \end{equation}
%%%%  \begin{equation} - (\frac12)^n  \leq x_n \end{equation}
%%%%  thus $x$ is nonnegative. 
%%%%
%%%%  The nonpositive case goes similar. 
%%%%\end{proof}   
%%% 
%%%
%%%\begin{lemma}
%%%  A Caucy sequence is nonnegative or nonpositive. 
%%%\end{lemma}

%\begin{lemma}
%  For any Cauchy sequence $p$, we have 
%  $(\forall (n:\mathbb N) p_n \leq (\frac12)^n) \vee (\forall (n:\mathbb N) p_n \geq -(\frac12)^n)$. 
%\end{lemma}
%\begin{proof}
%We 
%\end{proof}

\begin{definition}
Given two Cauchy sequences $p = (p_n)_{n\in\mathbb N}, q=(q_n)_{n\in\mathbb N}$, 
we define the proposition $p \sim_C  q$ as 
\begin{equation}
  p \sim_C q : = \forall (n,m : \mathbb N) ((| p_n - q_m| \leq  (\frac12)^n + (\frac12)^m))
\end{equation}
\end{definition}

%\begin{remark}
%  Note that $p\sim_C q$ is equivalent to 
%\begin{equation}
%  \forall (n : \mathbb N) | p_n - q_n| \leq  (\frac12)^{n-1}
%\end{equation}
%The equivalence doesn't hold, unless you cut off initial segments (which shouldn't matter). 
%\end{remark} 

\begin{definition}
  The type of \textbf{Cauchy reals} is given by 
  the type of Cauchy sequences modulo $\sim_C$.
\end{definition}

We claim that the inequality in \Cref{TODO} extends to a well-defined 
inequality between Cauchy reals and rational numbers. 

Furthermore, we claim that 
$\Pi_{x:\mathbb R} \Pi_{q:\mathbb Q} x \leq q \vee x \geq q$. 

%\begin{lemma}
%  For any Cauchy real $r$ any Cauchy sequence $p$ representing $r$, 
%  we have 
%  \begin{equation}
%    (\forall (n:\mathbb N) p_n \leq (\frac12)^n) \vee (\forall (n:\mathbb N) p_n \geq (\frac12)^n)
%  \end{equation}
%
%\end{lemma}

\begin{definition}
  A Cauchy sequence in the interval is a Cauchy sequence $x$ such that 
  for any $n:\mathbb N$, we have $0\leq x_n \leq 1$. 
 % 
  The interval of Cauchy reals is given by the type of Cauchy sequences in the interval 
  modulo $\sim_C$. We denote it by $[0,1]$. 
\end{definition}  


We want to show that the interval of Cauchy reals is Compact Hausdorff. 
Informally, to any binary sequence $\alpha : \mathbb N \to 2$, 
we can associate a Cauchy sequence 
\begin{equation}\label{eqnBinaryEncode}
  n\mapsto \sum\limits_{i = 0 }^n \frac {\alpha(i)}{2^{i+1}}
\end{equation}
and we are going to give a closed relation on Cantor space such that 
two binary sequences are equivalent iff they correspond to the same Cauchy reals. 
%
First, we'll need some notation.
\begin{definition}
Given a binary sequence $\alpha:\mathbb N \to 2$ and a natural number $n : \mathbb N$  
we denote $\alpha|_n: \mathbb N_{\leq n} \to 2$ for the 
restriction of $\alpha$ to a finite sequence of length $n$. 
We denote $\overline 0, \overline 1$ for the binary sequences which are constantly $0$ and $1$ respectively. 
We denote $0,1$ for the sequences of length $1$ hitting $0,1$ respectively. 
If $x$ is a finite sequence and $y$ is any sequence, denote $x\cdot y$ for their concatenation. 
\end{definition} 
Now we'll give a definition for when two finite binary sequences of length $n$ correspond 
to real numbers whose distance is $\leq (\frac12)^n$.
Basically, we want for every finite sequence $z$ that 
$(z \cdot 0 \cdot \overline 1)$ and  $(z \cdot 1 \cdot \overline 0)$ are equivalent. 

\begin{definition}
Now let $n:\mathbb N$ and $x,y:\mathbb N_{\leq n} \to 2$ be two sequences of length $n$. 
We say $x,y$ are near if we have an $m:\mathbb N$ with $m\leq n$
and some $a: \mathbb N_{\leq m} \to 2$, 
such that one of $(a \cdot 0 \cdot \overline 1)|_n,  ( a \cdot 1 \cdot \overline 0)|_n$
is equal to $x$ and the other is equal to $y$. 
We denote $\text{near}_n(x,y)$ if $x,y$ are near. 
%
To be precise, we define 
\begin{equation}
  \text{near}_n(x,y) = 
\Sigma(m:\mathbb N) m \leq n \wedge 
  \Sigma (a : Fin_m \to 2) 
\bigg( \big( (x,y) = 
((a \cdot 0 \cdot \overline 1)|_n,  ( a \cdot 1 \cdot \overline 0)|_n)
\big)
\bigvee 
\big(
  (y,x) = 
((a \cdot 0 \cdot \overline 1)|_n,  ( a \cdot 1 \cdot \overline 0)|_n)
\big)
\bigg)
\end{equation}
\end{definition}
\begin{remark}
Remark that when $x,y$ are near, $m$ and $a$ as above are unique. 
Thus $\text{near}_n(x,y)$ is a proposotion. 
%
Furthermore, to check whether $x,y$ are near, we need only make $n$ comparisons, 
thus $\text{near}_n(x,y)$ is decidable. 
%
Note that in the above definition, we allow $m = n$ and therefore $x$ is near to itself for any finite sequence $x$. 
Furthermore, we have defined nearness to be symmetric. 
However, it is not a transtive relation. 
After all, the sequence $010$ and $011$ are near and the sequence $011$ and $100$ are near, 
but $010$ is not near to $100$. This corresponds to the fact that $\frac14$ and $\frac38$ are distance $\leq (\frac12)^3$
apart, and so are $\frac38$ and $\frac12$, but $\frac14$ and $\frac12$ are not. 
\end{remark}
\begin{definition}
  We define the following relation on Cantor space for $\alpha, \beta: 2^\mathbb N$.
  \begin{equation}
    \alpha \sim_t \beta = \forall (n : \mathbb N) 
    \text{near}_n(\alpha|_n, \beta|_n)
  \end{equation}
\end{definition}
\begin{lemma}
  $\sim_t$ is a closed equivalence relation. 
\end{lemma}
\begin{proof}
   Let $\alpha, \beta, \gamma : 2^\mathbb N$. 
   As the dependent product of propositions is a proposition, $\alpha \sim_t\beta$ is a proposition. 
   %
   Furthermore, the closedness follows from decidability of $\text{near}_n(\alpha|_n, \beta|_n)$. 
   One could define $\gamma(n) = 1$ iff $\text{near}_n(\alpha|_n, \beta|_n)$
   
   As nearness is reflexive and symmetric, so is $\sim_t$. 

   Now suppose $\alpha \sim_t \beta$ and $\beta\sim_t \gamma$. 
   We claim that $\alpha \sim_t \gamma$. 

   Let $n:\mathbb N$, we need to show that 
   $\text{near}_n(\alpha|_n , \gamma|_n)$. 
   Let $(a,m)$ witness that $\text{near}_n(\alpha|_n, \beta|_n)$.
   and let $(b, k)$ witness that $\text{near}_n(\beta|_n, \gamma|_n)$
   We will make a case distinction on whether one of $m,k$ is equal to $n$, or
   both are strictly smaller than $n$. 
   \begin{itemize}
     \item 
       If $m=n$, we have that $\alpha|_n = \beta|_n$, and therefore 
       \begin{equation}
         \text{near}_n(\beta|_n, \gamma|_n) \leftrightarrow \text{near}_n(\alpha|_n, \gamma|_n)
       \end{equation} 
       The above also holds if $k = n$.
     \item 
       If $m< n$, we have that $\alpha(m+1) \neq \beta(m+1)$, thus 
       $\alpha|_l \neq \beta|_l$ for all $l>m$, 
       but we still have $\text{near}_l(\alpha|_l, \beta|_l)$ for these $l$. 
       Therefore $(\alpha, \beta)$ or $(\beta, \alpha)$ must be of the form
       $(a \cdot 0 \cdot \overline 1, a \cdot 1 \cdot \overline 0)$. 
       WLOG, we assume $\alpha = a \cdot 0 \cdot \overline 1$, and thus 
       $\beta = a \cdot 1 \cdot \overline 0$ (if not, we could do bitflips). 

       As $k<n$ also, by the same argument there is some $b$ such that one of 
       $(\beta,\gamma), (\gamma, \beta)$
       is equal to $(b\cdot 0 \cdot \overline 1, b \cdot 1 \cdot \overline 0)$. 
       However, $\beta$ is also of the form $a \cdot 1 \cdot \overline 0$, and 
       thus cannot also be of the form $b \cdot 0 \cdot \overline 1$. 
       Therefore we must have 
       $\beta = b\cdot 1 \cdot \overline 0$ and 
       $\gamma= b\cdot 0 \cdot \overline 1$. 

       But now $b \cdot 1 \cdot \overline 0 = a \cdot 1 \cdot \overline 0$, 
       The lengths of $a,b$ cannot be unequal, and by decidablity of natural numbers, 
       $a,b$ have the same length and it follows that $ a = b$. 
       Therefore $ \alpha = \gamma$, so $\alpha \sim_t\gamma$.
   \end{itemize}

   We conclude that $\sim_t$ is a closed equivalence relation. 
\end{proof}

%\begin{lemma}
  $b$ sends $\sim_n$ equivalent binary sequences to $\sim_C$ equivalent Cauchy sequences. 
\end{lemma}
\begin{proof}
  Let $\alpha, \beta$ be binary sequences.
  We claim that $|b(\alpha)_n - b(\beta)_n| \leq (\frac12)^{n+1}$ 
  whenever $\text{near}_n(\alpha, \beta)$. 
  It will follow that if $\alpha\sim_n \beta$, then 
  $b(\alpha)\sim_C b(\beta)$. 

  Let $n:\mathbb N$ and assume $m:\mathbb N$ with $m\leq n$ and 
  let $z$ be a sequence of length $m$ such that 
  $\alpha|_n = z\cdot 1 \cdot \overline 0|_n$ and $\beta|_n = z \cdot 0 \cdot \overline q |_n$. 
  then $b(\alpha)_n = \sum_{i\leq m} \frac{z(i)}{2^{i+1}} + (\frac12)^{m+2}$ and 
  $b(\beta)_n = \sum_{i\leq m} \frac{z(i)}{2^{i+1}} + \sum\limits_{m+2 \leq i \leq n}(\frac12)^{i+1}$. 
  Thus 
  $b(\alpha)_n - b(\beta)_n = (\frac12)^{m+2} - \sum\limits_{m+2 \leq i \leq n}(\frac12)^{i+1} = 
  (\frac12)^{n+1}$, 
  which is smaller than required. 
\end{proof}  

\begin{lemma}
  Whenever $b(\alpha) \sim_C b(\beta)$, 
  we have $\alpha \sim_n \beta$. 
\end{lemma}
\begin{proof}
  Assume $b(\alpha) \sim_Cb (\beta)$. 
  Let $n:\mathbb N$. 
  We shall show that $\text{near}_n(\alpha , \beta)$. 

  As we're only checking finitely many entries, 
  we either have $\alpha|_n = \beta|_n$, 
  or there exists a smallest $m\leq n$ with 
  $\alpha(m) \neq \beta(m)$. 

  If $\alpha|_n = \beta|_n$, we have $\text{near}_n(\alpha,\beta)$ and are done. 
  WLOG assume $\alpha(m) = 1, \beta(m) = 0$ for $m$ minimal. 

  Now note that 
  \begin{equation} 
    b(\alpha)_{k+1} - b(\beta)_{k+1} = 
    b(\alpha)_{k} - b(\beta)_{k} + 
    \frac{\alpha(k+1) - \beta(k+1)}{2^{k+2}}.
  \end{equation}

  For $k>m$, we have that 
  \begin{equation}
  |b(\alpha)_k - b(\beta)_k |= 
  |(\frac12)^{m+1} + \sum\limits_{i=m+1}^k \frac{ \alpha(i) -\beta(i)}{2^{i+1}}|. 
  \end{equation}
  Note that the right summand is always $\leq (\frac12)^{m+1}$. 
  Therefore, we can leave out the absolute value function. 

  We claim that for every $k\geq m+1$, we have $\alpha(k) = 0, \beta(k) = 1$. 
  We will use induction. 
  Suppose that for every $m <i<j$, we have $\alpha(i) = 0$, and $\beta(i) = 1$. 
  Then 
  \begin{equation}
    b(\alpha)_{j-1} - b(\beta)_{j-1} = 
    (\frac12)^{m+1} + 
    \sum\limits_{i=m+1}^{j-1} \frac{ -1}{2^{i+1}} 
    = (\frac12)^{j}
  \end{equation}
   
  \begin{itemize}
    \item 
      we claim that $\alpha(j) = 0$ 
      Suppose $\alpha(j) = 1$. 
      Then $\alpha(j) -\beta(j) \geq 0$. 
      And for $j + 2$, we have that 
  \begin{align}
    &b(\alpha)_{j+2} - b(\beta)_{j+2}
    \\
    =  
    &(b(\alpha)_{j-1} - b(\beta)_{j-1}) + 
    &\frac{\alpha(j)-\beta(j)}{2^{j+1}} +  
    &\frac{\alpha(j+1) - \beta_(j+1)}{2^{j+2}}
    +
    &\frac{\alpha(j+2) - \beta_(j+2)}{2^{j+3}}
    \\
    \geq  
      & (\frac12)^j + &0 
    + &\frac{-1}{2^{j+2}} 
    + &\frac{-1}{2^{j+3}} 
    \\
      > &(\frac12)^{j+1}
  \end{align}
  which contradicts $b(\alpha) \sim_Cb(\beta)$, 
  which would require that $|b(\alpha)_{j+2} - b(\beta_{j+2} | \leq (\frac{12})^{j+2}+ (\frac12)^{j+2} = (\frac12)^{j+1}$. 
  Therefore $\alpha(j) \neq 1$, and thus $\alpha(j) = 0$. 
    \item 
      We also claim that $\beta(i) = 1$. 
      If $\beta(i) = 0$, we also have 
      $\alpha(j) -\beta(j) \geq 0$, and the rest of the proof is similar as above. 
  \end{itemize}
\end{proof}


%\begin{lemma}
  The map $b: 2^\N \to [0,1]$ is surjective. 
\end{lemma}
\begin{proof}
  First, suppose we have a function 
  $d:\Pi_{x:\mathbb R} \Pi_{q: \mathbb Q} (x \leq q + x \geq q)$
  Then we could recursively define 
  $$\alpha(n) = \begin{cases}
    0 \text{ if } d(x - \sum\limits_{i<n} \frac{\alpha(i)}{2^{i+1}} , \frac{1}{2^{n+1}}) = inl(\cdot) \\
    1 \text{ otherwise}
  \end{cases}
  $$
%  Recall that inequality between rational numbers is decidable, therefore we can define
%  $$\alpha(n) = \begin{cases}
%    0 \text{ if } |x_n - \sum\limits_{i<n} \frac{\alpha(i)}{2^{i+1}}| \leq  \frac{1}{2^{n+1}} \\
%    1 \text{ otherwise}
%  \end{cases}
%  $$
  Note that 
  $$\alpha(n) = \begin{cases}
    0 \text{ if } d(x - b(\alpha)_{n-1} , \frac{1}{2^{n+1}}) = inl(\cdot) \\
    1 \text{ otherwise}
  \end{cases}
  $$
  We'll show by induction that $b(\alpha)_n \leq x$ for every $n:\N$. 
  First $b(\alpha)_0 = 0 \leq x$. 
  Assuming, $b(\alpha)_k \leq x$, for $b(\alpha)_{k+1}$, 
  there are two cases:
  \begin{itemize}
    \item 
     if $d(x -  b(\alpha)_k, \frac{1}{2^{n+1}}) = inl(\cdot)$, 
     then $b(\alpha)_{k+1} = b(\alpha)_k$, which is $\leq x$ by induction hypothesis. 
   \item 
     Otherwise, $ x - b(\alpha)_k \geq (\frac12)^{k+1}$
     So $x-b(\alpha)_k - (\frac12)^{k+1} \geq 0$, 
     and $b(\alpha)_{k+1} = b(\alpha)_k + (\frac12)^{k+1}$. 
     So $x-b(\alpha)_{k+1} \geq 0$, and $b(\alpha)_{k+1} \leq x$ as required. 
 \end{itemize}
 So by induction $b(\alpha)_n\leq x$ for every $n:\N$. 
 Therefore, $|x-b(\alpha)_n| = x-b(\alpha)_n$. 
  
  We shall also show by induction that 
  $ x- b(\alpha)_n \leq (\frac12)^{n+1} $
  for every natural number $n:\N$. 
%
  For $n = 0$, this follows from the assumption that $x\leq 1$. 
%
  Suppose that $ x- b(\alpha)_k  \leq (\frac12)^{k+1} $. 
  We make a case distinction on the form of $d(x-b(\alpha)_k, (\frac12)^{k+2})$.
  \begin{itemize}
    \item 
      If $d(x-b(\alpha)_k , (\frac12)^{k+2}) = inl(\cdot)$, 
      then $  x-b(\alpha)_k  \leq (\frac12)^{k+2}$, 
      and $b(\alpha)_{k+1} = b(\alpha)_k$, 
      and $x-b(\alpha)_{k+1}  \leq (\frac12)^{k+2}$ as well, 
      as required. 
    \item 
      Otherwise, we must have
      $ x- b(\alpha)_k  \geq (\frac12)^{k+2}$, 
      and $b(\alpha)_{k+1} = b(\alpha)_k + (\frac12)^{k+1}$.
      By induction hypothesis, we have 
      $x-b(\alpha)_k \leq (\frac12)^{k+1}$. 
      Thus \begin{equation}
        x-b(\alpha)_{k+1} = x - b(\alpha)_k - (\frac12)^{k+1}
        \leq (\frac12)^{k+1} - (\frac12)^{k+2} = (\frac12)^{k+2}
      \end{equation}
      as required. 
  \end{itemize}
  
  By induction, we conclude that 
  $ | b(\alpha)_n - x |  \leq (\frac12)^{n+1} $
  for every $n:\N$. 
  Therefore $b(\alpha)$ converges to $x$. 

  We may conclude that $\Pi_{x:[0,1]} \Pi_{q: \mathbb Q} (x \leq q + x \geq q)$ implies that 
  we can give for each $x: [0,1]$ a binary sequence $\alpha$ with $b(\alpha) = x$. 
  As we have the propositional trunctation of the premise by \Cref{ComparisonLemma}, 
  we may conclude that for each $x:[0,1]$ there merely exists $\alpha$ with $b(\alpha) = x$. 
  Therefore $b$ is surjective. 
\end{proof}



%
%
%
%\begin{theorem}
%  The interval of Cauchy reals is isomorphic to $2^\N / \sim_t$. 
%\end{theorem} 
%\begin{proof}
%  This follows from the fact that $b:2^\N$ is such that $\alpha\sim_n \beta$ iff $b(\alpha)\sim_t b(\beta)$. 
%  and for every Cauchy real, there is a binary sequence being sent to it, so the composition of $b$ and the 
%  quotient from Caucy sequences to Cauchy real is a surjection. 
%\end{proof}
%
%\begin{corollary}
%  The interval is compact Hausdorff. 
%\end{corollary}
