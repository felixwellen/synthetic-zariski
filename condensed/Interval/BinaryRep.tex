In this section we will introduce the unit interval $I$ as compact Hausdorff space. 
The definition is based on \cite{Bishop}. 
We will then calculate the cohomology of $I$. 
For a proof that the unit interval corresponds to the definition using Cauchy sequences, 
we refer to the appendix. 


\input{Interval/CauchySequences}
\input{Interval/BinaryClosedEquivalence}
\input{Interval/EquivalenceOfSims}
\input{Interval/Surjective}





\begin{theorem}
  The interval of Cauchy reals is isomorphic to $2^\N / \sim_t$. 
\end{theorem} 
\begin{proof}
  This follows from the fact that $b:2^\N$ is such that $\alpha\sim_n \beta$ iff $b(\alpha)\sim_t b(\beta)$. 
  and for every Cauchy real, there is a binary sequence being sent to it, so the composition of $b$ and the 
  quotient from Caucy sequences to Cauchy real is a surjection. 
\end{proof}

\begin{corollary}
  The interval is compact Hausdorff. 
\end{corollary}
