\paragraph{Extensional convergence }  
\begin{definition}
  Let $B_\infty$ be the Boolean algebra on countably many generators $(p_n)_{n:\N}$ 
  over the equivalence $p_n\wedge p_m = 0 $ whenever $n \neq m$. 
\end{definition} 
\begin{definition}
  We denote $\Noo$ be the spectrum of $B_\infty$. 
\end{definition} 
\begin{lemma}
  $B_\infty$ is isomorphic with the Boolean algebra of 
  finite/cofinite subsets of $\N$. 
\end{lemma}
\begin{proof}
  To go from $B_\infty$ to subsets of $\N$, we send
  the generators $p_n$ to the singleton $\{n\}$, which are clearly finite. 
  We call the induced Boolean operation $f$. 

  To go from finite/cofinite subsets of $\N$ to $B_\infty$,
  a finite subset $I$ of $\N$ is sent to the element 
  $\bigvee_{i \in I} p_i$, and a cofinite subset $J$ is sent to the element 
  $\bigwedge_{i \in J^C} \neg p_i$.  
  We call this function $g$ and we need to show that $g$ is a Boolean morphism. 
  \begin{itemize}
    \item By deMorgan's laws, $g$ preserves $\neg$. 
    \item To see that $g$ respects $\vee$, we need to check three cases
      \begin{itemize}
        \item If both $I,J$ are finite, then 
        \begin{equation} 
          g(I \cup J) = \bigvee_{i\in I \cup J} p_i= \bigvee_{i\in I} p_i \vee \bigvee_{j\in J} p_j 
        \end{equation}
      \item If both $I,J$ are cofinite, we have
        \begin{equation}
          g(I) \vee g(J) = 
          \bigwedge_{i \in I^C} \neg p_i \vee 
          \bigwedge_{j \in J^C} \neg p_j 
          = 
          \bigwedge_{i\in I^C} 
          \bigwedge_{j \in J^C}(\neg p_i \vee  \neg p_j) 
        \end{equation}
        Now note that $\neg p_i \vee \neg p_j = \neg ( p_i \wedge p_j)$, which 
        is $1$ if $i \neq j$ and $p_i$ if $i =j$. 
        We can leave $1$ out of the meet, and we are left with the intersection of $I^C$ and $J^C$, so
        \begin{equation}
          g(I) \vee g(J) = 
          \bigwedge_{i \in (I^C \cap J^C)} \neg p_i
          = 
          \bigwedge_{i \in (I \cup J)^C} \neg p_i 
        \end{equation} 
        as the union of $I$ and $J$ is also cofinite, this equals 
          $ g( I \cup J)$. 
        \item If $I$ is finite and $J$ cofinite, we have 
        \begin{equation}
        g(I) \vee g(J) = (\bigvee_{i\in I} p_i) \vee (\bigwedge_{j \in J^C} \neg p_j)
        = \bigwedge_{j \in J^C} (\bigvee_{i \in I}( p_i \vee \neg p_j))
        \end{equation}
        If $i\neq j$, then $p_i\wedge p_j = 0$, hence $\neg p_j \geq p_i$ and $p_i \vee \neg p_j  = \neg p_j$
        If $i = j$, then $p_i \vee \neg p_j = 1$.
%        \begin{equation}
%        g(I) \vee g(J) = 
%        = \bigwedge_{j \in J^C} (\bigvee_{i \in I-J}( p_i \vee \neg p_j))
%        \end{equation}
%
%        \item If $I$ is cofinite and $J$ is finite, we have that $I \cup J$ is cofinite.
%        Thus 
%        \begin{equation}
%          g(I \cup J) = \bigwedge_{i \in (I \cup J)^C} \neg p_i
%        \end{equation}
%
      \end{itemize}
    \item The case for $\wedge$ is completely dual to the case for $\vee$. 
  \end{itemize}
We conclude that $g$ is a Boolean morphism. 
Furthermore, $g$ and $f$ are clearly inverses, thus the Boolean algebras are isomorphic. 
\end{proof}

  \begin{lemma}\label{lemBinftyNormalForm}
  Any element of $B_\infty$ can be written as 
  either $\bigvee_{i\in I} p_i$  or
  as $\bigwedge_{j\in J} \neg p_j$ 
  for finite $I,J\subseteq \N$. 
\end{lemma}
\begin{proof}
  Remark that whenever $n \neq m$, we have that 
  $\neg p_n \geq p_m$ as $p_m \wedge p_n = 0$. 
\end{proof}
There is canonical embedding $\N \hookrightarrow \Noo$, 
wich sends $n$ to the unique function $\chi_{n}$ sending $p_n$ to $1$. 
We denote $\infty \in \Noo$ for the function which is constantly $0$. 
By \Cref{PropMarkov}, if an element is not $\infty$, 
it comes from the embedding $\N \hookrightarrow \Noo$. 
\begin{lemma}\label{LemmaOpensContainingInfty}
  Let $U$ be an open subset of $\Noo$ containing $\infty$.
  Then there merely exists an $N:\N$ such that whenever $n\geq N$, 
  $\chi_n\in U$ as well. 
\end{lemma}
\begin{proof}
  It is sufficient to prove the lemma for $U$ a basic open. 
  Assume $b : B_\infty $ is such that 
  $U = \{ \phi: B_\infty \to 2| \phi(b) = 1\}$.
  Assume furthermore that $\infty \in U$.
%  so $U$ contains the function sending every $p_i$ to $0$. 
  by \Cref{lemBinftyNormalForm}, $b$ can have two forms.
  If $b = \vee_{i\in I} p_i$, then as $\infty(b) = 0$, 
  we must have $I = \emptyset$, and thus $b = 0$, 
  which means $U$ is empty, contradicting $\infty\in U$. 
  Therefore, 
  $b$ must be of the form $\wedge_{j \in J} \neg p_j$. 
  Note that for $N = \max J + 1$, whenever $n>J$, 
  $\chi_n$  sends $b$ to $1$. 
  Thus $\chi_n \in U$ as well, and we are done. 
\end{proof}

\begin{definition}
  Let $\alpha$ be a sequence in $X$, we say that $\alpha$
  is convergent iff there exists an extension. 
  \begin{equation}\begin{tikzcd}
    \N \arrow[r, "\alpha"] \arrow[d,hook]  & X \\
    \Noo \arrow[ru,dashed]
  \end{tikzcd}\end{equation}  
\end{definition}  



\begin{proposition}
  A sequence is convergent iff it has a limit
\end{proposition}
\begin{proof}
  Let $\alpha$ be convergent, with extension $\overline \alpha$.
  we claim that $\overline \alpha(\infty)$ is a limit of $\alpha$.
  Let $U \subseteq X$ be an open containing $x$. 
  As $\overline\alpha^{-1}(U)$ is an open subset of $\Noo$ containing $\infty$,
  \Cref{LemmaOpensContainingInfty} tells us there exists some $N$ such that $[N,\infty]\subseteq \overline \alpha^{-1}(U)$. 
  Thus there exists an $N$ such that for $n\geq N$, we have $\alpha(n) \in U$, as required. 

  Conversely, suppose $\alpha$ has limit $x$. 
  Assume $X = Sp(B)$, and let $b\in B$. Then $b$ corresponds to a decidable subset $U\subseteq X$.
  For any decidable subset $U \subseteq X$, we have 
  $\alpha^{-1}(U)$ a decidable subset of $\N$. 
  We claim that $\alpha^{-1}(U)$ is either finite or cofinite. 
  As $U$ is decidable, we can decide wheter $x\in U$. If $x\in U$, $\alpha^{-1}(U)$ is cofinite, as 
  $\alpha(n) \in U$ for all $n \geq N$ for some $N$. 
  If $x\notin U$, we have $x\in U^C$, which is also decidable and therefore $\alpha^{-1}(U^C)$ is cofinite. 
  As $\alpha^{-1}(U) ^ C = \alpha^{-1}(U^C)$, it follows that $\alpha^{-1}(U)$ is finite. 
  Thus $\alpha^{-1}(U)$ is finite or cofinite for any decidable subset $U\subseteq X$. 
  Finite and cofinite subsets of $\N$ correspond to elements of $B_\infty$. 
  Therefore, $\alpha$ induces a map $B \to B_\infty$, which corresponds to a map 
  $\overline \alpha: \Noo \to X$. 

  We claim that $\overline \alpha$ extends $\alpha$. 
  Denote $\iota$ for the map $\N \to \Noo$. 
  We need to show that $\overline \alpha \circ \iota = \alpha$. 
  By definition, we have that $(\overline \alpha \circ \iota)^{-1}(U) = \alpha^{-1}(U)$ 
  for any decidable $U\subseteq X$. 
\end{proof}

\begin{lemma}
  Whenever $S = Sp(B)$ Stone, $f,g: A \to S$, and $f^{-1}(U) = g^{-1}(U)$ for any decidable 
  $U\subseteq S$, we have $f = g$. 
\end{lemma}
\begin{proof}
  By our assumption, we have for all $a:A$ that $f(a) \in U \iff g(a) \in U$ for 
  any decidable $U\subseteq X$. Such $U$ correspond to $b:B$.
  and $f(a) \in U \iff f(a)(b)  = 1$. 
  So the functions $f(a),g(a):B \to 2$ are such that 
  $f(a) (b) = g(a) (b)$ for all $b:B$. 
  This holds for all $a:A$ and by two uses of function extensionality we may conclude 
  $f=g$. 
\end{proof}


