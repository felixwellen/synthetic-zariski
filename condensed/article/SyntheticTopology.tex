\subsection{Types as spaces}
\begin{remark}
  For $X$ a type, a subtype $U$ of $X$ is a map $U:X\to\Prop$ 
  into the type of propositions. %a subtype of $X$. 
  We write $U\subseteq X$.
  If $X$ is a set, a we may call $U$ a subset. % for emphasis.
  For subtypes $A,B\subseteq X$, we write $A\subseteq B$ for pointwise implication.
  We will freely switch between subtypes $U:X\to\Prop$ and the corresponding embeddings
  $
    \sum_{x:X}U(x) \hookrightarrow  X.
  $
In particular, if we write $x\in U$ or $x:U$ 
we mean that $x:\sum_{y:X}U(y)$ -- but we might silently project $x$ to $X$.
%We will also denote $x\in U$ or $U(x)$ if we know that $x:X$. 
\end{remark}


The subobject $\Open$ of the type of propositions induces a topology on every type. 
This is the viewpoint taken in synthetic topology. 
We will follow the terminology of \cite{SyntheticTopologyEscardo, SyntheticTopologyLesnik}. 
%other references include \cite{SyntheticTopologyEscardo}%, TODOSortOutTaylorsReferences}.
%Defining a topology in this way has some benefits, which we summarize in this section. 

\begin{definition}
  Let $T$ be a type, and let $A\subseteq T$ be a subtype. 
  We call $A\subseteq T$ open (resp. closed) if $A(t)$ is open (resp. closed) for all $t:T$.
\end{definition}

\begin{remark}
  It follows immediately that the pre-image of an open by any map of types is open, so that any map is continuous. 
%  This is only relevant for a space if the topology we defined above matches the topology one would expect. 
  In \Cref{StoneClosedSubsets}, we shall see that the resulting topology is as expected for second countable Stone spaces.
  In \Cref{IntervalTopologyStandard}, we shall see that the same holds for the unit interval. 
\end{remark}



%\begin{remark}
%  Phao's principle is a special case of directed univalence. 
%\end{remark}
%\begin{proof}
%  \rednote{TODO}
%\end{proof}
