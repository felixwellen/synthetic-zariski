% latexmk -pdf -pvc main-fancy.tex
\documentclass[a4paper,UKenglish,cleveref, autoref, thm-restate, numberwithinsect]{lipics-v2021}
\RequirePackage{amsmath,amssymb,mathtools,amsxtra,stmaryrd}
\RequirePackage{tikz}
\RequirePackage{cleveref}
\usetikzlibrary{arrows, cd, babel}
\usepackage{zariski}
\newtheorem{axiom}{Axiom}

%This is a template for producing LIPIcs articles. 
%See lipics-v2021-authors-guidelines.pdf for further information.
%for A4 paper format use option "a4paper", for US-letter use option "letterpaper"
%for british hyphenation rules use option "UKenglish", for american hyphenation rules use option "USenglish"
%for section-numbered lemmas etc., use "numberwithinsect"
%for enabling cleveref support, use "cleveref"
%for enabling autoref support, use "autoref"
%for anonymousing the authors (e.g. for double-blind review), add "anonymous"
%for enabling thm-restate support, use "thm-restate"
%for enabling a two-column layout for the author/affilation part (only applicable for > 6 authors), use "authorcolumns"
%for producing a PDF according the PDF/A standard, add "pdfa"

%\pdfoutput=1 %uncomment to ensure pdflatex processing (mandatatory e.g. to submit to arXiv)
%\hideLIPIcs  %uncomment to remove references to LIPIcs series (logo, DOI, ...), e.g. when preparing a pre-final version to be uploaded to arXiv or another public repository

%\graphicspath{{./graphics/}}%helpful if your graphic files are in another directory

\bibliographystyle{plainurl}% the mandatory bibstyle

\title{A Foundation for Synthetic Stone Duality}

%\titlerunning{Dummy short title} %TODO optional, please use if title is longer than one line
%,  and 
\author{Felix {Cherubini}}{University of Gothenburg and Chalmers University of Technology, Sweden \and \url{felix-cherubini.de}}{felix.cherubini@posteo.de}{https://orcid.org/0000-0002-6589-1874}{}
\author{Thierry {Coquand}}{University of Gothenburg and Chalmers University of Technology, Sweden \and \url{https://www.cse.chalmers.se/~coquand/}}{Thierry.Coquand@cse.gu.se}{https://orcid.org/0000-0002-5429-5153}{}
\author{Freek {Geerligs}}{University of Gothenburg and Chalmers University of Technology, Sweden \and \url{https://www.gu.se/en/about/find-staff/freekgeerligs}}{geerligs@chalmers.se}{https://orcid.org/0009-0003-6938-4807}{}
\author{Hugo {Moeneclaey}}{University of Gothenburg and Chalmers University of Technology, Sweden \and \url{https://www.hugomoeneclaey.com/}}{hugomo@chalmers.se}{https://orcid.org/0000-0002-8579-0077}{}

\authorrunning{F.\ Cherubini, T.\ Coquand, F.\ Geerligs and H.\ Moeneclaey}

\Copyright{Felix Cherubini, Thierry Coquand, Freek Geerligs and Hugo Moeneclaey}

\begin{CCSXML}
  <ccs2012>
  <concept>
  <concept_id>10003752.10003790.10011740</concept_id>
  <concept_desc>Theory of computation~Type theory</concept_desc>
  <concept_significance>500</concept_significance>
  </concept>
  </ccs2012>
\end{CCSXML}

\ccsdesc[100]{Theory of computation~Type theory}

\keywords{Homotopy Type Theory, Synthetic Topology, Cohomology}

\category{} %optional, e.g. invited paper

\relatedversion{} %optional, e.g. full version hosted on arXiv, HAL, or other respository/website
%\relatedversiondetails[linktext={opt. text shown instead of the URL}, cite=DBLP:books/mk/GrayR93]{Classification (e.g. Full Version, Extended Version, Previous Version}{URL to related version} %linktext and cite are optional

%\supplement{}%optional, e.g. related research data, source code, ... hosted on a repository like zenodo, figshare, GitHub, ...
%\supplementdetails[linktext={opt. text shown instead of the URL}, cite=DBLP:books/mk/GrayR93, subcategory={Description, Subcategory}, swhid={Software Heritage Identifier}]{General Classification (e.g. Software, Dataset, Model, ...)}{URL to related version} %linktext, cite, and subcategory are optional

%\funding{(Optional) general funding statement \dots}%optional, to capture a funding statement, which applies to all authors. Please enter author specific funding statements as fifth argument of the \author macro.

\acknowledgements{We thank Thierry Coquand for discussions on the topic and in particular for explaining a proof of \Cref{extend-from-image} to us.
We thank Marc Nieper-Wißkirchen for a discussion which led to the explanation in \Cref{remark-sym-dual}.
Work on this article was supported by the ForCUTT project, ERC advanced grant 101053291.}%optional

%\nolinenumbers %uncomment to disable line numbering



%Editor-only macros:: begin (do not touch as author)%%%%%%%%%%%%%%%%%%%%%%%%%%%%%%%%%%
\EventEditors{John Q. Open and Joan R. Access}
\EventNoEds{2}
\EventLongTitle{42nd Conference on Very Important Topics (CVIT 2016)}
\EventShortTitle{CVIT 2016}
\EventAcronym{CVIT}
\EventYear{2016}
\EventDate{December 24--27, 2016}
\EventLocation{Little Whinging, United Kingdom}
\EventLogo{}
\SeriesVolume{42}
\ArticleNo{23}
%%%%%%%%%%%%%%%%%%%%%%%%%%%%%%%%%%%%%%%%%%%%%%%%%%%%%%

\begin{document}

\maketitle

%TODO mandatory: add short abstract of the document
\begin{abstract}
  \input{PaperAbstract}
\end{abstract}

\section*{Introduction}
Grothendieck advocated for a functor of points approach to schemes early on in his
project of foundation of algebraic geometry (see the introduction of \cite{EGAI}).
In this approach, a scheme is defined as a special kind of (covariant) set valued functor
on the category of commutative rings. This functor should in particular
be a sheaf w.r.t.\ the Zariski topology. As a typical example, the projective space $\bP^n$
is the functor, which to a ring $A$,
associates the set of finitely presented sub-modules of  rank $1$ of
$A^{n+1}$, which are direct factors \cite{Demazure,Eisenbud,Jantzen}.

In the 70s, Anders Kock suggested to use the language of higher-order logic \cite{Church40}
to describe the Zariski topos, the collection of sheaves for the Zariski topology \cite{Kock74,kockreyes}.
This allows for
a more suggestive and geometrical description of schemes, that can now be seen as a special kind
of types satisfying some properties. Morphisms of schemes in this setting are just general maps.
There is in particular a ``generic
local ring'' $R$, which associates to $A$ its underlying set. As described in \cite{kockreyes}
the projective space $\bP^n$ is then the set of lines in $R^{n+1}$.

A natural question is if we can show in this setting that the automorphism group of $\bP^n$
is  $\PGL_{n+1}(R)$.
More generally, can we show that any map $\bP^n\rightarrow \bP^m$ is given by $m+1$ homogeneous
polynomials of same degree in $n+1$ variables?
From this, it is possible to deduce the corresponding result about $\bP^n$ defined as
a functor of points (but the maps are now {\em natural transformations}) or about $\bP^n$ defined
as a scheme (but the maps are now {\em maps of schemes}).
While this result is a basic text book in the case of projective space over a field, the general
case is more subtle.
(This general result, though fundamental, is not in \cite{Hartshorne} for instance.)
One goal of this paper is to present such a proof.

In \cite{draft}, we presented an axiomatisation of the Zariski {\em higher topos} \cite{lurie-htt},
using instead of the language of higher-order logic the language of dependent type theory
with univalence \cite{hott}. The first axiom is that we have a local ring $R$. We then define
an affine scheme to be a type of the form $\Sp(A) = \Hom_{\Alg{R}}(A,R)$ for some finitely presented
$R$-algebra $A$. The second axiom, inspired from the work of Ingo Blechschmidt \cite{ingo-thesis},
states that the evaluation map $A\rightarrow R^{\Sp(A)}$ is a bijection. The last axiom states
that each $\Sp(A)$ satisfies some form of local choice \cite{draft}. We can then define a notion
of {\em open} proposition, with the corresponding notion of open subset, and define a scheme as a type
covered by a finite number of open subsets that are affine schemes. In particular, we define
$\bP^n$ as in \cite{kockreyes} and show that it is a scheme.
In this setting, dependent type theory with univalence extended with these 3 axioms,
we show the above result about maps between $\bP^n$ and $\bP^m$ and the result about automorphisms of $\bP^n$.

Interestingly, though these results are
about the Zariski $1$-topos, the proof makes use of types that
are not (homotopy) sets (in the sense of \cite{hott}),
since it proceeds in characterizing $\bP^n\rightarrow\KR$, where $\KR$ is the delooping
(thus a type which is not a set) of the multiplicative group of units of $R$.
More technically, we also use such higher types as an alternative to the technique
of Quillen patching \cite{Quillen,lombardi-quitte,Lam}.



%% Schemes as special kind of sheaf for Zariski topos.

%% Even nicer in a type theoretic framework

%% Anders Kock property of Zariski topos.

%% Zariski topos higher logic

%% Definition of $\bP^n$ as a set of lines in $R^{n+1}$ coincides with the definition
%% of projective as functor of points (Demazure? Eisenbud?)

%% ``Geometric'' definition

%% Meyers, Blechschmidt use of type theory with univalence

%% Axiomatisation of the Zariski (higher) topos

%% A scheme is defined as a type satisfying some property and a map of schemes is {\em any} function
%% between the corresponding types




\section{Stone duality}
\subsection{Preliminaries}

In this section, we introduce the type of countably presented Boolean algebras $\Boole$ and of Stone spaces $\Stone$. 
Both of these types carry a natural category structure. 
In later sections, we will axiomatize an anti-equivalence between these categories, 
which is classically valid and called Stone duality. 

%\subsection{Countably Presented Boolean Algebras}
%We will use the type of countably presented (c.p.) boolean algebras $\Boole$,
%for more definitions and notation see \Cref{A-cp-boolean-algebras}.

%\subsection{Countably presented Boolean algebras}
\begin{definition}
  A countably presented Boolean algebra $B$ is a Boolean algebra such that there merely are 
  countable sets $I,J$, 
  a set of generators $g_i,~{i\in I}$ and a set $f_j,~{j\in J}$ of Boolean expressions over these generators 
  such that $B$ is equivalent to the quotient of the free Boolean algebra over the generators by the relations
  $f_j=0$. We denote this algebra by $2[I]/(f_j)_{j:J}$.
\end{definition} 

\begin{remark}
By countable in the previous definition we mean sets that are merely equal to a decidable in $\N$. Note that any countably presented algebra is also merely of the form $2[\N]\rangle / (r_n)_{n:\N}$, if we add dummy variables that we equate to $0$, and dummy relations that equate $0$ to itself.
\end{remark}

We will call the family $(f_j)_{j\in J}$ as above a set of relations. 
If $I,J$ are finite, we call $B$ a finitely presented Boolean algebra. 
Once we have postulated the axiom of dependent choice, 
in \Cref{secBooleAsColimits}
we will be able to show that every countably presented algebra 
is actually a colimit of a sequence of finitely presented Boolean algebras.
They are therefore dual to pro-finite objects, which are used 
in the theory of light condensed sets \cite{Scholze,Dagur,TODO}.

\begin{remark}
  We denote the type of countably presented Boolean algebras $\Boole$. 
  Note that this type does not depend on a choice of universe. 
\end{remark}

\begin{example}
  If both the set of generators and relations are empty, we have the Boolean algebra $2$.
  We have $0\neq_2 1$, and the underlying set of $2$ is given by $\{0,1\}$.
\end{example}
Note that any Boolean algebra must contain the elements $0,1$. 
Therefore, $2$ is the initial Boolean algebra. 
We can therefore use it to define points of objects in the category dual to that of countably presented Boolean algebras. 

%\subsection{Stone spaces}
\begin{definition}
  For $B$ a countably presented Boolean algebra, we define $Sp(B)$ as the set of Boolean morphisms from $B$ to $2$. 
\end{definition}
\begin{definition}
  We define the predicate on types $\isSt$ by 
  \begin{equation}
    \isSt(X) := \sum\limits_{B : Boole} X = Sp(B)
  \end{equation} 
  A type $X$ is called \textit{Stone} if $\isSt(X)$ is inhabited.
\end{definition}


%\subsection{Examples}
\begin{example}
  \label{boolean-algebra-examples}
  \begin{enumerate}[(i)]
  \item There is only one Boolean map $2\to 2$, thus $Sp(2)$ is the singleton type $\top$. 
  \item   The trivial Boolean algebra is given by the empty set of generators and the relation $\{1\}$.
    We have $0=1$ in the trivial Boolean algebra. 
    As there cannot be a map from the trivial Boolean algebra into $2$ preserving both $0,1$, 
    the corresponding Stone space is the empty type $\bot$, 
  \item\label{ExampleBAunderCantor}   
    We denote by $C$ the Boolean algebra $2[\N]$ given by $\N$ as a set of generators and no relations. We write $p_n$ for the generator corresponding to $n$.
    A morphism $C\to 2$ corresponds to a function $\mathbb N\to 2$, which is a binary sequence. 
    The Stone space $Sp(C)$ of these binary sequences is denoted $2^{\N}$ and called \notion{Cantor space}.
  \item\label{ExampleBAunderNinfty}
    We denote by $B_\infty$ the quotient of $C$ by the relations $p_m\wedge p_n$ for $m\neq n$. 
    A morphism $B_\infty\to 2$ corresponds to a function $\mathbb N \to 2$ that hits $1$ at most once. 
    The corresponding Stone space is denoted by $\N_\infty$. 
  \end{enumerate}
\end{example}

\begin{remark}\label{BinftyTermsWriting}
  In \Cref{N-co-fin-cp}, we will show that $B_\infty$ is equivalent to the Boolean algebra on 
  subsets of $\N$ which are finite or co-finite. 
  Under this equivalence, the generator $p_n$ is sent to the singleton $\{n\}$. 
  Because of this, we have that any $b:B_\infty$ can be written 
  either as $\bigvee_{i\in I_0} p_i$ or as $\bigwedge_{i\in I_0} \neg p_i$ for some finite $I_0\subseteq \N$. 
\end{remark}


%\begin{remark}
%  As Boolean algebras are rings, any relation of the form $f=g$ with both $f,g$ Boolean expressions 
%  can be written as $h=0$ with $h=f-g$ a Boolean expression. 
%\end{remark} 




\subsection{Statement of the axioms}%
\label{statement-of-axioms}

We always assume there is a fixed commutative ring $\A$.
In addition, we assume the following three axioms about $\A$,
which were already mentioned in the introduction,
but we will indicate which of these axioms are used to prove each statement
by listing their shorthands.

\begin{axiom}[Loc]%
  \label{loc}\index{Loc}
  $\A$ is a local ring (\Cref{local-ring}).
\end{axiom}

\begin{axiom}[SQC]%
  \label{sqc}\index{sqc}
  For any finitely presented $\A$-algebra $A$, the homomorphism
  \[ a \mapsto (\varphi\mapsto \varphi(a)) : A \to (\Spec A \to \A)\]
  is an isomorphism of $\A$-algebras.
\end{axiom}

\begin{axiom}[Z-choice]%
  \label{Z-choice}\index{Z-choice}
  Let $A$ be a finitely presented $\A$-algebra
  and let $B : \Spec A \to \mU$ be a family of inhabited types.
  Then there merely exist unimodular $f_1, \dots, f_n : A$
  together with dependent functions $s_i : \Pi_{x : D(f_i)} B(x)$.
  As a formula\footnote{Using the notation from \Cref{unimodular}}:
  \[ (\Pi_{x : \Spec A} \propTrunc{B(x)}) \to
     \propTrunc{ ((f_1,\dots,f_n):\Um(A)) \times
      \Pi_i \Pi_{x : D(f_i)} B(x) }
     \rlap{.}
  \]
\end{axiom}

\subsection{First consequences}

Let us draw some first conclusions from the axiom (\axiomref{sqc}),
in combination with (\axiomref{loc}) where needed.

\begin{proposition}[using \axiomref{sqc}]%
  \label{spec-embedding}
  For all finitely presented $\A$-algebras $A$ and $B$ we have an equivalence
  \[
    f\mapsto \Spec f : \Hom_{\AAlg}(A,B) = (\Spec B \to \Spec A)
    \rlap{.}
  \]
\end{proposition}

\begin{proof}
  By \Cref{algebra-from-affine-scheme}, we have a natural equivalence
  \[
    X\to \Spec (\A^X)
  \]
  and by \axiomref{sqc}, the natural map
  \[
    A\to \A^{\Spec A}
  \]
  is an equivalence.
  We therefore have a contravariant equivalence between
  the category of finitely presented $\A$-algebras
  and the category of affine schemes.
  In particular, $\Spec$ is an embedding.
\end{proof}

An important consequence of \axiomref{sqc}, which may be called \notion{weak nullstellensatz}:

\begin{proposition}[using \axiomref{loc}, \axiomref{sqc}]%
  \label{weak-nullstellensatz}
  If $A$ is a finitely presented $\A$-algebra,
  then we have $\Spec A=\emptyset$ if and only if $A=0$.
\end{proposition}

\begin{proof}
  If $\Spec A = \emptyset$
  then $A = \A^{\Spec A} = \A^\emptyset = 0$
  by (\axiomref{sqc}).
  If $A = 0$
  then there are no homomorphisms $A \to \A$
  since $1 \neq 0$ in $\A$ by (\axiomref{loc}).
\end{proof}

For example, this weak nullstellensatz suffices
to prove the following properties of the ring $\A$,
which were already proven in
\cite{ingo-thesis}[Section 18.4].

\begin{proposition}[using \axiomref{loc}, \axiomref{sqc}]%
  \label{nilpotence-double-negation}\label{non-zero-invertible}\label{generalized-field-property}
  
  \begin{enumerate}[(a)]
  \item An element $x:\A$ is invertible,
    if and only if $x\neq 0$.
  \item A vector $x:\A^n$ is non-zero,
    if and only if one of its entries is invertible.
  \item An element $x:\A$ is nilpotent,
    if and only if $\neg \neg (x=0)$.
  \end{enumerate}
\end{proposition}

\begin{proof}
  Part (a) is the special case $n = 1$ of (b).
  For (b),
  consider the $\A$-algebra $A \colonequiv \A/(x_1, \dots, x_n)$.
  Then the set $\Spec A \equiv \Hom_{\AAlg}(A, \A)$
  is a proposition (that is, it has at most one element),
  and, more precisely, it is equivalent to the proposition $x = 0$.
  By \Cref{weak-nullstellensatz},
  the negation of this proposition is equivalent to $A = 0$
  and thus to $(x_1, \dots, x_n) = \A$.
  Using (\axiomref{loc}),
  this is the case if and only if one of the $x_i$ is invertible.

  For (c),
  we instead consider the algebra $A \colonequiv \A_x \equiv \A[\frac{1}{x}]$.
  Here we have $A = 0$ if and only if $x$ is nilpotent,
  while $\Spec A$ is the proposition $\inv(x)$.
  Thus, we can finish by \Cref{weak-nullstellensatz},
  together with part (a) to go from $\lnot \inv(x)$ to $\lnot \lnot (x = 0)$.
\end{proof}

The following lemma,
which is a variant of \cite{ingo-thesis}[Proposition 18.32],
shows that $\A$ is in a weak sense algebraically closed.
See \Cref{non-existence-of-roots} for a refutation of
a stronger formulation of algebraic closure of~$\A$.

\begin{lemma}[using \axiomref{loc}, \axiomref{sqc}]%
  \label{polynomials-notnot-decompose}
  Let $f : \A[X]$ be a polynomial.
  Then it is not not the case that:
  either $f = 0$ or
  $f = \alpha \cdot {(X - a_1)}^{e_1} \dots {(X - a_n)}^{e_n}$
  for some $\alpha : \A^\times$,
  $e_i \geq 1$ and pairwise distinct $a_i : \A$.
\end{lemma}

\begin{proof}
  Let $f : \A[X]$ be given.
  Since our goal is a proposition,
  we can assume we have a bound $n$ on the degree of $f$,
  so
  \[ f = \sum_{i = 0}^n c_i X^i \rlap{.} \]
  Since our goal is even double-negation stable,
  we can assume $c_n = 0 \lor c_n \neq 0$
  and by induction $f = 0$ (in which case we are done)
  or $c_n \neq 0$.
  If $n = 0$ we are done,
  setting $\alpha \colonequiv c_0$.
  Otherwise,
  $f$ is not invertible (using $0 \neq 1$ by (\axiomref{loc})),
  so $\A[X]/(f) \neq 0$,
  which by (\axiomref{sqc}) means that
  $\Spec(\A[X]/(f)) = \{ x : \A \mid f(x) = 0 \}$
  is not empty.
  Using the double-negation stability of our goal again,
  we can assume $f(a) = 0$ for some $a : \A$
  and factor $f = (X - a_1) f_{n - 1}$.
  By induction, we get $f = \alpha \cdot (X - a_1) \dots (X - a_n)$.
  Finally, we decide each of the finitely many propositions $a_i = a_j$,
  which we can assume is possible
  because our goal is still double-negation stable,
  to get the desired form
  $f = \alpha \cdot {(X - \widetilde{a}_1)}^{e_1} \dots {(X - \widetilde{a}_n)}^{e_n}$
  with distinct $\widetilde{a}_i$.
\end{proof}

\subsection{Anti-equivalence of $\Boole$ and $\Stone$}

\begin{remark}\label{SpIsAntiEquivalence}
Stone types will take over the role of affine scheme from \cite{draft}, 
and we repeat some results here. 
Analogously to Lemma 3.1.2 of \cite{draft}, 
for $X$ Stone, Stone duality tells us that $X = Sp(2^X)$. 
%
Proposition 2.2.1 of \cite{draft} now says that 
$Sp$ gives a natural equivalence 
\begin{equation}
   Hom_{\Boole} (A, B) = (Sp(B) \to Sp(A))
\end{equation}
Therefore $Sp$ is an embedding from $\Boole$ to any universe of types, and $\isSt$ is a proposition.

Its image, $\Stone$ also has a natural category structure.
By the above and Lemma 9.4.5 of \cite{hott}, the map $Sp$ defines a dual equivalence of categories between $\Boole$ and $\Stone$.
\end{remark}

\begin{lemma}\label{SpectrumEmptyIff01Equal}
  For $B:\Boole$, we have $0=_B1$ iff $\neg Sp(B)$.
\end{lemma}
\begin{proof}
  Note that whenever $0=1$ in $B$, there is no map $B\to 2$ respecting both $0$ and $1$ as $0\neq 1$ in $2$. 
  Thus $\neg Sp(B)$ whenever $0=1$ in $B$. 
  % 
  Conversely, if $\neg Sp(B)$, then $Sp(B) = \emptyset$, which is also the spectrum of the trivial Boolean algebra. 
  As $Sp$ is an embedding, $B$ is equivalent to the trivial Boolean algebra, and $0=_B1$. 
\end{proof}

%\begin{corollary}\label{MoreConcreteCompleteness}
%  By the above and propositional completeness, we have that $||Sp(B)||$ iff $0\neq_B1$. 
%\end{corollary}


\begin{remark}\label{StoneClosedUnderPullback}
%  By \Cref {SpIsAntiEquivalence} and the fact that that countably presented Boolean algebras form a 
%  finitely cocomplete category (\Cref{CoCompletenessBoole}), the category of Stone spaces is complete. 
  By \Cref {SpIsAntiEquivalence} and the fact that that countably presented Boolean algebras are closed under pushouts, 
  the category of Stone spaces is closed under pullbacks. 
\end{remark}

We conclude this section on the anti-equivalence of Stone and $\Boole$ by a relating surjections to injections. 
This theorem is actually equivalent to completeness of propositional logic, which we'll discuss in 
\Cref{NotesOnAxioms}. 

\begin{theorem}\label{FormalSurjectionsAreSurjections}
  Let $f:A\to B$ be a map of countably presented Boolean algebras. 
  If $f$ is injective, then the corresponding map $(\cdot) \circ f : Sp(B) \to Sp(A)$ is surjective. 
\end{theorem}

\begin{proof}
  Assume $f$ injective and let $x:Sp(A)$.
  By \Cref{FiberConstruction}, we have that $\left(\sum\limits_{y:Sp(B)} y\circ f = x \right) = Sp(B/R) $
  for $R=f(G)$ for some countable $G\subseteq A$ with $x(g) = 0$ for all $g\in G$. 
  By propositional completeness and \Cref{SpectrumEmptyIff01Equal}, 
  it's sufficient to show that $0\neq_{B/R}1$. 
  Note that $0=_{B/R} 1$ iff 
  $1 =_B \bigvee R_0$ for some $R_0\subseteq R$ finite. 
  But then $$1 = \bigvee f(G_0) = f(\bigvee  G_0)$$ for some $G_0\subseteq G$ finite. 
  And as $f$ is injective, $\bigvee G_0 = 1$. 
  However, 
  $$
  x(\bigvee G_0) = 
  x(\bigvee_{g\in G_0} g ) = \bigvee_{g \in G_0} x(g) = \bigvee_{g\in G_0} 0 = 0$$
  And as $x(1) = 1$, we get a contradiction. Therefore $0\neq_{B/R} 1$ as required. 
\end{proof}  
The converse to the above theorem is true as well, regardless of propositional completeness:
\begin{lemma}\label{SurjectionsAreFormalSurjections}
If $f:A\to B$ is a map in $\Boole$ and $(\cdot) \circ f :Sp(B) \to Sp(A)$ is surjective, 
$f$ is injective. 
\end{lemma}
\begin{proof}
  Suppose precomposition with $f$ is surjective. 
  Let $a:A$ be such that $f(a)= 0$. 
  By assumption, for every $x:A\to 2$, there is a $y:B\to 2$ with $y\circ f = x$. 
  Consequentely $x(a) = y(f(a)) = y(0) = 0$. 
  So $x(a) = 0$ for every $x:Sp(A)$. 
  Thus $Sp(A) = Sp(A/\{a\})$, and as $Sp$ is an embedding, 
  $A \simeq A/\{a\}$, and $a = 0$ in $A$. 
  So whenever $f(a) = 0$, we have $a=0$. Thus $f$ is injective. 
\end{proof}

\subsection{Principles of omniscience}
In constructive mathematics, we do not assume the law of excluded middle (LEM).
There are some principles called principles of omniscience that are weaker than LEM, which can be used to describe 
how close a logical system is to satisfying LEM.
In this section, we will show that two of them (Markov's principle and LLPO) hold, 
and one (WLPO) fails in our system.

\begin{theorem}[The negation of the weak lesser principle of omniscience ($\neg$WLPO)]\label{NotWLPO}
  It is not the case that the statement %There is no method which given $\alpha:2^\mathbb N$ decides whether 
  $\forall_{n:\mathbb N} \alpha(n) = 0$ is decidable for general $\alpha:2^\mathbb N$. 
\end{theorem}
\begin{proof}
  Such a decision method would give a function $f:2^\mathbb N \to 2$ such that 
  $f(\alpha) = 0$ iff $\forall_{n:\mathbb N} \alpha (n)= 0$. 
  By Stone duality, there must be some $c:C$ with 
  $f(\alpha) = 0 \iff \alpha(c) = 0$. 
  $c$ is expressable using only finitely many generators $(p_n)_{n\leq N}$. 
  Now consider $\beta,\gamma:C \to 2$ given by $\beta(p_n) = 0$ for all $n:\mathbb N$ and
  $\gamma(p_n) = 0$ iff $n\leq N$. 
  Note that these functions are equal on $(p_n)_{n\leq N}$, therefore, $\beta(c) = \gamma(c)$. 
  However, $f(\beta) = 0$ and $f(\gamma) = 1$.
  We thus have a contradiction, thus a decision method as required doesn't exist. 
\end{proof}

The following result is due to David W\"arn:
\begin{theorem}[Markov's principle (MP)]\label{MarkovPrinciple}
  For $\alpha:\Noo$, we have that 
  \begin{equation}
    (\neg (\forall_{n:\mathbb N} \alpha (n)= 0)) \to \Sigma_{n:\mathbb N} \alpha (n)= 1
  \end{equation}
\end{theorem}
\begin{proof}
  Assume $\neg (\forall_{n:\mathbb N} \alpha (n)= 0)$.
%  Suppose $\alpha \neq \infty$.%, then it is not the case that $\alpha(n) = 0$ for all $n:\N$. 
  It is sufficient to show that $2/\{\alpha(n)|n\in\N\}$ is the trivial Boolean algebra. 
  It will then follow that there is a finite subset $N_0\subseteq \N$ 
  with $\bigvee_{i:N_0} \alpha(i) = 1$.
  As $\alpha(i) \in \{0,1\}$ and $\alpha(i) = 1$ for at most one $i$, it then follows that 
  there exists an unique $n\in\mathbb N$ with $\alpha(n) = 1$. 

  To show that $2/\{\alpha(n)|n\in\N\}$ is trivial, we will show it has an empty spectrum. 
  Suppose $x: 2 \to 2$ is such that $x(\alpha(n)) = 0$ for every $n:\N$. 
  As $x(1) = 1$, we must have for every $n:\N$ that $\alpha(n) \neq 1$. 
  But then $\alpha(n) = 0$, contradicting our assumption. 
  We get a contradicition and there thus there are no points in the spectrum of $2/\{\alpha(n)|n\in\N\}$ as required. 
\end{proof}

\begin{corollary}
  For $\alpha:2^\mathbb N$, we have that 
  \begin{equation}
    (\neg (\forall_{n:\mathbb N} \alpha (n)= 0)) \to \Sigma_{n:\mathbb N} \alpha (n)= 1
  \end{equation}
\end{corollary}
\begin{proof}
  Given $\alpha:2^\mathbb N$, consider the sequence $\alpha':\Noo$ satisfying $\alpha'(n) = 1$ iff 
  $n$ is minimal with $\alpha(n) = 1$. Then apply the above theorem.
\end{proof}

\begin{theorem}[The lesser limited principle of omniscience (LLPO)]\label{LLPO}
  For $\alpha:\N_\infty$, 
  we have that 
  \begin{equation}\label{eqnLLPO}
    \forall_{k:\N} \alpha(2k) = 0  \vee \forall_{k:\N} \alpha(2k+1) = 0
  \end{equation}
\end{theorem}
\begin{proof}
%
%  We first will define a map $f:B_\infty \to B_\infty \times B_\infty$. 
%  Because of \Cref{rmkMorphismsOutOfQuotient}, it is sufficient to define $f$ on $(p_n)_{n:\N}$ with 
%  $f(p_n) \wedge f(p_m) = (0,0)$ for $n\neq m$. 
%  To define $f(p_n)$, we use a case distinction on whether $n$ is odd or even. 
  Define $f:B_\infty \to B_\infty \times B_\infty$ as follows:
  \begin{equation}
    f(p_n) =\begin{cases}
      (p_k,0) \text{ if } n = 2k\\
      (0,p_k) \text{ if } n = 2k+1\\
    \end{cases}
  \end{equation}
  To see that $f$ is well-defined, we can make a case distinction on parity. 
%  By making a case distinction on $n,m$ being odd or even, 
%  we can see that 
%  $f(p_n) \wedge f(p_m) = (0,0)$ when $n\neq m$, thus $f$ is well-defined. 
  We claim $f$ is injective. Assume $f(x) = 0$, 
  to see that $x=0$, we make a case distinction on whether $x$ corresponds to a finite or a cofinite set 
  (\Cref{BinftyTermsWriting}).
%
%  We also claim it is injective.
%  Now let $x:B_\infty$ with $f(x) = 0$. 
  We denote $E,O\subseteq \N$ for the even and odd numbers respectively. 
%  and we make a case distincition based on \Cref{BinftyTermsWriting}.
  \begin{itemize}
    \item Suppose 
      $x = \bigvee_{i\in I_0} p_i$ with $I_0$ finite. 
      Then 
      $$f(x) = (\bigvee_{i\in I_0 \cap E } p_{\frac i2} , \bigvee_{i\in I_0 \cap O } p_{\frac {i-1}2} ) = (0,0)$$
      As $p_j\neq 0$ for all $j\in\N$, we must have $I_0 \cap E = \emptyset = I_0 \cap O$. 
      Thus $I_0= \emptyset$, and $x = 0$. 
    \item Suppose 
%      Let $x$ correspond to a cofinite subset of $\N$. Write 
      $x = \bigwedge_{j\in J} \neg p_j$ with $J$ finite. % for $J$ finite. 
      We will derive a contradiction. %, from which we can conclude that $x=0$ after all. 
      Note that   
      $$f(x) = (\bigwedge_{j\in J \cap E } \neg p_j , \bigwedge_{j\in J \cap O } \neg p_j ) = (0,0)$$
%      As $f(x) = (0,0)$, we have that 
%      $\bigwedge_{j\in J \cap E } \neg p_j =0$ and
%      $\bigwedge_{j\in J \cap O } \neg p_j  = 0$.
      However, any finite meet of negations corresponds to a cofinite set, hence is nonzero. 
      We get a contradiction and conclude $x=0$. 
%      However, any finite meet of negations will correspond to a cofinite set,
%      in particular it will not correspond to the empty set, and thus not be $0$.
%      Thus $f(x)\neq 0$, contradicting the assumption that $f(x) = 0$, hence $x=0$ ex falso. 
  \end{itemize}
%  In both cases, we conclude $x=0$, thus $f$ is injective. 
  By \Cref{FormalSurjectionsAreSurjections}, $f$ corresponds to a surjection 
  $s:\Noo + \Noo \to \Noo$.
  Thus for $\alpha : \Noo$, 
  there exists some $x:\Noo + \Noo$ such that $s x = \alpha$. 
  If $x = inl(\beta)$, 
  for any $k:\N$, we have that 
  $$\alpha (p_{2k+1}) = s(x) (p_{2k+1}) = x(f(p_{2k+1})) = inl(\beta) (0,p_k)  = \beta(0) = 0.$$
  Similarly, if $x = inr(\beta)$, we have $\alpha(p_{2k}) = 0$ for all $k:\N$. 
  Thus \Cref{eqnLLPO} holds for $\alpha$ as required. 
\end{proof}

The use of \Cref{FormalSurjectionsAreSurjections}, and hence of propositional completeness, 
was helpful in the above proof, as the following shows:
\begin{lemma}
  The above function $f$ does not have a retraction. 
\end{lemma}
\begin{proof}
  Suppose $r:B_\infty \times B_\infty \to B_\infty$ is a retraction of $f$. 
  Note that $r(0,1):B_\infty$ is expressable using only finitely many generators $(p_n)_{n\leq N}$
  Note that $r(0,1) \geq r(0,p_k) = p_{2k+1}$ for all $k:\N$. 
  As a consequence, $r(0,1)$ cannot be of the form $\bigvee_{i\in I_0} p_i$, and by \Cref{BinftyTermsWriting}, 
  $r(0,1)$ corresponds to a cofinite subset of $\N$. % = \bigwedge_{i:I_0} \neg p_i$, where $i\leq N$ for $i\in I_0$. 
  By similar reasoning so does $r(1,0)$.% corresponds to a cofinite subset of $\N$. 
  But the intersection of cofinite subsets is cofinite, while 
  $$r(0,1) \wedge r(1,0) = r( (1,0) \wedge (0,1)) = r(0,0) = 0$$
  which gives a contradiction. Thus no retraction exists. 
\end{proof}

We finish with an equivalent formulation of LLPO:

\begin{lemma}\label{corAlternativeLLPO}
  Let $(\phi_n)_{n:\N}, (\psi_m)_{m:\N}$ be families of decidable propositions indexed over $\N$.
  We then have 
  \begin{equation}
    (\forall_{n:\N} \forall_{m:\N} (\phi_n \vee \psi_m) )
    \leftrightarrow
    ((\forall_{n:\N} \phi_n) \vee (\forall_{m:\N} \psi_m) )
  \end{equation}
\end{lemma}

\begin{proof}
  Note that the implication from right to left in the above equation always holds.
  Assume that for all $m,n:\mathbb N$ we have $\phi_n\vee \psi_m$ 
  Consider the sequence $\alpha:2^\mathbb N$ where $\alpha(2n) = 0$ iff $\phi_n$ and 
  $\alpha(2m+1) = 0$ iff $\psi_m$. 
  Let $\beta:\Noo$ be such that $\beta(i) = 1$ iff $i$ is minimal with $\alpha(i) = 1$
  By LLPO, we have that 
  $\beta$ is $0$ on all odd entries or on all even entries. 
  Suppose that $\beta$ hits $0$ on all odd entries. 
  We will show $\psi_m$ for all $m:\N$. 
  As $\beta(2m+1) = 0$, there are two options:
  \begin{itemize}
    	\item If $\alpha(l)=0$ for all $l\leq 2m+1$. Then in particular $\alpha(2m+1)=0$ and $\psi_m$ holds.
	\item Otherwise there is some $l<2m+1$ with $\beta(l) = 1$. 
  As $\beta$ hits $0$ on odd entries, $l$ is even. 
  So $\alpha(2n) = 1$ for $n = \frac{l}2$, meaning that $\neg \phi_n$. 
  By assumption, $\phi_n \vee \psi_m$ holds, hence $\psi_m$ must hold. 
  Thus for all $m:\N$, we have $\psi_m$ if $\beta$ hits $0$ on all odd entries. 
  By a symmetric argument, if $\beta$ hits $0$ on all even entries, we have $\phi_n$ for all $n:\N$. 
  We conclude that 
  $((\forall_{n:\N} \phi_n) \vee (\forall_{m:\N} \psi_m) )$ 
  as required. 
  \end{itemize}
\end{proof}

\begin{remark}
Note that the above statement implies LLPO as $\alpha(2n) =0 \vee \alpha(2m+1) =0$ for all $n,m:\mathbb N$ if $\alpha:\Noo$. 
\end{remark}

In this section, we will define the types of open and closed propositions. 
These will allow us to define a (synthetic) topology  \cite{SyntheticTopologyLesnik} on any type.
We will study this topology on Stone types in particular.

\subsection{Open and closed propositions}
In this section we will introduce a topology on the type of propositions, and 
study their logical properties.
We think of open and closed propositions respectively as countable disjunctions and conjunctions of decidable propositions.
Such a definition is universe-independent, and can be made internally.

\begin{definition}
  We define the types $\Open, \Closed$ of open and closed propositions as follows:
  \begin{itemize}
    \item 
    A proposition $P$ is open iff there merely exists some $\alpha:2^\N$ such that 
      $P \leftrightarrow \exists_{n:\mathbb N} \alpha(n) = 0$. 
    \item 
    A proposition $P$ is closed iff there merely exists some $\alpha:2^\N$ such that 
      $P \leftrightarrow \forall_{n:\mathbb N} \alpha(n) = 0$. 
  \end{itemize}
\end{definition}

\begin{remark}\label{rmkOpenClosedNegation}
  The negation of an open proposition is closed, 
  and by Markov's principle (\Cref{MarkovPrinciple}), the negation of a closed proposition is open. 
  Also by Markov's principle, we have $\neg \neg P \to P$ whenever $P$ is open or closed. 
  By the negation of WLPO (\Cref{NotWLPO}), 
  not every closed proposition is decidable. 
  Therefore, not every open proposition is decidable. 
  % Both therefore and similarly can be used here, by a similar proof we can show it, or we can use that 
  % if $P$ is closed and $\neg P$ is decidable, so is $\neg \neg P = P$. 
  Every decidable proposition is both open and closed, 
  and in \Cref{ClopenDecidable} we shall see the converse. 
\end{remark}

\begin{lemma}\label{ClosedCountableConjunction}
  Closed propositions are closed under countable conjunctions.
\end{lemma}
\begin{proof}
  Let $(P_n)_{n:\N}$ be a countable family of closed propositions. 
  By countable choice, for each 
  $n:\N$ we have an $\alpha_n:2^\N $ 
  such that $P_n \leftrightarrow \forall_{m:\N} \alpha_n(m)  =0$. 
  Consider a surjection $s:\N \to \N \times \N$.
  Let 
  $$\beta(k) = \alpha_{\pi_0(s(k))}(\pi_1 (s(k))).$$
  Note that $\forall_{k:\N} \beta(k) = 0$ iff 
  $\forall_{m,n:\N}\alpha_m(n) = 0$, which happens iff $\forall_{n:\N} P_n$. 
  Hence the countable conjunction of closed propositions is closed. 
\end{proof} 

\begin{lemma}\label{OpenCountableDisjunction}
  Open propositions are closed under countable disjunctions. 
\end{lemma}
\begin{proof}
  Similar to the previous lemma. 
\end{proof}

\begin{corollary}\label{ClopenDecidable}
  If a proposition is both open and closed, it is decidable. 
\end{corollary}
\begin{proof}
  If $P$ is open and closed, $P\vee \neg P$ is open, hence
  equivalent to $\neg \neg (P \vee \neg P)$, which is provable. 
\end{proof}

\begin{lemma}\label{ClosedFiniteDisjunction} 
  Closed propositions are closed under finite disjunctions. 
\end{lemma}
\begin{proof}
  We shall show that 
  $(\forall_{n:\N} \alpha(n) = 0 )\vee (\forall_{n:\N} \beta(n) = 0 )$ is closed for any $\alpha,\beta:2^\N$.
  By \Cref{corAlternativeLLPO}, the statement is equivalent to 
  $ \forall_{n:\N}  \forall_{m:\N}  (\alpha(n) = 0 \vee \beta(m) = 0)$, 
  which is a countable conjunction of decidable propositions, 
  hence closed by \Cref{ClosedCountableConjunction}.
\end{proof}
\begin{lemma}\label{OpenFiniteConjunction}
  Open propositions are closed under finite conjunctions. 
\end{lemma}
\begin{proof}
  We need to show that for any $\alpha,\beta:2^\N$, the following proposition is open:
  \begin{equation}\label{eqnConjunctionOpen}
    (\exists_{n:\N} \alpha(n) = 0 )\wedge(\exists_{n:\N} \beta(n) = 0 )
  \end{equation}
  Consider $\gamma:2^\N$ given by 
  $\gamma(l) = 1$ iff there exist some $k,k'\leq l$ with 
  $\alpha(k) = \beta(k') = 0$. 
  As we only need to check finitely many combinations 
  of $k,k'$, this is a decidable property for each $l:\N$ and $\gamma$ is well-defined. 
  Then $\exists_{k:\N}\gamma(k)=0$ if and only if \Cref{eqnConjunctionOpen} holds.
\end{proof}

\begin{lemma}\label{OpenDependentSums}
  Open propositions are closed under dependent sums.
\end{lemma}
\begin{proof}
  First note that for $D$ a decidable proposition, and $X:D \to \Open$,
  by case splitting on $D$, we can see 
  $\Sigma_{d:D} X(d)$ is open.
%
  Then note that for $P$ an open proposition, 
  there exists a sequence of decidable propositions $A_n$ with 
  $P = \exists_{n:\N} A_n $.
%
  So for $Y : P \to Open $, the dependent sum $\Sigma_P Y$ is given by 
  $\exists_{n:\N} (\Sigma_{a:A_n} Y(n,a))$. 
  which is a countable a countable disjunction of open propositions, 
  hence open by \Cref{OpenCountableDisjunction}.
\end{proof}

We will see the same holds for closed propositions in \Cref{ClosedDependentSums}.

\begin{remark}\label{ImplicationOpenClosed}
  If $P$ is open, $P \to \bot$ is only open if $P$ is decidable, which is not in general the case. 
  Thus $\Open$ is not closed under dependent products. Neither is $\Closed$. 
  However, as $(P\to Q)  \to \neg \neg (\neg P \vee Q)$,
  we have that if $P$ is open and $Q$ is closed, then $P\to Q$ is closed, and similarly $Q\to P$ is open.
\end{remark}

\subsection{Types as spaces}
The subobject $\Open$ of the type of propositions induces a topology on every type. 
This is the viewpoint taken in synthetic topology. 
We will follow the terminology of \cite{SyntheticTopologyLesnik}, 
other references include \cite{SyntheticTopologyEscardo, TODOSortOutTaylorsReferences}.
%Defining a topology in this way has some benefits, which we summarize in this section. 

\begin{definition}
  Let $T$ be a type, and let $A\subseteq T$ be a subtype. 
  We call $A\subseteq T$ open or closed iff $A(t)$ is open or closed respectively for all $t:T$.
\end{definition}

\begin{remark}
  It follows immediately that the pre-image of an open by any map of types sends is open, so that any map is continuous. 
  This is only relevant for a space if the topology we defined above matches the topology one would expect. 
  In \Cref{StoneClosedSubsets}, we shall see that it resembles the standard topology of Stone spaces.
  In \Cref{IntervalClosedSubsets}, we shall see that it is the standard topology for the unit interval. 
\end{remark}

\begin{lemma}[transitivity of openness]\label{OpenTransitive}
  Let $T$ be a type, let $V\subseteq T$ open and let $W\subseteq V$ open. 
  Then the composite $W\subseteq V\subseteq T$ is open as well. 
\end{lemma}
\begin{proof}
  Denote $W'\subseteq T$ for the composite. 
  Note that $W'(t) = \Sigma_{v:V(t)} W(t,v)$. 
  As open propositions are closed under dependent sums (\Cref{OpenDependentSums}), 
  $W'(t)$ is an open proposition, as required. 
\end{proof}

\begin{remark}\label{OpenDominance}
  As the true proposition is open and openness is transitive, 
  $\Open$ can be called a dominance according to Proposition 2.25 of \cite{SyntheticTopologyLesnik}
\end{remark}



%\begin{remark}
%  Phao's principle is a special case of directed univalence. 
%\end{remark}
%\begin{proof}
%  \rednote{TODO}
%\end{proof}


\section{Overtly discrete spaces}
\input{Odisc}
\input{OpenAreOdisc}
\subsection{Relating $\ODisc$ and $\Boole$}
\begin{lemma}\label{BooleIsODisc}
  Every countably presented Boolean algebra is a sequential colimit of finite Boolean algebras. 
\end{lemma}
\begin{proof}
  Consider a countably presented Boolean algebra of the form $B = 2[\N]/(r_n)_{n:\N}$. 
%  We will show there exists a diagram of shape $\N$ taking values in Boolean algebras 
%  with $B$ as colimit.
%  \paragraph{The diagram}
  For each $n:\N$, let $G_n$ be the union of $\{g_i\ |\ {i\leq n}\}$ and 
  the finite set of generators occurring in $r_i$ for some $i\leq n$. 
  Denote $B_n = 2[G_n]/(r_i)_{i\leq n}$. 
  Each $B_n$ is a finite Boolean algebra, and there are canonical maps $B_n \to B_{n+1}$.
%  The inclusion $G_n \hookrightarrow G_{n+1}$ induces maps $B_n \to B_{n+1}$.
%  Hence $B_n,~n:\N$ is an $(\N,\leq)$-indexed sequence of finite sets. 
  Then $B$ is the colimit of this sequence. 
%
%  \paragraph{The colimit}
%  As $G_n\subseteq G$ and $R_n \subseteq R$, 
%  \Cref{rmkMorphismsOutOfQuotient} also gives us a map $B_n\to \langle G \rangle \langle R \rangle$. 
%  We claim the resulting cocone is a colimit. 
%
%  Suppose we have a cocone $C$ on the diagram $(B_n)_{n\in\N}$. 
%  We need to show that there exists a map $\langle G \rangle / R\to C$ and
%  we need to show this map is unique as map between cocones. 
%  \begin{itemize}
%    \item To show there exists a map $\langle G \rangle / R \to C$, 
%      we use remark \Cref{rmkMorphismsOutOfQuotient} again. 
%      Let $g\in G$ be the $n$'th element of $G$, 
%      note that $g\in G_n$, and consider the image of $g$ under the map $B_n \to C$. 
%      This procedure defines a function from $G$ to the underlying set of $C$. 
%      Let $\phi \in R$ be the $n$'th element of $R$, 
%      note that $\phi \in R_n$, and the map $B_n \to C$ must send $\phi$ to $0$. 
%      Thus the function from $G$ to the underlying set of $C$ also sends $\phi$ to $0$. 
%      This thus defines a map $\langle G \rangle / R \to C$. 
%    \item To show uniqueness, consider that any map of cocones $\langle G \rangle / \langle R \rangle \to C$ 
%      must take the same values on all $g\in G_n$ for all $n\in\N$. 
%      Now all $g\in G$ occur in some $G_n$, so any map of cocones $\langle G \rangle /  \langle R \rangle \to C$ 
%      takes the same values for all $g\in G$. 
%      \Cref{rmkMorphismsOutOfQuotient} now tell us that these values uniquely determine the map. 
%  \end{itemize}
\end{proof}




\begin{corollary}\label{ODiscBAareBoole}
  A Boolean algebra $B$ is overtly discrete if and only if it is countably presented. 
\end{corollary}
\begin{proof}
  Assume $B:\ODisc$. 
  By \Cref{OdiscQuotientCountableByOpen}, we get a surjection $\N\twoheadrightarrow B$ and that $B$ has open equality. 
  Consider the induced surjective morphism $f:2[\N]\twoheadrightarrow B$.
  By countable choice, we get for each $b:2[\N]$
  a sequence $\alpha_{b}:2^\N$ such that 
  $(f(b) = 0)\leftrightarrow \exists_{k:\N} (\alpha_{b}(k) =1)$. 
  Consider 
  $r:2[\N] \to \N \to 2[\N]$ 
  given by 
  \[r(b,k) =\begin{cases}
    b &\text{ if } \alpha_{b}(k) = 1\\
    0   &\text{ if } \alpha_{b}(k) = 0
  \end{cases}
  \] 
  Then $B= 2[\N]/(r(b,k))_{b:2[\N],k:\N}$.
  %By countable choice, we get for each $a,b:2[\N]$
  %a sequence $\alpha_{a,b}:2^\N$ such that 
  %$(f(a) = f(b))\leftrightarrow \exists_{k:\N} (\alpha_{a,b}(k) =1)$. 
  %Consider 
  %$r:2[\N] \times 2[\N] \times \N \to 2[\N]$ 
  %given by 
  %$$r(a,b,k) =\begin{cases}
  %  a-b &\text{ if } \alpha_{(a,b)}(k) = 1\\
  %  0   &\text{ if } \alpha_{(a,b)}(k) = 0
  %\end{cases}
  %$$
  %Then $B= 2[\N]/(r(a,b,k))_{(a,b,n): F \times F \times \N}$. 
  The converse comes from \Cref{BooleIsODisc}.
\end{proof}

\begin{remark}\label{BooleEpiMono}
%  In particular equality in overtly discrete types is open. 
  By \Cref{OdiscSigma} and \Cref{ODiscBAareBoole}, 
  it follows that any 
  $g:B\to C$ in $\Boole$ has an overtly discrete kernel.
  As a consequence, the kernel is enumerable and $B/Ker(g)$ is in $\Boole$. 
  By uniqueness of epi-mono factorizations and \Cref{SurjectionsAreFormalSurjections}, 
  the factorization 
  $B\twoheadrightarrow B/Ker(g) \hookrightarrow C$ corresponds to 
  $\Sp(C) \twoheadrightarrow \Sp(B/Ker(g)) \hookrightarrow \Sp(B)$. 
\end{remark}
\begin{remark}\label{decompositionBooleMaps}
  Similarly to \Cref{lemDecompositionOfColimitMorphisms} and 
  \Cref{lemDecompositionOfEpiMonoFactorization}, a (resp. surjective, injective) morphism
  in $\Boole$ is a sequential colimit of (resp. surjective, injective) morphism between 
  finite Boolean algebras.
\end{remark}

%\input{OvertlyDiscrete/BooleMaps}
%\input{OvertlyDiscrete/KernelFactorization}
%\section{Topology}
\section{Stone spaces}

\subsection{Stone spaces as profinite sets}
%\begin{remark}\label{StoneClosedUnderPullback}
%  Note that Boolean algebras are closed under finite colimits. 
%  By \Cref{ODiscBAareBoole} and \Cref{ODiscFiniteColim}, $\Boole$ is closed under finite colimits.
%  By \Cref {SpIsAntiEquivalence}  it follows that the category of Stone spaces is closed under finite limits. 
%\end{remark}
Here we present Stone spaces as sequential limits of finite sets. 
This is the perspective taken in Condensed Mathematics \cite{Condensed,Dagur,Scholze}.
Some of the results in this section are versions of the axioms used in 
\cite{bc24}. A full proof of all these axioms is part of future work. 

\begin{lemma}
  Any $S:\Stone$ is a sequential limit of finite sets. 
\end{lemma}
\begin{proof}
  Assume $B:\Boole$. By \Cref{SpIsAntiEquivalence} and \Cref{BooleIsODisc}, %and \Cref{ODiscClosedUnderSequentialColimits}, 
 we have that $\Sp(B)$ is a sequential limit of spectrum of finite Boolean algebras, which are finite sets. 
\end{proof}

\begin{lemma}\label{StoneAreProfinite} 
  A sequential limit of finite sets is a Stone space. 
\end{lemma}
\begin{proof}
  %For finite sets, we have that $\Sp(2^{S_n}) = S_n$, hence each $S_n$ is Stone. 
  By \Cref{SpIsAntiEquivalence} and %\Cref{BooleIsODisc}
  \Cref{ODiscClosedUnderSequentialColimits}, 
  we have that $\Stone$ is closed under sequential limits, and finite sets are Stone.
%  As $\Boole$ is closed under sequential colimits, \Cref{SpIsAntiEquivalence} gives that the 
%  sequential limit of Stone space is Stone, hence $S$ is Stone. 
\end{proof}

%\begin{remark}
 % For all $S:\Stone$, we shall denote by $S_n$ any sequence of finite sets which limit is $S$. 
 % Whenever $n\leq m$, we denote $\pi_m^n$ for the maps $S_m \to S_n$, 
 % and $\pi_n:S\to S_n$. 
%\end{remark}
\begin{corollary}
Stone spaces are stable under finite limits.
\end{corollary}
\begin{remark}\label{StoneClosedUnderPullback}\label{ProFiniteMapsFactorization}
  %Dually to \Cref{ODiscFiniteColim} and \Cref{ODiscClosedUnderSequentialColimits}, 
  %Stone spaces are closed under $\Sigma$-types and sequential limits.
%  Dually to \Cref{lemDecompositionOfColimitMorphisms} 
%  maps of Stone spaces are sequential limits of maps of finite sets. 
  By \Cref{decompositionBooleMaps} and 
  \Cref{SurjectionsAreFormalSurjections}, maps (resp. surjections, injections) of Stone spaces
  are sequential limits of maps (resp. surjections, injections) of finite sets. 
%
%
%
%
%  when we have a map of Stone spaces $f:S\to T$, 
%  we have $(\N,\geq)$-indexed sequences of finite sets $S_n,T_n$ with limits $S$ and $T$ respectively
%  and maps $f_n:S_n\to T_n$ inducing $f$. Moreover if $f$ is surjective or injective, we 
%  can choose all $f_n$ to be surjective or injective respectively as well. 
\end{remark}

\begin{lemma}\label{ScottFiniteCodomain}
  For $(S_n)_{n:\N}$ a sequence of finite types with $S=\lim_nS_n$ and $k:\N$, we have that $\Fin(k)^{S}$ is the sequential colimit of $\Fin(k)^{S_n}$.
\end{lemma}
\begin{proof}
  By \Cref{SpIsAntiEquivalence} we have $\Fin(k)^S = \Hom(2^{k},2^S)$.
  Since $2^{k}$ is finite, we have that $\Hom(2^k,\_)$ commutes with sequential colimits, therefore $\Hom(2^{k},2^S)$ is the sequential colimit of $\Hom(2^{k},2^{S_n})$. 
  By applying \Cref{SpIsAntiEquivalence} again, %these types are 
  the latter type is $\Fin(k)^{S_n}$.% as required. 
\end{proof}

\begin{lemma}\label{MapsStoneToNareBounded}
  For $S:\Stone$ and $f:S \to \N$, there exists some $k:\N$ such that $f$ factors through $\Fin(k)$. 
\end{lemma}
\begin{proof}
  For each $n:\N$, the fiber of $f$ over $n$ is a decidable subset $f_n:S \to 2$. 
  We must have that $\Sp(2^S/(f_n)_{n:\N}) = \bot$, hence there exists some $k:\N$ with 
  $\bigvee_{n\leq k} f_n =_{2^S} 1 $. 
  It follows that $f(s)\leq k$ for all $s:S$ as required. 
\end{proof}

%\begin{lemma}\label{scott-continuity}
%  Let $E:\ODisc$ and $S:\Stone$, then 
%  Then $E^S$ is the colimit of the $(\N,\leq)$-indexed sequence $E^{S_n}$.
%%  $\mathrm{colim}_k(Z^{S_k}) \to \Z^S$
%%  is an equivalence.
%\end{lemma}
%\begin{proof}
%  Let $f:S \to E$. By \Cref{OdiscQuotientCountableByOpen}, 
%  we have an enumeration $\N\twoheadrightarrow 1 + E$. 
%  By \Cref{MapsStoneToNareBounded} and \Cref{AxLocalChoice}, there is some $N:\N$ such that 
%  $f(S)\subseteq e(\N_{\leq N})$. 
%\end{proof} 
\begin{corollary}\label{scott-continuity}
  For $(S_n)_{n:\N}$ a sequence of finite types with $S=\lim_nS_n$, we have that $\N^S$ is the sequential colimit of $\N^{S_n}$. 
\end{corollary}
\begin{proof}
  By \Cref{MapsStoneToNareBounded} we have that $\N^S$ is the sequential colimit of $\Fin(k)^S$. 
  By \Cref{ScottFiniteCodomain}, $\Fin(k)^S$ is the sequential colimit of the $\Fin(k)^{S_n}$ and we can swap the sequential colimits to conclude.
  \end{proof} 



\subsection{$\Closed$ and $\Stone$}
%\begin{lemma}\label{BooleEqualityOpen}
%  Whenever $B:\Boole$, $a,b:B$ the proposition $a=_Bb$ is open. 
%\end{lemma}
%\begin{proof}
%  Let $G,R$ be the generators and relations of $B$. 
%  Let $a,b$ be represented by $x,y$ in the free Boolean algebra on $G$. 
%  Now let $R_n$ denote the first $n$ elements of $R$. 
%  Note that $a=b$ iff there exists some $n:\N$ with $x-y \leq \bigvee_{r\in R_n} r$. 
%  Furthermore, inequality is decidable in the free Boolean algebra, hence
%  $a=b$ is a countable disjunction of decidable propositions, hence open. 
%\end{proof}

\begin{corollary}\label{TruncationStoneClosed}
  For all $S:\Stone$, the proposition $\propTrunc{S}$ is closed. 
\end{corollary}
\begin{proof}
  By \Cref{SpectrumEmptyIff01Equal}, $\neg S$ is equivalent to $0=_{2^S} 1$, which is open by \Cref{BooleIsODisc} and \Cref{OdiscQuotientCountableByOpen}. 
  Hence $\neg \neg S$ is a closed proposition, and by \Cref{LemSurjectionsFormalToCompleteness}, so is $\propTrunc{S}$. 
\end{proof}
%\begin{remark}\label{ExplicitTruncationStoneClosed}
%  \rednote{New check later}
%  The above lemma and corollary actually show that if we have an explicit 
%  presentation of a Stone space as $S = \Sp(2[G] / R)$, 
%  we can construct an explicit sequence $\alpha:2^\N$ such that $||S|| \leftrightarrow \forall_{n:\N} \alpha(n) = 0$. 
%\end{remark}


\begin{corollary}\label{PropositionsClosedIffStone}
  A proposition $P$ is closed if and only if it is a Stone space. 
\end{corollary}
\begin{proof}
  By the above, if $S$ is both a Stone space and a proposition, it is closed. 
  By \Cref{ClosedPropAsSpectrum}, any closed proposition is Stone. 
\end{proof}

\begin{lemma}\label{StoneEqualityClosed}
For all $S:\Stone$ and $s,t:S$, the proposition $s=t$ is closed. 
\end{lemma}
\begin{proof}
  Suppose $S= \Sp(B)$ and let $G$ be a countable set of generators for $B$. 
  Then $s=t$ if and only if $s(g) = t(g)$ for all $g:G$. 
  So $s=t$ is a countable conjunction of decidable propositions, hence 
  closed.
\end{proof}

\subsection{The topology on Stone spaces}
\begin{theorem}\label{StoneClosedSubsets}
  Let $A\subseteq S$ be a subset of a Stone space. TFAE:
  \begin{enumerate}[(i)]
    \item There exists a map $\alpha_{(\cdot)}:S \to 2^\N$ such that 
      $A (x) \leftrightarrow \forall_{n:\N} \alpha_x(n) = 0$ for any $x:S$. 
    \item There exists some countable family 
      $D_n,~{n:\N}$ 
      of decidable subsets of $S$ with $A = \bigcap_{n:\N} D_n$. 
    \item There exists a Stone space $T$ and some embedding $T\to S$ which image is $A$
    \item There exists a Stone space $T$ and some map $T\to S$ which image is $A$. 
    \item $A$ is closed.
  \end{enumerate}
\end{theorem}
\begin{proof}
\item 
  \begin{itemize}
  \item[$(i)\leftrightarrow (ii)$.] 
    $D_n$ and $\alpha_{(\cdot)}$ can be defined from each other by 
%    Define the decidable subsets of $S$ 
     $D_n(x) \leftrightarrow (\alpha_x(n) = 0)$. Then observe that %$A=\bigcap_{n:\N} D_n$ as 
     \begin{equation}
      (\bigcap_{n:\N} D_n) (x) \leftrightarrow 
      \forall_{n:\mathbb N} (\alpha_x(n) = 0) 
%      \leftrightarrow A s. 
     \end{equation}
   \item[$(ii) \to (iii)$.]
      Let $S=Sp(B)$. 
      By Stone duality, we have $d_n,~n:\N$ terms of $B$ such that $D_n = \{x:S| x(d_n) = 1\}$. 
      Let $C = B/(\neg d_n)_{n:\N}$.
      Then the map $Sp(C) \to S$ is as desired because
      $$Sp(C) = \{x:S| \forall_{n:\N} x(\neg d_n) =0\}  = \bigcap_{n:\N} D_n.$$
%      By \Cref{SurjectionsAreFormalSurjections}, t
%      The quotient map $B \twoheadrightarrow C$
%      corresponds to a map $\iota:Sp(C) \hookrightarrow  S$. 
%      For $s:S$, $s$ lies in the image of this map iff 
%      for all $n:\N$ we have  $s(\neg d_n) = 0$, 
%      \begin{equation}
%        x\in \iota(Sp(C)) \leftrightarrow x(\neg d_n) = 0 \leftrightarrow x(d_n) = 1 \leftrightarrow x\in D_n
%      \end{equation}
%      Thus the image of $\iota$ is given by $\bigcap_{n:\N} D_n$. 
   \item[$(iii) \to (iv)$] Immediate.
   \item[$(iv) \to (i)$.] 
     \rednote{TODO 
       The order of untracating is important in this proof, 
     and I struggle a bit with stressing this in a way this is clear (and concise). 
    Check with fresh eyes later. }
      Let $f:T\to S$ be a map between Stone spaces. 
      Assume $S = Sp(A), T = Sp (B)$. 
%      For this proof, we work with explicit presentations for $A,B$. 
%
      Let $G$ be a countable set of generators of $A$. 
      Assume also we have countable sets of generators and relations for $B$. 
%
      Following \Cref{FiberConstruction}, using $G$, for each $x:S$, we can construct 
      a countable set $I_x\subseteq B$ such that $$Sp(B/I_x) = (\Sigma_{y:T} f(y) = x) .$$
      By \Cref{ExplicitTruncationStoneClosed}, we can construct a sequence 
      $\alpha_x$ such that this type is inhabited iff $\forall_{n:\N} \alpha_x(n) = 0$,
      as required. 
%
%      Recall that the propositional truncation of a Stone is closed, as it is the negation of $0=1$ in the underlying 
%      Boolean algebra, which is open as it f
%
%
%      the core idea of the proof was that the closed proposition corresponds to checking equality in the underlying BA, 
%      which was closed as 
%
%
%
%
%      Note that $x$ in the image of $f$ iff $0\neq_{B/I_x} 1$. 
%      At this point, we have generators and relations of $B/I_x$ as data.
%      Hence using the proof of \Cref{BooleEqualityOpen}, we can construct a sequence 
%      $\alpha_x:2^\N$ such that $0 =_{B/I_x}1\leftrightarrow \exists_{n:\N} \alpha_x(n) = 0$. 
%      And for $\beta_x(n) = 1-\alpha_x(n)$, we conclude that 
%      \begin{equation}
%        x\in f(T) \leftrightarrow \forall_{n:\N} \beta_x(n) = 0
%      \end{equation}
%      Note that we did not use any choice axioms in the proof of this implication,
%      as we untruncated our assumptions before we specified $x$. 
   \item [$(i) \to (v)$.] By definition.
   \item[$(v) \to (iv)$.]
     As $A$ is closed, it induces a map $a:S\to \Closed$. 
     We can cover the closed propositions with Cantor space
     by sending 
     $\alpha \mapsto \forall_{n:\mathbb N} \alpha n = 0.$
     Now local choice gives us that there merely exists $T, e, \beta_\cdot$ as follows:
     \begin{equation}
       \begin{tikzcd}
         T \arrow[r,"\beta_\cdot"] \arrow[d, two heads,"e"] & 2^\mathbb N 
         \arrow[d,two heads, "\forall_{n:\mathbb N} (\cdot)n = 0"] \\
         S \arrow[r,"a"] & \Closed
       \end{tikzcd} 
     \end{equation} 
     Define $B(x) \leftrightarrow \forall_{n:\mathbb N} \beta_x(n) = 0$. 
     As $(i) \to (iii)$ by the above, $B$ is the image of some Stone space. 
     Furthermore, note that $A$ is the image of $B$, thus $A$ is the image of some Stone space. 
\end{itemize} 
\end{proof} 
\rednote{Ordering from here on out is WIP}

\begin{remark}\label{ClosedInStoneIsStone}
Using condition $(iii)$, the previous result implies that closed subtype of Stone spaces are Stone.
\end{remark}

\begin{corollary}\label{InhabitedClosedSubSpaceClosed}
  For $S:\Stone$ and $A\subseteq S$ closed, we have 
  $\exists_{x:S} A(x)$ is closed. 
\end{corollary}
\begin{proof}
  By \Cref{ClosedInStoneIsStone} we have that $\Sigma_{x:S}A(x)$ is Stone, so its truncation is closed by \Cref{TruncationStoneClosed}.
\end{proof}

\begin{corollary}\label{ClosedDependentSums}
  Closed propositions are closed under dependent sums. 
\end{corollary}
\begin{proof}
  Let $P:\Closed$ and $Q:P \to \Closed$. 
  Then $\Sigma_{p:P} Q(p) \leftrightarrow \exists_{p:P} Q(p)$.
  As $P$ is Stone by \Cref{PropositionsClosedIffStone}, 
%  As $P$ is Stone by \Cref{PropositionsClosedIffStone}, it is also compact Hausdorff, thus
  \Cref{InhabitedClosedSubSpaceClosed} gives that $\Sigma_{p:P} Q(p)$ is closed. 
\end{proof}
\begin{remark}
  Analogously to \Cref{OpenTransitive} and \Cref{OpenDominance}, it follows that 
  closedness is transitive and $\Closed$ forms a dominance. 
\end{remark}

We can get a dual to completeness.

\begin{lemma}\label{DualCompleteness}
Let $A$ and $B$ be c.p. boolean algebra with a map:
\[i:Sp(A)\to Sp(B)\] 
The following are equivalent:
\begin{enumerate}[(i)]
\item The induced map $B\to A$ is surjective.
\item The map $i$ is an embedding.
\item The map $i$ is a closed embedding.
\end{enumerate}
\end{lemma}

\begin{proof}
\begin{itemize}
\item[$(i)\to (ii)$] Immediate.
\item[$(ii)\to (iii)$] By \Cref{StoneEqualityClosed} the fibers of $i$ are closed in $Sp(A)$, so by \Cref{ClosedInStoneIsStone} they are Stone, so they are closed by \Cref{PropositionsClosedIffStone}.
\item[$(iii)\to (i)$] By we have that $Sp(A) = \cap_{n:\N}D_n$ for $D_n$ decidable in $Sp(B)$. Assuming $D_n$ correponds to $b_n:B$ though duality, we then have that $A=B/ (b_n)_{n:\N}$ and the induced map is the quotient map:
$$B\to B/(b_n)_{n:\N}$$
which is surjective.
\end{itemize}
\end{proof}

\begin{lemma}\label{StoneSeperated}
  If $S:\Stone $, and $F,G:S \to \Closed$ be such that $F\cap G = \emptyset$. 
  Then there exists a decidable subset $D:S \to 2$ such $F\subseteq D, G \subseteq \neg D$. 
\end{lemma}
\begin{proof}
  Assume $S = Sp(B)$. 
  By the above theorem, there exists sequences $f_n,g_n:B,~n:\N$ such that 
  $x\in F$ iff $x(f_n) = 1$ for all $n:\N$ and 
  $y\in G$ iff $y(g_m) = 1$ for all $m:\N$. 
%
  Denote $R\subseteq B$ for $\{\neg f_n|n:\N\} \cup\{\neg g_n|n:\N\}$. 
  Note that any inhabitant $Sp(B/R)$ gives a map $x:B\to 2$ such that
  $x(g_n)= x(f_n) = 1$ for all $n:\N$, hence $x\in F \cap G$. 
  As $F\cap G = \emptyset $, it follows that $Sp(B/R)$ is empty.
%
  Thus there exists finite sets $I,J\subseteq \N $ such that 
  $$1 =_B ((\bigvee_{i\in I} \neg f_i) \vee (\bigvee_{j\in J} \neg g_j)).$$
%
  Let $y\in G$. Then $y(\neg g_j) = 0$ for all $j \in J$. 
  Hence 
  $$
  1 = y(1) 
  = 
  y(\bigvee_{i\in I} \neg f_i) = y (\neg (\bigwedge_{i\in I} f_i))
  $$
  Thus $y(\bigwedge_{i\in I} f_i) = 0$. 
  Note that if $x\in F$, we have $x(f_i) = 1$ for all $i\in I$, hence 
  $x(\bigwedge_{i\in I} f_i) = 1$. 
  Thus for $D$ corresponding to $\bigwedge_{i\in I} f_i$, we have that 
  $F\subseteq D, G\subseteq \neg D$ as required. 
\end{proof} 

\begin{corollary}\label{StoneOpenSubsets}
  Let $A\subseteq S$ be a subset of a Stone space, then 
  $A$ is open iff there exists some countable family $D_n,~n:\N$ of decidable subsets of $S$ with 
  $A = \bigcup_{n:\N} D_n$. 
\end{corollary}
\begin{proof}
  By \Cref{rmkOpenClosedNegation}, $A$ is open iff $\neg A$ is closed and $A = \neg \neg A$. 
  By \Cref{StoneClosedSubsets}, $\neg A$ is closed iff 
  $\neg A = \bigcap_{n\in \N} E_n$ for some countably family $E_n,~n:\N$. 
  Thus $\neg \neg A = \neg (\bigcap_{n\in \N} E_n)$. 
  As a concequence of Markov's principle (\Cref{MarkovPrinciple}), we have that 
  $\neg (\bigcap_{n\in \N} E_n)= \bigcup_{n\in \N} \neg E_n$. 
  Thus $D_n := \neg E_n$ is as required. 
\end{proof}


\section{Compact Hausdorff spaces}
\begin{definition}
  A type $X$ is called a compact Hausdorff space if its identity types are closed propositions and there exists some $S:\Stone$ with a surjection $S\twoheadrightarrow X$. We write $\CHaus$ for the type of compact Hausdorff spaces.
\end{definition}

%This means that compact Hausdorff spaces are precisely quotients of Stone spaces by closed equivalence relations.

\subsection{Topology on compact Hausdorff spaces}

\begin{lemma}\label{CompactHausdorffClosed}
  Let $X:\CHaus$ with $S:\Stone$ and a surjective map $q:S\twoheadrightarrow X$.
  Then $A\subseteq X$ is closed if and only if it is the image of a closed subset of $S$ by $q$. 
\end{lemma}
\begin{proof}
%  If $A$ is closed, then it's pre-image under any map is also closed. 
%  In particular for $q:S\to X$ the quotient map, $q^{-1}(A)$ is closed. 
  As $q$ is surjective, we have $q(q^{-1}(A)) = A$.
  If $A$ is closed, so is $q^{-1}(A)$ and 
  hence $A$ is the image of a closed subset of $S$. 
  Conversely, let $B\subseteq S$ be closed. Then $x\in q(B)$ if and only if
   \[\exists_{t:S} (B(t) \wedge q(s) = x).\]
   Hence by \Cref{InhabitedClosedSubSpaceClosed}, $q(B)$ is closed. 
  % Define $A'\subseteq S$ by 
  %\[A'(s) = \exists_{t:S} (B(t) \wedge q(s) = q(t)).\]
  %Note that $B(t)$ and $q(s) = q(t)$ are closed. 
  %Hence by \Cref{InhabitedClosedSubSpaceClosed}, $A'$ is closed. 
  %Also $A'$ factors through $q$ as a map $A: X\to \Closed$.
  %Furthermore, $A'(s) \leftrightarrow (q(s)\in q(B))$. 
  %Hence $A=q(B)$. 
\end{proof}

The next two corollaries mean that compact Hausdorff spaces behave as finite sets for the purposes of unions/intersections of open/closed sets.

\begin{corollary}\label{InhabitedClosedSubSpaceClosedCHaus}
Assume given $X:\Chaus$ with $A\subseteq X$ closed. Then $\exists_{x:X} A(x)$ is closed, and equivalent to $A \neq \emptyset$. 
\end{corollary}

\begin{proof}
From \Cref{CompactHausdorffClosed} and \Cref{StoneClosedSubsets}, it follows that $A\subseteq X$ is closed if and only if it is the image of a map $T\to X$ for some $T:\Stone$. Then $\exists_{x:X} A(x)$ if and only $\propTrunc{T}$, which is closed by \Cref{TruncationStoneClosed}. Therefore $\exists_{x:X} A(x)$ is $\neg\neg$-stable and equivalent to $A \neq \emptyset$. 
  %If $A$ is closed, it follows from \Cref{InhabitedClosedSubSpaceClosed} that $\exists_{x:X} A(x)$ is closed as well, 
 % hence $\neg\neg$-stable, and equivalent to $A \neq \emptyset$. 
\end{proof}

%\begin{remark}\label{InhabitedClosedSubSpaceClosedCHaus}
%  Let $X:\Chaus$.
%  From \Cref{StoneClosedSubsets}, it follows that $A\subseteq X$ is closed if and only if it is the image of a map 
%  $T\to X$ for some $T:\Stone$. 
%  If $A$ is closed, it follows from \Cref{InhabitedClosedSubSpaceClosed} that $\exists_{x:X} A(x)$ is closed as well, 
%  hence $\neg\neg$-stable, and equivalent to $A \neq \emptyset$. 
%\end{remark}
%\begin{corollary}
%  For $X:\CHaus$ a subtype $A\subseteq X$ is closed iff it is the image of 
%  a map $T\to X$ for some $T:\Stone$. 
%\end{corollary}
%\begin{proof}
%  Directly from the above and \Cref{StoneClosedSubsets}.
%\end{proof}
%WhyDidWeNeedThis%\begin{remark}
%WhyDidWeNeedThis%  It is not the case that every closed subset of a compact Hausdorff space can be written 
%WhyDidWeNeedThis%  as countable intersection of decidable subsets. 
%WhyDidWeNeedThis%  In \Cref{UnitInterval}, we shall introduce the unit interval $[0,1]$ as a compact Hausdorff space with many closed 
%WhyDidWeNeedThis%  subsets, but only two decidable subsets. 
%WhyDidWeNeedThis%  In \Cref{ConnectedComponent}, we shall actually see that whenever every singleton of a compact Hausdorff space $X$
%WhyDidWeNeedThis%  can be written as countable intersection of decidable subsets, $X$ is Stone. 
%WhyDidWeNeedThis%  \rednote{Actually, we'll see that $\Sp(2^X)$ and $X$ are bijective sets, 
%WhyDidWeNeedThis%    which only implies that $X$ is Stone if $2^X:\Boole$, but this depends on our definition of countable, 
%WhyDidWeNeedThis%see \Cref{CountabilityDiscussion}}
%WhyDidWeNeedThis%\end{remark}


\begin{corollary}\label{AllOpenSubspaceOpen}
  Assume given $X:\Chaus$ with $U\subseteq X$ open. Then $\forall_{x:X} U(x)$ is open. 
\end{corollary}
%\begin{proof}
%  As $U$ is open, $\neg U$ is closed. 
%  So $\exists_{x:X} \neg U(x)$ is closed by \Cref{InhabitedClosedSubSpaceClosedCHaus}. 
%  Using \Cref{rmkOpenClosedNegation}, it follows that 
%  $\neg (\exists_{x:X} \neg U(x))$ is open. 
%  Furthermore, it is equivalent to $\forall_{x:X} \neg \neg U(x)$, 
%  which is equivalent to $\forall_{x:X} U(x)$ by \Cref{rmkOpenClosedNegation}.
%\end{proof}

Next lemma means that compact Hausdorff space are not too far from being compact in the classical sense.

\begin{lemma}\label{CHausFiniteIntersectionProperty}
  Given $X:\Chaus$ and $C_n:X\to \Closed$ closed subsets such that $\bigcap_{n:\N} C_n =\emptyset$, there is some $k:\N$ 
  with $\bigcap_{n\leq k} C_n  = \emptyset$. 
\end{lemma}
\begin{proof}
  By \Cref{CompactHausdorffClosed} it is enough to prove the result when $X$ is Stone, and by \Cref{StoneClosedSubsets} we can assume $C_n$ decidable.
  So assume 
  $X=\Sp(B)$ and $c_n:B$ such that
  \[C_n = \{x:B\to 2\ |\ x(c_n) = 0\}.\]
  Then we have that
  \[\Sp(B/(c_n)_{n:\N})%\ |\ n:\N\}) 
  \simeq \bigcap_{n:\N} C_n = \emptyset .\]
  Hence 
%  $0=_{B/(\neg c_n)_{n:\N}}1$ 
  $0=1$ in $B/(c_n)_{n:\N}$ %\ |\ n:\N\}$, 
  and there is some $k:\N$ with 
  $\bigvee_{n\leq k} c_n = 1$, which means that
  \[\emptyset = \Sp(B/(c_n)_{n\leq k}) %\ |\ n\leq k\})  
  \simeq \bigcap_{n\leq k} C_n \]
  as required.
\end{proof}

\begin{corollary}\label{ChausMapsPreserveIntersectionOfClosed}
  Let $X,Y:\CHaus$ and $f:X \to Y$. 
  Suppose $(G_n)_{n:\N}$ is a decreasing sequence of closed subsets of $X$. 
  Then $f(\bigcap_{n:\N} G_n) = \bigcap_{n:\N}f(G_n)$. 
\end{corollary}
\begin{proof}
  It is always the case that $f(\bigcap_{n:\N} G_n) \subseteq \bigcap_{n:\N} f(G_n)$. 
  For the converse direction, suppose that $y \in f(G_n)$ for all $n:\N$. 
  We define $F\subseteq X$ closed by $F=f^{-1}(y)$. 
  Then for all $n:\N$ we have that $F\cap G_n$ is %merely inhabited and therefore 
  non-empty. 
  By \Cref{CHausFiniteIntersectionProperty} this implies that $\bigcap_{n:\N}(F\cap G_n) \neq \emptyset$. 
  By \Cref{InhabitedClosedSubSpaceClosedCHaus},  we have that %and \Cref{rmkOpenClosedNegation}, 
  $\bigcap_{n:\N} (F\cap G_n)$ is merely inhabited. Thus $y\in f(\bigcap_{n:\N} G_n)$ as required. 
\end{proof}

\begin{corollary}\label{CompactHausdorffTopology}
Let $A\subseteq X$ be a subset of a compact Hausdorff space and $p:S\twoheadrightarrow X$ be a surjective map with $S:\Stone$. Then $A$ is closed (resp. open) if and only if there exists a sequence $(D_n)_{n:\N}$ of decidable subsets of $S$ such that $A = \bigcap_{n:\N} p(D_n)$ (resp. $A = \bigcup_{n:\N} \neg p(D_n)$).
%\begin{itemize}
%  \item $A$ is closed iff %if and only if 
%    it can be written as $\bigcap_{n:\N} p(D_n)$
%for some $D_n\subseteq S$ decidable. 
%  \item $A$ is open iff %if and only if 
%    it can be written as $\bigcup_{n:\N} \neg p(D_n)$
%for some $D_n\subseteq S$ decidable.
%\end{itemize}  
\end{corollary}
\begin{proof}
  The characterization of closed subsets follows from characterization (ii) in \Cref{StoneClosedSubsets}, 
  \Cref{CompactHausdorffClosed} 
  and \Cref{ChausMapsPreserveIntersectionOfClosed}. 
%  The characterization of open sets 
  To deduce the characterization of open subsets we use \Cref{rmkOpenClosedNegation} and
  \Cref{ClosedMarkov}.
\end{proof}
%
\begin{remark}
  For $S:\Stone$, there is a surjection $\N\twoheadrightarrow 2^S$. 
  It follows that for any $X:\CHaus$ there is a surjection from $\N$ to a basis of $X$. 
  Classically this means that $X$ is second countable. 
\end{remark}
%It follows that compact Hausdorff spaces are second countable:
%\begin{corollary}
%  Any $X:\Chaus$ is has a topological basis which is countable.
%\end{corollary}
%\begin{proof}
%  By \Cref{CompactHausdorffTopology}, 
%  a basis is given by the images of the decidable subsets of some $S:\Stone$. 
%  By \cref{ODiscBAareBoole}, $2^S$ is 
%  overtly discrete so we have a surjection $\N\to 2^S$.
%  \end{proof}
%

Next lemma means that compact Hausdorff spaces are normal.

\begin{lemma}\label{CHausSeperationOfClosedByOpens}
 Assume $X:\CHaus$ and $A,B\subseteq X$ closed such that $A\cap B=\emptyset$. 
  Then there exist $U,V\subseteq X$ open such that $A\subseteq U$, $B\subseteq V$ and $U\cap V=\emptyset$. 
\end{lemma}
\begin{proof}
  Let $q:S\to X$ be a surjective map with $S:\Stone$.
  As $q^{-1}(A)$ and $q^{-1}(B)$ are closed, 
  by \Cref{StoneSeperated}, there is some $D:S \to 2$ such that
  $q^{-1}(A) \subseteq D$ and $q^{-1}(B) \subseteq \neg D$. 
  Note that $q(D)$ and $q(\neg D)$ are closed by \Cref{CompactHausdorffClosed}. 
%  We define $U = \neg q(\neg D) $%\cap \neg B$ 
%  and $V=\neg  q(D) $.%\cap \neg A$. 
  As $q^{-1}(A) \cap \neg D  =\emptyset$, we have that 
  $A\subseteq \neg q(\neg D):=U$. 
%  As $A\cap B = \emptyset$, we have that $A\subseteq \neg B$ so $A\subseteq U$.
%  Similarly $B\subseteq V$. 
  Similarly $B\subseteq \neg q(D):=V$. 
  Then $U$ and $V$ are disjoint because $\neg q(D)\cap \neg q(\neg D) = \neg (q(D)\cup q(\neg D)) = \neg X = \emptyset$.
\end{proof}


\subsection{Compact Hausdorff spaces are stable under dependent sums}

\begin{lemma}
A type $X$ is Stone iff it is merely a closed in $2^\N$.
\end{lemma}

\begin{proof}
Any countably presented boolean algebra $B$ is enumerable, which gives a surjective morphism:
$$ 2[\N]\to B$$
so that by \Cref{DualCompleteness} we merely have a closed embedding:
$$ Sp(B)\to 2^\N$$
\end{proof}

\begin{lemma}\label{SigmaStoneCompactHausdorff}
Assume given $S:\Stone$ and $T:X\to \Stone$. Then $\Sigma_{x:S}T(x)$ is Compact Hausdorff.
\end{lemma}

\begin{proof}
By \Cref{ClosedDependentSums} we have that identity type in $\Sigma_{x:S}T(x)$ are closed.

We know that for any $x:S$ we have that $\exists_{C:2^\N\to \Closed} T_x = \Sigma_{y:2^\N}C(y)$. Using local choice we get $S':\Stone$ with a surjective map:
$$q:S'\to S$$
and:
$$ C : S'\to 2^\N\to\Closed$$
such that for all $x:S'$ we have $T(q(x)) = \Sigma_{y:2^\N}C(x,y)$. This gives a surjective map:
$$ \Sigma_{x:S'}\Sigma_{y:2^\N} C(x,y)\to \Sigma_{x:S}T(x)$$
The source is Stone by \Cref{StoneClosedUnderPullback} and \Cref{ClosedInStoneIsStone} so we can conclude.
\end{proof}

\begin{lemma}
Assume given $C:\CHaus$ and $D:X\to \CHaus$. Then $\Sigma_{x:C}D(x)$ is Compact Hausdorff.
\end{lemma}

\begin{proof}
By \Cref{ClosedDependentSums} we have that identity type in $\Sigma_{x:C}D(x)$ are closed.

We know that for any $x:C$ we have that $\exists_{T:\Stone} T\twoheadrightarrow C(x)$. Using a surjective map:
$$ S\to C$$
with $S:\Stone$ and local choice we get $S':\Stone$ with a surjective map:
$$q:S'\to C$$
such that for all $x:S'$ we have $T(x):\Stone$ and a surjective map $T(x)\to D(q(x))$. This gives a surjective map:
$$ \Sigma_{x:S'}T(x)\to \Sigma_{x:C}D(x)$$
By \Cref{SigmaStoneCompactHausdorff} we have a surjective map from a Stone space to the source so we can conclude.
\end{proof}
\subsection{Stone spaces are stable under dependent sums}

We will show that Stone spaces are precisely totally disconnected compact Hausdorff spaces. We will use this to prove that a dependent sum of Stone space is Stone.

\begin{lemma}\label{OpenInNAreDecidableInN}
For any open $U$ in $\N$, there merely exists a decidable $D$ in $\N$ such that $D=U$.
\end{lemma}

\begin{proof}
For any open proposition $U(n)$ we know that there merely exists $\alpha:\N_\infty$ such that:
$$ U(n) = \Sigma_{k:\N}\alpha_n(k)=0$$
Using countable choice we have that:
$$ U = \Sigma_{n,k:\N}\alpha_n(k)=0$$
and we conclude using $\N=\N\times\N$.
\end{proof}

\begin{lemma}\label{AlgebraCompactHausdorffCountablyPresented}
Assume $X$ compact Hausdorff, then $2^X$ is countably presented.
\end{lemma}

\begin{proof}
First we choose $S\to X$ surjective with $S$ Stone and prove that $2^X$ is an open subalgebra of $2^S$.

 The map $S\to X$ induces an injection of Boolean algebras $2^X \hookrightarrow 2^S$.
  Note that $a:S\to 2$ lies in $2^X$ iff for all $s,t:S$, we have $a(s) = a(t)$ whenever $s\sim t$.
  Note that $a(s) = a(t)$ is decidable and $s\sim t$ is closed, hence 
  $(s\sim t) \to (a(s) = a(t))$ is open (\Cref{ImplicationOpenClosed})
  By \Cref{AllOpenSubspaceOpen}, we conclude that 
  $\forall_{s:S} \forall_{t:S} ((s\sim t) \to (a(s) = a(t)))$ is open. 
  Hence $2^X$ is an open subalgebra of $2^S$. 

Now we prove that open subalgebras of countably presented agebras are countably presented. Assume $U\subset 2[\N] / F$ such a subalgebra. We have that $U$ is equivalent to the algebra generated by the $s:2[\N]$ such that $[s]\in U$ quotiented by the relation $s=t$ for all $s,t:2[\N]$ such that $[s],[t]\in U$ and $[s]=[t]$.

Using that $2[\N]$ is countable and that $[s]=[t]$ is open by \Cref{BooleEqualityOpen}, we see that $U$ is generated by variables and relations each indexed by an open in $\N$. But by \Cref{OpenInNAreDecidableInN} any open in $\N$ is countable, so $U$ is countably presented.
\end{proof}

\begin{lemma}\label{ConnectedComponentClosedInCompactHausdorff}
For all $X:\CHaus$ with $x:X$, then we have that $Q_x$ is a countable intersection of decidable in $X$.
\end{lemma}

\begin{proof}
By \Cref{AlgebraCompactHausdorffCountablyPresented} we have that $2^X$ is countably presented, therefore we can enumerate the elements of $2^X$, say as $(D_n)_{n:\N}$. if we define $E_n$ for $n:\N$ as $D_n$ if $x\in D_n$ and $X$ otherwise, we have that:
$$\cap_{n:\N}E_n = Q_x$$
\end{proof}

\begin{definition}
  Let $X:\Chaus$ and $x:X$. 
  We define the connected component of $x$ (denoted $Q_x$)
  as the intersection of all decidable subsets of $X$ containing $x$. 
\end{definition}

\begin{lemma}\label{ConnectedComponentSubOpenHasDecidableInbetween}
  Let $X:\Chaus, x:X$ and suppose $U\subseteq X$ is open with $Q_x\subseteq U$. 
  Then we have some decidable $E\subseteq X$ with $E(x)$ and $E\subseteq U$. 
\end{lemma}
\begin{proof}
  By \Cref{ConnectedComponentClosedInCompactHausdorff}, we have $Q_x = \bigcap_{n:\N}B_n$ with $B_n$ decidable. 
  If $Q_x \subseteq U$, we have that 
  $$Q_x\cap \neg U = \bigcap_{n:\N} (B_n \cap \neg U)$$ is empty. 
  By \Cref{CHausFiniteIntersectionProperty} there is some $N:\N$ with 
  $$(\bigcap_{n\leq N} B_n )\cap \neg U  = \bigcap_{n\leq N} (B_n \cap \neg U) = \emptyset.$$
  Therefore $\bigcap_{n\leq N} B_n \subseteq U$, furthermore a finite intersection of decidable subsets is decidable. 
  As $x\in B_n$ for all $n:\N$, $x\in \bigcap_{n\leq N} B_n$ as well and we're done. 
\end{proof}

\begin{lemma}\label{ConnectedComponentConnected}
Let $X$ be Compact Hausdorff with $x:X$. Then $Q_x$ is connected.
\end{lemma}
\begin{proof}
Assume given a separation $Q_x = A\cap B$ with $A,B$ disjoint and decidable in $Q_x$, and let us assume that $x\in A$. We want to prove that $B=\empty$. 

Since $Q_x$ is closed in $X$ by \Cref{ConnectedComponentClosedInCompactHausdorff}, we have that $A$ and $B$ are closed disjoint in $X$, so that by \Cref{CHausSeperationOfClosedByOpens} we have $U,V$ disjoint open such that $A\subset U$ and $B\subset V$. 

By \Cref{ConnectedComponentSubOpenHasDecidableInbetween} we have a decidable $E$ such that $Q_x\subset E\subset U\cup V$. Then we define $F = E\cap U$. We have that $F$ is open, it is closed as $F=E\cap \neg V$, therefore it is decidable by \Cref{ClopenDecidable}.

Then $Q_x\subset F$ with $F$ decidable and $B\cap F = \empty$ so that $Q_x\cap B = \empty$ and $B=\empty$.
\end{proof}

\begin{lemma}\label{StoneCompactHausdorffTotallyDisconnected}
Let $X:\CHaus$, then $X$ is Stone iff $\forall_{x:X}Q_x=\{x\}$.
\end{lemma}

\begin{proof}
By duality, it is clear that for all $x:S$ with $S$ Stone we have that $Q_x=\{x\}$.

For the converse, we show that the map:
\[X\to Sp(2^X)\]
is an equivalence and conclude by \Cref{AlgebraCompactHausdorffCountablyPresented}. 

Surjectivity always hold, indeed considering $q:S\to X$ surjective with $S$ Stone, we have that $2^X\subset 2^S$ as so that the by \Cref{FormalSurjectionsAreSurjections} the map:
$$S = Sp(2^S)\to Sp(2^X)$$
is surjective and it factors though $X$.

Now let us prove injectivity. Assume $x,y:X$ having the same image in $Sp(2^X)$. This means that any map in $X\to 2$ has the same value on $x$ and $y$, so $x\in Q_y$ and by hypothesis $x=y$.
\end{proof}

\begin{theorem}
Assume given $S:\Stone$ with $T:S\to\Stone$. Then $\Sigma_{x:S}T(x)$ is Stone.
\end{theorem}

\begin{proof}
By \Cref{SigmaStoneCompactHausdorff} we have that $\Sigma_{x:S}T(x)$ is compact Hausdorff. By \Cref{StoneCompactHausdorffTotallyDisconnected} it is enough to show that for all $x:S$ and $y:T(x)$ we have that $Q_{(x,y)}$ is a singleton.

Assume $(x',y')\in Q_{(x,y)}$, then for any map $q:S\to 2$ we have that:
$$ q(x) = q\circ \pi_1(x,y) = q\circ \pi_1(x',y') = q(x')$$
so that $x'\in Q_x$ and since $S$ is Stone by \Cref{StoneCompactHausdorffTotallyDisconnected} we have that $x=x'$.

Therefore we have $Q_{(x,y)}\subset \{x\}\times T_x$ so that by \Cref{ConnectedComponentConnected} we have an inhabited  connected subtype of a Stone space. Then any map $T_x\to 2$ is constant on $Q_{(x,y)}$ and by \Cref{StoneCompactHausdorffTotallyDisconnected} we conclude that it is a singleton.
\end{proof}




%\section{The Unit interval}
\subsection{The unit interval as a compact Hausdorff space}
Since we have dependent choice, the unit interval $\mathbb I = [0,1]$ can be defined using 
Cauchy reals or Dedekind reals. 
We can freely use results from constructive analysis \cite{Bishop}. 
As we have $\neg$WLPO, MP and LLPO, we can use the results from 
constructive reverse mathematics that follow from these principles \cite{ReverseMathsBishop, HannesDiener}. 
%\begin{definition}
%  We define $cs:2^\N \to \I$ as 
%  $cs(\alpha) = \sum_{n:\N} \frac{\alpha(i)}{2^{i+1}}$. 
%\end{definition}
\begin{definition}
  \label{def-cs-Interval}
  We define for each $n:\N$ the Stone space $2^n$ of binary sequences of length $n$.
  % = \Sp(2[\Fin(n)])$.
  And we define $cs_n:2^n \to \mathbb Q$ by 
  $cs_n(\alpha) = \sum_{i < n } \frac{\alpha(i)}{2^{i+1}}.$
  Finally we write $\sim_n$ for the binary relation on $2^n$ given by 
  $\alpha\sim_n \beta 
  \leftrightarrow \left|cs_n(\alpha) - cs_n(\beta)\right|\leq\frac{1}{2^n}$.
\end{definition}
\begin{remark}
  The inclusion $\mathrm{Fin}(n) \hookrightarrow \N$ induces a restriction 
  $\_|_n : 2^\N \to 2^n$ for each $n:\N$. 
\end{remark}
\begin{definition}
  We define $cs:2^\N \to \I$ as 
  $cs(\alpha) = 
  \sum_{i=0}^\infty \frac{\alpha(i)}{2^{i+1}}.
  $
%  \lim_{n\to\infty} cs_n(\alpha|_n)$. 
\end{definition}

\begin{theorem}\label{IntervalIsCHaus}
 The type $\I$ is a compact Hausdorff space.
\end{theorem}
\begin{proof}
  By LLPO, we have that $cs$ is surjective.   
  Note that $cs(\alpha) = cs(\beta)$ if and only if 
  for all $n:\N$ we have $\alpha|_n \sim_n \beta|_n$. 
  %$\left|cs_n(\alpha)-cs_n(\beta)\right|\leq \frac{1}{2^n}$
%  $$|\sum_{n=0}^{n-1} \frac{\alpha(i)}{2^{i+1}}-
%  \sum_{n=0}^{n-1} \frac{\beta(i)}{2^{i+1}}|\leq \frac{1}{2^n}$$
%  for all $n:\N$, 
  This is a countable conjunction of decidable propositions, so that equality in $\I$ is closed.
%  as inequality in $\mathbb Q$ is decidable. 
\end{proof}


%In this section we will introduce the unit interval $I$ as compact Hausdorff space. 
The definition is based on \cite{Bishop}. 
We will then calculate the cohomology of $I$. 
For a proof that the unit interval corresponds to the definition using Cauchy sequences, 
we refer to the appendix. 


\input{Interval/CauchySequences}
\input{Interval/BinaryClosedEquivalence}
\input{Interval/EquivalenceOfSims}
\input{Interval/Surjective}





\begin{theorem}
  The interval of Cauchy reals is isomorphic to $2^\N / \sim_t$. 
\end{theorem} 
\begin{proof}
  This follows from the fact that $b:2^\N$ is such that $\alpha\sim_n \beta$ iff $b(\alpha)\sim_t b(\beta)$. 
  and for every Cauchy real, there is a binary sequence being sent to it, so the composition of $b$ and the 
  quotient from Caucy sequences to Cauchy real is a surjection. 
\end{proof}

\begin{corollary}
  The interval is compact Hausdorff. 
\end{corollary}

%\subsection{Order on the interval}
%\begin{definition}
%  For $n:\N$ we define 
%  $cs_n:2^n \to \mathbb Q$ by 
%  \begin{equation}
%    cs_n(a) = \sum\limits_{i=0}^{n-1} \frac{a(i)} {2^{i+1}}
%  \end{equation}
%  And for $\alpha:2^\N$, we define the sequence $cs(\alpha) : \N \to \mathbb Q$ by 
%  \begin{equation}
%    cs(\alpha)_n = cs_n(\alpha|_n)
%  \end{equation}
%\end{definition}
%\begin{remark}\label{rmkPropertiesCSn}
%  $cs_n$ gives a bijection between $2^n$ and it's image 
%  $\{\frac{k}{2^n}|0\leq k \leq 2^{n}-1\}\subseteq \mathbb Q$.
%%  of rational numbers of the form  
%%  $\frac{k}{2^n}$ for $0\leq k \leq 2^n-1$. 
%  This observation has some corollaries: 
%  \begin{itemize}
%    \item In particular, each $cs_n$ is injective. 
%    \item Furthermore, whenever $a\neq b:2^n$, we must have that 
%      \begin{equation} 
%        |cs_n(a)-cs_n(b)|\geq \frac{1}{2^n}.
%      \end{equation}
%    \item It is known that $\bigcup_{n:\N} \{\frac{k}{2^n}|0\leq k \leq 2^{n}-1\}$ 
%      lies dense in the interval of Cauchy reals $[0,1]$. 
%      It follows that $cs$ induces a surjection from Cantor space to $[0,1]$. %the interval of Cauchy reals. 
%      We claim without proof it in fact induces an equivalence between $\I$ and $[0,1]$.
%%      between $I$ and the interval of Cauchy reals. 
%  \end{itemize}
%  Finally, let us repeat a well-known identity for all $m<n$ on such sums, which we'll make some use of 
%  \begin{equation}
%   \sum\limits_{i = m}^{n-1} \frac{1}{2^{i+1}} = \frac{1}{2^{m}} - \frac{1}{2^n}
%  \end{equation}
%\end{remark}
%\begin{lemma}\label{CauchyApproxLemma}
%  Let $n:\N$ and  $s,t:2^n$. Assume there is some $ m \leq n$ with $cs_m(s|_m) = cs_m(t|_m) + \frac{1}{2^m}$, and 
%  at the same time $cs_n(s) -cs_n(t)\leq \frac{1}{2^n}$. 
%  Then there is some $k< m$ and some $u:2^k$ such that 
%  \begin{equation}
%    (s = u \cdot 1 \cdot \overline 0|_n)
%    \wedge 
%    (t = u \cdot 0 \cdot \overline 1|_n)
%  \end{equation}
%\end{lemma}
%\begin{proof}
%%  By injectivity of $cs_m$, 
%  By assumption, we have that $s|_m \neq t|_m$. 
%  Then there must be some smallest number $k<m$ such that 
%  $s(k) \neq t(k)$. As $k$ is minimal, we have $s|_k = t|_k = : u$. 
%%  WLOG, we assume that $s(m) = 1, t(m) = 0$. 
%  It follows for all $l\leq n$ that 
%%  We thus have for all $k<l\leq n$ that 
%%  \begin{align}
%%    cs_l(s|_l) &= 
%%    cs_k(u|_k) + \sum\limits_{i = k}^{l-1} \frac{s(i)}{2^{i+1}}\\
%%    cs_l(t|_l) &= 
%%    cs_k(u|_k) + \sum\limits_{i = k}^{l-1} \frac{t(i)}{2^{i+1}}
%%  \end{align}
%%  And thus 
%  \begin{align}
%    cs_l(s|_l)-cs_l(t|_l) 
%    = \sum\limits_{i = k}^{l-1} \frac{s(i)-t(i)}{2^{i+1}}
%%    =\frac{s(k)-t(k)}{2^{k+1}} + \sum\limits_{i = k+1}^{l-1} \frac{s(i)-t(i)}{2^{i+1}}
%  \end{align}
%  Note that as $s(i),t(i) \in \{0,1\}$, we must have %that $s(i) -t(i) \in \{-1,0,1\}$. 
%  $|s(i)-t(i)| \leq 1$. 
%  Hence for any $k'<l$, we have 
%  \begin{equation}
%    \left|\sum\limits_{i = k'}^{l-1} \frac{s(i)-t(i)}{2^{i+1}}\right|
%    \leq 
%    \sum\limits_{i = k'}^{l-1} \frac{1}{2^{i+1}}
%    = 
%    \frac{1}{2^{k'}} - \frac{1}{2^{l}}
%  \end{equation}
%  Note that using the two equations above for $l=m$ and $k'=k+1$ we have:
%  \begin{align}
%    cs_m(s|_m) -cs_m(t|_m) 
%    =&
%    \frac{s(k)-t(k)}{2^{k+1}} + \sum\limits_{i = k+1}^{m-1} \frac{s(i)-t(i)}{2^{i+1}} \\
%    \leq& 
%    \frac{s(k)-t(k)}{2^{k+1}} + \left(\frac{1}{2^{k+1}} - \frac{1}{2^{m}}\right)
%  \end{align}
%  As the left hand side should equal $\frac{1}{2^m}$,
%  we must have that $s(k)-t(k) \neq -1$. 
%  As $s(k) \neq t(k)$ it follows that $s(k) = 1, t(k) = 0$.
%  But now 
%  \begin{equation}
%    cs_n(s) -cs_n(t) 
%    =
%    \frac{1}{2^{k+1}} + \sum\limits_{i = k+1}^{n-1} \frac{s(i)-t(i)}{2^{i+1}}
%    \geq 
%    \frac{1}{2^{k+1}} - \left(\frac{1}{2^{k+1}} - \frac{1}{2^n} \right)
%    =
%    \frac{1}{2^{n}}
%  \end{equation}
%  And as $cs_n(s)-cs_n(t) \leq \frac{1}{2^n}$ as well, we get that 
%  $cs_n(s)-cs_n(t) = \frac{1}{2^n}$. 
%  Note that this lower bound is only reached if $s(i)-t(i) = -1$ for all $k<i<n$. 
%  Hence for those $i$, we have $s(i) = 0, t(i) = 1$. 
%  Thus 
%  \begin{equation}
%    s = (u \cdot 1\cdot \overline 0) |_n \wedge 
%    t = (u \cdot 0\cdot \overline 1) |_n.
%  \end{equation}
%\end{proof}
%
% 
%\begin{corollary}\label{alternativeSimByCauchyDistance}
%  Let $n:\N$ and let $s,t:2^n$. Then 
%  \begin{equation}
%    s\sim_n t \leftrightarrow |cs_n(s) - cs_n(t)| \leq \frac{1}{2^{n}}.
%  \end{equation} 
%\end{corollary}
%
%\begin{proof}
%  \item  
%    Assume $ s \sim_n t$. If $s=t$, we have $cs_n(s) - cs_n(t) = 0$, 
%    otherwise, we may without loss of generality assume there is some $m<n$ and some $u:2^m$ such that 
%  \begin{equation}
%    (s = u \cdot 0 \cdot \overline 1|_n) \wedge ( t = u \cdot 1 \cdot \overline 0 |_n) . 
%  \end{equation}
%  Then 
%  \begin{align}
%    cs_n(s) &= 
%    cs_m(u) + 0 + \sum\limits_{i = m+1}^{n-1} \frac{1}{2^{i+1}}\\
%    cs_n(t) &= 
%    cs_m(u) + \frac{1}{2^{m+1}} + 0  
%  \end{align}
%  And hence 
%  \begin{equation}
%    cs_n(t) - cs_n(s) = \frac{1}{2^{m+1}} - \sum\limits_{i = m+1}^{n-1} \frac{1}{2^{i+1}} = \frac{1}{2^n}
%  \end{equation}
%  Thus in all cases, from $s\sim_n t$, we can conclude that 
%  \begin{equation}
%    |cs_n(s) -cs_n(t) |\leq \frac{1}{2^n}
%  \end{equation}
%  \item 
%  Conversely, assume that $|cs_n(s) - cs_n(t)| \leq \frac{1}{2^n}$. 
%  If $s = t$, it is clear that $s \sim_n t$.
%  If $s\neq t$, there must be some smallest number $m<n$ such that 
%  $s(m) \neq t(m)$. As $m$ is minimal, we have $s|_m = t|_m = : u$. 
%  WLOG, we assume that $s(m) = 1, t(m) = 0$. 
%  Then $cs_m(s|_{m+1})  = cs_{m+1}(t|_{m+1}) + \frac{1}{2^{m+1}}$
%  and by \Cref{CauchyApproxLemma} it follows that 
%  \begin{equation}
%    s = (u \cdot 1\cdot \overline 0) |_n \wedge 
%    t = (u \cdot 0\cdot \overline 1) |_n.
%  \end{equation}
%  and thus we can conclude $s\sim_n t$ as required. 
%\end{proof}
%

%Inspired by Definitions 2.7 and 2.10 \cite{Bishop}, 
%we define inequality on $\I$ as follows:
%\begin{definition}
%  Let $\alpha,\beta:2^\N$. 
%  We define $\alpha\leq_\I \beta$ and $\alpha<_\I\beta$ as follows:
%  \begin{align}
%  \alpha\leq_\I\beta : = \forall_{n:\N} \left( cs(\alpha)_n \leq cs(\beta)_n + \frac {1} {2^n}\right)\\ 
%    \alpha   <_\I \beta : = \exists_{n:\N} \left( cs(\alpha)_n < cs(\beta)_n - \frac {1} {2^n}\right)
%%    \\\rednote{Can become n\pm1, \leq ,<, +\frac1{2^n+2} }
%\end{align}
%\end{definition}
%\begin{lemma}
%  $\leq_\I$ respects $\sim_\I$. 
%\end{lemma}
%\begin{proof}
%  We will show that whenever $\alpha\leq_\I \gamma$ and $\alpha\sim_\I\beta$, we have $\beta\leq_\I\gamma$. 
%  The other proof obligation goes similarly. 
%%  The proof is similar to $\alpha'\leq_\I\gamma'$ and $\gamma'\sim_\I\beta'$, we have $\alpha'\leq_\I\beta'$.
%
%
%  As $\beta\leq_\I\gamma$ is closed, by \Cref{rmkOpenClosedNegation} it is double negation stable. 
%  By \Cref{MarkovPrinciple}, the negation is that there is some 
%  $N:\N$ with 
%  $cs(\beta)_N > cs(\gamma)_N + \frac{1}{2^N}.$
%  As $\alpha\leq_\I\gamma$, we have 
%  $cs(\gamma)_N + \frac{1}{2^N}\geq cs(\alpha)_N $. 
%  Thus $cs(\beta)_N > cs(\alpha)_N$ and therefore $cs(\beta)_N = cs(\alpha)_N+\frac{1}{2^N}$ using  $\alpha\sim_\I\beta$.
%%  Yet as $\alpha\sim_\I\beta$, from \Cref{alternativeSimByCauchyDistance}
%%  we have $cs(\beta)_n \leq cs(\alpha)_n + \frac{1}{2^n}$ for all $n:\N$. 
%%  Therefore, by \Cref{CauchyApproxLemma}, for $n\geq N$, we may conclude that 
%  It follows that 
%  $$
%  cs(\alpha)_N+\frac{1}{2^N} > cs(\gamma)_N + \frac{1}{2^N} \geq cs(\alpha)_N
%  $$
%  From \Cref{rmkPropertiesCSn}, we must have
%  $cs(\gamma)_N  + \frac{1}{2^N} = cs(\alpha)_N$, otherwise the distance 
%  between $cs(\gamma)_N$ and $cs(\alpha)_N$ 
%  would be smaller than $\frac{1}{2^N}$.
%%  Using again $\alpha\sim_\I\beta$ and \Cref{CauchyApproxLemma}, 
%%  for $n\geq N$ we get 
%%  $cs(\beta)_n = cs(\alpha)_n + \frac{1}{2^n}$.
%  As $cs(\alpha)_n \leq cs(\gamma)_n + \frac{1}{2^n}$ for all $n\geq N$, 
%  \Cref{CauchyApproxLemma} gives that 
%  $\alpha\sim_\I\gamma$. But also $\beta\sim_\I\gamma$. 
%  But now $\alpha,\beta,\gamma$ are all distinct yet related by $\sim_\I$, contradicting 
%  \Cref{IntervalFiberSizeAtMost2}. 
%\end{proof}
%
%\begin{remark}\label{NegationOfGeq}
%  By \Cref{MarkovPrinciple}, we have that $\neg (\alpha \leq \beta) \leftrightarrow (\beta <_\I \alpha)$. 
%  It follows immediately that $<_\I$ also respects $\I$. 
%  Therefore, $\leq_\I, <_\I$ induce relations $\leq,<$ on $\I$.
%  As the order in $\mathbb Q$ is decidable, $\leq, <$ are closed and open respectively. 
%\end{remark} 
%
%\begin{lemma}\label{IntervalOrderLeqOrGeq}
%  For any $x,y:\I$, we have $x\leq y \vee y \leq x$. 
%\end{lemma}
%\begin{proof}
%  Note that $x\leq y \vee y \leq x$ is the disjunction of two closed propositions, hence by 
%  \Cref{ClosedFiniteDisjunction} and \Cref{rmkOpenClosedNegation} we can show it's double negation instead. 
%  By the above remark, the negation implies that $x>y$ and $y<x$. We will show this is a contradiction. 
%  Let $\alpha,\beta:2^\N$ correspond to $x,y$ and assume $n,m:\N$ with 
%  $cs(\alpha)_n < cs(\beta)_n-\frac{1}{2^n}$ and 
%  $cs(\beta)_m < cs(\alpha)_m-\frac{1}{2^m}$. 
%  WLOG assume $n<m$. In this case for $\gamma$ any of $\alpha,\beta$, we have
%  $$0\leq cs(\gamma)_m - cs(\gamma)_n = \sum_{i = n}^{m-1} \frac{\gamma(i)}{2^{i+1}}\leq \frac{1}{2^n}-\frac{1}{2^m}$$
%  While at the same time, we have 
%  \begin{align}
%    cs(\beta)_m - cs(\beta)_n &\leq cs(\alpha)_m -\frac{1}{2^m} - cs (\beta)_n \\
%                              & = (cs(\alpha)_m-cs(\alpha)_n)  +      (cs(\alpha)_n -cs(\beta)_n) - \frac{1}{2^m}\\
%                              & \leq (\frac{1}{2^n} - \frac{1}{2^m}) -\frac{1}{2^n}               - \frac{1}{2^m}\\
%                              &<0
%  \end{align}
%  giving a contradiction as required. 
%\end{proof}
%
%\begin{remark}\label{rmkMapOutOfLeqGeq}
%  From \Cref{alternativeSimByCauchyDistance} we have $((x\leq y) \wedge (y \leq x )) \leftrightarrow (x = y)$. 
%  So in order to define a map $(x \leq y) \vee (y \leq x) \to P$, we need to define a map 
%  $f:x\leq y \to P$ and a map $g:y \leq x \to P$ such that $f|_{x = y} = g|_{x=y}$. 
%\end{remark}
%\rednote{These properties are nice but not necessary and paused WIP:}
%\rednote{It is no used for Bouwer's fixed point theorem}
%\begin{corollary}\label{inequality-lesser-greater-than}
%    For $x,y:\I$ we have $(x\leq y \wedge x \neq y) \leftrightarrow (x < y)$. 
%    Also $(x\neq y) \leftrightarrow (x < y + x > y)$. 
%\end{corollary} 
%\begin{proof}
%    By $(x<y)\leftrightarrow \neg (y\leq x)$
%    It's also immediate from the definitions that $x<y$ implies $x\neq y$. 
%    As $((x\leq y) \wedge (y \leq x )) \leftrightarrow (x = y)$, 
%    if $x\leq y \wedge x \neq y$, we have $\neg (y \leq x)$, hence $x<y$. 
%\end{proof}
%
%%    \item $(\exists_{y:I}(x\leq y \wedge y \leq z ))\leftrightarrow (x \leq z)$. 
%%    \item $(\exists_{y:I}(x<y \wedge y < z ))\leftrightarrow (x < z)$. 
%
%\begin{lemma}
%  Whenever $x,y:\I$ satisfy $x<y$, there is some $z:\I$ with  $x<z \wedge z< y$. 
%\end{lemma} 
%
%%
%%\rednote{TODO}
%%For any $x,y:I$ we have 
%%\begin{itemize}
%%  \item TODO $x\leq y \wedge x \neq y \to x < y$. 
%%\end{itemize}
%
%
%
%\subsection{The topology of the interval}
%
%
%\begin{definition}
%  Let $a,b:\I$. 
%  Following standard notation, we denote
%  \begin{equation}
%    [a,b]:= \Sigma_{x:\I} (a\leq x \wedge x \leq b),
%  \end{equation}
%  we call subsets of $\I$ of this form closed intervals. 
%%
%  We also denote 
%  \begin{align}
%    (-\infty,a) &:= \Sigma_{x:\I} (x < a)   \\
%    (a,\infty) &:= \Sigma_{x:\I} (a < x)  \\
%    (-\infty,\infty) &:= \I  \\
%    (a,b) &:= \Sigma_{x:\I} (a < x \wedge x < b),
%  \end{align}
%  We call subsets of $\I$ of these forms open intervals. 
%\end{definition}
%\begin{remark}
%  Note that closed intervals and open intervals are closed and open respectively. 
%\end{remark}
%
%
%%\begin{lemma}\label{IntervalQuotientMapIntersectionCommute}
%%  Let $D_n:2^\N \to 2$ be a sequence of decidable subsets with $D_{n+1}\subseteq D_n$.
%%  For $p$ the quotient map $2^\N \to I$, we have that 
%%  $p(\bigcap_{n:\N} D_n) = p(\bigcap_{n:\N} D_n)$
%%\end{lemma}
%%\begin{proof}
%%  It is always the case that $$p(\bigcap_{n:\N} D_n) \subseteq \bigcap_{n:\N} p(D_n).$$
%%  For the converse direction, let $(\bigcap_{n:\N} p(D_n))(x)$. 
%%  We will show that $ \neg \neg (p(\bigcap D_n)) (x)$, which is sufficient by \Cref{rmkOpenClosedNegation}. 
%%%
%%  As $(\bigcap_{n:\N} p(D_n))(x)$, there exists some $y\in D_0$ with $p(y) = x$. 
%%%
%%  If $x\notin p(\bigcap_{n:\N} D_n)$, we cannot have for all $n:\N$ that $y_0 \in  D_n$. 
%%  By Markov, there must exist some $k:\N$ with $\neg D_k(y_0)$. 
%%  As $D_{n+1}\subseteq D_n$ for all $n:\N$, it follows that $y_0\notin D_n$ for all $n\geq k$. 
%%%
%%  As $x\in \bigcap_{n:\N}p(D_n)$, there is however some $y_k\in D_k$ with $p(y_k) = x$. 
%%  By a similar argument, we have some $l>k$ with $y_k\notin D_l$, and some $y_l$ with $p(y_l) = x, y_l \in D_l$. 
%%  But now we have that $y_0, y_k, y_l:2^\N$ are all distinct, but $p(y_0) = p(y_k) = p(y_l) = x$. 
%%  This contradicts \Cref{IntervalFiberSizeAtMost2}, and we're done. 
%%\end{proof}
%

The following is also given by Definitions 2.7 and 2.10 of \cite{Bishop}.

\begin{definition}
  Assume given $x,y:\I$ and $\alpha,\beta:2^\N$ such that $x=cs(\alpha), y=cs(\beta)$.
  Then $x<y$ is the proposition $\exists_{n:\N}\, cs_n(\alpha)+\frac{1}{2^{n}}<_{\mathbb Q}cs_n(\beta)$, which is independent of the choice of $\alpha,\beta$.
\end{definition}

\begin{remark}\label{LesserOpenPropAndApartness}
  For all $x,y:\I$, we have that $x<y$ is an open proposition and that $x\neq y$ is equivalent to $(x<y) \vee (y<x)$.
\end{remark}
%
\begin{lemma}\label{ImageDecidableClosedInterval}
  For all $D\subseteq 2^\N$ decidable, we have that $cs(D)$ is a finite union of closed intervals. 
\end{lemma}
\begin{proof}
  If $D$ is contains precisely the $\alpha:2^\N$ with a fixed initial segment, $cs(D)$ is a closed interval. 
  Any decidable subset of $2^\N$ is a finite union of such subsets. 
\end{proof}
%\begin{proof}
%  We will show the above if there exists some $n:\N, u:2^n$ such that 
%  $D(\alpha) \leftrightarrow \alpha|_n = u$.
%  This is sufficient, as any decidable subset of $2^\N$ can be written as finite union of such decidable subsets. 
%  We claim that $p(D) = [p(u\cdot \overline 0) , p(u \cdot \overline 1)]$. 
%\item 
%  We will first show that $p(D) \subseteq [p(u\cdot \overline 0) , p(u \cdot \overline 1)]$. 
%  Suppose $D(\alpha)$. Then 
%  Then $\alpha|_n = u$ and hence 
%%  for $m\leq n$ we have 
%%  \begin{equation}
%%    cs(\alpha)_m = cs_m(u|_m) = cs(u\cdot \overline 0)_m= cs(u\cdot \overline 1)_m
%%  \end{equation}
%%  For $m>n$, we have that 
%%  \begin{align}
%%    cs(u\cdot \overline 1)_m =
%%    cs_n(u) +\sum_{i = n} ^{m-1} \frac{1}{2^{i+1}}
%%    \\
%%    cs(\alpha)_m =
%%    cs_n(u) +\sum_{i = n} ^{m-1} \frac{\alpha(i)}{2^{i+1}}
%%    \\
%%    cs(u\cdot \overline 0)_m = 
%%    cs_n(u) +\sum_{i = n} ^{m-1} \frac{0}{2^{i+1}}
%%  \end{align} 
%%  Hence for all $m:\N$, we have 
%  \begin{equation}
%    cs(u\cdot \overline 1)_m \geq 
%    cs(\alpha)_m \geq 
%    cs(u\cdot\overline 0)_m
%  \end{equation}
% which implies that $p(u\cdot \overline 1) \geq_\I p(\alpha) \geq_\I p(u\cdot\overline 0)$, as required. 
%\item 
%  To show that $[p(u\cdot \overline 0) , p(u \cdot \overline 1)]\subseteq p(D)$, 
%  Suppose
%  $(u\cdot \overline 0) \leq_\I \alpha \leq_\I (u \cdot \overline 1)$. 
%  It is sufficient to show that 
%  $$(\alpha|_n = u )\vee (\alpha \sim_\I u \cdot \overline 0 )\vee (\alpha \sim_\I u \cdot \overline 1).$$
%  As this is a disjunction of closed propositions, by \Cref{ClosedFiniteDisjunction} it's closed, and by 
%  \Cref{rmkOpenClosedNegation}, we can instead show the double negation. 
%  So suppose that none of the disjoints hold. 
%  As $\alpha|_n \neq u$, there is some minimal $m$ with $\alpha(m) \neq u(m)$. 
%  We assume that $\alpha(m) = 1, u(m) = 0$, the other case goes similarly. 
%  Then for all $k:\N$, we have 
%  $cs(\alpha)_k \geq cs(u \cdot \overline 1)|_k$. 
%  As also 
%  $(u\cdot \overline 1)\geq_\I \alpha$, we have 
%  $$cs(u \cdot \overline 1)|_k + \frac{1}{2^k} \geq cs(\alpha)_k \geq cs(u\cdot \overline 1)_k,$$
%  From which it follows that $|cs(u\cdot\overline 1)_k - cs(\alpha)_k|\leq \frac{1}{2^k}$. 
%  Hence $(u\cdot \overline 1)|_k \sim_k \alpha|_k$ by \Cref{alternativeSimByCauchyDistance}. 
%  Hence $x\sim_\I (a\cdot\overline 1)$, contradicting our assumption as required. 
%\end{proof}
%
\begin{lemma}\label{complementClosedIntervalOpenIntervals}
  The complement of a finite union of closed intervals is 
  a finite union of open intervals. 
\end{lemma}
%\begin{proof}
%  We'll use induction on the amount of closed intervals. 
%  The empty union of closed intervals is empty, and hence it's complement is $\I$, which is an open interval.  
%  Let $(C_i)_{i<k}$ be a finite set of closed intervals with $\neg (\bigcup_{i<k}C_i)$ 
%  a finite union of open intervals $\bigcup_{j<l} O_i$. 
%  Suppose $C_{k}$ is closed. We need to show that 
%  $\neg (\bigcup_{i\leq k} C_i)$ is also a finite union of open intervals. 
%  First note that in general, 
%  $(\neg (A \vee B ))\leftrightarrow (\neg A \wedge \neg B)$
%  hence 
%  $$
%  \neg (\bigcup_{i\leq k} C_i)
%  = 
%  \neg ((\bigcup_{i<k} C_i) \cup C_k) 
%  =
%  (\neg (\bigcup_{i<k} C_i) )\cap (\neg C_k) 
%  $$
%  And by the induction hypothesis and distributivity, this equals 
%  $$
%  (\bigcup_{j<l} O_i) ) \cap (\neg C_k) 
%  =
%  \bigcup_{j<l} (O_i \cap (\neg C_k) )
%  $$
%  So we need to show that the intersection of an open interval and the negation of a closed interval is a 
%  finite union of open intervals. We assume or open intervals are of the form $(a,b)$ for $a,b:\I$. 
%  The other cases are very similar. 
%  So let $a,b,c,d:\I$ and consider 
%  $U = (a,b) \cap (\neg [c,d])$. 
%  Then 
%  \begin{align} 
%    U(x) &= \Sigma_{x:\I}  
%  (a < x \wedge x < b) \wedge ( x < c \vee d < x)\\
%  &= \Sigma_{x:\I}
%  (a < x \wedge x < b \wedge x < c ) \vee ( d < x \vee a<x \wedge x < b)\\
%  &= 
%  \Sigma_{x:\I}
%  (a < x \wedge x < b \wedge x < c ) 
%  \cup 
%  \Sigma_{x:\I}
%  ( d < x \vee a<x \wedge x < b)
%  \end{align} 
%  We will show that 
%  $U' = \Sigma_{x:\I}(a < x \wedge x < b \wedge x < c ) $ is an open interval. 
%  By a similar argument, the other part will be as well, meaning that $U$ is the union of two open intervals. 
%  Consider that $b\leq c \vee c \leq b$. 
%  If $b \leq c$, $(x<b \wedge x< c) \leftrightarrow x<b$ and $U' = (a,b)$
%  If $c \leq b$, $(x<b \wedge x< c) \leftrightarrow x<c$ and $U' = (a,c)$
%  If $b=c$, these open intervals agree, hence from \Cref{rmkMapOutOfLeqGeq} we can conclude that $U'$ is an open interval. 
%  We conclude that $U$ is the union of two open intervals as required. 
%\end{proof}
%%
By \Cref{CompactHausdorffTopology} we can thus conclude:
\begin{lemma}\label{IntervalTopologyStandard}
  Every open $U\subseteq \I$ can be written as a countable union of open intervals.
\end{lemma} 
%\begin{proof}
%%  Let $U\subseteq I$ open, then $\neg U$ is closed and $U = \neg \neg U$ by \Cref{rmkOpenClosedNegation}. 
%%  By \Cref{StoneClosedSubsets}, \Cref{CompactHausdorffClosed} and \Cref{ChausMapsPreserveIntersectionOfClosed}
%  %\Cref{IntervalQuotientMapIntersectionCommute}, 
%  By \Cref{CompactHausdorffTopology}
%  there is some sequence of decidable subsets $D_n\subseteq 2^\N$ 
%%  with $\neg U = \bigcap_{n:\N} p(D_n)$. 
%%  Thus $U = \neg \bigcap_{n:\N} p(D_n)$. 
%%  By \Cref{ClosedMarkov}, it follows that 
%  with $U = \bigcup_{n:\N} \neg p(D_n)$. 
%  By \Cref{ImageDecidableClosedInterval}, each $p(D_n)$ is a finite union of closed intervals, 
%  and by \Cref{complementClosedIntervalOpenIntervals} it follows that each $\neg p(D_n)$ is a finite union of open intervals. 
%  We conclude that $U$ is a countable union of open intervals as required. 
%\end{proof}
%%
%%%  $\neg U$ is a countable intersection of finite unions of closed intervals. 
%%%  Thus $\neg\neg U$ is a countable union of finite intersections of complements of closed intervals. 
%%%  As complements of closed intervals are finite unions of open intervals (TODO), 
%%%  and finite intersections of such things are still finite unions of open intervals, 
%%%  it follows that $\neg\neg U$ is a countable union of open intervals. 
%%%  By \Cref{rmkOpenClosedNegation}, $\neg \neg U = U$ and we're done. 
%%%  \rednote{Lotta handwaving here, definitely not finished} 
%%\end{proof}
%%
%
%\begin{remark}\label{IntervalTopologyStandard}
  It follows that the topology of $\I$ is generated by open intervals, 
  which corresponds to the standard topology on $\I$. 
  Hence our notion of continuity agrees with the 
  $\epsilon,\delta$-definition of continuity one would expect and we get the following:
\begin{theorem}
  Every function $f:\I\to \I$ is continuous in the $\epsilon,\delta$-sense. 
\end{theorem}
%\end{remark}

 

\section{Cohomology}
In this section we compute $H^1(S,\Z) = 0$ for all $S$ Stone, and show that $H^1(X,\Z)$ for $X$ compact Hausdorff can be computed using \v{C}ech cohomology. We use this to compute $H^1(\I,\Z)=0$. 

\begin{remark}
We only work with the first cohomology group with coefficients in $\Z$ as it is sufficient for the proof of Brouwer's fixed-point theorem, but the results could be extended to $H^n(X,A)$ for $A$ any family of countably presented abelian groups indexed by $X$.
\end{remark}

\begin{remark}
We write $\mathrm{Ab}$ for the type of abelian groups and if $G:\mathrm{Ab}$ we write $\B G$ for the delooping of $G$ \cite{hott,davidw23}. This means that $H^1(X,G)$ is the set truncation of $X \to \B G$. 
\end{remark}

\subsection{\v{C}ech cohomology}

\begin{definition}
Given a type $S$, types $T_x$ for $x:S$ and $A:S\to\mathrm{Ab}$, we define $\check{C}(S,T,A)$ as the chain complex
\[
\begin{tikzcd}
     \prod_{x:S}A_x^{T_x} \ar[r,"d_0"] & \prod_{x:S}A_x^{T_x^2}\ar[r,"d_1"] &  \prod_{x:S}A_x^{T_x^3}
\end{tikzcd}
\]
where the boundary maps are defined as
\begin{align*}
d_0(\alpha)_x(u,v) =&\ \alpha_x(v)-\alpha_x(u)\\
d_1(\beta)_x(u,v,w) =&\ \beta_x(v,w) - \beta_x(u,w) + \beta_x(u,v)
\end{align*}
\end{definition}

\begin{definition}
Given a type $S$, types $T_x$ for $x:S$ and $A:S\to\mathrm{Ab}$, we define its \v{C}ech cohomology groups by
\[
  \check{H}^0(S,T,A) = \mathrm{ker}(d_0)\quad \quad \quad \check{H}^1(S,T,A) = \mathrm{ker}(d_1)/\mathrm{im}(d_0)
\]
We call elements of $\mathrm{ker}(d_1)$ cocycles and elements of $\mathrm{im}(d_0)$ coboundaries.
\end{definition}

This means that $\check{H}^1(S,T,A) = 0$ if and only if $\check{C}(S,T,A)$ is exact at the middle term. Now we give three general lemmas about \v{C}ech complexes.

\begin{lemma}\label{section-exact-cech-complex}
Assume a type $S$, types $T_x$ for $x:S$ and $A:S\to\mathrm{Ab}$ with $t:\prod_{x:S}T_x$. Then $\check{H}^1(S,T,A)=0$.
\end{lemma}

\begin{proof}
Assume given a cocycle, i.e. $\beta:\prod_{x:S}A_x^{T_x^2}$ such that for all $x:S$ and $u,v,w:T_x$ we have that $\beta_x(u,v)+\beta_x(v,w) = \beta_x(u,w)$. We define $\alpha:\prod_{x:S}A_x^{T_x}$ by $\alpha_x(u) = \beta_x(t_x,u)$. Then for all $x:S$ and $u,v:T_x$ we have that $d_0(\alpha)_x(u,v) =  \beta_x(t_x,v) - \beta_x(t_x,u) = \beta_x(u,v)$ so that $\beta$ is a coboundary.
\end{proof}

\begin{lemma}\label{canonical-exact-cech-complex}
Given a type $S$, types $T_x$ for $x:S$ and $A:S\to\mathrm{Ab}$, we have that $\check{H}^1(S,T,\lambda x.A_x^{T_x})=0$.
\end{lemma}

\begin{proof}
Assume given a cocycle, i.e. $\beta:\prod_{x:S}A_x^{T_x^3}$ such that for all $x:S$ and $u,v,w,t:T_x$ we have that $\beta_x(u,v,t)+\beta_x(v,w,t) = \beta_x(u,w,t)$. We define $\alpha:\prod_{x:S}A_x^{T_x^2}$ by $\alpha_x(u,t) = \beta_x(t,u,t)$. Then for all $x:S$ and $u,v,t:T_x$ we have that $d_0(\alpha)_x(u,v,t) = \beta_x(t,v,t) - \beta_x(t,u,t) = \beta_x(u,v,t)$ so that $\beta$ is a coboundary.
\end{proof}

\begin{lemma}\label{exact-cech-complex-vanishing-cohomology}
Assume a type $S$ and types $T_x$ for $x:S$ such that $\prod_{x:S}\propTrunc{T_x}$ and $A:S\to\mathrm{Ab}$ such that $\check{H}^1(S,T,A) = 0$.
Then given $\alpha:\prod_{x:S}\B A_x$ with $\beta:\prod_{x:S} (\alpha(x) = *)^{T_x}$, we can conclude $\alpha = *$.
\end{lemma}

\begin{proof}
We define $g : \prod_{x:S} A_x^{T_x^2}$ by $g_x(u,v) = \beta_x(v) - \beta_x(u)$.
It is a cocycle in the \v{C}ech complex, so that by exactness there is $f:\prod_{x:S}A_x^{T_x}$ such that for all $x:S$ and $u,v:T_x$ we have that $g_x(u,v)= f_x(v) - f_x(u)$.
Then we define $\beta' : \prod_{x:S}(\alpha(x)=*)^{T_x}$ by $\beta'_x(u) = \beta_x(u) - f_x(u)$
so that for all $x:S$ and $u,v:T_x$ we have that $\beta'_x(u) = \beta'_x(v)$ is equivalent to $f_x(v) - f_x(u) = \beta_x(v) - \beta_x(u)$, which holds by definition. So $\beta'$ is constant on each $T_x$ and therefore gives $\prod_{x:S} (\alpha(x)=*)^{\propTrunc{T_x}}$. By $\prod_{x:S}\propTrunc{T_x}$ we conclude $\alpha = *$.
\end{proof}


\subsection{Cohomology of Stone spaces}

%
%%\subsection{Needed results}
%
%%\rednote{Should probably be moved elsewhere}
%
\begin{lemma}\label{finite-approximation-surjection-stone}
Assume given $S:\Stone$ and $T:S\to\Stone$ such that $\prod_{x:S}\propTrunc{T(x)}$.
Then there exists a sequence of finite types $(S_k)_{k:\N}$ with limit $S$ 
%\rednote{Should the maps in the sequence be mentioned? (Maps $p_k$ are mentioned below)}
%\begin{equation}
%\begin{tikzcd}
%S_0 & S_1 \ar[l,"p_0"]& S_2\ar[l,"p_1"] & \cdots\ar[l]\\
%\end{tikzcd}
%\end{equation}
%such that: 
%\[\mathrm{lim}_kS_k = S\]
and a compatible sequence of families of finite types $T_k$ over $S_k$
with $\prod_{x:S_k}\propTrunc{T_k(x)}$ and 
$\mathrm{lim}_k\left(\sum_{x:S_k}T_k(x)\right) = \sum_{x:S}T(x)$. 
%
%Given $S:\Stone$ and $T:S\to\Stone$ such that $\prod_{x:S}\propTrunc{T(x)}$, there exists a sequence of finite types $(S_k)_{k:\N}$
%\rednote{Should the maps in the sequence be mentioned? (Maps $p_k$ are mentioned below)}
%%\begin{equation}
%%\begin{tikzcd}
%%S_0 & S_1 \ar[l,"p_0"]& S_2\ar[l,"p_1"] & \cdots\ar[l]\\
%%\end{tikzcd}
%%\end{equation}
%such that: 
%\[\mathrm{lim}_kS_k = S\]
%and for each $k:\N$ we have a family of finite types $T_k(x)$ for $x:S_k$ such that $\prod_{x:S_k}\propTrunc{T_k(x)}$ with maps $T_{k+1}(x) \to T_k(p_k(x))$ such that:
%\[\mathrm{lim}_k\left(\sum_{x:S_k}T_k(x)\right) = \sum_{x:S}T(x)\]
\end{lemma}

\begin{proof}
By theorem \Cref{stone-sigma-closed} and the usual correspondence between surjections and families of inhabited types, a family of inhabited Stone spaces over $S$ correspond to a Stone space $T$ with a surjection $T\to S$. Then we conclude using \Cref{ProFiniteMapsFactorization}.
%\rednote{ \@ Hugo This follows from \Cref{ProFiniteMapsFactorization} and \Cref{stone-sigma-closed} 
%  and considering the surjection $(\Sigma_{x:S} T(x)) \to S$, but we discussed whether it might be easier to 
%  refactor the proof where you use the above or make a remark after \Cref{stone-sigma-closed}}
\end{proof}

\begin{lemma}\label{cech-complex-vanishing-stone}
Assume given $S:\Stone$ with $T:S\to\Stone$ such that $\prod_{x:S}\propTrunc{T_x}$. Then we have that $\check{H}^1(S,T,\Z) = 0$.
\end{lemma}


\begin{proof}
We apply \cref{finite-approximation-surjection-stone} to get $S_k$ and $T_k$ finite. Then by \cref{scott-continuity} we have that $\check{C}(S,T,\Z)$ is the sequential colimit of the $\check{C}(S_k,T_k,\Z)$. By \cref{section-exact-cech-complex} we have that each of the $\check{C}(S_k,T_k,\Z)$ is exact, and a sequential colimit of exact sequences is exact.
\end{proof}

\begin{lemma}\label{eilenberg-stone-vanish}
Given $S:\Stone$, we have that $H^1(S,\Z) = 0$. 
\end{lemma}

\begin{proof}
Assume given a map $\alpha:S\to \B\Z$. We use local choice to get $T:S\to\Stone$ such that $\prod_{x:S}\propTrunc{T_x}$ with $\beta:\prod_{x:S}(\alpha(x)=*)^{T_x}$. Then we conclude by \cref{cech-complex-vanishing-stone} and \cref{exact-cech-complex-vanishing-cohomology}.
\end{proof}

\begin{corollary}\label{stone-commute-delooping}
For any $S:\Stone$ the canonical map $\B(\Z^S) \to (\B\Z)^S$ is an equivalence.
\end{corollary}

\begin{proof}
This map is always an embedding. To show it is surjective it is enough to prove that $(\B\Z)^S$ is connected, which is precisely \Cref{eilenberg-stone-vanish}.
\end{proof}


\subsection{\v{C}ech cohomology of compact Hausdorff spaces}

\begin{definition}
A \v{C}ech cover consists of $X:\CHaus$ and $S:X\to\Stone$ such that $\prod_{x:X}\propTrunc{S_x}$ and $\sum_{x:X}S_x:\Stone$.
\end{definition}

By definition any compact Hausdorff space $X$ is part of a \v{C}ech cover $(X,S)$.

\begin{lemma}\label{cech-eilenberg-0-agree}
Given a \v{C}ech cover $(X,S)$ and $A:X\to\mathrm{Ab}$, we have an isomorphism $H^0(X,A) = \check{H}^0(X,S,A)$ natural in $A$.
\end{lemma}

\begin{proof}
By definition an element in $\check{H}^0(X,S,A)$ is a map $f:\prod_{x:X}A_x^{S_x}$
such that for all $u,v:S_x$ we have $f(u)=f(v)$. Since $A_x$ is a set and the $S_x$ are merely inhabited, this is equivalent to $\prod_{x:X}A_x$. Naturality in $A$ is immediate.
\end{proof}

\begin{lemma}\label{eilenberg-exact}
Given a \v{C}ech cover $(X,S)$ we have an exact sequence
\[H^0(X,\lambda x.\Z^{S_x}) \to H^0(X,\lambda x.\Z^{S_x}/\Z) \to H^1(X,\Z)\to 0\]
\end{lemma}

\begin{proof}
We use the long exact cohomology sequence associated to
\[0 \to \Z \to \Z^{S_x} \to \Z^{S_x}/\Z\to 0\]
We just need $H^1(X,\lambda x.\Z^{S_x}) = 0$ to conclude. But by \cref{stone-commute-delooping} we have that $H^1(X,\lambda x.\Z^{S_x}) = H^1\left(\sum_{x:X}S_x,\Z\right)$ which vanishes by \cref{eilenberg-stone-vanish}.
\end{proof}

\begin{lemma}\label{cech-exact}
Given a \v{C}ech cover $(X,S)$ we have an exact sequence
\[\check{H}^0(X,S,\lambda x.\Z^{S_x}) \to \check{H}^0(X,S,\lambda x.\Z^{S_x}/\Z) \to \check{H}^1(X,S,\Z)\to 0\]
\end{lemma}

\begin{proof}
For $n=1,2,3$, we have that $\Sigma_{x:X}S_x^n$ is Stone so that  $H^1(\Sigma_{x:X}S_x^n, \Z) = 0$ by \cref{eilenberg-stone-vanish}, giving short exact sequences
\[0\to \Pi_{x:X}\Z^{S_x^n} \to \Pi_{x:X}(\Z^{S_x})^{S_x^n}\to \Pi_{x:X}(\Z^{S_x}/\Z)^{S_x^n}\to 0\]
They fit together in a short exact sequence of complexes
\[0 \to \check{C}(X,S,\Z) \to \check{C}(X,S,\lambda x.\Z^{S_x}) \to \check{C}(X,S,\lambda x.\Z^{S_x}/\Z)\to 0\]
But since $\check{H}^1(X,\lambda x.\Z^{S_x}) = 0$ by \cref{canonical-exact-cech-complex}, we conclude using the associated long exact sequence.
\end{proof}

\begin{theorem}\label{cech-eilenberg-1-agree}
Given a \v{C}ech cover $(X,S)$, we have that $H^1(X,\Z) = \check{H}^1(X,S,\Z)$
\end{theorem}

\begin{proof}
By applying \cref{cech-eilenberg-0-agree}, \cref{eilenberg-exact} and \cref{cech-exact} we get that $H^1(X,\Z)$ and $\check{H}^1(X,S,\Z)$ are cokernels of isomorphic maps, so they are isomorphic.
\end{proof}

This means that \v{C}ech cohomology does not depend on $S$.

\subsection{Cohomology of the interval}
%
%Recall that we denote $C_n=2^n$ with a binary relation $\sim_n$ on $C_n$ such that for all $x,y:2^\N$ we have that:
%\[\left(\forall(n:\N).\ x|_n\sim_n y|_n\right) \leftrightarrow x=_\I y\]
%
%\begin{lemma}\label{description-Cn-simn}
%We have that $(C_n,\sim_n)$ is equivalent to $(\Fin(2^n),\lambda x,y.\ |x-y|\leq 1)$.
%\end{lemma}
\begin{remark}\label{description-Cn-simn}
  Recall from \Cref{def-cs-Interval} that 
  there is a binary relation $\sim_n$ on $2^n=:\I_n$ such that 
  $(2^n,\sim_n)$ is equivalent to  $(\Fin(2^n),\lambda x,y.\ |x-y|\leq 1)$
  and for $\alpha,\beta:2^\N$ we have $(cs(\alpha) = cs(\beta)) \leftrightarrow 
  \left(\forall_{n:\N}\alpha|_n \sim_n \beta|_n\right)$. 
\end{remark}

We define $\I_n^{\sim2} = \Sigma_{x,y:\I_n}x\sim_n y$ and $\I_n^{\sim3} = \Sigma_{x,y,z:\I_n}x\sim_n y \land y\sim_n z\land x\sim_n z$.

\begin{lemma}\label{Cn-exact-sequence}
For any $n:\N$ we have an exact sequence
\[0\to \Z\overset{d_0}{\longrightarrow} \Z^{\I_n} \overset{d_1}{\longrightarrow} \Z^{\I_n^{\sim2}} \overset{d_2}{\longrightarrow} \Z^{\I_n^{\sim3}}\]
where $d_0(k) = (\_\mapsto k)$ and
\begin{eqnarray}
 d_1(\alpha)(u,v) &=& \alpha(v)-\alpha(u)\nonumber\\
 d_2(\beta)(u,v,w) &=& \beta(v,w)-\beta(u,w)+\beta(u,v).\nonumber
\end{eqnarray}
\end{lemma}

\begin{proof}
It is clear that the map $\Z\to \Z^{\I_n}$ is injective as $\I_n$ is inhabited, so the sequence is exact at $\Z$. Assume a cocycle $\alpha:\Z^{\I_n}$, meaning that for all $u,v:\I_n$, if $u\sim_nv$ then $\alpha(u)=\alpha(v)$. Then by \cref{description-Cn-simn} we see that $\alpha$ is constant, so the sequence is exact at $\Z^{\I_n}$.

Assume a cocycle $\beta:\Z^{\I_n^{\sim2}}$, meaning that for all $u,v,w:\I_n$ such that $u\sim_nv$, $v\sim_nw$ and $u\sim_nw$ we have that $\beta(u,v)+\beta(v,w) = \beta(u,w)$. %This is equivalent to asking $\beta(u,u)=0$ and $\beta(u,v) = -\beta(v,u)$.
Using \cref{description-Cn-simn} to pass along the equivalence between $2^n$ and $\Fin(2^n)$, we define $\alpha(k) = \beta(0,1)+\cdots+\beta(k-1,k)$.
We can check that $\beta(k,l) = \alpha(l)-\alpha(k)$, so that $\beta$ is indeed a coboundary and the sequence is exact at $\Z^{\I_n^{\sim2}}$.
\end{proof}

\begin{proposition}\label{cohomology-I}
We have that $H^0(\I,\Z) = \Z$ and $H^1(\I,\Z) = 0$.
\end{proposition}

\begin{proof}
Consider $cs:2^\N\to\I$ and the associated \v{C}ech cover $T$ of $\I$ defined by: 
\[T_x = \Sigma_{y:2^\N} (x=_\I cs(y))\]
Then for $l=2,3$ we have that $\mathrm{lim}_n\I_n^{\sim l} = \sum_{x:\I} T_x^l$. By \cref{Cn-exact-sequence} and stability of exactness under sequential colimit, we have an exact sequence
\[ 0\to \Z\to \mathrm{colim}_n \Z^{\I_n} \to \mathrm{colim}_n \Z^{\I_n^{\sim2}}\to \mathrm{colim}_n \Z^{\I_n^{\sim3}}\]
By \cref{scott-continuity} this sequence is equivalent to
\[ 0\to \Z\to \Pi_{x:\I}\Z^{T_x} \to  \Pi_{x:\mathbb{I}}\Z^{T_x^2} \to  \Pi_{x:\mathbb{I}}\Z^{T_x^3}\]
So it being exact implies that $\check{H}^0(\I,T,\Z) = \Z$ and $\check{H}^1(\I,T,\Z) = 0$.
We conclude by \cref{cech-eilenberg-0-agree} and \cref{cech-eilenberg-1-agree}.
\end{proof}

\begin{remark}
We could carry a similar computation for $\mathbb{S}^1$, by approximating it with $2^n$ with $0^n\sim_n1^n$ added. We would find $H^1(\mathbb{S}^1,\Z)=\Z$. We will give an alternative, more conceptual proof in the next section.
\end{remark}


\subsection{Brouwer's fixed-point theorem}

Here we consider the modality defined by localising at $\I$ as explained in \cite{modalities}. It is denoted by $L_\I$. We say that $X$ is $\I$-local if $L_\I(X) = X$ and that it is $\I$-contractible if $L_\I(X)=1$.

\begin{lemma}\label{Z-I-local}
$\Z$ and $2$ are $\I$-local.
\end{lemma}

\begin{proof}
By \cref{cohomology-I}, from $H^0(\I,\Z)=\Z$ we get that the map $\Z\to \Z^\I$ is an equivalence, so $\Z$ is $\I$-local. We see that $2$ is $\I$-local as it is a retract of $\Z$.
\end{proof}

\begin{remark}
Since $2$ is $\I$-local, we have that any Stone space is $\I$-local.
\end{remark}

\begin{lemma}\label{BZ-I-local}
$\B\Z$ is $\I$-local.
\end{lemma}

\begin{proof}
Any identity type in $\B\Z$ is a $\Z$-torsor, so it is $\I$-local by \cref{Z-I-local}. So the map $\B\Z\to \B\Z^{\I}$ is an embedding. From $H^1(\I,\Z)=0$ we get that it is surjective, hence an equivalence.
\end{proof}

\begin{lemma}\label{continuously-path-connected-contractible}
Assume $X$ a type with $x:X$ such that for all $y:X$ we have $f:\I\to X$ such that $f(0)=x$ and $f(1)=y$. Then $X$ is $\I$-contractible.
\end{lemma}

\begin{proof}
%First we prove that the map:
%\[\eta_X:X\to L_\I(X)\] 
%is surjective. Indeed its fiber are $\I$-contractible, but for any type $F$ we have a map:
%\[L_\I(F) \to L_\mathbb{F}(\propTrunc{F}) = \propTrunc{F}\] 
For all $y:X$ we get a map $g:\I\to L_\I(X)$ such that $g(0) = [x]$ and $g(1)=[y]$. Since $L_\I(X)$ is $\I$-local this means that $\prod_{y:X}[x]=[y]$. We conclude $\prod_{y:L_\I(X)}[x]=y$ by applying the elimination principle for the modality.
\end{proof}

\begin{corollary}\label{R-I-contractible}
We have that $\R$ and $\mathbb{D}^2=\{(x,y):\mathbb R^2\ \vert\ x^2+y^2\leq 1\}$ are $\I$-contractible.
\end{corollary}

\begin{proposition}\label{shape-S1-is-BZ}
$L_\I(\R/\Z) = \B\Z$.
\end{proposition}

\begin{proof}
As for any group quotient, the fibers of the map $\R\to\R/\Z$ are $\Z$-torsors, so we have an induced pullback square
\[
\begin{tikzcd}
\R\ar[r]\ar[d] & 1\ar[d] \\
\R/\Z\ar[r] & \B\Z
\end{tikzcd}
\]
Now we check that the bottom map is an $\I$-localisation. Since $\B\Z$ is $\I$-local by \cref{BZ-I-local}, it is enough to check that its fibers are $\I$-contractible. Since $\B\Z$ is connected it is enough to check that $\R$ is $\I$-contractible. This is \cref{R-I-contractible}.
\end{proof}

\begin{remark}
By \cref{BZ-I-local}, for any $X$ we have that $H^1(X,\Z) = H^1(L_{\I}(X),\Z)$, so that by \cref{shape-S1-is-BZ} we have that $H^1(\R/\Z,\Z) = H^1(\B\Z,\Z) = \Z$.
\end{remark}

We omit the proof that $\mathbb{S}^1=\{(x,y):\R^2\ \vert\ x^2+y^2=1\}$ is equivalent to $\R/\Z$.
The equivalence can be constructed using trigonometric functions, which exist by Proposition 4.12 in \cite{Bishop}.

\begin{proposition}
\label{no-retraction}
The map $\mathbb{S}^1\to \mathbb{D}^2$ has no retraction.
\end{proposition}

\begin{proof}
By \cref{R-I-contractible} and \cref{shape-S1-is-BZ} we would get a retraction of $\B\Z\to 1$, so $\B\Z$ would be contractible.
\end{proof}

\begin{theorem}[Intermediate value theorem]
  \label{ivt}
  For any $f: \I\to \I$ and $y:\I$ such that $f(0)\leq y$ and $y\leq f(1)$,
  there exists $x:\I$ such that $f(x)=y$.
\end{theorem}

\begin{proof}
  By \Cref{InhabitedClosedSubSpaceClosedCHaus}, the proposition $\exists_{x:\I}\, f(x)=y$ is closed and therefore $\neg\neg$-stable, so we can proceed with a proof by contradiction.
  If there is no such $x:\I$, we have $f(x)\neq y$ for all $x:\I$.
  By \cref{LesserOpenPropAndApartness} we have that $a<b$ or $b<a$ for all distinct numbers $a,b:\I$. So the following two sets cover $\I$
  \[
    U_0:= \{x:\I\mid f(x)<y\} \quad\quad
    U_1:= \{x:\I\mid y<f(x)\}
    \]
  Since $U_0$ and $U_1$ are disjoint, we have $\I=U_0+U_1$ which allows us to define a non-constant function $\I\to 2$, which contradicts \Cref{Z-I-local}.
\end{proof}

\begin{theorem}[Brouwer's fixed-point theorem]
  For all $f:\mathbb{D}^2\to \mathbb{D}^2$ there exists $x:\mathbb{D}^2$ such that $f(x)=x$.
\end{theorem}

\begin{proof}
  As above, by \Cref{InhabitedClosedSubSpaceClosedCHaus}, we can proceed with a proof by contradiction,
  so we assume $f(x)\neq x$ for all $x:\mathbb{D}^2$.
  For any $x:\mathbb{D}^2$ we set $d_x= x-f(x)$, so we have that one of the coordinates of $d_x$ is invertible.
  Let $H_x(t) = f(x) + t\cdot d_x $ be the line through $x$ and $f(x)$.
  The intersections of $H_x$ and $\partial\mathbb{D}^2=\mathbb{S}^1$ are given by the solutions of an equation quadratic in $t$. By invertibility of one of the coordinates of $d_x$, there is exactly one solution with $t> 0$.
  We denote this intersection by $r(x)$ and the resulting map $r:\mathbb D^2\to\mathbb S^1$ has the property that it preserves $\mathbb{S}^1$.
  Then $r$ is a retraction from $\mathbb{D}^2$ onto its boundary $\mathbb{S}^1$, which is a contradiction by \Cref{no-retraction}.
\end{proof}

\begin{remark}
In constructive reverse mathematics \cite{HannesDiener}, it is known that both the intermediate value theorem and Brouwer's fixed-point theorem are equivalent to LLPO. But LLPO does not hold in real cohesive homotopy type theory, so \cite{shulman-Brouwer-fixed-point} prove a variant of the statement involving a double negation.
\end{remark}


%%
%% Bibliography
%%

%% Please use bibtex, 

\bibliography{literature.bib}

\appendix

\end{document}
