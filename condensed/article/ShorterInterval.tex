Since we have dependent choice, the unit interval $\mathbb I = [0,1]$ can be defined using 
Cauchy reals or Dedekind reals. 
We can freely use results from constructive analysis \cite{Bishop}. 
\begin{definition}
  We define $cs:2^N \to \mathbb I$ as 
  $cs(\alpha) = \sum_{n:\N} \frac{\alpha(i)}{2^{i+1}}$. 
\end{definition}

\begin{theorem}
  $I$ is compact Hausdorff
\end{theorem}
\begin{proof}
  By LLPO, $cs$ is surjective.   
  Note that $cs(\alpha) = cs(\beta)$ iff 
  $$|\sum_{n=0}^{n-1} \frac{\alpha(i)}{2^{i+1}}-
  \sum_{n=0}^{n-1} \frac{\beta(i)}{2^{i+1}}|\leq \frac{1}{2^n}$$
  for all $n:\N$, which is a countable conjunction of decidable propositions as 
  inequality in $\mathbb Q$ is decidable. 
\end{proof}

As we have $\neg$ WLPO, MP and LLPO, we can use the results from 
constructive reverse mathematics that follow \cite{ReverseMathsBishop, HannesDiener}. 
