The language of homotopy type theory is a  dependent type theory enriched with the univalence axiom and higher inductive types. It has proven exceptionnally well-suited to
develop homotopy theory in a synthetic way \cite{hott}. It also provides
the precision needed to analyze categorical models of type theory \cite{vanderweide2024}.
Moreover, the arguments in this language can be rather directly represented in proof assistants. We use homotopy type theory to give a synthetic development of topology, which is analogous to the work on synthetic algebraic geometry \cite{draft}. 

We introduce 
four axioms which seem sufficient for expressing and proving basic notions of topology, based on the light condensed
sets approach, introduced in \cite{Scholze}.
Interestingly, this development establishes strong connections with constructive mathematics \cite{Bishop},
particularly constructive reverse mathematics \cite{ReverseMathsBishop,HannesDiener}. Several of Brouwer's principles, such that
any real function on the unit interval is continuous, or the celebrated fan theorem, are consequences of this system
of axioms. Furthermore, we can also prove principles that are not intuitionistically valid, such as Markov's Principle,
or even the so-called Lesser Limited Principle of Omniscience, a principle well studied in constructive reverse mathematics,
which is {\em not} valid effectively.

This development also aligns closely with the program of Synthetic
Topology \cite{SyntheticTopologyEscardo,SyntheticTopologyLesnik,abstractstone}:
there is a dominance of open propositions, providing any type with an intrinsic
topology, and we capture in this way synthetically the notion of (second-countable) compact Hausdorff spaces.
While working on this axiom system, we learnt about the related work \cite{bc24}, which provides a different axiomatisation
at the set level. We show that some of their axioms are consequences of our axiom system. In particular, we can introduce
in our setting a notion of ``Overtly Discrete'' spaces, dual in some way to the notion of compact Hausforff spaces, like
in Synthetic Topology\footnote{We actually have a
derivation of their ``directed univalence'', but this will be presented in a following paper.}.

A central theme of homotopy type theory is that the notion of {\em type} is more general than the notion of {\em set}. We illustrate
this theme here as well: we can form in our setting the types of Stone spaces and of compact Hausdorff spaces
(types which don't form a set but a groupoid),
and show these types are
closed under sigma types. It would be impossible to formulate such properties in the setting of simple type or set theory.
Additionally, leveraging the elegant definition of cohomology groups in homotopy type theory \cite{hott}, which relies
on higher types that are not sets, we prove, in a purely axiomatic way,
a special case of a theorem of Dyckhoff \cite{dyckhoff76}, describing
the cohomology of compact Hausdorff spaces. This characterisation also supports a type-theoretic proof of
Brouwer's fixed point theorem, similar to the proof in \cite{shulman-Brouwer-fixed-point}. In our setting the theorem can be formulated in the usual way, and not in an approximated form.

We expect our axioms to be validated by the interpretation of homotopy type theory into the higher topos of light condensed sets \cite{shulman2019all}, although checking this rigorously is still work in progress. We even expect this to be valid in a constructive metatheory
using the work \cite{CRS21}. It is important to stress that what we capture in this axiomatic way are the properties of light condensed
sets that are {\em internally} valid. David W\"arn \cite{warn2024} has proved that an important property of abelian
groups in the setting of light condensed sets, is {\em not} valid internally and thus cannot be proved in this axiomatic context.
We believe however that our axiom system can be convenient for proving the results that are internally valid, as we hope
is illustrated by the present paper. We also conjecture that the present axiom system is actually {\em complete}
for the properties that are internally valid.
