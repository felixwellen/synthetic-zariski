\begin{abstract}
In synthetic algebraic geometry (SAG) \cite{draft}, we study finitely presented algebras over a commutative ring. 
In this work, we study countably presented Boolean algebras instead. 
Where the finitely presented algebras over a commutative ring induce a Zariski topos, 
%the opposite category of these 
the countably presented Boolean algebras induce the topos of light condensed sets \cite{TODO}. 
\cite{draft} proposes an axiomatization of the Zariski topos in univalent homotopy type theory \cite{HoTT}. 
In this work, we propose a similar axiomatization for light condensed sets. 
% Furthermore, spectra of countably presented Boolean algebras correspond to quotients of Cantor space
% which is cool because reasons
\end{abstract} 

\begin{definition}
  A countably presented Boolean algebra $B$ is a Boolean algebra such that there mererly are countable sets $I,J$, 
  a set of generators $(g_i)_{i\in I}$ and a set $(f_j)_{j\in J}$ of Boolean expressions over these generators 
  such that $B$ is equivalent to the quotient of the free Boolean algebra over the generators by the relations
  $(f_j=0)_{j\in J}$. 
\end{definition} 


In SAG, we deal with a fixed commutative ring $R$. For this project, the role of $R$ is taken over by 
the Boolean algebra $2 = 1+1$. Note that we don't need to postulate an alternative for the \textbf{Loc} axiom. 


\begin{definition}
  For $B$ a countably presented Boolean algebra, we define $Sp(B)$ as the set of Boolean morphisms from $B$ to $2$. 
%  We assign a pointwise Boolean algebra structure on $Sp(B)$. 
  %This is not necessarily a Boolean algebra, structure, consider the free BA on 1 generator (the diamond).
  %The maps send the generator to either 0 or 1, but the pointwise conjunction of these maps sends both the generator 
  % and it's negation to 0, which shouldn't happen.
\end{definition}

\begin{axiom}[Stone duality]
  For any Boolean algebra $B$, there
  is an isomorphism of Boolean algebras
  $$B\simeq  2^{Sp(B)}.$$ 
\end{axiom} 
\begin{definition}
  A type $X$ is called \textit{Stone} if it comes equiped with a Boolean algebra $B$ such that
  $X = Sp(B)$. 
\end{definition}
\begin{remark}
  By Stone duality, if $Sp(B) \simeq  Sp(B')$, then $$B \simeq 2^{Sp(B)} \simeq 2^{Sp(B')} \simeq B'.$$
  Furthermore, for any Boolean algebra $B$, we have that the composite
  $$ 
  \sum\limits_{B' : BA} Sp(B') = Sp(B) \to 
  \sum\limits_{B' : BA} 2^{Sp(B')} = 2^{Sp(B)} \to
  \sum\limits_{B' : BA} B' = B 
  $$ 
  is an equivalence. As the latter type is contractible, being Stone is a proposition. 
\end{remark} 

\begin{axiom}[Local choice]
  Whenever $X$ Stone and $E\twoheadrightarrow X$ surjective, then there is some $Y$ Stone,
    a surjection $Y \twoheadrightarrow X$ and a map $Y\to E$ such that the following diagram commutes:
    \begin{equation}\begin{tikzcd}
      E \arrow[d,""',two heads]\\
      X & \arrow[l, "", two heads] Y\arrow[lu, ""']
    \end{tikzcd}\end{equation}  
\end{axiom} 

\begin{axiom}
  A map $f:Sp(B)\to Sp(B')$ is surjective iff the corresponding map $B' \to B$ is injective.
\end{axiom} 

