\subsection{Examples of open and closed propositions}
\begin{lemma}\label{BooleEqualityOpen}
  Whenever $B:\Boole$, $a,b:B$ the proposition $a=_Bb$ is open. 
\end{lemma}
\begin{proof}
  Let $G,R$ be the generators and relations of $B$. 
  Let $a,b$ be represented by $x,y$ in the free Boolean algebra on $G$. 
  Now let $R_n$ denote the first $n$ elements of $R$. 
  Note that $a=b$ iff there exists some $n:\N$ with 
  $x-y \leq \bigvee_{r\in R_n} r$. 
  Furthermore, inequality is decidable in the free Boolean algebra, hence
  $a=b$ is a countable disjunction of decidable propositions, hence open. 
\end{proof}


\begin{corollary}\label{StoneEqualityClosed}
  Whenever $S:\Stone$, and $s,t:S$, the proposition $s=t$ is closed. 
\end{corollary}
\begin{proof}
  Suppose $S= Sp(B)$ and let $G,R$ be the generators and relations of $B$. 
  Note that $x=y$ iff $x(g) =_2 y(g)$ for all $g:G$. 
  As $G$ is countable, and equality in $2$ is decidable, 
  $x=y$ is a countable conjunction of decidable propositions, hence closed. 
\end{proof}

