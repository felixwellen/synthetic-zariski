\subsection{$\Open$ and $\ODisc$} %Open propositions are Overtly discrete and Stone propositions.}
\begin{lemma}
  Whenever $P$ is a proposition and overtly discrete, $P$ is open. 
\end{lemma}
\begin{proof}
  If $P$ is overtly discrete, then $P\leftrightarrow \exists_{n:\N} P_n$. 
  As every $P_n$ is finite, it is decidable. 
  Hence $P$ is a countable disjunction of decidable propositions, hence open.% by \Cref{OpenCountableDisjunction}. 
\end{proof}
\begin{lemma}
  Whenever $P$ is a an open proposition, it is overtly discrete.
\end{lemma}
\begin{proof}
  Suppose $P\leftrightarrow \exists_{n:\N} \alpha_n = 1$. 
  Let $P_n = \exists_{k\leq n} (\alpha_k = 1)$, which is a decidable proposition, hence a finite set. 
  Then the colimit of $P_n$ is $P$. 
\end{proof} 
\begin{corollary}\label{PropOpenIffOdisc}
  A proposition is open iff it is overtly discrete.
\end{corollary}
\begin{proof}
  Immediate by the above two lemmas. 
\end{proof}
\begin{corollary}\label{OpenDependentSums}
  Open propositions are closed under dependent sums. 
\end{corollary}
\begin{proof}
  Immediate from \Cref{OdiscSigma} and \Cref{PropOpenIffOdisc}.
\end{proof}
\begin{remark}
  Note that the sequential colimit commutes with the propositional truncation, thus for $B:\ODisc$, we have 
  $||B||:\ODisc$. 
\end{remark}
%\begin{corollary}
%  Open propositions are closed under countable disjunctions. 
%\end{corollary}
%\begin{proof}
%  Clearly $\N:\ODisc$, and for $P:\N \to \Open$, and by the above 
%  $||\Sigma_{n:\N} P_n||:\ODisc$. 
%\end{proof} 

%\begin{corollary}\label{OpenFiniteConjunction}
%  Open propositions are closed under finite conjunctions. 
%\end{corollary}
%\begin{proof}
%  A conjunction of propositions is a product, which is a dependent sum. 
%\end{proof}
%FollowsAlsoFromNextLemma%\begin{lemma}\label{ODiscEqualityOpen}
%FollowsAlsoFromNextLemma%  Whenever $B$ is overtly discrete and $a,b:B$, the proposition $a=_B b$ is open. 
%FollowsAlsoFromNextLemma%\end{lemma}
%FollowsAlsoFromNextLemma%\begin{proof}
%FollowsAlsoFromNextLemma%  For $a,b:B$ there is some $n:\N$ with $a',b':B_n$ and $\iota_n(a') = a,\iota_n(b') = b$.
%FollowsAlsoFromNextLemma%  By \Cref{rmkEqualityColimit}, we have that $a=_B b$ iff 
%FollowsAlsoFromNextLemma%  there is some $m\geq n$ with $\iota_n^m (a') = \iota_n^m(b')$. 
%FollowsAlsoFromNextLemma%  As equality in finite sets is decidable, this is a countable disjunction of decidable propositions, hence open. 
%FollowsAlsoFromNextLemma%%  \rednote{That reference can also be a direct cite}.
%FollowsAlsoFromNextLemma%\end{proof}

\begin{lemma}
  A type $B$ is overtly discrete iff it is the quotient of a countable set by an open equivalence relation. 
\end{lemma}
\begin{proof}
  Is $B:\ODisc$, then $B= (\Sigma_{n:\N} B_n)/\sim_B$ where $\sim_B$ is the reflexive closure of  
  $(n,b)\sim(m,\iota_n^m b)$ for $n\leq m$. 
%
  Conversely, assume $B= D/R$ with $D\subseteq \N$ decidable and $R$ open. 
  By countable choice we get $\alpha_{(\cdot,\cdot)}:D \to D \to 2^\N$ such that 
  $R(x,y)\leftrightarrow \exists_{n:\N}\alpha(n) = 1$. 
  Define $D_n = (D \cap \N_{\leq n})$, and $R_n : D_n \to D_n \to 2$ so that $R_n(x,y)$ checks whether
  $\alpha_{(x,y)}(k) =1$ for some $k\leq n$. 
  Note that $B_n = D_n/R_n$ is a finite set, and has colimit $B$. 
\end{proof}
\begin{remark}\label{BooleEpiMono}
  In particular equality in overtly discrete types is open. 
  By \Cref{OdiscSigma}, it follows that any $g:B\to C$ in $\Boole$ has an overtly discrete kernel, 
  which must therefore be enumerable. Hence $B/Ker(g)$ is in $\Boole$. 
  By uniqueness of epi-mono factorizations and \Cref{SurjectionsAreFormalSurjections}, the factorization 
  $B\twoheadrightarrow B/Ker(g) \hookrightarrow C$ corresponds to 
  $Sp(C) \twoheadrightarrow Sp(B/Ker(g)) \hookrightarrow Sp(B)$. 
\end{remark}
%\begin{lemma}
%  A Boolean algebra $B$ is overtly discrete iff it is countably presented. 
%\end{lemma}
%\begin{proof}
%  In \Cref{BooleIsODisc}, we saw that countably presented Boolean algebras are overtly discrete. 
%  For the converse, assume $B:\ODisc$ and consider $2[\Sigma_{n:\N} B_n]/( r_i, s_j)_{i,j:\N}$, where 
%  $r_i ($ 
%\end{proof}





\subsection{$\Closed$ and $\Stone$}
%\begin{lemma}\label{BooleEqualityOpen}
%  Whenever $B:\Boole$, $a,b:B$ the proposition $a=_Bb$ is open. 
%\end{lemma}
%\begin{proof}
%  Let $G,R$ be the generators and relations of $B$. 
%  Let $a,b$ be represented by $x,y$ in the free Boolean algebra on $G$. 
%  Now let $R_n$ denote the first $n$ elements of $R$. 
%  Note that $a=b$ iff there exists some $n:\N$ with $x-y \leq \bigvee_{r\in R_n} r$. 
%  Furthermore, inequality is decidable in the free Boolean algebra, hence
%  $a=b$ is a countable disjunction of decidable propositions, hence open. 
%\end{proof}

\begin{corollary}\label{TruncationStoneClosed}
  Whenever $S:\Stone$, $||S||$ is closed. 
\end{corollary}
\begin{proof}
  By \Cref{SpectrumEmptyIff01Equal}, $\neg S$ is equivalent to $0=_B 1$, which is open by the above. 
  Hence $\neg \neg S$ is a closed proposition, and by \Cref{LemSurjectionsFormalToCompleteness}, so is $||S||$. 
\end{proof}
%\begin{remark}\label{ExplicitTruncationStoneClosed}
%  \rednote{New check later}
%  The above lemma and corollary actually show that if we have an explicit 
%  presentation of a Stone space as $S = Sp(2[G] / R)$, 
%  we can construct an explicit sequence $\alpha:2^\N$ such that $||S|| \leftrightarrow \forall_{n:\N} \alpha(n) = 0$. 
%\end{remark}


\begin{corollary}\label{PropositionsClosedIffStone}
  A proposition $P$ is closed iff it is a Stone space. 
\end{corollary}
\begin{proof}
  By the above, if $S$ is both a Stone space and a proposition, it is closed. 
  By \Cref{ClosedPropAsSpectrum}, any closed proposition is Stone. 
\end{proof}

\begin{lemma}\label{StoneEqualityClosed}
  Whenever $S:\Stone$, and $s,t:S$, the proposition $s=t$ is closed. 
\end{lemma}
\begin{proof}
  Suppose $S= Sp(B)$ and let $G$ be the generators of $B$. 
  Note that $s=t$ iff $s(g) =_2 t(g)$ for all $g:G$. 
  As $G$ is countable, and equality in $2$ is decidable, 
  $s=t$ is a countable conjunction of decidable propositions, hence 
  closed. 
\end{proof}
%
The following question was asked by Bas Spitters at TYPES 2024:
\begin{corollary}
  For $S:\Stone$ and $x,y,z:S$ 
  \begin{equation}\label{Apartness}
  x \neq y \to (x\neq z \vee y \neq z)
  \end{equation}
\end{corollary}
\begin{proof}
  As $x\neq y$, we can show that $\neg ( x = z \wedge y = z)$. 
  This in turn implies $\neg \neg ( x \neq  z \vee y \neq  z)$. 
  As, $x\neq z$ and $y \neq z$ are both open propositions, by \Cref{OpenCountableDisjunction} so is their disjunction. 
  By \Cref{rmkOpenClosedNegation}, that disjunction is double negation stable and \Cref{Apartness} follows. 
\end{proof}
\begin{remark}
  If \Cref{Apartness} holds in a type, we say that it's inequality is an apartness relation. 
  By a similar proof as above, it can be shown that in our setting inequality is an apartness relation 
  as soon as equality is open or closed. 
\end{remark}
