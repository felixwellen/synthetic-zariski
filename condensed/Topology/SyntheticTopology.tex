\subsection{Types as spaces}
The subobject $\Open$ of the type of propositions induces a topology on every type. 
This is the viewpoint taken in synthetic topology. 
We will follow the terminology of \cite{SyntheticTopologyLesnik}, 
other references include \cite{SyntheticTopologyEscardo, TODOSortOutTaylorsReferences}.
%Defining a topology in this way has some benefits, which we summarize in this section. 

\begin{definition}
  Let $T$ be a type, and let $A\subseteq T$ be a subtype. 
  We call $A\subseteq T$ open or closed iff $A(t)$ is open or closed respectively for all $t:T$.
\end{definition}

\begin{remark}
  It follows immediately that the pre-image of an open by any map of types sends is open, so that any map is continuous. 
  This is only relevant for a space if the topology we defined above matches the topology one would expect. 
  In \Cref{StoneClosedSubsets}, we shall see that it resembles the standard topology of Stone spaces.
  In \Cref{IntervalClosedSubsets}, we shall see that it is the standard topology for the unit interval. 
\end{remark}

\begin{lemma}[transitivity of openness]\label{OpenTransitive}
  Let $T$ be a type, let $V\subseteq T$ open and let $W\subseteq V$ open. 
  Then the composite $W\subseteq V\subseteq T$ is open as well. 
\end{lemma}
\begin{proof}
  Denote $W'\subseteq T$ for the composite. 
  Note that $W'(t) = \Sigma_{v:V(t)} W(t,v)$. 
  As open propositions are closed under dependent sums (\Cref{OpenDependentSums}), 
  $W'(t)$ is an open proposition, as required. 
\end{proof}

\begin{remark}\label{OpenDominance}
  As the true proposition is open and openness is transitive, 
  $\Open$ can be called a dominance according to Proposition 2.25 of \cite{SyntheticTopologyLesnik}
\end{remark}



%\begin{remark}
%  Phao's principle is a special case of directed univalence. 
%\end{remark}
%\begin{proof}
%  \rednote{TODO}
%\end{proof}
