\subsection{Principles of omniscience}
In constructive mathematics, we do not have general access to the law of excluded middle (LEM).
There are some principles that are weaker than LEM, which can be used to describe 
the proof-theoretic strength of a logical system, called principles of omniscience.
In this section, we will show that two of them (Markov's principle and LLPO) hold, 
and one (WLPO) fails in our system.

\begin{theorem}[The negation of the weak lesser principle of omniscience ($\neg$WLPO)]
  It is not the case that the statement %There is no method which given $\alpha:2^\mathbb N$ decides whether 
  $\forall_{n:\mathbb N} \alpha(n) = 0$ is decidable for general $\alpha:2^\mathbb N$. 
\end{theorem}
\begin{proof}
  Such a decission method would give a function $f:2^\mathbb N \to 2$ such that 
  $f(\alpha) = 0$ iff $\forall_{n:\mathbb N} \alpha (n)= 0$. 
  By Stone duality, there must be some $c:C$ with 
  $f(\alpha) = 0 \iff \alpha(c) = 0$. 
  $c$ is expressable using only finitely many generators $(p_n)_{n\leq N}$. 
  Now consider $\beta,\gamma:\langle C \rangle \to 2$ given by $\beta(p_n) = 0$ for all $n:\mathbb N$ and
  $\gamma(p_n) = 0$ iff $n\leq N$. 
  Note that these functions are equal on $(p_n)_{n\leq N}$, therefore, $\beta(c) = \gamma(c)$. 
  However, $f(\beta) = 0$ and $f(\gamma) = 1$.
  We thus have a contradiction, thus a decission method as required doesn't exist. 
\end{proof}

The following result is due to David W\"arn:
\begin{theorem}[Markov's principle (MP)]\label{MarkovPrincipleThm}
  For $\alpha:\Noo$, we have that 
  \begin{equation}
    (\neg (\forall_{n:\mathbb N} \alpha (n)= 0)) \to \Sigma_{n:\mathbb N} \alpha (n)= 1
  \end{equation}
\end{theorem}
\begin{proof}
  Assume $\neg (\forall_{n:\mathbb N} \alpha (n)= 0)$.
%  Suppose $\alpha \neq \infty$.%, then it is not the case that $\alpha(n) = 0$ for all $n:\N$. 
  It is sufficient to show that $2/\{\alpha(n)|n\in\N\}$ is the trivial Boolean algebra. 
  It will then follow that there is a finite subset $N_0\subseteq \N$ 
  with $\bigvee_{i:N_0} \alpha(i) = 1$.
  As $\alpha(i) \in \{0,1\}$ and $\alpha(i) = 1$ for at most one $i$, it then follows that 
  there exists an unique $n\in\mathbb N$ with $\alpha(n) = 1$. 

  To show that $2/\{\alpha(n)|n\in\N\}$ is trivial, we will show it has an empty spectrum. 
  Suppose $x: 2 \to 2$ is such that $x(\alpha(n)) = 0$ for every $n:\N$. 
  As $x(1) = 1\neq_2 0$, we must have for every $n:\N$ that $\alpha(n) \neq 1$. 
  But then $\alpha(n) = 0$, contradicting our assumption. 
  We get a contradicition and there thus there are no terms of the spectrum of $2/\{\alpha(n)|n\in\N\}$ as required. 
\end{proof}
\begin{corollary}
  For $\alpha:2^\mathbb N$, we have that 
  \begin{equation}
    (\neg (\forall_{n:\mathbb N} \alpha (n)= 0)) \to \Sigma_{n:\mathbb N} \alpha (n)= 1
  \end{equation}
\end{corollary}
\begin{proof}
  Given $\alpha:2^\mathbb N$, consider the sequence $\alpha':\Noo$ satisfying $\alpha'(n) = 1$ iff 
  $n$ is minimal with $\alpha(n) = 1$. Then apply the above theorem.
\end{proof}

\begin{theorem}[The lesser limited principle of omniscience (LLPO)]\label{LLPOThm}
  For $\alpha:\N_\infty$, 
  we have that 
  \begin{equation}\label{eqnLLPO}
    \forall_{k:\N} \alpha(2k) = 0  \vee \forall_{k:\N} \alpha(2k+1) = 0
  \end{equation}
\end{theorem}
\begin{proof}
%
%  We first will define a map $f:B_\infty \to B_\infty \times B_\infty$. 
%  Because of \Cref{rmkMorphismsOutOfQuotient}, it is sufficient to define $f$ on $(p_n)_{n:\N}$ with 
%  $f(p_n) \wedge f(p_m) = (0,0)$ for $n\neq m$. 
%  To define $f(p_n)$, we use a case distinction on whether $n$ is odd or even. 
  Define $f:B_\infty \to B_\infty \times B_\infty$ as follows:
  \begin{equation}
    f(p_n) =\begin{cases}
      (p_k,0) \text{ if } n = 2k\\
      (0,p_k) \text{ if } n = 2k+1\\
    \end{cases}
  \end{equation}
  By making a case distinction on $n,m$ being odd or even, 
  we can see that 
  $f(p_n) \wedge f(p_m) = (0,0)$ when $n\neq m$, thus $f$ is well-defined. 
  We also claim it is injective.
  Now let $x:B_\infty$ with $f(x) = 0$. 
  We denote $E,O\subseteq \N$ for the even and odd numbers, 
  and we make a case distincition based on \Cref{BinftyTermsWriting}.
  \begin{itemize}
    \item Suppose 
      $x = \bigvee_{i\in I_0} p_i$. 
      Then 
      $$f(x) = (\bigvee_{i\in I_0 \cap E } p_{\frac i2} , \bigvee_{i\in I_0 \cap O } p_{\frac {i-1}2} ) = (0,0)$$
      But now as $p_j\neq 0$ for all $j\in\N$, we must have $I_0 \cap E = \emptyset = I_0 \cap O$. 
      Thus $I_0= \emptyset$, and $x = 0$. 
    \item Suppose 
%      Let $x$ correspond to a cofinite subset of $\N$. Write 
      $x = \bigwedge_{j\in J} \neg p_j$. % for $J$ finite. 
      We will derive a contradiction. %, from which we can conclude that $x=0$ after all. 
      Note that   
      $$f(x) = (\bigwedge_{j\in J \cap E } \neg p_j , \bigwedge_{j\in J \cap O } \neg p_j )$$
      As $f(x) = (0,0)$, we have that 
      $\bigwedge_{j\in J \cap E } \neg p_j =0$ and
      $\bigwedge_{j\in J \cap O } \neg p_j  = 0$.
      However, any finite meet of negations will correspond to a cofinite set,
      in particular it will not correspond to the empty set, and thus not be $0$.
      Thus $f(x)\neq 0$, contradicting the assumption that $f(x) = 0$, hence $x=0$ ex falso. 
  \end{itemize}
  In both cases, we conclude $x=0$, thus $f$ is injective. 
  By \Cref{FormalSurjectionsAreSurjections}, $f$ corresponds to a surjection 
  $s:\Noo + \Noo \to \Noo$.
  Now let $\alpha : \Noo$, 
  let $x:\Noo + \Noo$ be such that $s x = \alpha$. 
  If $x = inl(\beta)$, 
  for any $k:\N$, we have that 
  $$\alpha (p_{2k+1}) = s(x) (p_{2k+1}) = x(f(p_{2k+1})) = inl(\beta) (0,p_k)  = \beta(0) = 0.$$
  Similarly, if $x = inr(\beta)$, we have $\alpha(p_{2k}) = 0$ for all $k:\N$. 
  Thus $\Cref{eqnLLPO}$ holds for $\alpha$ as required. 
\end{proof}

The use of \Cref{FormalSurjectionsAreSurjections}, and hence of propositional completeness, 
was not trivial in the above proof, as the following lemma shows:
\begin{lemma}
  The above function $f$ does not have a retraction. 
\end{lemma}
\begin{proof}
  Suppose $r:B_\infty \times B_\infty \to B_\infty$ is a retraction of $f$. 
  Note that $r(0,1):B_\infty$ is expressable using only finitely many generators $(p_n)_{n\leq N}$
  Note that $r(0,1) \geq r(0,p_k) = p_{2k+1}$ for all $k:\N$. 
  As a consequence, $r(0,1)$ cannot be of the form $\bigvee_{i\in I_0} p_i$, and by \Cref{BinftyTermsWriting}, 
  $r(0,1)$ corresponds to a cofinite subset of $\N$. % = \bigwedge_{i:I_0} \neg p_i$, where $i\leq N$ for $i\in I_0$. 
  By similar reasoning so does $r(1,0)$.% corresponds to a cofinite subset of $\N$. 
  But the intersection of cofinite subsets is cofinite, while 
  $$r(0,1) \wedge r(1,0) = r( (1,0) \wedge (0,1)) = r(0,0) = 0$$
  which gives a contradiction. Thus no retraction exists. 
\end{proof}

We finish with an euivalent formulation of LLPO:
\begin{lemma}\label{corAlternativeLLPO}
  Let $(\phi_n)_{n:\N}, (\psi_m)_{m:\N}$ be families of decidable propositions indexed over $\N$.
  We then have 
  \begin{equation}
    (\forall_{n:\N} \forall_{m:\N} (\phi_n \vee \psi_m) )
    \leftrightarrow
    ((\forall_{n:\N} \phi_n) \vee (\forall_{m:\N} \psi_m) )
  \end{equation}
\end{lemma}
Note that the above statement implies LLPO as 
$\alpha(2n) =0 \vee \alpha(2m+1) =0$ for all $n,m:\mathbb N$ if $\alpha:\Noo$. 
\begin{proof}
  Note that the implication from right to left in the above equation always holds.
%  \rednote{While \cite{ReverseMathsBishop} does mention LLPO, it doesn't mention this specific equivalence, 
%  and there should be a reference for this result}
  Assume $\forall_{n:\mathbb N} \forall_{m:\mathbb N} (\phi_n \vee \psi_m)$. 
  Consider the sequence $\alpha:2^\mathbb N$ where $\alpha(2n) = 0$ iff $\phi_n$ and 
  $\alpha(2m+1) = 0$ iff $\psi_m$. 
  Let $\\beta:\Noo$ be such that $\beta(i) = 1$ iff $i$ is minimal with $\alpha(i) = 1$
%
%  Assume that for every $n,m:\N$ we have $\phi_n \vee \psi_m$. 
%  We will define a binary sequence $\beta$ and show that $\beta(n)$ is $1$ at most once. 
%  First we define a binary sequence $\alpha$
%  such that $\alpha(2n) = 0$ iff $\phi_n$ holds and $\alpha(2m+1) = 0$ iff $\psi_m$ holds. 
%  We let $\beta(k) = 1$ iff $k$ is minimal with $\alpha(k) = 1$. 
%  By this minimality, we clearly have that $\beta$ is $1$ at most once and thus defines a term of $\Noo$. 
  By LLPO, we have that 
  $\forall_{k:\N} \beta(2k+1) = 0\vee \forall_{k:\N} \beta(2k+1) = 0$. 
%  As we're proving a proposition, we can unpack this truncation, and make a case distinction on
%  $\forall_{k:\N} \beta(2k+1) = 0 + \forall_{k:\N} \beta(2k+1) = 0$. 
  WLOG assume that 
  $\forall_{k:\N} \beta(2k+1) = 0$. We will show $\forall_{m:\mathbb N} \psi_m$. 
  Let $m:\N$
%  Let now $m:\N$. We claim that $\psi_m$ will hold. 
  Suppose $\neg \psi_m$, then $\alpha(2m+1) = 1$. 
  As $\beta(2m+1) = 0$, there is some $l<2m+1$ with $\alpha(l) = 1$. 
%  $k = 2m+1$ is not minimal such that $\alpha(k)  = 1$.
%  There is thus some  some $l\leq 2k+1$ with $\alpha(l) = 1$. 
  As $\beta(2k+1) = 0$ for all $k$, we have that $l$ must be even. 
  So $\alpha(2n) = 1$ for $n = \frac{l}2$, meaning that $\neg \phi_n$. 
  But now $\neg \phi_n \wedge \neg \psi_m$, which contradicts 
  $(\forall_{n:\N} \forall_{m:\N} (\phi_n \vee \psi_m) )$. 
  We get a contradiction, and as $\psi_m$ is decidable, we conclude $\psi_m$. 
  Thus for all $m:\N$, we have $\psi_m$. Thus 
  $((\forall_{n:\N} \phi_n) \vee (\forall_{m:\N} \psi_m) )$ 
  as required. 
%
  We conclude that 
  \begin{equation}
    (\forall_{n:\N} \forall_{m:\N} (\phi_n \vee \psi_m) )
    \to
    ((\forall_{n:\N} \phi_n) \vee (\forall_{m:\N} \psi_m) )
  \end{equation}
\end{proof}
