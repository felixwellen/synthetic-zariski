%\input{OvertlyDiscrete/ColimitRepresentation.tex}
\begin{definition}
  We call a type overtly discrete iff it can be described as 
  the colimit of an $(\N,\leq)$-indexed sequence of finite sets. 
\end{definition} 
\begin{remark}
  If we denote an overtly discrete type by $B$, we will denote the objects of the underlying sequence as 
  $B_n,~n:\N$, and for $n\leq m$, we denote the maps $B_n \to B_m$ with 
  Greek letters with lower index $n$ and upper index $m$. 
  So for example $\iota_n^m:B_n \to B_m$. 
  The maps $B_n \to B$ will in this case be denoted $\iota_n$. 
  If convenient, given a sequence $B_n$, we will denote $B_\infty$ for the colimit $B$.
\end{remark}
\begin{lemma}
  Every countably presented Boolean algebra is overtly discrete.
\end{lemma}
\begin{proof}
  Consider a countably presented Boolean algebra of the form $B = 2[\N]/(r_n)_{n:\N}$. 
%  We will show there exists a diagram of shape $\N$ taking values in Boolean algebras 
%  with $B$ as colimit.
%  \paragraph{The diagram}
  For each $n:\N$, let $G_n$ be the union of $\{g_i|{i\leq n}\}$ and 
  the finite set of terms occurring in $(r_i)_{i\leq n}$. 
  Denote $B_n = 2[G_n]/(r_i)_{i\leq n}$, and note that each $B_n$ is a finite Boolean algebra, 
  and we have maps $B_n \to B_{n+1}$.
%  The inclusion $G_n \hookrightarrow G_{n+1}$ induces maps $B_n \to B_{n+1}$.
  Hence $B_n,~n:\N$ is an $(\N,\leq)$-indexed sequence of finite sets. 
  We claim that $B$ is the colimit of this sequence. 
%
%  \paragraph{The colimit}
%  As $G_n\subseteq G$ and $R_n \subseteq R$, 
%  \Cref{rmkMorphismsOutOfQuotient} also gives us a map $B_n\to \langle G \rangle \langle R \rangle$. 
%  We claim the resulting cocone is a colimit. 
%
%  Suppose we have a cocone $C$ on the diagram $(B_n)_{n\in\N}$. 
%  We need to show that there exists a map $\langle G \rangle / R\to C$ and
%  we need to show this map is unique as map between cocones. 
%  \begin{itemize}
%    \item To show there exists a map $\langle G \rangle / R \to C$, 
%      we use remark \Cref{rmkMorphismsOutOfQuotient} again. 
%      Let $g\in G$ be the $n$'th element of $G$, 
%      note that $g\in G_n$, and consider the image of $g$ under the map $B_n \to C$. 
%      This procedure defines a function from $G$ to the underlying set of $C$. 
%      Let $\phi \in R$ be the $n$'th element of $R$, 
%      note that $\phi \in R_n$, and the map $B_n \to C$ must send $\phi$ to $0$. 
%      Thus the function from $G$ to the underlying set of $C$ also sends $\phi$ to $0$. 
%      This thus defines a map $\langle G \rangle / R \to C$. 
%    \item To show uniqueness, consider that any map of cocones $\langle G \rangle / \langle R \rangle \to C$ 
%      must take the same values on all $g\in G_n$ for all $n\in\N$. 
%      Now all $g\in G$ occur in some $G_n$, so any map of cocones $\langle G \rangle /  \langle R \rangle \to C$ 
%      takes the same values for all $g\in G$. 
%      \Cref{rmkMorphismsOutOfQuotient} now tell us that these values uniquely determine the map. 
%  \end{itemize}
\end{proof}
 
\begin{lemma}\label{colimitCompact}
  For any finite set $A$ and $(\N,\leq)$-indexed sequence of finite sets $B_n$ with colimit $B$, 
  the colimit of $B_n^A$ is $B^A$. 
%  and any overtly discrete type $B$ as colimit of the sequence $(B_n)_{n:\N}$, 
%  $(colim(B_n))^A \simeq colim (B_n^A)$
%  $B^A$ is the colimit of the sequence of sets $(B_n^A)_{n:\N}$. 
\end{lemma}  
\begin{proof}
%  First note that $B^A$ forms a cocone on $(B_n^A)_{n:\N}$ 
  Any map $A \to B_n$ induces a map $A \to B$, hence $B^A$ is a cocone on $(B_n^A)_{n:\N}$.
  Let $C$ form a cocone on $(B_n^A)_{n:\N}$. %with maps $F_n:B_n^A \to C$.
  For any $f:A \to B$, the finite image $f(A)$ must already occur in some $B_n$, 
  thus there is some $f':A\to B_n$ with $\iota_n\circ f' = f$.% occurs as some map $A\to B_n$, 
  As $C$ is a cocone, $f'$ corresponds to some $c$ in $C$, and this term does not depend on $n$. 
  Also any map $B^A \to C$ respecting the cocone conditions must send $f$ to $c$, 
  hence $B^A$ is indeed the colimit. 
%
%
%
%%  We shall show there is an unique morphism of cocones $B^A \to C$. 
%%  Denote $\iota_n:B_n \to B, F_n:B_n^A\to C$ for the cocone maps. 
%  \begin{itemize}
%    \item 
%      If $f:A\to B$, we have that $f(A)$ is a finite subset of $B$, and thus occurs already in some $B_n$. 
%      This induces a map $f'_n:A\to B_n$ with $\iota_n\circ f'_n = f$. 
%      As $C$ is a cocone, we have that $F_n(f'_n)$ does not depend on $n:\N$. 
%%      Thus $(F_n)_{n:\N}$ induces a map 
%      Thus we get a map $B^A\to C$. 
%    \item 
%      For uniqueness, by function extensionality maps $B^A \to C$ are uniquely determined by their values on 
%      $f:B^A$. By the above, the value of $f$ is uniquely determined by it's value on $B_n$ for 
%      any $n$ with the image of $f$ in $B_n$. Thus there is at most one morphism of cocones $B^A \to C$. 
%  \end{itemize}
\end{proof}
\begin{remark}\label{rmkEqualityColimit}
  In the above proof, we used that any element $b\in B$ already occurs in some $B_n$. 
  However, please note that it is not necessarily the case that it occurs uniquely in $B_n$.
%  there might be multiple elements in $B_n$ which can all be sent to $b$ in the end. 
%
  In general, if we assume $B$ is overtly discrete %is the colimit of any sequence $(B_n)_{n:\N}$, 
  and there exist some $B_n$ with two elements corresponding to the same element in $B$, 
  Theorem 7.4 from \cite{SequentialColimitHoTT} says that there merely exists some $m\geq n$
  such that these elements become equal in $B_m$. 
%  In general, if we have a sequence $(B_n)_{n:\N}$ with colimit $B$ and 
%  inclusion maps $\iota_n:B_n\to B$ and $\iota_n^m:B_n \to B_m$ for $n\leq m$, 
%  and there exist some $b,c:B_n$ with $\iota_n(b) = \iota_n(c)$, 
%  Theorem 7.4 from \cite{SequentialColimitHoTT} says that there merely exists some $m\geq n$
%  with $\iota_n^m(b) = \iota_n^m(c)$. 
\end{remark}
\begin{lemma}\label{lemDecompositionOfColimitMorphisms}
  Let $B,C$ be overtly discrete, 
  and let $f:B\to C$.
  There exists $(\N,\leq)$-indexed sequences of finite sets 
  $(B_n)_{n:\N}, (C_n)_{n:\N}$ with colimits $B,C$ respectively
  and compatible maps $f_n:B_n \to C_n$, 
  such that $f$ is the induced morphism $B\to C$.
\end{lemma}
\begin{proof}
  Let $(B_n)_{n:\N}, (C_n)_{n:\N}$ be 
  sequences of finite sets with colimits $B$ and $C$. 
  We will construct a subsequence of $(C_n)_{n:\N}$, using \Cref{axDependentChoice}.
%
  For $k:\N$, let $E_k$ consist of 
  strictly increasing sequences $(n_i)_{i<k}$ of natural numbers together with a finite family of maps 
  $(f_i: B_{i} \to C_{n_i})_{i<k}$ such that
  for all $0<i<k$ the following diagram commutes:
  \begin{equation}\label{eqnDecompositionOfColimitMorphisms}
    \begin{tikzcd}
      B_{i-1} \arrow[r,"\iota_{i-1}^i"] \arrow[d, "f_{i-1}"]& B_{{i}} \arrow[r, "\iota_i"] 
      \arrow[d,"f_{i}"]& B \arrow[d,"f"] \\
      C_{n_{i-1}} \arrow[r,"\kappa_{i-1}^i"] & C_{n_{i}} \arrow[r,"\kappa_i"] & C 
    \end{tikzcd}
  \end{equation}
  For $e:E_k, e':E_{k+1}$, we let 
  $R_k(e,e')$ denote  whether the underlying sequences of $e'$ extends that of $e$. 
  The empty sequence inhabits $E_0$. We will show that if $e:E_k$, there exists some $e':E_{k+1}$ with 
  $R_k(e,e')$. Then \Cref{axDependentChoice} will give the required sequence $(f_i:B_i\to C_{n_i})$.
%%
%  By \Cref{colimitCompact} 
%  there exists some $n_0:\N$ 
%  such that $B_0 \to B \to C$ factors as 
%  \begin{equation}
%    \begin{tikzcd}
%      B_{0} \arrow[r] \arrow[d, "f_0"]& B \arrow[d,"f"] \\
%      C_{n_0} \arrow[r] & C 
%    \end{tikzcd}
%  \end{equation}
%  Because our goal is a proposition, we can untruncate this existence to data. 
%  This data will form our $x_0:E_0$. %from \Cref{axDependentChoice}. 
%%

  Suppose we have $(f_i: B_{i} \to C_{n_i})_{i<k}$ for some $k\geq 0$ 
  such that for all $0<i<k$ the diagram of \Cref{eqnDecompositionOfColimitMorphisms} commutes.
  We shall show that in this case there exist $n_{k}:\N, f_{k}:B_k \to C_{n_k}$ 
  making the same diagram commute for $i = k$. 
  Consider the map $f\circ \iota_k: B_{k}\to C$. 
  As $B_k$ is finite, \Cref{colimitCompact} gives some $n_k':\N $ such that %, f_k':B_k \to C_{n_k'}$ such that 
  it factors over some $C_{n_k'}$.
%  \begin{equation}
%    \begin{tikzcd}
%    B_{k} \arrow[r,"\iota_k"]  \arrow[d,"f'_{k}"]& B \arrow[d,"f"]\\
%    C_{n'_{k}} \arrow[r, "\kappa_{n'_k}"'] & C
%    \end{tikzcd}
%  \end{equation}
  We may assume $n'_{k+1} > n_k$.
  Note that it is not necessarily the case that 
  $f'_{k} \circ \iota_{k-1}^k = \kappa_{n_{k-1}}^{n'_k}\circ f_{k-1}$. 
%  $f'_{k+1}$ is compatible with $f_k$, meaning the left square in the following diagram needn't commute:
%  \begin{equation}
%    \begin{tikzcd}
%      B_{k-1} \arrow[r] \arrow[d, "f_{k-1}"]& B_{{k}}  \arrow[r] \arrow[d,"f'_{k}"] & B \arrow[d,"f"] \\
%      C_{n_{k-1}} \arrow[r] & C_{n'_{k}} \arrow[r]  & C 
%    \end{tikzcd}
%  \end{equation}
  However, both $f'_{k}, f_{k-1}$ induce the same map $B_{k-1} \to C$. 
%  Recall by \Cref{rmkMorphismsOutOfQuotient} this map is induced by it's value on finitely many elements. 
  As $B_{k-1}$ is finite, from \Cref{rmkEqualityColimit} it follows there is some $n_{k} \geq {n'_{k}}$ 
  such that they become equal in $C_{n_k}$, and we have $f_k:B_k \to C_{n_k}$ such that the following does commute, 
  and we're done:
%  such that for $f_{k}$ the composition of $f'_{k+1}:B_{k+1} \to C_{n'_{k+1}}$ and 
%  the map $C_{n'_{k+1}} \to C_{n_{k+1}}$, the following diagram does commute:
  \begin{equation}
    \begin{tikzcd}
      B_{k-1} \arrow[d,"f_{k-1}"]\arrow[r] & B_{{k}} \arrow[rd, "f_{k}"] \arrow[rr] & & B \arrow[d,"f"] \\
      C_{n_{k-1}} \arrow[r] & C_{n'_{k}} \arrow[r] & C_{n_{k}} \arrow[r] & C 
    \end{tikzcd}
  \end{equation}
%  Now by dependent choice for the above $x_0, R_n, E_n$, we get a sequence $(f_i:B_i \to C_{n_i})$  for some 
%  strictly increasing sequence $n_i$ of natural numbers. 
%  Note that for such a sequence $(n_i)_{i:\N}$, 
%  $(C_{n_i})_{i:\N}$ converges to $C$. Also $(B_i)_{i:\N}$ still converges to $B$. 
%  Furthermore, by construction the map that sequence $f_i$ induces from $B \to C$ shares all values with $f$
%  and thus is equal to $f$. 
%  Thus our sequence $f_i$ is as required. 
\end{proof}

\begin{lemma}\label{lemDecompositionOfEpiMonoFactorization}
  Let $f:A_\infty\to B_\infty$ be a map between overtly discrete types, and suppose we have $f_n:A_n\to B_n$ such that 
  the following diagram commutes:
  \begin{equation}
    \begin{tikzcd}
      A_n \arrow[d,"f_n"]\arrow[r, "\iota_n^m"]  & A_m \arrow[d,"f_m"] \arrow[r,"\iota_m^\infty"]  & A_\infty \arrow[d,"f"] 
      \\
      B_n \arrow[r, "\kappa_n^m"'] & B_m \arrow[r,"\kappa_m^\infty"'] & B_\infty
    \end{tikzcd}
  \end{equation}
  Then $f(A)$ is the colimit of $f_n(A_n)$, 
  and the maps $A\twoheadrightarrow f(A)$ and $f(A) \hookrightarrow B$ 
  are induced by the maps $A_n\twoheadrightarrow f_n(A_n)$ and $f_n(A_n) \hookrightarrow B_n$ respectively. 
\end{lemma}
\begin{proof}
  For $n\leq m$, we have that $\kappa_n^m(f_n(A_n)) = f_m(\iota_n^m(A_n))\subseteq f_m(A_m)$, 
  hence we can take the corestriction of the map $f_n(A_n) \to B_m$ to $f_m(A_m)$ to get 
  maps $\lambda_n^m :f_n(A_n) \to f_m(A_m)$ making the following diagram commute:
  % https://q.uiver.app/#q=WzAsOSxbMSwwLCJBX20iXSxbMiwwLCJBX1xcaW5mdHkiXSxbMSwyLCJCX20iXSxbMiwyLCJCX1xcaW5mdHkiXSxbMiwxLCJmKEEpIl0sWzEsMSwiZl9tKEFfbSkiXSxbMCwwLCJBX24iXSxbMCwyLCJCX24iXSxbMCwxLCJmX24oQV9uKSJdLFswLDFdLFsyLDNdLFsxLDQsIiIsMCx7InN0eWxlIjp7ImhlYWQiOnsibmFtZSI6ImVwaSJ9fX1dLFs0LDMsIiIsMSx7InN0eWxlIjp7InRhaWwiOnsibmFtZSI6Imhvb2siLCJzaWRlIjoidG9wIn19fV0sWzAsNSwiIiwyLHsic3R5bGUiOnsiaGVhZCI6eyJuYW1lIjoiZXBpIn19fV0sWzUsMiwiIiwxLHsic3R5bGUiOnsidGFpbCI6eyJuYW1lIjoiaG9vayIsInNpZGUiOiJ0b3AifX19XSxbNywyLCJcXGthcHBhX25ebSIsMl0sWzYsMCwiXFxpb3RhX25ebSJdLFs2LDgsIiIsMix7InN0eWxlIjp7ImhlYWQiOnsibmFtZSI6ImVwaSJ9fX1dLFs4LDcsIiIsMSx7InN0eWxlIjp7InRhaWwiOnsibmFtZSI6Imhvb2siLCJzaWRlIjoidG9wIn19fV0sWzgsNSwiIiwxLHsic3R5bGUiOnsiYm9keSI6eyJuYW1lIjoiZGFzaGVkIn19fV0sWzUsNCwiIiwxLHsic3R5bGUiOnsiYm9keSI6eyJuYW1lIjoiZG90dGVkIn19fV1d
  \begin{equation}\label{eqnEpiMonoFactorizationDecomposition}
    \begin{tikzcd}
    {A_n} & {A_m} & {A_\infty} \\
    {f_n(A_n)} & {f_m(A_m)} & {f(A_\infty)} \\
    {B_n} & {B_m} & {B_\infty}
    \arrow["{\iota_n^m}", from=1-1, to=1-2]
    \arrow[two heads, from=1-1, to=2-1,"e_n"]
    \arrow[from=1-2, to=1-3,"\iota_m^\infty"]
    \arrow[two heads, from=1-2, to=2-2,"e_m"]
    \arrow[two heads, from=1-3, to=2-3,"e"]
    \arrow[dashed, from=2-1, to=2-2, "\lambda_n^m"]
    \arrow[hook, from=2-1, to=3-1, "i_n"]
    \arrow[dashed, from=2-2, to=2-3, "\lambda_m^\infty"]
    \arrow[hook, from=2-2, to=3-2,"i_m"]
    \arrow[hook, from=2-3, to=3-3,"i_\infty"]
    \arrow["{\kappa_n^m}"', from=3-1, to=3-2]
    \arrow[from=3-2, to=3-3,"\kappa_m^\infty"']
  \end{tikzcd}
\end{equation}
  To see that $f(A_\infty)$ is colimiting, note that whenever $b$ occurs in $f(A_\infty)$ as 
  $f(a)$ for some $a:A_\infty$, there is some $n:\N$ and $a':A_n$ with $\iota_n(a') = a$ and 
  $\kappa_n(e_n(a')) = \kappa_n(f_n(a')) = b$, so there is a unique extension from $f(A_\infty)$ into any other cocone. 
\end{proof}
\begin{corollary}
  In \Cref{lemDecompositionOfColimitMorphisms}, when $f$ is injective or surjective, 
  we can choose presentations such that each $f_n$ is also injective or surjective respectively. 
\end{corollary}
\begin{proof}
  Using \Cref{lemDecompositionOfColimitMorphisms} and \Cref{lemDecompositionOfEpiMonoFactorization}, 
  we get a factorization as in \Cref{eqnEpiMonoFactorizationDecomposition}. 
  If $f$ is injective, then $e$ is an isomorphism. 
  Hence $A$ is the colimit of $f_n(A_n)$, and we can take $f_n' = i_n$.
  Similarly, if $f$ is surjective $i$ is an isomorphism and we consider $B$ as colimit of $f_n(A_n)$ and 
  take $f_n' = e_n$.
\end{proof}



%
%\begin{lemma}
%  Let $A,B,C$ be overtly discrete, represented by sequences $(A_n)_{n:\N}, (B_n)_{n:\N},(C_n)_{n:\N}$. 
%  Let $u,v,(u_n)_{n:\N},(v_n)_{n:\N}$ be as in the following diagram of Boolean algebras:
%  \begin{equation}
%    \begin{tikzcd}
%      A_\infty \arrow[r,"u_\infty"] & B_\infty  \arrow[r,"v_\infty"] & C_\infty
%      \\
%      A_n \arrow[u,"i_n^\infty"] \arrow[r,"u_n"] & B_n \arrow[u,"j_n^\infty"] \arrow[r,"v_n"] & C_n \arrow[u,"k^\infty_n"]
%      \\
%      A_m \arrow[u,"i_m^n"] \arrow[r,"u_m"] & B_m \arrow[u,"j_m^n"] \arrow[r,"v_m"] & C_m \arrow[u,"k_m^n"]
%    \end{tikzcd} 
%  \end{equation} 
%  Furthermore, assume that $v\circ u = 0$ and $v_n \circ u_n = 0$ for all $n:\N$.
%  Then $Ker(v)/Im(u)$ is the colimit of the sequence $Ker(v_n)/Im(u_n)$. 
%\end{lemma}
%\begin{proof}
%  First, we will note what the maps in this sequence are, which by some abuse of notation also gives
%  the cocone maps. 
%\paragraph{If $n\leq m$, there are maps $Ker(v_n)/Im(u_n)\to Ker(v_m) / Im(u_m)$.}
%Let $x,y\in B_n, a \in A_n$ be such that $x - y  = u_n(a)$, 
%then $j_n^m(x) - j_n^m(y) = j_n^m(u_n(a)) = u_m(i_n^m(a))$.
%Thus whenever $x,y\in B_n$ are such that $x \sim_{Im(u_n)} y$, we have that 
%$j_n^m(x) \sim_{Im(u_m)} j_n^m(y)$. 
%
%%
%Furthermore, if $x\in Ker(v_n)$, then $v_n(x) = 0$, thus 
%\begin{equation}
%  v_m(j_n^m(x)) = k_n^m(v_n(x)) = k_n^m(0) = 0
%\end{equation} 
%and hence $j_n^m(x) \in Ker(v_m)$. 
%Thus $j_n^m$ induces a map $\iota_n^m:Ker(v_n)/Im(u_n) \to Ker(v_m)/Im(u_m)$, 
%with $\iota^n_m([x]) = [j_n^m(x)]$ for $x\in Ker(v_n)$. 
%
%\paragraph{These maps are compatible (in particular, the maps $j_n^\infty$ form a cocone).}. 
%If $k\leq n \leq m$, we have that $j_n^m \circ j_k^n = j_k^m$.
%We thus have that $\iota_n^m \circ \iota_k^n = \iota_k^m$.
%
%\paragraph{Given any cocone $\kappa_n : Ker(v_n)/Im(u_n)\to K$, 
%  there exists an extension $\kappa_\infty(v_\infty)/Im(u_\infty)\to K$}
%      If $\kappa_n$ forms a cocone, this means that for $n\leq m$ we have 
%      $\kappa_m = \kappa_n \circ \iota_m^n$.
%
%      We shall give a map $\kappa_\infty:Ker(v_\infty)/Im(u_\infty) \to K$ satisfying 
%      $\kappa_\infty \circ \iota_n^\infty= \kappa_n$ for all $n:\N$.
%      We're going to define a map $k:Ker(v_\infty) \to K$.
%%%
%      Let $x\in Ker(v_\infty)$. Then $x\in B_\infty$ and $v\infty(x) = 0$. 
%      As $B_\infty$ is the colimit of the sequence $B_n$, 
%      there is some $n:\N$ and some $x':B_n$ with $v_n(x') = 0$. 
%      We'd like to define $k(x) = \kappa_n([x'])$. We need to check this definition doesn't depend on $n$. 
%      \begin{itemize}
%        \item \textbf{$k$ is well-defined} 
%      Assume $n\leq m$ are such that we have $x':B_n, x'':B_m$ with $v_n(x') = 0, v_m(x'') = 0$ 
%      and $j_n(x') = j_m(x'') = x$. 
%      Then there exists some $l\geq m\geq n$ with 
%      $j_n^l (x') = j_m^l(x'')$, hence 
%      $$\iota_n^l[x'] = \iota_m^l[x'']$$ and thus 
%      $$\kappa_l(\iota_n^l[x']) = \kappa_l(\iota_m^l[x''])$$
%      But now $\kappa_l\circ \iota_n^l = \kappa_n$ and $\kappa_l\circ \iota_m^l = \kappa_m$, hence 
%      $$\kappa_n([x']) = \kappa_m([x''])$$
%      Thus $k$ is well-defined. 
%      \item \textbf{$k$ respects $Im(u)$}
%        Let $x,y:B$ and let $a:A$ be such that are such that $x-y = u(a)$ and $v(x) = v(y) = 0$.
%        Then there is some $n:\N$, and some $x',y':B_n, a':A_n$ with $x'-y'= u_n(a'), v_n(x') = v_n(y') = 0$ 
%        and $j_n(x') = x, j_n(y') = y, i_n(a') = a$. 
%        Hence $[x'] = [y']$, hence $\kappa_n([x']) = \kappa_n([y'])$.
%        Thus $k(x) = k(y)$. 
%      \end{itemize}
%      We conclude that $k$ induces a map $\kappa_\infty:Ker(v)/Im(u) \to K$. 
%    \paragraph{$\kappa$ is colimiting}
%    Let $\kappa_n$ be as above, and suppose that 
%    $\lambda \circ \iota_n^\infty = \kappa_n$. 
%    Then for all $n:\N$, $x':B_n$ such that $j_n(x) = x$, we have 
%    $$\lambda ([x]) = \lambda \circ \iota_n([x']) = \kappa_n([x']) = k(x)$$
%    As such $n,x'$ always exist, it follows that 
%    $\lambda([x]) = k(x)$, hence $\lambda[x] = \kappa[x]$, so $\lambda = \kappa$ as required. 
%\end{proof} 
%
%
%
%







%%\begin{remark}
%%  For any overtly discrete type $B$, we can choose a presentation of the underlying sequence 
%%  $B_n' = \iota_n(B_n)$ such that $B$ is the colimit of $B_n'$, and the inclusion maps $\iota'_n$ are injective. 
%%  Applying the above lemma to some $B$ of this form, we see that if $f$ is injective, so are all the $f_n$. 
%%%  
%%%
%%%  For the above lemma, if $f$ is injective, so are all the $f_n$. This isn't clear at all. Consider 
%%%  the sequence of $\{x,y\} \to \{x\} \to \{x,y\} \to \{x \} \to \cdots$ with colimit $\{x\}$. 
%%\end{remark}
%\begin{remark}
%  Let $B:\Boole$. 
%  Recall that $B$ can be seen as the colimit of $(B_n)_{n:\N}$ with $B_n$ finite Boolean algebras. 
%  Now in the category of Boolean algebras, we have $(B\to 2) \simeq lim(B_n\to 2)$.
%  As $B_n\to 2$ is finite whenever $B_n$ is, it follows that Stone spaces are limits of $\N$-indexed diagrams of finite sets. 
%\end{remark}
%
%
%\begin{remark}\label{rmkEpiMonoFactorizationCommutes}
%  For $f,(f_i)_{i:\N}$ as above, whenever $f_n(x) = 0$, we have $f_{n+1}(x \circ \iota_{n,n+1}) = 0$
%  for $\iota_{n,n+1}$ the map $A_n \to A_{n+1}$. 
%  By \Cref{rmkMorphismsOutOfQuotient}, $\iota_{n,n+1}$ induces a map $A_n/Ker(f_n)\to A_{n+1}/Ker(f_{n+1})$. 
%  This induced map is such that the following diagram commutes:
%  \begin{equation}\begin{tikzcd}
%    A_n \arrow[d, two heads] \arrow[r, "\iota_{n,n+1}"] & A_{n+1} \arrow[d,two heads]\\
%    A_n /Ker(f_n) \arrow[d,hook] \arrow[r] & A_{n+1} /Ker(f_{n+1}) \arrow[d,hook] \\
%    B_n \arrow[r] & B_{n+1}
%  \end{tikzcd}\end{equation}  
%  As the induced maps be epi's / mono's  is epi /mono, the colimit of the sequence 
%  $A_n / Ker(f_n)$ will fit into an epi-mono factorization of $f$ and thus be iso to $A/Ker(f)$. 
%  Thus the epi-mono factorization of the colimit is the colimit of the epi-mono factorizations. 
%\end{remark}
%\begin{remark}\label{rmkIsoEpiMonoMapColimit}
%  Whenever $f:B \to C$ is an iso, any sequence with $B$ as colimit, also has $C$ as colimit. 
%  Thus any iso can be represented this way as sequence of iso's. 
%  Conversely, any sequence of isomorphisms induces an isomorphism of their colimits. 
%
%  It follows from \Cref{rmkEpiMonoFactorizationCommutes} that when $f$ is epi/mono, 
%  we can say that $f$ can be induced by a sequence 
%  $(f_i)_{i\in \N}$ with all $f_i$ epi/mono. 
%\end{remark}
%
%
%
